\addchap{Spelling conventions}
\label{Orthographic conventions}

The writing system used in this grammar largely follows previous works on other Dargwa varieties \citep{Sumbatova.Mutalov2003, Sumbatova.Lander2014}. Given below as well is the Cyrillic \isi{orthography}, which I use with the Sanzhi community and which has been established in \citet{Forker.Gadzhimuradov2017}. It is almost identical to the established \isi{orthography} of Standard Dargwa (plus sounds that Standard Dargwa lacks, minus sounds that do not exist in Sanzhi Dargwa). The letters given in brackets represent phonemes that occur only in loan words.
%
\largerpage[2]
\begin{table}[h!]
	\centering
	\small
	\begin{tabularx}{0.8\textwidth}[]{lllXlll}
	\lsptoprule
			Cyrillic &	orthographic &	IPA	&	{}	&	Cyrillic &	orthographic &	IPA\\
		\cmidrule{1-3} \cmidrule{5-7}
			а	&	a	&	a	&	{}	&	сс	&	sː	&	sː\\
			б	&	b	&	b	&	{}	&	т	&	t	&	t\\
			в	&	w, ʷ	&	w, ʷ	&	{}	&	тт	&	tː	&	tː\\
			г	&	g 	&	g	&	{}	&	тI	&	t’	&	t’\\
			гI	&	ʡ	&	ʕ, ʡ	&	{}	&	у	&	u	&	u\\
			гъ	&	ʁ	&	ʁ	&	{}	&	(ф)	&	(f)	&	f\\
			гь	&	h	&	h	&	{}	&	х	&	χ	&	χ\\
			д	&	d	&	d	&	{}	&	хх	&	χː	&	χː\\
			е	&	e, je	&	e, je	&	{}	&	хъ	&	q	&	q\\	
			ж	&	ž	&	ʒ	&	{}	&	хь	&	x	&	x\\   
			з	&	z	&	z	&	{}	&	хьхь	&	xː	&	xː\\
			и	&	i	&	i	&	{}	&	хI	&	ħ	&	ħ\\
			й	&	j	&	j	&	{}	&	ц	&	c	&	ts\\
			к	&	k	&	k	&	{}	&	цI	&	c’	&	ts’\\
			кк	&	kː	&	kː	&	{}	&	ч	&	č	&	tʃ\\
			кI	&	k’	&	k’	&	{}	&	чː	&	čː	&	tʃː\\
			къ	&	qː	&	qː	&	{}	&	чI	&	č’	&	tʃ’\\
			кь	&	q’	&	q’	&	{}	&	ш	&	š	&	ʃ\\
			л	&	l	&	l	&	{}	&	\mbox{шш, щ}& šː	&	ʃː\\
			м	&	m	&	m	&	{}	&	ъ	&	Ɂ	&	Ɂ\\
			н	&	n	&	n	&	{}	&	э	&	e	&	e\\
			(о)	&	(o)	&	o	&	{}	&	уI	&	uˁ	&	uˁ\\
			п	&	p	&	p	&	{}	&	ю	&	uˁ	&	uˁ\\
			пI	&	p’	&	p’	&	{}	&	ю	&	ju	&	ju\\
			р	&	r	&	r	&	{}	&	я	&	aˁ	&	aˁ\\
			с	&	s	&	s	&	{}	&	я	&	ja	&	ja\\
			\lspbottomrule
	\end{tabularx}
\end{table}


%%%%%%%%%%%%%%%%%%%%%%%%%%%%%%%%%%%%%%%%%%%%%%%%%%%%%%%%%%%%%%%%%%%%%%%%%%%%%%%%

\addchap{Glosses and other abbreviations}
\section*{Glosses}
	\begin{multicols}{2}
    \largerpage[2]
		\begin{tabbing}
            \tsc{in.front}\hspace{\tabcolsep}\= human feminine singular\kill
			\tsc{1}	\>	first person\\
			\tsc{2}	\>	second person\\
			\tsc{3}	\>	third person\\
			\tsc{abl}	\>	\isi{ablative}\\
			\tsc{ad} \>	\isi{spatial case} `at' animate \\
			{}		\> reference point\\
			\tsc{add}	\>	\isi{additive}\\
			\tsc{adjvz}	\>	\isi{adjectivizer}\\
			\tsc{advz}	\>	\isi{adverbializer}\\
			\tsc{ante}	\>	location \sqt{in front}\\
			{}		\>	(\isi{spatial case}, \isi{preverb})\\
			\tsc{assoc}	\>	associative plural\\
			\tsc{aux}	\>	auxiliary\\
			\tsc{behind}	\>	spatial \isi{preverb} `behind'\\
			\tsc{caus}	\>	\isi{causative}\\
			\tsc{comit}	\>	\isi{comitative}\\
			\tsc{conc}	\>	\isi{concessive}\\
			\tsc{cond}	\>	\isi{conditional}\\
			\tsc{cop}	\>	\isi{copula}\\
			\tsc{cvb}	\>	converb\\
			\tsc{dat}	\>	\isi{dative}\\
			\tsc{dd}	\>	definite description\\
			\tsc{dem}	\>	demonstrative\\
			\tsc{dir}	\>	directional case\\
			\tsc{down}	\>	spatial \isi{preverb} `down'\\
			\tsc{emph}	\>	emphatic \isi{particle}\\
			\tsc{eq}	\>	equative \isi{enclitic}\\
			\tsc{erg}	\>	\isi{ergative}\\
			\tsc{f}	\>	human feminine singular\\
			\tsc{gen}	\>	\isi{genitive}\\
			\tsc{gm}	\> \isi{gender} marker\\
			\tsc{group}	\> \isi{derivation} of \is{group numeral}group numerals\\
			\tsc{hab}	\>	habitual\\
			\tsc{hither}	\>	\isi{preverb} `to the speaker, hither'\\
			\tsc{hpl}	\>	human plural\\
			\tsc{icvb}	\>	\isi{imperfective converb}\\
			\tsc{imp}	\>	\isi{imperative}\\
			\tsc{in}	\>	location \sqt{in}; \isi{preverb} \sqt{in};\\
			{}		\>	    \isi{spatial case} `in, on, at, \\ \> among' \\
			\tsc{indef}	\>	indefinite\\
			\tsc{indq}	\>	embedded question\\
			\tsc{inf}	\>	\isi{infinitive}\\
			\tsc{in.front}	\>	\isi{preverb} `in front'\\
			\tsc{in.the.hands}	\>	\\
						\> \isi{preverb} `in the hands'\\
			\tsc{intr}	\>	stem augment vowel for \\ \> \is{intransitive verb}intransitive verbs in certain \\ \> verb forms\\
			\tsc{ipfv}	\>	imperfective\\
			\tsc{loc}	\>	locative (\isi{participle});\\
			{}		\>	locative case `in, on, to'\\
			\tsc{m}		\>	human masculine singular\\
			\tsc{mod}	\>	modal\\
			\tsc{modq}	\>	\isi{modal interrogative}\\
			\tsc{msd}	\>	\isi{masdar}\\
			\tsc{n}		\>	neuter singular\\
			\tsc{neg}	\>	\isi{negation}\\
			\tsc{nmlz}	\>	\isi{nominalizer}\\
			\tsc{npl}	\>	neuter plural\\
			\tsc{num}	\>	numeral\\
			\tsc{obl}	\>	\isi{oblique stem} marker\\
			\tsc{opt}	\>	\isi{optative}\\
			\tsc{ord}	\>	ordinal\\
			\tsc{outside}	\>	spatial \isi{preverb} `outside'\\
			\tsc{pfv}	\>	perfective\\
			\tsc{pl}	\>	plural\\
			\tsc{post}	\>	posteriority \\
			{}		\>	    temporal suffix \sqt{since, after};\\
			{}		\>	    \isi{spatial case} \sqt{behind} \\
			\tsc{pret}	\>	preterite\\
			\tsc{proh}	\>	\isi{prohibitive}\\
			\tsc{prs}	\>	present\\
			\tsc{prt}	\>	\isi{particle}\\
			\tsc{pst}	\>	past\\
			\tsc{ptcp}	\>	\isi{participle}\\
			\tsc{pvb}	\>	\isi{preverb}\\
			\tsc{q}		\>	question\\
			\tsc{ref}	\>	referential\\
			\tsc{refl}	\>	reflexive\\
			\tsc{sg}	\>	singular\\
			\tsc{spr}	\>	spatial \isi{preverb} `on'\\
			\tsc{sub}	\>	location \sqt{under}\\
			{}		\>	(\isi{spatial case}, \isi{preverb})\\
			\tsc{subj}	\>	\isi{subjunctive}\\
			\tsc{ten}	\>	\isi{derivation} of numerals \\ \> multiples of ten\\
			\tsc{thither}	\>	\isi{preverb} `away from \\ \> speaker, thither'\\
			\tsc{time}	\>	\isi{derivation} of multiplicative \\ \> numerals\\
			\tsc{up}	\>	spatial meaning `up(wards)'\\
		\end{tabbing}
	\end{multicols}

\section*{Other abbreviations}
	\begin{multicols}{2}
		\begin{tabbing}
		  STIM\hspace{\tabcolsep}\= single argument of an\kill
			A 		\>	\isi{agent}\\
			C		\>  consonant\\
			cond.	\>	conditional\\
			dem.	\>	demonstrative\\
			ditr.	\>	ditransitive\\
			E	\>	elicited example\\
			EXP 	\>	\isi{experiencer}\\
			G 		\>	\isi{goal}\\
			intr. 	\>	intransitive\\
			IPA 	\>	International Phonetic\\
			{}		\>	Alphabet\\
			lit. 	\>	literally\\
			N 		\>	noun\\
			n		\>	no\\
			NP 		\>	\isi{noun phrase}\\
			O		\>	object\\
			P 		\>	\isi{patient}\\
			pro.		\>	pronoun\\
			R 		\>	\isi{recipient}\\
			refl.		\>	reflexive\\
			S 		\>	single argument of an\\
			{}		\>	intransitive clause\\
			S		\>	subject\\
			s.o.		\>	someone\\
			T 		\>	\isi{theme}\\
			TAM 	\>	tense-aspect-mood\\
			tr. 	\>	transitive\\
			V 		\>	verb\\
			V		\> 	vowel\\
			y		\> 	yes\\
		\end{tabbing}
	\end{multicols}
