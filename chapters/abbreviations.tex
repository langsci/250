\addchap{Spelling conventions}
\label{Orthographic conventions}

The writing system used in this grammar largely follows previous works on other Dargwa varieties \citep{Sumbatova.Mutalov2003, Sumbatova.Lander2014}. Given below as well is the Cyrillic orthography, which I use with the Sanzhi community and which has been established in \citet{Forker.Gadzhimuradov2017}. It is almost identical to the established orthography of Standard Dargwa (plus sounds that Standard Dargwa lacks, minus sounds that do not exist in Sanzhi Dargwa). The letters given in brackets represent phonemes that occur only in loan words.
%
\largerpage[2]
\begin{table}[h!]
	\centering
	\small
	\begin{tabularx}{0.8\textwidth}[]{lllXlll}
	\lsptoprule
			Cyrillic &	orthographic &	IPA	&	{}	&	Cyrillic &	orthographic &	IPA\\
		\cmidrule{1-3} \cmidrule{5-7}
			а	&	a	&	a	&	{}	&	сс	&	sː	&	sː\\
			б	&	b	&	b	&	{}	&	т	&	t	&	t\\
			в	&	w, ʷ	&	w, ʷ	&	{}	&	тт	&	tː	&	tː\\
			г	&	g 	&	g	&	{}	&	тI	&	t’	&	t’\\
			гI	&	ʡ	&	ʕ, ʡ	&	{}	&	у	&	u	&	u\\
			гъ	&	ʁ	&	ʁ	&	{}	&	(ф)	&	(f)	&	f\\
			гь	&	h	&	h	&	{}	&	х	&	χ	&	χ\\
			д	&	d	&	d	&	{}	&	хх	&	χː	&	χː\\
			е	&	e, je	&	e, je	&	{}	&	хъ	&	q	&	q\\	
			ж	&	ž	&	ʒ	&	{}	&	хь	&	x	&	x\\   
			з	&	z	&	z	&	{}	&	хьхь	&	xː	&	xː\\
			и	&	i	&	i	&	{}	&	хI	&	ħ	&	ħ\\
			й	&	j	&	j	&	{}	&	ц	&	c	&	ts\\
			к	&	k	&	k	&	{}	&	цI	&	c’	&	ts’\\
			кк	&	kː	&	kː	&	{}	&	ч	&	č	&	tʃ\\
			кI	&	k’	&	k’	&	{}	&	чː	&	čː	&	tʃː\\
			къ	&	qː	&	qː	&	{}	&	чI	&	č’	&	tʃ’\\
			кь	&	q’	&	q’	&	{}	&	ш	&	š	&	ʃ\\
			л	&	l	&	l	&	{}	&	\mbox{шш, щ}& šː	&	ʃː\\
			м	&	m	&	m	&	{}	&	ъ	&	Ɂ	&	Ɂ\\
			н	&	n	&	n	&	{}	&	э	&	e	&	e\\
			(о)	&	(o)	&	o	&	{}	&	ю	&	uˁ	&	uˁ\\
			п	&	p	&	p	&	{}	&	ю	&	ju	&	ju\\
			пI	&	p’	&	p’	&	{}	&	уI	&	uˁ	&	uˁ\\
			р	&	r	&	r	&	{}	&	я	&	aˁ	&	aˁ\\
			с	&	s	&	s	&	{}	&	я	&	ja	&	ja\\
			\lspbottomrule
	\end{tabularx}
\end{table}


%%%%%%%%%%%%%%%%%%%%%%%%%%%%%%%%%%%%%%%%%%%%%%%%%%%%%%%%%%%%%%%%%%%%%%%%%%%%%%%%

\addchap{Glosses and other abbreviations}
\section*{Glosses}
	\begin{multicols}{2}
    \largerpage[2]
		\begin{tabbing}
            \tsc{indef}\hspace{\tabcolsep}\= human feminine singular\kill
\tsc{indef}\hspace{\tabcolsep}\= human feminine singular\kill
			\tsc{1}	\>	first person\\
			\tsc{2}	\>	second person\\
			\tsc{3}	\>	third person\\
			\tsc{abl}	\>	ablative\\
			\tsc{ad} \>	spatial case `at' animate \\
			{}		\> reference point\\
			\tsc{add}	\>	additive\\
			\tsc{adjvz}	\>	adjectivizer\\
			\tsc{advz}	\>	adverbializer\\
			\tsc{ante}	\>	location \sqt{in front}\\
			{}		\>	(spatial case, preverb)\\
			\tsc{assoc}	\>	associative plural\\
			\tsc{aux}	\>	auxiliary\\
			\tsc{caus}	\>	causative\\
			\tsc{comit}	\>	comitative\\
			\tsc{comp}	\>	comparative\\
			\tsc{conc}	\>	concessive\\
			\tsc{cond}	\>	conditional\\
			\tsc{cop}	\>	copula\\
			\tsc{cvb}	\>	converb\\
			\tsc{dat}	\>	dative\\
			\tsc{dd}	\>	definite description\\
			\tsc{dem}	\>	demonstrative\\
			\tsc{dir}	\>	directional case\\
			\tsc{down}	\>	spatial preverb `down'\\
			\tsc{emph}	\>	emphatic particle\\
			\tsc{eq}	\>	equative enclitic\\
			\tsc{erg}	\>	ergative\\
			\tsc{f}	\>	human feminine singular\\
			\tsc{gen}	\>	genitive\\
			\tsc{gm}	\> gender marker\\
			\tsc{hab}	\>	habitual\\
			\tsc{hpl}	\>	human plural\\
			\tsc{icvb}	\>	imperfective converb\\
			\tsc{imp}	\>	imperative\\
			\tsc{in}	\>	location \sqt{in}\\
			{}		\>	    preverb \sqt{in};\\
			{}		\>	    spatial case \sqt{in, on, at, among} \\
			\tsc{indef}	\>	indefinite\\
			\tsc{indq}	\>	embedded question\\
			\tsc{inf}	\>	infinitive\\
			\tsc{in.front}	\>	preverb `in front'\\
			\tsc{ipfv}	\>	imperfective\\
			\tsc{loc}	\>	locative (participle);\\
			{}		\>	locative case `in, on, to'\\
			\tsc{m}		\>	human masculine singular\\
			\tsc{mod}	\>	modal\\
			\tsc{modq}	\>	modal interrogative\\
			\tsc{n}		\>	neuter singular\\
			\tsc{neg}	\>	negation\\
			\tsc{nmlz}	\>	nominalizer\\
			\tsc{npl}	\>	neuter plural\\
			\tsc{num}	\>	numeral\\
			\tsc{obl}	\>	oblique stem marker\\
			\tsc{opt}	\>	optative\\
			\tsc{ord}	\>	ordinal\\
			\tsc{pfv}	\>	perfective\\
			\tsc{pl}	\>	plural\\
			\tsc{post}	\>	posteriority \\
			{}		\>	    temporal suffix \sqt{since, after};\\
			{}		\>	    spatial case \sqt{behind} \\
			\tsc{pret}	\>	preterite\\
			\tsc{proh}	\>	prohibitive\\
			\tsc{prs}	\>	present\\
			\tsc{prt}	\>	particle\\
			\tsc{pst}	\>	past\\
			\tsc{ptcp}	\>	participle\\
			\tsc{pvb}	\>	preverb\\
			\tsc{q}		\>	question\\
			\tsc{ref}	\>	referential\\
			\tsc{refl}	\>	reflexive\\
			\tsc{sg}	\>	singular\\
			\tsc{spr}	\>	spatial preverb `on'\\
			\tsc{sub}	\>	location \sqt{under}\\
			{}		\>	(spatial case, preverb)\\
			\tsc{subj}	\>	subjunctive\\
			\tsc{up}	\>	spatial meaning `up(wards)'
		\end{tabbing}
	\end{multicols}


\section*{Other abbreviations}
	\begin{multicols}{2}
		\begin{tabbing}
		  STIM\hspace{\tabcolsep}\= single argument of an\kill
			A 		\>	agent\\
			E	\>	elicited example\\
			EXP 	\>	experiencer\\
			C		\>  consonant\\
			G 		\>	goal\\
			intr. 	\>	intransitive\\
			IPA 	\>	International Phonetic\\
			{}		\>	Alphabet\\
			lit. 	\>	literally\\
			N 		\>	noun\\
			NP 		\>	noun phrase\\
			O		\>	object\\
			P 		\>	patient\\
			R 		\>	recipient\\
			S 		\>	single argument of an\\
			{}		\>	intransitive clause\\
			S		\>	subject\\
			s.o.		\>	someone\\
			T 		\>	theme\\
			TAM 	\>	tense-aspect-mood\\
			tr. 	\>	transitive\\
			V 		\>	verb\\
			V		\> 	vowel\\
		\end{tabbing}
	\end{multicols}
