\chapter{Adjectives}
\label{cpt:morph-adjectives}
\section{Introduction}
\label{sec:Other syntactic properties}

Adjectives in Sanzhi can clearly be distinguished from \isi{nouns} or verbs since they are not lexically specified for \isi{gender}, and they cannot take tense suffixes or other inflectional morphology reserved for verbs. They are formally rather heterogeneous (\refsec{sec:adjmorphclasses}).  
Sanzhi \isi{adjectives} cover the typical semantic domains of this word class \refexrange{ex:dimensionADJ}{ex:otherADJ}.
%
\begin{exe}
	\ex	dimension	\label{ex:dimensionADJ} \\
\TabPositions{12em}
		\tit{qːant'} \sqt{short} \tab 		\tit{χːula} \sqt{big, old} \\
		\tit{nik'a} \sqt{small} \tab 		\tit{aq} \sqt{high, tall}

	\ex	age	\label{ex:ageADJ} \\
	\TabPositions{12em}
		\tit{b-uqna} \sqt{old} \tab 		\tit{jangi} \sqt{new} \\
		\tit{žahil} \sqt{young}

	\ex	evaluation:	\label{ex:evaluationADJ} \\
	\TabPositions{12em}
		\tit{wahi} \sqt{bad, evil} 		\tab	\tit{ʡaˁħ} \sqt{good} \\
		\tit{ʡaˁziz} \sqt{beloved, dear} 	\tab	\tit{durqa} \sqt{dear, expensive} \\
		\tit{durha} \sqt{cheap} 	\tab 	\tit{ʡaˁžib} \sqt{surprising}

	\ex	colour	\label{ex:colourADJ} \\
	\TabPositions{12em}
		\tit{c'utːar} \sqt{black} \tab \tit{c'ub} \sqt{white} \\
		\tit{it'in} \sqt{red} \tab  	\tit{xanc'} \sqt{blue} \\
		\tit{b-uqu} \sqt{yellow} \tab  	\tit{šiniš} \sqt{green} 
	

	\ex	physical property (humans and non-humans)	\label{ex:physicalpropertyADJ} \\
	\TabPositions{12em}
		\tit{b-arx} \sqt{direct, straight, right} \tab 	\tit{dirq'} \sqt{plain} \\
		\tit{qːuʁa} \sqt{beautiful} \tab 	\tit{čakːʷal} \sqt{handsome} \\
		\tit{c'aˁb} \sqt{dark} \tab 	\tit{kuk} \sqt{light} (\tie\ not heavy) \\
		\tit{dek'ʷ} \sqt{heavy} \tab 	\tit{gʷana} \sqt{warm} \\
		\tit{buχːar} \sqt{cold} \tab 	\tit{jazuq}; \tit{usal} \sqt{weak} \\
		\tit{q'amc'} \sqt{sour} \tab 	\tit{mizi} \sqt{sweet} \\
		\tit{bicːi} \sqt{tasty, aromatic} \tab 	\tit{b-uqen} \sqt{long}\\
		\tit{b-uˁc} \sqt{thick, dense} \tab \tit{b-aˁršu} \sqt{thick} (only inanimate referents) \\
		\tit{b-uk'ul} \sqt{thin} \tab \tit{ʁʷirc'} \sqt{thin} (only with inanimate referents) \\
		\tit{mic'ir} \sqt{alive} \tab 	\tit{c'aq'} \sqt{strong, mighty} \\
		\tit{debga} \sqt{tight} \tab 	\tit{laˁʁun} \sqt{smooth} \\
		\tit{b-ac'} \sqt{empty} \tab 	\tit{k'ant'i} \sqt{soft} \\
		\tit{duc'} \sqt{hot} \tab		\tit{dibaˁʁ} \sqt{ugly} \\
		\tit{ač} \sqt{open} \tab \tit{sːuqːur} \sqt{blind}

	\ex	human characteristics	\label{ex:humancharacteristics} \\
	\TabPositions{12em}
		\tit{razi} \sqt{happy} \tab 	\tit{duˁʡ} \sqt{wild, unrestricted} \\
		\tit{baˁħ} \sqt{crazy} \tab 	\tit{q'irq'ir} \sqt{greedy} \\
		\tit{basrak} \sqt{greedy} \tab 	\tit{ʡaˁsi} \sqt{angry} \\
		\tit{duχːu} \sqt{clever} \tab 	\tit{ʁaj adalχan} \sqt{mute} \\
		\tit{taliħči-b} \sqt{lucky, happy} \tab 		\tit{dawlači-b} \sqt{rich}\\
		\tit{pašman} \sqt{sad} \tab 	\tit{tašmiš} \sqt{sad} \\
		\tit{sark} \sqt{open-hearted}
	

	\ex	speed	\label{ex:speedADJ} \\
	\TabPositions{12em}
		\tit{bahla} \sqt{slow, quiet} \tab \tit{halak} \sqt{fast}

	\ex	difficulty	\label{ex:difficultyADJ} \\
	\TabPositions{12em}
		\tit{qːihin} \sqt{difficult} \tab 		\tit{raˁħaˁt} \sqt{easy}

	\ex	similarity	\label{ex:similarityADJ} \\
	\TabPositions{12em}
		\tit{miši} \sqt{similar} \tab \tit{dik'ar} \sqt{separate, different}

	\ex	 quantification	\label{ex:quantificationADJ} \\
\TabPositions{12em}
		\tit{har} \sqt{every} \tab 		\tit{li<b>il} \sqt{all} \\
		\tit{cara} \sqt{other} \tab 		\tit{ʡaˁbra} \sqt{much, many}\\
		\tit{kam} \sqt{little, few} \tab 	\tit{imc'a} \sqt{additional, superfluous} \\
		\tit{b-aq} \sqt{much, many}

	\ex	position	\label{ex:position} \\
	\TabPositions{12em}	
		\tit{guq} \sqt{low}	\tab 		\tit{xːar} \sqt{low} \\
		\tit{hek} \sqt{near} \tab 		\tit{qar} \sqt{upper} \\
		\tit{haraq} \sqt{far}

	\ex	other	\label{ex:otherADJ} \\
	\TabPositions{12em}
		\tit{busan} \sqt{rainy} \tab \tit{urra} \sqt{foreign}
	
\end{exe}

A few underived \isi{adjectives} have agreement markers as can be see from the examples above. In addition, all derived \isi{adjectives} containing the essive case plus \tit{-il, -či-b} and \tit{-b-azi-b} and all constructions with \tit{b-ah} (\refsec{sec:Derivation of adjectives}) also agree. Adjectives agree with the head noun in \isi{gender} and \isi{number} \refex{ex:He immediately broke one long branch (off a tree)}, \refex{ex:I know good people}, and \refex{ex:There is even a woman who looks old}. More information on \isi{gender}/\isi{number} agreement rules is provided in \refsec{sec:Gender/number agreement}. 

\begin{exe}
	\ex	\label{ex:He immediately broke one long branch (off a tree)}
	\gll	či-r-ix-ub	ca-b	halak-le	ca	b-uqen	q'aˁli \\
		\tsc{spr-abl}-take.off.\tsc{pfv-pret}	\tsc{cop-n}	fast-\tsc{advz}	one	\tsc{n}-long	branch \\
	\glt	\sqt{He immediately broke one long branch (off a tree).}
\end{exe}

Adjectives can be modified by adverbs, most commonly by \isi{degree adverbs} that precede the \isi{adjectives}, for example \tit{c'aq'le} \refex{ex:He was very very good}, \tit{ħaˁq'le} \sqt{very}, \tit{arindan} \sqt{too, too much} \refex{ex:The apples are too expensive}, \tit{b-aq} \sqt{much}, \tit{bah} \sqt{most} \refex{ex:Was (our) grandfather the oldest among his brothers}, \tit{χːʷalle} \sqt{largely}, \tit{q'ʷila, bara, kamle} \sqt{little, few, a bit} \refex{ex:My shirt was a bit long}. 
%
\begin{exe}
	\ex	\label{ex:He was very very good}
	\gll	c'aq'-le	χːʷal-le	ʡaˁħ	Ø-iχ-ub ca-w \\
		very-\tsc{advz}	big-\tsc{advz}	good	\tsc{m}-be.\tsc{pfv-pret} \tsc{cop-m} \\
	\glt	\sqt{He was very, very good.}

	\ex	\label{ex:The apples are too expensive}
	\gll	hetːi	hinc-be	arindan	durqa-te	ca<d>i \\
		those	apple\tsc{-pl}	too	expensive-\tsc{dd.pl} 	\tsc{cop<npl>}\\
	\glt	\sqt{The apples are too expensive.}

	\ex	\label{ex:My shirt was a bit long}
	\gll	di-la	q'ʷila	b-uqen	kːurtːi=de \\
		\tsc{1sg-gen}	a.little	\tsc{n}-long	dress=\tsc{pst} \\
	\glt	\sqt{My shirt was a bit long.}
\end{exe}

There is no derivational means of forming negative \isi{adjectives}. Only \isi{participles} used like \isi{adjectives} can have a negative variant if the verbal \isi{negation} prefix \tit{a-} is added, e.g. \textit{a-b-ucː-an} (\tsc{neg-n}-work-\tsc{ptcp}) `inoperative, spoiled, not working'. Otherwise \isi{negation} is expressed on the verb that heads the clause containing the adjective (see, \teg\ \refsec{sec:copulaclauses} on \isi{copula} clauses).

Adjectives usually precede the head noun, but the reverse order is also possible. Modifying adverbs, in turn, precede the adjective. \refsec{ssec:The structure and order of constituents within the noun phrase} provides information about \isi{constituent order} in the \isi{noun phrase}.


%%%%%%%%%%%%%%%%%%%%%%%%%%%%%%%%%%%%%%%%%%%%%%%%%%%%%%%%%%%%%%%%%%%%%%%%%%%%%%%%

\section{Adjectives and the cross-categorical suffixes -\textit{ce} and -\textit{il}}
\label{sec:adjmorphclasses} 

As is characteristic for Dargwa varieties, \isi{adjectives} occur in the form of bare roots when they are used as attributes to nominals \refex{ex:He immediately broke one long branch (off a tree)}, \refex{ex:Probably he was a bad person}. Many but not all of the \isi{adjectives} in \refexrange{ex:dimensionADJ}{ex:otherADJ} belong to the class of adjectival roots. Some of these \isi{adjectives} are also used in \isi{compounding}, especially for the formation of \isi{compound verbs} (\refsec{sec:compoundswithshortadjectives}), e.g. \textit{aq b-ik'ʷ-ij} `increase, enlarge, elevate, rise' (high \tsc{n}-aux.\tsc{ipfv-inf}).
%
\begin{exe}
	\ex	\label{ex:Probably he was a bad person}
	\gll	wahi	admi	už-ib ca-w	a-b-iχʷ-ar \\
		evil	person	be\tsc{-pret} \tsc{cop-m}	\tsc{neg-n-}be.\tsc{pfv-prs}\\
	\glt	\sqt{Probably he was a bad person.}
\end{exe}

The adjectival roots cannot be used substantively or predicatively. They must take the suffix \tit{-ce} and can then fulfill all three functions: attribution \refex{ex:We had a bad road}, predication \refex{ex:The apples are too expensive}, \refex{ex:Khabaci was also not bad_1} and reference \refex{ex:then like in order not to do bad}. In the plural, -\textit{ce} is replaced by \tit{-te} \refex{ex:then like in order not to do bad}.
%
\begin{exe}
	\ex	\label{ex:We had a bad road}
	\gll	wahi-ce	xːun	b-irχ-i	nišːa-la\\
		bad-\tsc{dd.sg}	way	\tsc{n}-be.\tsc{ipfv-hab.pst}	\tsc{1pl-gen}\\
	\glt	\sqt{We had a bad road.} (or \sqt{There was a bad road in our (area).})

	\ex	\label{ex:Khabaci was also not bad_1}
	\gll	χabacːi	dik'ar	wahi-ce	akːʷ-i \\
		Khabaci	too	bad-\tsc{dd.sg}	\tsc{cop.neg-hab.pst}\\
	\glt	\sqt{Khabaci (= personal name) was also not bad.}

	\ex	\label{ex:then like in order not to do bad} {
	\gll	c'il	wahi-te	a-d-arq'-ij	daˁʡle	\ldots\\
		then	evil-\tsc{dd.pl} 	\tsc{neg-npl}-do.\tsc{pfv-inf} as\\
	\glt	\sqt{then like in order not to do bad (things) \ldots}}
\end{exe}

When occurring in the canonical position before the head noun, adjectival roots and \isi{adjectives} with the suffix \tit{-ce} do not differ in their morphosyntactic or semantic properties. For example, both types of \isi{adjectives} can modify coordinated noun phrases \refex{ex:In Sanijats class there are good}. This behavior differentiates Sanzhi Dargwa from other Dargwa varieties such as Tanti Dargwa or Standard (Akusha) Dargwa, for which syntactic differences between adjectival roots and the so-called ``long'' \isi{adjectives} have been attested (\citealp[26]{vandenBerg2001}, \citealp[207\tnd208]{AbdullaevEtAl2014}, \citealp{Lander2014}).
%
\begin{exe}
	\ex	\label{ex:In Sanijats class there are good}
	\gll	Sanijat-la	kːalas-le-b	ʡaˁħ(-te)	[durħ-ne=ra	rurs-be=ra]	χe-b\\
		Sanijat-\tsc{gen}	class-\tsc{loc-hpl}	good-\tsc{dd.pl} 	boy\tsc{-pl=add}	girl\tsc{-pl=add}	exist.\tsc{down-hpl} \\
	‎\glt	‎‎\sqt{In Sanijat's class there are good [boys and girls].} (E)
\end{exe}

When nominalized, case suffixes are directly added to \tit{-ce} if the nominalized adjective occurs in the singular \refex{ex:Does he give medicine to the little one}. In the plural, the suffix \tit{-t-a} (instead of \tit{-t-e}) is used when case suffixes follow.
%
\begin{exe}
	\ex	\label{ex:Does he give medicine to the little one}
	\gll	nik'a-ce-li-j	darman	lukː-unne=w? \\
		small-\tsc{dd.sg-obl-dat}	medicine	give.\tsc{ipfv-icvb=q} \\
	\glt	\sqt{Does he give medicine to the little one?}
\end{exe}


The suffix -\textit{ce} attaches not only to adjectival roots, but also to other parts of speech such as inflected \isi{nouns} or verbs. Thus, its use is not restricted to \isi{adjectives}, but it applies across a range of lexical categories. Generally speaking, it forms definite descriptions that function as referential attributes, and syntactically behave like nominals. A detailed description of the functions of -\textit{ce} is given in \refsec{ssec:The -ce / -te attributive}.

Apart from the suffix \tit{-ce} Sanzhi has another suffix -\textit{il} for the formation of referential attributes that have similar morphosyntactic properties like items with -\textit{ce}, but its application is far more restricted. Only two quantitative \isi{adjectives} need the suffix \tit{-il} in order to be used not only attributively, but also substantively or predicatively: \tit{har-il} \sqt{every} and \tit{b-aq-il} \sqt{much, many}. Furthermore, it is arguably a part of the \isi{quantifier} \textit{li<b>il} `all', and when added to the \isi{preterite participle} of the verb \tit{ʔ-} \sqt{say}, the resulting verb form is used as a marker for \isi{ordinal numerals} (\refsec{sec:ordinalnumerals}), which are also adjectival in nature. More information on -\textit{il} can be found in \refsec{ssec:The -il attributive}.



%%%%%%%%%%%%%%%%%%%%%%%%%%%%%%%%%%%%%%%%%%%%%%%%%%%%%%%%%%%%%%%%%%%%%%%%%%%%%%%%

\section{Formation of adjectival attributes}
\label{sec:Derivation of adjectives}
Sanzhi does not have very productive means of forming new \isi{adjectives}, but there are a few suffixes that take \isi{nouns} as base and derive \isi{adjectives}. Other ways of extending the lexicon is by means of \isi{genitive} attributes and a special construction with the noun `owner' (see below). Furthermore, \isi{participles} are used and nowadays Russian \isi{adjectives} also occur occasionally.  

Sanzhi has a \isi{number} of \isi{adjectives} that are derived from \isi{nouns} denoting body parts and personal qualities. These \isi{adjectives} express the \isi{possession} of this body part. The base noun is marked for plural and then the suffix \tit{-ar} is added whereby the final vowel of the plural suffix undergoes deletion \refex{ex:adjectivesWithBAR}. This suffix might be a cognate of the \isi{participle} suffix of the \isi{copula} \textit{-ar} (\refsec{sec:The copula}). The \isi{adjectives} form the plural by mean of the most common plural suffix \tit{-te}. Two examples are provided in \refex{ex:The cat is hairy} and \refex{ex:hairy cats}.
%
\begin{exe}
	\ex	\label{ex:adjectivesWithBAR}
	\begin{xlist}
		\TabPositions{16em}
		\ex	\tit{ʁiz-b-ar} \sqt{hairy}					\tab < \tit{ʁizbe} \sqt{hairs}
		\ex	\tit{qi-m-ar} \sqt{horned}					\tab	< \tit{qime} \sqt{horns}
		\ex	\tit{supen-t-ar} \sqt{whiskered, mustached} 	\tab	< \tit{supente} \sqt{mustache} 
		\ex	\tit{laˁpː-ar} \sqt{big-eared, having ears} 	\tab	 < \tit{laˁpːe} \sqt{ears} 	
		\ex	\tit{cul-b-ar} \sqt{having (big) teeth} \tab	 < \tit{culbe} \sqt{teeth}
		\ex	\tit{ul-b-ar} \sqt{having big eyes} \tab	< \tit{ulbe} \sqt{eyes} 
		\ex \textit{k'ult'-n-ar} \sqt{pregnant} \tab	< \textit{k'ult'ne} \sqt{bellies}
		\ex \textit{piš-n-ar} \sqt{naughty boy, scamp} \tab	< \textit{pišne} \sqt{habits, tricks}
		\ex	\tit{ʡaˁmul-t-ar} \sqt{talented}			\tab	 < \tit{ʡaˁmulte} \sqt{skills, talents}
	\end{xlist}
\end{exe}

\begin{exe}
	\ex
	\begin{xlist}
		\ex	\label{ex:The cat is hairy}
		\gll   	het	kːaˁta	ʁiz-b-ar	ca-b \\
			that	cat	hair-\tsc{pl-adjvz}	\tsc{cop-n}\\
		\glt  	\sqt{The cat is hairy.} (E)

		\ex	\label{ex:hairy cats}
		\gll   	ʁiz-b-ar-te kːaˁt-ne \\
			hair-\tsc{pl-adjvz-pl}	cat-\tsc{pl} \\
		\glt	\sqt{hairy cats} (E)
	\end{xlist}
\end{exe}

There are a few \isi{adjectives} involving \isi{compounding} with numerals and mostly plural \isi{nouns} and the suffix \textit{-(a)n}. As with the \isi{adjectives} given in \refex{ex:CompoundNounsNumeralsAdjectives}, the \isi{nouns} occur in the plural. It might be the case that this suffix is a cognate of the modal\slash future \isi{participle} \textit{-an} (\refsec{sssec:The modal participle -an}), the \isi{locative participle} \textit{-an} (\refsec{sssec:The locative participle}) and/or the suffix \textit{-an} that is used for the \isi{derivation} of terms denoting inhabitants of particular villages and other places (\refsec{cpt:morph-placenames}).

%
\begin{exe}
	\ex	\label{ex:CompoundNounsNumeralsAdjectives} 
	\begin{xlist}
		\TabPositions{12em,14em}
			\ex	\tit{aʁmuzan}	\sqt{quadratic}		\tab	<	\tab	 \tit{aʁʷ} \sqt{four} + \tit{muza-n} corner-\tsc{adjvz}
			\ex	\tit{ʡaˁbmuzan}	\sqt{triangular}		\tab	<	\tab	 \tit{ʡaˁb} \sqt{three} + \tit{muza-n} corner-\tsc{adjvz}		
			\ex	\tit{ʡaˁbkumran}	\sqt{three-layered}		\tab	<	\tab	 \tit{ʡaˁb} \sqt{three} + \tit{kam-r-an} layer-\tsc{pl-adjvz}\\
			\ex	\tit{ʡaˁbdusːan}	\sqt{three-year}		\tab	<	\tab	 \tit{ʡaˁb} \sqt{three} + \tit{dusː-an} year\tsc{-pl-adjvz}
	\end{xlist}
\end{exe}

Another type of derived adjectival attributes can be formed from \isi{adjectives} denoting relational qualities. To the base \isi{adjectives} the suffix \tit{-\tsc{gm}-azi-\tsc{gm}} is added and the resulting \isi{adjectives} denote an extreme quality. As can be seen in \refex{ex:adjectivesWithBAZI}, the base can already be a derived adjective. The resulting \isi{adjectives} occur in attributive, predicative and substantive function \xxref{ex:I know good people}{ex:There are other engineers}. In the predicative function the suffix \tit{-ce} (-\tit{te}) is required \refex{ex:Those people are good}.

\begin{exe}
	\ex	\label{ex:adjectivesWithBAZI}
	\begin{xlist}
		\TabPositions{15em}
		\ex	\tit{b-aq-b-azi-b, b-aq-il-b-azi-b} \sqt{very much, very many}\\
				~\hspace*{1em}				\tab	<		\tit{b-aq} \sqt{much, many}
		\ex	\tit{kam-b-azi-b} \sqt{very few, very little}				\tab	<		\tit{kam} \sqt{few, little}
		\ex	\tit{ʡaˁħ-b-azi-b} \sqt{very good, excellent}				\tab	<		\tit{ʡaˁħ} \sqt{good}
		\ex	\tit{qːuʁa-b-azi-b} \sqt{very beautiful}	\tab	<		\tit{qːuʁa} \sqt{beautiful} 
		\ex	\tit{ʁezbar-b-azib} \sqt{very hairy} \tab	<		\tit{ʁezbar} \sqt{hairy}		
		\ex	\tit{ʡaˁbra-b-azi-b} \sqt{very much, very many, plenty}\\
				~\hspace*{1em}				\tab	<		\tit{ʡaˁbra} \sqt{much, many}
	\end{xlist}
\end{exe}

\begin{exe}
	\ex
	\begin{xlist}
		\ex	\label{ex:I know good people}
		\gll	dam	ʡaˁħ-b-azi-b χalq'	b-alχ-ad \\
			\tsc{1sg.dat}	good-\tsc{hpl-adjvz-hpl} people	\tsc{hpl}-know.\tsc{ipfv-1.prs}\\
		\glt	\sqt{I know good people.} (E)

		\ex	\label{ex:Those people are good}
		\gll	hetːi χalq'	ʡaˁħ-b-azi-b-te	ca-b \\
			those	people	good-\tsc{hpl-adjvz-hpl-dd.pl}	\tsc{cop-hpl} \\
		\glt	\sqt{Those people are good.} (E)

		\ex	\label{ex:I remember very much}
		\gll	ʡaˁbra-d-azi-d	han d-irk-ur \\
			much-\tsc{npl-adjvz-npl}	remember \tsc{npl}-occur.\tsc{ipfv-pret} \\
		\glt	\sqt{I remember very well.}

		\ex	\label{ex:There are other engineers}
		\gll	cara-te	inžener-te=ra	meχanekːa-be=ra	cara-te	šːan-te	k'e-b	ʡaˁbra-b-azi-b	hextːu-b=ra \\
			other-\tsc{dd.pl}	engineer-\tsc{pl=add}	mechanic-\tsc{pl=add}	other-\tsc{dd.pl} 	fellow.villager-\tsc{pl}	exist.\tsc{up-hpl}	much\tsc{-hpl-adjvz-hpl}	there.\tsc{up-hpl=add}  \\
		\glt	\sqt{There are other (experts like) engineers, mechanics, among the villagers, there are plenty.}
	\end{xlist}
\end{exe}

I found three \isi{adjectives} with the suffix \tit{-či} followed by a \isi{gender}/\isi{number} agreement marker \refex{ex:adjectivesWithCHIB_1}. The base \isi{nouns} are all loans. When the \isi{adjectives} occur in predicative or nominal function they need one of the \isi{cross-categorical suffixes} -\textit{ce} or \tit{-il} \refex{ex:She is rich}, \refex{The animals had apparently more consciences thatn our rich (people).}.

\begin{exe}
	\ex	\label{ex:adjectivesWithCHIB_1}
	\begin{xlist}
		\TabPositions{10em,12em}
		\ex	\tit{dawla-či-b} \sqt{rich} 		\tab	<	\tab	\tit{dawla} \sqt{wealth}
		\ex	\tit{taliħ-či-b} \sqt{lucky, happy}  \tab	<	\tab	\tit{taliħ} \sqt{happiness, luck}
		\ex	\tit{ʡaˁq'lu-či-b} \sqt{intelligent}  \tab	<	\tab	\tit{ʡaˁq'lu} \sqt{intellect, mind}
	\end{xlist}

	\ex	\label{ex:She is rich}
	\gll	it dawla-či-r-il ca-r \\
		that	wealth-\tsc{adjvz-f-ref}	\tsc{cop-f} \\
	\glt	\sqt{She is rich.} (E)
	
	\ex	\label{The animals had apparently more consciences thatn our rich (people).}
	\gll žaniwar-t-a-lla				χʷal-le		jaˁħ=ra,	namus=ra		b-už-ib ca-b,	nišːa-lla		dawla-či-b-t-a-lla-ja-r \\
	animal-\tsc{pl-obl-gen}	big-\tsc{advz}	conscience=\tsc{add}		conscience=\tsc{add}	\tsc{n}-be-\tsc{pret}	\tsc{cop-n}		\tsc{1pl-gen}	wealth-\tsc{adjvz-hpl-pl-obl-gen-loc-abl} \\
	\glt	\sqt{The animals had apparently more consciences than our rich (people).}
	
\end{exe}

For Standard Dargwa, \citealp[212]{AbdullaevEtAl2014} have claimed that -\textit{či} is a cognate of the \isi{spatial postposition} \textit{či} `on, above' . However, for Sanzhi this cannot be true because the postposition \textit{či} can be used with both native and loan words and it governs the \tsc{loc}-series or, alternatively, the \isi{genitive} case (\refsec{ssec:postposition ci}). The \isi{adjectivizer} -\textit{či} can also not be equated with the Turkic loan suffix -\textit{či} (\refsec{subsection:Agent nouns with -či}), which derives \isi{agent} \isi{nouns} because \isi{nouns} do not inflect for \isi{gender} and the \isi{agent} \isi{nouns} do not need any further suffixes in order to be used in argument position or as predicates. Furthermore, in Standard Dargwa the form of the \isi{adjectivizer} is -\textit{če}, but the form of the borrowed \isi{nominalizer} is -\textit{či} throughout all Dagestanian languages.

Nouns denoting materials and other properties or adverbs and \isi{nouns} with temporal semantics can be inflected for the \isi{genitive} and then yield the meaning of relational \isi{adjectives} \refex{ex:adjectivesWithLA} (see also \refsec{sssec:Genitive} for more examples).
%
\begin{exe}
\ex	\label{ex:adjectivesWithLA}
		\TabPositions{12em}
		\tit{murhe-la} \sqt{golden}		\tab	\tit{ižal-la}	\sqt{today's} \\
		\tit{urcu-la} \sqt{wooden}				\tab	\tit{haniša-la} \sqt{summer} (adjective)\\
		\tit{dešːa-la} \sqt{ancient} < \tit{dešːa} \sqt{antiquity, old times}\\
		\tit{ʡaˁb-bac-la}	\sqt{three-month}			<	 \tit{ʡaˁb} \sqt{three} + \tit{bac-la} \sqt{month's} 
\end{exe}


It is possible to express attribution with a possessive construction consisting of a noun in the \isi{genitive} denoting the possessed and the noun \tit{b-ah} \sqt{owner}.\footnote{This is one of the very few \isi{nouns} that has a \isi{gender} marker. See \refsec{sec:noungender} for more information.} This construction represents a standard \isi{genitive} phrase. The noun agrees with the head noun in \isi{gender} and \isi{number}. These constructions can occur as predicates \refex{ex:They are searching for who among them is guilty} and as attributes \refex{ex:He is looking around}.
%
\begin{exe}
	\ex	\label{ex:adjectivesWithBAH}
	\begin{xlist}
		\ex	\tit{muc'ur-ra w-ah} \sqt{bearded} (\tsc{pl} \tit{muc'ur-ra b-ah-inte})
		\ex	\tit{č'imi-la b-ah} \sqt{having a tail}
		\ex	\tit{abrazovanie-la w-ah} \sqt{educated} (from Russian \tit{obrazovanie} \sqt{education})
	\end{xlist}
\end{exe}

\begin{exe}
	\ex	\label{ex:They are searching for who among them is guilty}
	\gll	umc'-un ca-b	ča 	ca-w=el	ʡaˁjb-la	w-ah	hel-tː-a-cːe-rka \\
		search.\tsc{ipfv-icvb} \tsc{cop-hpl}	who	\tsc{cop-m=indq}	guilt\tsc{-gen}	\tsc{m}-owner	that-\tsc{pl-obl-in-abl} \\
	\glt	\sqt{They are searching for who among them is guilty.}

	\ex	\label{ex:He is looking around}
	\gll	er w-ik'-ul ca-w.	ca	ul-la	b-ah	šajt'an	ka-b-isː-un-ne \\
		look \tsc{m}-look.at\tsc{.ipfv-icvb} \tsc{cop-m}	 one	eye-\tsc{gen}	\tsc{n}-owner	devil	\tsc{down-n}-sleep.\tsc{pfv-pret-cvb} \\
	\glt	\sqt{He is looking around. The devil with one eye is asleep.}
\end{exe}


%%%%%%%%%%%%%%%%%%%%%%%%%%%%%%%%%%%%%%%%%%%%%%%%%%%%%%%%%%%%%%%%%%%%%%%%%%%%%%%%

\section{Comparative constructions with adjectives}
\label{sec:Comparison}

Comparative constructions can express (i) equality or similarity, (ii) comparative, and (iii) superlative. Similarity or equality can be expressed by means of the adverbs \tit{daˁʡle} \sqt{like, as} \refex{ex:There is even a woman who looks old}, \tit{mišil} \sqt{similar} or the \isi{enclitic} \tit{=ʁuna} \refex{ex:His wife, in my mind, is bad like a dog} as well as through \isi{manner adverbs} with the meaning \sqt{like this, like that} \refex{ex:They were such good friends}.
%
\begin{exe}
	\ex	\label{ex:There is even a woman who looks old}
	\gll	heχ	xːunul	bulan	r-uqna-ce	daˁʡle	či-r-ig-ul ca-r \\
		\tsc{dem.down}	woman	even	\tsc{f}-old-\tsc{dd.sg}	as	\tsc{spr-f}-see.\tsc{ipfv-icvb} \tsc{cop-f} \\
	\glt	\sqt{There is even a woman who looks old.}

	\ex	\label{ex:His wife, in my mind, is bad like a dog}
	\gll	hež-i-la	xːunul,	di-la	pikri	ħaˁsible, χʷe	ʁuna	wahi-ce	ca-r \\
		this-\tsc{obl-gen}	woman	\tsc{1sg-gen}	thought	following	dog	\tsc{eq}	bad-\tsc{dd.sg}	\tsc{cop-f} \\
	\glt	\sqt{His wife, in my mind, is bad like a dog.}

	\ex	\label{ex:They were such good friends}
	\gll	hel-itːe	ʡaˁħ	juldašː-e	b-už-ib-te ca-b	il-tːi \\
		that-\tsc{advz}	good	friend-\tsc{pl}	\tsc{hpl}-be-\tsc{pret-dd.pl} \tsc{cop-hpl}	that-\tsc{pl} \\
	\glt	\sqt{They were such good friends.}
\end{exe}

Adjectives do not have a special comparative form. Instead, the \isi{standard of comparison} takes the \tsc{loc}-\isi{ablative} suffix \refex{ex:There were (brothers) older than grandfather}. 
%
\begin{exe}
	\ex	\label{ex:There were (brothers) older than grandfather}
	\gll	atːa-ja-r	χːula-te=ra	b-irχ-i \\
		father-\tsc{loc-abl} 	big-\tsc{dd.pl=add}	\tsc{hpl}-be.\tsc{ipfv-hab.pst} \\
	\glt	\sqt{There were (brothers) older than grandfather.} (lit. father)
\end{exe}

The superlative is formed by means of the \isi{degree adverb} \textit{bah} \sqt{most} (emphatic variant \tit{bahlalla}) that occurs before the adjective \refex{ex:Was (our) grandfather the oldest among his brothers} (for other \isi{degree adverbs} see \refsec{sec:Other syntactic properties} and \refsec{sec:Degree adverbs}).
%
\begin{exe}
	\ex	\label{ex:Was (our) grandfather the oldest among his brothers}
	\gll	bah	χːula-ce	w-irχ-i=w	χatːaj	ču-la	ucː-b-a-cːe-r \\
		most	big-\tsc{dd.sg}	\tsc{m}-be.\tsc{ipfv-hab.pst=q}	grandfather	\tsc{refl.pl-gen}	brother-\tsc{pl-obl-in-abl} \\
	\glt	\sqt{Was (our) grandfather the oldest among his brothers?}
\end{exe}

More details and additional examples of \isi{comparative constructions} can be found in \refsec{sec:Comparative constructions}.




