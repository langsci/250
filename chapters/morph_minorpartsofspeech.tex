\chapter[Particles, conjunctions, and cross-categorical suffixes]{Predicative particles and other particles, conjunctions, and cross-categorical suffixes}
\label{cpt:Minor parts of speech}

This chapter discusses the morphosyntactic properties as well as the semantic and pragmatic functions of predicative particles, conjunctions, temporal enclitics, pragmatic particles, and cross-categorical suffixes. They do not form a part of speech or a homogeneous category, although they can be subgrouped into relatively coherent classes:
%
\begin{itemize}
	\item	predicative particles (\refsec{sec:Predicative particles})
	\item	conjunctions (\refsec{sec:Conjunctions})
	\item  temporal enclitics (\refsec{sec:Temporal enclitics})
	\item	discourse and modal enclitics (\refsec{sec:Discourse and modal enclitics})
	\item	pause fillers, address particles, exclamatives, interjections, and other particles (\refsec{sec:Pause fillers, address particles, exclamatives, and interjections})
	\item	cross-categorical derivational suffixes: attributive suffixes and the adverbializing suffix (\refsec{sec:Cross-categorical derivational suffixes}).
\end{itemize}

They are mainly treated together in one chapter because they either do not fit into any of the previous chapters or because they have a special relevance for the grammar of Sanzhi such that a separate treatment is legitimate.
%%%%%%%%%%%%%%%%%%%%%%%%%%%%%%%%%%%%%%%%%%%%%%%%%%%%%%%%%%%%%%%%%%%%%%%%%%%%%%%%

\section{Predicative particles}
\label{sec:Predicative particles}

In recent studies of Dargwa varieties researchers have introduced the term \dqt{predicative particles} to refer to a closed class of grammatical elements that fulfill the functions of copula-like auxiliaries (e.g. \citeb{Sumbatova.Mutalov2003}; \citeb{Kalinina.Sumbatova2007}; \citeb{Sumbatova.Lander2014}). This means that they function as heads of nominal predicate clauses and similar clauses that do not contain other verbs, and that they are used in analytic verb forms together with non-finite verb forms in order to form full main clauses. In other words, they are responsible for the finiteness of certain clauses, and their use depends on the clause type and the TAM form. In the following, I discuss these particles for Sanzhi. I employ the label \sqt{predicative particles}, but my analysis diverges from the analysis put forward by Sumbatova and colleagues.

\reftab{tab:Predicative particles in Sanzhi} presents the predicative particles of Sanzhi. They are enclitics because they cannot form their own phonological word. They always need a host to which they attach, but unlike suffixes they can be added to various parts of speech or phrase types, that is, to verbs, but also to nominals (noun phrases), adjectives, or adverbs.

%
\begin{table}
	\caption{Predicative particles in Sanzhi}
	\label{tab:Predicative particles in Sanzhi}
	\small
	\begin{tabularx}{0.98\textwidth}[]{%
		>{\raggedright\arraybackslash}p{65pt}
		>{\raggedright\arraybackslash}p{28pt}
		>{\raggedright\arraybackslash\hangindent=0.5em}X}
		
		\lsptoprule
			particle			&	gloss		&	short description\\
		\midrule
			\tit{=da}			&	\tsc{1, 2pl}	&	person enclitic for first person singular and plural and for second person plural (see \refsec{sec:Person agreement} on agreement)\\	   
			\tit{=de}			&	\tsc{2sg}	&	person enclitic for second person singular\\   
			\tit{=de}			&	\tsc{pst}	&	past tense marker\\
			\tit{=q'al}			&	\tsc{mod}	&	modal particle (\refsec{ssec:The enclitic =q'al})\\
			\tit{=e\slash =ja}		&	\tsc{q}	&	marker for content questions (\refsec{sec:Content questions})\\
			\tit{=w\slash =uw\slash =ew}	&	\tsc{q}	&	marker for polar questions (\refsec{sec:Simple polar questions and disjunctive polar questions})\\
			\tit{=l \slash =jal\slash =el}		&	\tsc{indq}	&	marker for embedded questions (\refsec{sec:Subordinate questions})\\
		\lspbottomrule
	\end{tabularx}
\end{table}


Due to this freedom in host selection they can be used in term focus constructions (\refsec{ssec:Contrastive focus and floating predicative particles}). However, most commonly they occur in the position in which auxiliary verbs (e.g. auxiliaries expressing aspect or modality) occur, namely following the lexical verb. In copula clauses they are normally attached to the head of the predicate (\refsec{sec:copulaclauses}). They partially express verbal categories such as person or tense, but they are not verbs themselves.

\citet[138\tnd140]{Sumbatova.Mutalov2003} and \citet[153\tnd163]{Sumbatova.Lander2014} include in their list of predicative particles three more items: the standard copula, the negative copula, and locational/existential copulas. For Sanzhi these are the copula \tit{ca-b} (\refsec{sec:The copula}), the locational copulas \tit{le-b}, \tit{te-b}, \tit{k'e-b,} and \tit{χe-b} (\refsec{sec:Locational copulae}), and the negative copula \tit{(b-)akːʷ-} in its present tense and past tense forms. However, I consider these copulas to be verbs with defective paradigms that overlap in their functions with the predicative particles because they also occur in copula clauses and analytic verb forms, but they diverge from the enclitics in \reftab{tab:Predicative particles in Sanzhi} in a number of ways.

First, they are not genuine enclitics; they can occur on their own without a host and can form their own clause, though some of them may also be used in the form of enclitics. Second, they express far more verbal categories than the predicative particles. The negative copula shares a great number of inflectional forms with standard verbs (e.g. it can inflect for habitual present and habitual past, masdar, etc.). The copula and the locational copulas have the same gender/number agreement affix as other verbs (even though all other verbs have gender/number prefixes and not suffixes). They convey present time reference, third person agreement, and are specified for affirmative polarity. Third, the predicative particles can be attached to the copula and to the locational copulas, including those particles that express verbal categories (i.e. the person enclitics and the past tense enclitic), so that all copulas can express first and second person agreement or past tense \refex{ex:Yes, I was minor}, but the person enclitics and the past tense enclitic strictly exclude each other.

The predicative particles can be divided into two groups. The first consists of the enclitics that express categories, which are most commonly marked on the verb (person enclitics \tit{=da} and \tit{=de} and the past tense enclitic \tit{=de}), and the second group are the pragmatic markers (modal particle, interrogative particles). The two groups differ in their properties:

\begin{description}
\item[Verby predicative particles (person enclitics, past enclitic)]
\begin{itemize}[leftmargin=*]
    \item[]
	\item	are regularly used for the formation of analytic verb forms such as the compound present and past, preterite, perfect, etc. (\refsec{cpt:Analytic verb forms}) and certain clause types (e.g. declarative copula clauses)
	\item	cannot occur in clauses with verb forms that have person suffixes (e.g. habitual present, habitual past, conditional forms, imperative, optative, etc.)
	\item	can never occur in most types of subordinate clauses such as adverbial clauses, relative clauses, conditional clauses and many complement clauses
	\item	are in complementary distribution with each other, i.e. person enclitics and the past enclitic exclude each other
\end{itemize}

\item[Pragmatic predicative particles (modal particle, interrogative particles)]
%
\begin{itemize}[leftmargin=*]
    \item[]
	\item	are normally not used as heads of copula clauses or for the formation of analytic verb forms, though such a usage is possible for third person subject-like arguments
	\item	can occur in clauses with verb forms that have person suffixes
	\item their use is never obligatory	and they cannot replace verby predicative particles in certain verb forms and certain clause types in which the verby particles are obligatory
	\item can occur in most types of subordinate clauses; only embedded questions are excluded
	\item	are in complementary distribution with each other, i.e., the modal particle cannot co-occur with any of the interrogative particles
\end{itemize}
\end{description}

Predicative particles of the two groups can co-occur with each other \refex{ex:‎‎‎Am I will not go there, I went yesterday minor}, \refex{ex:Are you going home because of what she said minor}; the verby particles always precede the pragmatic particles, and (most) other discourse particles. This means that the interrogative markers and the modal enclitic normally occur together with a person enclitic, the past enclitic or some kind of copula, and their function is primarily pragmatic (e.g. to convey a certain modal meaning or interrogative illocutionary force) and syntactic (for the interrogative markers).
%
\begin{exe}
	\ex	\label{ex:‎‎‎Am I will not go there, I went yesterday minor}
	\gll	itːu	a-r-ax-an=da=q'al,	sːa	ag-ur=da	ʁubza\\
		there	\tsc{neg-f-}go\tsc{-ptcp=1=mod}	yesterday	go\tsc{.pfv-pret=1}	\tsc{emph}\\
	\glt	\sqt{I will not go there, I went yesterday.}

	\ex	\label{ex:Are you going home because of what she said minor}
	\gll	hel-i-la	ʁaj-li-j	qili	arg-ul=de=w	u?\\
		that\tsc{-obl-gen}	word\tsc{-obl-dat}	home	go\tsc{.ipfv-icvb=2sg=q}	\tsc{2sg}\\
	\glt	\sqt{Are you going home because of what she said?} (lit. \sqt{because of her word})
\end{exe}

There are two types of clauses that may require the use of a predicative particle instead of a copula or another type of auxiliary verb. The first type is copula clauses (\refsec{sec:copulaclauses}) and the second type is main clauses with analytic tense forms (\refcpt{cpt:Analytic verb forms}). Thus, person enclitics and the past enclitic in the sentences in \xxref{ex:copula clauses minor}{ex:Yes, I was minor} cannot be replaced by copulas or other auxiliary verbs without changing the semantics of the clause or verb form or even making the sentence ungrammatical. The copula can be added to the clauses \refex{ex:I am a master minor@a}, \refex{ex:I was a master minor@b} without noticeably altering the semantics or pragmatics of the sentences, but not to \refex{ex:I am laughing minor@c}, \refex{ex:I was laughing minor@d}. An example is provided in \refex{ex:Yes, I was minor}. This means that in the analytic verb forms the copula can never co-occur with the person markers or with the past tense enclitic.

\paragraph*{First and second person or past time reference}
%
\begin{exe}
	\ex	copula clauses	\label{ex:copula clauses minor}
	\begin{xlist}
		\ex	\label{ex:I am a master minor@a}
		\gll	du	ustːa=da\\
			\tsc{1sg}	master\tsc{=1}\\
		\glt	\sqt{I am a master.} (E)

		\ex	\label{ex:I was a master minor@b}
		\gll	du	ustːa=de\\
			\tsc{1sg}	master\tsc{=pst}\\
		\glt	\sqt{I was a master.} (E)

		\ex	\label{ex:Are you a master minor}
		\gll	u	ustːa=de=w?\\
			2sg	master\tsc{=2sg=q}\\
		\glt	\sqt{Are you a master?} (E)
	\end{xlist}

	\ex	analytic verb forms	\label{ex:analytic verb forms minor}
	\begin{xlist}
		\ex	\label{ex:I am laughing minor@c}
		\gll	du	ħaˁħaˁ	r-ik'-ul=da\\
			\tsc{1sg}	laugh	\tsc{f-}say\tsc{.ipfv-icvb=1}\\
		\glt	\sqt{I am laughing.} (E)

		\ex	\label{ex:I was laughing minor@d}
		\gll	du	ħaˁħaˁ	r-ik'-ul=de\\
			\tsc{1sg}	laugh	\tsc{f-}say\tsc{.ipfv-icvb=1}\\
		\glt	\sqt{I was laughing.} (E)

		\ex	\label{ex:Are you laughing minor}
		\gll	u	ħaˁħaˁ	r-ik'-ul=de=w?\\
			\tsc{2sg}	laughter	\tsc{f-}say\tsc{.ipfv-icvb=2sg=q}\\
		\glt	\sqt{Are you laughing?} (E)
	\end{xlist}

	\ex	{[In the year 1971 you were in the army, right?]}\\\label{ex:Yes, I was minor}
	\gll	ca-w=de\\
		\tsc{cop-m=pst}\\
	\glt	\sqt{Yes, I (masc.) was.}
\end{exe}

In clauses with third person agreement controllers the copula is normally used \refex{ex:He is a master minor}, \refex{ex:Who is she minor}. However, it can be omitted when the pragmatic predicative particles are used if the concomitant pragmatic meaning needs to be conveyed \refex{ex:In this bucket there is water minor}, \refex{ex:This is probably my money minor} or if the speaker wants to utter a question \refex{ex:Is s/he a master minor}, \refex{ex:This, who is it minor}. 

\paragraph*{Third person non-past time reference}
%
\begin{exe}
	\ex	\label{ex:He is a master minor}
	\gll	iž	ustːa	ca-w\\
		this	master	\tsc{cop-m}\\
	\glt	\sqt{He is a master.}

	\ex	{copula clauses without a copula}\label{ex:copula clauses without a copula minor}
	\begin{xlist}
	
	\ex	\label{ex:In this bucket there is water minor}
		\gll	ij	badra-cːe-d	d-i-d	hin=q'al\\
			this	bucket\tsc{-in-npl}	\tsc{npl-}in\tsc{-npl}	water\tsc{=mod}\\
		\glt	\sqt{In this bucket there is water.}

		\ex	\label{ex:This is probably my money minor}
		\gll	di-la	arc=el	hel-tːi\\
			\tsc{1sg-gen}	money\tsc{=indq}	that\tsc{-pl}\\
		\glt	\sqt{That is probably my money.} (E)
		
		\ex	\label{ex:Is s/he a master minor}
		\gll	it	ustːa=w?\\
			that	master\tsc{=q}\\
		\glt	\sqt{Is s/he a master?} (E)
		
		\ex	\label{ex:This, who is it minor}
		\gll	ij,	ča=ja	iž?\\
			this	who\tsc{=q}	this\\
		\glt	\sqt{This, who is it?}

	\end{xlist}

	\ex	{analytic verb forms without a copula}\label{ex:analytic verb forms without a copula minor}
	\begin{xlist}
		\ex	\label{ex:She is laughing minor}
		\gll	ij	ħaˁħaˁ	r-ik'-ul=q'al\\
			this	laugh	\tsc{f-}say\tsc{.ipfv-icvb=mod}\\
		\glt	\sqt{She is laughing.}

		\ex	\label{ex:What is she saying minor}
		\gll	it	ce	r-ik'-ul=e?\\
			that	what	\tsc{f-}say\tsc{.ipfv-icvb=q}\\
		\glt	\sqt{What is she saying?}

		\ex	\label{ex:They are probably making a trial minor}
		\gll	heš-tː-a-l	sud	b-irq'-ul=el\\
			this\tsc{-pl-obl-erg}	trial	\tsc{n-}do\tsc{.ipfv-icvb=indq}\\
		\glt	\sqt{They are probably making a trial.}
	\end{xlist}
\end{exe}

It is always possible to add the copula. Thus, the following two examples show copula clauses and analytic verb forms with copulas and additional predicative particles. In \refex{ex:na hel rursilijra balxul akuqal il urxab ce cabel minor} the negative copula together with the modal particle and the affirmative copula with the indirecet question marker encliticized to it are used.
%
\begin{exe}
	\ex	\label{ex:Who is she minor}
	\gll	heχ	ča 	ca-r=e?\\
		\tsc{dem}.down	who	\tsc{cop-f=q}\\
	\glt	\sqt{Who is she?}

	\ex	\label{ex:na hel rursilijra balxul akuqal il urxab ce cabel minor}
	\gll	na	hel	rursːi-li-j=ra	b-alχ-ul akːu=q'al,	il	urχːab	ce	ca-b=el\\
		now	that	girl\tsc{-obl-dat=add}	\tsc{n-}know\tsc{.ipfv-icvb}	\tsc{cop.neg=mod}	that	mill	what	\tsc{cop-n=indq}\\
	\glt	\sqt{The girl also does not know if that is a mill or not.}
\end{exe}

Without the predicative particles (or a copula or another type of suitable auxiliary) the copula clauses would be ungrammatical:
%
\begin{exe}
	\ex	\label{ex:I am a master ungrammatical minor}
	\gll	{*}	du	ustːa\\
		{}	\tsc{1sg}	master\\
	\glt	(Intended meaning: \sqt{I am a master.}) (E)

	\ex	\label{ex:Are you a master ungrammatical minor}
	\gll	{*}	u	ustːa=w?\\
		{}	\tsc{2sg}	master\\
	\glt	(Intended meaning: \sqt{Are you a master?}) (E)

	\ex	\label{ex:This, who is it ungrammatical minor}
	\gll	{*}	ij,	ča	iž?\\
		{}	this	who	this\\
	\glt	(Intended meaning: \sqt{This, who is it?}) (E)
\end{exe}

Clauses with analytic verb forms are not ungrammatical, but they can only be used as subordinate clauses because of the non-finite verb forms \refex{ex:You sit and are laughing minor}.
%
\begin{exe}
	\ex	\label{ex:You sit and are laughing minor}
	\gll	u	ħaˁħaˁ	r-ik'-ul,	ka-r-iž-ib-le=de\\
		\tsc{2sg}	laugh	\tsc{f-}say\tsc{.ipfv-icvb}	\tsc{down-f-}sit\tsc{.pfv-pret-cvb=2sg}\\
	\glt	\sqt{You (fem.) are sitting and laughing.} (E)
\end{exe}


%%%%%%%%%%%%%%%%%%%%%%%%%%%%%%%%%%%%%%%%%%%%%%%%%%%%%%%%%%%%%%%%%%%%%%%%%%%%%%%%

\section{Conjunctions}
\label{sec:Conjunctions}

Sanzhi does not have native conjunctions, and this is typical for East Caucasian languages. The main way of conjoining phrases is the use of the additive enclitic (\refsec{ssec:The additive enclitic}), and at the clause level converbs are employed (\refsec{sec:The syntax of adverbial clauses}). However, there are a number of borrowed conjunctions whose use varies.

The monosyndetic conjunction \textit{wa} \sqt{and} occurs only in translated texts. The disjunctive particle \tit{ja \ldots\ ja} `or', `and', `either \ldots\ or', `neither \ldots\ nor' mostly occurs in the disjunction of clauses \refex{ex:‎Neither Xurija herself comes nor do I go minor} or more rarely of phrases \refex{ex:‎These are probably their pumpkins or watermelons minor}. Usually both disjunctions are introduced by \textit{ja}. However, sometimes there is only one clearly identifiable disjunction member in which \tit{ja} occurs, and in such examples \tit{ja} can also function as a conjunction \refex{ex:They went and they could not kill him minor}. The complex form \tit{ja=ra} (or\tsc{=add}) is used as well \refex{ex:‎These are probably their pumpkins or watermelons minor}. See \refsec{sec:Coordination of noun phrases and other phrases} and \refsec{ssec:Disjunctive coordination of clauses} for more information on the disjunction of phrases and clauses and their syntactic properties.
%
\begin{exe}
	\ex	\label{ex:‎Neither Xurija herself comes nor do I go minor}
	\gll	ħurija	ja	ca-r	ha-r-ax-ul	akːu,	ja	du	r-ax-ul	akːʷa-di\\
		Hurija	or	\tsc{refl-f}	\tsc{up}\tsc{-f-}go\tsc{-icvb}	\tsc{cop.neg}	or	\tsc{1sg}	\tsc{f-}go\tsc{-icvb}	\tsc{cop.neg-1}\\
	\glt	\sqt{‎Neither Hurija herself comes nor do I go.}

	\ex	\label{ex:‎These are probably their pumpkins or watermelons minor}
	\gll	e,	ču-la	hel=ʁuna	qːabuʁ-e	ja=ra	qːalpuz-e	ču-la	d-urkː-ar\\
		yes	\tsc{refl.pl-gen}	that\tsc{=eq}	pumpkin\tsc{-pl}	or\tsc{=add}	watermelon\tsc{-pl}	\tsc{refl.pl-gen}	\tsc{npl-}find\tsc{.ipfv-prs}\\
	\glt	\sqt{‎These are probably their pumpkins or watermelons.}

	\ex	\label{ex:They went and they could not kill him minor}
	\gll	ha-b-eʁ-ib-le,	ja	il	kaxʷ-ij	a-b-iχ-ub\\
		\tsc{up-hpl-}go\tsc{.pfv-pret-cvb}	and	that	kill\tsc{.pfv-inf}	\tsc{neg-hpl-}be.able\tsc{.pfv-pret}\\
	\glt	\sqt{They went and they could not kill him.}
\end{exe}

The conjunction \tit{amma} \sqt{but} introduces adversative clauses. Usually these clauses refer to situations that are contrasted with earlier mentioned events and the conjunction occurs in clause-initial position rather than between two clauses \refex{ex:‎But for 600, for 600, that is not bad at all minor}, but it can also be used like a normal clause conjunction between two main clauses \refex{ex:‎They say, We will drink and eat, but not until we become drunk minor} or very rarely at the end of the clause \refex{ex:‎But isn't this similar to a prison minor}.
%
\begin{exe}
	\ex	\label{ex:‎They say, We will drink and eat, but not until we become drunk minor}
	\gll	``b-učː-an=da,	b-uk-an=da''	b-ik'-ul	ca-b,	``amma	kep	a-d-irχʷ-an=da''\\
		\tsc{n-}drink\tsc{.ipfv-ptcp=1}	\tsc{n-}eat\tsc{.ipfv-ptcp=1}	\tsc{hpl-}say\tsc{.ipfv-icvb}	\tsc{cop-hpl}	but	drinking	\tsc{neg-1/2pl-}become\tsc{.ipfv-ptcp=1}\\
	\glt	\sqt{‎They say, ```We will drink and eat, but not get drunk.''} (a kind of saying, used by people who want to drink)

	\ex	\label{ex:‎But isn't this similar to a prison minor}
	\gll	tusnaq-li-j	miši-l	akːu	amma\\
		prison\tsc{-obl-dat}	similar\tsc{-advz}	\tsc{cop.neg}	but\\
	\glt	\sqt{‎But isn't this similar to a prison.}
\end{exe}

Moreover, it is employed to mark a switch of the topic of a conversation \refex{ex:‎But for 600, for 600, that is not bad at all minor}, just as Russian \tit{a} is used (see examples \refex{ex:and / but you also already told the story of Alibatir minor}, \refex{ex:‎And Nurishka, where is she minor} below).
%
\begin{exe}
	\ex	{[topic switch to back to the previous topic, namely the price of flour]}\\\label{ex:‎But for 600, for 600, that is not bad at all minor}
	\gll	amma	urek	darš-li-j,	urek	darš-li-j	wahi-l	akːu	garam=ra\\
		but	six	hundred\tsc{-obl-dat}	six	hundred\tsc{-obl-dat}	bad\tsc{-advz}	\tsc{cop.neg}	gram\tsc{=add}\\
	\glt	\sqt{‎But for 600, for 600, that is not bad at all.}
\end{exe}

The subordinating conjunction \tit{raχle} \sqt{if} is a native item with the morphological structure of an adverbial derived by means of the adverbializing suffix \textit{-le} (compare \textit{raχ-raχle} `sometimes'). It introduces conditional clauses \refex{ex:‎If I leave prison well, healthy, if nothing happens to me minor}. Because Sanzhi has specialized conditionals for this function, \tit{raχle} always co-occurs with one of the conditional forms (\refcpt{cpt:conditionalconcessiveclauses}). The use of \tit{raχle} is optional, whereas the conditional forms are mandatory. There is another borrowed conjunction with a similar meaning, \tit{egena} (< Persian \textit{eger} `if'), which occurs only in translated or elicited clauses.
%
\begin{exe}
	\ex	\label{ex:‎If I leave prison well, healthy, if nothing happens to me minor}
	\gll	raχle	tusnaq-le-r	tːura	uq-ulle,	cik'al	a-b-iχʷ-ar,	ʡaˁħ-le	saʁ-le ...\\
		if	prison\tsc{-loc-abl}	outside	go\tsc{.m.pfv-cond.1.prs}	something	\tsc{neg-n-}be\tsc{.pfv-cond.3}	good\tsc{-advz}	healthy\tsc{-advz}\\
	\glt	\sqt{‎If I leave prison well, healthy, if nothing happens to me, [‎‎I will probably become a dentist].}
\end{exe}

Sanzhi also has a couple of conjunctions borrowed from Russian: \tit{i} \sqt{and}, \tit{a} \sqt{and, but}, \tit{no} \sqt{but}, and \tit{ili} \sqt{or} (see \citet{ForkerSubmitteda} for code switching between Sanzhi and Russian). Among them, \tit{i} and \tit{a} are very frequently used by speakers of all ages in various types of texts, most often to conjoin stretches of discourse (not necessarily sentences) in the case of \tit{i}. The conjunction \textit{a} is used to mark a switch of the discourse topic \refex{ex:and / but you also already told the story of Alibatir minor}, \refex{ex:‎And Nurishka, where is she minor}. In addition, they coordinate clauses, but do not conjoin phrases, since in this function \tit{=ra} is used (\refsec{sec:Coordination of noun phrases and other phrases}).
%
\begin{exe}
	\ex	\label{ex:‎Patima took a sack and went into the forest and on the way she met a wolf minor}
	\gll	Pat'ima-l	h-asː-ib-le	qːap=ra,	ag-ur	ca-r	wac'a-cːe	i xun-ni-sa-b	suk	b-ič-ib	ca-b	bec'\\
		Patima\tsc{-erg}	\tsc{up}-take\tsc{.pfv-pret-cvb}	sack\tsc{=add}	go\tsc{.pfv-pret}	\tsc{cop-f}	forest-\tsc{in}	and	road\tsc{-obl-ante-n}	meet	\tsc{n-}occur\tsc{.pfv-pret}	\tsc{cop-n}	wolf\\
	\glt	\sqt{‎Patima took a sack and went into the forest and on the way she met a wolf.}

	\ex	\label{ex:and / but you also already told the story of Alibatir minor}
	\gll	a	il	ʡaˁlibatir=ra	χabar	b-urs-ib=de=q'al	u-l\\
		but	that	Alibatir\tsc{=add}	story	\tsc{n-}tell\tsc{-pret=2sg=mod}	\tsc{2sg-erg}\\
	\glt	\sqt{and\slash but you also already told the story of Alibatir.}

	\ex	\label{ex:‎And Nurishka, where is she minor}
	\gll	a	Nuriška	čina-r=e?\\
		but	Nurishka	where\tsc{-f=q}\\
	\glt	\sqt{‎And Nurishka, where is she?}
\end{exe}

The Russian disjunction \tit{ili} \sqt{or} conjoins disjunctive clauses (see \refsec{ssec:Disjunctive coordination of clauses} for examples). Furthermore, it is employed in clause-initial or clause-final position when expressing uncertainty together with the indirect question marker \refex{ex:‎Or this is a married couple minor} or an interrogative particle \refex{ex:Or did he beat her minor}.
%
\begin{exe}
	\ex	\label{ex:‎Or this is a married couple minor}
	\gll	sub	xːunul=el	ili?\\
		husband	woman\tsc{=indq}	or\\
	\glt	\sqt{‎Or this is a married couple?}

	\ex	\label{ex:Or did he beat her minor}
	\gll	ili	ik'-i-l	r-it-ib-le=w	iχ?\\
		or	\tsc{dem.up}\tsc{-obl-erg}	\tsc{f-}beat.up\tsc{-pret-cvb=q}	\tsc{dem.down}\\
	\glt	\sqt{Or did he beat her?}
\end{exe}

%%%%%%%%%%%%%%%%%%%%%%%%%%%%%%%%%%%%%%%%%%%
\section{Temporal enclitics}
\label{sec:Temporal enclitics}

Temporal enclitics, i.e. enclitics used in specialized converbal clauses and for the expression of other adverbial phrases, are a group of two particles that are encliticized to verbs and nominals. Their meanings are rather adverbial \refex{ex:subordinating enclitics} but because they are phonologically dependent on a host and can be hosted by a variety of parts of speech (verbs, pronouns, nouns, adjectives) I do not categorize them as genuine adverbials but treat them separately. They most commonly occur with non-finite verb forms (participles and infinitive/subjunctive) in adverbial clauses, a usage which corresponds to temporal and non-temporal specialized converbs. This function is only briefly illustrated in the current section, and more information and examples can be found in \refsec{sec:enclitic =qella} for \tit{=qːel(la)} and \refsec{sec:enclitic =satin} for \tit{=sat\slash =satːin\slash =satːinna}. 


\begin{exe}
	\ex	\label{ex:subordinating enclitics}
	\begin{xlist}
		\TabPositions{14em}
		\ex	\tit{=qːel(la)} \sqt{when, while, because}	\tab	(simultaneity, anteriority, causality)	   
		\ex	\tit{=sat\slash =satːin\slash =satːinna} \sqt{until, before, as much as, as long as}
		\sn	~\hspace*{1em}					\tab	(posteriority, manner)	      
	\end{xlist}
\end{exe}

The enclitics are not subordinating conjunctions even if their meaning corresponds to subordinating conjunctions in other languages, because they do not fulfill the function of syntactic subordination as genuine subordinating conjunctions or complementizers would. From a morphosyntactic point of view, they can occur in subordinate clauses because they are added to non-finite verb forms that are used to function as heads of subordinate clauses due to their non-finiteness. The enclitics themselves only contribute to the semantics of those clauses, not to their syntactic properties. 

One might argue that the enclitics resemble case markers or postpositions, but in contrast to the former they are not added to oblique stem forms, and in contrast to the latter they do not govern any cases. They have phrases in their scope and they are normally encliticized to the head of the phrase that they scope over, e.g., to the noun in a noun phrase \refex{ex:‎‎‎Around as much as 700 pictures I made when we went (there) now}, \refex{ex:‎‎‎The ones that I left in passion (i.e. that fell in love with me)}, \refex{ex:‎‎‎Every fourth or fifth night there was a circle (of people) in my mother's house though it was a small house}. They share this property with the focus-sensitive particles such as the additive and the modal particles (\refsec{sec:Discourse and modal enclitics}). In the following, I will describe the functions of the two enclitics in more detail, concentrating on the uses with non-verbal hosts.

The particle \tit{=qːel(la)}, of which the short form is used more often than the long form, is encliticized to the preterite and modal participle and to the negative copula (usually in its participial form), and expresses temporal simultaneity \refex{ex:‎Grandfather died when he was 77 years old} and occasionally anteriority or causality. 

\begin{exe}
	\ex	\label{ex:‎Grandfather died when he was 77 years old}
	\gll	w-ebč'-ib	χatːaj	wer-c'a	nu	wer-ra	dus	Ø-iχ-ub=qːel\\
		\tsc{m-}die\tsc{.pfv-pret}	grandfather	seven-\tsc{ten}	well seven\tsc{-num}	year	\tsc{m-}be\tsc{.pfv-pret=}when\\
	\glt	\sqt{‎Grandfather died when he was 77 years old.}
\end{exe}

Temporal simultaneity is also expressed when it is hosted by nominals such as demonstrative pronouns, nouns, numerals or adjectives and by adverbs. In the first place, the enclitic is attached to demonstrative pronouns yielding the deictic meaning \sqt{at that time, then}, which transparently derives from the meaning of the demonstrative and the meaning of the enclitic \refex{ex:‎Socialism was at that time}. As can be seen in example \refex{ex:(Is this) that time also one and the same (shirt)?}, the enclitic can be preceded by the additive, which indicates that it is not a derivational suffix that forms temporal adverbials, but rather a syntactically independent item that scopes over the entire combination of demonstrative and additive. 

\begin{exe}
	\ex	\label{ex:‎Socialism was at that time}
	\gll	socijalizma=de	het=qːella, het=qːella het=qːella het=ʁuna parjadok le-b=de hetːu-b\\
		socialism\tsc{=pst}	that=when that=when that=when that=\tsc{eq} order exist-\tsc{n=pst} there-\tsc{n}\\
	\glt	\sqt{‎Socialism was at that time, order (tidiness) like this was at that time there.}
		
	\ex	\label{ex:(Is this) that time also one and the same (shirt)?}
		\gll it=ra=qːel	ca	ʁuna=w?\\
			that=\tsc{add}=when	one	\tsc{eq=q} \\
		\glt \sqt{(Is it) that time also one and the same (shirt)?}
\end{exe}

More rarely the enclitic appears on nominals with and without additional case markers \refex{ex:‎Because when (you are) in prison it is difficult}, and also yields the meaning `when'. For instance, a noun denoting a profession to which \tit{=qːella} is added is interpreted as \sqt{when performing the relevant profession}; a noun denoting a location plus \tit{=qːella} leads to the meaning \sqt{when being in that location} \refex{ex:‎Because when (you are) in prison it is difficult}. Furthermore, the interrogative adverb \tit{ceqːel} \sqt{when} can diachronically be analyzed as \tit{ce} \sqt{what} and \tit{=qːel}, and the indefinite pronouns \tit{ca=qːel} and \tit{ca-ca=qːel} \sqt{sometimes, from time to time} as \tit{ca} \sqt{one} plus \tit{=qːel}. 
%

\begin{exe}

	\ex	\label{ex:‎Because when (you are) in prison it is difficult}
	\gll	cellij	akːu=n	tusnaq-le-w=qːella	qihin-ne	ca-b\\
		why	\tsc{cop.neg=}but	prison\tsc{-loc-m=}when	difficult\tsc{-advz}	\tsc{cop-n}\\
	\glt	\sqt{‎Because when (you are) in prison it is difficult.}

	\ex	\label{ex:‎When it is warm (i.e. in warm places) the houses are built like thisMorph}
	\gll	guna=qːel	ca-b	hel-itːe	daˁʡle	b-arq'-ib	qal\\
		warm=when	\tsc{cop-n}	that\tsc{-advz}	as	\tsc{n-}do\tsc{.pfv-pret}	house	\\
	\glt	\sqt{‎When it is warm (i.e. in warm places) the houses are built like this.}
\end{exe} 


The enclitic \tit{=sat\slash =satːin\slash =satːinna} occurs in three different variants that are functionally equivalent, but differ in their frequency of use. It originates from the postposition \tit{sa} \sqt{in front, ago}. When it is used with the infinitive/subjunctive the meaning is \sqt{before, until} \refex{ex:‎Until I came my mother died1}, i.e. temporal posteriority, which corresponds to the meaning of the postposition from which it is derived. When the enclitic occurs with the modal participle the meaning is \sqt{as much as, as long as} \refex{ex:(The wind) began to blow as strong as it could}. More examples can be found in \refsec{sec:enclitic =satin}.

\begin{exe}
	\ex	\label{ex:‎Until I came my mother died1}
	\gll	du	sa-jʁ-ij=satːinna,	r-ebč'-ib-le=de	aba\\
		\tsc{1sg}	\tsc{hither}-come\tsc{.m.pfv-inf=}until	\tsc{f-}die\tsc{.pfv-pret-cvb=pst}	mother\\
	\glt	\sqt{‎Until (before) I came my mother died.}
	
		\ex	\label{ex:(The wind) began to blow as strong as it could}
	\gll	uf	b-ik'-ul	b-aʔ-axː-ib	b-irχ-an=satːinna\\
		blow	\tsc{n-}say\tsc{.ipfv-icvb}	\tsc{n-}begin-put\tsc{.pfv-pret}	\tsc{n-}be.able\tsc{.ipfv-ptcp=}as.much\\
	\glt	\sqt{(The wind) began to blow as strong as it could}
\end{exe}


The latter meaning is also attested when the enclitic follows nouns \refex{ex:‎‎‎Around as much as 700 pictures I made when we went (there) now}, \refex{ex:‎‎‎The ones that I left in passion (i.e. that fell in love with me)}. As both examples prove, the enclitic is directly attached to the stem (after plural suffixes) without additional case marking and therefore does not qualify as a spatial case. Furthermore, it has the entire noun phrase in its scope.
%
\begin{exe}
	\ex	\label{ex:‎‎‎Around as much as 700 pictures I made when we went (there) now}
	\gll	[gde.to	wer	darš]=sat	sːurrat	ha-jt'-un=da,	hana	ag-ur=qːel\\
		somewhere	seven	hundred=as.much	picture	\tsc{up}-take.away\tsc{.pfv-pret=1}	now	go\tsc{.pfv-pret=}when\\
	\glt	\sqt{‎‎Around as many as 700 pictures I made when we went (there) now.}

	\ex	\label{ex:‎‎‎The ones that I left in passion (i.e. that fell in love with me)}
	\gll	du	ħaˁsrat-le	b-at-ur-te,	[nuˁq-b-a-lla	t'upː-e]=sat=de\\
		\tsc{1sg}	passion\tsc{-advz}	\tsc{hpl-}let\tsc{.pfv-pret-dd.pl} hand\tsc{-pl-obl-gen}	finger\tsc{-pl=}as.much\tsc{=pst}\\
	\glt	\sqt{‎‎‎The ones that I left in passion (i.e. that fell in love with me), (they) were as much as the hand's fingers.}
\end{exe}

Finally, the enclitic can be added to demonstrative pronouns and forms manner demonstrative pronouns that are used in comparison \sqt{like this, like that, such}:
%
\begin{exe}
	\ex	\label{ex:‎[‎‎From her small finger he pulled out his parents], so big was his sister}
	\gll	hel=sat	χːula	r-eʁ-ib-le	r-už-ib-le	hel	rucːi\\
		that=as.much	big	\tsc{f-}go\tsc{.pfv-pret-cvb}	\tsc{f-}be\tsc{-pret-cvb}	that	sister\\
	\glt	\sqt{‎(‎‎From her small finger he pulled out his parents), so big was his sister.}
\end{exe}



%%%%%%%%%%%%%%%%%%%%%%%%%%%%%%%%%%%%%%%%%%%%%%%%%%%%%%%%%%%%%%%%%%%%%%%%%%%%%%%%

\section{Discourse and modal enclitics}
\label{sec:Discourse and modal enclitics}


% --------------------------------------------------------------------------------------------------------------------------------------------------------------------------------------------------------------------- %

\subsection{The additive enclitic}
\label{ssec:The additive enclitic}

The additive enclitic \tit{=ra} covers all of the functions typical for additives in East Caucasian languages and other language families:
%
\begin{enumerate}
	\item	simple bisyndetic and emphatic conjunction of phrases, usually noun phrases (but not of clauses) (see \refsec{sec:Coordination of noun phrases and other phrases} on noun phrase coordination).
	%
	\begin{exe}
		\ex	\label{ex:‎(Both) ‎‎I and my elder (brother) Zalimkhan went. Kampaj and I went to Sanzhi minor}
		\gll	Q'ampaj=ra	du=ra	ag-ur=da	Sanži\\
			Kampaj\tsc{=add}	\tsc{1sg=add}	go\tsc{.pfv-pret=1}	Sanzhi\\
		\glt	\sqt{Kampaj and I went to Sanzhi.}
	\end{exe}

	\item	additive and scalar additive function (comparable to English \sqt{also}, \sqt{too}, \sqt{as well}, and \sqt{even}), that is, used as focus-sensitive particle that associates with an element of the proposition in which it occurs and indicates that what is said about this element also holds for an alternative \refex{ex:There were also ones older than grandfather minor}. In Sanzhi, the scalar additive function is particularly frequent in negative clauses, and when the additive is encliticized to \tit{hati} \sqt{more} \refex{ex:And then (the witch) said, even more angry, Say where you are minor}.
	%
	\begin{exe}
		\ex	{[Was grandfather the oldest son?]}\\\label{ex:There were also ones older than grandfather minor}
		\gll	atːa-ja-r	χːula-te=ra	b-irχ-i\\
			father-tsc{loc-abl}	big\tsc{-dd.pl=add}	\tsc{hpl-}be\tsc{.ipfv-pst.hab}\\
		\glt	\sqt{There were also ones older than grandfather.}

		\ex	\label{ex:And then (the witch) said, even more angry, Say where you are minor}
		\gll	c'il=ra	r-ik'-ul	ca-r	hati=ra	ʡaˁsi	r-iχ-ub-le,	``čina-w=de=kːʷa	u	b-urs-a!''\\
			then\tsc{=add}	\tsc{f-}say\tsc{.ipfv-icvb}	\tsc{cop-f}	more\tsc{=add}	angry	\tsc{f-}be\tsc{.pfv-pret-cvb}		where\tsc{-m=2sg=prt}	\tsc{2sg}	\tsc{n-}say\tsc{-imp}\\
		\glt	\sqt{And then (the witch) said, even more angry, ``Say where you are!''}
	\end{exe}

	\item	in contrastive topicalization \refex{ex:And Patima agreed, saying, Good minor} and topic switch constructions: when sentence topics are switched in a narrative, such topic switches are often accompanied by adding \tit{=ra} to the switched topics \refex{ex:And Patima agreed, saying, Good minor}. Sentence \refex{ex:And now also they are like Russians minor} is from a narrative in which the speaker talks about people from the neighboring village and says that earlier they were like Russians when they were still living in the mountains and that this has not changed, but that Sanzhi people were always and are still different from them.
	%
	\begin{exe}
		\ex	{[The fox said to Patimat: Do such and such!]}\\\label{ex:And Patima agreed, saying, Good minor}
		\gll	``ʡaˁħ-le,''	r-ik'-ul ca-r.	Pat'ima=ra	razi	r-iχ-ub ca-r\\
			 good\tsc{-advz}	\tsc{f-}say\tsc{.ipfv-icvb} \tsc{cop-f}	Patima\tsc{=add}	agree	\tsc{f-}be\tsc{.pfv-pret} \tsc{cop-f}\\
		\glt	\sqt{And Patima agreed, saying, ``Good.''}

		\ex	\label{ex:And now also they are like Russians minor}
		\gll	itːi	itwaj=ra	ʡuˁrusː-e	ʁunab-te	ca-b	hana=ra\\
			those	like.this\tsc{=add}	Russian\tsc{-pl}	like\tsc{-dd.pl}	\tsc{cop-hpl}	now\tsc{=add}\\
		\glt	\sqt{And now also they are like Russians.}
	\end{exe}

	\item	adverbial conjunction \sqt{and then}: the additive introduces a clause that is part of a stretch of connected discourse.
	%
	\begin{exe}
		\ex	\label{ex:And then slowly I got to know this language, and I did my (military) service minor}
		\gll	c'il=ra	hel-tːi	bahla.bahlal	ʁaj=ra	d-aχ-ur-re,	bahla.bahlal	islužba=ra	b-iqː-ul,	 \ldots\\
			then\tsc{=add}	that\tsc{-pl}	slowly	language\tsc{=add}	\tsc{npl-}know\tsc{-pret-cvb}		slowly	service\tsc{=add}	\tsc{n-}carry\tsc{.ipfv-icvb}\\
		\glt	\sqt{And then slowly I got to know this language, and I did my (military) service ...}
	\end{exe}
\end{enumerate}

Furthermore, the additive is used in the formation of concessive clauses by adding it to conditional markers (\refsec{sec:concessiveconditionals}). It also has derivational uses, namely the derivation of indefinite pronouns (\refsec{sec:Indefinite pronouns}), collective numerals (\refsec{sec:collectivenumerals}), and direct cardinal numerals from eleven up (\refsec{sec:cardinalnumerals}). For an account of additive pronouns by means of the semantic map method see \citet{Forker2016b}.


% --------------------------------------------------------------------------------------------------------------------------------------------------------------------------------------------------------------------- %

\subsection{The enclitic \tit{=q'al}}
\label{ssec:The enclitic =q'al}

The enclitic \tit{=q'al} is a frequently occurring focus-sensitive modal particle that in its frequency of occurrence is only outstripped by the additive (273 occurrences of the modal particle vs. around 1,700 occurrences of the additive in a corpus with 46,000 tokens). It belongs to the class of predicative particles, that is, when it is used together with certain non-finite verb forms such as the imperfective converb it turns the clause into an independent main clause \refex{ex:These (pictures) do not fit (on the table or in that order) minor} (\refsec{sec:Predicative particles}). The particle cannot occur in utterances that are of a sentence type other than assertions and exclamations in the optative mood. Commands or questions are thus excluded. It occurs in main clauses \refex{ex:Yesterday I already told the story about the birch tree minor} and in subordinate clauses \refex{ex:Hamid, if you know him minor} and is usually hosted by the verb, but in verbless predications by the predicate \refex{ex:In this bucket is water, right minor}. It co-occurs with other predicative particles and follows them (e.g. person markers, past tense marker, embedded question marker), but not together with the interrogative enclitics (including the marker of embedded questions). It can be used in term focus constructions when it is encliticized to the item that is in focus. 

The function of \tit{=q'al} is to mark an utterance as presupposed, and thus as, in principle, known to the hearer (and the speaker), but potentially in need of being activated and brought to the conscious attention of the addressee, similar to English \sqt{you know} or Russian \tit{že} (which is used for the translation of \tit{=q'al} into Russian) \xxref{ex:These (pictures) do not fit (on the table or in that order) minor}{ex:Hamid, if you know him minor}.
%
\begin{exe}
	\ex	\label{ex:These (pictures) do not fit (on the table or in that order) minor}
	\gll	a-ka-d-urc-ul=q'al	iš-tːi\\
		\tsc{neg-down}\tsc{-npl-}keep\tsc{.ipfv-icvb=mod}	this\tsc{-pl.abs}\\
	\glt	\sqt{These (pictures) do not fit (on the table or in that order).}\newline[presupposition: we both can see this, you should agree with me on this point]

	\ex	\label{ex:Yesterday I already told the story about the birch tree minor}
	\gll	biriz-la	kːalkːi-la	b-urs-ib=da=q'al	χabar	sːa\\
		birch\tsc{-gen}	tree\tsc{-gen}	\tsc{n-}tell\tsc{-pret=1=mod}	story	yesterday\\
	\glt	\sqt{Yesterday I already told the story about the birch tree.}\newline[presupposition: you should know and remember since you were present]

	\ex	\label{ex:If (the drinks) would be good, he would drink them minor}
	\gll	ʡaˁħ-te	d-iχʷ-ardel,	heχ-i-l	d-učː-an=de=q'al\\
		good\tsc{-dd.pl} 	\tsc{npl-}be\tsc{.pfv-cond.pst}	\tsc{dem.down-obl-erg} \tsc{npl-}drink\tsc{.ipfv-ptcp=pst=mod}\\
	\glt	\sqt{If (the drinks) were, he would drink them.}\newline[presupposition: you know that men never refuse good drinks]
\end{exe}

Example \refex{ex:Hamid, if you know him minor} was uttered when the speaker was talking about an event in which Hamid was involved and supposes that the addressee knows Hamid, which is the case.
%
\begin{exe}
	\ex	\label{ex:Hamid, if you know him minor}
	\gll	ħaˁmid	w-alχ-atːe=q'al	at, ...\\
		Hamid	\tsc{m-}know\tsc{.ipfv-cond.2=mod}	\tsc{2sg.dat}\\
	\glt	\sqt{Hamid, if you know him ...}
\end{exe}

The addressee is sometimes implicitly or explicitly asked to agree with the speaker \refex{ex:In this bucket is water, right minor}. In \refex{ex:Gather did, (we/you) don't say so minor} the speaker criticizes the use of the Russian verb \textit{sabrat} `gather, collect' instead of a native term and invites the hearer to agree with her and to remember the Sanzhi word.
%
\begin{exe}
	\ex	\label{ex:In this bucket is water, right minor}
	\gll	ij	badra-cːe-d	d-i-d	hin=q'al,	akːu=w?\\
		this	bucket-\tsc{in-npl}	\tsc{npl-}in\tsc{-npl}	water\tsc{=mod}	\tsc{cop.neg=q}\\
	\glt	\sqt{In this bucket is water, right?}

	\ex	\label{ex:Gather did, (we/you) don't say so minor}
	\gll	sabrat		d-arq'-ib,	herʔ-an	akːu=q'al\\
		gather	\tsc{npl-}do\tsc{.pfv-pret}	say\tsc{.ipfv-ptcp}	\tsc{cop.neg=mod}\\
	\glt	\sqt{``sabrat'' did, (we/you) don't say so.}\newline[presupposition: we both know that we have our own Sanzhi word for this]
\end{exe}

It is also used in questions with a strong presupposition that the addressee knows the answer \refex{ex:‎These herbs here, how are they called in our (language) minor}.
%
\begin{exe}
	\ex	\label{ex:‎These herbs here, how are they called in our (language) minor}
	\gll	hej	q'ar	ce=jal	b-ik'-u=q'al	nišːa-la?\\
		this	herbs	what\tsc{=indef}	\tsc{hpl-}say\tsc{.ipfv-prs.3=mod}	\tsc{1pl-gen}\\
	\glt	\sqt{‎These herbs here, how are they called in our (language)?}
\end{exe}

Another common usage is existential clauses with the locational copula \tit{le-b}, confirming the existence of a referent that is going to be the topic of the following discourse \refex{ex:There is the one that does not burn, the cow-parsnip minor}. They seem to correspond to the Russian phrase \tit{X est' že}, a typical Dagestanian expression that is almost never used by speakers of Standard Russian.
%
\begin{exe}
	\ex	\label{ex:There is the one that does not burn, the cow-parsnip minor}
	\gll	a-rurg-an	le-b=q'al	it,	birikːalla.ʁut'	le-b=q'al	het\\
		\tsc{neg-}burn\tsc{-ptcp}	exist\tsc{-n=mod}	that	cow.parsnip	exist\tsc{-n=mod}	that\\
	\glt	\sqt{There is the one that does not burn, the cow-parsnip.}
\end{exe}

If the information is new, it is still treated as presupposition that requires immediate accommodation. For instance, in \refex{ex:In Sanzhi, they did not have them (the iron item that you use on wooden ploughs) minor} the speaker is talking about how his grandfather for the first time brought iron parts for ploughs to Sanzhi which were unknown in Sanzhi and the addressee does not necessarily know this fact, but it is marked as presupposed by means of \tit{=q'al}.
%
\begin{exe}
	\ex	\label{ex:In Sanzhi, they did not have them (the iron item that you use on wooden ploughs) minor}
	\gll	Sanži-d	d-a-d-už-ib-le=q'al	hel-tːi\\
		Sanzhi\tsc{-npl}	\tsc{npl-neg-npl-}be\tsc{-pret-cvb=mod}	that\tsc{-pl.abs}\\
	\glt	\sqt{In Sanzhi, they did not have them (the iron item that you use on wooden ploughs).}
\end{exe}

The enclitic \tit{=q'al} widely occurs in other Dargwa varieties. In Standard (Akusha) Dargwa, there is \tit{q'alli}, which \citet[748\tnd75]{vandenBerg2001} analyzes as a focus particle. \citet{Tatevosov2001} analyzes Icari Dargwa \tit{=q'al} as a mirative marker. \citet{Sumbatova2009} in her account of questions in Icari calls it a focus-marking clitic translated with \sqt{but} in the glosses and described as marking the proposition \dqt{as known to both communicants}. In the Icari grammar as well as in \citet{Kalinina.Sumbatova2007} the same enclitic is also glossed with \sqt{but} and described as \dqt{actualization particle}. \citet[338\tnd339]{Sumbatova.Lander2014} treat Tanti Dargwa \tit{=q'ale} as an actualizing marker with a functional range very similar to the Sanzhi particle. For a detailed analysis of the morphosyntactic and semanto-pragmatic properties of \tit{=q'al} in Sanzhi see \citet{ForkerSubmittedc}.


% --------------------------------------------------------------------------------------------------------------------------------------------------------------------------------------------------------------------- %

\subsection{The enclitic \tit{=q'ar}}
\label{ssec:The enclitic =q'ar}

The enclitic \tit{=q'ar} is a modal particle that partially overlaps in its uses with \tit{=q'al} (\refsec{ssec:The enclitic =q'al}) and \tit{=n(u)} (\refsec{ssec:The enclitic =n(u)}). Like \tit{=q'al} it cannot be used in questions together with the interrogative particles. However, it does not belong to the class of predicative particles. It bears some resemblances to German \tit{doch} and Russian \tit{že}.

The enclitic is used when correcting utterances \refex{ex:These are not potato sacks, I say, they are people minor} or contradicting expectations. Thus, the speaker of \refex{ex:Apparently they knew it and they did not tell it to me minor} expected her children to inform her about the death of her son, but in order to preserve the mother from the very devastating news they did not tell her everything, but discussed the issue only among themselves.
%
\begin{exe}
	\ex	\label{ex:These are not potato sacks, I say, they are people minor}
	\gll	kartuška-la	qːup-re=q'ar	akːu,	Ø-ik'-ul=da,	χalq'	ca-b,	Ø-ik'-ul=da,	heχ-tːi\\
		potato\tsc{-gen}	sack\tsc{-pl=mod}	\tsc{cop.neg}	\tsc{m-}say\tsc{.ipfv-icvb=1}	people	\tsc{cop-hpl}	\tsc{m-}say\tsc{.ipfv-icvb=1}	\tsc{dem.down}\tsc{-pl}\\
	\glt	\sqt{These are not potato sacks, I say, they are people.}

	\ex	\label{ex:Apparently they knew it and they did not tell it to me minor}
	\gll	hel-tː-a-l,	b-aχ-ur-re	b-už-ib	ca-b	hel-tː-a-j,	di-cːe=q'ar	a-b-urs-ib\\
		that\tsc{-pl-obl-erg}	\tsc{n-}know\tsc{.pfv-pret-cvb}	\tsc{n-}stay\tsc{-pret}	\tsc{cop-n}	that\tsc{-pl-obl-dat}	\tsc{1sg-in=mod}	\tsc{neg-n-}tell\tsc{.pfv-pret}\\
	\glt	\sqt{Apparently they knew it and they did not tell it to me.}
\end{exe}

More generally, \tit{=q'ar} signals contrast between the utterance in which it occurs and some other utterance or previously discussed issues, i.e., it marks contrastive topicalization (`and as for X, P'). For example, in \refex{ex:‎After he came back, he refused very well minor} the speaker contrasts the behavior of a person after he had been in prison with his behavior before he went to prison, when he never refused a drink with his friends. Similarly, \refex{ex:The (fox) says, For me they made a bed from herbs and grass, and for themselves they took probably cotton wool minor} exemplifies a parallel structure of two clauses that immediately follow each other and contain contrasting propositions.
%
\begin{exe}
	\ex	\label{ex:‎After he came back, he refused very well minor}
	\gll	heχ	sa-jʁ-ib hitːi,	čar	Ø-iχ-ub zamana=q'ar,	ʡaˁħ-le	qːuʁa-l	atkaz	Ø-iχ-ub	ca-w\\
		\tsc{dem.down}	\tsc{hither}-come\tsc{.m.pfv-pret}	after	back	\tsc{m-}be\tsc{.pfv-pret}	time\tsc{=mod}	good\tsc{-advz}	beautiful\tsc{-advz}	refusal	\tsc{m-}be\tsc{.pfv-pret}	\tsc{cop-m}\\
	\glt	\sqt{‎After he came back, he refused very well.}

	\ex	\label{ex:The (fox) says, For me they made a bed from herbs and grass, and for themselves they took probably cotton wool minor}
	\gll	``dam=q'ar,''	b-ik'-ul	ca-b,	``nekʷ-la	buruš	b-arq'-ib,	a ču-la	baˁmbag-la	b-urkː-ar''\\
		\tsc{1sg.dat=mod}	\tsc{n-}say\tsc{.ipfv-icvb}	\tsc{cop-n}	straw\tsc{-gen}	mattress	\tsc{n-}do\tsc{.pfv-pret	}	but		\tsc{refl.pl-gen}	cotton.wool\tsc{-gen}	\tsc{n-}find\tsc{.ipfv-prs}\\
	\glt	\sqt{The (fox) says, ``For me they made a bed from herbs and grass, and for themselves they took probably cotton wool.''}
\end{exe}

Similarly to \tit{=q'al} as described above, the enclitic \tit{=q'ar} is also used as an actualizing modal particle that relates the utterance to the argumentative background and in this way indicates what is assumed to be common ground. In other words, it signals what the speaker assumes to be known by the hearer. For instance, \refex{ex:‎‎‎They know the time when (the wife) comes back from work minor} is from a narrative about some people who stole money during the absence of the main character and his wife. The speaker stresses the fact that it is clear to everyone that the people knew the times of the day when nobody was at home and when the wife was supposed to come home again. And \refex{ex:The rain does not reach there minor} describes the place close to Sanzhi where there are old paintings on rocks that are still visible, although they are assumed to have been made thousands of years ago, and what the speaker says is a fact known to every Sanzhi person.
%
\begin{exe}
	\ex	\label{ex:‎‎‎They know the time when (the wife) comes back from work minor}
	\gll	ʡaˁči-le-r	sa-q'-aˁn	zamana	sːaˁʡaˁt-e=q'ar	d-alχ-ul ca-d	hex-tː-a-j\\
		work\tsc{-loc-abl}	\tsc{hither}-go\tsc{-ptcp}	time	hour\tsc{-pl=mod}	\tsc{npl-}know\tsc{.ipfv-icvb}	\tsc{cop-npl}	\tsc{dem.up}\tsc{-pl-obl-dat}\\
	\glt	\sqt{‎‎‎They know the time when (the wife) comes back from work.}

	\ex	\label{ex:The rain does not reach there minor}
	\gll	ixtːu=q'ar	marka	či-ikː-ul	akːu\\
		there.\tsc{up=mod}	rain	\tsc{on}-get\tsc{.ipfv-icvb}	\tsc{cop.neg}\\
	\glt	\sqt{The rain does not reach there.}
\end{exe}


% --------------------------------------------------------------------------------------------------------------------------------------------------------------------------------------------------------------------- %

\subsection{The enclitic \tit{=n(u)}}
\label{ssec:The enclitic =n(u)}

The enclitic \tit{=nu} (allomorph \tit{=n} after vowels) is used when the speaker wants to attract the attention of the addressee. It is mostly encliticized to verbs. Its meaning can be paraphrased as \sqt{watch out, pay attention, something is happening or is going to happen in the near future that is of relevance and important for you}. There are several contexts in which it usually occurs. For example, \tit{=n(u)} often occurs in clauses with first person subject-like arguments when the speaker wants to stress the fact that s/he is already performing an action or is in a certain state or is about to perform an action in the near future \refex{ex:‎Now wait, I (fem.) am coming minor}.
%
\begin{exe}
	\ex	\label{ex:‎Now wait, I (fem.) am coming minor}
	\gll	hana	t'aš	r-icː-e!	r-ax-ul=da=n\\
		now	stop	\tsc{f-}stand\tsc{.pfv-imp}	\tsc{f-}go\tsc{-icvb=1=prt}\\
	\glt	\sqt{‎Now wait, I (fem.) am coming.}
\end{exe}

Such clauses can also have second or third person subjects, but again they warn that soon something will happen that is of importance for the addressee \refex{ex:‎‎[Your sister turned into a monster, she ate people], and she will eat you (masc.), don't go minor}.
%
\begin{exe}
	\ex	\label{ex:‎‎[Your sister turned into a monster, she ate people], and she will eat you (masc.), don't go minor}
	\gll	u=ra	ukː-an=de=n,	maˁ-q'-aˁtːa!\\
		\tsc{2sg=add}	eat\tsc{.m.ipfv-ptcp=2sg=prt}	\tsc{proh-}go\tsc{-proh.sg}\\
	\glt	\sqt{‎‎[Your sister turned into a monster, she ate people], and she will eat you (masc.), don't go!}
\end{exe}

The enclitic is part of the phrase \tit{celij akːu=n} (why \tsc{cop.neg=}but) with the meaning \sqt{because}. It introduces clauses that deliver an important explanation that the speaker wants the addressee to pay attention to \refex{ex:Because when you are in prison it is difficult minor}.
%
\begin{exe}
	\ex	\label{ex:Because when you are in prison it is difficult minor}
	\gll	cellij	akːu=n	tusnaq-le-w=qːella	qihin-ne	ca-b\\
		why	\tsc{cop.neg=prt}	prison\tsc{-loc-m=}when	difficult\tsc{-advz}	\tsc{cop-n}\\
	\glt	\sqt{Because when you are in prison it is difficult.}
\end{exe}

The second context is the use with imperatives and optatives, because they also occur in utterances that are of special importance and relevance for the addressee who, for instance, has been ordered to do something \refex{ex:‎‎‎If you want, take the (stuff) and empty it here in our place, go minor}.
%
\begin{exe}
	\ex	\label{ex:‎‎‎If you want, take the (stuff) and empty it here in our place, go minor}
	\gll	nu	b-ikː-aχː-at,	nišːa-la	heχtːu	d-uk-a=n,	d-ac'	d-arq'-a=nu,	uq'-aˁ=nu!\\
		well	\tsc{n-}want\tsc{.ipfv-cond-cond.2}	\tsc{1pl-gen}	there.\tsc{down}	\tsc{npl-}gather\tsc{.ipfv-imp=}but	\tsc{npl-}empty	\tsc{npl-}do\tsc{.pfv-imp=prt}		go\tsc{.m-imp=}but\\
	\glt	\sqt{‎‎‎If you want, take the (stuff) and empty it there in our place, go!}
\end{exe}

The third context is contrastive focus constructions that are used to correct wrong assumptions, assertions or beliefs. The enclitic occurs in the clause that rejects the assertion and is followed by the correction:
%
\begin{exe}
	\ex	\label{ex:‎No, not home, they take him to the sobering-up station minor}
	\gll	qili	akːu=nu,	witrezwitel-le	uqː-ul	ca-w\\
		home	\tsc{cop.neg=prt}	sobering.up\tsc{-loc}	carry\tsc{.m.pfv-icvb}	\tsc{cop-m}\\
	\glt	\sqt{‎No, not home, they take him to the sobering-up station.}

	\ex	\label{ex:‎No, he is not going towards them, but sitting together with them minor}
	\gll	či-haˁ-q'-uˁn-ne=kːu=n	ka-jž-ib	ca-w	hel-tː-a-cːella	w-alli\\
		\tsc{spr-up}-go\tsc{-pret-cvb=}\tsc{cop.neg=}but	\tsc{down}-remain\tsc{.m.pfv-pret}	\tsc{cop-m}	that\tsc{-pl-obl-comit}	\tsc{m-}together\\
	\glt	\sqt{‎No, he is not going towards them, but sitting together with them.}
\end{exe}

It is not necessary that the rejecting clause contain a negation; it can also be an affirmative clause that functions as a correction. For instance, people tried to destroy a mill by hitting the turning mill stone. They did not immediately succeed although they tried hard and thus \refex{ex:The heart remained, hit it! The heart remained, hit it minor} contradicts the expectation that they had already finished their destruction.
%
\begin{exe}
	\ex	\label{ex:The heart remained, hit it! The heart remained, hit it minor}
	\gll	urk'i	b-el=nu,	b-aˁq-aˁjaˁ!	urk'i	b-el=nu,	b-aˁq-aˁjaˁ!\\
		heart	\tsc{n-}remain\tsc{.pfv=prt}	\tsc{n-}hit\tsc{.pfv-imp.pl}	heart	\tsc{n-}remain\tsc{.pfv=prt} \tsc{n-}hit\tsc{.pfv-imp.pl}\\
	\glt	\sqt{The heart remained, hit it! The heart remained, hit it!}
\end{exe}


% --------------------------------------------------------------------------------------------------------------------------------------------------------------------------------------------------------------------- %

\subsection{Other enclitics that manipulate information structure}
\label{ssec:Further enclitics that manipulate the information structure}

Sanzhi has further focus-sensitive enclitics with a more specific semantics: \tit{=cun} \sqt{only} \refex{ex:‎‎‎For one month I was cured only with pills, for 30 days minor}, \tit{=gina} \sqt{alone, only} \refex{ex:Only I / I alone will go home, not you minor@6}, \tit{malle} \sqt{even} \refex{ex:‎‎‎Even his sisters daughter did he throw out, after some years minor}, and \tit{arrah} \sqt{at least} \refex{ex:‎‎‎Give me at least the skin back minor}. Note that in \refex{ex:Only I / I alone will go home, not you minor@6} the enclitic \tit{=gina} is followed by a person enclitic; the reverse order would be ungrammatical. See also \refsec{sec:Focus-sensitive particles} for more information on the position and use of focus-sensitive particles.
%
\begin{exe}
	\ex	\label{ex:‎‎‎For one month I was cured only with pills, for 30 days minor}
	\gll	ca	bac	darman-t-a-lla=cun	lečenie	b-arq'-ib=da,	ʡaˁb-c'al	bari\\
		one	month	medicine\tsc{-pl-obl-gen=}only	cure	\tsc{n-}do\tsc{.pfv-pret=1}	three-\tsc{ten}	day\\
	\glt	\sqt{‎‎‎For one month I was cured only with pills, for 30 days.}

	\ex	\label{ex:Only I / I alone will go home, not you minor@6}
	\gll	du=gina=da	qili	arg-an,	u	akːʷa-tːe\\
		\tsc{1sg=}only\tsc{=1}	home	go\tsc{.ipfv-ptcp}	\tsc{2sg}	\tsc{cop.neg-2sg}\\
	\glt	\sqt{Only I\slash I alone will go home, not you.} (E)

	\ex	\label{ex:‎‎‎Even his sisters daughter did he throw out, after some years minor}
	\gll	rucːi-la	rursːi	hel	malle	t'ut'u	r-arq'-ib-le	čum=el	dus	hitːille\\
		sister\tsc{-gen}	girl	that	even	throw.out	\tsc{f-}do\tsc{.pfv-pret-cvb}	how.many\tsc{=indq}	year	later\\
	\glt	\sqt{‎‎‎Even his sister's daughter did he throw out, after some years.}
\end{exe}

The particle \tit{arrah} \sqt{at least} is used in commands \refex{ex:‎‎‎Give me at least the skin back minor}, irrealis conditional clauses, and negative clauses together with the quantifier \tit{ca} \sqt{on} with a scalar additive meaning \refex{ex:Of our (people) not even one man fell down, he says minor}. It mostly occurs following nominals and then has scope over the nominals, but it can also scope over verbal predicates. In the latter case, it is possible to insert the particle between the locational and the deixis/gravitation preverbs. For instance, in \refex{ex:Let's go if you did not go there, at least to see minor} the verb is prefixed with the locational preverb \textit{či}- and the particle follows it. This preverb is a lexicalized part of the verb `see' because the root almost never occurs without the preverb, and thus the particle is inserted into a verbal stem.
%
\begin{exe}
	\ex	\label{ex:‎‎‎Give me at least the skin back minor}
	\gll	kːul-be	arrah	d-iqː-a	dam\\
		skin\tsc{-pl}	at.least	\tsc{npl-}carry\tsc{.ipfv-imp}	\tsc{1sg.dat}\\
	\glt	\sqt{‎‎‎Give me at least the skin back!}

	\ex	\label{ex:Of our (people) not even one man fell down, he says minor}
	\gll	``nišːa-la	ca	arrah	admi,''	Ø-ik'-ul	ca-w,	``a-ka-jč-ib''\\
		\tsc{1pl-gen}	one	at.least	person	\tsc{m-}say\tsc{.ipfv-icvb}	\tsc{cop-m}	\tsc{neg-down}-occur\tsc{.m.pfv-pret}\\
	\glt	\sqt{``Of our (people) not even one man fell down,'' he says.}

	\ex	\label{ex:Let's go if you did not go there, at least to see minor}
	\gll	w-aš-e	a-ag-ur-il	Ø-iχ-utːe,	či=arrah-b-až-ij\\
		\tsc{m-}go\tsc{-imp}	\tsc{neg-}go\tsc{.pfv-pret-ref}		\tsc{m-}be\tsc{.pfv-cond.2sg.prs} \tsc{spr=}at.least\tsc{-n-}see\tsc{.pfv-inf}\\
	\glt	\sqt{Let's go if you did not go there, at least to see.}
\end{exe}

There is an emphatic enclitic \tit{=le}, which, however, occurs only twice in the corpus, and speakers do not have clear intuitions about its meaning, making it difficult to analyze in detail. These are the two examples:
%
\begin{exe}
	\ex	\label{ex:Among the graves, in the grass, who finds (him), at night minor}
	\gll	c'elt-m-a-cːe-w	q'ar-ri-cːe-w	hi-l	urkː-u=le	dučːi-la	itːu\\
		gravestone\tsc{-pl-obl-in}\tsc{-m}	herbs\tsc{-obl-in}\tsc{-m}	who\tsc{.obl-erg}	find\tsc{.m.ipfv-prs=emph} night\tsc{-gen}	there\\
	\glt	\sqt{Among the graves, in the grass, who finds (him), at night.}

	\ex	\label{ex:‎(Look at) the way the girl is holding the child in her hands minor}
	\gll	iχ	rursːi-la qːuʁa-l	kʷi-sa-b-uc-ala=le	nik'a-ce\\
		\tsc{dem.down}	girl\tsc{-gen}	beautiful\tsc{-advz}	\tsc{in.hands-hither-n-}keep\tsc{.pfv-nmlz=emph}	small\tsc{-dd.sg}\\
	\glt	\sqt{‎(Look at) the way the girl is holding the child in her hands.}
\end{exe}

And there is another enclitic \tit{=k'u} that is also roughly described as emphatic or modal. Like the two modal enclitics \tit{=q'al} and \tit{=q'ar} it is usually translated by \textit{že} or \textit{ved'} into Russian. The enclitic is also used for the formation of specific indefinite pronouns (\refsec{ssec:Specific indefinite pronouns}). In the corpus, there are three occurrences of the emphatic/modal use, of which two are given here:
  
\begin{exe}
	\ex	\label{I said, ``You said that they did not find him.''}
	\gll	``u=k'u	ik'ʷ-a-tːe,''	haʔ-ib=da,	``w-arčː-ib-le=kːu'' \\
		\tsc{2sg=emph}	say.\tsc{ipfv.m-hab.pst-2sg}	say.\tsc{pfv-pret=1}		\tsc{m}-find.\tsc{pfv-pret-cvb}=\tsc{cop.neg} \\
	\glt	\sqt{I said, ``You said that they did not find him.''}
	
		\ex	\label{One year, you should know it, where the places were, Mahammad.}
	\gll	ca	dus=k'u,	ašːi-j	b-aχ-ij	d-urkː-a-tːa	čina	musːa-t=te=l,	Maˁħaˁmmad\\
		one	year=\tsc{emph}	\tsc{2pl-dat}	\tsc{n}-know.\tsc{pfv-inf}	\tsc{npl}-find.\tsc{ipfv-hab.pst-2pl}		where	place-\tsc{pl=pst=indq}	Mahammad\\
	\glt	\sqt{One year, you should know it, where the places were, Mahammad.}

\end{exe}


Interrogative markers for polar questions (\refsec{sec:Simple polar questions and disjunctive polar questions}), content questions (\refsec{sec:Content questions}), and embedded questions (\refsec{sec:Subordinate questions}) also play a role in the information structure of utterances and are analyzed in separate sections.


%%%%%%%%%%%%%%%%%%%%%%%%%%%%%%%%%%%%%%%%%%%%%%%%%%%%%%%%%%%%%%%%%%%%%%%%%%%%%%%%

\section{Pause fillers, address particles, exclamatives, and interjections}
\label{sec:Pause fillers, address particles, exclamatives, and interjections}

Sanzhi has two politeness particles that are used in imperatives and prohibitives in order to soften the command, \tit{=kːʷa} and the rarely used \tit{=ri}. The first enclitic is also used in polite questions \refex{ex:‎Uh, what was it (that I wanted to say) minor}. In my corpus there are 31 occurrences of \tit{=kːʷa} (and just one of \tit{=ri}), and two thirds of them were uttered by female speakers. Thus, it might be the case that the use of \tit{=kːʷa} is more common among female speakers.
%
\begin{exe}
	\ex	\label{ex:‎‎‎Calm down minor}
	\gll	r-už-e=ri!\\
		\tsc{f-}be\tsc{-imp=prt}\\
	\glt	\sqt{‎‎‎Calm down!}

	\ex	\label{ex:‎‎‎Do not keep the spider like this minor}
	\gll	hel-itːe	ma-b-urc-itːa=kːʷa	paʔuk!\\
		that\tsc{-advz}	\tsc{proh-n}-keep\tsc{.ipfv-proh.sg=prt}	spider\\
	\glt	\sqt{‎‎‎Do not keep the spider like this!}

	\ex	\label{ex:Then I said to Sanijat, hey, take, these are for you minor}
	\gll	c'il	heba	Sanijat-li-cːe,	``ma,	ha,	ma=kːʷa''	haʔ-ib=da,	``at	heštːi!''\\
		then	then	Sanijat\tsc{-obl-in}	take	uh	take\tsc{=prt}	say\tsc{.pfv-pret=1}	\tsc{2sg.dat}	these\\
	\glt	\sqt{Then I said to Sanijat, ``Hey, take, these are for you!''}

	\ex	\label{ex:‎Uh, what was it (that I wanted to say) minor}
	\gll	ha	ce=de=kːʷa?\\
		uh	what\tsc{=pst=prt}\\
	\glt	\sqt{‎Uh, what was it (that I wanted to say)?}
\end{exe}

The genitive reflexive pronouns \tit{cinna} (singular) and \tit{čula} function as pause fillers. The same has been reported for the neighboring Dargwa variety Icari \citep[187, fn.~107]{Sumbatova.Mutalov2003}. It seems that the singular pronoun occurs when the subject-like argument is singular \refex{ex:He is a person that is very extroverted minor@7a}, \refex{ex:‎He is really thinking (or worrying), and sitting, it looks like he is in prison minor}, and the plural pronoun when it is plural \refex{ex:Well, probably they are like under the ground (growing) minor@7b}. The full paradigms of the reflexive pronouns are listed in \refsec{sec:Reflexive pronouns} and their use in reflexive constructions is analyzed in \refsec{sec:Reflexive constructions}.
%
\begin{exe}
	\ex	\label{ex:He is a person that is very extroverted minor@7a}
	\gll	heχ	cinna	c'aq'-le	w-artaq-ib	admi	ca-w\\
		\tsc{dem.down}	pause.filler	very\tsc{-advz}	\tsc{m-}enjoy.oneself\tsc{.pfv-pret}	person	\tsc{cop-m}\\
	\glt	\sqt{He is a person that is very extroverted.}

	\ex	\label{ex:‎He is really thinking (or worrying), and sitting, it looks like he is in prison minor}
	\gll	nu	hež	dejstwitelno	pikri	Ø-ik'-ul	ka-jž-ib	ca-w	cinna		tusnaq-le	ka-jž-ib-il-li-j	miši-l	ca-w	iž\\
		well	this	really	thought	\tsc{m-}say\tsc{.ipfv-icvb}	\tsc{down}-remain\tsc{.m.pfv-pret}	\tsc{cop-m}	pause.filler	prison\tsc{-loc}	\tsc{down}-be\tsc{.m.pfv-pret-ref-obl-dat}	similar\tsc{-advz}	\tsc{cop-m}	this\\
	\glt	\sqt{‎He is really thinking (or worrying), and sitting, because it is like he is in prison.}

	\ex	\label{ex:Well, probably they are like under the ground (growing) minor@7b}
	\gll	čula	d-urkː-ar	iχ-tːi	ganza-l-gu-d	gu-d	daˁʡle	ca-d\\
		pause.filler	\tsc{npl-}find\tsc{.ipfv-cond.3}	\tsc{dem.down}\tsc{-pl}	ground\tsc{-obl-sub-npl}	down\tsc{-npl}	as	\tsc{cop-npl}\\
	\glt	\sqt{Well, probably they are like under the ground (growing).}
\end{exe}

It is not always easy to identify the pause fillers because often the reflexive pronouns can be interpreted as possessive pronouns with an omitted head noun. For instance, example \refex{ex:Well, probably they are like under the ground (growing) minor@7b} refers to a picture showing plants or roots that grow in the earth and the reflexive \tit{čula} could serve as a pronoun in a phrase like \sqt{their (plants)}.

Common address particles are \refex{ex:pause fillers interjections minor}. Some examples are given in \xxref{ex:‎He came, and she said (to him), Hey Iljas. (He said), Hi minor}{ex:‎Ooh, my dear boy, I thought, who are you minor}.
%
\begin{exe}
	\ex	\label{ex:pause fillers interjections minor}
	\begin{xlist}
		\ex	\tit{ja, wa} \sqt{ey, hey}	
		\ex	\tit{haj} \sqt{hi, oh} (informal answer to greeting and astonishment)	
		\ex	\tit{ej} \sqt{eh}	
		\ex \textit{ulkːa(s)} `hey'
		\ex	\tit{žan} \sqt{beloved}	
	\end{xlist}

	\ex	\label{ex:‎He came, and she said (to him), Hey Iljas. (He said), Hi minor}
	\gll	ha-jʁ-ib,	r-ik'ʷ-ar,	``wa	Iljas.''	``haj''\\
		\tsc{up}-come\tsc{.m.pfv-pret}	\tsc{f-}say\tsc{.ipfv-prs}	hey	Ilyas		hi\\
	\glt	\sqt{‎He came, and she said (to him), ``Hey Ilyas.'' (He said), ``Hi.''}

\end{exe}

The particle \tit{ulkːa(s)} is used as an address term when trying to prompt the reaction of the addressee, for instance when asking him to answer a question \refex{ex:Hey, Hasanali, tell the truth minor}, but it also seems to be a pause filler. 

%
\begin{exe}
	\ex	\label{ex:Hey, Hasanali, tell the truth minor}
	\gll	ulkːas,	ħaˁsanʡaˁli,	b-arx-le	b-urs-a=kːʷa!\\
		hey	Hassanali	\tsc{n-}direct\tsc{-advz}	\tsc{n-}tell\tsc{.pfv-imp=prt}\\
	\glt	\sqt{Hey, Hasanali, tell the truth!}
\end{exe}

The particle \tit{žan} \sqt{beloved} is a loan from Persian with the meaning `life, soul, spirit' that is used as an address particle before names or kinship terms when expressing endearment and affection towards the addressed person. It is also used as a noun with the meaning `body, vital essence'.

\begin{exe}
	\ex	\label{ex:‎Ooh, my dear boy, I thought, who are you minor}
	\gll	ellelej,	žan	durħuˁ,	haʔ-ib=da	ča=de=l\\
		\tsc{prt} beloved	boy	say\tsc{.pfv-pret=1}	who\tsc{=2sg=q}\\
	\glt	\sqt{‎Ooh, my dear boy, I thought, who are you?}
\end{exe}

There are two particles \tit{ma} \sqt{take}	and \tit{hara} \sqt{come, go, look, here is, here you are} that are used in commands when requesting the addressee to take something or to come to the speaker. These particles thus function like verbs inflected for the imperative. Therefore, they can also attach the suffix \tit{-(j)a} \refex{ex:‎They say to the guys passing by, Take (a drink) minor}, which is used in commands and other kinds of non-indicative utterances when the addressee is plural and most often co-occurs with the imperative, the prohibitive, and the optative.

\begin{exe}
	\ex	\label{ex:‎They say to the guys passing by, Take (a drink) minor}
	\gll	heχ-tːi		satːi	arg-an	durħ-n-a-cːe	``ma=ja!''		b-ik'-ul	ca-b\\
		\tsc{dem.down}\tsc{-pl}	in.front	go\tsc{.ipfv-ptcp}	boy\tsc{-pl-obl-in}	take!\tsc{=pl}	\tsc{hpl-}say\tsc{.ipfv-icvb}	\tsc{cop-hpl}\\
	\glt	\sqt{‎They say to the guys passing by, ``Take (a drink)!''}

	\ex	\label{ex:‎Come, you son is arriving, he says minor}
	\gll	``hara,	ala	durħuˁ,''	Ø-ik'-ul	ca-w,	``haˁ-q'-uˁn-ne''\\
		come	\tsc{2sg.gen}	boy	\tsc{m-}say\tsc{.ipfv-icvb}	\tsc{cop-m}	\tsc{up}-go\tsc{-pret-cvb}\\
	\glt	\sqt{``‎Come, your son has come,'' he says.}
\end{exe}	

Interjections expressing astonishment or excitement are \tit{huja, waħ} \sqt{wow}, \tit{ellelej(-q'u)} \sqt{oh, oh, oh} (astonishment, slightly negative evaluation) \refex{ex:‎Ooh, my dear boy, I thought, who are you minor}, and \tit{ʁubza} \sqt{oh man} \refex{ex:‎I swear ‎‎by God it happened minor}. The latter originates from the noun \tit{ʁʷabza} \sqt{dzhigit, true man}.

Sanzhi has no real words for \sqt{yes} and \sqt{no}, instead the copula \tit{ca-b} is used or the respective verb forms is repeated when affirming what has been said or agreeing with the addressee. For rejection or disaffirmation the negated verb is used. However, the exclamations \tit{e} \sqt{yes, agreed} and \tit{aʔa} \sqt{no} can also be employed in these functions. For more examples of question-answer pairs, see \refsec{sec:Simple polar questions and disjunctive polar questions}.
%
\begin{exe}
	\ex	\label{ex:‎Gutter (ruger), this is a channel or what is it? Yes, the place where the water runs (to the water mill) minor}
	\gll	``rurger''	qːanaw=aw	il	ce	ca-b=e? e,	e,	hik'	hin	d-ax-an	musːa\\
		gutter	channel\tsc{=q}	that	what	\tsc{cop-n=q}	yes	yes	\tsc{dem.}up	water	\tsc{npl-}go\tsc{-ptcp}	place\\
	\glt	\sqt{``‎Gutter,'' this is a channel or what is it? Yes, the place where the water runs (to the water mill).}

	\ex	\label{ex:‎‎‎Was he himself Kumyk? ‎‎No, he was Dargi minor}
	\gll	ca-w	qːumuqlan=de=w?	aʔa,	darkːʷan=de\\
		\tsc{refl-m}	Kumyk\tsc{=pst=q}	no	Dargwa\tsc{=pst}\\
	\glt	\sqt{‎‎‎Was he himself Kumyk? ‎‎No, he was Dargwa.}
\end{exe}

Other particles and exclamations are presented in \refex{ex:exclamations minor}.
%
\begin{exe}
	\ex	\label{ex:exclamations minor}
	\begin{xlist}
		\ex	\tit{inardi} \sqt{believe me, think yourself}	
		\ex	\tit{jaʁari(b)} \sqt{listen, my dear}	
		\ex	\tit{hu} \sqt{well, now, right, come on}	
		\ex	\tit{wari} \sqt{no, no way} (emphatic warning)	
		\ex	\tit{ixʷixʷle} \sqt{of course} (to express irony and when the speaker does not believe the addressee)	
	\end{xlist}

	\ex	\label{ex:‎Oh, it turned out that the Sanzhi person is such a strong man, he says minor}
	\gll	``jaʁarib,''	Ø-ik'-ul ca-w,	``c'aq'-ce	admi	už-ib-le=q'al,''	Ø-ik'-ul ca-w,	``ik'	sunglan''\\
		\tsc{prt}	\tsc{m-}say\tsc{.ipfv-icvb}	\tsc{cop-m}	strong\tsc{-dd.sg}	person	be\tsc{.m-pret-cvb=mod}	\tsc{m-}say\tsc{.ipfv-icvb}	\tsc{cop-m}	\tsc{dem.up}	Sanzhi\\
	\glt	\sqt{``‎Oh, it turned out that the Sanzhi person is such a strong man,'' he says.}

	\ex	\label{ex:Well, (whatever, it does not matter), they will come (another day) minor}
	\gll	hu=kːʷa,	sa-d-irʁ-an-ne\\
		well\tsc{=prt}	\tsc{hither-npl-}come\tsc{.ipfv-ptcp-fut.3}\\
	\glt	\sqt{Well, they will come (another day).}
\end{exe}

There are a couple of exclamative phrases and words from Arabic that are common in the Muslim world and are also used by Sanzhi speakers \refex{ex:whatever minor}, \refex{ex:‎I swear ‎‎by God it happened minor}.
%
\begin{exe}
	\ex	\label{ex:whatever minor}
	\begin{xlist}
		\ex	\textit{aj Allah, ja Allah} \sqt{oh God}	\label{ex:oh whatever minor}
		\ex	\textit{ʡaˁlħaˁmdullilah} \sqt{Praise be to God!}	\label{ex:praise be whatever minor}
		\ex	\textit{inša-Allah} \sqt{if Allah wills}	\label{ex:if whatever wills minor}		
		\ex	\textit{aman} \sqt{alas, mercy, pity, oh, ah!} (lit. security, safety, peacefulness)	\label{ex:alas minor}
		\ex	\textit{wallah, wallahi tallahi, billah, wallah tallah} \sqt{(I promise, I swear) by God}	\label{ex:I promise whatever minor}
		\ex	\textit{mašaʔallah} \sqt{God has willed it} (appreciation, joy, praise or thankfulness for an event or person that was just mentioned)	\label{ex:whatever has willed it minor}
	\end{xlist}

	\ex	{[You do not believe me and don't think that this has happened to me!?]}\\\label{ex:‎I swear ‎‎by God it happened minor}
	\gll	billah=ra,	wallah		ʁubza		ca-b=de\\
		by.God\tsc{=add}	by.God	\tsc{emph}		\tsc{cop-n=pst}\\
	\glt	\sqt{‎I swear ‎‎by God it happened.}
\end{exe}

Nowadays, speakers also employ Russian words or phrases as interjections or pause fillers \refex{ex:Russian fillers minor}.
%
\begin{exe}
	\ex	\label{ex:Russian fillers minor}
	\begin{xlist}
		\ex	\tit{kiljanus} \sqt{I swear}		
		\ex	\tit{karoče} \sqt{in short} (pause filler)	
		\ex	\tit{značit} \sqt{thus, this means} (pause filler)	
		\ex	\tit{wat'} \sqt{well, here is}	
		\ex	\tit{tak} \sqt{like this, so, well}	
		\ex	\tit{že} (modal particle)
		\ex \tit{dawaj} \sqt{let's go, come} (invitations and requests)	
	\end{xlist}
\end{exe}

The greeting phrase used among men is the traditional Arabic phrase \tit{as-salam ʡaˁlaykum}. Other greetings are given in \refex{ex:Are you sitting minor}, \refex{ex:‎‎‎Good night minor}. The first is used for greeting women, for example when they are sitting in front of their house because with women the Arabic phrase or its shorter form \textit{salam} is not used. The phrase in \refex{ex:‎‎‎Good night minor} is uttered at night when leaving or going to bed, but not when greeting people at night. 
%
\begin{exe}
	\ex	\label{ex:Are you sitting minor}
	\gll	ka-d-iž-ib-le=da=w\\
		\tsc{down-1/2pl-}sit\tsc{.pfv-pret=cvb=1=q}\\
	\glt	\sqt{Hello!} (lit. \sqt{Are you (pl.) sitting?})

	\ex	\label{ex:‎‎‎Good night minor}
	\gll	dučːi	ʡaˁħ	d-iχʷ-ab!\\
		night	good	\tsc{npl-}be\tsc{.pfv-opt.3}\\
	\glt	\sqt{‎‎‎Good night!}
\end{exe}

With outsiders, especially when they are female, Russian salutations are used (e.g. \tit{zdrastvujte} `hello', \tit{dobryj den'} `good day').


%%%%%%%%%%%%%%%%%%%%%%%%%%%%%%%%%%%%%%%%%%%%%%%%%%%%%%%%%%%%%%%%%%%%%%%%%%%%%%%%



\section{Cross-categorical suffixes}
\label{sec:Cross-categorical derivational suffixes}

These suffixes can be viewed as cross-categorical derivational suffixes that attach to a number of parts of speech (adjectives, verbs, adverbs, postpositions, nominals) and form referential attributes\slash definite descriptions with nominal properties (suffixes -\textit{ce} and -\textit{il}) or adverbials (adverbializing suffix).

% --------------------------------------------------------------------------------------------------------------------------------------------------------------------------------------------------------------------- %

\subsection{The suffix \tit{-ce} }
\label{ssec:The -ce / -te attributive}
\subsubsection{Function and distribution of the suffix \tit{-ce}}
\label{sssec:Function and distribution of the suffix -ce}

The semantic, syntactic and distributional properties of the suffix -\textit{ce} are quite complex. Its syntactic impact overlaps with that of the suffix -\textit{il} described below, but the distributions of both suffixes are rather complementary (see the end of \refsec{ssec:The -il attributive} for a comparison). The suffix -\textit{te}, which is, in fact, one of the most productive nominal plural suffixes (\refsec{sec:FrequentAndProductivePluralSuffixes}) is used as the plural form of -\textit{ce} and for the sake of simplicity will be treated as such in this section. However, there are small functional differences between both suffixes -\textit{ce} and -\textit{te} that will be pointed out whenever relevant.

The suffix -\textit{ce} is added to: 

\begin{itemize}
	\item	adjectives \xxref{ex:a long rope minor}{ex:Does an older (person) know it better minor}
	\item	various verb forms occurring in certain types of complement clauses (e.g. infinitive, participles, copulas) and relative clauses (preterite or modal participle) and very rarely to the negative copula when it is used as expressing the meaning `without'; this includes the `experiential' verb forms \xxref{Do you see?'' said the sun to the offended wind.}{ex:‎‎There is nothing more to tell.min}
		\item nominals inflected for the genitive case (noun, pronouns etc.) \refex{ex:‎‎‎It happens to me that I come across my (milk) there minor}
	\item	expressions with spatial meaning that are inflected for the essive case, in particular adverbials, postpositions, nouns, pronouns \refex{ex:‎The other one (son) who was at home did not drink that much minor}, \refex{ex:‎‎‎First he wanted to take one pear, when he saw the man who was on the tree minor}
\end{itemize}

The core function of the suffix can be described as forming definite descriptions that describe the referent via its location, its qualities, or its possessor:

\begin{itemize}
\item reference through location: the one that is located in/at/under/... X (when used on spatial expressions)
\item reference through qualities and more general characteristics: the one that is X\slash the one that lacks X (when used on adjectives and relative clauses)
\item reference through possessors: the one that belongs to X (when used on genitives)
\end{itemize}

When the referent is in the singular, -\textit{ce} is used; when it is plural, -\textit{te} is used. The descriptions can be used as referring expressions that function as phrasal or clausal arguments, predicates or detached topicalized items, etc. Based on the core function, the use of the suffix has further extended such that it is also optionally found on attributes such as adjectives and relative clauses that modify nominals. In the following, I will explain my approach by going systematically through the parts of speech listed above and the contexts of use.

First and foremost the suffix -\textit{ce} is found on adjectives. In my corpus, this usage exceeds all other uses. The suffix can optionally be added to adjectives in attributive function without leading to any noticeable semantic difference \refex{ex:a long rope minor}. As the same example shows, it can be added to adjectives with gender agreement prefixes and those lacking gender agreement prefixes. If the head noun is preceded by more than one adjective, all adjectives preceding it can but not need bear the attributive suffix.
%
\begin{exe}
	\ex	\label{ex:a long rope minor}
	\gll	b-uqen	t'alim	/	b-uqen-ce	t'alim\\
		\tsc{n-}long	rope		/	\tsc{n-}long\tsc{-dd.sg}	rope\\
	\glt	\sqt{a long rope} (E)

	\ex	\label{ex:‎‎a young, beautiful girl minor}
	\gll	žahil	qːuʁa	rursːi\\
		young	beautiful	girl\\
	\glt	\sqt{‎‎a young, beautiful girl} (E)

	\ex	\label{ex:‎‎a young, beautiful girl minor second}
	\begin{xlist}
		\ex	\tit{žahil-ce	qːuʁa	rursːi}		\label{ex:‎‎a young, beautiful girl minor@A}
		\ex	\tit{žahil	qːuʁa-ce	rursːi}	\label{ex:‎‎a young, beautiful girl minor@B}
		\ex	\tit{žahil-ce	qːuʁa-ce	rursːi}	\label{ex:‎‎a young, beautiful girl minor@C}
	\end{xlist}
\end{exe}

In order for adjectives to be used as predicates \refex{ex:Khabaci (name) was also not bad minor} or nominals \refex{ex:Does an older (person) know it better minor} the suffix is obligatorily added, and this rule includes Russian loan words as well \refex{ex:About them there were, are many stories, interesting (ones) minor@19c}. Adjectives that bear the suffix -\textit{ce} are referential nominals and thus can occur in a position detached from the noun even if they semantically rather seem to function as nominal modifiers \refex{ex:About them there were, are many stories, interesting (ones) minor@19c}. Examples such as \refex{ex:About them there were, are many stories, interesting (ones) minor@19c} do not represent discontinuous noun phrases. The adjective is rather an independent referential constituent that occurs to the right of the clause as an afterthought. This will be analyzed in more detail in \refsec{sssec:Analyzing the suffix -ce and its cognates in other Dargwa languages} below.

\begin{exe}
	\ex	\label{ex:Khabaci (name) was also not bad minor}
	\gll	χabacːi	dik'ar	wahi-ce	akːʷ-i\\
		Khabaci	too	bad\tsc{-dd.sg}	\tsc{cop.neg-hab.pst}\\
	\glt	\sqt{Khabaci (name) was also not bad.}

	\ex	\label{ex:About them there were, are many stories, interesting (ones) minor@19c}
	\gll	il-tː-a-la	d-aqil	χabur-te	k'e-d=de	ca-d,	interesni-te\\
		that\tsc{-pl-obl-gen}	\tsc{npl-}much	story\tsc{-pl}	exist.\tsc{up-pl=pst}	\tsc{cop-npl}	interesting\tsc{-dd.pl}\\
	\glt	\sqt{About them there were, are many stories, interesting (ones).}

\end{exe}

Adjectives (and other items) bearing the suffix can take case suffixes after the oblique stem suffix -\textit{li} has been added \refex{ex:Does an older (person) know it better minor}. In the plural, -\textit{te} is replaced by -\textit{ta} when cases are added (in the same way as for nouns that make use of the plural suffix -\textit{te}). 

\begin{exe}
	\ex	\label{ex:Does an older (person) know it better minor}
	\gll	χːula-ce-li-j	ʡaˁħ-le	ʡaˁq'lu	b-alχ-u=w?\\
		big\tsc{-dd.sg-obl-dat}	good\tsc{-advz}	mind	\tsc{n-}know\tsc{.ipfv-prs=q}\\
	\glt	\sqt{Does an older (person) know it better?}
\end{exe}

Second, the suffix appears on participles (modal and preterite participle) that form relative clauses. Its use is optional and relatively rare for relative clauses in the canonical prenominal position and seems to be preferred for head nouns in the plural and mass nouns that control plural agreement (in which case -\textit{te} instead of -\textit{ce} is used) \refex{ex:the gifts that the friends had brought to him@16a}, \refex{ex:I saw the details that I did not see for 25 years}. For head nouns in the singular, the use of the suffix -\textit{il} is more common than -\textit{ce} (\refsec{ssec:The -il attributive}). Example \refex{Do you see?'' said the sun to the offended wind.} is part of a translation of the famous fable `The North Wind and the Sun’. Example \refex{ex:the gifts that the friends had brought to him@16a} comes from the translation of a Standard Dargwa folktale. 

\begin{exe}

\ex	\label{Do you see?'' said the sun to the offended wind.}
\gll ``či-b-ig-ul=de=w?''	b-ik'ʷ-ar	bari	[q'uc'	b-iχ-ub-ce]	č'an-ni-cːe\\
\tsc{spr-n}-see.\tsc{ipfv-icvb=2sg=q}	\tsc{n}-say.\tsc{ipfv-prs}	sun	offence	\tsc{n}-be.\tsc{pfv-pret-dd.sg}	wind-\tsc{obl-in}\\
\glt \sqt{``Do you see?'' said the sun to the offended wind.}

	\ex	\label{ex:the gifts that the friends had brought to him@16a}
	\gll	[juldašː-a-l	cin-i-j	sa-qː-ib-te]	xunul-be\\
		friend\tsc{-obl.pl-erg}	\tsc{refl.sg-obl-dat}	\tsc{hither}-carry\tsc{-pret-dd.pl}	gift\tsc{-pl}\\
	\glt	\sqt{the gifts that the friends had brought to him}
	
			\ex	\label{ex:I saw the details that I did not see for 25 years}
	\gll	[ʁanu	xu-ra	dus	či-a-d-až-ib-te]	dalga=ra	či-d-až-ib=da\\
		twenty	five-\tsc{num}	year	\tsc{spr-neg-npl-}see\tsc{.pfv-pret-dd.pl} 	detail\tsc{=add}	\tsc{spr-npl-}see\tsc{.pfv-pret=1}\\
	\glt	\sqt{I also saw the details that I did not see for 25 years.}
\end{exe}

The use of -\textit{ce} becomes obligatory when relative clauses with the preterite participle occur in a position after or detached from the noun that they semantically belong to \refex{ex:the brothers who came home with empty hands} or when they are used without a head \refex{I am the sun that shines (lit. goes).}, \refex{ex:How the ones like made by little children, from the old times, they were interesting minor@16c}. In other words, relative clauses that do not function as attributes but as nominals are marked by -\textit{ce}.

\begin{exe}
		\ex	\label{ex:the brothers who came home with empty hands}
		\gll	uc-be	[čar	b-iχ-ub-te	d-ac'	nuˁq-b-a-cːella]\\
			brother\tsc{-pl}	back	\tsc{hpl-}be\tsc{.pfv-pret-dd.pl} \tsc{npl-}empty	hand\tsc{-pl-obl-comit}\\
		\glt	\sqt{the brothers who came back with empty hands}

\ex	\label{I am the sun that shines (lit. goes).}
	\gll bari=da=nu	[r-uq-un-ce]\\
	sun=1=\tsc{prt}		\tsc{f}-go.\tsc{pfv-pret-dd.sg}\\
	\glt	\sqt{I am the sun that shines (lit. goes).}

	\ex	\label{ex:How the ones like made by little children, from the old times, they were interesting minor@16c}
	\gll	cet'le	[nik'a	durħ-n-a-l	d-arq'-ib-te]	ʁunab-te,	sala-lla	zamana, 	intersna=de	ix-tːi\\
		how	small	boy\tsc{-pl-obl-erg}	\tsc{npl-}do\tsc{.pfv-pret-dd.pl} 	\tsc{eq-dd.pl}	before\tsc{-gen}	time	interesting\tsc{=pst}	\tsc{dem.up}\tsc{-pl}\\
	\glt	\sqt{Like the ones made by little children, from the old times, they were interesting.}
\end{exe}

In the function of marking relative clauses the suffix in principle competes with -\textit{il} (\refsec{ssec:The -il attributive} below), but we find a clear distribution. The suffix -\textit{ce} can only be used with singular referents \refex{Do you see?'' said the sun to the offended wind.}, \refex{I am the sun that shines (lit. goes).}, but its use is relatively rare and -\textit{il} is normally used instead. By contrast, in the plural -\textit{il} cannot be used and only -\textit{te} is available \refex{ex:the gifts that the friends had brought to him@16a}, \refex{ex:the brothers who came home with empty hands}, and \refex{ex:How the ones like made by little children, from the old times, they were interesting minor@16c}.

Relative clauses are not the only types of clauses that can be turned into referential definite descriptions by means of -\textit{ce}. Factual complement clauses with matrix verbs denoting emotions, cognition as well as evaluative predicates can also be marked by the preterite participle and -\textit{ce} (as an alternative to, e.g., the masdar suffix) (\refsec{ssec:The attributive marker -ce (-te)COMPL}). This use is straightforward: a fact is expressed as a proposition by means of -\textit{ce}, i.e., as a definite description, and can then be used in argument position. In this function, the use of -\textit{te} is not allowed.

\begin{exe}

	\ex	\label{ex:‎‎ She apparently got to know that he had died}
	\gll	[w-ebč'-ib-ce]	b-aχ-ur-re	b-už-ib-le ...\\
	\tsc{m}-die.\tsc{pfv-pret-dd.sg}	\tsc{n}-know.\tsc{pfv-pret-cvb}	\tsc{n}-be-\tsc{pret-cvb} \\
	\glt	\sqt{‎‎She apparently got to know that he had died, ... .}

	\ex	\label{ex:‎‎‎I am happy that you came_1}
	\gll	du	razi-l=da	[u		sa-r-eʁ-ib-ce]\\
		\tsc{1sg}	happy\tsc{-advz=1}	\tsc{2sg}	\tsc{hither-f-}go\tsc{.pfv-pret-dd.sg}\\
	\glt	\sqt{‎‎‎I am happy that you came.} (E)
\end{exe}

Similarly, -\textit{ce} (but not -\textit{te}) can be added to the infinitive and used as the complement of the copula in existential clauses \refex{ex:‎‎There is nothing more to tell.min}. The infinitive + -\textit{ce} combinations of the verbs `eat' and `drink' have been lexicalized as nouns, e.g. \textit{b-erkʷ-ij-ce} `food' (\tsc{n}-eat.\tsc{pfv-inf-dd.sg}).

\begin{exe}
	\ex	\label{ex:‎‎There is nothing more to tell.min}
	\gll cara	cik'al	b-urs-ij-ce	b-akːu\\
	other	something	\tsc{n}-tell-\tsc{inf-dd.sg}	\tsc{n}-\tsc{cop.neg-prs} \\
	\glt	\sqt{‎‎There is nothing more to tell.}
\end{exe}

There is one more context in which the suffix is used on verbs, namely for the formation of the analytic verb forms called `experiential' in this grammar. These verb forms consist of the preterite participle plus \textit{-ce} (or \textit{-il}) and a copula, and have perfect-like semantics. They are predominantly used when speakers talk about their own experiences and about situations they were personally involved in (\refsec{ssec:Experiential I and experiential II} and \refsec{ssec:Experiential past I and experiential past II}). The semantic contribution of the suffix \textit{-ce} to these verb forms is unclear to me, but their syntactic impact is obvious. The experiential tenses are close to forming a clause union or biclausal structure, i.e., the participle with \textit{-ce} functions like a headless relative clause.

Third, the suffix can be added to nominals that are marked for the genitive case \refex{ex:‎‎‎It happens to me that I come across my (milk) there minor} or for the essive case \refex{ex:‎The other one (son) who was at home did not drink that much minor} and also to spatial adverbs and postpositions that are inflected for the essive case \refex{ex:‎‎‎First he wanted to take one pear, when he saw the man who was on the tree minor}. Thus, in \refex{ex:‎‎‎First he wanted to take one pear, when he saw the man who was on the tree minor} the suffix has the entire postpositional phrase in its scope. As with the adjectives and the relative clauses, the so-formed constituents are definite descriptions that function as attributes of nouns or are referentially independent.
%
\begin{exe}
	\ex	\label{ex:‎‎‎It happens to me that I come across my (milk) there minor}
	\gll	di-la-ce	qːarči	b-ič-ib-le	χajri	b-irχʷ-u	heχ-tːu-b	dam\\
		\tsc{1sg-gen-dd.sg}	meet	\tsc{n-}occur\tsc{.pfv-pret-cvb}	benefit	\tsc{n-}become\tsc{.ipfv-prs.3}		\tsc{dem.down}\tsc{-loc-n}	\tsc{1sg.dat}\\
	\glt	\sqt{‎‎‎It happens to me that I come across my (milk) there.}

	\ex	\label{ex:‎The other one (son) who was at home did not drink that much minor}
	\gll	ij	qili-w-ce	iž-itːe	c'aq'-le	a-učː-i\\
		this	home\tsc{-m-dd.sg}	this\tsc{-advz}	strong\tsc{-advz}	\tsc{neg-}drink\tsc{.ipfv-hab.pst}\\
	\glt	\sqt{‎The other one (son) who was at home did not drink that much.}

	\ex	\label{ex:‎‎‎First he wanted to take one pear, when he saw the man who was on the tree minor}
	\gll	bahsar	ca	qaˁr	h-asː-ib	Ø-ikː-ul=de,	či-w-až-ib-le	admi	kːalkːi-cːe-w	či-w-ce\\
		first	one	pear	\tsc{up}-take\tsc{.pfv-pret}	\tsc{m-}want\tsc{.ipfv-icvb=pst}	\tsc{spr-m-}see\tsc{.pfv-pret-cvb}	person	tree-\tsc{in}\tsc{-m}	on\tsc{-m-dd.sg}\\
	\glt	\sqt{‎‎‎First he wanted to take one pear, when he saw the man who was in the tree.}
\end{exe}

The difference between modifiers or adjuncts bearing \tit{-ce} and those not bearing \tit{-ce} can be illustrated by the following minimal pair. The first sentence has two interpretations, one in which the noun with the spatial case suffix modifies the whole clause, and another one in which it modifies only the following noun phrase. By contrast, if the suffix \tit{-ce} is added to the noun with the spatial case, only the second interpretation is available.
%
\begin{exe}
	\ex	\label{ex:‎‎The car has to be washed in the yard minor}
	\gll	azbar-re-b	mašin	ic-an	ca-b\\
		yard\tsc{-loc-n}	car	wash\tsc{.ipfv-ptcp}	\tsc{cop-n}\\
	\glt	\sqt{‎‎The car has to be washed in the yard.} OR \sqt{The car that is in the yard has to be washed.} (E)

	\ex	\label{ex:‎‎‎The car that is in the yard has to be washed minor}
	\gll	azbar-re-b-ce	mašin	ic-an	ca-b\\
		yard\tsc{-loc-n-dd.sg}	car	wash\tsc{.ipfv-ptcp}	\tsc{cop-n}\\
	\glt	\sqt{‎‎‎The car that is in the yard has to be washed.} (E)
\end{exe}


\subsubsection{Analyzing the suffix \textit{-ce} and its cognates in other Dargwa languages}
\label{sssec:Analyzing the suffix -ce and its cognates in other Dargwa languages}

Cognates of Sanzhi -\textit{ce} are found in most if not all Dargwa languages (e.g. -\textit{ci} in Standard Dargwa and Icari Dargwa, -\textit{se} in Tanti Dargwa, -\textit{ze} in Chirag Dargwa). In the literature, they have mostly been analyzed with respect to their occurrence on adjectives. Thus, adjectives have been divided into `short adjectives' without the suffix and `long adjectives' that bear the suffix.

In grammars of Standard Dargwa, the short adjectives are said to be more archaic and basically only used in poetry and other types of fictional literature as expressive means to describe emotions and feelings \citep[26]{vandenBerg2001}; \citep[207\tnd208]{AbdullaevEtAl2014}. According to the latter grammar, adjectives with gender prefixes do not have a short form. This is in plain contrast to Sanzhi Dargwa, where they have a short form, e.g. \tit{ca b-uqen q'aˁli} (one \tsc{n-}long branch) \sqt{one long branch}. Furthermore, in Sanzhi short adjectives are at least as common as adjectives with the attributive suffix, if not more common.

\citet{Lander2014} (see also \citeb{Sumbatova.Lander2014}) describes short adjectives in Tanti Dargwa as formally and functionally marked and opposed to the unmarked long adjectives bearing the suffix \tit{-se} (the cognate of Sanzhi \tit{-ce}) because the former are rarely used and are restricted in their distribution. By contrast, the long adjectives allow for a large range of constructions. \citet{Lander2014} analyzes them as basically equivalent to relative clauses. He rejects an analysis of \tit{-se} as a nominalizer because adjectives to which \tit{-se} is suffixed differ in some properties from standard nouns. First, they cannot be modified by short adjectives. Second, they can modify personal pronouns, indefinite pronouns, and reflexive pronouns. Third, when case-marked, long adjectives cannot follow the noun as would be expected for a noun in an appositive construction.

For Sanzhi Dargwa the question of markedness is not fully clear, but if we can apply this label at all, it is modifiers having the attributive suffix (e.g. \sqt{long adjectives}) that are marked, rather than the other way around. First, they are clearly formally marked by the suffix. Second, they seem to be slightly less common than short adjectives, can occur in positions that most nominal modifiers cannot occur in, and occasionally have marked, contrastive semantics that is absent from unmarked modifiers (see the discussion below). Furthermore, when occurring outside of their canonical position, they are syntactically not part of the noun phrase to which they semantically belong. This becomes apparent when the head noun of the noun phrase appears in a case other than the unmarked absolutive. In such a case, the full adjective can only follow a noun when it is also case-marked and interpreted as forming its own phrase. In other words, it is nominalized and takes an argument or adjunct position in the clause \refex{ex:He went to a good doctor, he did not go to a bad one ungrammatical minor}, \refex{ex:He went to a doctor who is good; he did not go to a bad one minor}. A similar behavior is observed with floating quantifiers, which are also syntactically not part of the noun phrase (see \refsec{ssec:The structure and order of constituents within the noun phrase}).
%
\begin{exe}
	\ex	\label{ex:He went to a good doctor, he did not go to a bad one ungrammatical minor}
	\gll	{*}	it	sa-jʁ-ib	tuχtur-ri-šːu ʡaˁħ-ce,	wahi-ce-lli-šːu	a-ag-ur\\
		{}	that	\tsc{hither}-come\tsc{.m.pfv-pret}	doctor\tsc{-obl-ad} good\tsc{-dd.sg}	bad\tsc{-dd.sg-obl-ad}	\tsc{neg-}go\tsc{.pfv-pret}\\
	\glt	(Intended meaning: \sqt{He went to a good doctor, he did not go to a bad one.}) (E)

	\ex	\label{ex:He went to a doctor who is good; he did not go to a bad one minor}
	\gll	it	sa-jʁ-ib	tuχtur-ri-šːu ʡaˁħ-ce-lli-šːu,	wahi-ce-lli-šːu	a-ag-ur\\
		that	\tsc{hither}-come\tsc{.m.pfv-pret}	doctor\tsc{-obl-ad} good\tsc{-dd.sg-obl-ad}	bad\tsc{-dd.sg-obl-ad}	\tsc{neg-}go\tsc{.pfv-pret}\\
	\glt	\sqt{He went to a doctor who is good; he did not go to a bad one.} (E)
\end{exe}

Furthermore, modifiers with attributive suffixes can also be modified by modifiers without attributive suffixes, even in those cases where the former are used as nominals \refex{ex:Buy a ripe red one minor}, though it would preferable to use attributive suffixes on both adjectives in this example (i.e. \textit{b-iq'-ur-ce}	\textit{it'in-ce}).
%
\begin{exe}
	\ex	\label{ex:Buy a ripe red one minor}
	\gll	asː-a	b-iq'-ur	it'in-ce!\\
		buy\tsc{.pfv-imp}	\tsc{n-}ripen\tsc{-pret}	red\tsc{-dd.sg}\\
	\glt	\sqt{Buy a ripe red one!} (E)
\end{exe}

This behavior points again towards an analysis of the attributive suffix as a nominalization marker. If nouns bearing attributive suffixes are nominalized, we can opt for an analysis in terms of appositional constructions. In appositional constructions, the head noun is modified by one (or occasionally more than one) noun preceding it. Case marking occurs only once, namely on the head noun \refex{ex:He went to a good doctor minor}. It cannot occur on the modifier, be it a full adjective or an appositive noun.
%
\begin{exe}
	\ex	\label{ex:He went to a good doctor minor}
	\gll	it	sa-jʁ-ib	ʡaˁħ-ce	tuχtur-ri-šːu\\
		that	\tsc{hither}-come\tsc{.m.pfv-pret}	good\tsc{-dd.sg}	doctor\tsc{-obl-ad}\\
	\glt	\sqt{He went to a good doctor.} (E)

	\ex	\label{ex:He went to a good doctor ungrammatical minor}
	\gll	{*}	it	sa-jʁ-ib	ʡaˁħ-ce-li-šu	tuχtur-ri-šːu	/	ʡaˁħ-ce-li-šu	tuχtur\\
		{}	that	\tsc{hither}-come\tsc{.m.pfv-pret}	good\tsc{-dd.sg-obl-ad}	doctor\tsc{-obl-ad}	/	good\tsc{-dd.sg-obl-ad}	doctor\\
	\glt	‎(Intended meaning: \sqt{He went to a good doctor.}) (E)
\end{exe}

A similar analysis has been proposed in the Icari Dargwa grammar: adjectives and other words bearing \tit{-ci/-ti} are analyzed as free attributes alongside cardinal numerals, other derived adjectives and some other words. \citet[48, 129]{Sumbatova.Mutalov2003} claim that \dqt{free attributes and nouns could probably be considered to form a single syntactic class (nouns). The main difference is that free attributes are much more common in the attributive position than nouns.} Furthermore, free attributes \dqt{usually stress the restrictive character of the attribute or even imply contrastive emphasis on the attribute.} This characterization fits well the Sanzhi data. Modifiers bearing the attributive suffixes can have a contrastive reading, but this reading is normally due to their position (e.g. after the noun) and is not part of the meaning of the suffixes. The suffix just makes it morphosyntactically possible for the modifier to follow the head. For instance, the following elicited example refers to a situation in which large and small plates are contrasted, but the translation of the sentence contains only one occurrence of the attributive suffix on the second adjective, because it occurs without a head noun. This means that the use of the attributive suffix has a purely morphosyntactic explanation.
%
\begin{exe}
	\ex	\label{ex:Take the large plate, put away the small one minor}
	\gll	h-asː-a	χːula	waq,	kʷi-r	ka-b-ix-a	nik'a-ce!\\
		\tsc{up}-take\tsc{.pfv-imp}	big	plate	in.the.hands\tsc{-abl}	down\tsc{-n-}throw\tsc{.pfv-imp}	small\tsc{-dd.sg}\\
	\glt	\sqt{Take the large plate, put away the small one!} (E)
\end{exe}

Modifiers with \tit{-ce} can precede pronouns and occur on non-restrictive relative clauses \refex{ex:You who has finished the university will get a good job minor}, which also demonstrates that they do not convey contrastive or restrictive semantics. For example, \refex{ex:My old mother already since long ago does not leave the house minor} does not imply that the speaker has another mother who is not old.
%
\begin{exe}
	\ex	\label{ex:You who has finished the university will get a good job minor}
	\gll	[uniwersitet	ha-b-erχː-aq-ur-ce]	at	ʡaˁħ	ʡaˁči	b-irk-u\\
		university	\tsc{up-n-}fulfill\tsc{.pfv-caus-pret-dd.sg}	\tsc{2sg.dat}	good	work	\tsc{n-}occur\tsc{.ipfv-prs}\\
	\glt	\sqt{You who has finished the university will get a good job.} (E)
	
	\ex	\label{ex:My old mother already since long ago does not leave the house minor}
	\gll	di-la	r-uqna-ce	aba	na	ixʷbel=ra	qili-r	tːura	a-r-ax-u\\
		\tsc{1sg-gen}	\tsc{f-}old\tsc{-dd.sg}	mother	already	long.ago\tsc{=add}	home\tsc{-abl}	outside	\tsc{neg-f-}go\tsc{-prs}\\
	\glt	\sqt{My old mother already since long ago does not leave the house.} (E)
\end{exe}

However, if they modify personal names the interpretation is normally contrastive. For instance, the use of a noun phrase such as \refex{ex:the good Murad minor} implies that there is another person called Murad who is not good.
%
\begin{exe}
	\ex	\label{ex:the good Murad minor}
	\gll	ʡaˁħ-ce	Murad\\
		good\tsc{-dd.sg}	Murad\\
	\glt	\sqt{the good Murad} OR \sqt{the Murad who is good} (E)
\end{exe}


I finish this section with a final comment. During a guest lecture at the University of Potsdam the audience suggested that -\textit{ce} bears some similarity to quantifiers. It might serve to express number similar to what we observe in English \textit{the red one}, and resembles indefinite pronouns such as \textit{some}. In fact, -\textit{ce} is homophonous with the interrogative pronoun \textit{ce} `what', which can also be used as an indefinite pronoun meaning `something'. The similarity is also attested in other Dargwa languages, e.g. Tanti (-\textit{se} and \textit{se} `what'). As already mentioned, the plural marker -\textit{te} is identical to one of the normal plural suffixes for nouns, and becomes -\textit{ta} when further case suffixes are added. This suggests that, in contrast to -\textit{ce}, the suffix -\textit{te} is morphologically complex, and -\textit{ce} and -\textit{te} are not diachronically related, but go back to different sources. From this it naturally follows that -\textit{ce} and -\textit{te} do not have to have identical distributions. Following this suggestion, items bearing -\textit{ce} could be analyzed as quantificational expressions rather than as referring expressions. However, further research is needed in order to test this and other proposals and to reach a full account of -\textit{ce}, -\textit{te} (and -\textit{il}).


% --------------------------------------------------------------------------------------------------------------------------------------------------------------------------------------------------------------------- %

\subsection{The suffix \tit{-il}}
\label{ssec:The -il attributive}

The cross-categorical suffix \textit{-il} is functionally very close to the suffix \textit{-ce} (\refsec{ssec:The -ce / -te attributive}), but shows a different morphosyntactic distribution. It is added to 
\begin{itemize}
	\item verbs, more specifically to the preterite or modal participles, to copulas (including negative and existential\slash locational copulas), and to the morphologically defective verb \textit{b-el} `remain'
	\item expressions with spatial meaning that are inflected for the essive case, namely adverbials, postpositions, nouns, pronouns, etc.
\end{itemize}

As illustrated in examples \refex{ex:‎‎‎And the calf before him fell down minor}, \refex{ex:‎When he suspected, that he had left, the boy in the box started to scream minor} below, the second usage is roughly identical to the employment of \textit{-ce}. 

The suffix \textit{-il} is used for the formation of referential attributes, i.e., lexemes with attributival meaning that are used as referring expressions and can make up their own phrase, but can also occur in apposition to a noun that they modify. In the latter case they occur in the position before the noun just like other nominal modifiers (adjectives, genitives, relative clauses). With non-verbal base words (i.e. expressions marked with the essive case) the suffix is required in order to turn the spatial expression into an attribute of the noun. Without the suffix the spatial expression would function as a modifier at the event level (the same was shown for -\textit{ce} in \refsec{sssec:Function and distribution of the suffix -ce} above). For instance, if we omit the suffix \textit{-il} in \refex{ex:‎‎‎And the calf before him fell down minor}, the meaning of the sentence would change to \sqt{The calf fell down before him.} because now the spatial expression would function as adverb and modify the action expressed by the verb.

\begin{exe}
	\ex	\label{ex:‎‎‎And the calf before him fell down minor}
	\gll	a	[cin-na	sala-b-il]	qːačːa	k-ag-ur\\
		and	\tsc{refl-gen}	front\tsc{-n-ref}	calf	\tsc{down}-go\tsc{.pfv-pret}\\
	\glt	\sqt{‎‎‎And the calf before him fell down.}
	
	\ex	\label{ex:‎When he suspected, that he had left, the boy in the box started to scream minor}
	\gll	q'ʷani-l-cːe-w-il	durħuˁ	ʁaˁʁ	Ø-ik'ʷ-ij	w-aʔ-išː-ib	ca-w\\
		box\tsc{-obl-in}\tsc{-m-ref}	boy	scream	\tsc{m-}say\tsc{.ipfv-inf}	\tsc{m-}begin-put\tsc{.pfv-pret}	\tsc{cop-m}\\
	\glt	\sqt{The boy in the box started to scream.}
\end{exe}

When added to verbs the resulting construction is a relative clause that can be restrictive or non-restrictive (see \refcpt{cpt:Relative clauses} for examples of both types). The use of \textit{-il} in relative clauses is not obligatory  when the relative clause occurs in its canonical position before the noun and there are only very few examples in my corpus \refex{ex:the village where I was born}. But in elicitation of relative clauses the use of \textit{-il} is common. 

\begin{exe}
	\ex	\label{ex:the village where I was born}
	\gll	[du	hak'-ub-il]	di-la	šːi\\
		\tsc{1sg}	appear.\tsc{pfv-pret-ref}	\tsc{1sg-gen}	village\\
	\glt	\sqt{the village where I was born}
\end{exe}

When relative clauses occur in a position detached from the head noun, e.g. following it, the use of \textit{-il} becomes obligatory. This happens because noun phrases are head-final and modifiers can never follow the noun they modify (e.g. demonstrative pronouns). However, a relative clause with the suffix \textit{-il} forms its own phrase and can thus directly follow the noun as in \refex{ex:The snake who sat in a pit together with the rich} or even occur after the finite verb as in \refex{ex:‎This also and this also is probably the man}, a position that is commonly used to express afterthoughts (see \refsec{ssec:The structure and order of constituents within the noun phrase} for the constituent order of the noun phrase and \refsec{ssec:Extraposed adjectives, postpositional phrases, and relative clauses} for a discussion of extraposed modifiers).


\begin{exe}
	\ex	\label{ex:The snake who sat in a pit together with the rich}
	\gll	dam	b-ičː-ib	iž	maˁlʡuˁn-ni	[ca	kur-re	ka-b-iž-ib-il	dawla.či-w	Ismaˁʔil-li-cːella]	\\
		\tsc{1sg.dat}	\tsc{n}-give.\tsc{pfv-pret}	this	snake-\tsc{erg}	one	pit-\tsc{loc}	\tsc{down-n}-be.\tsc{pfv-pret-ref}	rich-\tsc{m}	Ismail-\tsc{obl-comit}\\
	\glt	\sqt{The snake who sat in a pit together with the rich Ismail gave it to me.}

	\ex	\label{ex:‎This also and this also is probably the man}
	\gll	iž=ra	het=ra,	het	ʡaˁχːuˁl	Ø-iχʷ-ij	[xːunul-la	qajqaj-li-cːe	b-aˁq-ib-il]\\
		this\tsc{=add}	that\tsc{=add}	that	guest	\tsc{m-}be\tsc{.pfv-inf}	woman\tsc{-gen}	jaw\tsc{-obl-in}	\tsc{n-}hit\tsc{.pfv-pret-ref}\\
	\glt	\sqt{‎This also and this also is probably the man who hit the woman on the jaw.}
	
\end{exe}


Constituents bearing \textit{-il} are referential and can therefore occur without a head noun. This includes headless relative clauses (for headless relative clauses formed with the preterite participle the use of \textit{-il} or -\textit{ce} is obligatory), but also all other constituents. For instance, without the suffix the word in \refex{ex:The (thing) in the parcel?} would not be referential.

\begin{exe}
	\ex	\label{ex:The (thing) in the parcel?}
	\gll	paket-le-b-il?\\
		parcel-\tsc{loc-n-ref}\\
	\glt	\sqt{The (thing) in the parcel?} NOT \sqt{Is it in the parcel?}
\end{exe}

Therefore, the suffix is often found in topicalization constructions in which the topicalized constituent occurs to the left of the clause in \refex{ex:‎This in it (i.e. his hand), what is it minor} or in right-dislocated afterthoughts that provide more information on the referent such that its identification is facilitated for the hearer \refex{ex:He, who is it, the one of the garage there minor}. The referential attributes are often co-referenced by nominals in the clause, as in the following two examples:

\begin{exe}
	\ex	\label{ex:‎This in it (i.e. his hand), what is it minor}
	\gll	iž-i-cːe-b-il,	ce	ca-b=el	iž?\\
		this\tsc{-obl-}in\tsc{-n-ref}	what	\tsc{cop-n=indq}	this\\
	\glt	\sqt{The (one) in it (i.e. in his hand), what is it?}
	
	\ex	\label{ex:He, who is it, the one of the garage there minor}
	\gll	het,	ča 	ca-w=e,	het	[gaˁraˁž-la	hetːu-w-il]?\\
		that	who	\tsc{cop-m=q}	that	garage\tsc{-gen}	there\tsc{-m-ref}\\
	\glt	\sqt{He, who is it, the one of the garage there?}
	
\end{exe}

In the next example \refex{ex:‎Then, for my fields, my sown fields, my fruits, all the stuff, I went to the garden, I found help} the referential attributes form topicalized noun phrases that are preceding the clause and are not co-referenced in the clause.
%
\begin{exe}
	\ex	\label{ex:‎Then, for my fields, my sown fields, my fruits, all the stuff, I went to the garden, I found help}
	\gll	c'il	di-la	qu-ja-b-il,	di-la	b-ax-un-il,	di-la	c'idex,	itil.ižili,	agarud-le ag-ur-re, kumek=ra	b-arčː-ib-le,	di-la	r-iχ-ub-il=ra	b-arq'-ib-le,	du=ra	ka-r-iž-ib-le=da\\
		then	\tsc{1sg-gen}	garden\tsc{-loc-n-ref}	\tsc{1sg-gen}	\tsc{n-}sow\tsc{.pfv-pret-ref}	\tsc{1sg-gen}	fruit		one.thing.and.another	garden\tsc{-loc}	go\tsc{.pfv-pret-cvb}	help\tsc{=add}	\tsc{n-}find\tsc{.pfv-pret-cvb}	\tsc{1sg-gen}	\tsc{f-}be.able\tsc{.pfv-pret-ref=add}	\tsc{n-}do\tsc{.pfv-pret-cvb}	\tsc{1sg=add}	\tsc{down-f-}be\tsc{.pfv-pret-cvb=1}\\
	\glt	\sqt{‎Then, the (things) in my garden, my sown (fields), my fruits, all the stuff, I went to the garden, I found help, my things that I was able to do I did, and then I was sitting (relaxing).}
\end{exe}
%

The verb forms to which \textit{-il} is added are able to take case markers (preceded by the oblique suffix \textit{-li}) and then they function as referring expressions like nominals, i.e., as headless relative clauses \xxref{ex:‎‎the drunkenness passed, of this person who is here (on this picture)}{ex:the son of (the one) who took away our ploughshare} (see also \refsec{sssec:The attributive markers -il and -ce / -te in combination with the participles} and \refsec{sec:Headless relative clauses}). The other items that take \textit{-il}, i.e. the spatial expressions in the essive case, are not further inflected. For example, the form \textit{χe-w-il-la} in \refex{ex:‎‎the drunkenness passed, of this person who is here (on this picture)} functions as possessor marked by the genitive, and the possessum is the clause-initial noun \textit{kep-dex}.

%
\begin{exe}

	\ex	\label{ex:‎‎the drunkenness passed, of this person who is here (on this picture)}
	\gll	kep-dex či-r-ag-ur ca-d, hej admi-la, heštːu-w χe-w-il-la\\
		drinking-\tsc{nmlz}	\tsc{spr-abl}-go.\tsc{pfv-pret}	\tsc{cop-npl}	this	person-\tsc{gen}	here-\tsc{m}	exist.\tsc{down-m-ref-gen} \\
	\glt	\sqt{T‎he drunkenness passed, of this person who is here down (in the picture).}
	
	\ex	\label{ex:The one who wanted (milk) lifted (the cans) up and}
	\gll	[b-ikː-an-il-li]	aq	či-ha-d-arq'-ib-le,	d-učː-i	heχ-tːi\\
		\tsc{n}-want.\tsc{ipfv-ptcp-ref-erg}	high	\tsc{spr-up-npl}-do.\tsc{pfv-pret-cvb}	\tsc{npl}-drink.\tsc{ipfv-hab.pst}	\tsc{dem.down-pl} \\
	\glt	\sqt{The one who wanted (milk) lifted (the cans) up and drank.}

	\ex	\label{ex:the son of (the one) who took away our ploughshare}
	\gll	[nišːa-lla	ʁʷab-ne	d-erqː-ib-il-la]	duˁrħuˁ\\
		\tsc{1pl-gen}	plowshare-\tsc{pl}	\tsc{npl}-take.\tsc{pfv-pret-ref-gen}	boy\\
	\glt	\sqt{the son of (the one) who took away our plowshare}
	
		\ex	\label{ex:‎Then I was embarrassed for what I had done; I felt ashamed}
	\gll	c'il	uruc	Ø-iχ-ub=da	[du-l	b-arq'-ib-il-li-j],	du	c'aχ	ka-b-icː-ur	dam\\
		then	embarrassed	\tsc{m-}be\tsc{.pfv-pret=1}	\tsc{1sg-erg}	\tsc{n-}do\tsc{.pfv-pret-ptcp-obl-dat}	\tsc{1sg}	shame	\tsc{down-n-}stand\tsc{.pfv-pret}	\tsc{1sg.dat}\\
	\glt	\sqt{‎Then I was embarrassed because of what I had done; I felt ashamed.}
\end{exe}	

There are three more uses\slash meanings of this suffix that have not been discussed so far. First of all, \textit{-il} is used for the formation of experiential forms in the same way as it was mentioned for the suffix -\textit{ce} in \refsec{sssec:Function and distribution of the suffix -ce}. Second, the suffix \tit{-il} can be added to the locational/existential copulas when they are followed by the standard copula, such that the result looks like an analytic verb form. In this case, the use of the suffix restricts the meaning of the locational/existential copula to the existential meaning, excluding the locational meaning. Thus, the sentence in \refex{ex:The son of Khalirbihin should be there (i.e. be still alive) minor} cannot be translated by \sqt{The son of Khalirbihin is (located) down there.}.
%
\begin{exe}
	\ex	\label{ex:The son of Khalirbihin should be there (i.e. be still alive) minor}
	\gll	heχ	χalirbihin-na	durħuˁ	χe-w-il	ca-w\\
		\tsc{dem.down}	Khalirbihin\tsc{-gen}	boy	exist.\tsc{down-m-ref}	\tsc{cop-m}\\
	\glt	\sqt{The son of Khalirbihin exists (i.e. is still alive).} (E)
\end{exe}
	
Third, when added to the preterite participle of the verb \tit{ʔ-} \sqt{say}, the resulting verb form is used as a marker for ordinal numerals (\refsec{sec:ordinalnumerals}). It is also part of the quantifiers \textit{li<b>il} `all', \textit{har-il} `every', and \tit{b-aq-il} \sqt{much, many}. With the first quantifier the use of \textit{-il} is obligatory, i.e. *\textit{lib}. To the other two quantifiers \textit{-il} is only attached when they are used referentially (i.e. as predicates or arguments). 

The constituents marked with \tit{-il} occasionally have a flavor of contrastiveness, but this is a pragmatic implicature from the context, not part of the meaning of \tit{-il}. Furthermore, many example sentences with \tit{-il} do not have a contrastive meaning. For instance, in \refex{ex:‎When he suspected, that he had left, the boy in the box started to scream minor} the boy in the box is not contrasted with any other boy. Similarly, the fruits of the gardens and fields in \refex{ex:‎Then, for my fields, my sown fields, my fruits, all the stuff, I went to the garden, I found help} are not contrasted with other items.

Finally I will briefly compare \textit{-ce}, \textit{-te} and \textit{-il}. The suffix \tit{-ce} has a larger range of applications because it is added to a greater variety of base words. Almost all morphosyntactic contexts that allow for \tit{-il} also allow for \tit{-ce}, but not vice versa (which might be partially explained by the fact that \tit{-ce} starts with a consonant, and can therefore follow consonants and vowels, but \tit{-il} can only be added after consonants):

\begin{itemize}
	\item	The suffix \textit{-ce} can only be used with singular referents. It occurs on adjectives, to a restricted extent on relative clauses (where it competes with \textit{-il}), nominals marked for genitive and essive case and in complement clauses of the fact type.
	\item	The suffix \textit{-te} can only be used with plural referents and largely mirrors \textit{-ce}. It occurs on adjectives, relative clauses and nominals in the genitive or essive case, but not in complement clauses.
	\item	The suffix \textit{-il} shares with \textit{-ce} the restriction to singular referents and thus partially competes with it. It is primarily used in relative clauses where its use is preferred over \textit{-ce}, but also with nominals bearing the essive case. It cannot be used with adjectives (except for three quantifiers and the formation of ordinal numerals), nor can it occur in complement clauses.
\end{itemize}
 
All three suffixes are used in experiential verb forms.

% --------------------------------------------------------------------------------------------------------------------------------------------------------------------------------------------------------------------- %

\subsection{The adverbializer \tit{-le}}
\label{ssec:The adverbializer -le}

The adverbializer \tit{-le} (which has the variant \tit{-lle} and the predictable allomorphs -\textit{ne} and -\textit{re}) forms (manner) adverbs from short adjectives and nouns (\refsec{sec:FormationOfAdverbialsWithTheSuffixLe}). It is also used for the formation of the simple converbs, i.e., the imperfective and the perfective converb, which are also widely used in analytic tenses (\refsec{ssec:Simple converbs}), and it can follow items bearing spatial cases such as adverbs \refex{ex:While she had the child in her arms, he hit his wife minor} or nouns \refex{ex:‎‎Just with the back (turned to me), I do not see this person minor} that then also function like manner adverbials. As example \refex{ex:‎While I am a good husband, you are a bad wife minor} illustrates, it can also be used with nouns in the absolutive case. This sentence shows a copula construction in which the copula complement of the first clause has been turned into an adverbial by means of \tit{-le}.
%
\begin{exe}
	\ex	\label{ex:While she had the child in her arms, he hit his wife minor}
	\gll	hel-i-l	nik'a-ce	kʷi-w-le,	xːunul-li-j	b-aˁq-ib	ca-b	hel-i-l\\
		that\tsc{-obl-erg}	small\tsc{-dd.sg}	in.the.hands\tsc{-m-advz}	woman\tsc{-obl-dat}	\tsc{n-}hit.\tsc{pfv-pret}	\tsc{cop-n}	that\tsc{-obl-erg}\\
	\glt	\sqt{While she (had) the child in her arms, he hit his wife.}

	\ex	\label{ex:‎‎Just with the back (turned to me), I do not see this person minor}
	\gll	prosto	qːaq-sa-lle,	či-w-ig-ul	akːʷa-di	du-l	heχ	admi\\
		just	back\tsc{-ante-advz}	\tsc{spr-m-}see\tsc{.ipfv-icvb}	\tsc{cop.neg-1}	\tsc{1sg-erg}	\tsc{dem.down}	person\\
	\glt	\sqt{‎‎Just with the back (turned to me), I do not see this person.}

	\ex	\label{ex:‎While I am a good husband, you are a bad wife minor}
	\gll	du	ʡaˁħ	sub-le,	u	xːunul	wahi-ce=de\\
		\tsc{1sg}	good	husband\tsc{-advz}		\tsc{2sg}	woman	bad\tsc{-dd.sg=2sg}\\
	\glt	\sqt{‎While I am a good husband, you are a bad wife.} (E)
\end{exe}
