\chapter{Numerals}
\label{cpt:numerals}

Sanzhi has (i) \isi{cardinal numerals} (\refsec{sec:cardinalnumerals}), (ii) \isi{ordinal numerals} (\refsec{sec:ordinalnumerals}), (iii) distributive numerals (\refsec{sec:distributivenumerals}), (iv) \is{group numeral}group numerals (\refsec{sec:groupnumerals}), (v) \isi{multiplicative numerals} (\refsec{sec:multiplicativenumerals}), and (vi) \isi{collective numerals} (\refsec{sec:collectivenumerals}).

Most of the numerals have the morphosyntactic properties of \isi{adjectives} or occasionally adverbs. Generally, numerals can be used as nominal modifiers with a following noun in the singular. For verbal agreement the \isi{noun phrase} is nevertheless treated as plural \refsec{ssec:The structure and order of constituents within the noun phrase}. In this chapter, I also treat some other numeral expressions and basic ways of counting (\refsec{sec:othernumeralexpressions}). Quantifiers such as \sqt{all} are treated in \refsec{sec:Universal indefinites and other quantifiers} together with \is{indefinite pronoun}indefinite pronouns.


%%%%%%%%%%%%%%%%%%%%%%%%%%%%%%%%%%%%%%%%%%%%%%%%%%%%%%%%%%%%%%%%%%%%%%%%%%%%%%%%%
%%%%%%%%%%%%%%%%%%%%%%%%%%%%%%%%%%%%%%%%%%%%%%%%%%%%%%%%%%%%%%%%%%%%%%%%%%%%%%%%%


\section{Cardinal numerals}
\label{sec:cardinalnumerals}

The \isi{cardinal numerals} 1\tnd101 are given in \reftab{tab:cardinalnumerals}. All numerals except for \tit{ca} \sqt{one} are morphologically complex, containing a root and a derivational suffix. The numerals 2 to 10, 20, as well as 100 are formed by means of the suffix \tit{-al} (allomorph \tit{-jal} after vowels). The decimal numerals 10 and 30\tnd90 are built by adding the suffix \tit{-c\ej al} to the roots. When decimals and the numerals 1\tnd9 are combined, both the decimals and the numerals 1\tnd9 take suffixes. On the decimals \tit{-al} is replaced by \tit{-nu}\slash\tit{-anu}, e.g. \tit{wec\ej-al} \sqt{10} and \tit{wec\ej-nu}, \tit{wer-c\ej-al} \sqt{70} and \tit{wer-c\ej-anu}. To the numerals 1\tnd9 the suffix \tit{-ra} is added.

\begin{table}
	\caption{Cardinal numerals 1\tnd101}
	\label{tab:cardinalnumerals}
	\begin{tabularx}{0.95\textwidth}[]{>{\raggedleft\arraybackslash}p{15pt} >{\itshape\raggedright\arraybackslash}p{60pt} >{\raggedleft\arraybackslash}p{15pt} >{\itshape\raggedright\arraybackslash}X >{\raggedleft\arraybackslash}p{15pt} >{\itshape\raggedright\arraybackslash}X}
		\lsptoprule
				1	&	ca
			&	11	&	wec\ej-nu ca-ra
			&	21	&	\vuvfr a-nu ca-ra\\

				2	&	k\ej\lab el (k\ej\lab i-)
			&	12	&	wec\ej-nu k\ej\lab i-ra
			&	22	&	\vuvfr a-nu k\ej\lab i-ra\\

				3	&	\eppl a\pha b-al
			&	13	&	wec\ej-nu \eppl a\pha b-ra
			&	23	&	\vuvfr a-nu \eppl a\pha b-ra\\

				4	&	a\vuvfr\lab-al
			&	14	&	wec\ej-nu a\vuvfr\lab-ra
			&	24	&	\vuvfr a-nu \eppl a\pha b-ra\\

				5	&	xu-jal
			&	15	&	wec\ej-nu xu-ra
			&	25	&	\vuvfr a-nu xu-ra\\

				6	&	urek\lmk-al
			&	16	&	wec\ej-nu urek\lmk-ra
			&	26	&	\vuvfr a-nu urek\lmk-ra\\

				7	&	wer-al
			&	17	&	wec\ej-nu wer-ra 
			&	27	&	\vuvfr a-nu wer-ra\\

				8	&	k\lmk a\glpl-al
			&	18	&	wec\ej-nu k\lmk a\glpl-ra 
			&	28	&	\vuvfr a-nu k\lmk a\glpl-ra\\

				9	&	ur\paaf\ej em-al
			&	19	&	wec\ej-nu ur\paaf\ej em-ra
			&	29	&	\vuvfr a-nu ur\paaf\ej em-ra\\

				10	&	wec\ej-al
			&	20	&	\vuvfr a-jal\\[0.3cm]

			%%%%%%%%%%

				30	&	\eppl a\pha b-c\ej al
			&	31	&	\eppl a\pha b-c\ej anu ca-ra\\

				40	&	a\vuvfr\lab-c\ej al
			&	41	&	a\vuvfr\lab-c\ej anu ca-ra\\

				50	&	xu-c\ej al
			&	51	&	xu-c\ej anu ca-ra\\

				60	&	urek-c\ej al
			&	61	&	urek-c\ej anu ca-ra\\

				70	&	wer-c\ej al
			&	71	&	wer-c\ej anu ca-ra\\

				80	&	k\lmk a\glpl-c\ej al
			&	81	&	k\lmk a\glpl-c\ej anu ca-ra\\

				90	&	ur\paaf\ej em-c\ej al
			&	91	&	\multicolumn{2}{l}{\tit{ur\paaf\ej em-c\ej anu ca-ra}}\\

				100	&	dar\pafr\lmk-al
			&	101	&	dar\pafr-lim ca\\
		\lspbottomrule
	\end{tabularx}
\end{table}

The \isi{cardinal numerals} for hundreds and thousands are provided in \reftab{tab:cardinalnumeralslarge}. Complex numerals containing hundreds need the derivational suffix \tit{-lim} added to \tit{dar\pafr} \sqt{1000}. For the higher \isi{cardinal numerals} (millions, billions, etc.) the Russian terms are used. 


\begin{table}
	\caption{Cardinal numerals 100\tnd20,000}
	\label{tab:cardinalnumeralslarge}
	\begin{tabularx}{0.80\textwidth}[]{>{\raggedleft\arraybackslash}p{30pt} >{\itshape\raggedright\arraybackslash}X >{\raggedleft\arraybackslash}p{30pt} >{\itshape\raggedright\arraybackslash}X}
		\lsptoprule
				100	&	dar\pafr\lmk-al
			&	101	&	dar\pafr-lim ca\\

				200	&	k\ej\lab i-dar\pafr
			&	201	&	k\ej\lab i-dar\pafr-lim ca\\

				300	&	\eppl a\pha b-dar\pafr
			&	301	&	\eppl a\pha b-dar\pafr-lim ca\\

				400	&	a\vuvfr\lab-dar\pafr
			&	401	&	a\vuvfr\lab-dar\pafr-lim ca\\

				500	&	xu-dar\pafr
			&	501	&	xu-dar\pafr-lim ca\\

				600	&	urek-dar\pafr
			&	601	&	urek-dar\pafr-lim ca\\

				700	&	wer-dar\pafr
			&	701	&	wer-dar\pafr-lim ca\\

				800	&	k\lmk a\glpl-dar\pafr
			&	801	&	k\lmk a\glpl-dar\pafr-lim ca\\

				900	&	ur\paaf\ej em-dar\pafr
			&	901	&	ur\paaf\ej em-dar\pafr-lim ca\\

				1,000	&	azir
			&	2,000	&	k\ej\lab el azir\\

				10,000 &	wec\ej al azir
			&	20,000 &	\vuvfr a-jal azir\\[0.3cm]

				%%%%%%%%%%

				123	&	\multicolumn{3}{l}{\tit{dar\pafr-lim \vuvfr a-nu \eppl a\pha b-ra}}\\
				1,234	&	\multicolumn{3}{l}{\tit{azir-lim k\ej\lab i-dar\pafr-lim \eppl a\pha b-c\ej anu a\vuvfr\lab -ra}}\\
		\lspbottomrule
	\end{tabularx}
\end{table}

Cardinal numerals are used in counting and as modifiers of \isi{nouns} in noun phrases. In the latter function the noun appears in the singular form, but it controls plural agreement on the verb \refex{ex:childrenarethere10}, \refex{ex:Ifound99friends}. Examples of \isi{cardinal numerals} in use are \refexrange{ex:whenIwas20yearsold}{ex:Ifound99friends}.



\ea\label{ex:whenIwas20yearsold}
\gll	\vuvfr ajal dus w-i\uvfr-ub-le,~\ldots\\
	twenty year \tsc{m-}be\tsc{.pfv-pret-cvb}\\
\glt	\sqt{when (I) was 20 years old, \ldots}

\ex\label{ex:Iwouldhavegivenhimmoney}
\gll	arc luk\lmk-adi du-l k\lmk a\glpl al azir ak\lmk u=n, k\lmk a\glpl-c\ej al azir\\
	money give\tsc{.ipfv-cond.1} \tsc{1sg-erg} eight thousand \tsc{cop.neg=prt} eight-\tsc{ten} thousand\\
\glt	\sqt{I would have given him money, not just 8,000, but 80,000.}

\ex\label{ex:childrenarethere10}
\gll	dur\phfr-ne le-b kːaʔal, xujal rurs\lmk i ca\tang{b}i, \eppl a\pha bal dur\phfr u\pha\\
	boy\tsc{-pl} exist\tsc{-hpl} eight five girl \tsc{cop<hpl>} three boy\\
\glt	\sqt{I have 8 children (lit. there are 8), five daughters and three sons.}

\ex\label{ex:Ifound99friends}
\gll	du-l ur\paaf\ej em-c\ej anu ur\paaf\ej em-ra {julda\pafr} b-ar\paaf\lmk-ib=da\\
	\tsc{1sg-erg} nine-\tsc{ten} nine-\tsc{num} friend \tsc{hpl-}find\tsc{.pfv-pret=1}\\
\glt	\sqt{I found 99 friends.}
\z

Cardinal numerals can be nominalized. Case endings are directly added to numerals ending with a consonant. With numerals ending in a vowel an oblique marker \tit{-l} sometimes precedes the case suffixes; see \reftab{tab:inflectionalparadigmscardinalnumerals}. Examples are given in \refexrange{ex:tobuyflourfor600}{ex:fromthe1000rubles}.


\begin{table}
	\caption{Inflectional paradigms of selected cardinal numerals}
	\label{tab:inflectionalparadigmscardinalnumerals}
	\begin{tabularx}{0.95\textwidth}[]{>{\raggedright\arraybackslash}p{60pt} >{\itshape\raggedright\arraybackslash}X >{\itshape\raggedright\arraybackslash}X >{\itshape\raggedright\arraybackslash}X}
		\lsptoprule
			{}
		&	\multicolumn{1}{l}{\sqt{1}}
		&	\multicolumn{1}{l}{\sqt{2}}
		&	\multicolumn{1}{l}{\sqt{24}}\\

		\midrule

			\isit{absolutive}
		&	ca
		&	k\ej\lab el
		&	\vuvfr anu~a\vuvfr\lab ra\\

			\isit{ergative}
		&	ca-(l)-li
		&	k\ej\lab el-li
		&	\mbox{\vuvfr anu~a\vuvfr\lab ra-(l)-li}\\

			\isit{genitive}
		&	ca-(l)-la
		&	k\ej\lab el-la
		&	\vuvfr anu~a\vuvfr\lab ra-l-la\\

			\isit{dative}
		&	ca-(l)-li-j
		&	k\ej\lab el-li-j
		&	\vuvfr anu~a\vuvfr\lab ra-l-li-j\\

			\textsc{in}-lative
		&	ca-l-li-c\lmk e
		&	k\ej\lab el-li-c\lmk e
		&	\vuvfr anu~a\vuvfr\lab ra-li-c\lmk e\\\midrule

		%%%%%%%%%%

			{}
		&	\multicolumn{1}{l}{\sqt{100}}
		&	\multicolumn{1}{l}{\sqt{1,000}}\\

		\midrule

			\isit{absolutive}
		&	dˈ{a}r\pafr\lmk al
		&	ˈ{a}zir\\

			\isit{ergative}
		&	dˈ{a}r\pafr\lmk al-li
		&	ˈ{a}zir-li\\

			\isit{genitive}
		&	dˈ{a}r\pafr\lmk al-la
		&	ˈ{a}zir-la\\

			\isit{dative}
		&	dˈ{a}r\pafr\lmk al-li-j
		&	ˈ{a}zir-li-j\\

			\textsc{in}-lative
		&	dˈ{a}r\pafr\lmk al-li-c\lmk e
		&	ˈ{a}zir-li-c\lmk e\\
		\lspbottomrule
	\end{tabularx}
\end{table}


%
\ea\label{ex:tobuyflourfor600}
\gll	urek-dar\pafr-li-j wahi-l ak\lmk u\\
	six-hundred\tsc{-obl-dat} bad\tsc{-advz} \tsc{cop.neg}\\
\glt	\sqt{(To buy flour) for 600 (rubles per sack) is not bad.}

\ex\label{ex:fromthe1000rubles}
\gll	{a\uvfr\lab-dar\pafr} q\lmk uru{\pafr} \paaf ar d-arq\ej-ib azar-li-c\lmk e-r\\
	four-hundred ruble back \tsc{npl-}do\tsc{.pfv-pret} thousand\tsc{-obl-in-abl}\\
\glt	\sqt{From the 1,000 rubles he returned 400.}
\z


%%%%%%%%%%%%%%%%%%%%%%%%%%%%%%%%%%%%%%%%%%%%%%%%%%%%%%%%%%%%%%%%%%%%%%%%%%%%%%%%%


\section{Ordinal numerals}
\label{sec:ordinalnumerals}

Ordinal numerals are formed by adding the suffix \tit{-\glpl ib-il} (allomorph \tit{-\glpl ubil} with the stem of the numeral \sqt{four}, which contains a labialized consonant) or its short variant \tit{-\glpl ib}. The first part of this suffix originates from the root of the verb \sqt{say}, which is \tit{-\glpl-} plus the preterite suffix \tit{-ib}. The second part \tit{-il} in the long variant is the \isi{cross-categorical suffix} \tit{-il}, \refsec{ssec:The -il attributive}). Similar ways of forming \isi{ordinal numerals} have been reported for other Dagestanian languages (\teg Lezgian, see \citealp[233]{Haspelmath1993}; Akusha Dargwa, see \citealp[30 fn.10]{vandenBerg2001}; Hinuq, see \citealp[401\tnd403]{Forker2013a}).

\begin{table}
	\caption{Ordinal numerals}
	\label{tab:ordinalnumerals}
	\begin{tabularx}{0.85\textwidth}[]{>{\raggedleft\arraybackslash}p{35pt} >{\itshape\raggedright\arraybackslash}p{70pt} >{\raggedleft\arraybackslash}p{40pt} >{\itshape\raggedright\arraybackslash}X}
		\lsptoprule
				1st		&	ca-{\glpl}ibil
			&	11th		&	wec{\ej}nu cara-{\glpl}ibil\\
	
				2nd		&	k{\ej}{\lab}i-{\glpl}ibil
			&	20nd		&	{\vuvfr}a-{\glpl}ibil\\
	
				3rd		&	{\eppl}a{\pha}b-{\glpl}ibil
			&	30th		&	{\eppl}a{\pha}b-c{\ej}al-{\glpl}ibil\\
	
				4th		&	a{\vuvfr}-{\glpl}ubil
			&	41st		&	a{\vuvfr}{\lab}-c{\ej}anu ca-ra-{\glpl}ibil\\
	
				5th		&	xu-{\glpl}ibil
			&	52nd		&	xu-c{\ej}anu k{\ej}{\lab}i-ra-{\glpl}ibil\\
	
				6th		&	urek-{\glpl}ibil
			&	100th		&	dar{\pafr}al-{\glpl}ibil	\\
	
				7th		&	wer-{\glpl}ibil
			&	1,000th	&	azir-{\glpl}ibil	\\
	
				8th		&	k{\lmk}a{\glpl}-{\glpl}ibil
			&			&	\\
	
				9th		&	urč{\ej}em-{\glpl}ibil
			&		&	\\
	
				10th		&	wec{\ej}-{\glpl}ibil
			&	10,000th	&	wec{\ej}al azir-{\glpl}ibil\\[0.3cm]

				123rd		&	\multicolumn{3}{l}{\tit{dar{\pafr}lim {\vuvfr}anu {\eppl}a{\pha}bra-{\glpl}ibil}}\\
				1,234th	&	\multicolumn{3}{l}{\tit{azir-lim k{\ej}{\lab}i-dar{\pafr}-lim {\eppl}a{\pha}b-c{\ej}anu a{\vuvfr}{\lab}-ra-{\glpl}ibil}}\\
		\lspbottomrule
	\end{tabularx}
\end{table}

Ordinal numerals are inflected just like any other nominal, e.g. \tit{ca\glpl ibil} \sqt{first}, \isi{ergative} \tit{ca\glpl ibil-li}, \isi{genitive} \tit{ca\glpl ibil-la}, \isi{dative} \tit{ca\glpl ibil-li-j}, \tsc{in}-lative \tit{ca\glpl ibil-li-c\lmk e}, and so on.
%
\ea\label{ex:wemovedherein68}
\gll	he\pafr t\lmk u	pereselica	d-i\uvfr-ub=da			urek-c\ej anu	k\lmk a\glpl -ra-\glpl ib	dusːi-c\lmk e-d\\
	here			move		\tsc{1/2pl-}be\tsc{.pfv-pret=1}	six-\tsc{ten}		eight\tsc{-num-ord}		year\tsc{.obl-in-1/2pl}\\
\glt	\sqt{We moved here in (19)68.}

\ex\label{ex:sayinggotoIcaritogradefour}
\gll	``Uc\ej ari	a\vuvfr-\glpl ubil-li-c\lmk e		r-aš!''			b-ik\ej-ul\\
	Icari		four\tsc{-ord-obl-in}		\tsc{f-}go\tsc{.ipfv.imp}	\tsc{hpl-}say\tsc{.ipfv-icvb}\\
\glt	\sqt{They (were) saying, ``Go to Icari to grade four!''}

\ex\label{ex:hecametothe11thlock}
\gll	\paaf i-sa-{\O}-j\vuvfr-ib				wec\ej-nu ca-ra-\glpl ibil	me\uvfr-li-\pafr\lmk u\\
	\tsc{spr-hither}\tsc{-m-}come\tsc{.pfv-pret}	ten-\tsc{num} one\tsc{-num-ord}		iron\tsc{-obl-ad}\\
\glt	\sqt{He came to the 11th lock.}
\z

Ordinal numerals can also form the plural. In this case, the final \textit{-il} part is omitted because this suffix is not compatible with plural referents (\refsec{ssec:The -il attributive}). Example of plural \isi{ordinal numerals} are \textit{ca-{\glpl}ib-te} `the first ones', \textit{k{\ej}{\lab}i-{\glpl}ib-te} `the second ones', \textit{{\eppl}a{\pha}b-{\glpl}ib} `the third ones', etc. The oblique plural is formed according to the regular pattern of plural nominals with the suffix -\textit{te}, i.e., by using -\textit{ta}, e.g. the \isi{ergative} form of `the first ones' is \textit{ca-{\glpl}ib-t-a-l}.


%%%%%%%%%%%%%%%%%%%%%%%%%%%%%%%%%%%%%%%%%%%%%%%%%%%%%%%%%%%%%%%%%%%%%%%%%%%%%%%%%


\section{Distributive numerals}
\label{sec:distributivenumerals}

Distributive numerals are formed by reduplicating the root. Optionally the suffix \tit{-l(e)} follows the reduplicated numeral; see \reftab{tab:distributivenumerals}. The suffix \tit{-l(e)} seems to be the \isi{adverbializer} (\tcf \refsec{ssec:The adverbializer -le}). Note that with distributive numerals the modified noun bears overt plural marking \refex{ex:everysonalsohastwosonseach}.

\begin{table}
	\caption{Distributive numerals}
	\label{tab:distributivenumerals}
	\begin{tabularx}{0.45\textwidth}[]{>{\itshape\raggedleft\arraybackslash}X >{\raggedright\arraybackslash}X}
		\lsptoprule
			ca-ca(l)					&	\sqt{one each}\\
			k{\ej}{\lab}i-k{\ej}{\lab}i(l)		&	\sqt{two each}\\
			{\eppl}a{\pha}b-{\eppl}a{\pha}b(le)	&	\sqt{three each}\\
			a{\vuvfr}{\lab}-a{\vuvfr}{\lab}(le)	&	\sqt{four each}\\
			xu-xu(l)					&	\sqt{five each}\\
		\lspbottomrule
	\end{tabularx}
\end{table}

\ea\label{ex:hetookthreepearsoneeachforeveryone}
\gll	h-as{\lmk}-ib	{\eppl}a{\pha}bal	qa{\pha}r	haril-li-j	ca-ca\\
	\tsc{up}-take\tsc{.pfv-pret}	three	pear	every\tsc{-obl-dat}	one-one\\
\glt	\sqt{(He) took three pears, one each for everyone.}

\ex\label{ex:everysonalsohastwosonseach}
\gll	har	dur{\phfr}u{\pha}-la	k{\ej}{\lab}i-k{\ej}{\lab}i	dur{\phfr}-ne=ra	le-b\\
	every	boy\tsc{-gen}	two-two	boy\tsc{-pl=add}	exist\tsc{-hpl}\\
\glt	\sqt{Every son also has two sons each.}

\ex\label{ex:putthemdownfiveeach}
\gll	xu-xu-l	ka-d-ix{\lmk}-a!\\
	five-five\tsc{-advz}	\tsc{down}\tsc{-npl-}put\tsc{.pfv-imp}\\
\glt	\sqt{Put them down five each!} (E)
\z



%%%%%%%%%%%%%%%%%%%%%%%%%%%%%%%%%%%%%%%%%%%%%%%%%%%%%%%%%%%%%%%%%%%%%%%%%%%%%%%%%


\section{Group numerals}
\label{sec:groupnumerals}

Group numerals are formed by adding the suffix \tit{-\tsc{gm}-a} to the root; see \reftab{tab:groupnumerals}. The \isi{gender} marker shows only plural agreement (\tit{-b} or \tit{-d}). The suffix can also be added to the \isi{quantifier} \tit{b-aq} \sqt{many, much}.

\begin{table}
	\caption{Some group numerals}
	\label{tab:groupnumerals}
	\begin{tabularx}{0.40\textwidth}[]{>{\raggedleft\arraybackslash}p{30pt} >{\itshape\raggedright\arraybackslash}X}
		\lsptoprule
			1	&	ca-b-a, ca-d-a\\
			2	&	k{\ej}{\lab}i-b-a, k{\ej}{\lab}i-d-a\\
			3	&	{\eppl}a{\pha}b-d-a\\	
			4	&	a{\vuvfr}{\lab}-d-a\\
			5	&	xu-d-a\\
			10	&	wec{\ej}-d-a\\
			20	&	{\vuvfr}a-d-a\\
			1000	&	azir-d-a\\
		\lspbottomrule
	\end{tabularx}
\end{table}

Group numerals denote groups or pairs of items. In my corpus, only the \isi{group numeral} of \tit{ca} \sqt{one} is used with human plural agreement (\tit{ca-b-a}) and its meaning is very similar to the \isi{indefinite pronoun} \sqt{some(one), somebody}. The human plural form \textit{ca-b-a} is also used as \isi{reciprocal pronoun} (see \refsec{sec:Reciprocal pronouns} for the case paradigm and  \refsec{sec:Reciprocal constructionss} for one example). All other \is{group numeral}group numerals carry the neuter plural suffix \tit{-d} \xxref{ex:thedoctortriedtenpairsofglasses}{ex:killingsomewoundingothers}. Group numerals can be attributes of \isi{nouns}, which normally occur in the plural \refex{ex:thedoctortriedtenpairsofglasses}. Group numerals can be inflected by adding case suffixes to the suffix \tit{-\tsc{gm}-a} \refex{ex:somepeopletookpitchforks}. They can also take the attributive plural suffix \tit{-te} (oblique form \tit{-t-a-}) as in the same example.

\ea\label{ex:thedoctortriedtenpairsofglasses}
\gll	wec{\ej}-d-a	{\eppl}a{\pha}{\paaf}k{\lmk}a-be=ra	sa-d-uc-ib\\
	ten\tsc{-npl-group}	glass\tsc{-pl=add}	\tsc{ante-npl-}keep\tsc{.pfv-pret}\\
\glt	\sqt{(The doctor) tried (with me) ten pairs of glasses.}

\ex\label{ex:afterIsawyoumanythoughtsarose}
\gll	u	{\paaf}i-r-až-ib-la\parn{,}	wer-d-a	hak{\ej}-ub-le	pikru-me	le-d\\
	\tsc{2sg}	\tsc{spr-f-}see\tsc{.pfv-pret-post}	seven\tsc{-npl-group}	appear\tsc{.pfv-pret-cvb}	thought\tsc{-pl}	exist\tsc{-npl}\\
\glt	\sqt{After I saw you, many (lit. seven groups of) thoughts arose.}

\ex\label{ex:killingsomewoundingothers}
\gll	kax-ub-le	ca-b-a-te	ca-b-a-te	b-a{\pha}q-ib-le	t{\ej}ut{\ej}u.q{\ej}a{\pha}t{\ej}	b-arq{\ej}-ib-le	hel-t{\lmk}i\\
	kill\tsc{.pfv}-\tsc{pret-cvb}	one\tsc{-hpl-group-dd.pl}		one\tsc{-hpl-group-dd.pl} 	\tsc{hpl-}wound\tsc{.pfv-pret-cvb}		drive.away	\tsc{hpl-}do\tsc{.pfv-pret-cvb}	that\tsc{-pl}\\
\glt	\sqt{Killing some, wounding others, he drove them away.}

\ex\label{ex:somepeopletookpitchforks}
\gll	ca-b-a-l	q{\ej}ig-me,	ca-d-a-l	beret-me	cara-t-a-l	k{\lmk}alk-me	k{\lab}i	sa-b-e{\vuvfr}-ib-il	{\phfr}a{\pha}sible	ca-l-li	b-alli	b-erq{\lmk}-ib	ca-b	unc-a-la	duk{\ej}\\
	one\tsc{-hpl-group-erg}	pitchfork\tsc{-pl}	one\tsc{-npl-group-erg}	ax\tsc{-pl}	other\tsc{-pl-obl-erg}	tree\tsc{-pl}	in.the.hands	\tsc{hither-n-}go\tsc{.pfv-pret-ref}	according.to	one\tsc{-obl-erg}	\tsc{n-}together	\tsc{n-}carry\tsc{.pfv-pret}	\tsc{cop-n}	ox\tsc{-obl-gen}	yoke\\
\glt	\sqt{Some (people) took pitchforks, some axes, whatever was at hand, one carried with himself the yoke of an ox.}
\z



%%%%%%%%%%%%%%%%%%%%%%%%%%%%%%%%%%%%%%%%%%%%%%%%%%%%%%%%%%%%%%%%%%%%%%%%%%%%%%%%%


\section{Multiplicative numerals}
\label{sec:multiplicativenumerals}

Multiplicative numerals are formed by means of the suffix \tit{-na} (\tit{-jna} after vowels) that is added to the root; see \reftab{tab:multiplicativenumerals} and \refex{ex:hewentthreetimestothehajj}. This suffix can also be added to the interrogative pronoun \tit{{\paaf}um} \sqt{how much, how many} plus the suffix \tit{-ra}, and then leads to the \isi{indefinite pronoun} \sqt{how often ever, many times} (\refsec{sssec:cujna how many times}).

\begin{table}
	\caption{Some multiplicative numerals}
	\label{tab:multiplicativenumerals}
	\begin{tabularx}{0.45\textwidth}[]{>{\itshape\raggedleft\arraybackslash}X >{\raggedright\arraybackslash}X}
		\lsptoprule
			ca-jna			&	\sqt{once}\\
			k{\ej}{\lab}i-jna	&	\sqt{twice}\\
			{\eppl}a{\pha}-jna	&	\sqt{three times}\\
			a{\vuvfr}{\lab}-na	&	\sqt{four times}\\
			wec{\ej}-na		&	\sqt{ten times}\\
			{\vuvfr}a-jna	&	\sqt{20 times}\\
			 dar{\pafr}-na	&	\sqt{100 times}\\	
			azir-na		&	\sqt{1,000 times}\\
		\lspbottomrule
	\end{tabularx}
\end{table}

\ea\label{ex:hewentthreetimestothehajj}
\gll	{\eppl}a{\pha}j-na	ag-ur-ce=de,	ij	dam	a-ag-ur	caj-na	arrah\\
	three-\tsc{time}	go\tsc{.pfv-pret-dd.sg=pst}	this	\tsc{1sg.dat}	\tsc{neg-}go\tsc{.pfv-pret}	one-\tsc{time}	at.least\\
\glt	\sqt{He went three times (to the Hajj), to me this happened not even once.}
\z

From the \isi{multiplicative numerals} expressions referring to time points can be formed by means of various derivational and inflectional suffixes. The words in \refex{ex:atthesecondtime} all mean \sqt{(at) the second time}.\footnote{In \refex{ex:atthesecondtime}, although the suffix \tit{-lla} in two of the given words strongly resembles the \isi{genitive}, it is, at least synchronically, distinct from the case marker, since it is possible to add the \isi{genitive} to an adverb with \tit{-lla}, e.g. \tit{k{\ej}{\lab}i-jna-lla-la} \sqt{of the second time}.} Examples from texts are presented in \refexrange{ex:twotimeshesendhimaway}{ex:thesecondtimetherewashazhittaj}.


	\ea	\label{ex:atthesecondtime}
	
	\TabPositions{10em,12em}
		\tit{k{\ej}{\lab}i-jna-l 	\tab k{\ej}{\lab}i-jna-lla}	 \\
		\tit{k{\ej}{\lab}i-jna-l-li \tab	k{\ej}{\lab}i-jna-lla-li-j}	 \\
		\tit{k{\ej}{\lab}i-jna-le}					
	

\ex\label{ex:twotimeshesendhimaway}
\gll	k{\ej}{\lab}i-jna	t{\lmk}ura	w-er{\paaf}-ib il	ni{\pafr}{\lmk}a-la	qili-rka		{\eppl}a{\pha}j-na-la-li-j	t{\lmk}ura	w-er{\paaf}-ib\\
	two-\tsc{time}	outside	\tsc{m-}lead\tsc{.pfv-pret}	that	\tsc{1pl-gen}	home\tsc{-abl}		three-\tsc{time}\tsc{-?-obl-dat}	outside	\tsc{m-}lead\tsc{.pfv-pret}\\
\glt	\sqt{Two times (he) send him away, our (guy), out of the room, the third time (he) sent him away.}

\ex\label{ex:thesecondtimetherewashazhittaj}
\gll	k{\ej}{\lab}i-jna-le	{\phfr}a{\pha}žit{\lmk}aj=de,	``w-alli	ka-{\O}-jž-e,''	{\O}-ik{\ej}{\lab}-ar, ``hež-i-c{\lmk}ella	w-alli!''\\
	two-\tsc{time}\tsc{-loc}	Hazhittaj\tsc{=pst}	\tsc{m-}together	\tsc{down}\tsc{-m-}remain\tsc{-imp}	\tsc{m-}say\tsc{.ipfv-prs}	this\tsc{-obl-comit}	\tsc{m-}together\\
\glt	\sqt{The second time there was Hazhittaj saying, ``Sit together, with him together (in the back of the car)!''}
\z



%%%%%%%%%%%%%%%%%%%%%%%%%%%%%%%%%%%%%%%%%%%%%%%%%%%%%%%%%%%%%%%%%%%%%%%%%%%%%%%%%


\section{Collective numerals}
\label{sec:collectivenumerals}

Collective numerals are formed by adding the \isi{additive enclitic} \tit{=ra} to the \isi{cardinal numerals}; see \refex{ex:collectivenumerals}. They can function as attributes of \isi{nouns} \refex{ex:mythreechildrenallplayedtricks} or they can be nominalized and then occur on their own \refex{ex:allchildrenwerebroughtupwell} or be modified by \is{demonstrative pronoun}demonstrative pronouns \refex{ex:thenthesetwoalsometonthemarket}.


	\ea	\label{ex:collectivenumerals}
		\TabPositions{12em}
		\tit{k{\ej}{\lab}el=ra} \sqt{both, all two} \tab \tit{{\eppl}a{\pha}b-al=ra} \sqt{all three} \\
		\tit{wec{\ej}-al=ra}		\sqt{all ten} \tab \tit{dar{\pafr}al=ra}	\sqt{all 100} \\
		\tit{azir=ra}				\sqt{all 1,000}


\ex\label{ex:mythreechildrenallplayedtricks}
\gll	he{\uvfr}-t{\lmk}i	{\eppl}a{\pha}bal=ra	dur{\phfr}u{\pha}-l	a{\vuvfr}{\lab}al=ra	{\eppl}a{\pha}mal	b-arq{\ej}-ib\\
	\tsc{dem.down}\tsc{-pl}	three\tsc{=add}	boy\tsc{-erg}	four\tsc{=add}	trick	\tsc{n-}do\tsc{.pfv-pret}\\
\glt	\sqt{The (my) three children all played tricks on me (\tie caused trouble).}

\ex\label{ex:allchildrenwerebroughtupwell}
\gll	wec{\ej}al=ra	{\eppl}a{\pha}{\phfr}-le	ha-b-iq{\ej}-un-te=de\\
	ten\tsc{=add}	good\tsc{-advz}	\tsc{up-hpl-}bring.up\tsc{-pret-dd.pl=pst}\\
\glt	\sqt{All ten (children) were brought up well.}

\ex\label{ex:thenthesetwoalsometonthemarket}
\gll	c{\ej}il	heba=ra	na	can	b-i{\paaf}-ib ca-b	bazar-re-b	Kuba{\paaf}i-b	il-t{\lmk}i	k{\ej}{\lab}el=ra\\
	then	then\tsc{=add}	now	meet	\tsc{hpl-}occur\tsc{.pfv-pret} \tsc{cop-hpl}	market\tsc{-loc-hpl}	Kubachi\tsc{-hpl}	that\tsc{-pl}	two\tsc{=add}\\
\glt	\sqt{Then these two also met on the market in Kubachi.}
\z

%%%%%%%%%%%%%%%%%%%%%%%%%%%%%%%%%%%%%%%%%%%%%%%%%%%%%%%%%%%%%%%%%%%%%%%%%%%%%%%%%



\section{Other numeral expressions and compounds involving numerals}
\label{sec:othernumeralexpressions}

Expressions for fractions are given in \refex{ex:fractions} and \refexrange{ex:thentheywentandwent}{ex:hespentthreeandahalfyears}; \tit{but{\ej}a} translates as \sqt{piece, part} into English. The word \tit{b-abq{\ej}i} \sqt{half} agrees in \isi{gender} and \isi{number} with its head noun if it occurs in attributive function \refex{ex:thentheywentandwent}, \refex{ex:hespentthreeandahalfyears}.
%

\begin{exe}\ex\label{ex:fractions}
		\TabPositions{12em}
		\tit{b-abq{\ej}i}	\sqt{(one) half} \tab \tit{a{\vuvfr}ubil but{\ej}a} \sqt{one fourth} \\
		\tit{{\eppl}a{\pha}bibil but{\ej}a}	\sqt{one third} \tab \tit{a{\vuvfr}{\lab}allic{\lmk}er {\eppl}a{\pha}bal (but{\ej}a)}	\sqt{three fourth}


\ex\label{ex:thentheywentandwent}
\gll	c{\ej}il d-ax-ul d-ax-ul	d-abq{\ej}i x{\lmk}un aq-ib	zamana,~\ldots\\
	then \tsc{nhpl-}go\tsc{.ipfv-icvb} \tsc{nhpl-}go\tsc{.ipfv-icvb}	\tsc{nhpl-}half way go.through\tsc{.pfv-pret}	time\\
\glt	\sqt{Then they went and went, and when they went half of the way, \ldots}

\ex\label{ex:hespentthreeandahalfyears}
\gll	ʡa{\pha}bal=ra	b-abq{\ej}i	dus	{\eppl}a{\pha}rmija-c{\lmk}e-w	kelg-un\\
	three\tsc{=add}	\tsc{n-}half	year	army\tsc{-in-m}	remain\tsc{.pfv-pret}\\
\glt‎\sqt{He spent three and a half years in the army.}
\end{exe}

The Sanzhi terms for the school grades are formed by adding \tit{-la} (\tit{-lla} after vowels) to the root of the numerals 1\tnd5 (with 5 being the best grade and 1 the worst): \tit{calla} \sqt{one}, \tit{k{\ej}{\lab}illa} \sqt{two}, \tit{{\eppl}a{\pha}bla} \sqt{three}, \tit{a{\vuvfr}{\lab}la} \sqt{four}, \tit{xulla} \sqt{five} \refex{ex:whatgradedidyougetIgotafive}. 

\ea\label{ex:whatgradedidyougetIgotafive}
\gll	ce	b-i{\paaf}-ib=e	at?	dam	b-i{\paaf}-ib	xulla\\
	what	\tsc{n-}get\tsc{.pfv-pret=q}	\tsc{2sg.dat}	\tsc{1sg.dat}	\tsc{n-}get\tsc{.pfv-pret}	five\\
\glt	\sqt{What did you get (\tie which grade)? I got a five.} (E)
\z

Other words that are derived from numerals are the terms \tit{k{\ej}{\lab}idarq{\ej}i} \sqt{twins} and \tit{{\eppl}a{\pha}bdarq{\ej}i} \sqt{triplets} (from the numerals \sqt{two} and \sqt{three} and the verb \tit{b-arq{\ej}-ij} \sqt{do, make}). Then there are terms for traditional events and rituals that occur after the death of a person, namely, 
%

	\ea	\label{ex:daysafterdeath}
		\TabPositions{12em}
		\tit{{\eppl}a{\pha}bil-la}			\sqt{three days} \tab \tit{a{\vuvfr}{\lab}c{\ej}al-la}		\sqt{40 days} \\
		\tit{xuc{\ej}anu k{\ej}{\lab}ira-la}	\sqt{52 days} \tab 	\tit{dus{\lmk}i-la}				\sqt{one year}


\ex\label{ex:afterthreedayspeoplegivealms}
\gll	{\eppl}a{\pha}billa-li-j=ra	arc luk{\lmk}-an ca-d,	a{\vuvfr}{\lab}c{\ej}alla-li-j=ra	arc	luk{\lmk}-un ca-d,		dus{\lmk}i-la	a{\vuvfr}{\lab}al	zikru\\
	three.days\tsc{-obl-dat=add}	money give\tsc{.ipfv-ptcp} \tsc{cop-npl}	forty.days\tsc{-obl-dat=add}	money	give\tsc{.ipfv-pret} \tsc{cop-npl}	year\tsc{.obl-gen}	four	dhikr\\
\glt	\sqt{After three days (people) give (alms), after 40 days, after one year, four dhikrs.}
\z

Compound \isi{nouns} and \isi{adjectives} can contain numerals, e.g., \tit{k{\ej}{\lab}i-dus{\lmk}-an k{\ej}a{\pafr}{\lmk}a}  \sqt{two-year old bull}, \tit{{\eppl}a{\pha}b-da{\pha}r{\uvfr}-la qul-be} (three-floor\tsc{-gen} house\tsc{-pl}) \sqt{three-floor houses} (see \refsec{ssec:AdjN} and \refsec{sec:Derivation of adjectives} for more examples).

Counting is exemplified in \refex{ex:twoplusthreeequalsfive} and \refex{ex:sevenminusoneequalssix}.
%
\ea\label{ex:twoplusthreeequalsfive}
\gll	k{\ej}{\lab}el-le	{\eppl}a{\pha}bal	{\paaf}i-ka-b-ix-ar	b-ir{\uvfr}{\lab}-u	arg-u	xujal\\
	two\tsc{-loc} three	\tsc{spr-down}\tsc{-n-}throw\tsc{.pfv-cond}	\tsc{n-}become\tsc{.ipfv-prs} go\tsc{.ipfv-prs} five\\
\glt	\sqt{Two plus three equals five.} (lit. if you throw three onto two five happens)

\ex\label{ex:sevenminusoneequalssix}
\gll	weral-li-c{\lmk}e-r	gu-r-h-as{\lmk}-ar	{\paaf}i-r-h-as{\lmk}-ar	ca	arg-u	urek{\lmk}al\\
	seven\tsc{-obl-in-abl}	\tsc{sub-abl-up}-take\tsc{.pfv-prs}	\tsc{spr-abl-up}-take\tsc{.pfv-prs}	one	go\tsc{.ipfv-prs}	six\\
\glt	\sqt{Seven minus one equals six.} (if you take away one from seven it goes six)
\z

%%%%%%%%%%%%%%%%%%%%%%%%%%%%%%%%%%%%%%%%%%%%%%%%%%%%%%%%%%%%%%%%%%%%%%%%%%%%%%%%%
