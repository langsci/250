\chapter{Place names and microtoponyms}
\label{cpt:morph-placenames}

Tables \ref{tab:Names for villages, towns, and districts, and their inhabitants} and \ref{tab:Generic locations and their inhabitants} show names for the villages, towns, and districts that are relevant to the Sanzhi people. The tables first provide the citation form of the place name followed by the essive case, \tie\ the word form that needs to be used when answering the question \tit{Where are you?} The last two place names, Druzhba and Makahchkala, morphosyntactically differ from all the others because they represent recent borrowings. In order to form the essive case they need to employ the locational case suffix -\textit{le} (\refsec{sssec:spr-lative -le/-ja/-a, spr-essive -le-b/-ja-b/-a-b and spr-ablative -le-r/-ja-r/-a-r}). The other place names do not need such an additional \isi{spatial case} because the place names have by themselves spatial meaning just like spatial adverbials because this is their default use. With these older place names it might diachronically be possible to identify a root morpheme that represents the place name followed by a \isi{spatial case} suffix, but synchronically Sanzhi has no \is{spatial case}spatial cases that consist of a vowel \textit{i} (the most frequent word-final segment of the place names in Table \ref{tab:Names for villages, towns, and districts, and their inhabitants}).  Other Dargwa varieties such as Chirag \citep{GanenkovChiragSketch}, however, have a \isi{spatial case} expressed by a suffix \textit{-i} that functionally resemble the Sanzhi locational case.

The third column contains referential-attributive terms that are semantically related to the respective places. These terms are formed by adding \tit{-(a)n} to a root that can be the place name or some other root related to it. This suffix might be a cognate of the \isi{locative participle} suffix \textit{-an} (\refsec{sssec:The locative participle}) and/or the \is{interrogative clause}interrogative clauses -\textit{an} \refsec{sssec:The modal participle -an}. Another possible cognate is the \isi{adjectivizer} \textit{-(a)n}, which is used for the formation of a few \isi{adjectives} involving \isi{compounding} with numerals and mostly plural \isi{nouns} (\refsec{sec:Derivation of adjectives}). The same suffix seems to occur in the \isi{derivation} of the adjective \textit{b-urkːa-l-an} \sqt{middle} from the postposition \textit{b-urkːa} \sqt{between}. In the default case, these terms refer to the inhabitants of the respective places as the term \textit{the English} can refer to English people. They are also used as attributes of head \isi{nouns} that do not refer to human beings but to their language, customs, clothes, etc., just like the use of \textit{English} in the phrase \textit{the English language}.

Syntactically, the referential attributes in the third column function like other referential attributes formed by means of the two \is{cross-categorical suffix}cross-categorical suffixes -\textit{ce} and -\textit{il} (\refsec{ssec:The -ce / -te attributive} and \refsec{ssec:The -il attributive}). This means that they largely possess the syntactic properties of \isi{nouns}. They are used in argument position \refex{ex:The Urkarakh people complained and put him into prison} or as predicates \refex{ex:Was he Kumyk No (he) was Dargwa}. They can also modify \isi{nouns} as the last column `language' shows. The constructions in the last column, which resemble compound \isi{nouns} a bit (\refsec{ssec:AdjN}), can probably be analyzed as nominal appositions similar to the combination of proper names and kinship terms (\refsec{ssec:Lexical, phrasal, and clausal modifiers in noun phrases}). Just like with referential attributes that are marked with -\textit{ce}, plural formation occurs by means of the most common plural suffix \tit{-te} \refex{ex:The Urkarakh people complained and put him into prison}. 

The fourth column contains terms referring to the ethnic group. These terms are a kind of mass \isi{nouns} that trigger human plural agreement like the word \tit{χalq'} \sqt{people}. The last column contains the terms for the language. Language names contain the word \tit{ʁaj} \sqt{language}, which is preceded by either (i) the singular term for the inhabitants, (ii) the \isi{genitive} of the term for the ethnic group, or (iii) the \isi{genitive} of the place name.

\begin{table}
	\caption{Names for villages, towns, and districts, and their inhabitants}
	\label{tab:Names for villages, towns, and districts, and their inhabitants}
	\small
	\begin{tabularx}{1\textwidth}[]{%
		>{\raggedright\arraybackslash\hangindent=0.5em}p{42pt}
		>{\raggedright\arraybackslash\hangindent=0.5em\itshape}p{38pt}
		>{\raggedright\arraybackslash\hangindent=0.5em\itshape}p{54pt}
		>{\raggedright\arraybackslash\hangindent=0.5em\itshape}p{60pt}
		>{\raggedright\arraybackslash\hangindent=0.5em\itshape}p{34pt}
		>{\raggedright\arraybackslash\hangindent=0.5em\itshape}p{62pt}}
		
		\lsptoprule
		{}		&	\upshape place	&	\upshape essive			&	\upshape referential &	\upshape ethnic\\
		{}		&	\upshape name	&	\upshape case	&	\upshape attribute	&	\upshape group	&	\upshape language\\
		\midrule
		Sanzhi		&	sːanži		&	sːanži-b		&	sːunglan(te)		&	sːungul		&	sːunglan /\\
		{}		&	{}		&	{}			&	{}			&	{}		&	~sːunglila,\\
		{}		&	{}		&	{}			&	{}			&	{}		&	~sːungulla ʁaj\\
		Icari		&	uc'ari		&	uc'ari-b		&	uc'ran(te)		&	\tmd		&	uc'ran /\\
		{}		&	{}		&	{}			&	{}			&	{}		&	~uc'rila ʁaj\\
		Chakhri	&	čːiħri		&	čːiħri-b			&	čːuˁħrugan(te)	&	čːuˁħrug	&	čːuˁħrugan,\\
		{}		&	{}		&	{}			&	{}			&	{}		&	~čːuˁħrugla ʁaj\\
		Kubachi	&	ʡuˁrbuži 	&	ʡuˁrbuži-b		&	ʡuˁrbugan(te),	&	ʡuˁrbug	&	ʡuˁrbugan /\\
		{}		&	{}		&	{}			&	ʡuˁrbuglan(te)	&	{}		&	~ʡuˁrbugla /\\
		{}		&	{}		&	{}			&	{}			&	{}		&	~ʡuˁrbužila ʁaj\\
		Shari		&	šurgli		&	šurgli-b		&	šurglan(te)		&	\tmd		&	šurglan /\\
		{}		&	{}		&	{}			&	{}			&	{}		&	~šurglila ʁaj\\
		Sursar-	&	sursarbač'i	&	sursarbač'i-b		&	sursbuk'an(te)	&	sursbuk'	&	sursbuk'an /\\
		~~Bachi	&	{}		&	{}			&	{}			&	{}		&	~sursbuč'ila ʁaj\\
		Sanakari	&	sanaqari	&	sanaqari-b		&	sunqlugan(te)	&	sunqlug	&	sunqlugan /\\
		{}		&	{}		&	{}			&	{}			&	{}		&	~sunqlužila ʁaj \\
		Khuduc	&	xuduc'a	&	xudec'a-b		&	xudec'an(te)		&	\tmd		&	xudec'an ʁaj\\
		Ashti		&	eštːa		&	eštːa-b			&	eštːan(te)		&	\tmd		&	eštːan ʁaj\\
		Ankluk		&	ank'luʁ	&	ank'luʁ-a-b		&	ank'luʁan(te)		&	ank'luʁi	&	ank'luʁila /\\
		{}		&	{}		&	{}			&	{}			&	{}		&	~ank'luʁan ʁaj\\
		Urkarakh	&	urkuq(i)	&	urkuqi-b, 		&	urkuqan(te)		&	urkuq		&	urkuqla /\\
		{}		&	{}		&	~\mbox{urkaraqari-b,}	&	{}			&	{}		&	~urkuqan /\\
		{}		&	{}		&	~urkaraq-le-b	&	{}			&	{}		&	~urkuqila ʁaj\\
		Kala-		&	urc'mucːi	&	urc'mucːi-b		&	urc'mucːan(te) 	&	urc'muc	&	urc'mucːan /\\
		~~Kurejsh	&	{}		&	{}			&	{}			&	{}		&	~urc'mucːila ʁaj\\
		Sirga  		&	sarħaˁ 	&	sarħaˁ-b		&	sarħaˁn(te)		&	\tmd		&	sarħaˁntala /\\
		~~(district)	&	{}		&	{}			&	{}			&	{}		&	~sarħaˁla ʁaj\\
		Druzhba	&	družba	&	družba-le-b		&	družbala~šːante~/ 	&	\tmd		&	\tmd\\
		{}		&	{}		&	{}			&	~družbalan(te)\\
		Makhach-	&	maˁħaˁč-	&	maˁħaˁč- 		&	maˁħaˁč-		&	\tmd		&	\tmd\\
		~~-kala	&	~~-qːala	&	~~-qːala-le-b	&	~~-qːalan(te)\\
		\lspbottomrule
	\end{tabularx}
\end{table}
%

As can be seen in Table \ref{tab:Generic locations and their inhabitants}, the noun \textit{qːatːa} forms the essive case by changing the pitch accent to the final vowel (this is an irregular way to form the locational case; it is also found with a few other \isi{nouns}). The noun \textit{šːi} `village' also has an irregular locational case form, whereas \textit{dubur} is regularly inflected for either the \textsc{loc}-series (suffix -\textit{le}, assimilated to -\textit{re}) or the \textsc{in}-series (suffix -\textit{cːe}). If not specified otherwise, \textit{šːi} `village' refers to the village of Sanzhi.


\begin{table}
	\caption{Generic locations and their inhabitants}
	\label{tab:Generic locations and their inhabitants}
	\small
	\begin{tabularx}{0.85\textwidth}[]{%
		>{\raggedright\arraybackslash}p{42pt}
		>{\raggedright\arraybackslash\itshape}p{27pt}
		>{\raggedright\arraybackslash\itshape}p{66pt}
		>{\raggedright\arraybackslash\itshape}p{53pt}
		>{\raggedright\arraybackslash\itshape}p{62pt}}
		
		\lsptoprule
		{}		&	\upshape place	&	{}			&	\upshape referential\\
		{}		&	\upshape name	&	\upshape essive	&	\upshape attribute	&	\upshape language\\
		\midrule
		canyon	&	qːátːa		&	qːatːá-b		&	qːatːigan(te) 	&	\tmd\\
		mountain	&	dubur		&	dubur-t-a-cːe-b /	&	duburlan(te)		&	duburla ʁaj\\
		{}		&	{}		&	~dubur-re-b\\
		village		&	šːi		&	šːa-b,			&	šːan(te)		&	šːila ʁaj\\
		{}		&	{}		&	~šːi-l-cːe-b\\
		\lspbottomrule
	\end{tabularx}
\end{table}

The place names only inflect for directional cases (essive, lative, \isi{ablative}). As can be seen when comparing the two columns in \reftab{tab:Names for villages, towns, and districts, and their inhabitants}, the place names mostly have directional meaning, \tie\ the lative is identical to the place names themselves. Examples are given in \refexrange{ex:I came to Urkarakh by call}{ex:one through the peak on which there is}. In the speech of a few younger speakers (age 30 or younger) I noticed the use of the  \textsc{loc}-series marker with the word \textit{sːanži}, i.e., they used the explicit marking \textit{sːanži-le} instead of \textit{sːanži} when talking about going to the village \refex{ex:He always told us}. This might be due to Russian influence because Russian place names do not have inherent locative meaning, but require explicit case marking (in Sanzhi and Russian) as the last two lines in Tables \ref{tab:Names for villages, towns, and districts, and their inhabitants} show.


\begin{exe}
	\ex	\label{ex:I came to Urkarakh by call}
	\gll   	du	priziw-li	ka-∅-ač'-ib=da	urkaraqari\\
		\textsc{1sg}	call-\textsc{erg}	\textsc{down-m}-come.\textsc{pfv-pret}=1	Urkarakh\\
	\glt 	 \sqt{I (masc.) came to Urkarakh by call.} (\tie\ \sqt{I was called to Urkarakh.})

	\ex	\label{ex:My mother was in Chakhri, at her brother's place}
	\gll  	aba	čːiħri-r=de	cin-na	ucːi-li-šːu-r\\
		mother	Chakhri-\textsc{f=pst}		\textsc{refl.sg-gen}	brother-\textsc{obl-ad-f}\\
	\glt	\sqt{My mother was in Chakhri, at her brother's place.}

	\ex	\label{ex:one through the peak on which there is}[There were four ways leading to our village,]\\
	\gll  	ca	ce	či-b-il	bek'-le-rka,	ca	uc'ari-rka,	ca	χudec'a-rka,		ca	šaˁrʡaˁ-rka\\
		one	what	on-\textsc{n-adjvz}	head-\textsc{loc-abl}	one	Icari-\textsc{abl}	one	Khuduc-\textsc{abl}	one	Shari-\textsc{abl}\\
	\glt  	\sqt{‎‎one through the peak on which there is something, one from Icari, one from Khuduc, and one from Shari.}
\end{exe}

The referential attributive terms and the terms for the ethnic groups (fourth and fifth column) inflect like standard \isi{nouns}, for example \textit{sungul} \sqt{Sanzhi people}, \isi{ergative} \textit{sungul-li}, \isi{genitive} \textit{sungul-la}\slash\textit{sungli-la}, \isi{dative} \textit{sungul-li-j}, and \textit{sunglante} \sqt{Sanzhi villagers}, \isi{ergative} \textit{sunglan-t-a-l}, \isi{genitive} \textit{sunglan-t-a-la}, and so on.

\begin{exe}
	\ex	\label{ex:All Sanzhi people came out (of their houses)}
	\gll	tːura	ka-b-uq-un-ne	li<b>il=ra	sungul\\
		outside	\textsc{down-hpl}-go.\textsc{pfv-pret-cvb}	all<\textsc{hpl>=add}	Sanzhi.people\\
	\glt  	\sqt{All Sanzhi people came out (of their houses).}

	\ex	\label{ex:We look: the village of Icari}
	\gll	er d-ik'-ul=da:	uc'ri-la	šːi  \\
		look 1/2\textsc{pl}-look.at.\textsc{ipfv-icvb}=1	Icari-\textsc{gen}	village\\
	\glt	\sqt{We are looking: the village of Icari.}

	\ex	\label{ex:The Urkarakh people complained and put him into prison}
	\gll	ʡaˁrz	w-arq'-ib-le,	tusnaq	w-arq'-ib	urkuqan-t-a-l   \\
		complain	\textsc{m-}do.\textsc{pfv-pret-cvb}	prison	\textsc{m}-do.\textsc{pfv-pret} Urkarakh.person-\textsc{pl-obl-erg}\\
	\glt	\sqt{The Urkarakh people complained and put him into prison.}
\end{exe}

\reftab{tab:Ethnic groups} displays terms for referential attributes that mostly denotate ethnic groups of the Caucasus and the names of the respective languages. Many of the referential attributes are also formed by means of the suffix \textit{-an}. Some examples illustrating the usage are given in \refexrange{ex:This is called nature reserve in Russian}{ex:Was he Kumyk No (he) was Dargwa}. As example \refex{ex:This is called nature reserve in Russian} shows, the terms that contain \isi{genitive} suffixes can also be used without  head \isi{nouns} (e.g. \tit{ʁaj} \sqt{language} in this examples) if the reference is clear from context.  

\begin{table}
	\caption{Ethnic groups}
	\label{tab:Ethnic groups}
	\small
	\begin{tabularx}{0.86\textwidth}[]{%
		>{\raggedright\arraybackslash}p{56pt}
		>{\raggedright\arraybackslash\itshape}p{108pt}
		>{\raggedright\arraybackslash\itshape}X}
		
		\lsptoprule
		\upshape ethnic group	&	\upshape attributes (\tsc{sg}, \tsc{pl})	&	\upshape language\\
		\midrule
		Avar		&	k'araqan(te)				&	k'araqan\slash k'araqala ʁaj\\
		Lak 		& 	belekːʷan(te)				&	belekːʷan\slash belekːʷala ʁaj\\
		{}			& 	~ \upshape{(<} \textit{belekːʷa} \upshape{`Lakia'})	&	{}\\
		Lezgian		&	lezgi(be)					&	lezgi ʁaj\\
		Tabasaran	&	tabasran(te)				&	tabasran ʁaj\\
		Dargwa		&	darkːʷan(te)				&	darkːʷan\slash darkːʷala ʁaj\\
		Kumyk		&	žaˁndar(te), 				&	žaˁndar /\\
		{}			&	~qːumuq(ːte), qːumuqlan(te)	&	~qːumuq\slash qːumuqlan\\
		Jewish		&	žuhut'(e) 				&	žuhut' ʁaj\\
		Aghul		&	aʁul(te) 				&	aʁul ʁaj\\
		Russian		&	ʡuˁrus, ʡuˁrusːe			&	ʡuˁrus ʁaj\\
		Noghaj		&	nuʁaj(te)				&	nuʁaj(tala) ʁaj\\
		Chechen		&	čaˁčaˁn(te),				&	čaˁčaˁn /\\
		{}		&	~mičiχičlan(te)			&	~mičiχičlan ʁaj\\
		Dagestanian	&	daʁistan(te)				&	daʁistanna ʁaj\\
		Georgian	&	gurži(be) 				&	gurži(la) ʁaj\\
		German	&	nemec, nemcːabe			&	nemcːabala ʁaj\\
		\lspbottomrule
	\end{tabularx}
\end{table}
%


		
		
\begin{exe}
	\ex	\label{ex:This is called nature reserve in Russian}
	\gll	zapowednik	b-ik'-u	ʡuˁrusː-a-la  \\
		nature.reserve	\textsc{hpl}-say.\textsc{ipfv-prs}	Russian-\textsc{obl-gen}\\
	\glt	\sqt{This is called `nature reserve' in the Russian (language).}

	\ex	\label{ex:Well, the Jew already felt better}
	\gll	nu	uže	žuhut'-li-j	ʡaˁħ-le	ag-ur   \\
		well	already	Jew-\textsc{obl-dat}	good-\textsc{advz}	go.\textsc{pfv-pret}\\
	\glt	\sqt{Well, the Jew already felt better.}

	\ex	\label{ex:Was he Kumyk No (he) was Dargwa}
	\gll	ca-w	qːumuqlan=de=w?	aʔa,	darkːʷan=de   \\
		\textsc{refl-m}	Kumyk=\textsc{pst=q} no Dargwa=\textsc{pst}\\
	\glt	\sqt{Was he Kumyk? No, (he) was Dargwa.}
\end{exe}

Some microtoponyms can be found in \reftab{tab:Microtoponyms}. The first column provides the citation form of the name and the second column the essive case form (all other \is{spatial case}spatial cases are formed accordingly). The second column shows that the essive forms are sometimes transparently built from the \tsc{loc}-series (-\textit{le}) and in one case from the \tsc{ad}-series (-\textit{šːu}) \refex{ex:They sent us to Shari it is probably one kilometer}. All terms for microtoponyms do not contain morphemes that synchronically can be identified as \isi{spatial case} suffixes \refex{ex:As I remember they were in Shike}. The third column provides explanations for those place names for which I was able to find one. Unfortunately not all place names are still remembered after more than 50 years since Sanzhi people resettled from their original village to the lowlands. 

\begin{table}
	\caption{Microtoponyms}
	\label{tab:Microtoponyms}
	\small
	\begin{tabularx}{0.98\textwidth}[]{%
		>{\raggedright\arraybackslash\hangindent=0.5em\itshape}p{60pt}
		>{\raggedright\arraybackslash\hangindent=0.5em\itshape}p{85pt}
		>{\raggedright\arraybackslash\hangindent=0.5em}X}
		
		\lsptoprule
		\upshape \isit{microtoponym}	&	\upshape essive 	&	explanation\\
		\midrule
		zejnuq' 		&	zejnuq'-le-b 		&	\\
		iqanna		&	iqanna-b		&	\\
		χːula sukri		&	χːula sukri-b		&	around 500 meters from Sanzhi to the west, an area about of the length of about one kilometer; location of some terraced fields\\
		kuzu			&	kuzu-le-b		&	\\
		šːik'e			&	šːik'e-b		&	\\
		paˁχ-paˁχ		&	paˁχ-paˁχ-le-b 	&	main spring on the other side of the river in front of the village from where the Sanzhi people used to fetch their water\\ 
		c'aˁl darkːʷi 		&	c'aˁl darkːʷi-le-b	&	an Icari farm located on the main road to Icari; Sanzhi people used to work there\\
		ħaˁpraqu 		&	ħaˁpraqu-le-b	&	\\
		irč'milla baˁʡ		&	irč'milla baˁʡ-li-šːu-b	&	\\
		kʷasːala qal(li)sa	&	kʷasːala qalsa-b 	&	\\
		sana			&	sana-b		&	sunny site of the mountain valley\\
		ʡaˁragu 		&	ʡaˁragu-b		&	\\
		qirabaj		&	qirabaj-le-b		&	\\
		χaˁnhara 		&	χaˁnhara-b		&	\\
		čibk'ila bek'		&	čibik'ila bek'-le-b	&	elevation above the village, on the northern side\\
		\lspbottomrule
	\end{tabularx}
\end{table}

\begin{exe}
	\ex	\label{ex:They sent us to Shari it is probably one kilometer}
	\gll	šaˁrʡaˁ	d-at	aˁʁ-ib-le	ca	kilametru	k'e-b	b-urkː-ar	hextːu-b		ʁʷaž-le-r	či-d-a,	muʁar-la	bek'-le-r   \\
		Shari	1/2\textsc{pl}-free	do.\textsc{pfv-pret-cvb}		one	kilometer	exist.\textsc{up-n}	\textsc{n}-find.\textsc{ipfv-prs}	there.\textsc{up-n}	Ghwazh-\textsc{loc-abl}	on-1/2\textsc{pl-dir} Mughar-\textsc{gen}	head-\textsc{loc-abl}\\
	\glt	\sqt{They sent us to Shari, it is probably one kilometer, through the hill Ghwazh, ‎‎through the top Mughar.}

	\ex	\label{ex:As I remember they were in Shike}
	\gll	han b-irk-u	ix-tːi	šːik'e-b  \\
		remember \textsc{hpl}-occur.\textsc{ipfv-prs}	\textsc{dem.up}-\textsc{pl}	Shike-\textsc{hpl}\\
	\glt	\sqt{As I remember, they were in Shike.}
\end{exe}
