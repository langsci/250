\chapter{Postpositions}
\label{cpt:postpositions}

Sanzhi has spatial and non-spatial postpositions. Some of the spatial postpositions also have temporal readings. The majority of the spatial postpositions are widely used as adverbs and then occur without a dependent noun phrase (\refsec{ssec:SpatialAdverbsDerivedFromPostpositions}). Thus, the distinction between postpositions and adverbs is rather blurred. The distinction between postpositions and spatial cases is is, by contrast, relatively clear-cut with respect to the morphosyntax, although there are no clear intonational and often also no clear semantic differences. Most postpositions govern the genitive case; otherwise two spatial cases or the absolutive case are used (\reftab{tab:Spatial postpositions}). This is in contrast with spatial cases, which are suffixed directly to the nominal stem or to the oblique/ergative suffix. Furthermore, only the postposition \textit{sa} has a clear cognate form used as spatial case (\refsec{sssec:ante-lative -sa, ante-essive -sa-b, and ante-ablative -sa-r}). The postposition \textit{sa} is shown in \refex{in front of the house}; examples with the cognate spatial case are given in \refex{In front of his house there is also a good area, he says} and \refex{he himself also slept in front of it}. 
%
\begin{exe}
	\ex	
	\begin{xlist}
			\ex	\label{in front of the house}
		\gll	qal-la	sa-b \\
			house-\textsc{gen}	in.front-\textsc{n}\\
		\glt	\sqt{in front of the house} (E)
		
		\ex	\label{In front of his house there is also a good area, he says}
		\gll	cin-na	qal-li-sa-b	musːa=ra	ʡaˁħ-ce	ca-b Ø-ik'ʷ-ar\\
			\textsc{refl}.\textsc{sg}-\textsc{gen}	house-\textsc{obl}-\textsc{ante}-\textsc{n}	place=\textsc{add} good-\textsc{attr}.\textsc{sg} \textsc{cop-n}	\textsc{m}-say.\textsc{ipfv}-\textsc{prs}\\
		\glt	\sqt{In front of his house there is also a good area, he says.}

		\ex	\label{he himself also slept in front of it}
		\gll	it	ca-w=ra	hel-i-sa	sa-ka-jsː-un-ne\\
			that	\textsc{refl}-\textsc{m}=\textsc{add}	that-\textsc{obl}-\textsc{ante}	\textsc{ante-down}-lay.\textsc{m}.\textsc{pfv}-\textsc{pret}-\textsc{cvb}\\
		\glt	\sqt{he himself also slept in front of it (a horse, in order to watch over it)}
	\end{xlist}
\end{exe}

There is another class of morphemes with which postpositions formally and semantically overlap, namely spatial preverbs. The postpositions \textit{sa}, \textit{hitːi}, \textit{či}, \textit{b}-\textit{i} and \textit{tːura} also occur as location preverbs (\refsec{ssec:Location preverbs and spatial cases expressing direction}) that can be combined with the postposition or case marker or occur on their own. In example \refex{he himself also slept in front of it} both the spatial case \textit{-sa} and the preverb \textit{sa-} are used.

This chapter treats spatial postpositions (including those with temporal meanings) (\refsec{sec:Spatialpostpositions}) and non-spatial postpositions (\refsec{sec:Non-spatial postpositions}). 

%%%%%%%%%%%%%%%%%%%%%%%%%%%%%%%%%%%%%%%%%%%%%%%%%%%%%%%%%%%%%%%%%%%%%%%%%%%%%%%%

\section{Spatial postpositions}
\label{sec:Spatialpostpositions}

\reftab{tab:Spatial postpositions} displays the spatial postpositions and the cases they govern (in the last column). Most postpositions govern the genitive case, which is typical for Dargwa varieties. Postpositions can be inflected for directional cases. The inflected postpositions in brackets can be elicited, but are not commonly used.
%
\begin{table}
	\caption{Spatial postpositions}
	\label{tab:Spatial postpositions}
	\small
	\begin{tabularx}{0.92\textwidth}[]{%
		>{\raggedright\arraybackslash}X
		>{\raggedright\arraybackslash\itshape}p{24pt}
		>{\raggedright\arraybackslash\itshape}p{30pt}
		>{\raggedright\arraybackslash\itshape}p{47pt}
		>{\raggedright\arraybackslash\itshape}p{40pt}
		>{\raggedright\arraybackslash}p{42pt}}
		
		\lsptoprule
		{}				&	\upshape lative	&	\upshape essive	&	\upshape ablative	&	\upshape directional	&	case\\ 
		\midrule
		\sqt{in front}		&	sala			&	sala-b			&	sala-r-(ka)		&	sala-b-a		&	\tsc{gen}\\
		\sqt{in front, ago}		&	sa			&	sa-b			&	sa-r-(ka)		&	\tmd			&	\tsc{gen/abs}\\
		\sqt{behind, after}		&	hila			&	hila-b			&	hila-r-(ka)		&	hila-b-a		&	\tsc{gen}\\
		\sqt{after, behind}		&	hitːi			&	hitːi-b			&	hitːi-r-(ka)		&	(hitːi-b-a)		&	\tsc{gen}\\
		`at the bottom,		& 	xːari			&	xːari-b		&	xːari-r-(ka)		&	xːari-b-a		&	\tsc{gen}\\
		~~down, under'\\
		`at the top, above,		&	qari			&	qari-b			&	qari-r-(ka)		&	qari-b-a		&	\tsc{gen}\\
		~~on, about'\\
		\sqt{on}			&	či			&	či-b			&	či-r-(ka)		&	či-b-a			&	\tsc{gen/loc}\\
		`between,  			&	urkːa			&	urkːa-b		&	urkːa-r-(ka)		&	urkːa-b-a		&	\tsc{gen/abs}\\
		~~in the middle'\\
		\sqt{in(side)}		&	b-i			&	b-i-b			&	b-i-r-(ka)		&	b-i-b-a		&	\tsc{loc/in/gen}\\
		\sqt{aside, next to}	&	šːule			&	šːule-b		&	šːule-r-(ka)		&	(šːule-b-a)		&	\tsc{gen}\\
		\sqt{outside}		&	tːura			&	tːura-b		&	tːura-r-(ka)		&	tːura-b-a		&	\tsc{gen}\\
		\lspbottomrule
	\end{tabularx}
\end{table}



% --------------------------------------------------------------------------------------------------------------------------------------------------------------------------------------------------------------------- %

\subsection{\textit{sala} \sqt{in front of}}
\label{ssec:postposition sala}
The postposition \tit{sala} has only spatial meaning, but the cognate adverb has spatial and temporal readings (e.g. \textit{salar(ka)} \sqt{formerly, in former times}). It governs the genitive.
%
\begin{exe}
	\ex
	\begin{xlist}
		\ex	\label{Put this before of this, and this also}
		\gll	hež-i-la	sala	ka-b-iž-aq-a,	hej=ra!\\
			this-\textsc{obl}-\textsc{gen}	in.front	\textsc{down-n}-be.\textsc{pfv}-\textsc{caus}-\textsc{imp}	this=\textsc{add}\\
		\glt	\sqt{Put (it) before of this, and this also!}

		\ex	\label{Is there a farm in front of the graveyard}
		\gll	χːuˁrba-la	sala-b	pirma	te-b=uw?\\
			tomb.-\textsc{obl.pl-gen}	in.front-\textsc{n}	farm	exist-\textsc{n}=\textsc{q}\\
		\glt	\sqt{Is there a farm in front of the graveyard?}
	\end{xlist}
\end{exe}


% --------------------------------------------------------------------------------------------------------------------------------------------------------------------------------------------------------------------- %

\subsection{\tit{sa} \sqt{in front, ago}}
\label{ssec:postposition sa}

The postposition \textit{sa} is a cognate of \textit{sala}. When having a spatial reading it governs the genitive \refex{the bandits caught (him) at the throat}. With the temporal meaning \sqt{ago it governs} the absolutive \refex{Grandfather’s wife died 30 years ago}. The ablative \tit{sar(ka)} is more commonly used as temporal adverb with the meaning \sqt{before, earlier, until} (\refsec{sec:TemporalAdverbs}).
%
\begin{exe}
	\ex
	\begin{xlist}
		\ex	\label{the bandits caught (him) at the throat}
		\gll	uc-ib-le	qːačuʁ-a-l	susm-a-la	sa-w\\
			catch.\textsc{pfv.m}-\textsc{pret}-\textsc{cvb}	bandit-\textsc{obl}.\textsc{pl}-\textsc{erg}	throat-\textsc{obl}-\textsc{gen}	in.front-\textsc{m}\\
		\glt	\sqt{the bandits caught (him) by the throat}

		\ex	\label{Grandfather’s wife died 30 years ago}
		\gll	χatːaj-la	xːunul	r-ebč'-ib-il=de	ʡaˁb-c'al	dus	sa-r\\
			grandfather-\textsc{gen}	woman	\textsc{f}-die.\textsc{pfv}-\textsc{pret}-\textsc{ref}=\textsc{pst}	three-\textsc{ten}	year	ago-\textsc{abl}\\
		\glt	\sqt{Grandfather's wife died 30 years ago.}
	\end{xlist}
\end{exe}


% --------------------------------------------------------------------------------------------------------------------------------------------------------------------------------------------------------------------- %

\subsection{\tit{hila} \sqt{behind, after}}
\label{ssec:postposition hila}

The postposition \tit{hila}, which governs the genitive, has spatial and occasionally temporal uses. It is sometimes followed by \textit{hitːi} and it is widely used as a spatial and temporal adverb (\refsec{ssec:SpatialAdverbsDerivedFromPostpositions}).
%
\begin{exe}
	\ex
	\begin{xlist}
		\ex	\label{I carried (the cheese) behind the house and fed it to the chicken}
		\gll	b-erqː-ib-le	qal-la	hila,	ʡuˁrʡ-aˁ-j	s-ix-ub=da  \\
			\textsc{n}-carry.\textsc{pfv}-\textsc{pret}-\textsc{cvb}	house-\textsc{gen}	behind	chicken-\textsc{obl}.\textsc{pl}-\textsc{dat}	\textsc{ante}-throw.\textsc{pfv}-\textsc{pret}=1\\
		\glt	\sqt{I carried (the cheese) behind the house and fed it to the chicken.}

		\ex	\label{There behind the hill there is immediately something like a field}
		\gll	tum-la	hila-b	srazu	majdan ʁuna	k'e-b\\
			hill-\textsc{gen}	behind-\textsc{n}	immediately	field	\textsc{eq}	exist.\textsc{up-n}\\
		\glt	\sqt{There behind the hill there is immediately something like a field.}

		\ex	\label{a rainbow after the rain}
		\gll	marka-la	hila-b	cuχaˁb  \\
			rain-\textsc{gen}	behind-\textsc{n}	rainbow\\
		\glt	\sqt{a rainbow after the rain} (E)
	\end{xlist}
\end{exe}


% --------------------------------------------------------------------------------------------------------------------------------------------------------------------------------------------------------------------- %

\subsection{\tit{hitːi} \sqt{after, behind}}
\label{ssec:postposition hiti}

This postposition has spatial and temporal semantics. There are examples that allow for both readings, e.g. \refex{After this (one) must put these, these where they steal} is a sentence from the \textit{Family Problems Picture Task} \citep{SanRoqueEtAl2012} and it can refer to the spatial ordering of the pictures on the table or to the temporal ordering of the events that the pictures are illustrating. There are two examples with spatial meaning: in both examples \textit{hitːi} is preceded by \textit{hila} and thus it might be \textit{hila} that, in fact, provides for the spatial interpretation \refex{He tied it behind the saddle}. It governs the genitive and mostly occurs with a preceding demonstrative pronoun and the meaning \sqt{after this\slash that}, e.g. \textit{hežila} \textit{hitːi}. There are also lexicalized variants of such phrases, e.g. \textit{helila} \textit{hitːi} > \textit{helilitːi}.
%
\begin{exe}
	\ex
	\begin{xlist}
		\ex	\label{Once after the rain (she) went up to sweep in front of the house}
		\gll	caj-na	marka-la	hitːi	če-r-uq-un ca-r	qar	qal-sa	qʷaˁrš	b-arq'-ij  \\
			one-\textsc{time}	rain-\textsc{gen}	after	\textsc{spr.up}-\textsc{f}-go.\textsc{pfv}-\textsc{pret} \textsc{cop-f}	up	house-\textsc{ante}	sweep	\textsc{n}-do.\textsc{pfv}-\textsc{inf}\\
		\glt	\sqt{Once after the rain (she) went up to sweep in front of the house.}

		\ex	\label{After this (one) must put these, these where they steal}
		\gll	het-i-la	hitːi	ka-d-irxː-an-te ca-d	heštːi	d-ilʡ-aˁn-te \\
			that-\textsc{obl}-\textsc{gen}	after	\textsc{down-npl}-put.\textsc{ipfv}-\textsc{ptcp}-\textsc{dd}.\textsc{pl} be-\textsc{npl}	these	\textsc{npl}-steal.\textsc{ipfv}-\textsc{ptcp}-\textsc{dd}.\textsc{pl}\\
		\glt	\sqt{After this (one) must put these, these where they steal.}

		\ex	\label{He tied it behind the saddle}
		\gll	urči-la	žilixʷa-la	hila	hitːi	b-iχ-un  \\
			horse-\textsc{gen}	saddle-\textsc{gen}	behind	behind	\textsc{n}-tie.\textsc{pfv}-\textsc{pret}\\
		\glt	\sqt{(He) tied it behind the saddle.}
	\end{xlist}
\end{exe}

In addition, \textit{hitːi} is widely used as a temporal adverb (\refsec{sec:TemporalAdverbs}), including temporal adverbial clauses (\refsec{sec:temporalcausal postposition hiti}), and the short encliticized version \textit{=itːi} occurs within compound verbs (\refsec{ssec:compoundswithnouns}, example \refex{ex:markednouncpafter}).


% --------------------------------------------------------------------------------------------------------------------------------------------------------------------------------------------------------------------- %

\subsection{\textit{xːar(i)} \sqt{down, at the bottom, under}}
\label{ssec:postposition xari}

This postposition has exclusively spatial meaning \sqt{to the bottom, down, under} and governs the genitive \refex{I put the cup under the table / to the bottom of the table}-\refex{I went to the hospital down at the sea}. It is semantically close to the spatial case \textit{-gu} (\refsec{sssec:sub-lative -gu, sub-essive -gu-b, and sub-ablative -gu-r}) and the spatial case marker can be suffixed to the postposition in which case the meaning is solely \sqt{under}.
%
\begin{exe}
	\ex
	\begin{xlist}
		\ex	\label{I put the cup under the table / to the bottom of the table}
		\gll	ust'u-la	xːari	pihala	ka-b-irx-ul=da  \\
			table-\textsc{gen}	down	cup	\textsc{down-n}-put.\textsc{ipfv}-\textsc{icvb}=1\\
		\glt	\sqt{I put the cup under the table\slash to the bottom of the table.} (E)

		\ex	\label{From the waist down I was wet}
		\gll	 w-ag-la	xːar	w-aˁħ-un-ni=de \\
			\textsc{m}-waist-\textsc{gen}	down	\textsc{m}-get.wet.\textsc{pfv}-\textsc{pret}-\textsc{msd}=\textsc{pst}\\
		\glt	\sqt{From the waist down I (masc.) was wet.}

		\ex	\label{I went to the hospital down at the sea}
		\gll	heχ	urx-m-a-la	xːari	balnicːa-b-a-j	r-ax-ul\\
			\textsc{dem.down}	sea-\textsc{pl}-\textsc{obl}-\textsc{gen}	down	hospital-\textsc{pl}-\textsc{obl}-\textsc{dat}	\textsc{f}-go.\textsc{ipfv}-\textsc{icvb}\\
		\glt	\sqt{I was going to the hospital down at the sea ...}
	\end{xlist}
\end{exe}


% --------------------------------------------------------------------------------------------------------------------------------------------------------------------------------------------------------------------- %

\subsection{\tit{qari} \sqt{at the top, above, on, about}}
\label{ssec:postposition qari}

This postposition, which governs the genitive, means \sqt{at/on the top, above} and is the counterpart to \textit{xːar(i)} \refex{In those times upwards from our village there was another village}-\refex{The lamp hangs above the table.}. 
% 
\begin{exe}
	\ex
	\begin{xlist}
		\ex	\label{In those times upwards from our village there was another village}
		\gll	hel	zamana	šːi-la	qari-b	cara	šːi	k'e-b\\
			that	time	village-\textsc{gen}	at.top-\textsc{n}	other	village	exist.\textsc{up-n}\\
		\glt	\sqt{In those times upwards from our village there was another village.}

		\ex	\label{At that time the puppy began to jump to climb up the tree's top}
		\gll	il	zamana	kac'i	či-ka-b-iħ-ib ca-b	taˁħ	b-ax-araj	kːalkːi-la	qari \\
			that	time	puppy	\textsc{spr-down}-\textsc{n}-begin.\textsc{pfv}-\textsc{pret} \textsc{cop-n}	jump	\textsc{n}-go.\textsc{ipfv}-\textsc{subj.3}	tree-\textsc{gen}	at.top\\
		\glt	\sqt{At that time the puppy began to jump to climb up the tree's top.}


		\ex	\label{The lamp hangs above the table.}
		\gll	lampːučkːa	ust'u-la	qari-b	kemq-un	ca-b  \\
			lamp	table-\textsc{gen}	above-\textsc{n}	hang-\textsc{pret}	\textsc{cop-n}\\
		\glt	\sqt{The lamp hangs above the table.} (E)
	\end{xlist}
\end{exe}

In combination with the postposition \textit{či} (\refsec{ssec:postposition ci}) it is also used to express the topic of a conversation or the contents of thoughts \refex{She is complaining about him}, \refex{to talk about what had happened}.
% 
\begin{exe}
	\ex
	\begin{xlist}
		\ex	\label{She is complaining about him}
		\gll	ʡaˁrz	r-ik'-ul ca-r	iχ-i-la	qari=či-r  \\
			complain	\textsc{f}-say.\textsc{ipfv}-\textsc{icvb} \textsc{cop-f}	\textsc{dem.down}-\textsc{obl}-\textsc{gen}	at.top=on-\textsc{f}\\
		\glt	\sqt{She is complaining about him.}

		\ex	\label{to talk about what had happened}
		\gll	cinna	d-iχ-ub-t-a-la	qari=či-d	b-urs-ij  \\
			pause.filler	\textsc{npl}-be.\textsc{pfv}-\textsc{pret}-\textsc{pl}-\textsc{obl}-\textsc{gen}	at.top=on-\textsc{npl}	\textsc{n}-tell.\textsc{pfv}-\textsc{inf}\\
		\glt	\sqt{to talk about what had happened}
	\end{xlist}
\end{exe}


% --------------------------------------------------------------------------------------------------------------------------------------------------------------------------------------------------------------------- %

\subsection{\tit{či} \sqt{on, above}}
\label{ssec:postposition ci}

This postposition, which is often pronounced together with the complement nominal as an enclitic, normally governs a spatial case, the \textsc{loc}-series \refex{They are making food on a gas cooker}, \refex{He fell down from the horse into a deep pit} (\refsec{sssec:spr-lative -le/-ja/-a, spr-essive -le-b/-ja-b/-a-b and spr-ablative -le-r/-ja-r/-a-r}), but it can, in principle, also be used with the dependent noun bearing the genitive \refex{There is fog on / above the mountain}. When the \textsc{loc}-series is used, then the direction markers of the case and of the postposition need to coincide, i.e. both are marked for the essive \refex{They are making food on a gas cooker}, lative, or ablative \refex{He fell down from the horse into a deep pit}. The postposition can be encliticized to \textit{qari} when referring to the content of conversations or thoughts \refex{She is complaining about him}, \refex{to talk about what had happened}. 
%
\begin{exe}
	\ex
	\begin{xlist}
		\ex	\label{They are making food on a gas cooker}
		\gll	berkʷijce	b-irq'-ul	ca-b	iχ-tː-a-l	gaz-le-b	či-b  \\
			food	\textsc{n}-do.\textsc{ipfv}-\textsc{icvb} \textsc{cop}-\textsc{n}	\textsc{dem.down}-\textsc{pl}-\textsc{obl}-\textsc{erg}		gas-\textsc{loc}-\textsc{n}	on-\textsc{n}\\
		\glt	\sqt{They are making food on a gas cooker.}

		\ex	\label{He fell down from the horse into a deep pit}
		\gll	urči-le-r	či-r	ka-jč-ib-le,	w-i-ka-ag-ur	ca-w	kur	kur-ri-cːe   \\
			horse-\textsc{loc}-\textsc{abl}	on-\textsc{abl}	\tsc{down}-occur.\textsc{m}.\textsc{pfv}-\textsc{pret}-\textsc{cvb}	\textsc{m-in-down}-go.\textsc{pfv}-\textsc{pret}	\textsc{cop-m}	deep	pit-\textsc{obl}-\textsc{in}\\
		\glt	\sqt{He fell down from the horse into a deep pit.}

		\ex	\label{There is fog on / above the mountain}
		\gll	dubur-ra	či-b	dirixʷ	k'e-b\\
			mountain-\textsc{gen}	on-\textsc{n}	fog	exist.\textsc{up-n}\\
		\glt	\sqt{There is fog on\slash above the mountain.} (E)
	\end{xlist}
\end{exe}

Since Sanzhi also has a preverb \tit{či-} with a very similar if not identical meaning (\refsec{ssec:Location preverbs and spatial cases expressing direction}) it is sometimes not easy to decide whether an occurrence of \tit{či} functions as postposition\slash adverbial or as preverb. Thus, instead of \refex{He fell down from the horse into a deep pit} with \tit{či-r} as postposition, we can also write it together with the verb and interpret it as preverb \refex{ex:He fell from the horse verbs_4}.  But we can also manipulate the constituent order in \refex{He fell down from the horse into a deep pit} and place the verb before the postpositional phrase \refex{ex:He fell from the horse verbs_2} or have both the postposition and the preverb \refex{ex:He fell from the horse verbs_3}. In the last two examples, \tit{či-r} is unambiguously a postposition.

\begin{exe}
		\ex	\label{ex:He fell from the horse verbs_4}
	\gll	urči-le-r či-r-ka-jč-ib\\
		horse\tsc{-loc-abl} \tsc{spr-abl-down}-occur\tsc{.pfv.m-pret}\\
	\glt	\sqt{He fell from the horse.} (E)
	
	\ex	\label{ex:He fell from the horse verbs_2}
	\gll	ka-jč-ib urči-le-r	či-r \\
		\tsc{down}-occur\tsc{.pfv.m-pret} horse\tsc{-loc-abl} on-\tsc{abl}	\\
	\glt	\sqt{He fell from the horse.} (E)
	
		\ex	\label{ex:He fell from the horse verbs_3}
	\gll	urči-le-r či-r	či-r-ka-jč-ib\\
		horse\tsc{-loc-abl} on-\tsc{abl} \tsc{spr-abl-down}-occur\tsc{.pfv.m-pret}\\
	\glt	\sqt{He fell from the horse.} (E)
	
\end{exe}

% --------------------------------------------------------------------------------------------------------------------------------------------------------------------------------------------------------------------- %

\subsection{\tit{urkːa} \sqt{between, among, within, in the middle}}
\label{ssec:postposition urka}

The postposition \textit{urkːa} has spatial and temporal meanings. For the spatial reading only genitive marking on the dependent noun is admissible. When used with nouns and noun phrases denoting a plurality it means \sqt{between, among} \refex{(The sun) itself came out of the middle of the clouds}-\refex{I am between the houses}. 

%
\begin{exe}
	\ex
	\begin{xlist}
		\ex	\label{(The sun) itself came out of the middle of the clouds}
		\gll	ca-b	qːirma-la	urkːa-rka	tːura	sa-b-uq-un  \\
			\textsc{refl}-\textsc{n}	black.clouds-\textsc{gen}	between-\textsc{abl}	outside	\textsc{hither-n}-go-\textsc{pret}\\
		\glt	\sqt{(The sun) itself came out of the middle of the clouds.}

		\ex	\label{A stone kept between two rocks or the like is there up}
		\gll	k'ʷel=ra	qič'-m-a-la	urkːa	hek'	qːarqːa	b-uc-ib-il ʁuna musːa	k'e-b=q'al   \\
			two=\textsc{add}	rock-\textsc{pl}-\textsc{obl}-\textsc{gen}	between	\textsc{dem.up}	stone	\textsc{n}-keep.\textsc{pfv}-\textsc{pret}-\textsc{ref}	\textsc{eq}	place	exist.\textsc{up-n}=\textsc{mod}\\
		\glt	\sqt{A stone kept between two rocks or the like is up there.}

		\ex	\label{Now the women are whispering among themselves}
		\gll	na	xːun-re	ču-la	urkːa-b	qit.qit	b-ik'-ul	ca-b  \\
			now	woman-\textsc{pl}	\textsc{refl}.\textsc{pl}-\textsc{gen}	between-\textsc{hpl}	whisper	\textsc{hpl}-say.\textsc{ipfv}-\textsc{icvb}	\textsc{cop-hpl}\\
		\glt	\sqt{Now the women are whispering among themselves.}
	\end{xlist}
\end{exe}

But it can also occur with singular nouns and the meaning \sqt{in}. The following minimal pair illustrates the difference:

\begin{exe}
	\ex
	\begin{xlist}
		\ex	\label{I am between the houses}
		\gll	qul-b-a-la urkːa-w=da  \\
			house-\textsc{pl}-\textsc{obl}-\textsc{gen}	between-\textsc{m}=1\\
		\glt	\sqt{I am between the houses.} (E)

		\ex	\label{I am in the house.}
		\gll	qal-la urkːa-w=da  \\
			house-\textsc{gen}	between-\textsc{m}=1\\
		\glt	\sqt{I am in the house.} (E)
	\end{xlist}
\end{exe}

With the temporal reading the postposition governs the genitive \refex{withinoneweek}, or the absolutive \refex{After two months they were doing (the medical treatment again}. When reduplicated the postposition can be used as an adverb with the meaning \textit{urkːa} \textit{urkːa-b} \sqt{from time to time, sometimes}.


\begin{exe}
	\ex
	\begin{xlist}
		\ex	\label{withinoneweek}
		\gll	žumaˁʡ-la	urkːa-r  \\
			week-\textsc{gen}	between-\textsc{abl}\\
		\glt	\sqt{within one week}

		\ex	\label{After two months they were doing (the medical treatment again}
		\gll	k'ʷel	bac	urkːa-r	d-irq'-ul ...\\
			two	month	between-\textsc{abl}	\textsc{npl}-do.\textsc{ipfv}-\textsc{icvb}\\
		\glt	\sqt{After two months they were doing (the medical treatment) ...}
	\end{xlist}
\end{exe}


% --------------------------------------------------------------------------------------------------------------------------------------------------------------------------------------------------------------------- %

\subsection{\tit{b-i} \sqt{in, inside}}
\label{ssec:postposition bi}

The postposition \tit{b-i}, which only has spatial meanings, contains a gender/number prefix agreeing with the absolutive argument of the clause to which the postpositional phrase belongs. In all examples from natural texts the postposition governs the \textsc{in}-series or the \textsc{loc}-series (depending on the noun employed, see \refsec{sssec:in-lative -cːe, in-essive -cːe-b, and in-ablative -cːe-r} and \refsec{sssec:spr-lative -le/-ja/-a, spr-essive -le-b/-ja-b/-a-b and spr-ablative -le-r/-ja-r/-a-r}) \refex{(It) threw the boy and the dog into the water}, \refex{Iam inside your box}. However, in elicitation the genitive is also available \refex{Iaminside the house}.
%
\begin{exe}
	\ex
	\begin{xlist}
		\ex	\label{(It) threw the boy and the dog into the water}
		\gll	lak'	w-arq'-ib	ca-w	duˁrħuˁ=ra	χːʷe=ra	hin-ni-cːe b-i  \\
			throw	\textsc{m}-do.\textsc{pfv}-\textsc{pret}	\textsc{refl}-\textsc{m}	boy=\textsc{add}	dog=\textsc{add} water-\textsc{obl}-\textsc{in} \textsc{hpl}-in\\
		\glt	\sqt{(It) threw the boy and the dog into the water.}

		\ex	\label{Iam inside your box}
		\gll	du	ala	sunduq'-le-w	w-i-w	le-w=da  \\
			1\textsc{sg}	2\textsc{sg}.\textsc{gen}	box-\textsc{loc}-\textsc{m}	\textsc{m}-in-\textsc{m}	exist-\textsc{m}=1\\
		\glt	\sqt{I (masc.) am inside your box.}

		\ex	\label{Iaminside the house}
		\gll	qal-la r-i-r=da  \\
			house-\textsc{gen} \textsc{f}-in-\textsc{f}=1\\
		\glt	\sqt{I (fem.) am inside the house.} (E)
	\end{xlist}
\end{exe}


% --------------------------------------------------------------------------------------------------------------------------------------------------------------------------------------------------------------------- %

\subsection{\textit{šːule} \sqt{at side, next to, near}}
\label{ssec:postposition sule}

This postposition exclusively expresses spatial meanings. It requires the dependent noun to appear in the genitive case. Examples \refex{Next to the village under a stone wall there is our graveyard}-\refex{The grave was near the top.} show the postposition inflected for the essive case. In \refex{Then these boys passed by from his, the father's, side}, in bears the ablative case suffix.
%
\begin{exe}
	\ex
	\begin{xlist}
		\ex	\label{Next to the village under a stone wall there is our graveyard}
		\gll	šːi-la	šːule-d	qič'a-la	baˁʡ-li-gu-d	k'e-d	nišːa-la	χːuˁrbe  \\
			village-\textsc{gen}	at.side-\textsc{npl}	rock-\textsc{gen}	wall-\textsc{obl}-\textsc{sub}-\textsc{npl}	exist.\textsc{up-npl}	1\textsc{pl}-\textsc{gen}	tomb.\textsc{pl}\\
		\glt	\sqt{Next\slash near to the village down at a stone wall, there is our graveyard.}

						\ex	\label{The ball was at my feet.}
		\gll	tup di-la t'uˁ-ma-lla šːule-b=de \\
ball	\textsc{1sg-gen}	leg-\textsc{pl-obl-gen}	at.side-\textsc{n=pst} 	\\
		\glt	\sqt{The ball was at my feet.}	(E)		
		
			\ex	\label{The grave was near the top.}
		\gll	χːaˁb muza-la šːule-b b-už-ib ca-b \\
grave	top-\textsc{gen}	at.side-\textsc{n}	\textsc{n}-be-\textsc{pret}	\textsc{cop-n}\\
		\glt	\sqt{The grave was near the top.}	(E)
		
		\ex	\label{Then these boys passed by from his, the father's, side}
		\gll	c'il	hetːi	duˁrħ-ne	ag-ur	hel-i-la	atːa-la šːule-r\\
			then	those	boy-\textsc{pl}	go.\textsc{pfv}-\textsc{pret}	that-\textsc{obl}-\textsc{gen}	father-\textsc{gen} at.side-\textsc{abl}\\
		\glt	\sqt{Then these boys passed by from his, the father's, side.}
		
	\end{xlist}
\end{exe}

This postposition probably originates from the noun \textit{šːal} \sqt{side}. Though it looks like it could be the \textsc{loc}-case of this noun, this is synchronically not the case, since the \textsc{loc}-lative of the noun is \textit{šːal}-\textit{le} and not \textit{šːu(l)le} \refex{And I am at the side of Isakadi, at this end}. Nevertheless the origin from a spatial noun explains why the postposition governs only the genitive.
%
\begin{exe}
	\ex	\label{And I am at the side of Isakadi, at this end}
	\gll	a	du	rjadom	Isaq'adi-la	šːal-li-cːe-w	hej	b-aʔ-le-w=da \\
		but	1\textsc{sg}	next.to	Isakadi-\textsc{gen}	side-\textsc{obl}-\textsc{in}-\textsc{m}	this	\textsc{n}-edge-\textsc{loc}-\textsc{m}=1\\
	\glt	\sqt{And I am at the side of Isakadi, at this end.}
\end{exe}


% --------------------------------------------------------------------------------------------------------------------------------------------------------------------------------------------------------------------- %

\subsection{\tit{tːura} \sqt{out, outside}}
\label{ssec:postposition tura}

This postposition governs the genitive \refex{Hungry and wild, we went out of the village}-\refex{also all people outside of the villag}. However, it more frequently occurs as an adverb and as a spatial preverb with preceding nouns in the \textsc{in}-ablative or \textsc{loc}-ablative (\refsec{ssec:SpatialAdverbsDerivedFromPostpositions}, \refsec{ssec:Location preverbs and spatial cases expressing direction}).
%
\begin{exe}
	\ex
	\begin{xlist}
		\ex	\label{Hungry and wild, we went out of the village}
		\gll	kːuš-le	duˁʡ-le=de	nušːa	šːi-la	tːura	ag-ur=da  \\
			hungry-\textsc{advz}	wild-\textsc{advz}=\textsc{pst}	\textsc{1pl}	village-\textsc{gen}	outside	go.\textsc{pfv}-\textsc{pret}=1\\
		\glt	\sqt{Hungry and wild, we went out of the village.}

		\ex	\label{From outside the gates an evil scream was made}
		\gll	qːapu-la	tːura-r	wahi	ʁaˁʁ-la	t'ama	ha-d-eʁ-ib ca-d\\
			gate-\textsc{gen}	outside-\textsc{abl}	evil	scream-\textsc{gen}	sound	\textsc{up}-\textsc{npl}-do-\textsc{pret} \textsc{cop-npl}\\
		\glt	\sqt{From outside the gates an evil scream was made.}

		\ex	\label{also all people outside of the villag}
		\gll	heχtːu	šːi-la	tːura-b-te	χalq'	li<b>il=ra\\
			there.\textsc{down}	village-\textsc{gen}	outside-\textsc{hpl}-\textsc{dd.pl} 	people	all<\textsc{hpl}>=\textsc{add}\\
		\glt	\sqt{also all people outside of the village}
	\end{xlist}
\end{exe}

The postposition \textit{tːura} also expresses the non-spatial meaning \sqt{apart from, except for}. In this case the governed nominal can be not only in the genitive \refex{If I can tell you now (something else) apart from this}, but also in the \textsc{loc}-ablative \refex{Moreover, he defamed (him) and wanted (him) to be beheaded}, \refex{Apart from Rasul I also brought another one}. For instance, if a demonstrative pronoun precedes the postposition the whole phrase reads as \sqt{besides, and what is more, moreover}.
%
\begin{exe}
	\ex
	\begin{xlist}
		\ex	\label{If I can tell you now (something else) apart from this}
		\gll	il-i-la	tːura-b	a-cːe	du-l	b-urs-ille  ...\\
			that-\textsc{obl}-\textsc{gen}	outside-\textsc{n}	2\textsc{sg}-\textsc{in}	1\textsc{sg}-\textsc{erg}	\textsc{n}-tell.\textsc{pfv}-\textsc{cond}.1.\textsc{prs}\\
		\glt	\sqt{If I can tell you now (something else) apart from this, …} (E)

		\ex	\label{Moreover, he defamed (him) and wanted (him) to be beheaded}
		\gll	ile-rka tːura		ʁaj	či-Ø-ik'-ul,	b-ikː-ul ca-b	il	qaˁb-la	w-aˁq-ij\\
			that.\textsc{loc}-\textsc{abl}	outside	word	\textsc{spr}-\textsc{m}-say.\textsc{ipfv}-\textsc{icvb}	\textsc{n}-want.\textsc{ipfv}-\textsc{icvb} \textsc{cop-n}	that	neck-\textsc{gen}	\textsc{m}-hit.\textsc{pfv}-\textsc{inf}\\
		\glt	\sqt{Moreover, he defamed (him) and wanted (him) to be beheaded.}

		\ex	\label{Apart from Rasul I also brought another one}
		\gll	Rasul-la	/	Rasul-le-rka	tːura	cara=ra	sa-č-ib=da  \\
			Rasul-\textsc{gen}	/ Rasul-\textsc{loc}-\textsc{abl}	outside	other=\textsc{add}	\textsc{hither}-lead.\textsc{pfv}-\textsc{pret}=1\\
		\glt	\sqt{ Apart from Rasul I also brought another one.} (E)
	\end{xlist}
\end{exe}


%%%%%%%%%%%%%%%%%%%%%%%%%%%%%%%%%%%%%%%%%%%%%%%%%%%%%%%%%%%%%%%%%%%%%%%%%%%%%%%%

\section{Non-spatial postpositions}
\label{sec:Non-spatial postpositions}


% --------------------------------------------------------------------------------------------------------------------------------------------------------------------------------------------------------------------- %

\subsection{\tit{b-alli} \sqt{together, with}}
\label{ssec:postposition balli}

This postposition, which seems to be a cognate of the preverb \textit{b-al} `matching, together, in unison', governs the comitative \refex{He is sitting together with them}, and with young speakers the \tsc{in}-ablative case \refex{He went to sleep together with the dog}. It has a gender/number prefix and agreement is controlled by the absolutive argument of the clause to which the postposition belongs. Both nominals, i.e. the governed one marked for comitative or \tsc{in}-ablative and the noun in the absolutive, can be absent.
%
\begin{exe}
	\ex
	\begin{xlist}
		\ex	\label{He is sitting together with them}
		\gll	či-haˁ-Ø-q'-uˁn-ne=kːu=n	ka-Ø-jž-ib	ca-w	hel-tː-a-cːella	w-alli  \\
			\textsc{spr-up}-\textsc{m}-go-\textsc{pret}-\textsc{cvb}=\textsc{cop.neg}=but	\textsc{down-m}-remain-\textsc{pret}	\textsc{cop-m}	that-\textsc{pl}-\textsc{obl}-\textsc{comit}	\textsc{m}-together\\
		\glt	\sqt{He is sitting together with them.}

		\ex	\label{He went to sleep together with the dog}
		\gll	w-arcː-ur-le,	ag-ur	ca-w	ka-Ø-jsː-ij	χːʷe-cːe-r	w-alli  \\
			\textsc{m}-get.tired.\textsc{pfv}-\textsc{pret}-\textsc{cvb}	go.\textsc{pfv}-\textsc{pret}	\textsc{cop}-\textsc{m} \textsc{down-m}-sleep.\textsc{pfv}-\textsc{inf}	dog-\textsc{in}-\textsc{abl}	\textsc{m}-together\\
		\glt	\sqt{He got tired and went to sleep together with the dog.}
	\end{xlist}
\end{exe}

If the governed noun phrase is overt it mostly precedes the postposition \refex{He is sitting together with them}, \refex{He went to sleep together with the dog}, though it can also follow it \refex{Then the ran together with them} or occur in another non-adjacent position \refex{Myfriend was together with me}. If the governed noun is absent the noun in the absolutive frequently takes its position right before the postposition \refex{If I had known that I will look at picture}.
%
\begin{exe}
	\ex
	\begin{xlist}
		\ex	\label{Then the ran together with them}
		\gll	c'il	ka-b-ič-ib	ca-b	b-alli	hel-tː-a-cːe-r   \\
			then	\textsc{down-hpl}-occur.\textsc{pfv}-\textsc{pret}	\textsc{cop-hpl}	\textsc{hpl}-together	that-\textsc{pl}-\textsc{obl}-\textsc{in}-\textsc{abl}\\
		\glt	\sqt{Then they ran together with them.}

		\ex	\label{Myfriend was together with me}
		\gll	w-alli,	di-la	hej	juldaš	w-alli	le-w=de	di-cːella\\
			\textsc{m}-together	1\textsc{sg}-\textsc{gen}	this	friend	\textsc{m}-together	exist-\textsc{m}=\textsc{pst}	1\textsc{sg}-\textsc{comit}\\
		\glt	\sqt{Together, my friend was together with me.}

		\ex	\label{If I had known that I will look at picture}
		\gll	ulbasne	d-alli	ha-d-iqː-a-di=q'al\\
			glasses	\textsc{npl}-together	\textsc{up}-\textsc{npl}-carry.\textsc{ipfv}-\textsc{hab}-1=\textsc{mod}\\
		\glt	\sqt{[If I had known that I will look at pictures], I would have brought my glasses.}
	\end{xlist}
\end{exe}

Because of the general closeness of adverbs and postpositions, examples such as \refex{Then the ran together with them}, \refex{Myfriend was together with me}, in which \textit{b-alli} and the case-marked noun occur in the reverse order and/or not immediately following each other can be treated as adverbial uses. Similarly, in \refex{If I had known that I will look at picture} a full postpositional phrase would be `glasses with me', but the governed nominal is absent from the clause and thus the example rather represents the adverbial use.
% --------------------------------------------------------------------------------------------------------------------------------------------------------------------------------------------------------------------- %

\subsection{\textit{canille} \sqt{together, with}}
\label{ssec:postposition canille}

This postposition, which probably originates from the numeral \textit{ca} `one', can also govern the comitative case \refex{ex:There is a house together with the pakh-pakh}. However, more frequently it is used as an adverb with the meaning \sqt{together} \refex{ex:Men and women are sitting together and drinking}.
%
\begin{exe}
	\ex	\label{ex:There is a house together with the pakh-pakh}
	\gll	qal	k'e-b=q'al	paˁχ.paˁχ-li-cːella	canille  \\
		house	exist.\textsc{up}-\textsc{n}=\textsc{mod}	pakh.pakh-\textsc{obl}-\textsc{comit}	together\\
	\glt	\sqt{There is a house together with the pakh-pakh.} (i.e. next to the pakh-pakh, which is a place in Sanzhi)

	\ex	\label{ex:Men and women are sitting together and drinking}
	\gll	xːun-re=ra	murgl-e=ra	canille	b-učː-ul	ka-b-iž-ib	ca-b  \\
		woman-\textsc{pl}=\textsc{add}	man-\textsc{pl}=\textsc{add}	together	\textsc{hpl}-drink.\textsc{ipfv}-\textsc{icvb}	\textsc{down-hpl}-be.\textsc{pfv}-\textsc{pret}	\textsc{cop-hpl}\\
	\glt	\sqt{Men and women are sitting together and drinking.}
\end{exe}


% --------------------------------------------------------------------------------------------------------------------------------------------------------------------------------------------------------------------- %

\subsection{\tit{bahanne\slash bahandan} \sqt{because of}}
\label{ssec:postposition bahanne}

This postposition originates from the noun \textit{bahana} \sqt{reason}. It governs the absolutive.
%
\begin{exe}
	\ex
	\begin{xlist}
		\ex	\label{At home the sisters were apparently arguing because of the nut}
		\gll	qili-b	ruc-be	b-iħ-ib-le	b-už-ib ca-b	hel	qix	bahanne  \\
			home-\textsc{hpl}	sister-\textsc{pl}	\textsc{hpl}-wrestle.\textsc{ipfv}-\textsc{pret}-\textsc{cvb}	\textsc{hpl}-stay-\textsc{pret} \textsc{cop-hpl}	that	nut	because.of\\
		\glt	\sqt{At home the sisters were apparently arguing because of the nut.}

		\ex	\label{Because of you I sat in prison once}
		\gll	ušːa	bahanne	caj-na	ka-Ø-jž-ib=da  \\
			2\textsc{pl}	because.of	one-\tsc{time}	\textsc{down-m}-remain-\textsc{pret}=1\\
		\glt	\sqt{Because of you I sat in prison once.}

		\ex	\label{For God’s sake, let me, he says}
		\gll	``Allah	bahandan	w-at-abaj!''	Ø-ik'-ul	ca-w  \\
			Allah	because.of	\textsc{m}-let.\textsc{pfv}-\textsc{opt}.3	\textsc{m}-say.\textsc{ipfv}-\textsc{icvb}	\textsc{cop-m}\\
		\glt	\sqt{``For God's sake, let me!'' he says.}
	\end{xlist}
\end{exe}


% --------------------------------------------------------------------------------------------------------------------------------------------------------------------------------------------------------------------- %

\subsection{\textit{akːʷar} \sqt{without, except, apart}}
\label{ssec:postposition akwar}

The present tense form of the negative copula with the meaning \sqt{not being} is used in constructions that have a meaning similar to adpositions like \sqt{except, without}. To it the cross-categorical suffixes -\textit{ce} or -\textit{il} can be added without changing the meaning \refex{For children, for adults we make stockings, those with a drawing, those without a drawing}-\refex{‎‎He said, there where is no water a mill cannot be}. The governed nominal is in the absolutive because of the verbal origin of \textit{akːʷar} as a copula that governs the absolutive case.
%
\begin{exe}
	\ex
	\begin{xlist}
		\ex	\label{Yes, except Alibatir, in my mind, I do not remember}
		\gll	e,	hel	ʡaˁlibatir	akːʷ-ar,	di-la	pikri	ħisab-le,	han	w-akːu	dam   \\
			yes	that	Alibatir	\textsc{cop.neg}-\textsc{prs}	1\textsc{sg}-\textsc{gen}	thought	account-\textsc{advz}	remember	\textsc{m}-\textsc{cop.neg}	1\textsc{sg.}\textsc{dat}\\
		\glt	\sqt{Yes, except Alibatir, in my mind, I do not remember.}

		\ex	\label{For children, for adults we make stockings, those with a drawing, those without a drawing}
		\gll	nik'a-t-a-la	χːula-t-a-la	dind-be	d-irq'-id	naˁq'iš-la	naˁq'iš	akːʷ-ar-te \\
			small-\textsc{pl}-\textsc{obl}-\textsc{gen}	big-\textsc{pl}-\textsc{obl}-\textsc{gen}	stocking-\textsc{pl}	\textsc{npl}-do.\textsc{ipfv}-1	drawing-\textsc{gen} drawing	\textsc{cop.neg}-\textsc{prs}-\textsc{dd.pl }\\
		\glt	\sqt{For children, for adults we make stockings, those with a drawing, those without a drawing.}

		\ex	\label{(He) lacks patience}
		\gll	barkat	akːʷ-ar-ce	ca-w  \\
			patience	\textsc{cop.neg}-\textsc{prs}-\textsc{attr}.\textsc{sg}	\textsc{cop-m}\\
		\glt	\sqt{(He) lacks patience.}

		\ex	\label{‎‎He said, there where is no water a mill cannot be}
		\gll	c'il	Ø-ik'-ul ca-w		``hin	akːʷ-ar-il-le-b	urχːab	a-b-irχʷ-ni=q'al''  \\
			then	\textsc{m}-say.\textsc{ipfv}-\textsc{icvb} \textsc{cop-m} water \textsc{cop.neg}-\textsc{prs}-\textsc{ref}-\textsc{loc}-\textsc{n}	mill	\textsc{neg}-\textsc{n}-be.able.\textsc{ipfv}-\textsc{msd}=\textsc{mod}\\
		\glt	\sqt{‎‎He said, ``Where there is no water a mill cannot be.''}
	\end{xlist}
\end{exe}


% --------------------------------------------------------------------------------------------------------------------------------------------------------------------------------------------------------------------- %

\subsection{\textit{q'atːin(na)} \sqt{for the sake of, because of}}
\label{ssec:postposition qatinna}

This postposition governs the absolutive. There are no examples of this postposition in my corpus, but \refex{Because of you I bought a suit}-\refex{For the sake of Hazhimurad, Muslimat came here} show three elicited sentences.

\begin{exe}
	\ex
	\begin{xlist}
		\ex	\label{Because of you I bought a suit}
		\gll	u	q'atːin	kast'um	isː-ul=da\\
			2\textsc{sg}	for.sb's.sake	suit	buy.\textsc{ipfv}-\textsc{icvb}=1\\
		\glt	\sqt{Because of you I bought a suit.} (E)

		\ex	\label{For my sake do not go}
		\gll	du	q'atːin	ma-w-ax-utːa!\\
			1\textsc{sg}	for.sb's.sake	\textsc{proh}-\textsc{m}-go.\textsc{ipfv}-\textsc{proh}.\textsc{sg}\\
		\glt	\sqt{For my sake do not go.} (E)

		\ex	\label{For the sake of Hazhimurad, Muslimat came here}
		\gll	ħaˁžimurad	q'atːin	Muslimat	heštːu	sa-r-eʁ-ib\\
			Hazhimurad	for.sb's.sake	Muslimat	here	\textsc{hither}-\textsc{f}-go.\textsc{pfv}-\textsc{pret}\\
		\glt	\sqt{For the sake of Hazhimurad, Muslimat came here.} (E)
	\end{xlist}
\end{exe}


% --------------------------------------------------------------------------------------------------------------------------------------------------------------------------------------------------------------------- %

\subsection{\tit{ħaˁsible} \sqt{according to}}
\label{ssec:postposition hasible}

This postposition is almost exclusively used in the phrase \textit{dila pikri} \textit{ħaˁsible} \sqt{in my mind} \refex{Yes, except Alibatir, in my mind, I do not remember}, \refex{In my mind there was their 3-litre can with wine}, but it can also be used with other nouns that always occur in the absolutive \refex{And these friends, what they are saying, only by means of the picture, (one) cannot know, can}. It was originally borrowed from Arabic \textit{ħaːsib} `counting', which has a similar meaning.
%
\begin{exe}
	\ex
	\begin{xlist}
		\ex	\label{In my mind there was their 3-litre can with wine}
		\gll	di-la	pikri	ħaˁsible	hel-tː-a-la	ʡaˁbal	litru-la	balun	čaˁʁir-la le-b=de\\
			1\textsc{sg}-\textsc{gen}	thought	following	that-\textsc{pl}-\textsc{obl}-\textsc{gen}	three	liter-\textsc{gen} can	wine-\textsc{gen}	exist-\textsc{n}=\textsc{pst}\\
		\glt	\sqt{‎‎In my mind there was their 3-liter can with wine.}

		\ex	\label{And these friends, what they are saying, only by means of the picture, (one) cannot know, can}
		\gll	a	iš-tːi	juldašː-e,	ce	b-ik'-ul=el,	tolko	hel	sːurrat	ħaˁsible	b-aχ-ij	a-w-irχʷ-ar,	w-irχʷ-an-ne=w?\\
			but	this-\textsc{pl}	friend-\textsc{pl}	what	\textsc{n}-say.\textsc{ipfv}-\textsc{icvb}=\textsc{indq}	only	that	picture	following	\textsc{n}-know.\textsc{pfv}-\textsc{inf}	\textsc{neg}-\textsc{m}-be.able.\textsc{ipfv}-\textsc{prs}	\textsc{m}-be.able.\textsc{ipfv}-\textsc{ptcp-3=q}\\
		\glt	\sqt{And these friends, what they are saying, only by means of the picture, (one) cannot know, can one?}
	\end{xlist}
\end{exe}

