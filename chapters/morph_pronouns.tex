\chapter{Pronouns}\label{cpt:pronouns}
\addtocontents{toc}{\protect\enlargethispage{\baselineskip}}


%%%%%%%%%%%%%%%%%%%%%%%%%%%%%%%%%%%%%%%%%%%%%%%%%%%%%%%%%%%%%%%%%%%%%%%%%%%%%%%%

%\section{Pronouns and some other quantifiers}
%\label{sec:Pronouns and some other quantifiers}

Sanzhi Dargwa has the following types of pronouns:

\begin{itemize}
	\item	personal pronouns (\refsec{sec:Personal pronouns})
	\item	\is{demonstrative pronoun}demonstrative pronouns (\refsec{sec:Demonstrative pronouns})
	\item	\is{reflexive pronoun}reflexive pronouns (\refsec{sec:Reflexive pronouns})
	\item	\is{reciprocal pronoun}reciprocal pronouns (\refsec{sec:Reciprocal pronouns})
	\item	interrogative pronouns (\refsec{sec:Interrogative pronouns})
	\item	various types of \is{indefinite pronoun}indefinite pronouns (\refsec{sec:Indefinite pronouns} and \refsec{sec:Universal indefinites and other quantifiers})
\end{itemize}

This chapter also includes a subsection on quantifiers such as \sqt{some}, \sqt{every} and \sqt{all} (\refsec{sec:Universal indefinites and other quantifiers}). 

Pro\isi{nouns} express the typical features of nominals, namely case, \isi{number}, and to a very limited extent \isi{gender} (only \is{reflexive pronoun}reflexive pronouns in the \isi{absolutive} case, one type of \isi{reciprocal pronoun}, essive-case forms of pronouns, e.g. of the pronoun `where'). Case marking of pronouns is almost fully regular and identical to the case marking of \isi{nouns} (and nominalized \isi{adjectives}, verbs, etc.). As for \isi{number} marking, only the \is{demonstrative pronoun}demonstrative pronouns and the interrogative `who' form the plural by means of special suffixes; personal and \is{reflexive pronoun}reflexive pronouns use suppletive stems and \is{indefinite pronoun}indefinite pronouns mostly do not have plural forms. The \isi{gender} exponents are the usual markers that are used across all parts of speech that express \isi{gender}.


% --------------------------------------------------------------------------------------------------------------------------------------------------------------------------------------------------------------------- %

\section{Personal pronouns}
\label{sec:Personal pronouns}

Sanzhi has personal pronouns for the first and for the second person. For the third person \is{demonstrative pronoun}demonstrative pronouns are used (\refsec{sec:Demonstrative pronouns}). \reftab{tab:Personal pronouns} displays a partial paradigm of the personal pronouns.

It is possible to make a few generalizations about the morphophonological structure of the personal pronouns. The \isi{absolutive} and the \isi{ergative} make use of the same root. Most other cases are formed from a distinct oblique root that is formed via ablaut from the \isi{absolutive} root (the first root vowel changes \textit{u} > \textit{i} for the first person, \textit{u} > \textit{a} for the second person; the second root vowel of the plural pronouns changes \textit{a} > \textit{i}). The only exceptional forms are the \isi{dative} forms of the singular pronouns that do not contain segments that could be identified as \isi{dative} case exponents, and the \isi{genitive} forms of the plural pronouns that are a mixture of \isi{absolutive} and \isi{oblique stem}. Note, furthermore, that the plural pronouns have \textit{-lla} as the \isi{genitive} suffix. The same allomorph of the \isi{genitive} case suffixe is optionally used for plural \isi{nouns} \refsec{sec:nouncase}).

\begin{table}[h]
	\caption{Personal pronouns}
	\label{tab:Personal pronouns}
	\small
	\begin{tabularx}{0.75\textwidth}[]{%
		>{\raggedright\arraybackslash}p{56pt}
		>{\raggedright\arraybackslash}X
		>{\raggedright\arraybackslash}X
		>{\raggedright\arraybackslash\hangindent=0.5em}X
		>{\raggedright\arraybackslash}X}
		
		\lsptoprule
		{}			&	\tsc{1sg}	 	&	\tsc{2sg}		&	\tsc{1pl}		&	\tsc{2pl}\\
		\midrule 
		\isit{absolutive}		&	\tit{du}		&	\tit{u}			&	\tit{nušːa}		&	\tit{ušːa}\\   
		\isit{ergative}		&	\tit{du-l}		&	\tit{u-l}		&	\tit{nušːa-l}		&	\tit{ušːa-l}\\
		\isit{dative}			&	\tit{dam}		&	\tit{at}		&	\tit{nišːi-j}		&	\tit{ašːi-j}\\
		\isit{genitive}		&	\tit{di-la}		&	\tit{a-la}		&	\tit{nišːa-lla}		&	\tit{ašːa-lla}\\
		\isit{comitative}		&	\tit{di-cːella}		&	\tit{a-cːella}		&	\tit{nišːi-cːella}	&	\tit{ašːi-cːella}\\ 
		\tsc{ad}-lative	&	\tit{di-šːu}		&	\tit{a-šːu}		&	\tit{nišːi-šːu}		&	\tit{ašːi-šːu}\\
		\tsc{in}-lative 	&	\tit{di-cːe}		&	\tit{a-cːe}		&	\tit{nišːi-cːe}	&	\tit{ašːi-cːe}\\
		\tsc{loc}-lative	&	\tit{di-ja}		&	\tit{a-ja}		&	\tit{nišːi-ja}		&	\tit{ašːi-ja}\\
		\tsc{sub}-lative	&	\tit{di-gu}		&	\tit{a-gu}		&	\tit{nišːi-gu}		&	\tit{ašːi-gu}\\
		\tsc{ante}-lative	&	\tit{di-sa}		&	\tit{a-sa}		&	\tit{nišːi-sa}		&	\tit{ašːi-sa}\\
		\midrule
		oblique root		&	\tit{di-}		&	\tit{a-}		&	\tit{nišːi-}		&	\tit{ašːi-}\\
		\lspbottomrule
	\end{tabularx}
\end{table}


% --------------------------------------------------------------------------------------------------------------------------------------------------------------------------------------------------------------------- %

\section{Demonstrative pronouns and adverbials derived from them}
\label{sec:Demonstrative pronouns}

Sanzhi Dargwa has a rich system of demonstratives whose stems express \isi{number} and case, but not \isi{gender}. These demonstratives fulfill a variety of deictic and non-deictic functions. Their deictic uses can be exophoric (e.g. gestural) or discourse deictic when demonstratives refer to a chunk of discourse (\textit{She said this}). Non-deictic uses of Sanzhi demonstratives can be anaphoric or cataphoric. The demonstratives are organized along several formal and semantic dimensions:
%

\begin{itemize}
	\item	\isi{number} (singular vs. plural)
	\item	form class (i.e. usage as adnominal modifier vs. independent pronoun vs. adverbial) 
	\item	proximity to speech act participants
	\item	elevation
	\item	visibility, aformentionedness, familiarity, etc.
\end{itemize}
%

\reftab{tab:Demonstrative pronouns} displays the demonstratives that serve pronominal and adnominal functions. In the table, they are divided into three series in the columns in both the singular (\textit{iC, heC, hiC}) and the plural (\textit{i(C)tːi}, he\textit{(C)tːi}, \textit{hi(C)tːi}). The series in the columns are distinguished by the root-initial segments (\refsec{ssec:The vertical dimension: iC vs. heC vs. hiC and i(C)tːi vs. he(C)tːi vs. hi(C)tːi}). There are six series of pronouns in the lines of the table that differentiated by their last root consonant (before the plural suffix in case of the plural pronouns), i.e. \textit{ž (š)} vs. \textit{j} vs. \textit{l} vs. \textit{t} vs. \textit{k' (x)} vs.\textit{χ} (\refsec{ssec:Proximity, distance, cardinal directions, and height The six horizontal series}). The series with \textit{j} as the last root consonant is defective because it exists only for singular \isi{absolutive} pronouns; singular oblique forms as well as any plural forms are unattested.

\begin{table}
	\caption{Basic (i.e. absolutive case) forms of nominal demonstratives}
	\label{tab:Demonstrative pronouns}
	\small
	\begin{tabularx}{\textwidth}{llllllQ}
		\lsptoprule
		\multicolumn{3}{c}{singular}	&	\multicolumn{3}{c}{plural}\\\cmidrule(lr){1-3}\cmidrule(lr){4-6}
	 	\tit{iC}	&	\tit{heC}	&	\tit{hiC}	&	\tit{i(C)tːi}	&	\tit{he(C)tːi}	&	\tit{hi(C)tːi}\\
		\midrule
		\tit{iž}		&	\tit{hež}	&	\tit{hiž}	&	\tit{ištːi} 	&	\tit{heštːi} 	&	\tit{hištːi}	&	\sqt{this}\slash\sqt{these}; close to the speaker (deictic center)\\	   
		\tit{ij}		&	\tit{hej}	&	\tit{hij}	&	\tmd		&	\tmd		&	\tmd		&	\sqt{this}\slash\sqt{these}; close to the speaker (deictic center)\\	   
		\tit{il}		&	\tit{hel}	&	\tit{hil}	&	\tit{iltːi}	&	\tit{heltːi}	&	\tit{hiltːi}	&	\sqt{that}\slash\sqt{those}; away from speaker; \\	  
				{}		&	{}		&	{}		&	{}		&	{}		&	{}		&	can be close to the hearer\\ 
		\tit{it}		&	\tit{het}	&	\tit{hit}	&	\tit{itːi}	&	\tit{hetːi}	&	\tit{hitːi}	&	\sqt{that}\slash\sqt{those}; not close to speaker\\
		{}		&	{}		&	{}		&	{}		&	{}		&	{}		&	or hearer, undifferentiated\\
		\tit{ik'}	&	\tit{hek'}	&	\tit{hik'}	&	\tit{ixtːi}	&	\tit{hextːi}	&	\tit{hixtːi}	&	above the deictic center\\
		\tit{iχ}	&	\tit{heχ}	&	\tit{hiχ}	&	\tit{iχtːi}	&	\tit{heχtːi}	&	\tit{hiχtːi}	&	below the deictic center\\
		\lspbottomrule
	\end{tabularx}
\end{table}



The plural pronouns are mostly based on the singular pronouns by adding the plural suffix \textit{-tːi} to the singular stem and some minor phonological adjustments. The oblique stems of the singular pronouns are formed by adding the suffix \textit{-i} to the stem (\reftab{tab:Oblique stem formation of demonstrative pronouns}) to which in turn case suffixes are attached. These two suffixes are not used for the inflection of \isi{nouns}, but only with \is{demonstrative pronoun}demonstrative pronouns. Partial paradigms of inflected pronouns are provided in \reftab{tab:Partial paradigms of some demonstrative pronouns}. For the \isi{oblique stem} of the plural pronouns the stem-final vowel \textit{i} is replaced by \textit{-a}, a suffix generally used for the formation of oblique plural stems of \isi{nouns} (\refsec{sec:nouncase}).

\begin{table}
	\caption{Oblique stem formation of demonstrative pronouns}
	\label{tab:Oblique stem formation of demonstrative pronouns}
	\small
	\begin{tabular}{*{8}{>{\itshape}l}}
		
		\lsptoprule
		\multicolumn{4}{c}{singular}	&	\multicolumn{4}{c}{plural}\\\cmidrule(lr){1-4}\cmidrule(lr){5-8}
 		\multicolumn{2}{c}{\tit{iC}}	&	\multicolumn{2}{c}{\tit{heC}}	&	\multicolumn{2}{c}{\tit{i(C)tːi}}	&	\multicolumn{2}{c}{\tit{he(C)tːi}}\\\cmidrule(lr){1-2}\cmidrule(lr){3-4}\cmidrule(lr){5-6}\cmidrule(lr){7-8}
		\multicolumn{1}{c}{\tsc{abs}}	&	\multicolumn{1}{c}{\tsc{obl}}	&	\multicolumn{1}{c}{\tsc{abs}}	&	\multicolumn{1}{c}{\tsc{obl}}
							&	\multicolumn{1}{c}{\tsc{abs}}	&	\multicolumn{1}{c}{\tsc{obl}}	&	\multicolumn{1}{c}{\tsc{abs}}
							&	\multicolumn{1}{c}{\tsc{obl}}\\
		\midrule
		iž 		&	iž-i- 		&	hež 		&	hež-i- 		&	ištːi		&	ištː-a- 	&	heštːi 		&	heštː-a-\\
		ij 		&	\tmd 		&	hej 		&	\tmd 		&	\tmd		&	\tmd		&	\tmd		&	\tmd\\
		il 		&	il-i- 		&	hel 		&	hel-i- 		&	iltːi		&	iltː-a-		&	heltːi		&	heltː-a-\\
		it 		&	it-i- 		&	het 		&	het-i- 		&	itːi		&	itː-a-		&	hetːi		&	hetː-a-\\
		ik' 		&	ik'-i- 		&	hek' 		&	hek'-i- 	&	ixtːi		&	ixtː-a-		&	hextːi		&	hextː-a-\\
		iχ 		&	iχ-i- 		&	heχ 		&	heχ-i-		&	iχtːi		&	iχtː-a-		&	heχtːi		&	heχtː-a-\\
		\lspbottomrule
	\end{tabular}
\end{table}

\begin{table}
	\caption{Partial paradigms of some demonstrative pronouns}
	\label{tab:Partial paradigms of some demonstrative pronouns}
	\small
	\begin{tabularx}{1\textwidth}[]{%
		>{\raggedright\arraybackslash}p{28pt}
		>{\raggedright\arraybackslash\hangindent=0.5em\itshape}p{36pt}
		>{\raggedright\arraybackslash\itshape}p{36pt}
		>{\raggedright\arraybackslash\itshape}p{37pt}
		>{\raggedright\arraybackslash\hangindent=0.5em\itshape}p{42pt}
		>{\raggedright\arraybackslash\itshape}p{43pt}
		>{\raggedright\arraybackslash\itshape}X
		>{\raggedright\arraybackslash\itshape}X}
		
		\lsptoprule
		{}			&	\multicolumn{1}{c}{\sqt{that}}	&	\multicolumn{1}{c}{\sqt{that}}	&	\multicolumn{1}{c}{\sqt{this}}	&	\multicolumn{1}{c}{\sqt{that}}	&	\multicolumn{1}{c}{\sqt{those}}	&	\multicolumn{1}{c}{\sqt{those}}\\
		\midrule
		abs.		&	il		&	it		&	iž		&	hel		&	iltːi		&	heltːi\\
		erg.		&	il-i-l		&	it-i-l		&	iž-i-l		&	hel-i-l		&	iltː-a-		&	heltː-a-l\\
		gen.		&	il-i-la		&	it-i-la		&	iž-i-la		&	hel-i-la	&	iltː-a-lla	&	heltː-a-lla\\
		dat.		&	il-i-j		&	it-i-j		&	iž-i-j		&	hel-i-j		&	iltː-a-j		&	heltː-a-j\\
		comit.		&	il-i-cːella	&	it-i-cːella	&	iž-i-cːella	&	hel-i-cːella	&	iltː-a-cːella	&	heltː-a-cːella\\
		\tsc{ad}-lat.	&	il-i-šːu		&	it-i-šːu		&	iž-i-šːu	&	hel-i-šːu	&	iltː-a-šːu	&	heltː-a-šːu\\
		\tsc{in}-lat. 	&	il-i-cːe		&	it-i-cːe		&	iž-i-cːe	&	hel-i-cːe	&	iltː-a-cːe	&	heltː-a-cːe\\
		\mbox{\tsc{loc}-lat.} &	ile\slash il-i-ja	&	it-i-ja		&	iž-i-ja		&	hele\slash hel-i-ja	&	iltː-a-ja	&	heltː-a-ja\\
		\lspbottomrule
	\end{tabularx}
\end{table}

The deictic meaning of the demonstratives is participant-oriented. Three semantic dimensions along the scale `proximity/distance to speech act participants' are distinguished: (i) near hearer (root-final \isi{consonants} \textit{ž/š} and \textit{j}), (ii) near addressee (\textit{l}), and (iii) undifferentiated or not close to speaker or addressee (\textit{t}).

Another aspect of the deictic semantics is elevation (or height), namely higher (up) or lower (down) location than the deictic center which is most commonly the speaker. Elevation distinctions in demonstratives are widespread in Dagestanian languages \citep{Schulze2003, ForkerLTSanzhi}, and the Sanzhi Dargwa system represent a typical instance.



% - - - - - - - - - - - - - - - - - - - - - - - - - - - - - - - - - - - - - - - - - - - - - - - - - - - - - - - - - - - - - - - - - - - - - - - - - - - - - - - - - - - - - - - - - - - - - - - - - - - - - - - - - - - - - - - - - - - - - - - - - - %

\subsection{The demonstrative series in the columns: \textit{iC} vs. \textit{heC} vs. \textit{hiC} and \textit{i(C)tːi} vs. \textit{he(C)tːi} vs. \textit{hi(C)tːi}}
\label{ssec:The vertical dimension: iC vs. heC vs. hiC and i(C)tːi vs. he(C)tːi vs. hi(C)tːi}

There is a pronounced difference in frequency between the three series. The \textit{he-} series is by far the most commonly used and the \textit{hi-} series is only very rarely used. Speakers are aware of the three different series but do not seem to notice a difference in semantics. The phonetic difference between the \textit{hi}-series and the \textit{i}-series is rather small and hard to hear. Thus, one of the reasons why the latter is so rare in the corpus might be that some of the tokens might incorrectly have been transcribed as \textit{i-}. In the following, I will only discuss the \textit{heC} and the \textit{iC} series.

When looking into natural texts it is not difficult to find tendencies hinting at the functional difference between the \textit{heC}-pronouns and the \textit{iC}-pronouns. The \textit{heC}-pronouns preferably refer to items or persons that are or have been:

\begin{itemize}
	\item	in the immediate deictic sphere of speaker (and addressee) and/or part of the knowledge sphere or social world of the speaker
	\item	aforementioned or are assumed to be part of the ongoing conversation
	\item	common knowledge
\end{itemize}

First of all, \textit{heC}- pronouns are used for denoting visible referents, for instance in pointing events. For example, after \refex{Who, he says to them, is Asijats brother? This one} has been uttered the speaker stresses the fact that the person in the narrative was only pointing at the man, but not saying anything:

\begin{exe}
	\ex	{[Three men were standing there like this.]}\\	\label{Who, he says to them, is Asijats brother? This one}
	\gll	``kutːi''	∅-ik'ʷ-ar	hetːa-cːe-r	``Asijat-la	ucːi?'' 	``heχ'' \\
		which	\textsc{m}-say.\textsc{ipfv}-\textsc{prs}	those.\textsc{obl}-\textsc{in}-\textsc{abl}	Asiyat-\textsc{gen}	brother \textsc{dem.down}\\
	\glt	\sqt{``Who,'' he says to them, ``is Asiyat's brother?'' ``This one.''}

	\ex	{[The speaker pointed at a similar bottle.]} \\	\label{He gave me such a bottle and sent me away}
	\gll	heχ=ʁuna	šuša	kʷi-b-ičː-ib	r-at	k-aʁ-ib=da\\
		\textsc{dem.down=eq}	bottle	\textsc{in.hands}-\textsc{n}-give.\textsc{pfv}-\textsc{pret}	\textsc{f}-send	\textsc{down}-do-\textsc{pret}=1 \\
	\glt	\sqt{He gave me such a bottle and sent me (fem.) away.}

	\ex	{[referring to a boy that turned up unexpectedly during the conversation]}\\	\label{[referring to a boy that turned up unexpectedly during the conversation}
	\gll	aman!		het	ceqːel	ha-∅-jʁ-ib=e?\\
		oh		that	when	\textsc{up}-\textsc{m}-come.\textsc{pfv}-\textsc{pret}=\textsc{q}\\
	\glt	\sqt{Oh! When did he come?}
\end{exe}

The \textit{heC}-pronouns are used for referents within the personal social sphere of the speaker such as her/his close relatives and other people well-known to the speaker \refex{To Bashlikent, to MR, to the ones that you know we will bring you}, body parts of the speaker \refex{Now the (i.e. my) lips are swollen, it is difficult to talk}, etc. These items or persons can be assumed to be implicitly present in the discourse and can be identified via their close relationship to the speaker.

\begin{exe}
	\ex	\label{To Bashlikent, to MR, to the ones that you know we will bring you}
	\gll 	Baršlikːent-le	Maˁħaˁmmarasul-li-šːu	hex-tːi	a-la	b-alχ-an-t-a-šːu	r-ik-an=da \\
		Barshlikent-\textsc{loc}	Mahammarasul-\textsc{obl-ad}	\textsc{dem.up-pl}	\textsc{2sg-gen}	\textsc{hpl}-know.\textsc{ipfv}-\textsc{ptcp}-\textsc{pl}-\textsc{obl-ad} \textsc{f}-lead.\textsc{ipfv}-\textsc{ptcp}=1\\
	\glt	\sqt{To Bashlikent, to Mahammarasul, to the ones that you know we will bring you (fem.).}

	\ex	\label{Now the (i.e. my) lips are swollen, it is difficult to talk}
	\gll	hana	heš-tːi	k'unt'-be	d-emtː-un-ne		ʁaj	r-ik'ʷ-ij	wahi-l	ca-b \\
now	this-\textsc{pl}	lip-\textsc{pl}	\textsc{npl}-swell.\textsc{pfv}-\textsc{pret}-\textsc{cvb}	word	\textsc{f}-say.\textsc{ipfv}-\textsc{inf}	bad-\textsc{advz}	\textsc{cop-n}\\
	\glt	\sqt{Now the (i.e. my) lips are swollen, it is difficult to talk.}
\end{exe}

Second, the \textit{heC}-pronouns refer to referents that have been introduced in the preceding discourse, either in the immediately preceding sentence such that they establish a kind of topic continuity or when referring back to something said some time ago. Thus, sentence \refex{There is the forest that is called the mill’s forest, the ground, if you know it directly} brings up a new topic, the mill forest. The speaker is then constantly referring back to the forest with the pronouns \textit{hek'}, \textit{het} and \textit{hej} \refex{A forest like this exists there, the mill’s forest it is called}, \refex{There in the forest nothing should be cut}. The first clause of \refex{We made sticks and hit them into the ground} introduces a new referent, the sticks, and the following clause refers to them by means of a \textit{heC}-pronoun.

\begin{exe}
	\ex
	\begin{xlist}
		\ex	\label{There is the forest that is called the mill’s forest, the ground, if you know it directly}
		\gll	urχːab-la	wac'a	b-ik'-ul,	ganza	te-b	u-l	b-alχ-atːe	prjama	hek' ...\\
			mill-\textsc{gen}	forest	\textsc{hpl}-say.\textsc{ipfv}-\textsc{icvb}	ground	exist-\textsc{n}	\textsc{2sg}-\textsc{erg}	\textsc{n}-know.\textsc{ipfv}-\textsc{cond}.2	directly	\textsc{dem.up}\\
		\glt	\sqt{The forest that is called the mill's forest, there is the ground, if you know it directly ...}

		\ex	\label{A forest like this exists there, the mill’s forest it is called}
		\gll	hej=ʁuna	wac'a	k'e-b,	urχːab-la	wac'a	b-ik'ʷ-ar	hek'-i-j  \\
			this=\textsc{eq}	forest	exist.\textsc{up}-\textsc{n}	mill-\textsc{gen}	forest	\textsc{hpl}-say.\textsc{ipfv}-\textsc{prs}	\textsc{dem.up}-\textsc{obl}-\textsc{dat}\\
		\glt	\sqt{A forest like this exists there, the mill's forest it is called}

		\ex	\label{There in the forest nothing should be cut}
		\gll	hextːu		hek'	wac'a-cːe-b	cik'al	ka-b-irčː-an akːu  \\
			there		\textsc{dem.up}	forest-\textsc{in-n}	nothing	\textsc{down-n}-cut.\textsc{ipfv}-\textsc{ptcp} \textsc{cop.neg}\\
		\glt	\sqt{There in the forest nothing should be cut}
	\end{xlist}

	\ex	{[‎‎When we were little we had a game.]}	\label{We made sticks and hit them into the ground}
	\sn
	\gll	dirx-me	d-umkːa	d-arq'-ib-le,	ganza-l-cːe	d-urq-aˁ-di	hel-tːi\\
		stick-\textsc{pl}	\textsc{npl}-sharp	\textsc{npl}-do.\textsc{pfv}-\textsc{pret}-\textsc{cvb}	ground-\textsc{obl-in}	\textsc{npl}-hit.\textsc{ipfv}-\textsc{hab}-1	that-\textsc{pl} \\
	\glt	\sqt{We made sticks and hit them into the ground.}
\end{exe}

While speakers performed the \textit{Family Problems Picture Task} \citep{SanRoqueEtAl2012}, they constantly needed to refer to the people and objects depicted on the pictures. Frequently they first used an \tit{iC}-pronoun to establish a new referent, and then, in a kind of afterthought following the clause, repeated the reference again by employing a \tit{heC}-pronoun \refex{They already carried him away}, \refex{and these are watermelons, I don't know, these}. The first \is{demonstrative pronoun}demonstrative pronouns in such clauses can be interpreted as deictic, whereas the second demonstratives in the same examples represent the anaphoric use. In the following two examples the relevant demonstratives are given in boldface.

\begin{exe}
	\ex	\label{They already carried him away}
	\gll	\textbf{iž}	uže	w-erč-ib-le	\textbf{hež} \\
		this	already	\textsc{m}-lead.\textsc{pfv}-\textsc{pret}-\textsc{cvb}	this\\
	\glt	\sqt{They already carried him away.}
	
	\ex	\label{and these are watermelons, I don't know, these}
	\gll	a	\textbf{iš-tːi}	qːalpuz-e=jal,	aχːu,	\textbf{heš-tːi} \\
		and	this-\textsc{pl}	watermelon-\textsc{pl}=\textsc{indq}	not.know	this-\textsc{pl}\\
	\glt 	\sqt{and these are watermelons, I don't know, these.}
\end{exe}


This function is reflected in the meaning of the adverb \textit{hel-i-j} (that-\textsc{obl}-\textsc{dat}) \sqt{therefore} and the phrases \tit{hel bahandan} \sqt{for this reason} and \tit{hel zamana} \sqt{that time}, which link causally or temporally connected passages in a stretch of discourse.

Third, the \textit{heC}-pronouns denote objects and persons that the speaker assumes to be familiar for the hearer, i.e. that are common knowledge such as certain places, famous people, etc.:

\begin{exe}
	\ex	\label{when I reached the crossover of Mazhalis}
	\gll	hej	Mažalis-la	pawarut'e-le	w-iteʁ-ib=qːella, \ldots\\
		this	Mazhalis-\textsc{gen}	crossover-\textsc{loc}	\textsc{m}-reach.\textsc{pfv}-\textsc{pret}=when\\
	\glt	\sqt{when (I) reached the crossover of Mazhalis, \ldots}
\end{exe}

The use of the \textit{iC}-pronouns diverges from the use of the \textit{heC}-pro\-nouns. The \textit{iC}-pro\-nouns preferably occur when new topics and referents are introduced into the discourse \refex{And this man called Arsen, he was also there, with his wife from Usisha.} or when topics switch \refex{What did you say about uncle Abdulkhalik's father?}, \refex{And the boy went slowly away, and the three boys went there to that place, on that road.}.

\begin{exe}
	\ex	\label{And this man called Arsen, he was also there, with his wife from Usisha.}
	\gll	iž	Arsen	b-ik'-ul	iž=ra	le-w=de	iž	Usːan	xːunul-li-cːe-r\\
	this	Arsen	\textsc{hpl}-say.\textsc{ipfv}-\textsc{icvb} this=\textsc{add}	exist-\textsc{m}=\textsc{pst}	this	Usisha.person	woman-\textsc{obl}-\textsc{in}-\textsc{abl}\\
	\glt	\sqt{And this man called Arsen, he was also there, with his wife from Usisha.}

	\ex	{[switching back the topic of the conversation to a person known to both speaker and hearer]}	\label{What did you say about uncle Abdulkhalik's father?}
	\sn
	\gll	c'il	it	ʡaˁbdulχaliq'	acːi-la	atːa-l	ce=jal	∅-ik'ʷ-a-tːe=q'al	u\\
		then	that	Abdulkhalik	uncle-\textsc{gen}	father-\textsc{erg}	what=\textsc{indef}	\textsc{m}-say.\textsc{ipfv}-\textsc{hab}.\textsc{pst}-\textsc{2sg}=\textsc{mod}	\textsc{2sg}\\
	\glt	\sqt{What did you say about uncle Abdulkhalik's father?}\footnote{The noun \sqt{father} bears the \isi{ergative} case because the speaker intended to ask for something that Abdulkhalik's father had done, without explicitly saying so in his utterance.}

	\ex	{[Then when they had gathered all these pears, they put them again on the bike.]}	\label{And the boy went slowly away, and the three boys went there to that place, on that road.}
	\sn
	\gll	il	durħuˁ	arg-ul=de	bahla-l,	itːi	ʡaˁbal	durħ-ne	arg-ul=de	het	sa-∅-jʁ-ib	musːa-r het	xːun-ni-cːe-r\\
		that	boy	go.\textsc{ipfv}-\textsc{icvb}=\textsc{pst}	slow-\textsc{advz}	\textsc{dem.pl} three	boy-\textsc{pl}	go.\textsc{ipfv}-\textsc{icvb}=\textsc{pst}	that \textsc{hither-m}-come.\textsc{pfv}-\textsc{pret} place.\textsc{loc}-\textsc{abl} that	road-\textsc{obl-in}-\textsc{abl} \\
	\glt	\sqt{And the boy went slowly away, and the three boys went there to that place, on that road.}

	%\ex	\label{Every year, he went this year.}
	%\gll	har	dus,	ij	dus	ag-ur-il=de\\
		%every	year	this	year	go.\textsc{pfv}-\textsc{pret}-\textsc{ptcp}=\textsc{pst}\\
	%\glt	\sqt{Every year, he went this year.}
\end{exe}

The \textit{iC}-pronouns are also used when the referent or the topic of the conversation has been introduced into the discourse, but the speaker considers them to be out of his/her personal sphere. For instance, in \refex{In Urkarakh there is this son of my friend} and \refex{She does not know anything, (my daughter) says.2} the speaker continues to talk about acquaintances of hers who are not close friends or relatives of herself:

\begin{exe}
	\ex	{[‎‎In Urkarakh there is this son of my friend.]}	\label{In Urkarakh there is this son of my friend}
	\sn
	\gll	ik'-i-l	har	cik'al	di-la=ra	d-irq'-u\\
		\textsc{dem.up-obl}-\textsc{erg} 	every	something	\textsc{1sg}-\textsc{gen}=\textsc{add}	\textsc{npl}-do.\textsc{ipfv}-\textsc{prs}\\
	\glt	\sqt{He does all my things. (i.e. does everything)}

	\ex	\label{She does not know anything, (my daughter) says.2}
	\gll	it-i-l	b-alχ-an	b-akːu=q'al	r-ik'ʷ-ar\\
		that-\textsc{obl}-\textsc{erg}	\textsc{n}-know.\textsc{ipfv}-\textsc{ptcp}	\textsc{n}-\textsc{cop.neg}=\textsc{mod}	\textsc{f}-say.\textsc{ipfv-prs}\\
	\glt	\sqt{She does not know anything, (my daughter) says.}
\end{exe}

In example \refex{There were plants that she gathered.} the speaker is talking about a person who is present but does not belong to the Sanzhi community, and who does not understand Sanzhi (later the speaker switches to \textit{hel} when referring to the same person):

\begin{exe}
	\ex	\label{There were plants that she gathered.}
	\gll	iž-i-l	d-alc'-un	q'ar	le-d=de\\
		this-\textsc{obl}-\textsc{erg}	\textsc{npl}-gather.\textsc{pfv}-\textsc{pret}	plant	exist-\textsc{pl}=\textsc{pst}\\
	\glt	\sqt{There were plants that she gathered.}
\end{exe}

In \refex{Prepare (the groceries), when your little sister comes,} the speaker is contradicting and correcting the addressee (who is his wife) and perhaps distancing himself a bit form the referent (his sister-in-law):

\begin{exe}
	\ex	{[Prepare (the groceries), when your little sister comes, for her to take them.]}\label{Prepare (the groceries), when your little sister comes,}
	\sn
	\gll	iž-i-l	d-uqː-ij	a-r-irχ-u	itːi\\
		this-\textsc{obl}-\textsc{erg}	\textsc{npl}-carry.\textsc{pfv}-\textsc{inf}	\textsc{neg}-\textsc{f}-be.able.\textsc{ipfv}-\textsc{prs} \tsc{dem.pl}\\
	\glt	\sqt{She cannot carry them.}
\end{exe}

However, these are only tendencies, not strict rules. Speakers play around with the pronouns, use different pronouns for one and the same referent or correct themselves. Thus, in \refex{By God, when two pills (i.e. medicines) were left behind, (I) also} and \refex{He is our good friend, but he has a bit this habit} the same objects (the pills) and person (the friend) are first referred to by means of a \textit{heC}-pronoun and then immediately later by an \textit{iC}-pronoun. This is the opposite order of what I found in the data from the \textit{Family Problems Picture Task} presented above in \refex{They already carried him away}, \refex{and these are watermelons, I don't know, these}.

\begin{exe}
	\ex	\label{By God, when two pills (i.e. medicines) were left behind, (I) also}
	\gll	wallah	k'ʷel	darman	hila	d-uq-un=qːel,	urk'i	xul-le	\textbf{hel-tːi}=ra	d-erčː-ib=da	\textbf{hix-tːi}=ra \\
		by.God two medicine behind \textsc{npl}-go.\textsc{pfv-pret}=when heart	wish-\textsc{advz} that-\textsc{pl}=\textsc{add} \textsc{npl}-drink.\textsc{pfv}-\textsc{pret}=1 \textsc{dem.up-pl}=\textsc{add}\\
	\glt	\sqt{By God, when two pills (i.e. medicines) were left behind, (I) also wanted them, and I also drank them.}
	
	\ex	{[He is like this, he is always busy, has many friends, etc.]} \label{He is our good friend, but he has a bit this habit}
	\sn
	\gll	c'il	ca-w	ʡaˁħ	juldaš	ca<w>i	nišːa-la	nu	\textbf{heχ}	q'ʷila	q'ʷila	\textbf{iχ-i-cːe-b}	χasijat	χe-b-il	ca-b\\
		then	\textsc{refl-m}	good	friend	\textsc{cop<m>}	\textsc{1pl}-\textsc{gen}	well		\textsc{dem.down}	a.little	a.little	\textsc{dem.down}-\textsc{obl-in}-\textsc{n}	habit	exist.\textsc{down}-\textsc{n}-\textsc{ref}	\textsc{cop-n}\\
	\glt	\sqt{He is our good friend, but he has a bit of this habit.}
\end{exe}

In the following sections, I will discuss the differences between the horizontal series (i.e. the pronouns in the six different lines of \reftab{tab:Demonstrative pronouns}) and largely ignore the differences between the columns.

% - - - - - - - - - - - - - - - - - - - - - - - - - - - - - - - - - - - - - - - - - - - - - - - - - - - - - - - - - - - - - - - - - - - - - - - - - - - - - - - - - - - - - - - - - - - - - - - - - - - - - - - - - - - - - - - - - - - - - - - - - - %

\subsection{Proximity, distance, and elevation}
\label{ssec:Proximity, distance, cardinal directions, and height The six horizontal series}


% - - - - - - - - - - - - - - - - - - - - - - - - - - - - - - - - - - - - - - - - - - - - - - - - - - - - - - - - - - - - - - - - - - - - - - - - - - - - - - - - - - - - - - - - - - - - - - - - - - - - - - - - - - - - - - - - - - - - - - - - - - %

\subsubsection{\textit{ž}-pronouns: \textit{iž}, \textit{hež}, \textit{hiž}, \textit{ištːi}, \textit{heštːi}, \textit{hištːi}; and \textit{j}-pronouns: \textit{ij}, \textit{hej}, \textit{hij}}
\label{sssec:z-pronouns j-pronouns}

These pronouns express proximity and normally denote referents close to the speaker. The \textit{ž}-pronouns are preferably used as independent pronouns \refex{There were plants that she gathered.}, \refex{Prepare (the groceries), when your little sister comes,}, \refex{These happy ones (=happy people on a picture) give them here!}, \refex{They have only one son, right?}, where\-as the \textit{j}-pronouns predominantly occur as deictic modifiers of \isi{nouns} and definite markers similar to articles \refex{This picture, what picture is it/this}, \refex{This is when he came back; this (iž) then needs to be the (hij) very last}, \refex{This (iž) is the (ij) trial}, but again these are tendencies, not strict rules. The \textit{j}-pronouns have only singular \isi{absolutive} forms, lacking entirely singular oblique and all plural forms.

All following examples are from the \textit{Family Problems Picture Task} when speakers where referring to pictures and people on the pictures that were lying close to them on the table.

\begin{exe}
	\ex	\label{This picture, what picture is it/this}
	\gll	hej	sːurrat,	ce	sːurrat=e	iž?\\
		this	picture	what	picture=\textsc{q}	this\\
	\glt	\sqt{This picture, what picture is it/this?}

	\ex	\label{This is when he came back; this (iž) then needs to be the (hij) very last}
	\gll	sa-∅-jʁ-ib-il=de=q'al,	iž	hij	bah	hila	b-ax-an ca-b	abuχar\\
		\textsc{hither-m}-come.\textsc{pfv}-\textsc{pret}-\textsc{ref}=\textsc{pst}=\textsc{mod}	this	this	most	behind	\textsc{n}-go.\textsc{ipfv}-\textsc{ptcp} \textsc{cop-n}	then\\
	\glt	\sqt{This is when he came back; this (\tit{iž}) then needs to be the (\tit{hij}) very last.}

	\ex	\label{This (iž) is the (ij) trial}
	\gll	ij	sud	ca-b	iž\\
		this	trial	\textsc{cop-n}	this\\
	\glt \sqt {This (\tit{iž}) is the (\tit{ij}) trial.}
	
\ex	\label{These happy ones (=happy people on a picture) give them here!}
	\gll	heštːi	razi-te	heštːu	b-iqː-a! \\
		these	happy-\textsc{dd}\textsc{.pl} 	here	\textsc{hpl}-take.out.\textsc{ipfv}-\textsc{imp}\\
	\glt	\sqt{These happy ones (=happy people on a picture) give them here!}

	\ex	\label{They have only one son, right?}
	\gll	iš-tː-a-la	ca	akːʷ-ar	durħuˁ,	w-akːu=w?\\
		this-\textsc{pl}-\textsc{obl}-\textsc{gen}	one	\textsc{cop.neg}-\textsc{prs} boy	\textsc{m}-\textsc{cop.neg}=\textsc{q}\\
	\glt	\sqt{They have only one son, right?}


\end{exe}


% - - - - - - - - - - - - - - - - - - - - - - - - - - - - - - - - - - - - - - - - - - - - - - - - - - - - - - - - - - - - - - - - - - - - - - - - - - - - - - - - - - - - - - - - - - - - - - - - - - - - - - - - - - - - - - - - - - - - - - - - - - %

\subsubsection{\textit{l}-pronouns: \textit{il}, \textit{hel}, \textit{hil}, \textit{iltːi}, \textit{heltːi}, \textit{hiltːi}}
\label{sssec:l-pronouns}

These pronouns denote objects or persons that are not in the proximity of the speaker, but close to the addressee \refex{Do not talk to them! (i.e. to the pictures)}.

\begin{exe}
	\ex	{[The addressee starts talking to the picture in front of her. The other speaker says to her:]}	\label{Do not talk to them! (i.e. to the pictures)}
	\sn
	\gll	u	il-tː-a-cːe	ʁaj	ma-r-ik'-ut!\\
		2sg	that-\textsc{pl}-\textsc{obl}-in	word	\textsc{proh}-\textsc{f}-say.\textsc{ipfv}-\textsc{proh.sg}\\
	\glt	\sqt{Do not talk to them!} (i.e. to the pictures)
\end{exe}

They are also employed when talking about absent referents or items located further away, not necessarily in the proximity of the hearer \refex{The color does not fly off, nothing happened to it.}, \refex{They remained there, my two sons and those two neighbors.}. Finally, they are the default pronouns in fiction such as traditional narratives, legends, etc. \refex{She turned into a monster.}.

\begin{exe}
	\ex	{[talking about the colors used for the rock paintings; the conversation takes place far away from the painting]}	\label{The color does not fly off, nothing happened to it.}
	\sn
	\gll	il	kraska	atletit	b-iχ-ub-le akːu, 	il-i-j	cik'al	ag-ur-re=kːu\\
		that	color	fly.away	\textsc{n}-be.\textsc{pfv}-\textsc{pret}-\textsc{cvb} \textsc{cop.neg}	that-\textsc{obl}-\textsc{dat} nothing	go.\textsc{pfv}-\textsc{pret}-\textsc{cvb}=\textsc{cop.neg}\\
	\glt	\sqt{The color does not fly off, nothing happened to it.}

	\ex	{[I went to my house.]}	\label{They remained there, my two sons and those two neighbors.}
	\sn
	\gll	hel-tːi	kelg-un	heltːu-b	di-la	k'ʷel=ra	durħuˁ=ra,	hel-tːi	k'ʷel=ra	zunra	admi=ra\\
		that-\textsc{pl}	remain.\textsc{pfv}-\textsc{pret}	there-\textsc{hpl}	\textsc{1sg-gen}	two=\textsc{add}	boy=\textsc{add}	that-\textsc{pl}	two=\textsc{add}	neighbor	person=\textsc{add}\\
	\glt	\sqt{They remained there, my two sons and those two neighbors.}

	\ex	\label{She turned into a monster.}
	\gll	aždaha	ag-ur ca-r	hel-i-cːe-r \\
		monster	go.\textsc{pfv}-\textsc{pret} \textsc{cop-f}	that-\textsc{obl-in}-\textsc{abl}\\
	\glt	\sqt{She turned into a monster.}
\end{exe}


% - - - - - - - - - - - - - - - - - - - - - - - - - - - - - - - - - - - - - - - - - - - - - - - - - - - - - - - - - - - - - - - - - - - - - - - - - - - - - - - - - - - - - - - - - - - - - - - - - - - - - - - - - - - - - - - - - - - - - - - - - - %

\subsubsection{\textit{t}-pronouns: \textit{it, het, hit, itːi, hetːi, hitːi}}
\label{sssec:t-pronouns}

These pronouns refer to persons or objects whose location is undifferentiated, irrelevant, or impossible to determine or that are not close to the speaker or the hearer. They are used, for instance, when talking about people that are not present, or about unknown referents, of which it is not important where they are located \refex{Who was that (masc.) who told me that}, \refex{He left her and went to that, he left her and when to the next.}. They are also very frequently used in elicitation.

\begin{exe}
	\ex	\label{Who was that (masc.) who told me that}
	\gll	ča=de	it	di-cːe	∅-ik'ʷ-an?\\
		who=\textsc{pst}	that	\textsc{1sg-in}	\textsc{m}-say.\textsc{ipfv-ptcp}\\
	\glt	\sqt{Who was that (masc.) who told me that?}

	\ex	{[talking about the former lovers of the husband of the speaker]}	\label{He left her and went to that, he left her and when to the next.}
	\sn
	\gll	it	r-ax	r-at-ur,	het-i-šːu	w-ax-ul,	it	r-ax	r-at-ur	hetilil-li-šːu	w-ax-ul\\
		that	\tsc{f-}let	\tsc{f-}let\tsc{.pfv-pret}	that\tsc{-obl-ad}	\tsc{m-}go\tsc{.ipfv-icvb}	that	\tsc{f-}let	\textsc{f}-let.\textsc{pfv}-\textsc{pret}	other-\textsc{obl-ad}	\textsc{m}-go.\textsc{ipfv}-\textsc{icvb}\\
	\glt	\sqt{He left her and went to that one; he left her and went to the next one.}

    \pagebreak
	\ex	{[talking about a stone fence that the speaker is building; both speaker and hearer are located somewhere away from the fence]}	\label{talking about a stone fence that the speaker is building; both speaker} 
	\sn
	\gll	c'il=ra	het	šːal-le-b	lac	či-b-irq'-an=uw?\\	
		then=\textsc{add}	that	side-\textsc{loc}-\textsc{n}	fence	\textsc{spr}-\textsc{n}-do.\textsc{ipfv}-\textsc{ptcp}=\textsc{q} \\
	\glt	\sqt{Then you also have to build the fence from that side?}
\end{exe}

With \refex{At that time this hand (of mine) was still working} the speaker refers back to former times and \refex{I have been to this wedding as well.} is the typical final statement of a traditional story that ends with the wedding of the protagonist:

\begin{exe}
	\ex	\label{At that time this hand (of mine) was still working}
	\gll	ij	naˁq	hit=qːel	b-ucː-ul	b-el=de\\
		this	hand	that=when	\textsc{n}-work-\textsc{icvb}	\textsc{n}-remain.\textsc{pfv}=\textsc{pst}\\
	\glt	\sqt{At that time this hand (of mine) was still working.}

	\ex	\label{I have been to this wedding as well.}
	\gll	du=ra	het	meq-le-w	kelg-un=da\\
		\textsc{1sg}=\textsc{add}	that	wedding-\textsc{loc}-\textsc{m}	remain.\textsc{pfv}-\textsc{pret=1}\\
	\glt	\sqt {I have been to this wedding as well.}
\end{exe}


% - - - - - - - - - - - - - - - - - - - - - - - - - - - - - - - - - - - - - - - - - - - - - - - - - - - - - - - - - - - - - - - - - - - - - - - - - - - - - - - - - - - - - - - - - - - - - - - - - - - - - - - - - - - - - - - - - - - - - - - - - - %

\subsubsection{\textit{k'-/x}-pronouns: \textit{ik'}, \textit{hek'}, \textit{hik'}, \textit{ixtːi}, \textit{hextːi}, \textit{hixtːi}}
\label{sssec:k-x-pronouns}

These pronouns are used when referring to items or people located above the level of the deictic center (which is most commonly the speaker), e.g. in the mountains as in \refex{They were interesting.}, \refex{At the time when we became Muslims, they did not become Muslims.} or higher than some other point of reference \refex{Whatever may happen, do not look at them}. For instance, in examples \refex{They were interesting.}, \refex{If now she comes from over there I am trembling.}, the deictic center is the speaker, but it can also be another location. See \citet{ForkerLTSanzhi} for more information on the deictic category of elevation in Sanzhi Dargwa.

\begin{exe}
	\ex	{[talking about rock paintings located in the mountains, higher up than Sanzhi]}	\label{They were interesting.}
	\sn
	\gll	intersna=de	ix-tːi \\
		interesting=\textsc{pst}	\textsc{dem.up}-\textsc{pl}\\
	\glt	\sqt{They were interesting.}

	\ex	{[referring to the inhabitants of a legendary village that is supposed to have existed on the mountains above Sanzhi]}	\label{At the time when we became Muslims, they did not become Muslims.}
	\sn
	\gll	islam	prinimat	b-irq'-an	zamana	hex-tː-a-l	prinimat	b-arq'-ib-le	a-b-určː-i\\
		Islam	accept	\textsc{hpl}-do.\textsc{ipfv}-\textsc{ptcp}	time		\textsc{dem.up}-\textsc{pl}-\textsc{obl}-\textsc{erg}	accept	\textsc{n}-do.\textsc{pfv}-\textsc{pret}-\textsc{cvb}	\textsc{neg}-\textsc{hpl}-be.\textsc{ipfv}-\textsc{hab.}\textsc{\textsc{pst}}\\
	\glt	\sqt{At the time when we became Muslims, they did not become Muslims.}

	\ex	{[In one place there are trees.]}	\label{Whatever may happen, do not look at them}
	\sn
	\gll	warilla.wari	u	ix-tː-a-j	er	či-ma-hark'-utːa!\\
		no.way	\textsc{2sg}	\textsc{dem.up}-\textsc{pl}-\textsc{obl}-\textsc{dat}	look	\textsc{spr}-\textsc{proh}-look.\textsc{ipfv}-\textsc{proh.sg}\\	
	\glt	\sqt{Whatever may happen, do not look at them (=trees)!}

\end{exe}

The factual elevation with respect to the deictic center can be minimal as long as speakers perceive a difference in height. For instance, the village of Druzhba where most Sanzhi speakers live is located on the flat land around 5 km from the Caspian sea coast. It stretches from the highway that runs parallel to the coast and to a point approximately one kilometer before the slops of some hills. If there is any difference in elevation between the two ends of the village it is minimal and not visible, but the part of the village closer to the sea is conceptualized as ‘lower' whereas the part closer to the hills is regarded as `higher'. Example \refex{If now she comes from over there I am trembling.} originates from a conversation about a woman who lives in the ‘higher' part of the village and the speaker uses \textit{hek'} with reference to that woman. Thus, it is not necessarily the location at the time of speaking that is relevant, but the usual location of the referent in relation to the deictic center can be decisive for the use of demonstratives.

\begin{exe}
	\ex	{[talking about a woman who lives in the `upper part' of the village]}	\label{If now she comes from over there I am trembling.}
	\sn
	\gll	hana	hek'	hek-ka	ka-r-eʁ-ij=al		gargar	gargar	r-ik'-ud	du \\
		now	\textsc{dem.up}	\textsc{dem.up-down}	\textsc{down-f}-go.\textsc{pfv}-\textsc{inf=indq}	trembling	trembling	\textsc{f-}say.\textsc{ipfv-1.prs}	\textsc{1sg}\\		
	\glt	\sqt{If now she comes from over there I am trembling.}

%	\ex	{[talking about a man who lives in the `upper part' of the village]}	\label{May nobody die like Husejni, like this Husejni.}
%	\sn
%	\gll	hik'	ħuˁsejni	daˁʡle	ma-ka-at-ab	admi		hek'	ħuˁsejni	daˁʡle\\
%		\textsc{dem.up}	Husejni	as	\textsc{proh-down}-let.\textsc{pfv}-\textsc{opt}.3	person	\textsc{dem.up}	Husejni	as\\
%	\glt	\sqt{May nobody die like Husejni, like this Husejni.}
\end{exe}

Sentence \refex{Or he did beat her?} has been uttered during a \textit{Family Problems Picture Task} discussion. The picture shows the arrest of the protagonist by the police. His wife is sitting on the ground and he is carried away by two policemen. In the picture, he is depicted higher than the woman. Example \refex{It turns out that they lived well in the beginning. Look at them both together!} is also part of a \textit{Family Problems Picture Task} discussion. The two \is{demonstrative pronoun}demonstrative pronouns refer to the main protagonists who are depicted in little bubbles above the main scene of the picture.

\begin{exe}
	\ex	\label{Or he did beat her?}
	\gll	ili	ik'-i-l	r-it-ib-le=w	iχ?\\
		or	\textsc{dem.up-obl-erg}	\textsc{f}-beat.up-\textsc{pret-cvb=q}	\textsc{dem.down}\\
	\glt	\sqt{Or did he beat her up?}

	\ex	\label{It turns out that they lived well in the beginning. Look at them both together!}
	\gll	bahsar	ix-tːi	qːuʁa-l	er	b-irχ-ul	b-už-ib ca-b;	hex-tːi	er	r-erč'-e	k'ʷel=ra	canille!\\
		first	\textsc{dem.up}-\textsc{pl}	beautiful-\textsc{advz}	life	\textsc{hpl}-be.\textsc{ipfv}-\textsc{icvb}	\textsc{hpl}-stay-\textsc{pret} \textsc{cop-hpl}	\textsc{dem.up}-\textsc{pl}	look	\textsc{f}-look.\textsc{pfv-im}\textsc{p}	two=\textsc{add}	together\\
	\glt	\sqt{It turns out that they lived well in the beginning. Look at them both together!}
\end{exe}


% - - - - - - - - - - - - - - - - - - - - - - - - - - - - - - - - - - - - - - - - - - - - - - - - - - - - - - - - - - - - - - - - - - - - - - - - - - - - - - - - - - - - - - - - - - - - - - - - - - - - - - - - - - - - - - - - - - - - - - - - - - %

\subsubsection{\textit{χ}-pronouns: \textit{iχ, heχ, hiχ, iχtːi, heχtːi, hiχtːi}}
\label{sssec:x-pronouns}

These pronouns denote referents located below the deictic center. For instance, example \refex{What happened to her?} originates from a conversation about a woman who lives in the part of the village closer to the sea and the speaker continuously uses \textit{iχ} with reference to that woman. Examples \refex{He and they, his friends also meet and sit together.} and \refex{(They) stole it, whatever it is, this, their watermelons.} refer to people and items on pictures. The people are sitting down and the pumpkins (referred to as watermelons) on the picture are lying on the ground.

\begin{exe}
	\ex	\label{What happened to her?}
	\gll	ce	ag-ur-re=l	iχ-i-j?\\
		what	go.\textsc{pfv-pret-cvb}=\textsc{indq}	\textsc{dem.down}-\textsc{obl}-\textsc{dat}\\
	\glt	\sqt{What happened to her?}

	\ex	\label{He and they, his friends also meet and sit together.}
	\gll	heχ=ra	heχ-tːi	heχ-tːi=ra	ču-la	juldašː-e	ca-b	guči<b>ič-ib-le	ka-b-iž-ib-le\\
		\textsc{dem.down}=\textsc{add}	\textsc{dem.down}-\textsc{pl}	\textsc{dem.down}-\textsc{pl}=\textsc{add}	\textsc{refl}.\textsc{pl}-\textsc{gen}	friend-\textsc{pl}	\textsc{cop-hpl}	meet<\textsc{hpl}>.\textsc{pfv-pret-cvb}	\textsc{down}-\textsc{hpl}-be-\textsc{pret}-\textsc{cvb}\\
	\glt	\sqt{He and they, his friends also meet and sit together.}

	\ex	\label{(They) stole it, whatever it is, this, their watermelons.}
	\gll	b-iʡ-uˁn ca-b	heχ,	ce	ca-b=el,	heχ,	heχ-tː-a-la	qːalpuz-e=w \\
		\tsc{n-}steal.\textsc{pfv}-\textsc{pret} \textsc{cop-n}	\textsc{dem.down}	what	\textsc{cop-n}=\textsc{indq}	\textsc{dem.down}	\textsc{dem.down}-\textsc{pl}-\textsc{obl}-\textsc{gen}	watermelon-\textsc{pl}=\textsc{q}\\
	\glt	\sqt{(They) stole it, whatever it is, this, their watermelons.}
\end{exe}

In the discourse deictic function, mostly the \textit{χ}-pronouns occur \refex{This is the end (of the story).}, but the \textit{k'-/x}-pronouns can also occasionally be found.

\begin{exe}
	\ex	\label{This is the end (of the story).}
	\gll	taman	ca-b	heχ \\
		end	\textsc{cop-n}	\textsc{dem.down}\\
	\glt	\sqt{This is the end (of the story).}
\end{exe}


Yet elevation cannot be the only criterion that governs the use of the \textit{χ}-pronouns vs. the \textit{k'-/x}-pronouns. For instance, in \refex{I (masc.) say, they have a (little) piece (of a diamond) like this, while I have one like this} the first \isi{demonstrative pronoun} denotes people who the speaker has seen on TV. They are described as being located lower than some unknown point of reference. At the same time the sentence is a good example for the contrast between \textit{iC}-pronouns and \textit{heC}-pronouns as discussed in \refsec{ssec:The vertical dimension: iC vs. heC vs. hiC and i(C)tːi vs. he(C)tːi vs. hi(C)tːi}. 

\begin{exe}
	\ex	\label{I (masc.) say, they have a (little) piece (of a diamond) like this, while I have one like this}
	\gll	du	∅-ik'-ul=da,	iχ-tː-a-la	ij=ʁuna	but'a	ca-b,	di-la	hej=ʁuna=qːel \\
		\tsc{1sg}	\tsc{m-}say\tsc{.ipfv-icvb=1}	\tsc{dem.down}\tsc{-pl-obl-gen}	this\tsc{=eq}	piece	\tsc{cop-n}	\tsc{1sg-gen}	this\tsc{=eq=}when\\
	\glt	\sqt{I (masc.) say, they have a (little) piece (of a diamond) like that, while I have one like that.}
\end{exe}

For a more thorough discussion of the elevational meaning in Sanzhi demonstratives and examples see the detailed account in \citet{ForkerLTSanzhi}.\pagebreak
% - - - - - - - - - - - - - - - - - - - - - - - - - - - - - - - - - - - - - - - - - - - - - - - - - - - - - - - - - - - - - - - - - - - - - - - - - - - - - - - - - - - - - - - - - - - - - - - - - - - - - - - - - - - - - - - - - - - - - - - - - - %

\subsection{Pronouns and adverbs derived from demonstrative pronouns}
\label{ssec:Pronouns and adverbs derived from demonstrative pronouns}

There are a few pro-forms derived from the \is{demonstrative pronoun}demonstrative pronouns such as \tit{hetilil} \refex{He left her and went to that, he left her and when to the next.}, \tit{itilil} \refex{ex:thehorseflyandtheman} \sqt{other, next}, \tit{itil-ižili} \sqt{one thing and another}, and \tit{it-heχ} \sqt{this and that, various}.

\begin{exe}
	\ex	{[He pointed with his finger on his forehead and said to the other man, Well here is the horsefly.]} \label{ex:thehorseflyandtheman}
	\sn
	\gll	itilil-li	ix-ub-le	tupang	antːa-l-cːe	zija=ra	kax-ub	ca-b	il	admi=ra\\
		other-\tsc{erg}	throw\tsc{.pfv-pret-cvb}	weapon	forehead\tsc{-obl-in} horsefly\tsc{=add}	kill\tsc{.pfv-pret}	\tsc{cop-n}	that	person\tsc{=add}\\
	\glt	\sqt{The other shot at the forehead and killed the horsefly and the man.}
\end{exe}

The temporal/clausal adverb \tit{helij} \sqt{therefore} is diachronically the \isi{dative} case form of the pronoun \tit{hel} \sqt{that}.

There is a series of \isi{manner adverbs} with the meaning \sqt{like this\slash like that} that is formed by adding the suffix \tit{-itːe} to the singular \is{demonstrative pronoun}demonstrative pronouns (except for the \tit{j-} pronouns, since they are mostly used in the noun-modifying function, see \reftab{tab:Manner adverbs derived from demonstrative pronouns}). Their meaning is based on the meanings of the \is{demonstrative pronoun}demonstrative pronouns as described in the preceding sections. Some examples can be found in \refex{ex:It must be done like this}, \refex{ex:But now you did not say this} (see also \refsec{sec:MannerAdverbs} for more examples).

\begin{table}
	\caption{Manner adverbs derived from demonstrative pronouns}
	\label{tab:Manner adverbs derived from demonstrative pronouns}
	\small
	\begin{tabularx}{1\textwidth}[]{*{3}{>{\itshape}l}Q}
		\lsptoprule
		\multicolumn{1}{c}{\tit{iC}}	&	\multicolumn{1}{c}{\tit{heC}}	&	\multicolumn{1}{c}{\tit{hiC}}\\
		\midrule
		\tit{iž-itːe}	&	\tit{hež-itːe}	&	\tit{hiž-itːe}	&	\sqt{like this, like something close to the speaker}\\   
		\tit{il-itːe}	&	\tit{hel-itːe}	&	\tit{hil-itːe} 	&	\sqt{like that, like something away from the speaker and/or close to the hearer}\\
		\tit{it-itːe}	&	\tit{het-itːe}	&	\tit{hit-itːe}	&	\sqt{like that, like something away from speaker and hearer or undifferentiated}\\
		\tit{ik'-itːe}	&	\tit{hek'-itːe}	&	\tit{hik'-itːe}	&	\sqt{like this/that above the deictic center}\\
		\tit{iχ-itːe}	&	\tit{heχ-itːe}	&	\tit{hiχ-itːe}	&	\sqt{like this/that below deictic center}\\
		\lspbottomrule
	\end{tabularx}
\end{table}

\begin{exe}
	\ex	\label{ex:It must be done like this}
	\gll	iž	hež-itːe	b-irq'-an ca-b\\
		this	this\tsc{-advz}	\tsc{n-}do\tsc{.ipfv-ptcp} \tsc{cop-n}\\
	\glt	\sqt{It must be done like this.}

	\ex	\label{ex:But now you did not say this}
	\gll	hana=q'ar	il-itːe	a-haʔ-ib=de\\
		now\tsc{=mod}	that\tsc{-advz}	\tsc{neg-}say\tsc{.pfv-pret=2sg}\\
	\glt	\sqt{But now you did not say this.} (i.e. you did not tell the stories that you told the other time)
\end{exe}

There is another group of four \isi{manner adverbs} with a similar meaning as the adverbs ending in \tit{-itːe}, namely \tit{itwaj}, \tit{hetwaj}, \tit{hitwaj}, and \tit{ižwaj} \sqt{like that, and so}. Their usage is illustrated in \xxref{ex:They are also like this, like Russians, even now}{ex:Uh, and so he did not marry}.

\begin{exe}
	\ex	\label{ex:They are also like this, like Russians, even now}
	\gll	itːi	itwaj=ra	ʡuˁrusː-e	ʁunab-te	ca-b	hana=ra\\
		those	like.that\tsc{=add}	Russian\tsc{-pl}	\tsc{eq-dd.pl} 	\tsc{cop-hpl}	now\tsc{=add}\\
	\glt	\sqt{They are also like this, like Russians, even now.}

	\ex	\label{ex:(He) is my (real) grandfather. For you he is only an old man (lit. he is like a grandfather of yours)}
	\gll	di-la	χatːaj	ca-w,	ala	itwaj	χatːaj	ca-w\\
		\tsc{1sg-gen}	grandfather	\tsc{cop-m}	\tsc{2sg.gen}	like.that	grandfather	\tsc{cop-m}\\
	\glt	\sqt{(He) is my (real) grandfather. For you he is only an old man.} (lit. \sqt{he is like a grandfather of yours})

	\ex	\label{ex:Uh, and so he did not marry}
	\gll	ha	itwaj	ka-b-iž-ib-te=kːu\\
		uh	like.that	\tsc{down-hpl-}be\tsc{.pfv-pret-dd.pl=cop.neg}\\
	\glt	\sqt{Uh, and so they did not marry.}
\end{exe}

Spatial adverbs with the basic meaning \sqt{here, there} are derived by adding the suffix \tit{-tːu} to the pronominal stems (\reftab{tab:Spatial adverbs derived from demonstrative pronouns}). The meaning of the \is{spatial adverb}spatial adverbs is transparently derived from the meaning of the demonstratives. As can be seen in the table, there are two series of \is{spatial adverb}spatial adverbs with the meaning \sqt{there above} because both the singular as well as the plural pronominal stem can serve as the base for the \isi{derivation}, but the adverbs with \tit{x} are far more common than the adverbs with \tit{k'}. The adverbs with \tit{k'} are very rarely used and might even be switches to another dialect of Dargwa.

\begin{table}
	\caption{Spatial adverbs derived from demonstrative pronouns}
	\label{tab:Spatial adverbs derived from demonstrative pronouns}
	\small
	\begin{tabular}{*{3}{>{\itshape}l}l}
		\lsptoprule
		\multicolumn{1}{l}{\tit{i(C)tːu}}	&	\multicolumn{1}{l}{\tit{he(C)tːu}}	&	\multicolumn{1}{l}{\tit{hi(C)tːu}}\\
		\midrule
		iš-tːu 		&	heš-tːu	&	hiš-tːu 	&	\sqt{here, close to the speaker}\\
		il-tːu		&	hel-tːu	&	hil-tːu		&	\sqt{there, away from the speaker and\slash or close to the hearer}\\
		i-tːu 		&	he-tːu 	&	hi-tːu		&	\sqt{there, further away, unspecific distance}\\
		ik'-tːu		&	hek'-tːu	&	hik'-tːu	&	\sqt{here/there above the deictic center}\\
		ix-tːu		&	hex-tːu	&	hix-tːu	&	\sqt{here/there above the deictic center}\\
		iχ-tːu		&	heχ-tːu	&	hiχ-tːu	&	\sqt{here/there below the deictic center}\\
		\lspbottomrule
	\end{tabular}
\end{table}

\largerpage Since the adverbs have inherent spatial semantics, locational cases cannot be added, but only directional suffixes just as with other spatial adverbials or nominals. The lative is zero-marked, the essive is expressed through \isi{gender}/\isi{number} agreement, the \isi{ablative} by means of the suffix \tit{-r(ka)} and the directive through the suffix \tit{\tsc{-gm}-a} including a \isi{gender}/\isi{number} agreement marker, e.g. lative \tit{heš-tːu}, essive \tit{heš-tːu-b}, \isi{ablative} \tit{heš-tːu-r(ka)}, directive \tit{heš-tːu-b-a}. Examples can be found in {ex:We crossed the border between Shurli and our (Sanzhi area), and up there we found a stone}{ex:Do they really allow people from here (to enter) the hospital}. More examples are given in \refsec{ssec:SpatialAdverbsDerivedFromDemonstrativePronouns}.

\begin{exe}
	\ex	\label{ex:We crossed the border between Shurli and our (Sanzhi area), and up there we found a stone}
	\gll	šuˁrʡli-la=ra	nišːa-la=ra	dazu-la	hetːu-r	tːura	d-ituq-un-ne	hek'tːu-b	b-arčː-ib-il=de	ca	qːarqːa\\
		Shurli-\tsc{gen=add}	\tsc{1pl-gen=add}	border\tsc{-gen}	there\tsc{-abl}	outside	\tsc{1/2pl-}cross\tsc{.pfv-pret-cvb}	there.\tsc{up-n}	\tsc{n-}find\tsc{.pfv-pret-ref=pst}	one	stone\\
	\glt	\sqt{We crossed the border between Shurli and our (Sanzhi area), and up there we found a stone.}

	\ex	\label{ex:Were there sheep up there}
	\gll	macːa	d-irχʷ-i=w	ixtːu-d?\\
		sheep	\tsc{npl-}be\tsc{.ipfv-hab.pst=q}	there.\tsc{up-npl}\\
	\glt	\sqt{Were there sheep up there?}

	\ex	\label{ex:They are walking on (along) the road there, right, these}
	\gll	iš-tːi	xːune-r	hitːu-b-a	arg-ul akːu=w	iš-tːi?\\
		this\tsc{-pl}	road\tsc{.spr-abl}	there\tsc{-hpl-dir}	go\tsc{.ipfv-icvb} \tsc{cop.neg=q}	this\tsc{-pl}\\
	\glt	\sqt{Are they walking on (along) the road there, right, these?}

	\ex	\label{ex:Do they really allow people from here (to enter) the hospital}
	\gll	balnicːa-le	b-i-b-aš-aq-u=w	ištːu-rka	ag-ur	χalq'\\
		hospital\tsc{-loc}	\tsc{hpl-in-hpl-}go\tsc{.ipfv-caus-prs=q}	here\tsc{-abl}	go\tsc{.pfv-pret}	people\\
	\glt	\sqt{Do they really allow people from here (to enter) the hospital?!}
\end{exe}

Another series of \is{spatial adverb}spatial adverbs denoting the source can be derived by means of the suffix \tit{-ka} (which is probably a cognate of the second part of the complex \isi{ablative} suffix \tit{-r-ka}), e.g. \tit{hež-ka} \sqt{from here}, \tit{hel-ka} \sqt{from there}, etc. (\refsec{ssec:Functions of semantic cases}). These adverbs can also have a temporal interpretation (`from time X on'). Moreover, there is a series of \is{spatial adverb}spatial adverbs with the meaning \sqt{from X to X} containing the suffix \tit{-k-itːu-b-a}, e.g. \tit{hež-kitːu-b-a} \sqt{from here to there} \refex{ex:(The dog) ran away to that side (from there to there)}. This suffix is a combination of the \isi{ablative} \tit{-ka} (shortened to \tit{-k}), the locational suffix \tit{-tːu} and the directive \tit{-\tsc{gm}-a}. Both series are available from all three stem types of demonstratives (\tit{heC,} \tit{iC,} and \tit{hiC}), but only the adverbs based on \tit{heC} are commonly used in my corpus. See \refsec{ssec:SpatialAdverbsDerivedFromDemonstrativePronouns} for Tables displaying all adverbs and more examples. 

\begin{exe}
	\ex	\label{ex:From here the scandal happened}
	\gll	qːalmaqːar	ag-ur-te	hej-ka	ca-d\\
		scandal	go\tsc{.pfv-pret-dd.pl} 	this\tsc{-abl}	\tsc{cop-npl}\\
	\glt	\sqt{From here the scandal happened.}

	\ex	\label{ex:Well we went from up there along the upper side}
	\gll	nu,	ik-ka	nušːa	qari-rka	ag-ur=da\\
		well	\tsc{dem.up}\tsc{-abl}	\tsc{1pl}	up\tsc{-abl}	go\tsc{.pfv-pret=1}\\
	\glt	\sqt{Well we went from up there along the upper side.}

	\ex	\label{ex:(The dog) ran away to that side (from there to there)}
	\gll	hana	hetkitːu-b-a	b-ibšː-ib\\
		now	from.there.to.there\tsc{-n-dir}	\tsc{n-}escape\tsc{-pret}\\
	\glt	\sqt{(The dog) ran away to that side (from there to there).}
\end{exe}
%
The equative \isi{enclitic} \tit{=ʁuna} \sqt{like, similar} and the \isi{temporal enclitic} \tit{=qːel} \sqt{when} can also be attached to the \is{demonstrative pronoun}demonstrative pronouns leading to pro-forms used when comparing referents \refex{He gave me such a bottle and sent me away}, \refex{A forest like this exists there, the mill’s forest it is called} and \is{temporal adverb}temporal adverbs with the meaning \sqt{then, at this/that time} \refex{At that time this hand (of mine) was still working}.


% --------------------------------------------------------------------------------------------------------------------------------------------------------------------------------------------------------------------- %

\section{Reflexive pronouns}
\label{sec:Reflexive pronouns}

Sanzhi Dargwa has simple \is{reflexive pronoun}reflexive pronouns (\reftab{tab:Simple reflexive pronouns}) and two types of complex \is{reflexive pronoun}reflexive pronouns (\reftab{tab:Complex reflexive pronouns}). In \is{reflexive construction}reflexive constructions, the \is{reflexive pronoun}reflexive pronouns refer only to third persons. For first and second person reflexivization personal pronouns are used. Reflexive pronouns are marked for \isi{gender} (in the \isi{absolutive} only), for \isi{number} and for case. The \isi{absolutive} case of the \isi{reflexive pronoun} is identical to the \isi{copula} and might be diachronically related to it. For all other cases the pronoun has two stems (singular and plural).

\begin{table}
	\caption{Simple reflexive pronouns}
	\label{tab:Simple reflexive pronouns}
	\small
	\begin{tabularx}{0.6\textwidth}[]{%
		>{\raggedright\arraybackslash}p{56pt}
		>{\raggedright\arraybackslash\itshape}X
		>{\raggedright\arraybackslash\itshape}X}
		
		\lsptoprule
		{}			&	\emph{singular}	&	\emph{plural}\\
		\midrule
		\isit{absolutive}		&	ca-w /-r /-b	&	ca-b /-d\\
		\isit{ergative}		&	cin-ni		&	ču-l\\
		\isit{genitive}		&	cin-na		&	ču-la\\
		\isit{dative}			&	cini-j		&	ču-j\\
		\isit{comitative}		&	cini-cːella	&	ču-cːella\\
		\tsc{ad}-lative	&	cini-šːu	&	ču-šːu\\
		\tsc{in}-lative 	&	cini-cːe	&	ču-cːe\\
		\tsc{loc}-lative	&	ci-ne		&	ču-ja\\
		\lspbottomrule
	\end{tabularx}
\end{table}

\begin{table}
	\caption{Complex reflexive pronouns}
	\label{tab:Complex reflexive pronouns}
	\small
	\begin{tabularx}{1\textwidth}[]{%
		>{\raggedright\arraybackslash}p{28pt}
		>{\raggedright\arraybackslash\itshape}X
		>{\raggedright\arraybackslash\itshape}X
		>{\raggedright\arraybackslash\itshape}X
		>{\raggedright\arraybackslash\itshape}X}
		
		\lsptoprule
		{}		&	\multicolumn{2}{c}{singular}		&	\multicolumn{2}{c}{plural}\\\cmidrule(lr){2-3}\cmidrule(lr){4-5}
		{}		&	\multicolumn{1}{l}{case copying\footnote{with \isi{ergative} controller}}	&	\multicolumn{1}{l}{\isit{genitive} refl.}
				&	\multicolumn{1}{l}{case copying\textsuperscript{\itshape a}}	&	\multicolumn{1}{l}{\isit{genitive} refl.}\\
		\midrule
		\textsc{abs}		&	cinni ca-w /-r /-b	&	\mbox{cinna ca-w /-r /-b}	&	čul ca-b /-d		&	čula ca-b /-d\\
		\textsc{erg}		&	\tmd			&	cinna cin-ni		&	\tmd			&	čula čul\\
		\textsc{gen}		&	cinni cin-na		&	\tmd			&	čul čula		&	\tmd\\
		\textsc{dat}		&	cinni cini-j		&	cinna cini-j		&	čul ču-j		&	čula ču-j\\
		\textsc{comit}		&	cinni cini-cːella	&	cinna cini-cːella	&	čul ču-cːella		&	čula ču-cːella\\
		\lspbottomrule
	\end{tabularx}
\end{table}

The simple \is{reflexive pronoun}reflexive pronouns occur in local and non-\isi{local reflexivization} (including logophoric contexts across clau\-sal boun\-daries, whereas the complex \is{reflexive pronoun}reflexive pronouns can only be bound within the clause. Both types of complex \is{reflexive pronoun}reflexive pronouns consist of a reduplicated form of the simple reflexive (\reftab{tab:Simple reflexive pronouns}). For the first variant of the complex \is{reflexive pronoun}reflexive pronouns, one part of the reflexive undergoes case-copying from the controller (in \reftab{tab:Complex reflexive pronouns} exemplified with an \isi{ergative} controller), and the second part takes the appropriate case-marking. In the second variant, the first part is invariably \isi{genitive}. The second variant, the complex \isi{genitive} reflexive, lacks a form for the \isi{genitive} case, so it can never occur as possessor. Other functions in addition to local and non-\isi{local reflexivization} are: emphatic reflexivization, \isi{comitative} constructions and pause fillers.

All types of \is{reflexive construction}reflexive constructions are analyzed in more detail in \refsec{sec:Reflexive constructions} and in \citet{Forker2014}. The \isi{genitive} singular and plural \is{reflexive pronoun}reflexive pronouns \tit{cinna} and \tit{čula} are used as pause fillers (\refsec{sec:Pause fillers, address particles, exclamatives, and interjections}). The \isi{absolutive} \is{reflexive pronoun}reflexive pronouns occur in \isi{comitative} constructions that have the formal structure of coordinated noun phrases (\refsec{sec:Comitative constructions}).

None of these additional functions are available for complex \is{reflexive pronoun}reflexive pronouns, which occur only in \isi{local reflexivization}, emphatic reflexivization and \isi{reciprocal constructions} (only plural \is{reflexive pronoun}reflexive pronouns).


% --------------------------------------------------------------------------------------------------------------------------------------------------------------------------------------------------------------------- %

\section{Reciprocal pronouns}
\label{sec:Reciprocal pronouns}

Reciprocal pronouns are very similar to complex \is{reflexive pronoun}reflexive pronouns in form as well as in morphosyntactic behavior. They consist of a reduplicated form of the numeral \tit{ca} \sqt{one}. Sanzhi Dargwa has three types of \is{reciprocal pronoun}reciprocal pronouns. Two of these pronouns always consist of the reduplicated numeral \tit{ca} \sqt{one}. Except for the \isi{genitive} they fully inflect for case, but do not distinguish \isi{gender}. One type of \is{reciprocal pronoun}reciprocal pronouns is the equivalent of the \isi{genitive} reflexive because its first part is always in the \isi{genitive}. The second reciprocal has always one part in the \isi{absolutive}. The third variant, \tit{ca-b-a}, is also based on \tit{ca} \sqt{one}, to which a plural suffix that exhibits \isi{gender}/\isi{number} agreement is added. It can also be reduplicated (this is not shown in the Table) and inflects for all cases. All reciprocals are shown in the partial paradigm in \reftab{tab:Reciprocal pronouns}. In addition, the language also makes use of plural \is{reflexive pronoun}reflexive pronouns (\reftab{tab:Complex reflexive pronouns}) for the expression of reciprocity.

Syntactically, \is{reciprocal pronoun}reciprocal pronouns behave similarly to complex reflexives because they are always locally bound. More information on reciprocalization can be found in \refsec{sec:Reciprocal constructionss} and in \citet{Forker2014}. 


\begin{table}
	\caption{Reciprocal pronouns}
	\label{tab:Reciprocal pronouns}
	\small
	\begin{tabularx}{1\textwidth}[]{%
		>{\raggedright\arraybackslash}p{56pt}
		>{\raggedright\arraybackslash\itshape}X
		>{\raggedright\arraybackslash\itshape}X
		>{\raggedright\arraybackslash\itshape}X}
		
		\lsptoprule
		{}			&	\normalfont\sqt{each other}\linebreak(\isit{genitive} variant)	&	\normalfont\sqt{each other}\linebreak\upshape(\isit{absolutive} variant) &	\normalfont\sqt{each other}\\
		\midrule
		\isit{absolutive}		&	calla ca		&	calli ca 		&	ca-b-a\\
		\isit{ergative}		&	calla ca-l-li		&	calli ca			&	ca-b-a-li\\
		\isit{genitive}		&	calla calla			&	ca-l-la ca		&	ca-b-a-la\\
		\isit{dative}			&	calla ca-l-li-j 		&	ca-l-li-j ca 		&	ca-b-a-li-j\\
		\isit{comitative}		&	calla ca-l-li-cːella	&	ca-l-li-cːella ca 	&	ca-b-a-li-cːella\\
		\tsc{ad}-lative	&	calla ca-l-li-šːu	&	ca-l-li-šːu ca		&	ca-b-a-li-šːu\\
		\tsc{in}-lative 	&	calla ca-l-li-cːe	&	ca-l-li-cːe ca		&	ca-b-a-li-cːe\\
		\tsc{loc}-lative 	&	calla ca-l-le		&	ca-l-le ca 		&	ca-b-a-l-le\\
		\lspbottomrule
	\end{tabularx}
\end{table}



% --------------------------------------------------------------------------------------------------------------------------------------------------------------------------------------------------------------------- %

\section{Interrogative pronouns}
\label{sec:Interrogative pronouns}

The interrogative pronouns of Sanzhi are given in \reftab{tab:Interrogative pronouns}. Some of the pronouns are morphologically complex, consisting of the root \textit{ce} \sqt{what} to which other morphemes are added:\pagebreak

\begin{itemize}
	\item	\tit{ce} + \tit{t'le}: the second part might contain the \isi{adverbializer} \tit{-le}
	\item	\tit{ce} + \tit{ʁuna}: the second part is the equative \isi{enclitic} \tit{=ʁuna} \sqt{like, as} (\refsec{sec:Equative constructions and the expression of similarity})
	\item	\tit{ce} + \tit{li-j}: the second part is the inflection for \isi{dative} case (\refsec{sssec:Dative})
	\item \tit{ce} + \textit{l}: the second part is the inflection for \isi{ergative} case (\refsec{sssec:Ergative})
	\item	\tit{ce} + \tit{qːel}: the second part is the \isi{temporal enclitic} \tit{=qːel} \sqt{when, while, at that time} (\refsec{sec:enclitic =qella})
\end{itemize}
%


\begin{table}
	\caption{Interrogative pronouns}
	\label{tab:Interrogative pronouns}
	\small
	\begin{tabular}{ll@{\hspace{4em}}ll@{\hspace{4em}}ll}
		\lsptoprule
		\tit{ča}	&	\sqt{who}	&	\tit{ceʁuna}	&	\sqt{which}	&	\tit{ceqːel}	&	\sqt{when}\\
		\tit{ce}	&	\sqt{what}	&	\tit{kutːi}	&	\sqt{which}	&	\tit{čujna}	&	\sqt{how many times}\\
		\tit{čina}	&	\sqt{where}	&	\tit{cel}	&	\sqt{why}	&	\tit{kusa}	&	\sqt{how much}\\
		\tit{cet'le}	&	\sqt{how}	&	\tit{celij}	&	\sqt{why}	&	\tit{čum}	&	\sqt{how many}\\
		\lspbottomrule
	\end{tabular}
\end{table}

The pronouns \tit{čujna} and \tit{čum} are also complex. They seem to contain the same root \mbox{\tit{ču-}.} In order to arrive at \tit{ču-jna} the derivational suffix \tit{-na} (allomorph \mbox{\tit{-jna}} after vowels) has been added. This suffix is also used to form \isi{multiplicative numerals} (\refsec{sec:multiplicativenumerals}). The pronoun \tit{kutːi} seems to be composed of a root \tit{ku-} and an ending \tit{-tːi}, the latter also found with plural \is{demonstrative pronoun}demonstrative pronouns (\refsec{sec:Demonstrative pronouns}).

In the following, all pronouns are described and illustrated with examples. More information on \is{interrogative clause}interrogative clauses can be found in \refcpt{cpt:Interrogative clauses}. Embedded interrogatives are treated in \refsec{sec:Subordinate questions}.


% - - - - - - - - - - - - - - - - - - - - - - - - - - - - - - - - - - - - - - - - - - - - - - - - - - - - - - - - - - - - - - - - - - - - - - - - - - - - - - - - - - - - - - - - - - - - - - - - - - - - - - - - - - - - - - - - - - - - - - - - - - %

\subsection{\tit{ča} \sqt{who} and \tit{ce} \sqt{what}}
\label{ssec:ca who and ce what}

Partial inflectional paradigms of the pronouns \tit{ča} \sqt{who} and \tit{ce} \sqt{what} are shown in \reftab{tab:Interrogative pronouns ca who and ce what}. The pronoun \tit{ča} has a suppletive stem \tit{hi-} for all cases except for the \isi{absolutive} \refex{ex:From whose clan was he}. The pronoun \tit{ča} can be used as a modifier to a nominal with human reference and translates then as \sqt{which, what kind of} \refex{ex:Which Khamis? Sutajs's sister}. It can be marked for plural by means of the associative plural suffix \tit{-qal} \refex{ex:Nice, who were they? he asks the Russian woman} (\refsec{ssec:MorphophonologicalrulesPlural}).

\begin{table}
	\caption{Interrogative pronouns \tit{ča} \sqt{who} and \tit{ce} \sqt{what}}
	\label{tab:Interrogative pronouns ca who and ce what}
	\small
	\begin{tabularx}{0.52\textwidth}[]{%
		>{\raggedright\arraybackslash}p{56pt}
		>{\raggedright\arraybackslash\itshape}X
		>{\raggedright\arraybackslash\itshape}X}
		
		\lsptoprule
		{}			&	\normalfont\sqt{who}	&	\normalfont\sqt{what}\\
		\midrule
		\isit{absolutive}		&	ča		&	ce\\
		\isit{ergative}		&	hi-l		&	ce-l-li\\
		\isit{genitive}		&	hi-la		&	ce-lla\\
		\isit{dative}			&	hi-j		&	ce-lli-j\\
		\isit{comitative}		&	hi-cːella	&	ce-lli-cːella\\
		\tsc{ad}-lative	&	hi-šːu		&	ce-lli-šːu\\
		\tsc{in}-lative 	&	hi-cːe		&	ce-lli-cːe\\
		\tsc{loc}-lative	&	hi-ja		&	ce-l-le\\
		\lspbottomrule
	\end{tabularx}
\end{table}

\begin{exe}
	\ex	\label{ex:Which Khamis? Sutajs's sister}
	\gll	ča 	χamis?		Sut'aj-la	rucːi\\
		who	Khamis	Sutaj\tsc{-gen}	sister\\
	\glt	\sqt{Which Khamis? Sutaj's sister.}

	\ex	\label{ex:Nice, who were they? he asks the Russian woman}
	\gll	``čakːʷa-l,	ča-qal=de?''	∅-ik'-ul	ca-w	het	ʡuˁrus	xːunul-li-cːe\\
		handsome\tsc{-advz}	who\tsc{-assoc=pst}	\tsc{m-}say\tsc{.ipfv-icvb}	\tsc{cop-m}	that	Russian	woman\tsc{-obl-in}\\
	\glt	\sqt{``Nice, who were they?'' he asks the Russian woman.}

	\ex	\label{ex:From whose clan was he}
	\gll	il	hi-la	q'am-la=de?\\
		that	who\tsc{-gen}	kin\tsc{-gen=pst}\\
	\glt	\sqt{From whose clan was he?}
\end{exe}

The pronoun \tit{ce} \sqt{what} \refex{ex:What do you (masc.) say} can also be used with the meanings \sqt{how} \refex{ex:How do you know who they are}, \sqt{where} \refex{ex:(The picture on which the people) run away, where is it} and, when functioning as a nominal modifier, \sqt{which, what kind of}. The \isi{dative} case of this pronoun \textit{celij} translates as \sqt{why} (\refsec{sssec:cel and celij why}).

\begin{exe}
	\ex	\label{ex:What do you (masc.) say}
	\gll	ce	∅-ik'-ul=de	u?\\
		what	\tsc{m-}say\tsc{.ipfv-icvb=2sg}	\tsc{2sg}\\
	\glt	\sqt{What do you (masc.) say?}

	\ex	\label{ex:How do you know who they are}
	\gll	ce	b-alχ-ul=de	ča-qal=el?\\
		what	\tsc{hpl-}know\tsc{.ipfv-icvb=2sg}	who\tsc{-assoc=indq}\\
	\glt	\sqt{How do you know who they are?}

	\ex	\label{ex:(The picture on which the people) run away, where is it}
	\gll	sa-r-b-ulq-an	ce	b-iχ-ub=e?\\
		\tsc{ante-abl-hpl-}direct\tsc{.ipfv-ptcp}	what	\tsc{n-}be\tsc{.pfv-pret=q}\\
	\glt	\sqt{(The picture on which the people) run away, where is it?}
\end{exe}


% - - - - - - - - - - - - - - - - - - - - - - - - - - - - - - - - - - - - - - - - - - - - - - - - - - - - - - - - - - - - - - - - - - - - - - - - - - - - - - - - - - - - - - - - - - - - - - - - - - - - - - - - - - - - - - - - - - - - - - - - - - %

\subsection{Other interrogative words}
\label{ssec:Other interrogative pronouns}


% - - - - - - - - - - - - - - - - - - - - - - - - - - - - - - - - - - - - - - - - - - - - - - - - - - - - - - - - - - - - - - - - - - - - - - - - - - - - - - - - - - - - - - - - - - - - - - - - - - - - - - - - - - - - - - - - - - - - - - - - - - %

\subsubsection{\tit{čina} \sqt{where}}\label{sssec:cina where}\largerpage

This pronoun has an inherent spatial meaning and can be further inflected for the directional cases just like other nominals or adverbials with spatial meaning, see \refsec{ssec:Functions of semantic cases}. Thus, we obtain:

\begin{itemize}
	\item	the zero-marked lative \tit{čina} for directed motion \refex{ex:Are you (masc.) asking where he went}
	\item	the essive \tit{čina-b} for location (with the \isi{gender}/\isi{number} agreement suffix) \refex{ex:Where was he}
	\item	the \isi{ablative} \tit{čina-r-(ka)} for movement from a source or through\slash along a reference point or with the meaning \sqt{how} (lit. \sqt{from where}) \refex{ex:How did you get to know them (= the medical plants)}
\end{itemize}

\begin{exe}
	\ex	\label{ex:Are you (masc.) asking where he went}
	\gll	čina	ag-ur-re	∅-ik'-ul=de	u?\\
		where	go\tsc{.pfv-pret-cvb}	\tsc{m-}say\tsc{.ipfv-icvb=2sg}	\tsc{2sg}\\
	\glt	\sqt{Are you (masc.) asking where he went?}

	\ex	\label{ex:Where was he}
	\gll	čina-w=de	it?\\
		where\tsc{-m=pst}	that\\
	\glt	\sqt{Where was he?}

	\ex	\label{ex:How did you get to know them (= the medical plants)}
	\gll	at	čina-r	d-aχ-ur=de	hel-tːi?\\
		\tsc{2sg.dat}	where\tsc{-abl}	\tsc{npl-}know\tsc{.pfv-pret=pst}	that\tsc{-pl}\\
	\glt	\sqt{How did you get to know them (= the medical plants)?}
\end{exe}

It can also take the \isi{genitive} suffix, then denoting origin in the sense of ethnic descent \refex{ex:Where is this person from? (He) is from Georgia (i.e. he is Georgian)}:

\begin{exe}
	\ex	\label{ex:Where is this person from? (He) is from Georgia (i.e. he is Georgian)}
	\gll	čina-la	admi=ja	iž?	gurži-le-r	/	gurži-la	ca-w\\
		where\tsc{-gen}	person\tsc{=q}	this	Georgia\tsc{.obl-loc-abl}	/	Georgia\tsc{.obl-gen}	\tsc{cop-m}\\
	\glt	\sqt{Where is this person from? (He) is from Georgia (i.e. he is Georgian).} (E)
\end{exe}


% - - - - - - - - - - - - - - - - - - - - - - - - - - - - - - - - - - - - - - - - - - - - - - - - - - - - - - - - - - - - - - - - - - - - - - - - - - - - - - - - - - - - - - - - - - - - - - - - - - - - - - - - - - - - - - - - - - - - - - - - - - %

\subsubsection{\tit{cet'le} \sqt{how}}
\label{sssec:cetle how}

The pronoun \tit{cet'le} refers to the manner of action.

\begin{exe}
	\ex	\label{ex:How can rain fall there? They are inside a cave}
	\gll	marka	cet'le	či-b-irʁ-ul=e	ixtːu?	neqːe-d	d-i-d=q'al	itːi\\
		rain	how	\tsc{spr-n-}come\tsc{.ipfv-icvb=q}	there.\tsc{up}	cave\tsc{.loc-npl}	\tsc{npl-in}\tsc{-npl=prt}	those\\
	\glt	\sqt{How can rain fall there? They are inside a cave.}

	\ex	\label{ex:How is she studying? She is studying well}
	\gll	cet'le il	r-uč'-unne?	iž ʡaˁħ-le r-uč'-un ca-r\\
		how	this	\tsc{f-}learn\tsc{.ipfv-icvb}	this	good\tsc{-advz}	\tsc{f-}learn\tsc{.ipfv-icvb} \tsc{cop-f}\\
	\glt	\sqt{How is she studying? She is studying well.} (E)
\end{exe}


% - - - - - - - - - - - - - - - - - - - - - - - - - - - - - - - - - - - - - - - - - - - - - - - - - - - - - - - - - - - - - - - - - - - - - - - - - - - - - - - - - - - - - - - - - - - - - - - - - - - - - - - - - - - - - - - - - - - - - - - - - - %

\subsubsection{\tit{kutːi} and \tit{ceʁuna} \sqt{which}}
\label{sssec:kuti and ceruna which}

The pronoun \tit{kutːi} asks for the indication of a specific item among a group of items. For instance, the first speaker in \refex{ex:This picture does not fit here. Which} wants to indicate to his interlocutor a picture that does not fit into a picture story. The second speaker does not understand to which of the pictures the first speaker is referring, and asks for clarification. It can be used as an \isi{indefinite pronoun} and then be inflected for various cases \refex{ex:The men did not know which was whose}.

\begin{exe}
	\ex	\label{ex:This picture does not fit here. Which}
	\gll	hež	sːurrat	heštːu	b-al	b-ič-ib-le	akːu. kutːi?\\
		this	picture	here	\tsc{n-}fit	\tsc{n-}occur\tsc{.pfv-pret-cvb}	\tsc.{cop.neg}	which\\
	\glt	\sqt{This picture does not fit here. Which?}

	\ex	\label{ex:Which (picture) goes behind which, which goes in front, which goes behind, take a look}
	\gll	kutːi	arg-ul=el	kutːi-l-li	hara	hitːi,	kutːi	arg-ul=el kutːi	sala-b=el,	kutːi	hila-b=el,		er d-irq'-aj!\\
		which	go\tsc{.ipfv-icvb=indq}	which\tsc{-obl-erg}	behind	after	which	go\tsc{.ipfv-icvb=indq}	which	front\tsc{-n=indq}		which	behind\tsc{-n=indq}		look \tsc{npl}-do.\tsc{ipfv-imp.pl}\\
	\glt	\sqt{Which (picture) goes behind which, which goes in front, which goes behind, take a look!}

	\ex	\label{ex:The men did not know which was whose}
	\gll	murgl-a-j	a-b-alχ-i=q'al	kutːi-la	ce	ca-d=el\\
		man\tsc{-obl.pl-dat}	\tsc{neg-n-}know\tsc{.ipfv-hab.pst=mod}	which\tsc{-gen}	what	\tsc{cop-npl=indq}\\
	\glt	\sqt{The men did not know which was whose (lit. of which) milk.}
\end{exe}

The pronoun \tit{ceʁuna} literally means \sqt{like what, similar to what} and requests the hearer to provide more information about the manner or the type as in \refex{ex:How is your car? Is it good?}. In example \refex{ex:Look what beautiful goats I bought}, the \isi{indefinite pronoun} modifies the following noun.

\begin{exe}
	\ex	\label{ex:How is your car? Is it good?}
	\gll	ceʁuna=ja	ala	mašin?		ʡaˁħ-ce=w?\\
		which\tsc{=q}	\tsc{2sg.gen}	car	good\tsc{-dd=q}\\
	\glt	\sqt{How is your car? Is it good?} (E)

	\ex	\label{ex:Look what beautiful goats I bought}
	\gll	či-d-ag-a	du-l	ceʁuna	qːuʁa-te	eč-ne	asː-ib=da=jal\\
		\tsc{spr-npl-}see\tsc{.pfv-imp}	\tsc{1sg-erg}	which	beautiful\tsc{-dd.pl} 	she.goat\tsc{-pl}	buy\tsc{.pfv-pret=1=indq}\\
	\glt	\sqt{Look what beautiful goats I bought.}
\end{exe}


% - - - - - - - - - - - - - - - - - - - - - - - - - - - - - - - - - - - - - - - - - - - - - - - - - - - - - - - - - - - - - - - - - - - - - - - - - - - - - - - - - - - - - - - - - - - - - - - - - - - - - - - - - - - - - - - - - - - - - - - - - - %

\subsubsection{\tit{cel} and \tit{celij} \sqt{why}}
\label{sssec:cel and celij why}

These pronouns are case-inflected forms of \tit{ce} \sqt{what}, more specifically \isi{ergative} \tit{cel} and \isi{dative} \tit{celij}, and the semantics of the case suffixes together with the base pronoun transparently explains the meaning `why' (< `what for'). The \isi{ergative} is used to express agents and instruments, and the \isi{dative} for the expression of causes.

\begin{exe}
	\ex	\label{ex:Why did the people not love him? Why did the women kill him}
	\gll	χalq'-li-j	il	cel	a-∅-jčː-aq-ul=de?	cellij	kax-ub=e	il xːun-r-a-l?\\
		people\tsc{-obl-dat}	that	why	\tsc{neg-m-}want\tsc{.ipfv-caus-icvb=pst}	why	kill\tsc{.pfv-pret=q}	that	woman\tsc{-pl-obl-erg}\\
	\glt	\sqt{Why did the people not love him? Why did the women kill him?}
\end{exe}


% - - - - - - - - - - - - - - - - - - - - - - - - - - - - - - - - - - - - - - - - - - - - - - - - - - - - - - - - - - - - - - - - - - - - - - - - - - - - - - - - - - - - - - - - - - - - - - - - - - - - - - - - - - - - - - - - - - - - - - - - - - %

\subsubsection{\tit{ceqːel} \sqt{when}}
\label{sssec:ceqel when}

This interrogative adverb is used when asking for time points. It can occur in the \isi{genitive} case without a change in meaning \refex{ex:When will we go to the theater?}.

\begin{exe}
	\ex	\label{ex:When did he come}
	\gll	het	ceqːel	ha-∅-jʁ-ib=e?\\
		that	when	\tsc{up-m-}come\tsc{.pfv-pret=q}\\
	\glt	\sqt{When did he come?}
\end{exe}
%


\begin{exe}
	\ex	\label{ex:When will we go to the theater?}
	\gll	ceqːel-la	/	ceqːel	nušːa	teatir-le	d-ax-an=da?\\
		when\tsc{-gen}	/	when	\tsc{1pl}	theater\tsc{-loc}	\tsc{1/2pl-}go\tsc{-ptcp=1}\\
	\glt	\sqt{When will we go to the theater?} (E)
\end{exe}


% - - - - - - - - - - - - - - - - - - - - - - - - - - - - - - - - - - - - - - - - - - - - - - - - - - - - - - - - - - - - - - - - - - - - - - - - - - - - - - - - - - - - - - - - - - - - - - - - - - - - - - - - - - - - - - - - - - - - - - - - - - %

\subsubsection{\tit{čujna} \sqt{how many times}}
\label{sssec:cujna how many times}

This interrogative adverb refers to the frequency with which a situation occurs. To the same adverb the suffixes can be added that can also be added to \isi{multiplicative numerals} when they are used for the formation of expressions referring to time points \refex{ex:At which time did you go? I went at the second time.} (\refsec{sec:multiplicativenumerals}).

\begin{exe}
	\ex	\label{ex:How many times do you pray (every day)? I pray five times}
	\gll	čujna	debʁalla	b-irq'-itːe? 	xu-jna	b-irq'-id\\
		how.often	prayer	\tsc{n-}do\tsc{.ipfv-2sg} 	five-time	\tsc{n-}do\tsc{.ipfv-1.prs}\\
	\glt	\sqt{How many times do you pray (every day)? I pray five times.} (E)
\end{exe}


\begin{exe}
	\ex	\label{ex:At which time did you go? I went at the second time.}
	\gll	čujna-lla	u	ag-ur-il=de?	k'ʷi-jna-lla	ag-ur-il=de\\
		how.often\tsc{-temp}	\tsc{2sg}	go\tsc{.pfv-pret-ref=pst}	two-time\tsc{-temp}	go\tsc{.pfv-pret-ref=pst}\\
	\glt	\sqt{At which time did you go? I went at the second time.} (E)
\end{exe}


% - - - - - - - - - - - - - - - - - - - - - - - - - - - - - - - - - - - - - - - - - - - - - - - - - - - - - - - - - - - - - - - - - - - - - - - - - - - - - - - - - - - - - - - - - - - - - - - - - - - - - - - - - - - - - - - - - - - - - - - - - - %

\subsubsection{\tit{čum} \sqt{how many}}
\label{sssec:cum how many}

The pronoun \tit{čum} \sqt{how many} is only used as a modifier to count \isi{nouns}. It can be inflected with the \isi{dative} yielding \tit{čum-li-j} if no head noun is following \refex{ex:How much does the chair cost?}. This form is used when asking for prices. Instead of directly adding the \isi{dative} to the interrogative pronoun it can also be added to the head noun, e.g. \tit{čum q'uruš-li-j?} (how many ruble\tsc{-obl-dat}) \sqt{for how many rubles?}.

\begin{exe}
	\ex	\label{ex:How many thousand (of rubles) does (the government) give per month (as child allowance)?}
	\gll	bac-li-j	čum	azir	lukː-unne?\\
		moon\tsc{-obl-dat}	how.many	thousand	give\tsc{.ipfv-icvb}\\
	\glt	\sqt{How many thousand (of rubles) does (the government) give per month (as child allowance)?}

	\ex	\label{ex:How much does the chair cost?}
	\gll	čum-li-j	b-ik'-ul=e	kursːi?\\
		how.many\tsc{-obl-dat}	\tsc{n-}say\tsc{.ipfv-icvb=q}	chair\\
	\glt	\sqt{How much does the chair cost?} (E)
\end{exe}

The forms \tit{čum-ib} and \tit{čum-ibil} ask for ordinal numbers (\refsec{sec:ordinalnumerals}):

\begin{exe}
	\ex	\label{ex:Even they do not remember in which year (the pictures) were drawn}
	\gll	daže	hel-tː-a-j d-alχ-ul akːu	hel-tːi	čum-ib	dusːi-cːe-d d-elk'-un-ne=l	hel-tːi\\
		even	that\tsc{-pl-obl-dat}	\tsc{npl-}know\tsc{.ipfv-icvb}	\tsc{cop.neg}	that\tsc{-pl}		how.many\tsc{-ord}	year\tsc{.obl-in-npl}	\tsc{npl-}write\tsc{.pfv-pret-cvb=prt}	that\tsc{-pl}\\
	\glt	\sqt{Even they do not remember in which year (the pictures) were drawn.}
\end{exe}


% - - - - - - - - - - - - - - - - - - - - - - - - - - - - - - - - - - - - - - - - - - - - - - - - - - - - - - - - - - - - - - - - - - - - - - - - - - - - - - - - - - - - - - - - - - - - - - - - - - - - - - - - - - - - - - - - - - - - - - - - - - %

\subsubsection{\tit{kusa} \sqt{how much, how many}}
\label{sssec:kusa how much how many}

The pronoun \tit{kusa} can be used together with count \isi{nouns} or mass \isi{nouns} \refex{ex:How much flour remained?} and without any head \isi{nouns} \xxref{ex:For how much did you buy it}{ex:How (much) early did he go}. It also has the more specific temporal meaning \sqt{(for) how long} \refex{ex:No one is able to know for how long they were away}.

\begin{exe}

	\ex	\label{ex:How much flour remained?}
	\gll	kusa	bet'u	d-el=e?\\
		how.much	flour	\tsc{npl-}remain\tsc{.pfv=q}\\
	\glt	\sqt{How much flour remained?} (E)

	\ex	\label{ex:For how much did you buy it}
	\gll	kusa-lli-j	asː-ib=de?\\
		how.much-\tsc{obl-dat}	buy\tsc{.pfv-pret=pst}\\
	\glt	\sqt{For how many (rubles) did you buy it?}
	
	\ex	\label{ex:No one is able to know for how long they were away}
	\gll	il-tːi	kusa	tːura-b	kelg-un=el	b-alχ-an	a-haq-ib\\
		that\tsc{-pl}	how.much	outside\tsc{-hpl}	remain\tsc{.pfv-pret=indq}	\tsc{n-}know\tsc{.ipfv-ptcp}	\tsc{neg-}manage\tsc{.pfv-pret}\\
	\glt	\sqt{No one is able to know for how long they were away.}

	\ex	\label{ex:How (much) early did he go}
	\gll	kusa	ixʷle	ag-ur-re?\\
		how.much	early	go\tsc{.pfv-pret-cvb}\\
	\glt	\sqt{How (much) early did he go?}
\end{exe}


% - - - - - - - - - - - - - - - - - - - - - - - - - - - - - - - - - - - - - - - - - - - - - - - - - - - - - - - - - - - - - - - - - - - - - - - - - - - - - - - - - - - - - - - - - - - - - - - - - - - - - - - - - - - - - - - - - - - - - - - - - - %

\subsection{Interrogative pronouns used as indefinites}
\label{ssec:Interrogative pronouns as indefinites}

Occasionally plain interrogative pronouns are used as \is{indefinite pronoun}indefinite pronouns as in the following example \refex{ex:He says, my hands got tired, I do something}.

\begin{exe}
	\ex	\label{ex:He says, my hands got tired, I do something}
	\gll	nuˁq-be	ʡaˁbħ-ib ca<d>i	∅-ik'-ul ca-w	ij,	``ce	d-irq'-ul=da''\\
		arm\tsc{-pl}	get.tired\tsc{.pfv-pret} \tsc{cop<npl>}	\tsc{m-}say\tsc{.ipfv-icvb} \tsc{cop-m}	this	what	\tsc{npl-}do\tsc{.ipfv-icvb=1}\\
	\glt	\sqt{He says, ``My hands got tired, I do something.''}
\end{exe}


% --------------------------------------------------------------------------------------------------------------------------------------------------------------------------------------------------------------------- %

\section{Indefinite pronouns}
\label{sec:Indefinite pronouns}

Sanzhi Dargwa has a rather wide range of \is{indefinite pronoun}indefinite pronouns that are regularly formed on the basis of the interrogative pronouns. Most of these pronouns make use of enclitics that are also otherwise used in the grammar as complementizers (\tit{=jal}\slash\tit{=el}, \tit{=del}), emphatic \isi{particle} (\tit{=k'u}) or \isi{additive enclitic} (\tit{=ra}). The pronominal stems are normally inflected just like the interrogative pronouns, and then the derivational markers are attached.\pagebreak

\begin{itemize}
	\item	\tit{=jal}\slash\tit{=el}: specific indefinite (\refsec{ssec:Specific indefinite pronouns})
		\item	\tit{=k'u}: specific indefinite (\refsec{ssec:Specific indefinite pronouns})
	\item	\tit{=del}: non-specific indefinite (\refsec{ssec:Non-specific indefinite pronouns})
	\item	\tit{-k'a}: free-choice indefinite (\refsec{ssec:Non-specific indefinite pronouns})
	\item	\tit{-k'al}: negative indefinite, specific indefinite, free-choice indefinite (\refsec{ssec:Negative indefinite pronouns})
		\item	\tit{=č'u}: negative indefinite, free-choice indefinite (\refsec{ssec:Negative indefinite pronouns})
	\item	\tit{=ra}: negative indefinite, universal indefinite, free-choice indefinite (\refsec{ssec:Negative indefinite pronouns})
\end{itemize}

For the formation of universal indefinites the \isi{quantifier} \tit{har} \sqt{every} or more rarely \tit{li<b>il} \sqt{all} is used (\refsec{sec:Universal indefinites and other quantifiers}).


% - - - - - - - - - - - - - - - - - - - - - - - - - - - - - - - - - - - - - - - - - - - - - - - - - - - - - - - - - - - - - - - - - - - - - - - - - - - - - - - - - - - - - - - - - - - - - - - - - - - - - - - - - - - - - - - - - - - - - - - - - - %

\subsection{Specific indefinite pronouns}
\label{ssec:Specific indefinite pronouns}

Specific \is{indefinite pronoun}indefinite pronouns (\reftab{tab:Specific indefinite pronouns}) are formed by means of the complementizer \tit{=jal} (after vowels)\slash\tit{=el} (after \isi{consonants}), which is otherwise used in embedded \isi{questions} (see \refsec{sec:Subordinate questions}) and certain epistemic modal constructions that have developed out of embedded \isi{questions} and can be labeled ``\isi{insubordination}'' (\refsec{sec:Subordinate questions}).

\begin{table}
	\caption{Specific indefinite pronouns}
	\label{tab:Specific indefinite pronouns}
	\small
	\begin{tabular}{ll@{\hspace{2.75em}}ll@{\hspace{2.75em}}ll}
		
		\lsptoprule
		\tit{ča=jal}		&	\sqt{somebody}		&	\mbox{\tit{čina-b=el}} &	\sqt{somewhere}	&	\tit{čum=el}		&	\sqt{some}\\
		\tit{ce=jal}		&	\sqt{something}		&	\tit{cet'le=jal}	&	\sqt{somehow}	&	\tit{kutːi=jal}	&	\sqt{some}\\
		\tit{čina=jal}		&	\sqt{to somewhere}	&	\tit{celij=jal}		&	\sqt{for some reason}	&	\tit{ceqːel=el}	&	\sqt{sometimes}\\
		\lspbottomrule
	\end{tabular}
\end{table}

Exemplary case forms of \tit{ča=jal} and \tit{ce=jal} are:

\begin{itemize}
	\item	\sqt{somebody}: \isi{ergative} \tit{hi-l=el}, \isi{genitive} \tit{hi-la=jal}, \isi{dative} \tit{hi-j=jal}, \isi{comitative} \tit{hi-cːe=jal}
	\item	\sqt{something}: \isi{ergative} \tit{ce-l-li=jal}, \isi{genitive} \tit{ce-l-la=jal}, \isi{dative} \tit{ce-li-j=jal}
\end{itemize}

\begin{exe}
	\ex	\label{ex:Someone bought it down (= the area around the village of Sanzhi), they say}
	\gll	ik'	gu	gu-r-asː-ib ca-b,	b-ik'ʷ-ar,	hi-l=el\\
		\tsc{dem.up}	under	\tsc{sub-abl}-buy.\tsc{pfv-pret} \tsc{cop-n}	\tsc{hpl-}say\tsc{.ipfv-prs}	who\tsc{.obl-erg=indef}\\
	\glt	\sqt{Someone bought it down (= the area around the village of Sanzhi), they say.}

	\ex	\label{ex:There down is a can of something}
	\gll	heχ	ce-lla=jal	banka	χe-b\\
		\tsc{dem.down}	what\tsc{-gen=indef}	can	exist.\tsc{down}\tsc{-n}\\
	\glt	\sqt{Down there is a can of something.}

	\ex	\label{ex:In Karka the bandits stayed for some days.}
	\gll	qːarka	hetːu-b	b-už-ib	čum=el	bari\\
		Karka	there\tsc{-n}	\tsc{hpl-}be\tsc{-pret}	how.many\tsc{=indef}	day\\
	\glt	\sqt{In Karka the bandits stayed for some days.}
\end{exe}

There is a second series of specific \is{indefinite pronoun}indefinite pronouns with the emphatic \isi{enclitic} \tit{=k'u} (\refsec{ssec:Further enclitics that manipulate the information structure}) that is used when the speaker does not remember a name of a person or thing and instead uses the indefinite as a kind of filler word. Of these pronouns \tit{ce=k'u} (what\tsc{-indef}) is especially frequent and can be translated as \sqt{whatchamacallit}. 

\begin{exe}
	\ex	\label{ex:These are the prison's whatchamacallits}
	\gll	hel-tːi	tusnaq-la	ce=k'u	ca-d\\
		that\tsc{-pl}	prison\tsc{-gen}	what\tsc{=indef}	\tsc{cop-npl}\\
	\glt	\sqt{These are the prison's whatchamacallits.}

	\ex	\label{ex:This one his sister, that Hasan who lives down there}
	\gll	ča=k'u-la	rucːi	heχ	ħaˁsan	χe-w=q'al\\
		who\tsc{=indef-gen}	sister	\tsc{dem.down}	Hasan	exist\tsc{.down-m=mod}\\
	\glt	\sqt{This one his sister, that Hasan who lives down there.}

	\ex	\label{ex:This one, how is he called, he is still alive}
	\gll	ik'	ča=k'u=q'ar	b-ik'ʷ-ar	ik'	mic'ir-re	w-el\\
		\tsc{dem.up}	who\tsc{=indef=mod}	\tsc{hpl-}say\tsc{.ipfv-prs}	\tsc{dem.up}	alive\tsc{-advz}	\tsc{m-}remain\\
	\glt	\sqt{This one, what is he called, he is still alive.}

	\ex	\label{ex:To someone I (masc.) said, well I will wash him}
	\gll	hi-l=k'u-cːe	``jaʁari,	du-l	ic-an=da''	∅-ik'ʷ-a-di\\
		who\tsc{.obl-obl=indef-in}	\tsc{prt}	\tsc{1sg-erg}	wash\tsc{.ipfv-ptcp=1}	\tsc{m-}say\tsc{.ipfv-hab-1}\\
	\glt	\sqt{To someone I (masc.) said, ``Well, I will wash him.''}
\end{exe}


% - - - - - - - - - - - - - - - - - - - - - - - - - - - - - - - - - - - - - - - - - - - - - - - - - - - - - - - - - - - - - - - - - - - - - - - - - - - - - - - - - - - - - - - - - - - - - - - - - - - - - - - - - - - - - - - - - - - - - - - - - - %

\subsection{Non-specific indefinite pronouns}
\label{ssec:Non-specific indefinite pronouns}

Non-specific \is{indefinite pronoun}indefinite pronouns are formed by adding \tit{=del} to the interrogative base. This suffix is morphologically complex consisting of \tit{=de} and \tit{=(e)l}. The first part might originate from the past \isi{enclitic} \tit{=de}. The second part represents the \isi{enclitic} used for embedded \isi{questions} (\refsec{sec:Subordinate questions}) and also for the formation of specific \is{indefinite pronoun}indefinite pronouns (\refsec{ssec:Specific indefinite pronouns}). The following examples illustrate reference to non-specific indefinite persons \refex{ex:He was called Abdukhaliq or something (lit. somebody), I don't know, I don't remember this name well}, \refex{ex:My cousin, Old Kurban, may his sins be relieved, brought them for me to give them to someone} and places \refex{ex:I put (the picture) somewhere}, \refex{ex:He was somewhere}.

\begin{exe}
	\ex	\label{ex:He was called Abdukhaliq or something (lit. somebody), I don't know, I don't remember this name well}
	\gll	ʡaˁbdulq'adir	b-ik'ʷ=el	aχːu	ča=del	na	zu	ʡaˁħ-le	han	d-il	akːu	hel\\
		Abdulkadir	\tsc{n-}say\tsc{.ipfv=indq}	not.know	who\tsc{=indef}	now	name	good\tsc{-advz}	remember	\tsc{npl-}remain	\tsc{cop.neg}	that\\
	\glt	\sqt{He was called Abdukhaliq or something (lit. somebody), I don't know, I don't remember the name well.}

	\ex	\label{ex:My cousin, Old Kurban, may his sins be relieved, brought them for me to give them to someone}
	\gll	hek'	di-la	ucːiq'ar ca-w,	χːula	Q'urban,	buna.χat'a	gu-r-ka-d-uc,		hek'-i-la	k-aqː-ib-le	hi-j=del	b-ičː-ij\\
		\tsc{dem.up}	\tsc{1sg-gen}	cousin	\tsc{cop-m}	big	Kurban	sin	\tsc{sub-abl-down}\tsc{-npl-}catch\tsc{.pfv}		\tsc{dem.up}\tsc{obl-gen}	\tsc{down}-carry\tsc{-pret-cvb}	who\tsc{.obl-dat=indef}	\tsc{n-}give\tsc{.pfv-inf}\\
	\glt	\sqt{My cousin, Old Kurban, may his sins be relieved, brought them for me to give them to someone.}
	
	\ex	\label{ex:I put (the picture) somewhere}
	\gll	ka-b-iž-ib-il	ka-b-išː-ib=da	heltːu	čina=del\\
		\tsc{down-hpl}-be.\tsc{pfv-pret-ref}	\tsc{down-n-}put\tsc{.pfv-pret=1}	there	where\tsc{=indef}\\
	\glt	\sqt{I put (the picture) somewhere.}

	\ex	\label{ex:He was somewhere}
	\gll	čina-w=del	le-w=de=q'al\\
		where\tsc{-m=indef}	exist\tsc{-m=pst=mod}\\
	\glt	\sqt{He was somewhere.}


\end{exe}


% - - - - - - - - - - - - - - - - - - - - - - - - - - - - - - - - - - - - - - - - - - - - - - - - - - - - - - - - - - - - - - - - - - - - - - - - - - - - - - - - - - - - - - - - - - - - - - - - - - - - - - - - - - - - - - - - - - - - - - - - - - %

\subsection{Free-choice indefinite pronouns}
\label{ssec:Free-choice indefinite pronouns}

Free-choice \is{indefinite pronoun}indefinite pronouns of the \sqt{any} or \sqt{WH-ever} type are formed by means of the suffix \tit{-k'a} that does not serve any other function. In the majority of the cases the pronoun is followed by the verb form \tit{b-iχʷ-ar=ra} (\tsc{n-}be\tsc{.pfv-cond.3=add}) that has a \isi{concessive} meaning that can approximately be translated with \sqt{even if it is} (\refsec{sec:Other verbs used in copula-functions and as auxiliaries}). The verb form \tit{b-iχʷ-ar=ra} mostly has the neuter singular prefix \tit{b-} (suspended agreement), but it can also agree with the \isi{absolutive} argument or even some other salient argument (see \refsec{ssec:Gender agreement with arguments in other than the absolutive case} for examples). In natural speech the suffix \tit{-k'a} is also added to the Russian free-choice \isi{indefinite pronoun} \tit{lubuj-cːella-k'a-li-j} (any\tsc{-comit-indef-obl-dat}) \sqt{for anything}. The following example \refex{ex:Whenever I make it, I will bring it, I said} contains not only a Sanzhi free-choice indefinite, but also the Russian free-choice indefinite \textit{kagda-nibud} `whenever'.

\begin{exe}
	\ex	\label{ex:Whenever I make it, I will bring it, I said}
	\gll	ceqːel-k'a	b-iχʷ-ar=ra	b-arq'-ille,	haʔ-ib=da	kagda.nibud	ka-b-iqː-an=da\\
		when\tsc{-indef}	\tsc{n-}be\tsc{.pfv-cond.3=add}	\tsc{n-}do\tsc{.pfv-cond.1}	say\tsc{.pfv-pret=1}		whenever	\tsc{down-n-}carry\tsc{.ipfv-ptcp=1}\\
	\glt	\sqt{Whenever I make it, I will bring it, I said.}

	\ex	\label{ex:He was hiding himself at the house of whomever. (at any house)}
	\gll	daˁʡaˁna	w-irx-ul	hi-la-k'a	b-iχʷ-ar=ra	qili\\
		secret	\tsc{m-}become\tsc{.ipfv-icvb}	who\tsc{-gen-indef}	\tsc{n-}be\tsc{.pfv-cond.3=add}	home\\
	\glt	\sqt{He was hiding himself at the house of whomever.} (i.e. at any house).

	\ex	\label{ex:15000 needs to be given to the doctors and every day these (i.e. this amount of money), however many days you stay}
	\gll	wec'-nu	xu-ra	azir	tuχtur-t-a-j=ra	lukː-an-te	ca-d	i	har	bari-j	hel-tːi=ra	čum-k'a	bar	kelg-an=ra\\
		ten-\tsc{ten}	five\tsc{-num}	thousand doctor\tsc{-pl-obl-dat=add} give\tsc{.ipfv-ptcp-dd.pl} \tsc{cop-npl}	and	every	day\tsc{-dat} that\tsc{-pl=add}	how.many\tsc{-indef} day	remain\tsc{.pfv-ptcp=add}\\
	\glt	\sqt{15,000 needs to be given to the doctors and every day these (i.e. this amount of money), however many days you stay.}

	\ex	\label{ex:I will give you whatever you can want}
	\gll	du-l	at	ce-k'a	b-ikː-ul	haq-itːe=ra	lukː-an=da=n\\
		\tsc{1sg-erg}	\tsc{2sg.dat}	what\tsc{-indef}	\tsc{n-}want\tsc{.ipfv-icvb}	be.enough\tsc{.pfv-2sg=add}	give\tsc{.ipfv-ptcp=1=prt}\\
	\glt	\sqt{I will give you whatever you may want.}
\end{exe}


% - - - - - - - - - - - - - - - - - - - - - - - - - - - - - - - - - - - - - - - - - - - - - - - - - - - - - - - - - - - - - - - - - - - - - - - - - - - - - - - - - - - - - - - - - - - - - - - - - - - - - - - - - - - - - - - - - - - - - - - - - - %

\subsection{Negative indefinite pronouns}
\label{ssec:Negative indefinite pronouns}

In general, the negative indefinite function of \is{indefinite pronoun}indefinite pronouns is only available in clauses with \isi{negation}. In affirmative clauses none of the pronouns described in this section has a negative indefinite reading, but readings such as free-choice indefinite or universal indefinite. 

The suffix \tit{-k'al} is used for the formation of \is{indefinite pronoun}indefinite pronouns that have the negative indefinite reading if they occur in a clause with negative polarity \refex{ex:There is no place where there are no bottles}, \refex{ex:Now nobody will betray me anymore}. This suffix can be analyzed as consisting of \tit{-k'a}, which forms free-choice \is{indefinite pronoun}indefinite pronouns (\refsec{ssec:Free-choice indefinite pronouns}) and the \isi{enclitic} used for embedded \isi{questions} (\refsec{sec:Subordinate questions}) and also for the formation of specific and non-specific \is{indefinite pronoun}indefinite pronouns (Sections \refsec{ssec:Specific indefinite pronouns}, \refsec{ssec:Non-specific indefinite pronouns}).

\begin{exe}
	\ex	\label{ex:There is no place where there are no bottles}
	\gll	šuša	akːʷ-ar	musːa	χe-b-akːu	čina-b-k'al\\
		bottle	\tsc{cop.neg-prs}	place	exist.\tsc{down-n}-\tsc{cop.neg}	where\tsc{-n-indef}\\
	\glt	\sqt{There is no place where there are no bottles.}

	\ex	\label{ex:Now nobody will betray me anymore}
	\gll	na=q'ar	du	hi-l-k'al-li	a-w-irʡ-aˁn=da\\
		now\tsc{=mod}	\tsc{1sg}	who\tsc{.obl-obl-indef-erg}	\tsc{neg-m-}betray\tsc{-ptcp=1}\\
	\glt	\sqt{Now nobody will betray me anymore.}
\end{exe}

Other meanings of pronouns with \tit{-k'al} are free-choice indefiniteness if they are used in a \isi{conditional clause} \refex{ex:Whoever came saying, do something, do this work, father did all works}, \refex{ex:And like this also (Isakadi’s) issues, things do not finish, forever, wherever he went} or non-specific indefinite if simply used in an affirmative clause \xxref{ex:Then this needs to be positioned somewhere (else) probably}{ex:Wine fit well with any of these (types of food)}.

\begin{exe}
	\ex	\label{ex:Whoever came saying, do something, do this work, father did all works}
	\gll	ča-k'al	sa-∅-jʁ-ardel,		``ci-k'al	b-arq'-a,	ʡaˁči	b-arq'-a!''	∅-ik'-ul,	li<d>il	ʡaˁči	d-irq'-i	atːa-l\\
		who\tsc{-indef}	\tsc{hither-m-}come\tsc{.pfv-cond.pst}		what\tsc{-indef}	\tsc{n-}do\tsc{.pfv-imp}		work	\tsc{n-}do\tsc{.pfv-imp}	\tsc{m-}say\tsc{.ipfv-icvb}	all\tsc{<npl>}	work	\tsc{npl-}do\tsc{.ipfv-hab.pst}	father\tsc{-erg}\\
	\glt	\sqt{No matter who came saying, ``Do something, do this work!'' father did all works.}

	\ex	\label{ex:And like this also (Isakadi’s) issues, things do not finish, forever, wherever he went}
	\gll	itwaj=ra	qːulluqː-e	a-ha-d-urχː-u,	it	wečna	čina-k'al	tːura-w-q-utːel\\
		like.this\tsc{=add}	matter\tsc{-pl}	\tsc{neg-up}\tsc{-npl-}finish\tsc{.ipfv-prs}	that	forever	where\tsc{-indef}	\tsc{out-m-}go\tsc{.pfv-cond.pst}\\
	\glt	\sqt{And like this also (Isakadi's) issues, things do not finish, forever, no matter where he went.}

	\ex	\label{ex:Then this needs to be positioned somewhere (else) probably}
	\gll	abuχar	hež	či-r-b-iqː-an	b-urkː-ar	čina-k'al\\
		then	this	\tsc{spr-abl-n-}carry\tsc{.ipfv-ptcp}	\tsc{n-}find\tsc{.ipfv-prs}	where\tsc{-indef}\\
	\glt	\sqt{Then this needs to be positioned somewhere (else) probably.}

	\ex	\label{ex:Did you kill anyone}
	\gll	ča-k'al	kax-ub=de=w?\\
		who\tsc{-indef}	kill\tsc{.pfv-pret=2sg=q}\\
	\glt	\sqt{Did you kill anyone?}

	\ex	\label{ex:And together with the son they are going to sit outside anywhere}
	\gll	durħuˁ=ra	ca-b=ra	arg-ul ca-b	heštːi	čina-k'al	tːura-ka-b-ig-ar-aj\\
		boy\tsc{=add}	\tsc{refl-hpl=add}	go\tsc{.ipfv-icvb} \tsc{cop-hpl}	these	where\tsc{-indef}	\tsc{out-down-n-}be\tsc{-prs-subj.3}\\
	\glt	\sqt{And together with the son they are going to sit outside anywhere.}

	\ex	\label{ex:Wine fit well with any of these (types of food)}
	\gll	har	ce-lla-k'al-li-j	čaˁʡir=ra	d-al d-irč-iri\\
		every	what\tsc{-gen-indef-obl-dat}	wine\tsc{=add}	\tsc{npl-}match \tsc{npl-}occur\tsc{.ipfv-hab.pst}\\
	\glt	\sqt{Wine fitted well with any of these (types of food).}
\end{exe}

Note that the word \tit{cik'al} (from \tit{ce} \sqt{what} plus \tit{-k'al}) has been lexicalized as a noun with the meaning \sqt{thing}. At the same time it is still used as an \isi{indefinite pronoun} with the meanings \sqt{nothing} (in negative clauses) and \sqt{something, anything} in positive clauses \refex{ex:Whoever came saying, do something, do this work, father did all works}. It can also precede \isi{nouns} as negative \isi{quantifier} with the meaning \sqt{no}.

Furthermore, the \isi{additive enclitic} \tit{=ra} (\refsec{ssec:The additive enclitic}) is used for the formation of \is{indefinite pronoun}indefinite pronouns. If these pronouns occur in clauses with positive polarity the reading is universal indefinite \refex{ex:I always went through the mountains}, if they occur in clauses with negative polarity the reading is universal negative \refex{ex:I did not go even once.1}, and if they occur in \isi{concessive} clauses the reading is free choice indefinite \refex{ex:However I hide them, (they always) find them}.

\begin{exe}
	\ex	\label{ex:I always went through the mountains}
	\gll	ceqːel=ra	w-aš-ib=da	dubur-t-a-cːe\\
		when\tsc{=add}	\tsc{m-}go\tsc{.ipfv-pret=1}	mountain\tsc{-pl-obl-in}\\
	\glt	\sqt{I always went through the mountains.}

	\ex	\label{ex:I did not go even once.1}
	\gll	čujna=ra	a-ag-ur=da\\
		how.often\tsc{=add}	\tsc{neg-}go\tsc{.pfv-pret=1}\\
	\glt	\sqt{I did not go even once.}

	\ex	\label{ex:However I hide them, (they always) find them}
	\gll	cet'le=ra	du-l	itːi	daˁʡaˁna	d-arq'-ib-le=xːar, amma d-urkː-ul ca-d\\
		how\tsc{=add}	\tsc{1sg-erg}	those	secret	\tsc{npl-}do\tsc{.pfv-pret-cvb=conc}	but	\tsc{npl-}find\tsc{.ipfv-icvb} \tsc{cop-npl}\\
	\glt	\sqt{No matter how I hide them, (they always) find them.} (E)
\end{exe}

However, in practice such \is{indefinite pronoun}indefinite pronouns are (almost) never attested in natural texts. Instead, the \isi{enclitic} \tit{=ra} is usually preceded by \tit{-k'al} \refex{ex:Nobody is at home} or occasionally \tit{-k'a} \refex{ex:I do not see any pit} for the negative indefinite meaning.

\begin{exe}
	\ex	\label{ex:Nobody is at home}
	\gll	ča-k'al=č'u=ra	qili-w	w-akːu\\
		who-\tsc{indef=emph=add}	home-\tsc{m}	\tsc{m}-\tsc{cop.neg}\\
	\glt	\sqt{Nobody is at home.}

	\ex	\label{ex:I do not see any pit}
	\gll	dam	ci-k'a=ʁuna=ra	kur	či-a-b-až-ib=da\\
		\tsc{1sg.dat}	what-\tsc{indef=eq=add}	pit	\tsc{spr-neg-n}-see.\tsc{pfv-pret=1}\\
	\glt	\sqt{I do not see any pit.}
\end{exe}

Similarly, the emphatic \isi{enclitic} \textit{=č'u} can form negative \is{indefinite pronoun}indefinite pronouns when it is attached to the usual base (interrogative pronoun or numeral `one') and used in clauses with negative polarity. As with the \isi{additive enclitic}, in almost all examples that contain the \isi{enclitic} it follows \textit{-k'al} (and frequently \textit{=č'u} is followed by the \isi{additive enclitic}) \refex{ex:Nothing bad will happen.}, \refex{ex:No, there was no one, except for Ashura he did not marry anyone.}. 

\begin{exe}
	\ex	\label{ex:Nothing bad will happen.} 
	\gll	ci-k'al=č'u=ra	a-b-irχ-u\\
		what-\tsc{indef=emph=add}	\tsc{neg-n}-become.\tsc{ipfv-prs.3}\\
	\glt	\sqt{Nothing bad will happen.} (lit. `There will be nothing.')

	\ex	\label{ex:No, there was no one, except for Ashura he did not marry anyone.} [So they did not marry.]\\
	\gll	akːu,		ca=č'u=ra	akːu,	ca=č'u	akːu	Ašura	ka-r-iž-ib\\
		\tsc{cop.neg}		one=\tsc{emph=add}	\tsc{cop.neg}		one=\tsc{emph}	\tsc{cop.neg}	Ashura	\tsc{down-f}-sit.\tsc{pfv-pret}\\
	\glt	\sqt{No, there was no one, except for Ashura he did not marry anyone.}
\end{exe}

There are only two examples of pronouns with \textit{=č'u} in clauses that do not have negative polarity such that the pronouns display their free-choice indefinite meaning. The first example in \refex{saying that they will also eat no matter what} illustrates the use of the pronoun in combination with the \isi{concessive} auxiliary \textit{b-iχʷ-ar=ra} as it has already been described for other free-choice \is{indefinite pronoun}indefinite pronouns (\refsec{ssec:Free-choice indefinite pronouns}). The second example in \refex{The ones who are drinking anything good, what do they do? They do not do anything (good).} contains two pronouns with \textit{=č'u}, of which the first has the free-choice indefinite reading whereas the second is a negative \isi{indefinite pronoun} because of the negated verb.

\begin{exe}
	\ex	\label{saying that they will also eat no matter what} 
	\gll	ca=ra	ci-k'al=č'u	b-iχʷ-ar=ra	b-uk-an-ne,	b-ik'-ul	…\\
		one=\tsc{add}	what-\tsc{indef=prt}	\tsc{n}-be.\tsc{pfv-cond.3=add}	\tsc{hpl}-eat.\tsc{ipfv-ptcp-fut.3}	\tsc{hpl}-say.\tsc{ipfv-icvb}\\
	\glt	\sqt{saying that they will also eat no matter what} (lit. `whatever it might be')
	
		\ex	\label{The ones who are drinking anything good, what do they do? They do not do anything (good).} 
	\gll	heštːi	deč-li	b-učː-an-t-a-l	ci-k'al=č'u	ʡaˁħ-dex,	iš-tː-a-l	ce	b-irq'-u=ja?		ci-k'al=č'u	a-b-irq'-u\\
		these	drinking-\tsc{erg}	\tsc{hpl}-drink.\tsc{ipfv-ptcp-pl-obl-erg}	what-\tsc{indef=prt}	good-\tsc{nmlz}	this-\tsc{pl-obl-erg}	what	\tsc{n}-do.\tsc{ipfv-prs=q}		what-\tsc{indef=prt}	\tsc{neg-n}-do.\tsc{ipfv-prs}\\
	\glt	\sqt{The ones who are drinking anything good, what do they do? They do not do anything (good).} 
	\end{exe}

Other negative \is{indefinite pronoun}indefinite pronouns are \tit{caʔarra} \sqt{no one} and \tit{cajnara} \sqt{never, not once} (\tit{ca-jna=ra} one-\tsc{time}\tsc{=add}). The first pronoun consists of (\tit{ca-ʔar=ra} one-?\tsc{=add}) and\linebreak seems to be related the \isi{focus-sensitive particle} \tit{arrah} \sqt{at least} (\refsec{ssec:Further enclitics that manipulate the information structure})

\begin{exe}
	\ex	\label{ex:No one was hungry}
	\gll	caʔarra	kːuš	∅-iχ-ub-il	akː-i\\
		no.one	hungry	\tsc{m-}be\tsc{.pfv-pret-ref}	\tsc{cop.neg-hab.pst}\\
	\glt	\sqt{No one was hungry.}
\end{exe}


% --------------------------------------------------------------------------------------------------------------------------------------------------------------------------------------------------------------------- %

\section{Universal indefinites and other quantifiers}
\label{sec:Universal indefinites and other quantifiers}

Universal indefinites are normally not formed from interrogative pronouns, but by\linebreak means of the \isi{quantifier} \tit{har} \sqt{every} (or \tit{li<b>il} \sqt{all}) plus a following noun:

\begin{itemize}
	\item	\tit{har admi} \sqt{everyone} (every man)
	\item	\tit{har}\slash\tit{li<d>il cik'al} \sqt{everything}
	\item	\tit{har}\slash\tit{li<b>il musːa}\slash\tit{musne} \sqt{everywhere} (lit. every place\slash all places)
	\item	\tit{har zamana} \sqt{always} (lit. every time)
\end{itemize}
%

Other quantifiers are \tit{sukːil}, \tit{li<b>il} \sqt{all, whole, complete}, \tit{har}, \tit{haril}, \tit{harki}, \tit{harkil} \sqt{every}, \tit{b-aqil}, \tit{ʡaˁbra}, \tit{ʡaˁbra-b-al} \sqt{much, many}, and \tit{kam} \sqt{little, few}. The quantifiers treated in this section have most morphosyntactic properties that \isi{adjectives} have and, as \isi{adjectives}, normally occur before teh noun when they function as nominal modifiers. But just like \isi{adjectives} and some other nominal modifiers they can also follow the noun under certain circumstances. See \refsec{ssec:The structure and order of constituents within the noun phrase} for \isi{quantifier} floating. 

The quantifiers \tit{sukːil} and \tit{li<b>il} can both be used as attributes and they can be nominalized. When they are used as attributes of \isi{nouns} in the plural they mean \sqt{all}; with singular \isi{nouns} they translate as \sqt{whole, complete}. The \isi{quantifier} \tit{li<b>il} has a \isi{gender}/\isi{number} agreement slot and follows the agreement rules for \isi{adjectives} and other nominal modifiers, i.e. agreement with the head noun.

\begin{exe}
	\ex	\label{ex:all girls all houses}
	\begin{xlist}
		\TabPositions{10.5em,12.5em}
		\ex	\tit{li<b>il rurs-be} \sqt{all girls}		\tab	vs.	\tab	\tit{li<r>il rursːi} \sqt{the whole girl}
		\ex	\tit{sukːil qulbe }\sqt{all houses}		\tab	vs.	\tab	\tit{sukːil qal} \sqt{the complete house}
	\end{xlist}

	\ex	\label{ex:He divided all (the bread) like one (i.e. everyone got the same amount of bread)}
	\gll	sukːil	d-ut'-ib	ca	daˁʡle\\
		all	\tsc{npl-}divide\tsc{-pret}	one	as\\
	\glt	\sqt{He divided all (the bread) like one (i.e. everyone got the same amount).}

	\ex	
	\gll	di-la	li<d>il	daluj-te\\
		\tsc{1sg-gen}	all\tsc{<npl>}	song\tsc{-pl}\\
	\glt	\sqt{all my songs}
\end{exe}

From the quantifiers listed above, \tit{har} can only be used attributively. All other quantifiers can also be nominalized. The head noun is in the singular, but mass \isi{nouns} that trigger plural agreement are also possible if an interpretation referring to a specific quantity is available.

\begin{exe}
	\ex	\label{ex:All people escaped in every direction}
	\gll	li<b>il	χalq'	b-ibšː-ib ca-b	har	šːal\\
		all\tsc{<hpl>}	people	\tsc{hpl-}escape\tsc{-pret} \tsc{cop-hpl}	every	side\\
	\glt	\sqt{All people escaped in every direction.}

	\ex	\label{ex:Bring bread from everyone}
	\gll	``t'ult'	s-aqː-a,''	b-ik'-ul		``haril-li-cːe-rka!''\\
		bread	\tsc{hither}-carry\tsc{-imp.sg}	\tsc{hpl-}say\tsc{.ipfv-icvb}	every\tsc{-obl-in-abl}\\
	\glt	\sqt{``Bring bread from everyone!'' they said.}

	\ex	\label{ex:I (fem.) gave everyone's sack back and came back}
	\gll	het	harkil-la	qːup-re	či-d-ičː-ib-le						čar ha-r-iχ-ub=da\\
		that	every\tsc{-gen}	sack\tsc{-pl}	\tsc{spr-npl-}give\tsc{.pfv-pret-cvb}	back \tsc{up}\tsc{-f-}become\tsc{.pfv-pret=1}\\
	\glt	\sqt{I (fem.) gave everyone's sack back and came back.}
\end{exe}

The quantifiers \tit{b-aqil}, \tit{ʡaˁbra}, \tit{ʡaˁbra-b-al} \sqt{much, many} also show \isi{gender}/\isi{number} agreement with the head noun in case there is any. Otherwise they express the \isi{gender} and \isi{number} of the item they are referring to.

\begin{exe}
	\ex	\label{ex:He took part in many arguments}
	\gll	d-aqil	ʁaj-li-cːe	w-ič-ib ca-w\\
		\tsc{npl}-much	word-\tsc{obl-in}	\tsc{m}-occur.\tsc{pfv-pret} \tsc{cop-m}\\
	\glt	\sqt{He took part in many arguments.} (i.e. he had many problems)

	\ex	\label{ex:There are many edible roots}
	\gll	ʡaˁbra	ʁut'-e	d-irχʷ-ar\\
		much	edible.roots-\tsc{pl}	\tsc{npl}-become.\tsc{ipfv-prs}\\
	\glt	\sqt{There are many edible roots.}
\end{exe}

The \isi{quantifier} \tit{kam} \sqt{little, few, less} can modify \isi{nouns}, it can be nominalized (by adding the \isi{cross-categorical suffix} \tit{-ce}; plural \tit{-te} ) and it occurs in \is{compound verb}compound verbs with the meaning \sqt{decrease, diminish, become less} \refex{ex:The scandals (i.e. fights) did not diminish}.

\begin{exe}
	\ex	\label{ex:The scandals (i.e. fights) did not diminish}
	\gll	qːalmaqːar-te	kam	d-irχ-ul	akːu\\
		scandal\tsc{-pl}	little	\tsc{npl-}be\tsc{.ipfv-icvb}	\tsc{cop.neg}\\
	\glt	\sqt{The scandals (i.e. fights) did not diminish.}
\end{exe}


