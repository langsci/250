\chapter{Phonology}
\label{cpt:phonology}

This chapter treats Sanzhi phonology. Sanzhi is a typical East Caucasian language in terms of its phonology because it has a relative large consonant inventory (\refsec{sec:Consonant inventory}) and a medium vowel inventory (\refsec{sec:Vowel inventory}). Other topics covered in this chapter are the syllable structure (\refsec{sec:Syllable and word structure}), pharyngealization (\refsec{sec:Pharyngealization}), stress (\refsec{sec:Word stress}), and phonological and morphophonological alternations (\refsec{sec:Phonological and morphophonological alternations}). 


%%%%%%%%%%%%%%%%%%%%%%%%%%%%%%%%%%%%%%%%%%%%%%%%%%%%%%%%%%%%%%%%%%%%%%%%%%%%%%%%

\section{Consonant inventory}
\label{sec:Consonant inventory}

\reftab{tab:The consonant inventory of Sanzhi Dargwa} displays the consonant inventory. The table gives the phonemic value of the consonants, and in italics displays the orthographic representation used in this grammar (see also page xvii for the Cyrillic orthography). The three series of stops are, in the order given in the table: voiceless non-ejective, voiced, and voiceless ejective. The two series of fricatives are voiceless and voiced. All velars and uvulars also occur in labialized form. All voiceless non-ejective stops and fricatives (except for the pharyngeal/epiglottal and the glottal sounds) also occur as geminates (i.e. tense).
%
\begin{table}
	\caption{The consonant inventory of Sanzhi Dargwa}
	\label{tab:The consonant inventory of Sanzhi Dargwa}
	\footnotesize
	\begin{tabularx}{1\textwidth}[]{%
		>{\raggedright\arraybackslash}p{20pt}
		>{\centering\arraybackslash}X@{\hskip 0em}
		>{\centering\arraybackslash}X@{\hskip 0em}
		>{\centering\arraybackslash}X@{\hskip 1em}
		>{\centering\arraybackslash}X@{\hskip 0em}
		>{\centering\arraybackslash}X@{\hskip 0em}
		>{\centering\arraybackslash}X@{\hskip 1em}
		>{\centering\arraybackslash}X@{\hskip 1em}
		>{\centering\arraybackslash}X@{\hskip 0em}
		>{\centering\arraybackslash}X@{\hskip 1em}
		>{\centering\arraybackslash}X@{\hskip 0em}
		>{\centering\arraybackslash}X@{\hskip 0em}
		>{\centering\arraybackslash}X@{\hskip 1em}
		>{\centering\arraybackslash}X@{\hskip 0em}
		>{\centering\arraybackslash}X@{\hskip 1em}
		>{\centering\arraybackslash}X@{\hskip 1em}
		>{\centering\arraybackslash}X@{\hskip 1em}
		>{\centering\arraybackslash}X@{\hskip 0em}}
		
		\lsptoprule
 			{}		
		& 	\multicolumn{3}{c}{\hspace*{-0.5em}bilabial}
		& 	\multicolumn{3}{c}{\hspace*{-1em}alveolar}
		& 	\multicolumn{2}{c}{\hspace*{-1em}postalv}
		& 	\multicolumn{1}{c}{\hspace*{-1em}palatal}
		& 	\multicolumn{3}{c}{\hspace*{-1em}velar}
		& 	\multicolumn{2}{c}{\hspace*{-1em}uvular}
		& 	\rotatebox{90}{\parbox{17mm}{pharyngeal~/\\[-0.5mm]epiglottal}}
		&	\rotatebox{90}{glottal}\\
		\midrule
			stop		& [p] 	& [b] 	& [pʼ] 	& [t] 	& [d] 	& [tʼ]	& {} 	& {} 	& {} 	& [k] 	& [ɡ] 	& [kʼ] 	& [q] 	& [qʼ] 	& [ʡ] 	& [ʔ]\\
			{}		& \tit{p} & \tit{b} & \tit{pʼ} & \tit{t} & \tit{d} & \tit{tʼ} & {} & {} & {} & \tit{k} & \tit{g} & \tit{kʼ} & \tit{q} & \tit{qʼ} & \tit{ʡ} & \tit{ʔ}\\
			{}		& {}	& {} 	& {} 	& {} 	& {} 	& {}	& {} 	& {} 	& {} 	& [kʷ] 	& [ɡʷ] & [kʼʷ]	& [qʷ] & [qʼʷ]	& {} 	& {}\\
			{}		& {} 	& {} 	& {} 	& {} 	& {} 	& {} 	& {} 	& {} 	& {} 	& \tit{kʷ} & \tit{gʷ}	& \tit{kʼʷ} & \tit{qʷ} & \tit{qʼʷ} & {} & {}\\
			{}		& [pː] 	& {} 	& {} 	& [tː]	& {} 	& {} 	& {} 	& {} 	& {} 	& [kː] 	& {} 	& {} 	& [qː] 	& {} 	& {} 	& {}\\
			{}		& \tit{pː } 	& {} & {} & \tit{tː }	& {} & {} & {} 	& {} 	& {} 	& \tit{kː } & {} & {} 	& \tit{qː } & {} & {} 	& {}\\
			{}		& {}	& {}	& {}	& {}	& {}	& {}	& {}	& {}	& {}	& [kːʷ]	& {}	& {}	& [qːʷ]	& {}	& {}	& {}\\
			{}		& {}	& {}	& {}	& {}	& {}	& {}	& {}	& {}	& {}	& \tit{kːʷ} & {} & {}	& \tit{qːʷ} & {}	& {}	& {}\\[3mm]

			fricative	& {}	& {}	& {}	& [s] 	& [z]	& {}	& [ʃ]	& [ʒ]	& {}	& {}	& [x]	& {}	& [χ]	& [ʁ]	& [ħ]	& [h]\\
			{}		& {}	& {}	& {}	& \tit{s}	& \tit{z}	& {}	& \tit{š}	&  \tit{ž}	& {}	& {}	& \tit{x}	& {}	& \tit{χ}	& \tit{ʁ}	& \tit{ħ}	& \tit{h}\\
			{}		& {}	& {}	& {}	& {}	& {}	& {}	& {}	& {}	& {}	& {}	& [xʷ]	& {}	& [χʷ] & [ʁʷ]	& {}	& {}\\
			{}		& {}	& {}	& {}	& {}	& {}	& {}	& {}	& {}	& {}	& {}	& \tit{xʷ}	& {}	& \tit{χʷ} 	& \tit{ʁʷ}	& {}	& {}\\
			{}		& {}	& {}	& {}	& [sː]	& {}	& {}	& [ʃː]	& [ʃː]	& {}	& {}	& [xː]	& {}	& [χː]	& {}	& {}	& {}\\
			{}		& {}	& {}	& {}	& \tit{sː}& {}	& {}	& \tit{šː}	& \tit{šː} & {}	& {}	& \tit{xː} & {}	& \tit{χː} & {}	& {}	& {}\\
			{}		& {}	& {}	& {}	& {}	& {}	& {}	& {}	& {}	& {}	& {}	& {}	& {}	& [χːʷ]	& {}	& {}	& {}\\
			{}		& {}	& {}	& {}	& {}	& {}	& {}	& {}	& {}	& {}	& {}	& {}	& {}	& \tit{χːʷ} & {} & {}	& {}\\[3mm]

			affricate	& {}	& {}	& {}	& [t͡s] & [t͡sʼ]	& {}	& [t͡ʃ]	& [t͡ʃʼ] & {}	& {}	& {}	& {}	& {}	& {}	& {}	& {}\\
			{}		& {}	& {}	& {}	& \tit{c} & \tit{cʼ} & {} & \tit{č} & \tit{čʼ}  & {} & {} & {} & {}	& {}	& {}	& {}	& {}\\
			{}		& {}	& {}	& {}	& [t͡sː]	& {}	& {}	& [t͡ʃː]	& {}	& {}	& {}	& {}	& {}	& {}	& {}	& {}	& {}\\
			{}		& {}	& {}	& {}	& \tit{cː} & {} & {}	& \tit{čː}	& {} & {} & {}	& {}	& {}	& {}	& {}	& {}	& {}\\[3mm]

			nasal		& {}	& [m]	& {}	& {}	& [n]	& {}	& {}	& {}	& {}	& {}	& {}	& {}	& {}	& {}	& {}	& {}\\
			{}		& {}	& \tit{m} & {} & {}	& \tit{n} & {} & {}	& {}	& {}	& {}	& {}	& {}	& {}	& {}	& {}	& {}\\[3mm]

			liquid		& {}	& {}	& {}	& [r]	& [l]	& {}	& {}	& {}	& {}	& {}	& {}	& {}	& {}	& {}	& {}	& {}\\
			{}		& {}	& {}	& {}	&  \tit{r}	& \tit{l} & {} & {} & {} & {}	& {}	& {}	& {}	& {}	& {}	& {}	& {}\\[3mm]

			semi-		& {}	& [w]	& {}	& {}	& {}	& {}	& {}	& {}	& [j]	& {}	& {}	& {}	& {}	& {}	& {}	& {}\\
			vowel		& {}	& \tit{w} & {} & {}	& {}	& {}	& {}	& {}	& \tit{j} & {} & {}	& {}	& {}	& {}	& {}	& {}\\
		\lspbottomrule
	\end{tabularx}
\end{table}

The uvular stops [q] and [qʷ] have strong friction that makes them sound almost like affricates [q͡χ] and [q͡χʷ]. The friction is absent from the ejective [qʼ] and the geminates [qː] and [qːʷ].

The phonemic glottal stop is found in the noun \tit{beʔe} \sqt{blood} and at the end of some words, for instance in the root-final position of two verbs \tit{ha-ʔ-} (\tsc{pfv})\slash\tit{h-erʔ-} (\tsc{ipfv}) \sqt{say}, and \tit{b-erʔ-} (\tsc{pfv})\slash\tit{b-uʔ-} (\tsc{ipfv}) \sqt{rot} and the numeral \tit{kːaʔ-al} \sqt{eight}. Except for \tit{beʔe} \sqt{blood}, only loan words and names contain the glottal stop in root-medial position, e.g. \tit{žaˁbraˁʔil} (male name), \tit{daʔim} \sqt{continuation}.

A non-phonemic glottal stop, which is not written, occurs before word-initial non-pharyngealized vowels, e.g. \tit{aba} [ʔaba] \sqt{mother}, including vowel-initial words in compounds, for example \tit{ca-ibil} [t͡saʔibıl] \sqt{first} (one\tsc{-ord}) or occasionally at other morpheme boundaries of inflected words, for example, \tit{a-uk-un} (\tsc{neg-}eat\tsc{.ipfv-icvb}) \sqt{not eating} can be pronounced [aʔukʊn] or [aʊ̯kʊn].

The semivowel /w/ is realized as a voiced labiodental fricative [v] or as a labial-velar approximant [w]. 

In addition to the segments listed in \reftab{tab:The consonant inventory of Sanzhi Dargwa}, the voiceless labiodental fricative [f] is attested in the ideophone \tit{uf b-ik'ʷ-ij} \sqt{blow} (uf \tsc{hpl-}say\tsc{.ipfv-inf}) and in loan words, mostly from Russian, e.g. \textit{forel} \sqt{trout}. In older loans it had been replaced with [p], e.g. \textit{purma} \sqt{uniform} (< \textit{forma}).

All plain consonants occur in initial, medial, and final position. Geminates are never found in word-final position (more generally, syllable-final position). And there are three labialized consonants (\tit{q'ʷ, χʷ, ʁʷ}) which are also not attested in final position. \reftab{tab:Distribution of consonants@A} shows the distribution of consonants by means of example words. The table contains a number of morphologically complex words for which the relevant sound happens to occur at the end of the root, but within the stem because the root is followed by suffixes (the root is given in boldface) . 
%
\begin{table}
	\caption{Distribution of consonants}
	\label{tab:Distribution of consonants@A}
	\footnotesize
	\begin{tabularx}{1\textwidth}[]{%
		>{\raggedright\arraybackslash\itshape}p{10pt}
		>{\raggedright\arraybackslash\hangindent=0.5em\itshape}X
		>{\raggedright\arraybackslash\hangindent=0.5em\itshape}X
		>{\raggedright\arraybackslash\hangindent=0.5em\itshape}X}

		\lsptoprule
 			{}	&	\upshape initial			&	\upshape medial			&	\upshape final\\
		\midrule
			p	&	puq'a \sqtx{nest}			&	qupi \sqtx{hoe}			&	t'up \sqtx{finger}\\
			b	&	bec' \sqtx{wolf}			&	heba \sqtx{then}			&	urχːab \sqtx{mill}\\
			p’	&	p'aq' \sqtx{shake off}		&	q'aˁp'i \sqtx{shutter}		&	lap' \sqtx{wave}\\
			pː	&	pːiħaˁla \sqtx{feather}		&	k'apːur \sqtx{leaf}			&	\tmd\\
			t	&	tum \sqtx{hill}			&	kːaˁta \sqtx{cat}			&	it \sqtx{that}\\
			d	&	du \sqtx{1\tsc{sg}}			&	juldaš \sqtx{friend}			&	ca-d \sqtx{be-\tsc{npl}}\\
			t’	&	t'up \sqtx{finger}			&	kːat'i \sqtx{scarf}			&	t'ult' \sqtx{bread}\\
			tː	&	tːaˁm \sqtx{trap}			&	tːutːu \sqtx{beak}			&	\tmd\\	   
			k	&	kabc \sqtx{skin, fell}		&	dukala \sqtx{apron}		&	dek \sqtx{dung}\\
			g	&	gurmedi \sqtx{type of kerchief}	&	zigar \sqtx{hurry}			&	dig \sqtx{meat}\\
			k’	&	k'apːur \sqtx{leaf}			&	nik'a \sqtx{little, small}		&	hek' \sqtx{this/that (up)}\\
			kʷ	&	kʷač'a \sqtx{paw}			&	mikʷa \sqtx{fingernail}		&	nekʷ \sqtx{straw}\\
			gʷ	&	gʷargʷal \sqtx{onion}		&	targʷa \sqtx{weasel}		&	mergʷ \sqtx{lair, den}\\
			kʼʷ	&	k'ʷel \sqtx{two}			&	r-\textbf{ik'ʷ}-ij\footnote{The relevant roots of morphologically complex words are given in boldface. In these words, the respective sound occurs at the end of the root, but within the stem because the root is followed by suffixes.} \sqtx{\tsc{f-}say\tsc{.ipfv-inf}}	&	erk'ʷ \sqtx{river}\\
			kː	&	kːaˁta \sqtx{cat}			&	kːalkːi \sqtx{tree}			&	\tmd\\
			kːʷ	&	kːʷacːa \sqtx{mare}			&	\textbf{akːʷ}-ar \sqtx{be\tsc{.neg-prs}, without}&	\tmd\\
			q	&	qaˁr \sqtx{pear}			&	b-aqil \sqtx{much}			&	qːaq \sqtx{back}\\
			q’	&	q'aˁp'i \sqtx{shutter}		&	puq'a \sqtx{nest}			&	aq' \sqtx{flock}\\
			qʷ	&	qʷesːa \sqtx{ashes}			&	ha-\textbf{lqʷ}-an \upshape(up-direct\tsc{.ipfv-ptcp}) \sqtx{the climbing one} & daˁrqʷ \sqtx{barn}\\
			qʼʷ	&	q'ʷaˁl \sqtx{cow}			&	b-\textbf{elq'ʷ}-ij \sqtx{\tsc{n-}break\tsc{.pfv-inf}}	&	\tmd\\
			qː	&	qːap \sqtx{sack}			&	qːuˁlqːuˁ \sqtx{scythe}		&	\tmd\\
			qːʷ	&	qːʷaz \sqtx{goose}			&	\textbf{miriqːʷ}-e \sqtx{worm\tsc{-pl}}		&	\tmd\\
			s	&	sala \sqtx{in front, before}	&	qusmuk \sqtx{cupboard}		&	dus \sqtx{year}\\
			z	&	zija \sqtx{horsefly}			&	zize \sqtx{strawberry}		&	keruz \sqtx{slope}\\
			sː	&	sːika \sqtx{bear}			&	musːa \sqtx{place}			&	\tmd\\
			š	&	šal \sqtx{direction, side}		&	haniša \sqtx{summer}		&	juldaš \sqtx{friend}\\
			ž	&	žergʷa \sqtx{wasp}			&	ižal \sqtx{today}			&	hež \sqtx{this}\\
			šː	&	šːi \sqtx{village}			&	dešːa \sqtx{ancient}		&	\tmd\\
			x	&	xujal \sqtx{five}			&	xurxe \sqtx{sobber}		&	c'erx \sqtx{fat}\\
			xʷ	&	xʷit' \sqtx{whistle} (ideophone)	&	\textbf{ixʷ}-le \sqtx{early}			&	dirixʷ \sqtx{fog}\\
			xː	&	xːamxːa \sqtx{foam}			&	dirxːa \sqtx{stick}			&	\tmd\\
			χ	&	χat'a \sqtx{bowl}			&	alχni \sqtx{saw}			&	maχ \sqtx{barrow}\\
			ʁ	&	ʁajal \sqtx{twenty}			&	pːurʁum \sqtx{carriage}		&	qːabaʁ \sqtx{pumpkin}\\
			χʷ	&	χʷal-le \sqtx{much, a lot}		&	b-\textbf{iχʷ}-ij \sqtx{\tsc{n-}be\tsc{.pfv-inf}}	&	\tmd\\
			ʁʷ	&	ʁʷab \sqtx{ploughshare}		&	aʁʷal \sqtx{four}			&	\tmd\\
			χː	&	χːula \sqtx{big, tall}			&	duχːu \sqtx{clever}			&	\tmd\\
	\end{tabularx}
\end{table}
\begin{table}
%	\caption{Distribution of consonants (cont.)}
%	\label{tab:Distribution of consonants@B}
	\footnotesize
	\begin{tabularx}{1\textwidth}[]{%
		>{\raggedright\arraybackslash\itshape}p{10pt}
		>{\raggedright\arraybackslash\hangindent=0.5em\itshape}X
		>{\raggedright\arraybackslash\hangindent=0.5em\itshape}X
		>{\raggedright\arraybackslash\hangindent=0.5em\itshape}X}

		\midrule
 			{}	&	\upshape initial			&	\upshape medial			&	\upshape final\\
		\midrule
			χːʷ	&	χːʷe \sqtx{dog}			&	ha-d-\textbf{erχːʷ}-ij \sqtx{up\tsc{-n-}fulfill\tsc{.pfv-inf}} &	\tmd\\
			c	&	ca \sqtx{one}			&	q'aca \sqtx{he-goat}		&	kabc \sqtx{skin, fell}\\
			c’	&	c'il \sqtx{then}			&	imc'a \sqtx{superflous}		&	bec' \sqtx{wolf}\\
			cː	&	cːab \sqtx{sky}			&	kːancːa \sqtx{step}			&	\tmd\\
			č	&	čina-b \sqtx{where\tsc{-n}}	&	ʡaˁči \sqtx{work}			&	deč \sqtx{drinking}\\
			čʼ	&	č'an \sqtx{wind, storm}		&	kʷač'a \sqtx{paw}			&	ʡaˁmč' \sqtx{peel\tsc{.pfv.opt}}\\
			čː	&	čːaˁʡaˁl \sqtx{tomorrow, morning} &	ečːa \sqtx{she-goat}		&	\tmd\\
			ʔ	&	aba \sqtx{mother}			&	beʔe \sqtx{blood}			&	b-aʔ \sqtx{begin}\\
			ħ	&	ħaˁšak \sqtx{pot}			&	pːiħaˁla \sqtx{feather}		&	ʡaˁħ \sqtx{good}\\
			ʡ	&	ʡaˁbal \sqtx{three}			&	čːaˁʡaˁl \sqtx{tomorrow, morning}&	daˁʡ \sqtx{face}\\
			h	&	hel \sqtx{that}			&	buhem \sqtx{bundle}		&	b-ah \sqtx{owner}\\
			m	&	mikʷa \sqtx{fingernail}		&	gurmendi \sqtx{type of kerchief}	&	t'em \sqtx{smell}\\
			n	&	nekʷ \sqtx{straw}			&	haniša \sqtx{summer}		&	arin \sqtx{too much}\\
			r	&	rucːi \sqtx{sister}			&	rursːi \sqtx{girl, daughter}		&	q'ar \sqtx{herbs}\\
			l	&	lazun \sqtx{dough}			&	ʡuˁla \sqtx{wheel}			&	hel \sqtx{that}\\
			w	&	weral \sqtx{seven}			&	gawhar \sqtx{pupil}		&	alaw \sqtx{around}\\
			j	&	jangi \sqtx{new}			&	zija \sqtx{horsefly}			&	hej \sqtx{this}\\
		\lspbottomrule
	\end{tabularx}
\end{table}

Final voiced stops do not undergo devoicing. Final voiceless non-ejective stops (\tit{p, t, k, q}) are post-aspirated. Stops in final position are released. They are also released when a homorganic consonant follows, e.g. \tit{urek-c'al} \sqt{six-ty}, \tit{ħaˁžat-ce} \sqt{necessary} (need\tsc{-dd.sg}), \tit{c'elt-ne} \sqt{gravestone\tsc{-pl}}, \tit{le-d=nu} \sqt{exist-\tsc{npl=prt}}. If the voiceless stops [t], [k], and the voiceless affricate [t͡s] occur at morpheme boundaries and are followed by homorganic consonants, all consonants are fully pronounced and released \refex{ex:gemination A phon}. Neither [t] nor [k] nor [t͡s] become geminates under the described conditions, although gemination is otherwise a frequent process that applies across morpheme boundaries (\refsec{ssec:Gemination and degemination}). However, the ejective stop [k’] can turn into a plain stop as shown in the examples in \refex{ex:ikka phon}.
%
\begin{exe}
	\ex	\label{ex:gemination A phon}
	\begin{xlist}
		\ex	\tit{b-uˁc-ce}\slash\tit{b-uˁc-te} \sqt{thick} (\tsc{n-}thick\tsc{-dd.sg}\slash\tsc{n-}thick\tsc{-dd.pl})	\label{ex:bucce phon}
		\ex	\tit{tunt-ce}\slash\tit{tunt-te} \sqt{daring}	\label{ex:tuntce phon}
		\ex	\tit{ik'-ka}\slash\tit{hek'-ka} \sqt{\tsc{dem.up}\tsc{-abl}} (alternatively \tit{ik-ka}\slash\tit{hek-ka})	\label{ex:ikka phon}
	\end{xlist}
\end{exe}

All velar and uvular consonants occur in plain and labialized forms.The labialized velars and uvulars can be followed by all vowels except /u/. Labialization is mostly found with syllable-initial consonants, but as \reftab{tab:Distribution of consonants@A} shows, there are also words with labialized consonants in final position. In most words, labialization is restricted to one consonant per root, but there are a number of words with two labialized consonants, e.g. \tit{gʷagʷa} \sqt{flower}, \tit{gʷargʷal} \sqt{onion}, and \tit{xʷixʷit'} \sqt{pipe}. In addition to labialization in roots, deletion of the vowel [u] triggers labialization of the preceding consonant or following consonant (\refsec{ssec:Labialization and delabialization}). Labialized consonants are mostly found in nouns, numerals, adjectives, adverbs, and verbs and also attested in a few particles, but not in pronouns or suffixes. Labialization is absent from Standard Dargwa and therefore speakers who have been trained in the Standard Dargwa orthography do not write them in Sanzhi, although they pronounce them. Younger speakers often replace labialized consonants by plain consonants and change a preceding or following \tit{a} to \tit{o} (in speech and writing). Minimal pairs for some labialized consonants are given in \refex{ex:labialization phon@A}.
%
\begin{exe}
	\ex	\label{ex:labialization phon@A}
	\begin{xlist}
		\TabPositions{11em}
		\ex	\tit{d-elq'-ij} (\tsc{pfv}) \sqt{grind}	\tab \tit{d-elq'ʷ-ij} (\tsc{pfv}) \sqt{break}\label{ex:delqij phon}
		
		\ex	\tit{b-iχ-ij} (\tsc{pfv}) \sqt{tie, fasten}	\tab \tit{b-iχʷ-ij} (\tsc{pfv}) \sqt{be, become, be able}	\label{ex:bixwij phon}
		\ex	\tit{akri} \sqt{Akri} (place name)	\tab \tit{akʷri} \sqt{be\tsc{.neg.msd}} \label{ex:akri phon}
		\ex	\tit{ik'-i-j} \sqt{\tsc{dem.}up\tsc{-obl-dat}}	\tab \tit{ik'ʷ-ij} \sqt{say\tsc{.m.ipfv-inf}}	\label{ex:ikwij phon}
	\end{xlist}
\end{exe}

Geminates are always voiceless, non-ejective, and not aspirated. All voiceless non-ejective obstruents except for pharyngeal/epiglottal and glottal segments occur as geminates, and even a number of labialized consonants are geminates. The phonemic status of geminates is proven by the minimal pairs and minimal oppositions in \refex{ex:gemination B phon@A}.
%
\begin{exe}
	\ex	\label{ex:gemination B phon@A}
	\begin{xlist}
\TabPositions{13em}
		\ex	\tit{iχ-i-j} \sqt{\tsc{dem.down}\tsc{-obl-dat}}	\tab \tit{iχː-ij} \sqt{guard, protect, care}	
		\ex	\tit{buqij} (\tsc{pfv}) \sqt{run, go} \tab \tit{buqːij} \sqt{carry, bring}	\label{ex:buqij phon}
	
		\ex	\tit{bus-ij} \sqt{rain}	\tab \tit{b-usː-ij} \sqt{sleep, fall asleep} \label{ex:busij phon}
		
		\ex	\tit{b-ač-ij} (\tsc{pfv}) \sqt{smear, spread} \tab \tit{ačː-ij} \sqt{strike, hit onself} 	\label{ex:bacij phonA}
		
		\ex	\tit{b-ac-ij} (\tsc{pfv}) \sqt{plough} \tab \tit{acːi-j} \sqt{uncle\tsc{-dat}}	\label{ex:bacij phonB}
		\ex	\tit{het-i-j} \sqt{\tsc{dem-obl-dat}} \tab \tit{hetːi} \sqt{\tsc{dem.pl}}	\label{ex:hetij phonC}
		
	\end{xlist}
\end{exe}

Geminate fricatives are not always easy to identify because fricatives can be tense in emphatic pronunciation. But geminate stops and affricates are clearly audible as such, because there is a significant difference in the closure duration between singletons and geminates. Gemination can probably be analyzed as a difference between lax and tense consonants, but the exact phonetic properties of geminates still need to be clarified by future research.

In addition to their occurrence in stems, geminates occur at morpheme boundaries (see \refsec{ssec:Gemination and degemination} below). A few sonorants can also occur as tense consonants within roots (/n/, /m/, /l/, /r/ and /w/) and/or at morpheme boundaries, but their phonemic status needs further clarification. Only geminates of /n/, /r/ and /l/ are found in native items \refex{ex:gemination C1 phon}; the other sonorants are only found in loan words \refex{ex:gemination C2 phon}.
%
\begin{exe}
	\ex	\label{ex:gemination C1 phon}
	%\begin{xlist}
	\TabPositions{11em}
		\tit{t'unneq} \sqt{basket}	 \tab \tit{=malle} (emphatic particle)	\\
		\tit{-lla\slash -la} (genitive suffix)	\tab \tit{-lle\slash -le} (adverbialzing suffix)	\\
		\tit{urra} \sqt{foreign}	
	%\end{xlist}

	\ex	\label{ex:gemination C2 phon}
	%\begin{xlist}
	\TabPositions{11em}
			\tit{Allah} \sqt{Allah}	 \tab 	\tit{amma} \sqt{but}	\\
			\tit{sːurrat} \sqt{picture}	 \tab 	\tit{Maˁħaˁmma} (male personal name)		
	%\end{xlist}
\end{exe}


%%%%%%%%%%%%%%%%%%%%%%%%%%%%%%%%%%%%%%%%%%%%%%%%%%%%%%%%%%%%%%%%%%%%%%%%%%%%%%%%

\section{Vowel inventory}
\label{sec:Vowel inventory}

Sanzhi has four plain vowels and three pharyngealized vowels, of which one (\tit{iˁ}) is very rare and whose phonemic status needs further clarification. Pharyngealized vowels and pharyngealization are treated in \refsec{sec:Pharyngealization}. The vowels \tit{i}, \tit{e} and \tit{u} have lax and tense variants, whose distribution is not entirely clear. \reftab{tab:The vowel inventory of Sanzhi Dargwa} shows the vowel inventory with the orthographic symbols used in this grammar. 
%
\begin{table}
	\caption{The vowel inventory of Sanzhi Dargwa}
	\label{tab:The vowel inventory of Sanzhi Dargwa}
	\footnotesize
	\begin{tabularx}{0.58\textwidth}[]{%
		>{\raggedright\arraybackslash}p{20pt}
		>{\centering\arraybackslash}X@{\hskip 0em}
		>{\centering\arraybackslash}X@{\hskip 0em}
		>{\centering\arraybackslash}X}
		
		\lsptoprule
			{}	&	front			&	central	&	back\\
		\midrule
			high	&	[ı], [i]; [ıˁ], [iˁ]	&	{}		&	[u], [ʊ]; [ʊˁ]\\
			{}	&	\tit{i; iˁ}			&	{}		&	\tit{u; uˁ}\\[2mm]

			mid	&	[ε], [e]			&	{}		&	{}\\
			{}	&	\tit{e}			&	{}		&	{}\\[2mm]

			low	&	{}			&	[a]; [aˁ]	&	{}\\
			{}	&	{}			&	\tit{a; aˁ}	&	{}\\
		\lspbottomrule
	\end{tabularx}
\end{table}

There is one long vowel [aː], which is not phonemic, but occurs relatively frequently. It shows up only as sequences of homorganic vowels at morpheme boundaries \refex{ex:long monophthongs phon} (most often in negated verb forms), and occasionally as a stressed variant of short vowels. Long vowels mostly occur in open syllables, but can occasionally also be found in closed syllables. The negative present-tense copula-auxiliary normally has a short vowel, but when it is used as existential or locational copula, the first vowel becomes long \refex{ex:akkuu akkuu phon}.
%
\begin{exe}
	\ex	\label{ex:long monophthongs phon}
	\begin{xlist}
		\ex	\tit{aːgur} < \tit{a-ag-ur} \sqt{did not go} (\tsc{neg-}go\tsc{.pfv-pret})	\label{ex:aagur phon}
		\ex	\tit{aːčːib} < \tit{a-ačː-ib} \sqt{did not get} (\tsc{neg-}get\tsc{.pfv-pret})	\label{ex:aciib phon}
		\ex	\tit{čiaːžib} < \tit{či-a-w-až-ib} \sqt{did not see} (\tsc{spr-neg-m-}see\tsc{.pfv-pret})	\label{ex:ciawazib phon}
		\ex	\tit{b-aːkːu} \sqt{does not exist} vs. \tit{akːu} \sqt{be\tsc{.neg.prs}}	\label{ex:akkuu akkuu phon}
	\end{xlist}
\end{exe}

The long high front vowel [iː] is rarely found when spatial preverbs are prefixed to some verbs having [i] as stem vowel (see \refsec{ssec:Sequences of identical vowels} below for examples).

Sanzhi also has four diphthongs [ʊɪ̯], [aɪ̯], [εɪ̯], and [aʊ̯] that can be analyzed as consisting of two phonemes, a vowel, and a semivowel. Examples are given in \refex{ex:diphthongs phon}.
%
\begin{exe}
	\ex	\label{ex:diphthongs phon}
	%\begin{xlist}
	\TabPositions{11em}
		\tit{čuj} \sqt{\tsc{refl.pl.dat}}	\tab \tit{nejg} \sqt{milk}	\\
		\tit{ʁaj} \sqt{word, talk}		\tab \tit{caw} \sqt{be\tsc{.prs.m}; \tsc{refl.sg.m}}\\
		\tit{alaw} \sqt{around}	
	%\end{xlist}
\end{exe}




%%%%%%%%%%%%%%%%%%%%%%%%%%%%%%%%%%%%%%%%%%%%%%%%%%%%%%%%%%%%%%%%%%%%%%%%%%%%%%%%

\section{Syllable and word structure}
\label{sec:Syllable and word structure}

The minimal syllable consists of a single vowel. Initial vowels are always preceded by a non-phonemic glottal stop not indicated in the orthography. The syllables in monomorphemic native words are V, VC, VCC, CV, CVC and CVCC. In other words, syllables never have complex onsets, but can have complex codas, and the general syllable structure is shown in \refex{ex:syllable structure A phon}.
%
\begin{exe}
	\ex	(C)V(C)(C) \label{ex:syllable structure A phon}
\end{exe}

In the onset, every consonantal phoneme can occur (see \reftab{tab:Distribution of consonants@A} above for examples), whereas in the coda not all consonants are allowed. Note, however, that simple underived verbs underlie stronger restrictions because they can basically only have /l/ and /r/ in the onset as well as the pharyngeal stop /ʡ/ (in addition to gender exponents and consonants used in the deixis/elevation preverbs, see \refsec{sec:The structure underived verbal stems} for more details on the structure of verbs). Labialized consonants in syllable-final position are rarer than in syllable-initial position, but they are attested. Ejective consonants are also found. By contrast, geminate (i.e. tense) consonants are prohibited in the coda of syllables. Thus, geminate consonants in roots that happen to occur at the end of syllables in morphologically complex words, for instance after suffixation, are regularly shortened (see \refsec{ssec:Gemination and degemination} for examples). The nucleus consists of one vowel, which under certain circumstances can be long (\refsec{sec:Vowel inventory}). Some words can be seen as containing diphthongs, but diphthongs are analyzed as a sequence of a vowel and a semivowel. The most frequent syllable type is CV \refex{ex:syllable CV phon}, but VC \refex{ex:syllable VC phon} and CVC \refex{ex:syllable CVC phon} are also relatively common.
%
\begin{exe}
\TabPositions{11em}
	\ex	V	\label{ex:syllable V phon} \\
		\tit{u} \sqt{\tsc{2sg}}	\tab \tit{\tbf{a}.law} \sqt{around}	
\end{exe}	

\begin{exe}
\TabPositions{11em}
	\ex	CV	\label{ex:syllable CV phon} \\
		\tit{šːi} \sqt{village}	\tab \tit{qu} \sqt{field}	\\
		\tit{χːʷe} \sqt{dog} 	\tab\tit{ʁuˁ.ra} \sqt{hare}		
		\tit{a.\textbf{ba}} \sqt{mother}	\tab \tit{du.ra.zi} \sqt{threshing floor}	
\end{exe}	

\begin{exe}
\TabPositions{11em}
	\ex	VC	\label{ex:syllable VC phon} \\
		\tit{at} \sqt{\tsc{2sg.dat}}	\tab \tit{\tbf{eb}.la} \sqt{in spring}	
\end{exe}

\begin{exe}
\TabPositions{11em}
	\ex	CVC	\label{ex:syllable CVC phon} \\
		\tit{dus} \sqt{year}	\tab 	\tit{ʁaj} \sqt{word, talk}	\\
		\tit{mi.\tbf{riqʷ}} \sqt{worm}	\tab 	\tit{ʡaˁ.\tbf{jar}} \sqt{dance}	\\
		\tit{\tbf{ʡaˁn}.čːi} \sqt{earth, clay}	\tab 	\tit{qaˁš.qaˁr} \sqt{scab}	
\end{exe}

As mentioned in \refsec{sec:Vowel inventory}, there are no phonemic long vowels. Long vowels occasionally show up at morpheme boundaries or when the vowels are stressed or emphasized.

The only types of superheavy syllables are VCC \refex{ex:syllable VCC phon} and CVCC \refex{ex:syllable CVCC phon}, with only sonorants (/r/, /l/, /n/, /m/, /j/) and /b/ permitted in the position of the first consonant in the coda. Up to now I found only one exceptional noun that has a fricative before the second consonant, this being \tit{q'ast} \sqt{aim, intention, plan}. This noun is a loan ultimately from Arabic (qaṣ̊d); in Standard Dargwa its form is \tit{q'as}. The syllable-final consonants of superheavy syllables can only be plain stops, fricatives, or affricates including ejectives, geminates, and labialized stops (i.e. obstruents). Although they are mostly voiceless, there are also a few examples of voiced fricatives in the final position of (C)VCC syllables \refex{ex:syllable VCC phon}, \refex{ex:syllable CVCC phon}.
%
\begin{exe}
\TabPositions{13em}
	\ex	VCC	\label{ex:syllable VCC phon}\\
		\tit{ims} \sqt{moth}	\tab 		\tit{\textbf{alχ}.ni} \sqt{saw}	\\
		\tit{arʁ} \sqt{weather}	\tab 		\tit{irk} \sqt{threshing board}	
\end{exe}

\begin{exe}
\TabPositions{13em}
	\ex	CVCC	\label{ex:syllable CVCC phon}\\
		\tit{laˁbz} \sqt{mortar}	\tab 		\tit{daˁrqʷ} \sqt{barn}	\\
		\tit{nejg} \sqt{milk} 	\tab 		\tit{laˁmc'} \sqt{lightning}	\\
		\tit{kabc} \sqt{skin, fell}	\tab 		\tit{\textbf{c'ult}.mi} \sqt{plum}	\\
		\tit{jebš} \sqt{base}	\tab 		\tit{t'ult'} \sqt{bread}	\\
		\tit{b-ark} \sqt{inside}	\tab 		\tit{b-arx} \sqt{direct, straight} 	\\
		\tit{ku.\textbf{bart}} \sqt{pressed dung}	\tab 		\tit{qːuˁš.\textbf{tːunk'}} \sqt{rolling pin}	\\
		\tit{q'ast} \sqt{target, intention, idea}	
\end{exe}
 
There are no native words with syllable-initial consonant clusters. Consonant clusters in (older loans) are broken up by insertion of epenthetic vowels either between initial consonant clusters or before them. In the first case, the vowels vary and are often identical to the following vowel as the first three words in \refex{ex:epenthesis phon} show. In the second case, the vowel is /i/ as in the last two words:
%
\begin{exe}
	\ex	\label{ex:epenthesis phon}
	\TabPositions{11em}
	\tit{purust'in} \sqt{bed sheet}	\tab 	< Russian \textit{prostynja}	\\
		\tit{kːalas} \sqt{class}	\tab 	< Russian \textit{klass}	\\
		\tit{kiniga} \sqt{book}		\tab 	< Russian \textit{kniga}	\\
		\tit{ispirt} \sqt{alcohol}	\tab < Russian \textit{spirt} \\
				\tit{ispakulan} \sqt{speculator} \tab  < Russian \textit{spekuljant} 		
\end{exe}

Another possibility is to apply metathesis, though this process is very rare, for example Russian \tit{brigadir} > Sanzhi \tit{birgadir} \sqt{brigadier}.

The minimal word (i.e. free root) has the shape V (see the example in \refex{ex:syllable V phon} above). Minimal bound roots seem to consist of a single consonant and are only found among verbs. Examples are \tit{ha-ʔ-} (\tsc{pfv}) \sqt{say} and \tit{ka-xʷ-}\slash\tit{ha-xʷ-} (\tsc{pfv}) \sqt{pour, add}. These verbs obligatorily contain preverbs and the vowel can be analyzed as either belonging to the preverb (which results then in the monosegmental verb stems) or to the verbs, or two both (\tit{ha-} + \tit{aʔ-} > \tit{haːʔ-} > \tit{haʔ-}).


%%%%%%%%%%%%%%%%%%%%%%%%%%%%%%%%%%%%%%%%%%%%%%%%%%%%%%%%%%%%%%%%%%%%%%%%%%%%%%%%

\section{Pharyngealization}
\label{sec:Pharyngealization}

The most frequent pharyngealized vowel is [aˁ], but [uˁ] is also relatively common, whereas [iˁ] is restricted to very few words. The vowel [aˁ] has phonemic status in Sanzhi as the following minimal pairs and minimal oppositions show \refex{ex:pharyngealization minimal pairs A phon@A}.
%
\begin{exe}
	\ex	\label{ex:pharyngealization minimal pairs A phon@A}
	\begin{xlist}
\TabPositions{13em}
		\ex	\tit{šaˁm} \sqt{candle} \tab \tit{šam} \sqt{one year old ram}
		\ex	\tit{qːaˁp} (preverb) \sqt{twitch} \tab \tit{qːap} \sqt{sack}
		\ex	\tit{b-aˁʡ} \sqt{leaf, side, face} \tab \tit{b-aʔ} \sqt{end, beginning, edge}
		\ex	\tit{waˁħ} (or \tit{wah}) (interjection) \tab \tit{w-ah} \sqt{owner} (masc. singular)
		\ex	\tit{ʡaˁʁʷa-l} \sqt{fat\tsc{-advz}} \tab \tit{aʁʷ-al} \sqt{four\tsc{-num}}
		\ex	\tit{χːaˁb} \sqt{grave} \tab \tit{urχːab} \sqt{mill}
	\end{xlist}
\end{exe}

The vowel [uˁ] is far less frequent than [aˁ], and thus I so far have found only one minimal pair and only few examples of minimal oppositions in which the pharyngealized vowels only occur after uvular and pharyngeal sounds \refex{ex:pharyngealization minimal pairs B phon@A}.
%
\begin{exe}
	\ex	\label{ex:pharyngealization minimal pairs B phon@A}
	\begin{xlist}
	\TabPositions{15em}
		\ex	\tit{ruˁqː-uˁl}\slash\tit{ruˁqː-ul} \sqt{educate\tsc{-icvb}} \tab \tit{r-uqː-ul} \sqt{\tsc{f-}bring\tsc{-icvb}}
		\ex	\tit{ʁuˁb-e} \sqt{potato\tsc{-pl}} \tab \tit{qːajʁu-be} \sqt{sorrow\tsc{-pl}}
		\ex	\tit{ʡaˁχːuˁl} \sqt{guest}		\tab \tit{duχːu-l} \sqt{clever\tsc{-advz}}
		\ex	\tit{ʡuˁla} \sqt{wheel}		\tab \tit{ul-la} \sqt{eye\tsc{-gen}}
	\end{xlist}
\end{exe}

There are a few words that seem to have a pharyngealized high front vowel, e.g. \tit{b-iħ-iˁb} \sqt{they fought} (\tsc{hpl-}fight\tsc{.pfv-pret}), \tit{w-irʡ-iˁb} \sqt{(they) betrayed him} (\tsc{m-}betray\tsc{.pfv-pret}), \tit{b-iˁʡ-iˁj} (\tsc{n-}steal\tsc{.pfv-inf}), \tit{čːiˁħri} (village name). However, speakers are uncertain about the presence of [iˁ] in Sanzhi words, and further research is needed.

The vast majority of pharyngealized vowels occur in the adjacency of the uvular or pharyngeal consonants (see \reftab{tab:Distribution of consonants@A}). When pharyngealized vowels occur in roots that contain those consonants, the vowels most frequently follow the consonants, but can also precede them \refex{ex:uvular pharyngealphon1}. The respective consonants are \tit{q}, \tit{q'}, \tit{qː}, \tit{χ}, \tit{ʁ}, \tit{χ}ː, \tit{ʡ}, \tit{ħ} for \tit{aˁ} and \tit{uˁ}, and for \tit{aˁ} also the labialized consonants \tit{qʷ}, \tit{q'ʷ}, \tit{χʷ}, and \tit{ʁʷ}. The remaining uvular and pharyngeal consonants (\tit{qːʷ}, \tit{χːʷ}) are in general rare and I have not found any words so far that contain both the consonants and pharyngealized vowels.
%
\begin{exe}
	\ex	\label{ex:uvular pharyngealphon1}
\TabPositions{12em}
\tit{ʡuˁrʡ-e} \sqt{chicken\tsc{-pl}}	\tab \tit{ʡaˁnčːi} \sqt{clay, earth}	\\
\tit{q'aˁlči} \sqt{foot}	\tab \tit{ʁuˁc} \sqt{arrow}	\\
\tit{naˁq'iš} \sqt{drawing}	\tab \tit{q'ʷaˁl} \sqt{cow}	\\
\tit{daˁrqʷ} \sqt{barn}	
\end{exe}

The pharyngeal stop [ʡ] cannot be followed by non-pharyngealized [a] or [u], but only by non-pharyngealized [e] or [i], that is *[ʡa] and *[ʡu] \refex{ex:uvular pharyngealphon1}. And the pharyngealized vowels [aˁ] and [uˁ] are never followed by the glottal fricative [h], but only by the pharyngeal fricative [ħ], that is *[aˁh], *[uˁh].

Nevertheless, pharyngeal [aˁ] and to a lesser extent [uˁ] can also be found in stems that do not contain uvular or pharyngeal phonemes \refex{ex:not uvular phon} (see also the first minimal pair in \refex{ex:pharyngealization minimal pairs A phon@A} above).
%
\begin{exe}
	\ex	\label{ex:not uvular phon}
\TabPositions{12em}
		\tit{naˁs} \sqt{dirt}	\tab 		\tit{baˁs} \sqt{argument}	\\
		\tit{laˁbz} \sqt{mortar}	\tab 		\tit{čaˁč} \sqt{hearcut}	\\
		\tit{t'uˁ} \sqt{leg}	\tab 		\tit{čaˁt} \sqt{mud}	\\
		\tit{jaˁlči} \sqt{worker}	\tab 		\tit{šuˁra} \sqt{puddle}	

\end{exe}

There are a number of words that contain two pharyngealized vowels that can be either identical or both [aˁ] and [uˁ] \refex{ex:two pharyngealized vowels phon}.
%
\begin{exe}
	\ex	\label{ex:two pharyngealized vowels phon}
\TabPositions{12em}
		\tit{muˁʡaˁlim} \sqt{teacher}	\tab 	\tit{daˁquˁpːe} \sqt{wounds}	\\
		\tit{qaˁjquˁjte} \sqt{jaw}		\tab 	\tit{ʡaˁχːuˁl} \sqt{guest}	\\
		\tit{qːuˁnqːuˁpːe} \sqt{noses}	\tab 	\tit{qːuˁlqːuˁ-l} \sqt{scythe\tsc{-erg}} \\	
		\tit{q'aˁq'aˁ} \sqt{basin}		\tab 	\tit{naˁqaˁ} \sqt{oat}			\\
		\tit{ʡaˁrʡaˁ} \sqt{chicken}		\tab 	\tit{daˁrχaˁ} \sqt{evening}	
\end{exe}

There is also some variation with those words that contain two pharyngealized vowels, in the sense that some speakers pharyngealize only one vowel whereas others pharyngealize both. The vowel that is optionally pharyngealized can be the first \refex{ex:pharyngealization of first vowel or both vowels} or the second vowel \refex{ex:pharyngealization of second vowel or both vowels}.
%
\begin{exe}
	\ex	{pharyngealization of first vowel or both vowels}\label{ex:pharyngealization of first vowel or both vowels}
	\begin{xlist}
		\ex	\tit{puˁšːuˁk'}\slash\tit{puˁšːuk'} \sqt{blister}
		\ex	\tit{maˁlʡuˁn}\slash\tit{maˁlʡun} \sqt{snake}
		\ex	\tit{ʡuˁruˁs}\slash\tit{ʡuˁrus} \sqt{Russian}
	\end{xlist}

	\ex	{pharyngealization of second vowel or both vowels}\label{ex:pharyngealization of second vowel or both vowels}
	\begin{xlist}
		\ex	\tit{daˁʡaˁna}\slash\tit{daʡaˁna} \sqt{secret, secretly}
		\ex	\tit{durħuˁ}\slash\tit{duˁrħuˁ} \sqt{boy, son}
	\end{xlist}
\end{exe}

I found very few words that contain only one vowel that can optionally be pharyngealized \refex{ex:optional pharyngealization vowels phon}.
%
\begin{exe}
	\ex	\label{ex:optional pharyngealization vowels phon}
	\begin{xlist}
		\ex	\tit{duħi}\slash\tit{duˁħi} \sqt{snow}
		\ex	\tit{zaˁnʁ}\slash\tit{zanʁ} \sqt{ring} (ideophone)
		\ex	\tit{čaˁʁir}\slash\tit{čaʁir} \sqt{wine}
	\end{xlist}
\end{exe}

There are also two derivational suffixes containing pharyngealized vowels. The suffixes are \tit{-q'aˁ} and \tit{-uˁq'}. They are not productive and derive agent nouns from other nouns, infinitives, and parts of compound verbs form (\refsec{ssec:Agent nouns with -q'aˁ, -uˁq' and -kar}). These suffixes do not have allomorphs with plain vowels.

Pharyngealization is a suprasegmental feature that spreads to inflectional prefixes and suffixes, even in those words that do not contain pharyngealized vowels in the root, but uvular/pharyngeal consonants. Only those prefixes and suffixes are affected that start with the vowels \tit{a} and \tit{u} such that in affixes only the pharyngealized vowels [aˁ] and [uˁ] occur, but no other pharyngealized vowels. Other affixes that contain the same vowels but start with a consonant do not have pharyngealized variants, for instance the vowel in the local participle suffix \tit{-na} cannot be pharyngealized \refex{ex:pharyngealization na phon}. 
%
\begin{exe}
	\ex	\label{ex:pharyngealization na phon}
\TabPositions{12em}
	\tit{guči d-urq-aˁdi} \sqt{I gathered} \tab vs. \tit{guči d-urq-na} \sqt{the place of gathering}
\end{exe}

In the case of the negation prefix \textit{a-} this leads to a long pharyngealized vowel (see \refsec{ssec:Sequences of identical vowels} for long vowels resulting from sequences of identical vowels):

\begin{exe}
	\ex	\label{ex:pharyngealization long vowel}
\begin{xlist}
\ex	\tit{či-aˁ-aˁħ-un} > \tit{či-aˁːħ-un} \sqt{did not fly on (something)} (\tsc{spr-neg-}fly\tsc{.pfv-pret})
\ex	\tit{aˁ-w-aˁq-ib} > \tit{aˁ-aˁq-ib} > \tit{aˁːq-ib} \sqt{did not hit} (msc.) (\tsc{neg-}hit\tsc{.pfv-pret})
	\end{xlist}
\end{exe}	


Furthermore, only affixes in immediately preceding or following syllables are affected. Pharyngealization does not spread over the entire word. For nouns the suffixes containing pharyngealized vowels are the plural and oblique plural suffixes, as well as one suffix deriving actions nouns (\tit{-a}; \refsec{ssec:Action and event nouns with -uti and -a}). For verbs, the suffixes can be derivational (the causative suffix, the spatial preverbs) or inflectional (negation prefixes \tit{a-} and \tit{ma-}, various TAM suffixes). Examples are provided in \refex{ex:pharyngealization spread phon}.
%
\begin{exe}
	\ex	\label{ex:pharyngealization spread phon}
\begin{xlist}
		\ex \tit{qːuˁnqː-uˁpːe} \sqt{nose\tsc{-pl}}	
		\ex \tit{baliqː-aˁ-lla} \sqt{fish\tsc{-obl.pl-gen}} 
		\ex \tit{ruˁrq-uˁl} \sqt{boiling} (boil\tsc{-icvb})	
		\ex \tit{b-iħ-aˁq-ib} \sqt{made fight} (\tsc{hpl-}fight\tsc{-pfv-caus-pret}) 
		\ex \tit{b-iʡ-uˁn} \sqt{stole} (\tsc{n-}steal\tsc{.pfv-pret})	
		\ex	\tit{b-aˁq-aˁjaˁ} \sqt{hit it!} (\tsc{n-}hit\tsc{.pfv-imp.pl})
\end{xlist}
\end{exe}

The pharyngealized articulation associated with the vowel is maintained when the vowel changes, that is when there is vowel mutation \tit{a} > \tit{u}, as, for instance, with plural forms of some nouns \refex{ex:vowel mutation plural phon}.
%
\begin{exe}
	\ex	\label{ex:vowel mutation plural phon}
	\begin{xlist}
		\ex	\tit{ʡaˁrʡaˁ} \sqt{chicken} > \tit{ʡuˁrʡ-e} \sqt{chicken\tsc{-pl}}
		\ex	\tit{q'ʷaˁl} \sqt{cow} > \tit{q'uˁl-e} \sqt{cow\tsc{-pl}}
	\end{xlist}
\end{exe}

There is one verb \sqt{go} that occurs without a root vowel when prefixes are attached and with a root vowel that can be pharyngealized or plain otherwise. The suffixes used with this verb are obligatorily pharyngealized, whereas for prefixes pharyngealization is optional \refex{ex:go pharyngealization phon}.
%
\begin{exe}
	\ex	\label{ex:go pharyngealization phon}
	\begin{xlist}
		\ex	\tit{maˁ-q'-aˁtːa} \sqt{do not go!} (\tsc{proh-}go\tsc{-proh.sg})
		\ex	\tit{b-uq'-aˁq-ij} \sqt{to make it go} (\tsc{n-}go\tsc{-caus-inf})
		\ex	\tit{saˁ-q'-aˁn} vs. \tit{sa-q'-aˁn} \sqt{going} (\tsc{hither}-go\tsc{-ptcp})
	\end{xlist}
\end{exe}

At least with some affixes, pharyngealization is optional, and one can find one and the same inflected word form with and without affixes that contain pharyngealized vowels \refex{ex:optional pharyngealization phon}.
%
\begin{exe}
	\ex	\label{ex:optional pharyngealization phon}
	\begin{xlist}
	\TabPositions{13em}
		\ex	\tit{b-aˁħ-uˁn-ce} vs. \tit{b-aˁħ-un-ce} 		\tab \sqt{wet} 
		\ex	\tit{guči b-aˁq-aˁraj} vs. \tit{b-aˁq-araj}	\tab \sqt{gather} (gather \tsc{hpl-}assemble\tsc{-subj})
		\ex	\tit{kaˁ-q-aˁja!} vs. \tit{ka-q-aˁja!} 		\tab \sqt{drag!} (\tsc{down}-drag\tsc{.pfv-imp.pl})
	\end{xlist}
\end{exe}




Pharyngealization includes loan words, even recent borrowings from Russian \refex{ex:loan pharyngealization phon}, which are not pharyngealized in the donor language. It is even noticeable when (older) Sanzhi people speak Russian.
%
\begin{exe}
	\ex	\label{ex:loan pharyngealization phon}
	\TabPositions{12em}
		\tit{čaˁj} \sqt{tea}	\tab 	< Russian \textit{čaj}	 \\
		\tit{šaˁbk'a} \sqt{hat}	\tab 	< Russian 	\textit{šapka} \\
		\tit{ʡaˁšibkːa} \sqt{mistake}	\tab 	< Russian \textit{ošibka}	 \\	
		\tit{šljaˁp'a} \sqt{hat} \tab 	< Russian	\textit{šljapa} \\
		\tit{ʡaˁčkːabe} \sqt{glasses} \tab 	< Russian	\textit{očki} \\
		\tit{luˁkːujte} \sqt{lungs} \tab 	< Russian	\textit{legkie} 
\end{exe}


%%%%%%%%%%%%%%%%%%%%%%%%%%%%%%%%%%%%%%%%%%%%%%%%%%%%%%%%%%%%%%%%%%%%%%%%%%%%%%%%

\section{Word stress}
\label{sec:Word stress}

Stress is not a very prominent category in Sanzhi Dargwa. The stress is quite weak and the stress properties of words are very hard to determine. Stress is dynamic and has no fixed position, but it is lexicalized. There are a few examples of minimal pairs or near minimal pairs that differ only in stress \refex{ex:stress minimal pairs phon}.
%
\begin{exe}
	\ex	\label{ex:stress minimal pairs phon}
	\begin{xlist}
\TabPositions{14em}
		\ex	\tit{búk’ul} \sqt{freezing}		\tab \tit{b-uk’úl} \sqt{thin, slender}
		\ex	\tit{sːála} \sqt{wedge}		\tab \tit{salá} \sqt{in front}		
		\ex	\tit{ákːʷar} \sqt{without} (postposition) \tab \tit{akːʷár} \sqt{not being} (participle)
		\ex	\tit{hána} \sqt{cast iron} \tab \tit{haná} \sqt{now}
	\end{xlist}
\end{exe}

Some affixes attract stress, so that the position of stress in roots and in inflected word forms of one and the same lexeme may differ. For instance, plural suffixes of nouns normally attract stress \refex{ex:plural attract stress phon}.
%
\begin{exe}
	\ex	\label{ex:plural attract stress phon}
	\begin{xlist}
		\ex	\tit{qːap} \sqt{sack} > \tit{qːup-né} \sqt{sack\tsc{-pl}}
		\ex	\tit{kur} \sqt{pit} > \tit{kur-mé} \sqt{pit\tsc{-pl}}
	\end{xlist}
\end{exe}

The factors influencing placement of stress require further research.


%%%%%%%%%%%%%%%%%%%%%%%%%%%%%%%%%%%%%%%%%%%%%%%%%%%%%%%%%%%%%%%%%%%%%%%%%%%%%%%%

\section{Phonological and morphophonological alternations}
\label{sec:Phonological and morphophonological alternations}

Sanzhi Dargwa has a variety of phonological and morphophonological alternations that affect vowels and consonants. Some of the processes that target vowels result from the fact that hiatus is not allowed, and therefore the underlying word forms have to be changed. A number of processes such as vowel deletion, alternation in the form of enclitics\slash affixes, degemination are syllable repair mechanisms, but others do not serve this function.

Processes affecting vowels are vowel deletion \refsec{ssec:Vowel deletion (vowel syncope)}, glide insertion \refsec{ssec:Glide insertion}, glottal stop insertion \refsec{ssec:Glottal stop insertion}, long vowels resulting from sequences of identical vowels \refsec{ssec:Sequences of identical vowels}, pharyngealization and formation of diphthongs \refsec{ssec:Other general processes affecting vowels}, and vowel mutation \refsec{ssec:Vowel mutation (apophony)}.

Processes affecting mainly consonants are assimilation \refsec{ssec:Assimilation}, palatalization \refsec{ssec:Palatalization}, labialization\slash delabialization \refsec{ssec:Labialization and delabialization}, gemination (in combination with devoicing) and degemination \refsec{ssec:Gemination and degemination} (although labialization and delabialization also have an effect on vowels).


% --------------------------------------------------------------------------------------------------------------------------------------------------------------------------------------------------------------------- %

\subsection{Vowel deletion (vowel syncope)}
\label{ssec:Vowel deletion (vowel syncope)}

Vowel deletion (vowel syncope) is one means of avoiding two subsequent vowels at a morpheme juncture. It is mainly found with encliticized negative auxiliaries. There are three types of vowel deletion within the domain of verbal morphology. First, sequences of identical vowels might lead to the deletion of one vowel or to vowel deletion in combination with mutation (\refsec{ssec:Vowel mutation (apophony)}). Second, the initial vowel of the negative auxiliary is deleted when the auxiliary is used as an enclitic \refex{ex:vowel syncope verbs phon}.
%
\begin{exe}
	\ex	\label{ex:vowel syncope verbs phon}
	\begin{xlist}
		\ex	\tit{biχuble + akːu} > \tit{b-iχ-ub-le=kːu} (\tsc{n-}become\tsc{.pfv-pret-cvb=}be\tsc{.neg})			
		\ex	\tit{qːuʁace + akːu} > \tit{qːuʁa-ce=kːu} (beautiful-\tsc{dd.sg=}be\tsc{.neg})		
		\ex	\tit{kabišːible + akːʷadi} > \tit{ka-b-išː-ib-le=kːʷadi}\newline\hspace*{1em}(\tsc{down}\tsc{-n-}put\tsc{.pfv-pret-cvb=}be\tsc{.pst.neg.1}) 
		\ex	\tit{χʷalle + akːʷi} > \tit{χʷal-le=kːʷi} (big\tsc{-advz=}be\tsc{.pst.neg})
	\end{xlist}
\end{exe}

Third, there is one verb of which the root vowel \tit{u} is deleted when the gender agreement is masculine singular and the verbal root is preceded by the deixis/elevation preverbs or the negation prefixes. These preverbs and prefixes end in \tit{a}. The process is accompanied by the labialization of the root consonant, that is, the labial feature turns from a vowel feature into a consonantal feature \refex{ex:habulqan phon}. See \refsec{ssec:Labialization and delabialization} below for more examples with the same verb stem.
%
\begin{exe}
	\ex	\tit{ha-(w)-ulq-an} > \tit{halqʷan} \sqt{the one that goes upwards}\newline\hspace*{1em}(\tsc{up-m-}direct\tsc{.ipfv-ptcp}) (compare with \tit{ha-b-ulq-an} \tsc{up-n-}direct\tsc{.ipfv-ptcp})	\label{ex:habulqan phon}
\end{exe}

As for nominal morphology, vowel deletion that is not caused by sequences of vowels is a regular component of plural formation. The deletion of \tit{a}, \tit{i}, or \tit{u} in the final syllable of mostly dissyllabic nouns is found with the suffixes \tit{-be}, \tit{-me}, \tit{-re}, \tit{-e}, \tit{-ne}, \tit{-upːe}, \tit{-urbe}, and \tit{-ube} (\refsec{sec:nounnumber}). If the last (usually stem-final) obstruent is a geminate it undergoes degemination. Examples are:
%
\begin{exe}
	\ex	\label{ex:vowel syncope nouns phon}
	\begin{xlist}
		\ex	\tit{šuša} > \tit{šuš-ne} \sqt{bottle}
		\ex	\tit{durħuˁ} > \tit{durħ-ne} \sqt{boy, son},
		\ex	\tit{daˁqaˁ} > \tit{daˁq-uˁpːe} \sqt{wound}
		\ex	\tit{rursːi} > \tit{rurs-be} \sqt{girl, daughter}
		\ex	\tit{murgul} > \tit{murgl-e} \sqt{man}
		\ex	\tit{k'apːur} > \tit{k'apr-e} \sqt{leaf}
	\end{xlist}
\end{exe}


% --------------------------------------------------------------------------------------------------------------------------------------------------------------------------------------------------------------------- %

\subsection{Alternations in the form of enclitics\slash suffixes}
\label{ssec:Alternations in the form of enclitics / suffixes}

There are suffixes and enclitics that have allomorphs whose use depends on the syllable structure of the item to which the suffixes or enclitics are added. The general rule is that suffixes\slash enclitics consisting of a single consonant are attached to vowel-final words and suffixes\slash enclitics of the form CV to consonant-final words. Relevant suffixes and enclitics are:
%
\begin{itemize}
	\item	the ergative suffix \tit{-l}\slash\tit{-li} (\refsec{sssec:Ergative}), e.g. \textit{du-l} (\tsc{1sg-erg}) vs. \textit{kulpat-li} (family-\tsc{erg})
	\item	the adverbializer \tit{-l}\slash\tit{-le} (\refsec{ssec:The adverbializer -le}), e.g. \textit{razi-l} \sqt{happily} vs. \textit{c'aq'-le} \sqt{strongly}
	\item	the emphatic enclitic \tit{=n}\slash\tit{=nu} (\refsec{ssec:The enclitic =n(u)}), e.g. \textit{akːu=n} (be\tsc{.neg.prs=prt}) vs. \textit{le-d=nu} (exist-\tsc{npl=prt})
	\item	the enclitic for polar questions \tit{=w}\slash\tit{=uw} (\refsec{sec:Simple polar questions and disjunctive polar questions}) \textit{arg-ul=de=w?} (go.\tsc{ipfv-icvb=2sg=q}) \sqt{Are you going?} vs. \textit{le-b=uw?} (exist-\tsc{n=q}) \sqt{Does it exist?}
\end{itemize}

The last enclitic has another allomorph \tit{=ew} that only occurs after the imperfective converb suffix \tit{-ul}, perhaps to avoid having two identical vowels in two adjacent syllables (although this is generally allowed).

One can argue that the underlying form of the suffixes is the one with the vowel (\tit{-li}, \tit{-le}) and that the vowel is deleted in the appropriate contexts, although there is no phonotactic need for such a deletion and it even goes against the general preference of open syllables in final position.


% --------------------------------------------------------------------------------------------------------------------------------------------------------------------------------------------------------------------- %

\subsection{Glide insertion}
\label{ssec:Glide insertion}

Glide insertion represents a regular form of allomorphy. It is only found with the palatal glide \textit{j} and only before certain suffixes or enclitics that start with the vowel \textit{a}. The respective suffixes and enclitics are:
%
\begin{itemize}
	\item	the derivational suffix used to form the numerals 2\tnd10, 20, as well as 100: \tit{-al}\slash\tit{-jal}, e.g. \tit{aʁʷ-al} \sqt{four} vs. \tit{xu-jal} \sqt{five} 
	\item	the derivational suffix \sqt{X-times} for the formation of multiplicative numerals \tit{-na}\slash\tit{-jna} 
	\item	one of the allomorphs of the spatial case suffix of the \tsc{loc}-series: \tit{-a}\slash\tit{-ja} 
	\item	optional marker for non-indicative verb forms that  serves as address particle for plural addressees \tit{-a}\slash\tit{-ja}, e.g. in the imperative plural suffix \tit{-ene-ja} (alternative variant of \tit{-ene}); the particle \tit{ma} vs. \tit{ma-ja} `Here, take!'
	\item	the enclitic marking content questions: \tit{=e} after consonants and \tit{=ja} after vowels 
	\item	the enclitic marking embedded questions and forming specific indefinite pronouns: \tit{=el} after consonants and \tit{=jal} after vowels, e.g. \tit{ča=jal} \sqt{somebody}, \tit{ce=jal}	\sqt{something} vs. \tit{čina-b=el} \sqt{somewhere} 
\end{itemize}


% --------------------------------------------------------------------------------------------------------------------------------------------------------------------------------------------------------------------- %

\subsection{Glottal stop insertion}
\label{ssec:Glottal stop insertion}

Another means of avoiding two adjacent vowels is the insertion of a glottal stop. This occurs when spatial preverbs and negation prefixes are added to vowel-initial verbs. In the following examples the glottal stops are given (although they are normally not written in this position).
%
\begin{exe}
	\ex	\label{ex:glottal stop inversion phon}
	\begin{xlist}
		\TabPositions{3em}
		\ex	\tit{iʔa:}	\tab	\tit{biʔat'un} < \tit{b-i-at'-un} (\tsc{n-in}-stick.\tsc{pfv-pret})
		\ex	\label{Don't take it!Phon} \tit{aʔi:} 	\tab	\tit{maʔisːit} (`Don't take it!')	(alternative: \tit{majsːit})	
		\ex	\tit{uʔi:}	\tab	\tit{guʔičib} < \tit{gu-ič-ib} 	(\tsc{sub}-occur.\tsc{pfv.m-pret})
		\ex	\tit{uʔa:}	\tab	\tit{gu-ʔagur, gu-ʔa-lik'un}	(\tsc{sub}-go.\tsc{pfv.pret}, \tsc{sub-neg}-listen.\tsc{icvb})
		\ex	\label{hitherGOPHON} \tit{aʔa:} 	\tab	\tit{sa-ʔargul} (\tsc{hither}-go.\tsc{ipfv.icvb}) (alternative: \tit{saːrgul})	
		\ex	\tit{iʔi:}	\tab	\tit{čiʔiʁij} \sqt{understand}	
		\ex	\label{insideADDPHON} \tit{iʔi:}	\tab	\tit{biʔiʁitːe} (\tsc{inside}.add.\tsc{ipfv.2sg}) (alternative: \tit{biːʁitːe})	
	\end{xlist}
\end{exe}

As can be seen in the above examples, in some cases alternative processes can be applied, namely the formation of a long vowel in the case of a sequence of two identical vowels \refex{hitherGOPHON}, \refex{insideADDPHON}, and the change from \tit{i} to \tit{j} if the first vowel is \tit{a} and the second is \tit{i} \refex{Don't take it!Phon}. The same can be observed for combinations of three vowels or compound verbs. For instance, if the verb form \tit{sa-ʔargul} given in \refex{hitherGOPHON} is negated, we get \tit{sa-a-ʔargul}. In principle, it is possible to pronounce all three vowels separately, although this is not the preferred variant in practice. Similarly, in compound verbs the first part can be pronounced as a phonologically independent word or the two vowels can fuse across word boundaries, for example \tit{dak'u uqandel} vs. \tit{dak'uːqandel} \sqt{if he appeared}. The precise conditions of glottal stop insertion still require clarification.


% --------------------------------------------------------------------------------------------------------------------------------------------------------------------------------------------------------------------- %

\subsection{Sequences of identical vowels}
\label{ssec:Sequences of identical vowels}

Long vowels can be the result of a sequence of two identical vowels or of the vowel \tit{i} plus the semivowel \tit{j}. The latter happens when the dative suffix is added to nominals. The only long vowels are [aː], [aˁː], [iː], and in one case after vowel mutation [eː]. The emergence of long vowels from two identical vowels is in many cases optional, with the insertion of a glottal stop being the usual alternative. Two identical vowels at morpheme boundaries occur only with verbs, either when the negation prefixes \tit{a-} and \tit{ma-} are used or with spatial preverbs. Note that in particular with the negation prefix it is the long vowel that carries the meaning of negation. If the two identical vowels would be shortened, the resulting verb form would be identical to the affirmative verb form and the negative meaning would be lost. Examples of the sequence are given in \refex{ex:identical vowels A phon} and for the long pharyngealized low central vowel in \refex{ex:pharyngealization long vowel}.
%
\begin{exe}
	\ex	\label{ex:identical vowels A phon}
	\begin{xlist}
		\ex	\tit{a-ag-ur} > \tit{aːgur} \sqt{did not go} (\tsc{neg-}go\tsc{.pfv-pret})
		\ex	\tit{sa-arg-ul} > \tit{saːrgul} \sqt{coming here} (\tsc{hither}-go\tsc{.ipfv-icvb})\newline\hspace*{1em}(alternative: \tit{saʔargul}, \refex{hitherGOPHON})
		\ex	\tit{qum a-art-u} > \tit{qum aːrtu} \sqt{does not forget}\newline\hspace*{1em}(forget \tsc{neg-}forget\tsc{.ipfv-prs.3}) 
		\ex	\tit{sa-a-ka-b-išː-ib} > \tit{saːkabišːib} \sqt{did not put down}\newline\hspace*{1em}(\tsc{in.front-neg-down-n-}put\tsc{.pfv-pret})
		\ex	\tit{a-erč-ur} > \tit{eːrčur}\slash\tit{aʔerčur} \sqt{did not saw} (\tsc{neg-}saw\tsc{.pfv-pret})
		\ex	\tit{b-i-iʁ-itːe} > \tit{biːʁitːe} \sqt{you add it in} (\tsc{n-in}-add\tsc{.ipfv-prs.2sg})\newline\hspace*{1em}(alternative: \tit{biʔiʁitːe}, \refex{insideADDPHON})
	\end{xlist}
\end{exe}

Furthermore, the masculine singular agreement prefix \tit{w} can optionally be deleted when it occurs between two identical vowels (and, more generally, before \tit{i}). This process also leads to long vowels [aː] and [iː] \refex{ex:identical vowels B phon}.
%
\begin{exe}
	\ex	\label{ex:identical vowels B phon}
	\begin{xlist}
		\ex	\tit{a-w-at-ur} > \tit{aːtur} \sqt{did not let him} (\tsc{neg-m-}let\tsc{.pfv-pret})\newline\hspace*{1em}(alternative: \tit{a-w-at-ur})
		\ex	\tit{ma-w-ax-utːa} > \tit{maːxutːa} \sqt{do not go!} (\tsc{proh-m-}go\tsc{-proh.sg})\newline\hspace*{1em}(alternative: \tit{ma-w-ax-utːa})
		\ex	\tit{w-i-a-w-ax-an=de} >  \tit{wiʔaːxande} \sqt{you will not go inside}\newline\hspace*{1em}(\tsc{m-}\tsc{in-neg-m-}go\tsc{-ptcp=2sg}) (alternative: \tit{w-i-a-w-ax-an=de})
		\ex	\tit{či-w-ig-ul=de} > \tit{čiːgulde} \sqt{you see him} (\tsc{spr-m-}see\tsc{.ipfv-icvb=2sg})\newline\hspace*{1em}(alternative: \tit{čiwigulde})
	\end{xlist}
\end{exe}


% --------------------------------------------------------------------------------------------------------------------------------------------------------------------------------------------------------------------- %

\subsection{Other general processes affecting vowels: Pharyngealization and formation of diphthongs}
\label{ssec:Other general processes affecting vowels}

Pharyngealization is a frequent process that is attested with verbal and nominal affixes containing \tit{u} or \tit{a}. The pharyngealization feature of verbal and nominal stems spreads to the closest prefixes or suffixes, but not to the entire word. The nominal affixes that have pharyngealized allomorphs are the plural suffix \tit{-upːe}, oblique plural suffix \tit{-a} and the suffix \tit{-a} deriving actions nouns from verbs. The verbal suffixes are a variety of derivational and inflectional suffixes. See \refsec{sec:Pharyngealization} above for more details and relevant examples.

There are two diphthongs that arise when vowels are followed by the semivowels \tit{w} and \tit{j}. The diphthongs [aɪ̯] and [aʊ̯], written \tit{aj} and \tit{aw}, are found in a few roots \refex{ex:diphthongs phon}. They also arise during certain inflectional or derivational processes. The first diphthong is attested with verbs having \tit{i} as the root vowel and consisting only of one consonant, that is, verbs of the structure \tit{(b-)iC(ː)}. They may have or may not have gender prefixes. When a spatial preverb \tit{ka-}, \tit{ha-},\tit{ sa-}, or the negation prefixes (\tit{a-}, \tit{ma-}) are added, the result is \tit{a} + \tit{i} > \tit{aj}. For verbs with gender prefixes the process only applies when the gender prefix \tit{w-} for masculine singular is omitted, which is always possible for verbs that have the root vowel \tit{i}. The forms with the gender prefixes \tit{b-} or \tit{r-} that do not contain the diphthong are given in brackets.
%
\begin{exe}
	\ex	{\tit{a} + \tit{i} > \tit{aj}} (iC(ː) > jC(ː))\label{ex:a i aj phon}
	\begin{xlist}
		\ex	\tit{ha-(w)-icː-ij} > \tit{hajcːij} \sqt{to stand up} (\tsc{up-m-}stand\tsc{.pfv-inf}) (\tit{ha-b-icː-ij})
		\ex	\tit{a-(w)-iχʷ-ij} > \tit{ajχʷij} \sqt{to not be able} (\tsc{neg-m-}can\tsc{.pfv-inf}) (\tit{a-b-iχʷ-ij})
		\ex	\tit{ma-isː-it} > \tit{majsːit} \sqt{Do not shave!} (\tsc{neg-}shave\tsc{.ipfv-proh.sg})
	\end{xlist}
\end{exe}

This process is optional to some degree. This means that under certain circumstances that need further investigation, the two adjacent vowels \tit{a} and \tit{i} can be pronounced separately, not forming a diphthong. For instance, \tit{majsːit} can alternatively be pronounced \tit{maʔisːit} \refex{Don't take it!Phon}.

The diphthong [aʊ̯] arises when spatial preverbs or negation prefixes with the final vowel \tit{a} are added to verbs with the root vowel \tit{u}. This can be verbs with a gender prefix (\tit{b-uC(ː)-}) that are inflected for masculine singular gender agreement \refsec{sec:Gender/number agreement}. The masculine singular prefix \tit{w-} is regularly dropped before verbs with the root vowel \tit{u} (e.g. \tit{uc-ib} \sqt{caught him} vs. \tit{r-uc-ib} \sqt{caught her}), and then the combination of the two subsequent vowels turns into a diphthong that will be written \tit{aw} \refex{ex:a u aw phon}. In the following examples forms with overt agreement prefixes \textit{b} or \textit{d} are given in brackets at the end of the example lines.
%
\begin{exe}
	\ex	{\tit{a} + \tit{u} > \tit{aw}} (uC(ː) > wC(ː)) \label{ex:a u aw phon}
	\begin{xlist}
		\ex	\tit{tːura-(w)-uq-un} > \tit{tːurawqun} \sqt{he went outside} (\tsc{outside-m-}go\tsc{.pfv-pret}) (vs. \tit{tːura b-uq-un}) 
		\ex	\tit{sa-(w)-uq-un} > \tit{sawqun} \sqt{he came} (\tsc{hither-m-}go\tsc{.pfv-pret}) (vs. \tit{sa-b-uq-un})
		\ex	\tit{gu-sa-(w)-uc-ib} > \tit{gusawcib} \sqt{kept him} (\tsc{down-hither-m-}keep\tsc{.pfv-pret}) (vs. \tit{gu-sa-b-uc-ib})
		\ex	\tit{a-(w)-uq'-idel} > \tit{awq'idel} \sqt{should I not go} (\tsc{neg-m-}go\tsc{-modq}) (vs. \tit{a-d-uq'-idel})
	\end{xlist}
\end{exe}

The same happens to verbs that do not have a gender prefix (\tit{uC(ː)-}) when the root is preceded by prefixes ending in \textit{a} (\ref{ex:a u aw phon2}). And again there are exceptions to the rule, e.g. in \textit{sauq'ij} \sqt{go towards, go to meet} the two vowels do not form a diphthong, but are separated by a glottal stop.
%
\begin{exe}
	\ex	{\tit{a} + \tit{u} > \tit{aw}} (uC(ː) > wC(ː)) \label{ex:a u aw phon2}
	\begin{xlist}
		\ex	\textit{ka-utː-ij} > \textit{kawtːij} \sqt{tear off, rip off} (\tsc{down}-tear.\tsc{ipfv-inf})
		\ex \textit{ha-utː-ij} > \textit{hawtːij} \sqt{pull out, disassemble, take apart} (\tsc{up}-tear.\tsc{ipfv-inf})
		\ex \textit{ha-uχːaq-ij} > \textit{hawχːaqij} \sqt{ignite, set fire} (\tsc{up}-sparkle-\tsc{caus-inf})
	\end{xlist}
\end{exe}
% --------------------------------------------------------------------------------------------------------------------------------------------------------------------------------------------------------------------- %

\subsection{Vowel mutation (apophony)}
\label{ssec:Vowel mutation (apophony)}

Vowel mutation is found with inflected nouns and verbs. In the case of nouns, it is triggered by suffixation, in the case of verbs by prefixation.

The vowel \tit{a} in the final syllable of nouns ending in a consonant is raised and backened when one of the plural suffixes \tit{-e}, \tit{-te}, \tit{-be}, and \tit{-re} containing close-mid vowels is added, that is, there is vowel mutation \tit{a} > \tit{u} (including \tit{aˁ} > \tit{uˁ}). The process can be accompanied by delabialization (\refsec{ssec:Labialization and delabialization} shows examples). There are also one instance each of \tit{e} > \tit{u} and \tit{e} > \tit{i} under the same conditions. Relevant examples are given in \refex{ex:a u e u e i phon}. See \refsec{sec:nounnumber} for more nouns.
%
\begin{exe}
	\ex	{\tit{a} > \tit{u}; \tit{e} > \tit{u}; \tit{e} > \tit{i}}\label{ex:a u e u e i phon}
	\begin{xlist}
		\ex	\tit{qːap} > \tit{qːup-re} \sqt{sack}
		\ex	\tit{χabar} > \tit{χabur-te} \sqt{story, news}
		\ex	\tit{nez} > \tit{nuz-be} \sqt{louse}
		\ex	\tit{ʁez} > \tit{ʁiz-be} \sqt{hair}
	\end{xlist}
\end{exe}

Vowel mutation with verbs occurs when the spatial preverbs or negation prefixes with the final vowel \tit{a} are prefixed. The first type of verbal vowel mutation happens with verbs containing the stem vowel \tit{i} that are inflected for masculine singular or lack gender agreement prefixes. The gender prefix is dropped and the two vowels merge. Verb forms with overt gender prefixes are given in brackets at the end of the example lines for comparison.
%
\begin{exe}
	\ex	{\tit{a} + \tit{i} > \tit{e}}\label{ex:a i e phon}
	\begin{xlist}
		\ex	\tit{sa-(w)-irʁ-an} > \tit{serʁan} \sqt{the one that comes} (\tsc{hither-m-}come\tsc{-ptcp}) (vs. \tit{sa-b-irʁ-an}) 
		\ex	\tit{a-(w)-irχʷ-ar} > \tit{erχʷar} \sqt{cannot} (\tsc{neg-m-}be.able\tsc{.ipfv-prs.3}) (vs. \tit{a-b-irχʷ-ar})
		\ex	\tit{ka-(w)-irg-an=da} > \tit{kerganda} \sqt{I will sit down} (\tsc{down-m-}be\tsc{.ipfv-ptcp=1}) (vs. \tit{ka-r-irg-an=da})
	\end{xlist}
\end{exe}

This process is optional, but again the circumstances under which alternatives are allowed need to be clarified \refex{ex:does not understand phon}.
%
\begin{exe}
	\ex	\label{ex:does not understand phon}
	\begin{xlist}
		\ex	\tit{han a-w-irk-u} > \tit{han awirku}\slash\tit{han.erku}\slash\tit{han.aʔirku} \sqt{does not remember\newline\hspace*{1em}him} (remember \tsc{neg-m-}occur\tsc{.ipfv-prs.3})  
		\ex	\tit{a-irʁ-ib=da} > \tit{erʁibda}\slash\tit{aʔirʁibda} \sqt{I did not understand}\newline\hspace*{1em}(\tsc{neg-}understand\tsc{.pfv-pret=1})
	\end{xlist}
\end{exe}
%
The second type of verbal vowel mutation happens with verbs that have the stem vowel \tit{e} and lack gender agreement prefixes \refex{ex:a e e phon}. Note that in the first verb given below the vowel mutation results in a long vowel because the negation prefix \textit{a-} assimilated to the stem vowel and this, in turn, leads to a sequence of two identical vowels, which then becomes a long vowel. This process commonly occurs when the negation prefix is added to verbs beginning with the vowel \textit{a} because if the sequence would be shortened, the negated form would be identical to the affirmative form and negation could not be expressed \refex{ex:identical vowels A phon}. The same logic applies to \tit{eːrčur} \refex{ex:sawPHON}.
%
\begin{exe}
	\ex	{\tit{a} + \tit{e} > \tit{e}}\label{ex:a e e phon}
	\begin{xlist}
		\ex	\label{ex:sawPHON} \tit{a-erč-ur} > \tit{aʔerčur}\slash\tit{eːrčur} (\tsc{neg-}saw\tsc{.pfv-pret})
		\ex	\tit{ha-erʔ-ul} > \tit{herʔul} (\tsc{up}-say.\tsc{ipfv-icvb})
	\end{xlist}
\end{exe}

Finally, the combination of spatial preverbs ending with \tit{i} and a verb without a gender prefix and \tit{a} as stem vowel or a following preverb \tit{ha-} \sqt{upwards} also leads to vowel mutation. In the second case, when two preverbs combine, then the vowel mutation is initiated by the disappearance of the glottal fricative. The affected preverbs are \tit{či-} \sqt{on} + \tit{ha-} > \tit{če-, kʷi-} \sqt{in the hands} + \tit{ha} > \tit{kʷe-, hitːi-} \sqt{behind, after} + \tit{ha} > \tit{hitːe-, b-i} \sqt{in, inside} + \tit{ha-} > \tit{be-}. We can analyze this process as lowering of the vowel of the second preverb \refex{ex:i a e phon}. Again the process is optional and does not occur in slow, careful speech.
%
\begin{exe}
	\ex	{\tit{i} + \tit{a} > \tit{e}}\label{ex:i a e phon}
	\begin{xlist}
		\ex	\tit{či-ag-ur} > \tit{čegur} \sqt{s/he went} (\tsc{spr-}go\tsc{.pfv-pret})
		\ex	\tit{či-ha-b-išː-ib} > \tit{čebišːib} \sqt{s/he put it up} (\tsc{spr-up}\tsc{-n-}put\tsc{.pfv-pret})
		\ex	\tit{kʷi-ha-b-uc-ib} > \tit{kʷebucib} \sqt{s/he kept it in the hands}\newline\hspace*{1em}(\tsc{in.hands-up-n}-keep\tsc{.pfv-pret})
	\end{xlist}
\end{exe}


% --------------------------------------------------------------------------------------------------------------------------------------------------------------------------------------------------------------------- %

\subsection{Assimilation}
\label{ssec:Assimilation}

Progressive assimilation occurs with all verbal and nominal suffixes that have initial \tit{l}. The liquid assimilates to a preceding sonorant \tit{n} or \tit{r} \refex{ex:n l nn r l rr phon}. The following suffixes are affected:
%
\begin{itemize}
	\ex	genitive case: \tit{-la} > \tit{-na}\slash\tit{-ra}
	\ex	\tsc{loc}-series (spatial case): \tit{-le} > \tit{-ne}\slash\tit{-re}
	\ex	ergative case\slash oblique stem marker: \tit{-li} > \tit{-ni}\slash\tit{-ri}
	\ex	perfective converb\slash adverbializer: \tit{-le} > \tit{-ne}\slash\tit{-re}
	\ex	anteriority\slash causality converb \tit{-la} > \tit{-na}\slash\tit{-ra}
\end{itemize}
%
\begin{exe}
	\ex	{\tit{n} + \tit{l} > \tit{nn}; \tit{r} + \tit{l} > \tit{rr}}\label{ex:n l nn r l rr phon}
	\begin{xlist}
		\ex	\tit{cin-la} > \tit{cinna} \sqt{his\slash her} (\tsc{refl.sg.obl-gen})
		\ex	\tit{tuχtur-li} > \tit{tuχturri} (doctor\tsc{-erg})
		\ex	\tit{b-uč'-un-le} > \tit{buč'unne} \sqt{have read} (\tsc{n-}read\tsc{.pfv-pret-cvb})
	\end{xlist}
\end{exe}

With many words the process is optional, and in careful speech no assimilation takes place.


% --------------------------------------------------------------------------------------------------------------------------------------------------------------------------------------------------------------------- %

\subsection{Palatalization}
\label{ssec:Palatalization}

Palatalization of velar consonants occurs with verbs when suffixes starting with the front vowels \tit{i} and \tit{e}, or the causative suffix \tit{-aq} are added, or occasionally when the masdar suffix \tit{-ni} is following.
%
\begin{exe}
	\ex	{\tit{x} > \tit{š}, \tit{xː} > \tit{šː}}\label{ex:x s xx s phon}
	\begin{xlist}
		\ex	\tit{či-ka-b-ixː-a} \sqt{put it on!} (\tsc{spr-down}\tsc{-n-}put\tsc{.pfv-imp.sg})\newline\hspace*{1em}vs. \tit{či-ka-b-išː-ij} \sqt{to put it on} (\tsc{spr-down}\tsc{-n-}put\tsc{.pfv-inf})
		\ex	\tit{b-ax-ul} \sqt{going} (\tsc{n-}go\tsc{-icvb})\newline\hspace*{1em}vs. \tit{w-aš-e!} \sqt{Go!} (\tsc{m-}go\tsc{-imp.sg})
	\end{xlist}

	\ex	{\tit{g} > \tit{ž}}\label{ex:g z phon}
	\begin{xlist}
		\ex	\tit{b-ug-ul} \sqt{remaining} (\tsc{n-}stay\tsc{-icvb})\newline\hspace*{1em}vs. \tit{b-už-ib} \sqt{remained} (\tsc{n-}stay\tsc{-pret}) 
	\end{xlist}

	\ex	{\tit{k} > \tit{č}, \tit{kː} > \tit{čː}, \tit{k'} > \tit{č'}}\label{ex:k c kk c k c phon}
	\begin{xlist}
		\ex	\tit{b-uk-ul} \sqt{gathering} (\tsc{n-}gather\tsc{-icvb})\newline\hspace*{1em}vs. \tit{b-uč-ib} \sqt{gathered it} (\tsc{n-}gather\tsc{-pret}) 
		\ex	\tit{b-ikː-a} \sqt{give it!} (\tsc{n-}give\tsc{.pfv-imp.sg})\newline\hspace*{1em}vs. \tit{b-ičː-ib} \sqt{gave it} (\tsc{n-}give\tsc{.pfv-pret})
		\ex	\tit{er w-erk'-araj} \sqt{in order to look at him} (look \tsc{m-}look\tsc{.pfv-subj})\newline\hspace*{1em}vs. \tit{er w-erč'-e} \sqt{Look!} (look \tsc{m-}look\tsc{.pfv-imp.sg})
		\ex	\tit{b-ebk'-a} \sqt{death} (\tsc{n-}die\tsc{.pfv-nmlz})\newline\hspace*{1em}vs. \tit{b-ebč'-ni} \sqt{death} (\tsc{n-}die\tsc{.pfv-msd})
	\end{xlist}
\end{exe}

When the masdar suffix is added the process is optional, at least with some verbs \refex{ex:masdar palatalization phon} (although it occurs when other suffixes are added). With a few verbs such as \tit{er b-ik'ʷ-ni} `looking' (look \tsc{-n-}say\tsc{.ipfv-msd}) it is downright ungrammatical.
%
\begin{exe}
	\ex	\label{ex:masdar palatalization phon}
	\begin{xlist}
		\ex	\tit{ubč'-ni}\slash\tit{ubk'-ni} (die\tsc{.m.ipfv-msd}) < \tit{b-ubk'-} (\tsc{n-}die\tsc{.ipfv-}) 
		\ex	\tit{b-arč-ni}\slash\tit{b-ark-ni} (\tsc{n-}find\tsc{.pfv-msd}) < \tit{b-arkː-} (\tsc{n-}find\tsc{.pfv-})
	\end{xlist}
\end{exe}


% --------------------------------------------------------------------------------------------------------------------------------------------------------------------------------------------------------------------- %

\subsection{Labialization and delabialization}
\label{ssec:Labialization and delabialization}

There are two instances of labialization of stops triggered by the round vowel \tit{u}. In the first instance, a preceding vowel is lost and the loss is compensated by labializing the following stop (another example with the same verb has been provided in \refsec{ssec:Vowel deletion (vowel syncope)} above):
%
\begin{exe} \label{ex:insidephon2}
	\ex	\tit{w-i-ha-(w)-ulq-an} > \tit{wihalqʷan} \sqt{the one that goes inside}\newline\hspace*{2em}(\tsc{m-in-up-(m)}-go\tsc{.ipfv-ptcp}) (compare with \tit{b-i-ha-b-ulq-an})
\end{exe}

The second instance represents the combination of the two spatial preverbs \tit{gu-} \sqt{under}, \tsc{sub} and \tit{ha-} \sqt{upwards}. The glottal fricative between the two vowels is lost and the round vowel disappears, leaving the initial stop labialized, that is \tit{gu-h-a} > \tit{gʷa} \refex{ex:She set up a phon}.
%
\begin{exe}
	\ex	\label{ex:She set up a phon}
	\gll	c'a	gʷa-b-iq'-un	ca-b\\
		fire	\tsc{from.under.up}\tsc{-n-}set.fire\tsc{.pfv-pret}	be\tsc{-n}\\
	\glt	\sqt{(She) set up a fire.}
\end{exe}

Delabialization is a more widespread and predictable process. It occurs when verbs that contain labialized stem consonants take suffixes beginning with the round vowel \tit{u} (i.e. one of the preterite allomorphs \tit{-ub}, \tit{-ur,} or \tit{-un}):
%
\begin{exe}
	\ex	\label{ex:delabialization A phon}
	\begin{xlist}
		\ex	\tit{b-elk'ʷ-ij} \sqt{\tsc{n-}write\tsc{.pfv-inf}} > \tit{b-elk-un} \sqt{wrote} (\tsc{n-}write\tsc{.pfv-pret})
		\ex	\tit{kaxʷ-ij} \sqt{kill} > \tit{kax-ub} \sqt{killed} (kill\tsc{.pfv-pret})
		\ex	\tit{ergʷ-ij} \sqt{sieve} > \tit{erg-ur} \sqt{sieved} (sieve\tsc{.pfv-pret})
	\end{xlist}
\end{exe}

With nouns delabialization occurs in the formation of the plural. When the plural suffix or the oblique plural suffix is added to nouns that have a vowel \tit{a}\slash\tit{aˁ} in the root that undergoes vowel mutation \tit{a}\slash\tit{aˁ} > \tit{u}\slash\tit{uˁ}, then the mutation is accompanied by delabialization of a stop that precedes or follows the mutated vowel. Furthermore, plural suffixes containing \tit{u} also trigger delabialization of preceding consonants when they are added \refex{ex:plural delabialization phon}.
%
\begin{exe}
	\ex	\label{ex:plural delabialization phon}
\TabPositions{13em}
		\tit{daˁrqʷ} \sqt{barn} > \tit{duˁrq-be} \tab 	\tit{q'ʷaˁl} \sqt{cow} > \tit{q'uˁl-e} \\
		\tit{qːʷaz} \sqt{goose} > \tit{qːuz-re}	\tab 	\tit{mikʷa} \sqt{fingernail} > \tit{mik-upːe} \\
		\tit{χːʷe} \sqt{dog} > \tit{χː-ude}
\end{exe}

Other plural suffixes do not lead to vowel mutation, and thus labialized consonants are preserved, for example:
%
\begin{exe}
	\ex	\label{ex:no plural vowel mutation phon}
\TabPositions{13em}
		\tit{kːʷacːa} \sqt{mare} > \tit{kːʷac-ne}	\tab 	\tit{targʷa} \sqt{weasel} > \tit{targʷ-ne} \\
		\tit{gʷagʷa} \sqt{flower} > \tit{gʷagʷne}	\tab 	\tit{žilixʷa} \sqt{saddle} > \tit{žilixʷme} 
\end{exe}


% --------------------------------------------------------------------------------------------------------------------------------------------------------------------------------------------------------------------- %

\subsection{Gemination and degemination}
\label{ssec:Gemination and degemination}
Gemination is not a common process, whereas degemination is frequent. There is optional gemination in combination with devoicing, which always involves at least one gender affix. This process occurs only with the gender affixes \tit{b} (neuter singular\slash human plural) and \tit{d} (neuter plural\slash first and second person plural). The two lax voiced consonants become tense and devoiced when they are preceded or followed by an identical consonant. This can either be the same gender affix or the past tense enclitic \tit{=de}, the attributive plural suffix \tit{-te} or occasionally when a preverb ending in \tit{p} is used in a complex verb (there are no preverbs ending in \tit{d}). Examples are given in \refex{ex:d dt t phon} and \refex{ex:bp b p phon}. In careful speech the two consonants are pronounced individually, and no gemination and devoicing take place.
%
\begin{exe}
	\ex	{\tit{d} + \tit{d/t} > \tit{tː}}\label{ex:d dt t phon}
	\begin{xlist}
		\ex	\tit{či-d-d-iχ-un} > \tit{čitːiχun} \sqt{(they) tied them} (\tsc{spr-npl-npl-}tie\tsc{.pfv-pret})
		\ex	\tit{le-d=de} > \tit{letːe} \sqt{we were there} (exist\tsc{-npl=pst}) (\tit{le-b=de})
		\ex	\tit{xari-d-te} > \tit{xaritːe} \sqt{the ones down} (down-\tsc{npl-dd.pl}) (\tit{xari-b-te})
	\end{xlist}

	\ex	{\tit{b/p} + \tit{b} > \tit{pː}}\label{ex:bp b p phon}
	\begin{xlist}
		\ex	\tit{gu-b-b-iči-b} > \tit{gupːičib} \sqt{it lost} (\tsc{sub-n-n-}occur\tsc{.pfv-pret})
		\ex	\tit{χːap b-arq'-ib} > \tit{χːapːarq'ib} \sqt{grabbed it} (grab \tsc{n-}do\tsc{.pfv-pret})
	\end{xlist}
\end{exe}

Gemination does not occur when two voiceless consonants follow each other, for example \tit{ħaˁžat-te} (need\tsc{-dd.pl}).

Furthermore, a number of verbal suffixes such as the present habitual suffixes contain geminates. These suffixes are probably diachronically complex in their morphology, but since they synchronically function as entire morphemes that are not further split up, they are not treated here.


Geminates are regularly degeminated when they end up in syllable-final position, because geminates in syllable-final position are prohibited (see Sections \ref{sec:Consonant inventory}, \ref{sec:Syllable and word structure}). Therefore, when suffixation leads to resyllabification, then degemination takes place, that is, tense consonants become lax. Voicing is not affected. Within the nominal morphology we find degemination of stops, fricatives, and affricates when the plural suffixes \tit{-be}, \tit{-ne,} and \tit{-me} are added \refex{ex:degemination be ne me phon}.
%
\begin{exe}
	\ex	\label{ex:degemination be ne me phon}
\TabPositions{14em} 
		\tit{rur.sːi} > \tit{rurs-be} \sqt{girl, daughter}	\tab 	\tit{cːa.cːi} > \tit{cːac-be} \sqt{thorn} \\
		\tit{c'el.tːa} > \tit{c'elt-me} \sqt{gravestone}		\tab 	\tit{e.čːa} > \tit{eč-ne} \sqt{she-goat}
\end{exe}

Similarly, a number of nouns have underlying geminates (stops and fricatives) in the word-final position that are only pronounced as geminates when suffixes that begin with a vowel (e.g. the plural suffixes \tit{-e} and \tit{-upːe}) are attached  \refex{ex:gemination e upee phon}. In those plural nouns the geminates occur in syllable-initial position. By contrast, when the nouns are used in the singular or when suffixes that start with consonants are added (e.g. the ergative suffix \textit{-li}), then the stops and fricatives are degeminated. More examples can be found in \refsec{sec:nounnumber}. In the examples in \refex{ex:gemination e upee phon} first the plural forms are given and then the singular forms.
  
%
\begin{exe}
	\ex	\label{ex:gemination e upee phon}
\TabPositions{13em} 
		 \tit{juldašːe} > \tit{juldaš} \sqt{friend}	\tab 		\tit{baliqːe} > \tit{baliq} \sqt{fish} \\
		\tit{ʡuˁrusːe} > \tit{ʡuˁrus} \sqt{Russian}	\tab 		\tit{ħaˁšukːe} > \tit{ħaˁšuk} \sqt{pot} \\
		\tit{miriqːʷe} > \tit{miriqʷ}  \sqt{worm}	\tab 		\tit{t'upːe} > \tit{t'up} \sqt{finger} \\
		\tit{qːuˁnqːuˁpːe} > \tit{qːuˁnq}  \sqt{nose}
\end{exe}



Within the verbal system, degemination can only occur when consonant-initial suffixes are added to verbal roots that have geminated consonants. The only relevant suffixes are the masdar suffix \tit{-ni} (or \tit{-ri}) and the locative participle \tit{-na}.
%
\begin{exe}
	\ex	\label{ex:degemination masdar locative phon}
	\begin{xlist}
		\ex	\tit{ha-qː-ij} (\tsc{up}-carry\tsc{-inf}) > \tit{haq-ni} (masdar)
		\ex	\tit{ka-b-ičː-ij} (\tsc{down-n-}cut.up\tsc{.pfv-inf}) > \tit{kabič-ni} (masdar)
		\ex	\tit{b-arcː-ij} (\tsc{n-}get.tired\tsc{.pfv-inf}) > \tit{barc-ni} (masdar)
		\ex	\tit{akːʷ-} (be\tsc{.neg}) > \tit{akʷ-ni}\slash\tit{akʷ-ri} (masdar)
		\ex	\tit{b-učː-ij} (\tsc{n-}drink\tsc{.pfv-inf}) > \tit{buč-na} (locative participle)
	\end{xlist}
\end{exe}
