\chapter{Syntactic properties of adverbial and conditional clauses}
\label{cpt:Syntactic properties of adverbial and conditional clauses}

This chapter analyzes the syntax of adverbial and conditional clauses in Sanzhi and compares them to the syntactic properties of similar clauses in other East Caucasian languages.


%%%%%%%%%%%%%%%%%%%%%%%%%%%%%%%%%%%%%%%%%%%%%%%%%%%%%%%%%%%%%%%%%%%%%%%%%%%%%%%%

\section{The syntax of adverbial clauses}
\label{sec:The syntax of adverbial clauses}

Sanzhi has different types of adverbial clauses that can be distinguished by the morphological make-up of the verb forms in the subordinate clause and by their semantics. Semantically, we can distinguish between simple converbs with a fairly general meaning and specialized converbs with a rather specific temporal or non-temporal meaning. The first group consists of the imperfective (\refsec{sssec:The imperfective converb}) and perfective converb (\refsec{sssec:The perfective converb}). The second group contains temporal, causal, and other converbs (\refsec{cpt:specializedconverbssubordinatingenclitics}). A similar distinction is found in many East Caucasian languages (e.g. in Tsezic, see \citealp{Comrie.Forker.Khalilova2012}, and in Dargwa varieties, see \citealp{Belyaev2010}). The syntactic characteristics of constructions with general converbs have repeatedly been discussed in the literature  because they exhibit a mixed behavior, showing features of subordination as well as of coordination (see, among others, \citealp{Kazenin.Testelets2004}; \citealp{Haspelmath1995}; \citealp{Belyaev2010}; \citealp{Comrie.Forker.Khalilova2012}; \citealp{Creissels2010, Creissels2012}; \citealp{Forker2013b}). Sentences in Sanzhi can be fairly complex, containing a number of adverbial clauses that are combined with one main clause. Semantically, these clauses either resemble coordination, as in \refex{ex:‎‎‎The donkey got tired, fell down, and died}, or subordination, when the meaning of the adverbial clause is causal \refex{ex:‎‎‎First, (because) the stones of father's house had fallen down}.
%
\begin{exe}
	\ex	\label{ex:‎‎‎The donkey got tired, fell down, and died}
	\gll	amχa	[b-arcː-ur-re]	[ka-b-ič-ib-le]	b-ebč'-ib	ca-b\\
		donkey	\tsc{n-}get.tired\tsc{.pfv-pret-cvb}	\tsc{down-n-}occur\tsc{.pfv-pret-cvb}	\tsc{n-}die\tsc{.pfv-pret}	\tsc{cop-n}\\
	\glt	\sqt{‎‎‎The donkey got tired, fell down, and died.}

	\ex	\label{ex:‎‎‎First, (because) the stones of father's house had fallen down}
	\gll	bahsar	[heχ	cin-na	atːa-la	jurt-la	qːarqːa	ʡaˁbal	qal-la	xːari	k-ag-ur-re]	[qːaq-li-j	či-ka-d-irxː-ul]	[ha-d-iqː-ul]	qːarqːa=ra	gu-r-h-aqː-ib=da\\
		first	\tsc{dem.down}	\tsc{refl.sg-gen}	father\tsc{-gen}	house\tsc{-gen}	stone	three	house\tsc{-gen}	down	\tsc{down}-go\tsc{.pfv-pret-cvb}		back\tsc{-obl-dat}	\tsc{spr-down}\tsc{-npl-}put\tsc{.ipfv-icvb}	\tsc{up-npl-}carry\tsc{.ipfv-icvb}	stone\tsc{=add}	\tsc{sub-abl-up}-carry\tsc{-pret=1}\\
	\glt	\sqt{‎‎‎First, (because) the stones of father's house had fallen down three floors, we put them on the back and carried them, carried the stones.}
\end{exe}

The perfective converb is widely used in procedural texts, such as the description of how to prepare dishes. These texts consist of a list of actions that are expressed by verbs bearing perfective converb suffixes with a main clause at the end. The actions are supposed to occur in the order in which the clauses follow each other, i.e., there is iconicity, and the order of the clauses cannot be changed without changing the meaning of the whole sentence. This is generally interpreted as a semantic feature of coordination, as opposed to subordination, where the order of the clauses does not reflect the temporal order of the events and can therefore be changed without a concomitant change in the meaning. Linear order will be discussed in more detail in \refsec{ssec:Linear order and iconicity} below.

Converbs are non-finite in the sense that they head only subordinate clauses. The two general converbs (imperfective, perfective) also occur in analytic tenses in main clauses (\refcpt{cpt:Analytic verb forms}), but only when combined with a copula or a predicative particle (\refsec{sec:Predicative particles}). Therefore, they are unable to express illocutionary force or absolute temporal reference but share those properties with the verb form in the main clause (see \refsec{ssec:Scope properties} below). They are also not marked for person by person suffixes or enclitics, in contrast to the verb forms in the superordinate clause. However, they express aspect, because aspect is mainly conveyed through the verbal stem and there are no restrictions concerning the use of perfective or imperfective stems in adverbial clauses. Moreover, they can have their own arguments that fulfill the same grammatical roles as arguments in main clauses, i.e., case marking patterns in adverbial clauses and main clauses do not differ. Furthermore, gender agreement is present in adverbial clauses. In contrast to main clauses, it is strictly controlled by the absolutive argument. By contrast, in main clauses copulas can exhibit gender agreement with ergative or dative arguments. However, these copulas cannot occur in subordinate clauses.

The constituent order in adverbial clauses shows a far greater tendency for verb-final order than is observed for main clauses \refex{ex:‎‎‎First, (because) the stones of father's house had fallen down}, but adverbial clauses in which the verb is followed by other constituents can be found as well \refex{ex:‎[when I was in that situation], when I also was in a place like this, I also did not feel well}, \refex{ex:Stalin died, and the cars, the trains were stopped making tooot}.
%
\begin{exe}
	\ex	\label{ex:‎[when I was in that situation], when I also was in a place like this, I also did not feel well}
	\gll	[hel=ʁuna	musna-w	ink	w-aq-ib=qːel	du=ra]	dam=ra	ʡaˁħ-le=kːʷi\\
		that\tsc{=eq}	place\tsc{.loc-m}	meet	\tsc{m-}go.through\tsc{.pfv-pret=}when	\tsc{1sg=add}	\tsc{1sg.dat=add}	good\tsc{-advz=neg.pst}\\
	\glt	\sqt{‎[When I was in that situation], when I also was in a place like this, I also did not feel well.}
\end{exe}

In the following discussion, I will adopt the typology of \citet{Bickel2010} for the investigation of clause-linkage patterns. Bickel's typology consists of eleven variables, which are reproduced in the first column of \reftab{tab:Syntactic variables for the analysis of adverbial clauses}. A short description is given in the second column of the same table. 
%
\begin{table}
	\caption{Syntactic variables for the analysis of adverbial clauses}
	\label{tab:Syntactic variables for the analysis of adverbial clauses}
	\small
	\begin{tabularx}{0.98\textwidth}[]{%
		>{\raggedright\arraybackslash}p{90pt}
		>{\raggedright\arraybackslash\hangindent=0.5em}X}
		
		\lsptoprule
			Variable			&	Description\\
		\midrule
			Illocutionary scope		&	Which clauses fall within the scope of illocutionary force operators?\\
			Illocutionary marking	&	Can the dependent clause contain illocutionary force operators?\\
			Tense scope			&	Which clauses fall within the scope of tense operators?\\
			Tense marking		&	Can the dependent clause contain tense markers?\\
			Finiteness			&	Does the dependent clause express fewer (non-finite) or the same number (finite) of categories?\\
			Symmetry			&	Can the range of expressed categories in the dependent and in the main clause be different or not?\\
			WH				&	Are question words and focus enclitics inside dependent clauses allowed or not?\\
			Focus				&	Can focus marking appear on the dependent clause?\\
			Extraction			&	Is extraction of elements of dependent clauses allowed?\\
			Position			&	Can the dependent clause appear before and after the main clause? Can it be separated by other clauses?\\
			Layer				&	Can the dependent clause be center-embedded?\\
		\lspbottomrule
	\end{tabularx}
\end{table}

I will additionally use a number of other criteria that have been proposed in order to differentiate between coordination and subordination, namely co-reference and expression of shared arguments, morphosyntactic locus, and relativization of constituents of adverbial clauses.

I will mainly analyze the two general converbs as well as the temporal converb \tit{=qːel(la)} \sqt{when, while, because}, which expresses temporal simultaneity and anteriority as well as causality, because these converbs show the largest semantic overlaps and are semantically close to coordination.


% --------------------------------------------------------------------------------------------------------------------------------------------------------------------------------------------------------------------- %

\subsection{Scope properties}
\label{ssec:Scope properties}

Adverbial clauses do not contain markers for illocutionary force, such as the imperative, optative suffixes, or the interrogative particles (``banned''). Those markers can only occur in the main clause. Their scope can be restricted to the main clause (``local''), but, in the appropriate context, it can also extent across the adverbial clause (``extensible''). However, the latter possibility is noticeably less common in texts. Examples \refexrange{ex:When you put the roof beam at this (at one) side}{ex:‎‎The legs are trembling, jump} illustrate local scope restricted to the main clause.
%
\begin{exe}
	\ex	\label{ex:When you put the roof beam at this (at one) side}
	\gll	[hej	šːal-li-cːe	cːiχːin	ka-b-alt-an=qːel]	het	šːal-la	ʡaˁnčːi	a-ka-d-ax-u=w?\\
		this	side\tsc{-obl-in}	roof.beam	\tsc{down-n-}put\tsc{.ipfv-ptcp=}when	that	side\tsc{-gen}	earth	\tsc{neg-down}\tsc{-npl}-go-\tsc{prs.3=q}\\
	\glt	\sqt{When you put the roof beam at this (at one) side, does the clay of that (the other side) not fall down?}

	\ex	\label{ex:Do you go home because your wife told you to}
	\gll	[heχ	xːunul-la	ʁaj-li-gu	aq-ib-le]	qili	arg-ul=de=w?\\
		\tsc{dem.down}	woman\tsc{-gen}	word\tsc{-obl-sub}	go.through\tsc{.pfv-pret-cvb}	home	go\tsc{.ipfv-icvb=2sg=q}\\
	\glt	\sqt{Do you go home because your wife told you to?}

	\ex	\label{ex:‎‎The legs are trembling, jump}
	\gll	[t'uˁ-me	rurčː-ul]	taˁħ	d-uq-ene!\\
		leg\tsc{-pl}	tremble\tsc{-icvb}	jump	\tsc{1/2pl-}go\tsc{.pfv-imp.pl}\\
	\glt	\sqt{‎‎The legs are trembling, jump!}\footnote{Within the contexts from which this example originates the subjects of the adverbial clause and the main clause differ. The speaker who was guiding a truck full of people urged them to jump off the car because he had problems controlling it. This means that the full translation is `While/because my legs are trembling, jump!’ Out of context, however, the most natural reading is rather: `While your legs are trembling, jump!’ with a same-subject interpretation.}
\end{exe}

Some converbs seem to fully ban joint scope of illocutionary operators. For instance, interrogative markers \refex{ex:When you put the roof beam at this (at one) side} or imperative markers \refex{ex:‎When you buy bread, eat it with cheese!} cannot scope over the temporal converb \tit{=qːel}, although tense suffixes can. 
%
\begin{exe}
	\ex	\label{ex:‎When you buy bread, eat it with cheese!}
	\gll	[t'ult'	asː-ib=qːel]		nisːe-cːella	b-erkʷ-en!\\
		bread	buy\tsc{.pfv-pret=}when	cheese\tsc{-comit}	\tsc{n-}eat\tsc{.pfv-imp}\\
	\glt	\sqt{‎When you buy bread, eat it with cheese!} (NOT: \sqt{Buy bread and eat it with cheese!}) (E)
\end{exe}

But at least with the perfective and the imperfective converbs it is also possible that the two clauses have joint scope:
%
\begin{exe}
	\ex	\label{ex:Go and bring it}
	\gll	[ag-ur-re]	h-aqː-a!\\
		go\tsc{.pfv-pret-cvb}	\tsc{up}-carry\tsc{-imp}\\
	\glt	\sqt{Go and bring it!}

	\ex	\label{ex:‎Sing a song and dig the field}
	\gll	[dalaj	Ø-ik'-ul]	qu	b-urqː-a!\\
		song	\tsc{m-}say\tsc{.ipfv-icvb}	field	\tsc{n-}dig\tsc{.pfv-imp}\\
	\glt	\sqt{‎Sing a song and dig the field!} (E)
\end{exe}

Similarly, adverbial clauses can only express aspectual distinctions because this is a property of the verbal stem. Other semantic categories of verbs such as tense and evidentiality are only available to verb forms in main clauses. The converbs have relative temporal reference. This means that they refer to situations that take place before, after or during the situation that is expressed by the matrix clause. For instance, in \refex{ex:I will go to the forest and bring nuts@37} the verb form in the main clause has future/modal meaning, which is extended to the adverbial clause with the preterite converb. Sentence \refex{ex:‎Very slowly making small steps we went across (the river)} conveys past time reference due to the preterite in the main clause, and \refex{ex:Being alone at home, I stay not crying} conveys present time reference because of the compound present tense. Both sentences contain adverbial clauses with the imperfective converb that only expresses that the situation in the adverbial clause took place during the situation described in the main clause.
%
\begin{exe}
	\ex	\label{ex:I will go to the forest and bring nuts@37}
	\gll	du-l	[ag-ur-re	wac'a-cːe]	ka-d-iqː-an=da	qix-be\\
		\tsc{1sg-erg}	go\tsc{.pfv-pret-cvb}	forest-\tsc{in}	\tsc{down-npl-}carry\tsc{.ipfv-ptcp=1}	nut\tsc{-pl}\\
	\glt	\sqt{I will go to the forest and bring nuts.} 

	\ex	\label{ex:‎Very slowly making small steps we went across (the river)}
	\gll	[bahla-l	bahla-l	nik'a	kːanc	ka-b-ircː-ul]	bahla-l	či-r-ag-ur=da\\
		slow\tsc{-advz}	slow\tsc{-advz}	small	step	\tsc{down-n-}stand\tsc{.ipfv-icvb}	slow\tsc{-advz}	\tsc{spr-abl-}go\tsc{.pfv-pret=1}\\
	\glt	\sqt{‎Very slowly making small steps we went across (the river).} 

	\ex	\label{ex:Being alone at home, I stay not crying}
	\gll	[qili-r	du=gina	r-irχ-ul]	[a-r-isː-ul]	r-ug-ul=da\\
		home\tsc{-f}	\tsc{1sg=}only	\tsc{f-}be\tsc{.ipfv-icvb}	\tsc{neg-f-}cry\tsc{-icvb}	\tsc{f-}stay\tsc{.ipfv-icvb=1}\\
	\glt	\sqt{Being alone at home, I (fem.) stay not crying.} 
\end{exe}

Similarly, the past perfect used in the main clause of \refex{ex:From her (he) took 16 000, and he sent (that money to us)} expresses not only past time reference but also indirect evidentiality, which extends to the meaning of the full sentence including the adverbial clause with the preterite converb.
%
\begin{exe}
	\ex	\label{ex:From her (he) took 16 000, and he sent (that money to us)}
	\gll	[it-i-sa-r	s-asː-ib-le	wec'-nu	urek-ra	azir]	it-i-l=ra d-ataʁ-ib-le=de\\
		that\tsc{-obl-ante-abl}	\tsc{hither}-take\tsc{.pfv-pret-cvb}	ten-\tsc{ten}	six\tsc{-num}	thousand that\tsc{-obl-erg=add}	\tsc{npl-}send\tsc{.pfv-pret-cvb=pst}\\
	\glt	\sqt{From her (he) took 16\thinspace 000, and he sent (that money to us).}
\end{exe}

In short, fewer categories are expressed in adverbial clauses than in main clauses, because person agreement, tense, evidentiality, and illocutionary force are absent. This means that Sanzhi adverbial clauses are, in Bickel's terms, ``asymmetrical'' and non-finite \citep{Bickel2010}. 


% --------------------------------------------------------------------------------------------------------------------------------------------------------------------------------------------------------------------- %

\subsection{Focus and question words}
\label{ssec:Focus and question words}

Most but not all focus-sensitive particles can appear in adverbial clauses attached to the converbs. The following examples show the enclitic \tit{=cun} \sqt{only} and the emphatic modal particle \tit{=q'ar} in clauses together with the perfective converb and the \tit{=qːel} converb. The modal particle \tit{=q'al} can also be employed in certain types of adverbial clauses, but in general its use in subordinate clauses is subject to many restrictions \refex{But when I saw the teacher, my face changed (i.e. turned red)}, \refex{ex:‎‎‎Was he at home when I came back from Derbent@B}. The restrictions are specific to this particle and therefore not relevant for a discussion of the morphosyntactic properties of adverbial clauses.
%
\begin{exe}
	\ex	\label{ex:‎‎‎I understood (everything) only wrongly)}
	\gll	[b-alk'-un-ne=cun]	irʁ-ul=de\\
		\tsc{n-}bend\tsc{-pret-cvb=}only	understand\tsc{.ipfv-icvb=pst}\\
	\glt	\sqt{‎‎‎I understood (everything) only wrongly (i.e. I had only bad thoughts.).}

	\ex	\label{ex:Fallen down there are bottles there}
	\gll	ka-d-ič-ib-le=q'ar	χe-d	heχtːu-d	šuš-ne\\
		down\tsc{-npl-}occur\tsc{.pfv-pret-cvb=mod}	exist.\tsc{down-npl}	there.\tsc{down-npl}	bottle\tsc{-pl}\\
	\glt	\sqt{Fallen down there are bottles there.}

	
		\ex	\label{But when I saw the teacher, my face changed (i.e. turned red)}
		\gll	[a	učitil	či-w-až-ib=qːel=q'al]	c'il	di-la	daˁʡ d-ars	d-iχ-ub\\
			but	teacher	\tsc{spr-m}-see\tsc{.pfv-pret=}when\tsc{=mod}	then	\tsc{1sg-gen}	face \tsc{npl}-change 	\tsc{npl}-be.\tsc{pfv-pret}	\\
		\glt	\sqt{But when I saw the teacher, my face changed (i.e. turned red).} 
	
		\ex	\label{ex:‎‎‎Was he at home when I came back from Derbent@B}
		\gll	{*} 	du		Derbent-le-r	sa-jʁ-ib=qːel=q'al	it	qili-w=de?\\
			{}	\tsc{1sg}	Derbent\tsc{-loc-abl}	hither-come\tsc{.pfv-pret=}when\tsc{=mod}	that	home\tsc{-m=pst}	\\
		\glt	(Intended meaning: \sqt{‎‎‎Was he at home when I came back from Derbent?}) (E)

\end{exe}

As mentioned in \refsec{ssec:Scope properties} above, interrogative particles (which also belong to the focus-sensitive particles) cannot be used in adverbial clauses. However, adverbial clauses with various converbs can contain interrogative pronouns as the following examples with the perfective converb \refex{ex:When Hazhimurad was given what we were happy} and the converb \tit{=qːel} \refex{ex:‎When who is guiding the car the girls get afraid} show. 
%
\begin{exe}
	\ex	\label{ex:When Hazhimurad was given what we were happy}
	\gll	[ħaˁžimurad-li-j	ce	b-ičː-ib-le]	ušːa	razi	d-iχ-ub=da=ja?\\
		Hazhimurad\tsc{-obl-dat}	what	\tsc{n-}give\tsc{.pfv-pret-cvb}	\tsc{2pl}	happy	\tsc{1/2pl-}be\tsc{.pfv-pret=1=q}\\
	\glt	\sqt{When Hazhimurad was given what were we happy?} (E) 

	\ex	\label{ex:‎When who is guiding the car the girls get afraid}
	\gll	[hi-l	mašin	b-ik-an=qːel]	rurs-be	uruχ	b-ik'-ul=e?\\
		who\tsc{.obl-erg}	car	\tsc{n-}lead\tsc{.ipfv-ptcp=}when	girl\tsc{-pl}	fear	\tsc{hpl-}\tsc{aux.ipfv-icvb=q}	\\
	\glt	\sqt{‎When who is guiding the car do the girls get afraid?} (E) 
\end{exe}


% --------------------------------------------------------------------------------------------------------------------------------------------------------------------------------------------------------------------- %

\subsection{Co-reference and expression of shared arguments}
\label{ssec:Co-reference and expression of shared arguments}

Converb clauses can almost always have their own subjects that do not need to be co-referential with the subject in the main clause. Examples of adverbial clauses with differing subjects can be found in \refex{ex:Stalin died, and the cars, the trains were stopped making tooot} for the perfective converb, in \refex{ex:‎Two days it was freezing and raining} for the imperfective converb, and in \refex{ex:When you put the roof beam at this (at one) side} and \refex{But when I saw the teacher, my face changed (i.e. turned red)} for constructions with \tit{=qːel}. However, for the sentence in \refex{ex:‎Two days it was freezing and raining} there is no alternative possibility of using a same-subject construction because the two weather verbs grammatically require different subjects. Thus, syntactically \refex{ex:‎Two days it was freezing and raining} is a complex clause with two different subjects, but semantically there is a clear relationship between the two clauses.
%
\begin{exe}
	\ex	\label{ex:Stalin died, and the cars, the trains were stopped making tooot}
	\gll	[w-ebč'-ib-le	Istalin]	[mašin-te	pojezd-e	t'aš aʁ-ib-le] 	tːuːˁtː-d-ik'-ul, \ldots\\
		\tsc{m-}die\tsc{.pfv-pret-cvb}	Stalin,	car\tsc{-pl}	train\tsc{-pl}	stop do\tsc{-pret-cvb} 	toot\tsc{-npl-}say\tsc{.ipfv-icvb}\\
	\glt	\sqt{Stalin died, and the cars, the trains were stopped making tooot, \ldots}

	\ex	\label{ex:‎Two days it was freezing and raining}
	\gll	k'ʷel	bar	[wiz	b-ik'-ul]	b-us-ib\\
		two	day	freeze	\tsc{n}-\tsc{aux.ipfv-icvb}	\tsc{n-}rain\tsc{-pret}\\
	\glt	\sqt{‎Two days it was freezing and raining.}
\end{exe}

If the subjects differ, it is possible that other arguments are co-referential instead. In \refex{ex:‎‎The legs are trembling, jump} the subject of the first clause with the imperfective converb is not identical to that of the following, but can be identical to the omitted possessor (see the comment in the footnote). In \refex{But when I saw the teacher, my face changed (i.e. turned red)}, the omitted dative subject of the adverbial clause shares the referent with the possessive pronoun in the main clause. Similarly, in \refex{ex:When they got married, they had a good life} the omitted subject of the adverbial clause is identical to the referent of the possessive pronoun in the main clause. It can also be the case that a string of adverbial clauses shares the subject with an adjunct in the main clause.
%
\begin{exe}
	\ex	\label{ex:When they got married, they had a good life}
	\gll	[can	ka-b-iž-ib=qːel]	ču-la	jašaw-li-cːe-b	zamana	ca-b\\
		meet	\tsc{down-hpl-}be\tsc{.pfv-pret=}when	\tsc{refl.pl-gen}	being\tsc{-obl-in-n}	time	\tsc{cop-n}\\
	\glt	\sqt{When they got married, they had a good life.} (lit. When they met it is the time of their well-being.)
\end{exe}

The sharing of the subject argument is clearly preferred for the perfective converb and can be seen in most examples in this section. Even in example \refex{ex:Stalin died, and the cars, the trains were stopped making tooot} there is at least a causal relationship between the described events: because of the death of Stalin the trains tooted and honked. If no such causal relationship can be found, a complex clause with different subjects is impossible \refex{ex:When Ali came home, Indira was sewing a dress}.
%
\begin{exe}
	\ex	\label{ex:When Ali came home, Indira was sewing a dress}
	\gll	{??}	[ʡaˁli	qili	w-i-ha-w-q-un-ne]	Indira-l	kːurtːi	b-urχ-ul=de\\
		{}	Ali	home	\tsc{m-in-up}\tsc{-m-}go\tsc{.pfv-pret-cvb}	Indira\tsc{-erg}	dress	\tsc{n-}sew\tsc{.ipfv-icvb=pst}\\
	\glt	(Intended meaning: \sqt{When Ali came home, Indira was sewing a dress.})
\end{exe}

The requirement for shared subjects is even stronger for the imperfective converb, for which it is almost the only attested possibility in natural texts. By contrast, for \tit{=qːel} it is easy to find examples with differing subjects \refex{ex:‎When they did not calm down, (he) put them into the box, frightening them}, but still around half to two third of the examples share the subject \refex{ex:‎When he left, he prayed to Allah}, \refex{ex:‎When you buy bread, eat it with cheese!}
%
\begin{exe}
	\ex	\label{ex:‎When he left, he prayed to Allah}
	\gll	[tːura	sa-w-q-un=qːel]	heχ	Allah-li-cːe	ulkː-un-ne\\
		outside	\tsc{hither-m-}go\tsc{.pfv-pret=}when	\tsc{dem.down}	Allah\tsc{-obl-in}	pray\tsc{-icvb-cvb}\\
	\glt	\sqt{‎When he left, he prayed to Allah.}
\end{exe}

In clauses with disjoint subjects, normally at least one of the subjects \refex{ex:Stalin died, and the cars, the trains were stopped making tooot}, \refex{ex:When they got married, they had a good life}, if not both are overt. However, even in those cases it is possible that both subjects are absent, as in example \refex{ex:‎When they did not calm down, (he) put them into the box, frightening them}, in which it is clear from the context that the referent of the subject of the first clause is the children, and that the referent of the subject in the main clause as well as in the following adverbial clause is the main character of the story.
%
\begin{exe}
	\ex	\label{ex:‎When they did not calm down, (he) put them into the box, frightening them}
	\gll	[a-b-ug-an=qːel]	b-i-ka-b-at-ur	ca-b	[q'ʷani-l-cːe	uruχ	b-arq'-ib-le]\\
		\tsc{neg-hpl-}be.calm\tsc{.ipfv-ptcp=}when	\tsc{hpl-in-down}\tsc{-hpl-}leave\tsc{.pfv-pret}	\tsc{cop-hpl}	box\tsc{-obl-in}	fear	\tsc{hpl-}do\tsc{.pfv-pret-cvb}\\
	\glt	\sqt{‎When they did not calm down, (he) put (the children) into the box, frightening them.}
\end{exe}

Co-referential arguments are omitted, so zeroes commonly occur in the subordinate clause. Therefore, cataphora is very frequent. In example \refex{ex:When he felt the warmth of the sun, he thanked the sun@35} the omitted argument in the first clause corresponds to the agent in the second clause.
%
\begin{exe}
	\ex	\label{ex:When he felt the warmth of the sun, he thanked the sun@35}
	\gll	[bari-la	gʷana-dex-li-j	šak ič-ib-le]	il-i-l	bari-li-j	barkalla 	b-aχ-aq-ur\\
		sun\tsc{-gen}	warm\tsc{-nmlz-obl-dat}	feel occur\tsc{.m.pfv-pret-cvb}	that\tsc{-obl-erg}	sun\tsc{-obl-dat}		thanks 	\tsc{n-}know\tsc{.pfv-caus-pret}\\
	\glt	\sqt{When he felt the warmth of the sun, he thanked the sun.} 
\end{exe}

But anaphora is also attested \refex{ex:The bird run (i.e. flies) after him and his dog, and they run and run, and shout, but they did not find the frog@36}. In this example, we find G=S=S=A, with only the first G argument being a full noun phrase and all other occurrences of the same argument left implicit, so that no grammatical relations are involved.
%
\begin{exe}
	\ex	\label{ex:The bird run (i.e. flies) after him and his dog, and they run and run, and shout, but they did not find the frog@36}
	\gll	[hitːi	b-uq-un-ne	č'aka	χːʷe-j=ra	hel-i-j=ra]	[sa-r-b-uq-un-ne,	sa-r-b-uq-un-ne] 	[waˁw b-ik'-ul]	b-arčː-ib-le=kːu	ʡaˁt'a\\
		after	\tsc{n-}go\tsc{.pfv-pret-cvb}	eagle	dog\tsc{-dat=add}	that\tsc{-obl-dat=add}	\tsc{ante-abl-hpl-}go\tsc{.pfv-pret-cvb}	\tsc{ante-abl-hpl-}go\tsc{.pfv-pret-cvb }	call \tsc{hpl-}say\tsc{.ipfv-icvb}	\tsc{n-}find\tsc{.pfv-pret-cvb=neg}	frog\\
	\glt	\sqt{The bird runs (i.e. flies) after him and his dog, and they run and run, and shout, but they did not find the frog.}
\end{exe}

Another strategy commonly employed is to have the co-referential NP in clause-initial position, syntactically belonging to the main clause, but separated from the rest of the main clause in terms of linear order. The controlee is in the embedded clause, resulting in center embedding. In \refex{ex:I will go to the forest and bring nuts@37}, the adverbial clause contains an intransitive predicate; therefore, the pronoun \tit{dul} \sqt{\tsc{1sg.erg}} must be part of the main clause. If both clauses have the same valency frame, it is in principle impossible to decide to which of the two clauses the overt argument belongs. In general, arguments whose referents the speaker assumes to be known to the hearer are left implicit such that often none of the clauses contains an occurrence of the shared arguments.

Though shared arguments are very common, this is not a necessity. In \refex{ex:Stalin died, and the cars, the trains were stopped making tooot} the first adverbial clause contains an overt S, \tit{Istalin}, which is not shared in the subsequent adverbial and main clause.

The adverbial clause mostly precedes the main clause, but the reverse order is also attested (\refsec{ssec:Linear order and iconicity}). Shared S and A arguments in either order are frequently found in texts \refex{ex:When he felt the warmth of the sun, he thanked the sun@35}, \refex{ex:I will go to the forest and bring nuts@37}, and are easily provided in elicitation \refex{ex:Mother came and fed Madina@39a}, \refex{ex:Murad saw Madina and went away@39b}. The situation gets more complicated if P arguments are also involved. An overt S argument in the first clause can correspond to a covert P in the second clause but not if the verb in the subordinate clause bears the converb suffix \tit{-le}. Instead, the more specific construction with \tit{=qːella} must be used such that the first clause is not only syntactically but also semantically an adverbial clause \refex{ex:When the daughter came, the mother fed@39c}. According to my Sanzhi consultants, the more general converb \tit{-le} can only be used if the S in the converbal clause corresponds to an S or A in the main clause.
%
\begin{exe}
	\ex	\label{ex:Mother, Madina, feeding@39}
	\begin{xlist}
		\ex	\label{ex:Mother came and fed Madina@39a}
		\gll	[aba$_{i}$	sa-r-eʁ-ib-le]	\_$_{i}$	Madina	r-aχː-un\\
			mother	\tsc{hither-f}-come\tsc{-pret-cvb}	\tsc{erg}	Madina	\tsc{f-}feed\tsc{-pret}\\
		\glt	\sqt{Mother came and fed Madina.} (S = A) (E)

		\ex	\label{ex:Murad saw Madina and went away@39b}
		\gll	[Murad-li-j$_{i}$	Madina	či-r-až-ib-le]	\_$_{i}$	ag-ur\\
			Murad\tsc{-obl-dat}	Madina	\tsc{spr-f}-see\tsc{.pfv-pret-cvb}	\tsc{abs}	go\tsc{-pret}\\
		\glt	\sqt{Murad saw Madina and went away.} (A = S) (E)

		\ex	\label{ex:When the daughter came, the mother fed@39c}
		\gll	[rursːi$_{i}$	sa-r-eʁ-ib=qːella]	aba-l	\_$_{i}$	r-aχː-un\\
			daughter	\tsc{hither-f}-come\tsc{-pret}=when	mother\tsc{-erg}	\tsc{abs}	\tsc{f-}feed\tsc{-pret}\\
		\glt	\sqt{When the daughter came, the mother fed (her).} (S = P) (E)
	\end{xlist}
\end{exe}

If the first clause contains two arguments A and P, then an implicit S in the second clause can, in principle, be co-referential with any of these two arguments. However, co-reference with P is less preferable, i.e. in example \refex{ex:Father saw Madina and (she) got happy@40}, the S argument in the second clause can be co-referential with P in the first clause, or with another argument previously established in the context. In natural texts the co-referential argument would rather be expressed as S in the main clause and left implicit in the adverbial clause. In \refex{ex:Murad saw Madina and went away@39b}, co-reference between the A in the first clause and S in the second clause is the preferred reading, and co-reference with a third person is rather unlikely. 
%
\begin{exe}
	\ex	\label{ex:Father saw Madina and (she) got happy@40}
	\gll	[atːa-j	Madina$_{i}$	či-r-až-ib-le]	\_$_{i/j}$	razi	r-iχ-ub\\
		father\tsc{-dat} 	Madina	\tsc{spr-f}-see\tsc{.pfv-pret-cvb}	\tsc{abs}	happy	\tsc{f-}become\tsc{-pret}\\
	\glt	\sqt{Father saw Madina and (she) got happy.} (P = S) (E)
\end{exe}

If we exchange the predicate in the second clause in \refex{ex:When the daughter came, the mother fed@39c} with a transitive predicate, we again encounter the same situation. If the shared argument occurs as P in the adverbial clause, the whole sentence becomes rather marginal because out of context the referent of the omitted A in the main clause could be either the mother or the daughter. Therefore, speakers prefer to express the shared argument as A in the main clause \refex{ex:Mother called her daughter and she swept the house@41}.
%
 \begin{exe}
	\ex	\label{ex:Mother called her daughter and she swept the house@41}
	\gll	[aba-l	\_$_{i}$ až-aq-ur-re]	rursːi-l$_{i}$	qal	qʷaˁrš b-arq'-ib\\
		mother\tsc{-erg} \tsc{abs}	go\tsc{.pfv-caus-pret-cvb}	girl\tsc{-erg}	house	sweep \tsc{n-}do\tsc{.pfv-pret}\\
	\glt	\sqt{Mother called her daughter and she (= the daughter) swept the house.} (P=A) (E)
\end{exe}

Thus, there is some evidence that shared arguments are preferably expressed as S or A instead of P. However, co-reference is never a grammatical necessity. In each of the sentences an implicit argument can always be co-referential with other referents in the contexts that do not occur in the sentence to which the omitted argument belongs.

Pronouns (demonstrative or reflexive) in combination with co-referential noun phrases are usually not employed to express shared arguments, because the use of pronouns often leads to disjoint reference as the only available interpretation. Adverbial clauses preceding the main clause never allow for pronominal cataphoras as we know them from European languages. This means that the demonstrative or reflexive pronoun in \refex{ex:‎When s/he bought bread, Zainab ate (it) with cheese} cannot be co-referential with a following noun phrase.
%
\begin{exe}
	\ex	\label{ex:‎When s/he bought bread, Zainab ate (it) with cheese}
	\gll	[cin-ni	/	it-i-l	t'ult'	asː-ib=qːel]	Zajnab-li	nisːe-li-cːella	b-erk-un\\
		\tsc{refl.sg-erg}	/	that\tsc{-obl-erg}	bread	buy\tsc{.pfv-pret=}when	Zainab\tsc{-erg}	cheese\tsc{-obl-comit}	\tsc{n-}eat\tsc{.pfv-pret}\\
	\glt	\sqt{‎When s/he (i.e. not Zainab) bought bread, Zainab ate (it) with cheese.} (E)
\end{exe}

If we reverse the order of pronoun and noun we also have disjoint reference for the demonstrative pronoun \refex{ex:‎When Hurija came, s/he milked the cow1}. However, with the reflexive pronoun the situation is more complicated because this pronoun can be interpreted as fulfilling a purely emphatic function, which means that the main clause actually lacks an overt subject. This makes it possible, in turn, to arrive at a co-referential reading \refex{ex:‎When Hurija came, s/he milked the cow2}, \refex{ex:‎While Zapir was singing a song while he dug the field}. If we exclude the emphatic interpretation of the reflexive, then in clauses with the \tit{=qːel} converb, disjoint reference is the only possible interpretation, but perfective converbs still seem to allow co-reference. 
%
\begin{exe}
	\ex	\label{ex:‎When Hurija came, s/he milked the cow1}
	\gll	[ħuˁrija	sa-r-eʁ-ib=qːel]	cin-ni	q'ʷal	b-ircː-ib\\
		Hurija	\tsc{hither-f-}go\tsc{.pfv-pret=}when	\tsc{refl.sg-erg}	cow	\tsc{n-}milk\tsc{.pfv-pret}\\
	\glt	\sqt{‎When Hurija came, s/he (i.e. not Hurijat) milked the cow.} (E)

	\ex	\label{ex:‎When Hurija came, s/he milked the cow2}
	\gll	[ħuˁrija	sa-r-eʁ-ib-le]	cin-ni	q'ʷal	b-ircː-ib\\
		Hirija	\tsc{hither-f-}go\tsc{.pfv-pret-cvb}	\tsc{refl.sg-erg}	cow	\tsc{n-}milk\tsc{.pfv-pret}\\
	\glt	\sqt{‎When Hurija came, s/he (Hurijat herself or another person) milked the cow.} (E)

	\ex	\label{ex:‎While Zapir was singing a song while he dug the field}
	\gll	[Zapir	dalaj	Ø-ik'-ul]	cin-ni	qu	b-urqː-ib\\
		Zapir	song	\tsc{m-}say\tsc{.ipfv-icvb}	\tsc{refl.sg-erg}	garden	\tsc{n-}dig\tsc{.pfv-pret}\\
	\glt	\sqt{‎While Zapir was singing a song he (another person or Zapir himself) dug the field.} (E)
\end{exe}

We can also swap around the order of the clauses. In sentences in which the main clause precedes the adverbial clause, no cataphora whatsoever is allowed \refex{ex:S/he ate the bread when Zainab bought it}, \refex{ex:While Zapir was singing a song}. This means that neither zeroes nor pronouns can express co-reference with subject arguments in the following subordinate clauses. A pronoun (or a zero anaphora) may not both precede and c-command its antecedent (\citealp[185]{Langacker1969}; \citealp[8]{Reinhart1976}). Note that if we use demonstrative pronouns or zero, the person reference in the first clause remains unspecified. By contrast, the reflexive pronoun would be used if we continue to talk about a person who already was the topic of the conversation.
%
\begin{exe}
	\ex	\label{ex:S/he ate the bread when Zainab bought it}
	\gll	(cin-ni	/	it-i-l)	t'ult'	b-erk-un,	[Zajnab-li	asː-ib=qːel]\\
		\tsc{refl.sg-erg}	/	that\tsc{-obl-erg}	bread	\tsc{n-}eat\tsc{.pfv-pret}	Zajnab\tsc{-erg}	buy\tsc{.pfv-pret=}when\\
	\glt	\sqt{S/he (i.e. not Zajnab) ate the bread when Zajnab bought it.} (E)

	\ex	\label{ex:While Zapir was singing a song}
	\gll	(cin-ni	/	it-i-l)	qu	b-urqː-ib, 	[Marko	dalaj	Ø-ik'-ul]\\
		\tsc{refl.sg-erg}	/	that\tsc{-obl-erg}	garden	\tsc{n-}dig\tsc{.pfv-pret}	Marko	song	\tsc{m-}say\tsc{.ipfv-icvb}\\
	\glt	\sqt{While Marko was singing a song, s/he (i.e. not Marko) dug the field.} (E)
\end{exe}


% --------------------------------------------------------------------------------------------------------------------------------------------------------------------------------------------------------------------- %

\subsection{Linear order and iconicity}
\label{ssec:Linear order and iconicity}

This criterion concerns the linear order of adverbial clause and main clause (``position'' and ``layer'' in the terminology of \citealp{Bickel2010}). Although the adverbial clauses most frequently precede the main clause, they may also follow it \xxref{ex:‎‎‎I remained there for two years, unable to eat food made of corn}{ex:‎‎‎I came to know the truth from you, I said, because he (the other doctor) did not tell me (the truth)}, \refex{ex:‎When they did not calm down, (he) put them into the box, frightening them}, and they may be separated by other subordinate clauses from the main clause, e.g. by other adverbial clauses.

In \refex{ex:‎‎‎I remained there for two years, unable to eat food made of corn}, the imperfective converb clause follows the main clause and shares with the main clause the subject referent and the past time reference. In \refex{ex:Then, in the maternity hospital, here you do not give money when you go to take (the child) out (of the hospital and home)} the converbal clause with \tit{=qːel} also follows the main clause and most probably shares the subject-like argument. In \refex{ex:‎‎‎I came to know the truth from you, I said, because he (the other doctor) did not tell me (the truth)} we again have a converbal clause with \tit{=qːel} that follows the main clause and has a causal interpretation.
%
\begin{exe}
	\ex	\label{ex:‎‎‎I remained there for two years, unable to eat food made of corn}
	\gll	k'ʷi	dus	kelg-un=da		[ʡaˁžlač'i-la	χurejg	b-erkʷ-ij	a-r-irχ-ul]\\
		two	year	remain\tsc{.pfv-pret=1}	corn\tsc{-gen}	food	\tsc{n-}eat\tsc{.pfv-inf}	\tsc{neg-f-}be.able\tsc{.ipfv-icvb}\\
	\glt	\sqt{‎‎‎I remained there for two years, unable to eat food made of corn.}

	\ex	\label{ex:Then, in the maternity hospital, here you do not give money when you go to take (the child) out (of the hospital and home)}
	\gll	c'il	roddom-le	heštːu-d	lukː-unne=kːu=w	ce=ja	arc	[tːura	h-asː-ij	r-ax-an=qːel]?\\
		then	maternity.hospital\tsc{-loc}	here\tsc{-npl}	give\tsc{.ipfv-icvb=}\tsc{cop.neg=q}	what\tsc{=q}	money		outside	\tsc{up}-take\tsc{.pfv-inf}	\tsc{f-}go\tsc{-ptcp=}when\\
	\glt	\sqt{Then, in the maternity hospital, here you do not give money when you go to take (the child) out (of the hospital and home)?}

	\ex	\label{ex:‎‎‎I came to know the truth from you, I said, because he (the other doctor) did not tell me (the truth)}
	\gll	wallah,	haʔ-ib=da,	[a-cːe	hel	b-arx-dex	b-aχ-ij	bahanne]	sa-r-ač'-ib-il=da		[ik'-i-l	a-b-urs-ib=qːel]\\
		by.God	say\tsc{.pfv-pret=1}	\tsc{2sg-in}	that	\tsc{n-}right\tsc{-nmlz}	\tsc{n-}know\tsc{.pfv-inf}	because.of	\tsc{hither}\tsc{-f-}come\tsc{.pfv-pret-ref=1}	\tsc{dem.up}\tsc{-obl-erg}	\tsc{neg-n-}tell\tsc{-pret=}when\\
	\glt	\sqt{‎‎‎I came to know the truth from you, I said, because he (the other doctor) did not tell me (the truth).}
\end{exe}

Examples \xxref{ex:‎When / Because Murad got ill he did not build the fence}{ex:‎Madina, having come home, washed the dishes} show center-embedding, i.e. adverbial clauses that occur within the main clause. That it is in fact center-embedding and not adverbial clauses preceding the main clauses is indicated by the case-marking on the shared argument. The verb in the adverbial clauses differs from the verb in the main clause in transitivity, and the case of the shared argument is assigned by the predicate in the main clause. Note that in all examples the only interpretation available is the shared subject interpretation. 
%
\begin{exe}
	\ex	\label{ex:‎When / Because Murad got ill he did not build the fence}
	\gll	Murad-li	[ʡaˁrkːa	Ø-iχ-ub-le]	lac	a-b-arq'-ib\\
		Murad\tsc{-erg}	ill	\tsc{m-}be\tsc{.pfv-pret-cvb}	fence	\tsc{neg-n-}do\tsc{.pfv-pret}\\
	\glt	\sqt{‎When\slash Because Murad got ill he did not build the fence.} (E)

	\ex	\label{ex:Musa is singing a song and building the fence}
	\gll	Musa-l	[dalaj	Ø-ik'-ul]	lac	b-irq'-ul	ca-b\\
		Musa\tsc{-erg}	song	\tsc{m-}say\tsc{.ipfv-icvb}	fence	\tsc{n-}do\tsc{.ipfv-icvb}	\tsc{cop-n}\\
	\glt	\sqt{Musa is singing a song and building the fence.} (E)

	\ex	\label{ex:‎Madina, having come home, washed the dishes}
	\gll	Madina-l	[qili	sa-r-eʁ-ib=qːel]	t'alaħ-ne	d-irc-ib\\
		Madina\tsc{-erg}	home	\tsc{hither-f-}go\tsc{.pfv-pret=}when	dishes\tsc{-pl}	\tsc{npl-}wash\tsc{.pfv-pret}\\
	\glt	\sqt{‎Madina, having come home, washed the dishes.} (E)
\end{exe}

It has been observed for the perfective converb in other Dargwa varieties and other East Caucasian languages that when the subjects are not identical, the order of main clause and adverbial clause can be changed, but then only the causal interpretation is possible (\citealp{Belyaev2010}; \citealp{Kustova2015}; \citealp{Kazenin.Testelets2004}). In other words, when the adverbial clause precedes the main clause, we can have both a same-subject and a different-subject reading \refex{ex:‎When / Because Murad got ill he (= Murad or some other person) did not build the fence}. However, the different-subject reading is rather marginal and only available in the right context (see the discussion in \refsec{ssec:Co-reference and expression of shared arguments} about example \refex{ex:When Ali came home, Indira was sewing a dress}).
%
\begin{exe}
	\ex	\label{ex:‎When / Because Murad got ill he (= Murad or some other person) did not build the fence}
	\gll	[Murad	ʡaˁrkːa	Ø-iχ-ub-le]	lac	a-b-arq'-ib\\
		Murad	ill	\tsc{m-}be\tsc{.pfv-pret-cvb}	fence	\tsc{neg-n-}do\tsc{.pfv-pret}\\
	\glt	\sqt{‎When\slash Because Murad got ill he (= Murad or some other person) did not build the fence.} (E)
\end{exe}

If we reverse the order, interpretations with shared subjects are more frequently disapproved, e.g. \refex{ex:‎‎‎(He) is digging the field while Musa is singing} means that an unspecified person is digging the field while Murad is singing. For the perfective converb, a reversal of the order means that a causal interpretation between the two described situations is required \refex{ex:Because Murad got ill, he (= Murad or another person) did not build the fence2}, whereas in the default order, in which the adverbial clause precedes the main clause, a causal interpretation is possible, but not necessary. Sentences such as \refex{ex:‎When / Because Murad got ill he (= Murad or some other person) did not build the fence} can also simply express the temporal order of the events as occurring simultaneously or sequentially without implying a causal relationship.
%
\begin{exe}
	\ex	\label{ex:‎‎‎(He) is digging the field while Musa is singing}
	\gll	qu	uqː-ul	ca-w	[Musa	dalaj	Ø-ik'-ul]\\
		garden	dig\tsc{.ipfv-icvb}	\tsc{cop-m}	Musa	song	\tsc{m-}say\tsc{.ipfv-icvb}\\
	\glt	\sqt{‎‎‎(He) is digging the field while Musa is singing.} (E)

	\ex	\label{ex:Because Murad got ill, he (= Murad or another person) did not build the fence2}
	\gll	lac	a-b-arq'-ib	[Murad	ʡaˁrkːa	Ø-iχ-ub-le]\\
		fence	\tsc{neg-n-}do\tsc{.pfv-pret}	Murad	ill	\tsc{m-}be\tsc{.pfv-pret-cvb}\\
	\glt	\sqt{Because Murad got ill, he (= Murad or another person) did not build the fence.} (E)
\end{exe}

This means that the order of the clauses in constructions with perfective and imperfective converbs cannot be changed without a concomitant change in the interpretations. This property makes the respective converb constructions slightly similar to clause coordination, which also depicts the order of the events if they do not occur simultaneously: the first clause refers to the first event, the second clause to the second event. By contrast, for other converbs such as the temporal converb \tit{=qːel}, it is possible to reverse the order of the clauses without changing the interpretation, which makes them more similar to subordination \refex{ex:Then, in the maternity hospital, here you do not give money when you go to take (the child) out (of the hospital and home)}, \refex{ex:‎‎‎I came to know the truth from you, I said, because he (the other doctor) did not tell me (the truth)}.


% --------------------------------------------------------------------------------------------------------------------------------------------------------------------------------------------------------------------- %

\subsection{Morphosyntactic locus}
\label{ssec:Morphosyntactic locus}

In addition to the properties discussed, I also tested for morphosyntactic locus \citep{Kazenin.Testelets2004}, i.e. the locus of marking a complement clause as dependent on the main clause. For coordination embedded into a complement clause, the formal marking of embedding is expected to occur on each member of the coordination. By contrast, in case of subordination we can expect the formal marking to occur only on the head or within the head constituent of the complement, but not within another adverbial clause that is part of the complement. This is the case for Sanzhi adverbial clauses that can occur in complement constructions. For instance, in \refex{ex:‎‎‎I am sad because Murad got ill and did not build the fence4} and \refex{ex:‎I am happy when Murad got healthy and finished building the fence5} the masdar suffix that marks the complement clause as dependent occurs only on one verb, whereas the other verb in the complement retains its converbal suffix. In \refex{ex:‎I am happy that Fatimat dug the field while singing a song} complementation is achieved by means of the cross-categorical suffix -\textit{ce} added to the preterite.
%
\begin{exe}
	\ex	\label{ex:‎‎‎I am sad because Murad got ill and did not build the fence4}
	\gll	du	pašman-ne=da [[Murad	ʡaˁrkːa	Ø-iχ-ub-le]	lac	a-b-arq'-ni]\\
		\tsc{1sg}	sad\tsc{-advz=1}	Murad	ill	\tsc{m-}be\tsc{.pfv-pret-cvb}	fence	\tsc{neg-n-}do\tsc{.pfv-msd}\\
	\glt	\sqt{‎‎‎I am sad because Murad got ill and did not build the fence.} (E)

	\ex	\label{ex:‎I am happy when Murad got healthy and finished building the fence5}
	\gll	du	razi-l=da	[[Murad	ʡaˁħ	Ø-iχ-ub=qːel]	lac	taman	b-arq'-ni]\\
		\tsc{1sg}	happy\tsc{-advz=1}	Murad	good	\tsc{m-}be\tsc{.pfv-pret=}when	fence	end	\tsc{n-}do\tsc{.pfv-msd}\\
	\glt	\sqt{‎I am happy when Murad got healthy and finished building the fence.} (E)

	\ex	\label{ex:‎I am happy that Fatimat dug the field while singing a song}
	\gll	du	razi-l=da	[[Fat'imat	dalaj	r-ik'-ul]	qu	b-urqː-ib-ce]\\
		\tsc{1sg}	happy\tsc{-advz=1}	Fatimat	song	\tsc{f-}say\tsc{.ipfv-icvb}	garden	\tsc{n-}dig\tsc{.pfv-pret-dd.sg}\\
	\glt	\sqt{‎I am happy that Fatimat dug the field while singing a song.} (E)
\end{exe}


% --------------------------------------------------------------------------------------------------------------------------------------------------------------------------------------------------------------------- %

\subsection{Island constraints: relativization and extraction}
\label{ssec:Island constraints: relativization and extraction}

The data concerning extraction out of relative clauses varies depending on the converb used and on the interpretations available. The converb \tit{=qːel} blocks extraction, as example \refex{ex:The driver who when driving the car the girls got afraid is our father} shows. By contrast, the perfective converb allows for extraction \refex{ex:‎Give me the i-phone that when it was given to Hazhimurad we got happy.}. Although the data in \xxref{ex:When father was guiding the car, the girls became afraid}{ex:‎Give me the i-phone that when it was given to Hazhimurad we got happy.} generally fits what has been observed for other East Caucasian languages (e.g. \citealp{Kazenin.Testelets2004}; \citealp{Creissels2012}; \citealp{Bickel2010}), two divergent examples are not enough to understand whether Sanzhi adverbial clauses show the behavior of coordination or of subordination and to what extent this depends on the converbs themselves or on the available interpretations.
%
\begin{exe}
	\ex	\label{ex:father was guiding the car}
	\begin{xlist}

	\ex	\label{ex:When father was guiding the car, the girls became afraid}
	\gll	[šupir-ri mašin	b-ik-an=qːel]	rurs-be	uruχ	b-iχ-ub	\\
		driver\tsc{-erg}	car	\tsc{n-}lead\tsc{.ipfv-ptcp=}when	girl\tsc{-pl}	fear	\tsc{hpl-}be\tsc{.pfv-pret}\\
	\glt	\sqt{When the driver was guiding the car, the girls became afraid.} (E)

	\ex	\label{ex:The driver who when driving the car the girls got afraid is our father}
	\gll	{*}	[[\_$_{i}$ mašin	b-ik-an=qːel]	rurs-be	uruχ	b-iχ-ub]	šupir$_{i}$	nušːa	atːa	ca-w\\
		{}	\tsc{erg} car	\tsc{n-}lead\tsc{.ipfv-ptcp=}when	girl\tsc{-pl}	fear	\tsc{hpl-}be\tsc{.pfv-pret}	driver	\tsc{1pl}	father	\tsc{cop-m}\\
	\glt	(Intended meaning: \sqt{The driver who when driving the car the girls got afraid is our father.}) (E)
	\end{xlist}
\end{exe}

\begin{exe}
	\ex	\label{ex:father an iphone was given}
	\begin{xlist}
		\ex	\label{ex:When an iphone was given to Hazhimurad we got happy}
	\gll	[ħaˁžimurad-li-j	ajpun b-ičː-ib-le]	nušːa	razi	d-iχ-ub=da\\
		Hazhimurad-\tsc{obl-dat}	i-phone \tsc{n-}give\tsc{.pfv-pret-cvb}	\tsc{1pl}	happy	\tsc{1/2pl-}be\tsc{.pfv-pret=1}\\
	\glt	\sqt{When an i-phone was given to Hazhimurad we got happy.} (E)

	\ex	\label{ex:‎Give me the i-phone that when it was given to Hazhimurad we got happy.}
	\gll	[[ħaˁžimurad-li-j	\_$_{i}$	b-ičː-ib-le]	nušːa	razi	d-iχ-ub-il]	ajpun$_{i}$	b-iqː-a!\\
		Hazhimurad\tsc{-obl-dat}	\tsc{abs}	\tsc{n-}give\tsc{.pfv-pret-cvb}	\tsc{1pl}	happy	\tsc{1/2pl-}be\tsc{.pfv-pret-ref}	i-phone	\tsc{n-}take.out\tsc{.ipfv-imp}\\
	\glt	\sqt{‎Give me the i-phone that when it was given to Hazhimurad we got happy.} (E)
	\end{xlist}
\end{exe}


% --------------------------------------------------------------------------------------------------------------------------------------------------------------------------------------------------------------------- %

\subsection{Summary}
\label{ssec:adverbialclausessummary}

\reftab{tab:Morphosyntactic properties of adverbial clauses} summarizes some of the morphosyntactic properties of perfective and imperfective converb clauses as well as adverbial clauses with \tit{=qːel} that have been discussed in the previous sections. The table shows that the three converbs by and large share most of their properties. If we compare the behavior of Sanzhi adverbial clause constructions with adverbial clauses in other East Caucasian languages, we also find that Sanzhi converb constructions strongly resemble their counterparts in other languages of the family (e.g. \citealp{Forker2013b} on Tsezic; \citealp{Creissels2010, Creissels2012} on Akhvakh; \citealp{Bickel2010} on Chechen).
%
\begin{table}
	\caption{Morphosyntactic properties of adverbial clauses}
	\label{tab:Morphosyntactic properties of adverbial clauses}
	\small
	\begin{tabularx}{0.88\textwidth}[]{%
		>{\raggedright\arraybackslash}X
		>{\raggedright\arraybackslash}p{90pt}
		>{\raggedright\arraybackslash}p{80pt}}
		
		\lsptoprule
			Variable			&	\tsc{ipfv}\slash\tsc{pfv} converb	&	\tit{=qːel}\\
		\midrule
			Illocutionary scope		&	local\slash extensible	&	local\\
			Illocutionary marking	&	\multicolumn{2}{c}{banned}	\\
			Tense scope			&	\multicolumn{2}{c}{conjunct}	\\
			Tense marking		&	\multicolumn{2}{c}{banned}		\\
			Finiteness			&	\multicolumn{2}{c}{non-finite}	\\
			Symmetry			&	\multicolumn{2}{c}{asymmetrical}\\
			WH-words			&	\multicolumn{2}{c}{allowed}		\\
			Focus-sensitive particles	&	\multicolumn{2}{c}{allowed}		\\
			Extraction			&	no data\slash allowed	&	disallowed\\
			Position			&	\multicolumn{2}{c}{flexible-relational}	\\
		\lspbottomrule
	\end{tabularx}
\end{table}


% --------------------------------------------------------------------------------------------------------------------------------------------------------------------------------------------------------------------- %

\subsection{Adverbial clauses as independent utterances?}
\label{ssec:Adverbial clauses as independent utterances}

When examining natural texts it is striking to notice that adverbial clauses headed by perfective and imperfective converbs occur sometimes without a main clause that is obviously connected to it. Example \refex{ex:‎‎Then slowly I learned the language and I did my (military) service} illustrates a perfective converb clause, followed by an imperfective converb clause, and then the speaker concludes his narrative about his military service with a comment that is not directly related to the two preceding adverbial clauses. The utterance in \refex{ex:‎‎The brother of my father (= Abdulkhalik) tore down the wall}, which consists of three adverbial clauses with preterite converbs, describes what the speaker's uncle Abdulkhalik did in order to build himself a house. It is followed by a comment that explicitly states the name of the uncle, but not by a main clause referring to the building of the house, which would be expected based on the general rules of use for the perfective converb.
%
\begin{exe}
	\ex	\label{ex:‎‎Then slowly I learned the language and I did my (military) service}
	\gll	c'il=ra	hel-tːi	bahla		bahla-l	ʁaj=ra	d-aχ-ur-re		bahla		bahla-l	islužba=ra b-iqː-ul\ldots\\
		then=\tsc{add}		that-\tsc{pl}	slow		slow-\tsc{advz}	language=\tsc{add}	\tsc{npl}-know.\tsc{pfv-pret-cvb}	slow	slow-\tsc{advz}	service=\tsc{add} \tsc{n}-carry.\tsc{ipfv-icvb}\\
	\glt	\sqt{‎‎Then slowly I learned the language and I did my (military) service, (To be honest, I stayed for three years, and I was not one single hour at the guardhouse.)} 

\ex	\label{ex:‎‎The brother of my father (= Abdulkhalik) tore down the wall}
\gll [di-la	atːa-la	ucːi-l	ha-b-ertː-ib-le	il	b-aʔ]		[ʡaˁħ-te	[cin-na	taχna	b-arq'-ij]	d-erqː-ib-le	už-ib-le]		[wahi-te	heχtːu	lak'	d-i-ka-d-arq'-ib-le	d-už-ib-le	ʡaˁbdulχaliq'-li]\ldots \\
\tsc{1sg-gen}	father-\tsc{gen}	brother-\tsc{erg}	\tsc{up-n}-take.\tsc{pfv-pret-cvb}	that	\tsc{n}-edge	good-\tsc{dd.pl} 	\tsc{refl.sg.obl-gen}	room	\tsc{n}-do.\tsc{pfv-inf}	\tsc{npl}-carry.\tsc{pfv-pret-cvb}	be-\tsc{pret-cvb}		bad-\tsc{dd.pl} 	there.\tsc{down}	throw	\tsc{npl-in-down-npl}-do.\tsc{pfv-pret-cvb}	\tsc{npl}-be-\tsc{pret-cvb}	Abdulkhalik-\tsc{erg} \\
\glt	\sqt{‎‎The brother of my father (= Abdulkhalik) tore down the wall, apparently took the good (materials) in order to build his house (=room), the bad (materials) Abdulkhalik threw away there, (My fathers brother was called Abdulkhalik.)}

\end{exe}

Therefore, we might wonder if we perhaps observe an ongoing change in which subordinate verb forms develop into forms that can head independent main clauses. For Mehweb Dargwa, it has been observed in elicitation that some speakers allow perfective and imperfective converbs to head main clauses \citep{KustovaForthcoming}, although the corpus does not contain any examples. In Sanzhi, the situation is reversed: in elicitation, examples such as \refex{ex:‎‎Then slowly I learned the language and I did my (military) service} and \refex{ex:‎‎The brother of my father (= Abdulkhalik) tore down the wall} are clearly judged as subordinate clauses, but in narrations we find again and again subordinate clauses with a missing main clause. The following excerpt from a discussion between two speakers illustrates the phenomenon. The conversation starts with a question by speaker A \refex{ex:‎‎‎Why did he die}, which is then answered by speaker B. About half of the utterances by speaker B are formally subordinate clauses.
%
\begin{enumerate}
	\item	finite clause (speaker A)
	%
	\begin{exe}
		\ex	\label{ex:‎‎‎Why did he die}
		\gll	c'il	cellij	w-ebč'-ib-le=de?\\
			then	why	\tsc{m-}die\tsc{.pfv-pret-cvb=pst}	\\
		\glt	\sqt{‎‎‎Why did he die?}
	\end{exe}

	\item	non-finite clause as answer (speaker B)
	%
	\begin{exe}
		\ex	\label{ex:‎in the ditches, he died because of hunger}
		\gll	cin-na	hetːu	qːanaw-t-a-cːe-w	kːiši-l	w-ebč'-ib-le, \ldots\\
			\tsc{refl.sg-gen}	there	ditch\tsc{-pl-obl-in-m}	hunger\tsc{-adv}	\tsc{m-}die\tsc{.pfv-pret-cvb}\\
		\glt	\sqt{‎in the ditches, he died because of hunger, \ldots}
	\end{exe}

	\item	finite clause (speaker B)
	%
	\begin{exe}
		\ex	\label{ex:‎What could it be, why should he die}
		\gll	Ø-irχʷ-an=de	cellij	ubk'-an-ne	c'il\\
			\tsc{m-}be\tsc{.ipfv-ptcp=pst}	why	die\tsc{.m.ipfv-ptcp-prs.3}	then\\
		\glt	\sqt{Something must have happened to him, why should he die (i.e. what other reasons were there to die at that time).}
	\end{exe}

	\item	non-finite clause (speaker B)
	%
	\begin{exe}
		\ex	\label{ex:He got ill, and they let him go}
		\gll	zaˁʡip	Ø-ič-ib-le,	w-ataʁ-ib-le, \ldots\\
			ill	\tsc{m-}occur\tsc{.pfv-pret-cvb}	\tsc{m-}let\tsc{.pfv-pret-cvb}\\
		\glt	\sqt{He got ill, and they let him go, \ldots}
	\end{exe}

	\item	non-finite clause (speaker B)
	%
	\begin{exe}
		\ex	\label{ex:AAAAAAAAAAAAHHH}
		\gll	nuz-b-a-l	b-ukː-unne,	χalq'	nuz-b-a-l	t'ut'u	ka-b-ik'-ul, \ldots\\
			louse\tsc{-pl-obl-erg}	\tsc{hpl-}eat\tsc{.ipfv-icvb}	people	louse\tsc{-pl-obl-erg}	spread	down\tsc{-hpl-}move\tsc{.ipfv-icvb}\\
		\glt	\sqt{‎‎‎The lice were eating (the people), lice were all over (the people), \ldots}
	\end{exe}

	\item	finite clause (speaker B)
	%
	\begin{exe}
		\ex	\label{ex:‎They died without stopping, in Sanzhi (people) died}
		\gll	b-ubč'-i	naχadu.	Sanži-b	b-ebč'-ib\\
			\tsc{hpl-}die\tsc{.pfv-hab.pst.3}	without.break	Sanzhi\tsc{-hpl}	\tsc{hpl-}die\tsc{.pfv-pret}\\
		\glt	\sqt{‎They died without stopping. In Sanzhi (people) died.}
	\end{exe}
\end{enumerate}

\citet{Mithun2008}, examines the development of subordinate clauses into main clauses in Navajo, Central Alaskan, Yup'ik, and a few other languages, and notes that the respective sentences contain background information, evaluations or comments that do not advance the storyline. However, this does not seem to be the case in Sanzhi. In both examples \refex{ex:‎‎Then slowly I learned the language and I did my (military) service} and \refex{ex:‎‎The brother of my father (= Abdulkhalik) tore down the wall}, it is rather the other way around. The adverbial clauses drive forward the narrative and the main clauses that follow them provide background information or evaluations. And when we compare the main clauses with the subordinate clauses in \refex{ex:‎‎‎Why did he die} to \refex{ex:‎They died without stopping, in Sanzhi (people) died}, there is no obvious division into story line and background information that correlates with the use of converbs and finite verb forms. Only a more detailed study of the Sanzhi corpus can help to clarify whether we really observe an ongoing change, or whether utterances such as the ones discussed in this Section can simply be explained as natural, unprepared spoken text or perhaps performance errors.


%%%%%%%%%%%%%%%%%%%%%%%%%%%%%%%%%%%%%%%%%%%%%%%%%%%%%%%%%%%%%%%%%%%%%%%%%%%%%%%%

\section{The syntax of conditional clauses}
\label{sec:The syntax of conditional clauses}

Conditional clauses behave syntactically like adverbial clauses, but also show some differences; for the morphological structure and their functions see \refsec{cpt:conditionalconcessiveclauses}. Firstly, conditional clauses have person agreement. Secondly, conditional clauses express the difference between present/future time and past time reference, and they can also express irrealis mood. Thirdly, an imperative marker in a main clause does not have scope over the conditional clause \refex{ex:‎‎‎If you need me, burn one hair}, such that the illocutionary scope is always ``local''. Conditional clauses may share their subject or other arguments with the main clause, but this is not a requirement. They mostly precede the main clause, but can also follow it \refex{ex:‎‎‎(They) will operate your eye, she said, if you go (to the doctor)}.
%
\begin{exe}
	\ex	\label{ex:‎‎‎If you need me, burn one hair}
	\gll	[du	ħaˁžat	b-ik'-ulle]	b-ikːʷ-a	ca	ʁez!\\
		\tsc{1sg}	need	\tsc{n-}say\tsc{.ipfv-cond.1.prs}	\tsc{n-}burn\tsc{-imp}	one	hair\\
	\glt	\sqt{‎‎‎If you need me, burn one hair!}

	\ex	\label{ex:‎‎‎(They) will operate your eye, she said, if you go (to the doctor)}
	\gll	``ala	ul-li-j	aparacija	b-irq'-u,''	r-ik'ʷ-ar,	``[r-uq'-utːe]''\\
		\tsc{2sg.gen}	eye\tsc{-obl-dat}	operation	\tsc{n-}do\tsc{.ipfv-prs}	\tsc{f-}say\tsc{.ipfv-prs}	\tsc{f-}go\tsc{-cond.2sg.prs}\\
	\glt	\sqt{‎‎‎``(They) will operate your eye,'' she said, ``if you (fem.) go (to the doctor).''}
\end{exe}

The past conditional occurs recurrently without an apodosis \refex{ex:‎if he did not drink first}. Such sentences can also express wishes \refex{ex:‎‎‎In which year it was, beloved Allah, if I would remember the years}.
%
\begin{exe}
	\ex	\label{ex:‎if he did not drink first}
	\gll	bahsar	a-b-učː-an-del	iž	ce-del, \ldots	\\
		first	\tsc{neg-n-}drink\tsc{.ipfv-ptcp-cond.pst}	this	what\tsc{-indef}\\
	\glt	\sqt{‎if he did not drink first, \ldots}

	\ex	\label{ex:‎‎‎In which year it was, beloved Allah, if I would remember the years}
	\gll	čum-ib	dusː-i-cːe-b=de=l	w-ikː-an	Allah	dus-me	han	d-ik-ar-del, \ldots\\
		how.many\tsc{-ord}	year\tsc{-obl-in-n=pst=indq}	\tsc{m-}want\tsc{.ipfv-ptcp}	Allah	year\tsc{-pl}	remember	\tsc{npl-}occur\tsc{.pfv-prs-cond.pst}\\
	\glt	\sqt{‎‎‎In which year it was, beloved Allah, if I would remember the years, \ldots}
\end{exe}

Interrogative pronouns \refex{ex:If Zapir buys what Zainab will marry him} and focus-sensitive particles \refex{ex:If Zapir buys only a car, Zainab will not marry him} are allowed to occur in conditional clauses. Extraction out of conditional clauses is blocked \refex{ex:The car that if Zapir buys it Zainab will marry him is a foreign car}:
%
\begin{exe}
	\ex	\label{ex:Zapir, Zainab, and Heloise}
	\begin{xlist}
		\ex	\label{ex:If Zapir buys what Zainab will marry him}
		\gll	[Zapir-ri	ce	asː-ar]	Zajnab	xadi	r-ax-an-ne?\\
			Zapir\tsc{-erg}	what buy\tsc{.pfv-cond.3}	Zainab	married	\tsc{f-}go\tsc{-ptcp-fut.3}\\
		\glt	\sqt{If Zapir buys what Zainab will marry him?} (E) 

		\ex	\label{ex:If Zapir buys only a car, Zainab will not marry him}
		\gll	[Zapir-ri	mašin=cun	asː-ar]	Zajnab	xadi	a-r-ax-an-ne\\
			Zapir\tsc{-erg}	car=only buy\tsc{.pfv-cond.3}	Zainab	married	\tsc{neg-f-}go\tsc{-ptcp-3}\\
		\glt	\sqt{If Zapir buys only a car, Zainab will not marry him.} (E) 

		\ex	\label{ex:If Zapir buys a foreign car, Zajnab will marry him}
		\gll	[Zapir-ri	mašin asː-ar]	Zajnab	xadi	r-ax-an-ne\\
			Zapir\tsc{-erg}	car buy\tsc{.pfv-cond.3}	Zainab	married	\tsc{f-}go\tsc{-ptcp-fut.3}\\
		\glt	\sqt{If Zapir buys a car, Zajnab will marry him.} (E) 

		\ex	\label{ex:The car that if Zapir buys it Zainab will marry him is a foreign car}
		\gll	{*}	[[Zapir-ri	\_$_{i}$	asː-ar]	Zajnab	xadi	r-ax-an]	mašin$_{i}$	inomarka	ca-b\\
			{}	Zapir\tsc{-erg}	\tsc{abs}	buy\tsc{.pfv-cond.3}	Zainab	married	\tsc{f-}go\tsc{-ptcp}	car	foreign.car	\tsc{cop-n}\\
		\glt	(Intended meaning: \sqt{The car that if Zapir buys it Zainab will marry him is a foreign car.}) (E) 
	\end{xlist}
\end{exe}
