\chapter{Agreement}
\label{cpt:Agreement}

Sanzhi Dargwa has gender, number and person agreement. Formally, there are several systems of agreement exponents that act completely independently from each other and are therefore treated separately. We can distinguish between pure number agreement, combined gender/number agreement and person agreement. Pure number agreement occurs noun-phrase internally and at the clausal level with a restricted number of TAM forms (\refsec{sec:Pure number agreement}). Combined gender and number agreement is attested for the vast majority of East Caucasian languages, including Sanzhi Dargwa (\refsec{sec:Gender/number agreement}). It is often found within the noun phrase and at the clausal level with all TAM forms, including verb forms such as converbs and participles. Person agreement is rather rare for East Caucasian languages. Among the languages that have it are Dargwa languages such as Sanzhi (\refsec{sec:Person agreement}), Lak, Tabasaran, Batsbi (Tsova-Tush), Udi, and to a lesser extend Hunzib, Akhvakh, and some Avar varieties (see \citealp{Helmbrecht1996, vandenBerg1999, Schulze2007}). It only occurs at the level of the clause.

I will use the terms \textit{agreement}, \textit{target}, and \textit{controller} in the sense of \citealp{Corbett2006} to describe the properties of the three types of agreement in Sanzhi.


%%%%%%%%%%%%%%%%%%%%%%%%%%%%%%%%%%%%%%%%%%%%%%%%%%%%%%%%%%%%%%%%%%%%%%%%%%%%%%%%

\section{Pure number agreement}
\label{sec:Pure number agreement}
Pure number agreement is found in the noun phrase and at the clausal level. Within the noun phrase, demonstrative pronouns (\refsec{sec:Demonstrative pronouns}) and definite descriptions formed by means of the cross-categorical suffix -\textit{ce} (\refsec{ssec:The -ce / -te attributive}) agree with the head noun in number. If the head noun is in the plural the demonstrative pronoun must occur in the plural and the cross-categorical suffix must change to \tit{-te} (or be omitted)  \refex{ex:‎a big puddle of water}, \refex{ex:‎There are big targets}.
%
\begin{exe}
	\ex	\label{ex:‎a big puddle of water}
	\gll	hin-na	χːula-ce	šuˁra\\
		water\tsc{-gen}	big\tsc{-dd.sg}	puddle\\
	\glt	\sqt{‎a big puddle of water}

	\ex	\label{ex:‎There are big targets}
	\gll	χːula-te	q'asta-ne	le-d\\
		big\tsc{-dd.pl} 	target\tsc{-pl}	exist\tsc{-npl}\\
	\glt	\sqt{‎There are big targets.}
\end{exe}

Noun phrases modified by numerals other than \textit{ca} `one' are semantically plural and thus require demonstratives to appear in the plural and prohibit the use of the singular cross-categorical suffix\footnote{The suffix can also be omitted because the adjectives in attributive function can generally occur with or without it.} although no overt plural marking on the noun occurs \refex{ex:‎‎‎three old cowsAGREE}. Some mass nouns also require plural agreement even though they are not overtly marked for plural, e.g. terms for ethnic groups of inhabitants of villages \refex{ex:‎When all Chakhri people moved to the lowlands, we (also) moved2}.

\begin{exe}
	\ex	\label{ex:‎‎‎three old cowsAGREE}
	\gll	hel-tːi ʡaˁbal	d-uqna(-te)	q'ʷal\\
		that\tsc{-pl} three \tsc{npl-}old\tsc{-dd.pl} 	cow\\
	\glt	\sqt{those ‎‎‎three old cows} (E)
\end{exe}

At the clausal level pure number agreement is expressed by means of the special plural suffix of the optative, \textit{-ar-te}, which is only used for plural addressees \refex{ex:‎‎‎May Allah leave you (plural) wellAGREE} (\refsec{sec:optative}), and through the cross-categorical suffixes -\textit{ce} (plural -te) and -\textit{il} in those periphrastic verb forms, which make use of the suffixes (experiential I, experiential II, obligative present). Singular agreement controllers require -\textit{ce} \refex{ex:I will go / have to go analytic2} or -\textit{il} \refex{ex:‎I gave birth to (my son) under a blanket@B}; plural agreement controllers require -\textit{te} \refex{ex:‎When all Chakhri people moved to the lowlands, we (also) moved2}, \refex{ex:Then you have to find them2}. This type of agreement follows ergative alignment. For one-place verbs and extended intransitive verbs the number agreement controller is the single argument in the absolutive \refex{ex:I will go / have to go analytic2}, \refex{ex:‎When all Chakhri people moved to the lowlands, we (also) moved2}; with transitive verbs and affective verbs the number agreement controller is the absolutive patient or stimulus \refex{ex:‎I gave birth to (my son) under a blanket@B}, \refex{ex:Then you have to find them2}. More examples are given in Section \refsec{ssec:Person agreement rules}.

\begin{exe}

	\ex	\label{ex:‎‎‎May Allah leave you (plural) wellAGREE}
	\gll	Allah-li	ʡaˁħ-le	d-at-arte!\\
		Allah\tsc{-erg}	good\tsc{-advz}	\tsc{1/2pl-}let\tsc{.pfv-opt.pl}\\
	\glt	\sqt{‎‎‎May Allah leave you (plural) well!}
	
	
	\ex	\label{ex:I will go / have to go analytic2}
	\gll	du	w-ax-an-ce=da\\
		\tsc{1sg}	\tsc{m-}go\tsc{.ipfv-ptcp-dd.sg=1}\\
	\glt	\sqt{I (masc.) will go\slash have to go.} (E)
	
	\ex	\label{ex:‎When all Chakhri people moved to the lowlands, we (also) moved2}
	\gll	hetːi	li<b>il=ra	čːuˁħrug	ka-b-eʁ-ib=qːel,	ka-d-eʁ-ib-te=da	\\
		those	all\tsc{<hpl>=add}	Chakhri.people	\tsc{down-hpl-}go\tsc{.pfv-pret=}when	\tsc{down-1/2pl-}go\tsc{.pfv-pret-dd.pl=1}\\
	\glt	\sqt{‎When all Chakhri people moved to the lowlands, we (also) moved.}
	
			\ex	\label{ex:‎I gave birth to (my son) under a blanket@B}
		\gll	du-l	julʁan-ni-gu-w	w-arq'-ib-il	ca-w\\
			\tsc{1sg-erg} blanket\tsc{-obl-sub-m}	\tsc{m-}do\tsc{.pfv-pret-dd.sg} 	\tsc{cop-m}\\
		\glt	\sqt{‎I gave birth to (my son) under a blanket.} [modified corpus example]
		
	\ex	{[The others are hiding.]}\\	\label{ex:Then you have to find them2}
	\gll	c'il	u-l	b-urkː-an-te	ca-b	hel-tːi\\
		then	\tsc{2sg-erg}	\tsc{hpl-}find\tsc{.ipfv-ptcp-dd.pl} 	\tsc{cop-hpl}	that\tsc{-pl}\\
	\glt	\sqt{Then you have to find them.} [modified corpus example]
\end{exe}

%%%%%%%%%%%%%%%%%%%%%%%%%%%%%%%%%%%%%%%%%%%%%%%%%%%%%%%%%%%%%%%%%%%%%%%%%%%%%%%%

\section{Combined Gender/number agreement}
\label{sec:Gender/number agreement}
\subsection{General remarks on gender/number agreement}
\label{General remarks on gender/number agreement} 
Combined gender/number agreement is a pervasive feature of East Caucasian languages including Sanzhi Dargwa. It is possible that within one clause three, four, or even more linguistic items agree with one and the same agreement controller. Sanzhi has three genders that have a transparent semantic basis: masculine, feminine, and neuter (\refsec{sec:noungender}). Agreement targets for gender/number agreement can be divided according to the same two agreement domains that have been mentioned for pure number agreement in the previous section, i.e. (i) the clausal domain (\refsec{sec:Simple clauses headed by verbs other than copulae}), and (ii) domain of the noun phrase (\refsec{sec:Noun phrases}). Within the domains the various targets can co-occur, depending on the morphosyntactic context (i.e. a noun in the essive case can but need not to be accompanied by an agreeing postposition). Example \refex{ex:‎Now he is crying in prisonAGREE} illustrates agreement within a clause. Four targets (lexical verb, copula, noun and postposition) agree with the agreement controller (a nominal with a masculine singular referent), which is not overtly expressed. The noun phrase in \refex{ex:All empty bottlesAGREE} contains two agreeing modifiers, a quantifier and an adjective.


\begin{description}
\item[Clausal domain]
\begin{itemize}[leftmargin=*]
    \item[]
	\item	most vowel-initial verbs (\refsec{sec:Gender agreementVerb})
	\item a few compound verbs with bound lexical stems (e.g. \textit{b-al} ‘together', \textit{b-at} ‘set free, let'), the spatial preverbs \textit{b-i-} `in, inside' and \textit{b-it}- ‘thither' (\refsec{sec:Preverbs})
	\item the standard copula (\refsec{sec:The copula}) as well as the locative copulas (\refsec{sec:Locational copulae}) (including the negative locative/existential copula \textit{b-akːu})
	\item the postpositions/adverbs \textit{b-i} ‘in', \textit{b-alli} ‘together', \textit{b-arxle} ‘directly, straight'
	\item	all items that can be inflected for the essive case, e.g. nouns, pronouns, spatial adverbs, postpositions, and all items that inflect for the directional case, i.e. mostly spatial adverbs (\refsec{sec:nouncase})
\end{itemize}
\end{description}

\begin{exe}
	\ex	\label{ex:‎Now he is crying in prisonAGREE}
	\gll	na	w-isː-ul	ca-w	tusnaq-le-w	w-i-w\\
		now	\tsc{m-}cry\tsc{-icvb}	\tsc{cop-m}	prison\tsc{-loc-m}	\tsc{m-}in\tsc{-m}\\
	\glt	\sqt{‎Now he is crying in prison.}
\end{exe}


\begin{description}
\item[Domain of the noun phrase]
\begin{itemize}[leftmargin=*]
    \item[]
    \item	a handful of adjectives (\refsec{sec:Other syntactic properties})
	\item	the quantifier \textit{li<b>il} ‘all' and group numerals (\refsec{sec:groupnumerals})
	\item	the derivational suffixes \tit{-či-b} and \textit{-azi-b}, which derive adjectives (\refsec{sec:Derivation of adjectives})
	\item	nouns, which function as modifiers in noun phrases and are inflected for the essive case (\refsec{sec:nouncase})
\end{itemize}
\end{description}


\begin{exe}
	\ex	\label{ex:All empty bottlesAGREE}
	\gll	li<d>il d-ac' šuš-ne\\
		all<\tsc{npl}> 	\tsc{npl}-empty	bottle\tsc{-pl}\\
	\glt	\sqt{all empty bottles} (E)
\end{exe}



Furthermore, a small number of nouns (e.g. \textit{b-ah} ‘owner, master') (\refsec{sec:noungender}) and reflexive pronouns in the absolutive (\refsec{sec:Reflexive pronouns}) and one reciprocal pronoun (\refsec{sec:Reciprocal pronouns}) contain gender exponents that express the gender of the referent.

The agreement affixes are given in \reftab{tab:Agreement affixes in Sanzhi}. (Almost) all forms can occur as prefixes, suffixes, and infixes.\footnote{There are only two agreement targets that have infixes, namely the quantifier \textit{li<b>il} `all' \refex{ex:All empty bottlesAGREE} and a variant of the standard copula \textit{ca<b>i}. The form \textit{ca<b>i} is used by a few speakers of Sanzhi in free variation with the much more common form \textit{ca-b}. The quantifier \textit{li-b-il} is diachronically complex and the gender marker is rather a suffix added to a stem \textit{li}- and followed by the referential attributive suffix -\textit{il}.} The only exception to this rule is the zero marking for masculine singular agreement, which is only possible in the prefixal position (see below for examples). Verbs (except for copulas) and adjectives have prefixes; the other agreement targets have suffixes or infixes. The agreement slots for prefixes, suffixes, and infixes are obligatorily filled for all targets that have them (i.e. all agreement targets with agreement slots always exhibit agreement).
%
\begin{table}
	\caption{Agreement affixes in Sanzhi}
	\label{tab:Agreement affixes in Sanzhi}
	\small
	\begin{tabularx}{0.46\textwidth}[]{%
		>{\raggedright\arraybackslash}X
		>{\centering\arraybackslash}p{24pt}
		>{\centering\arraybackslash}p{24pt}
		>{\centering\arraybackslash}p{24pt}}
		
		\lsptoprule
		{}			&	\tsc{sg}	 	&	\tsc{1/2pl}		&	\tsc{3pl}\\
		\midrule 
		masculine		&	\tit{w}\slash\O		&	\tit{d}			&	\tit{b}\\
		feminine		&	\tit{r}			&	\tit{d}			&	\tit{b}\\
		neuter		&	\tit{b}			&	\multicolumn{2}{c}{\tit{d}}\\
		\lspbottomrule
	\end{tabularx}
\end{table}

As \reftab{tab:Agreement affixes in Sanzhi} shows, there are fewer distinctions in the plural than in the singular, because masculine and feminine are united in human plural agreement. In addition, human plural is conditioned by person: first and second person plural agreement controllers are marked with \tit{d}, third person with \tit{b}. This phenomenon is also found in other Dargwa varieties, Archi, Ingush, and Chechen (see, e.g. \citealp{Chumakina.Kibort.Corbett2007} and \citealp[239\tnd251]{Corbett2012} for analyses of Archi) \refex{ex:We wanted to cry@14c}.

The prefix for masculine singular is \tit{w-}, but it is (optionally) deleted when it occurs between vowels or in initial position when followed by the vowels /i/ or /u/. Deletion of /w/ between two vowels leads to vowel lengthening when the two vowels have the same quality, e.g. \tit{a-w-ax-an=da} (\tsc{neg-m-}go\tsc{-ptcp=1}) > \tit{aːxanda} \sqt{I will not go} (vs. \tit{a-r-ax-an=da} for female speakers), or the vowel quality changes according to the standard sandhi rules. For instance, \tit{a-w-irχ-ud} (\tsc{neg-m-}be.able\tsc{.ipfv-1}) > \tit{a-irχud} > \tit{erχud} \sqt{I cannot} (vs. \tit{a-r-irχ-ud} for female speakers) (see \refsec{sec:Phonological and morphophonological alternations} for morphophonological rules). When occurring in initial position before \tit{i} the prefix \textit{w-} is optionally omitted, e.g. \tit{Ø-ik'-ud}\slash\tit{w-ik'-ud} (\tsc{m-}say\tsc{.ipfv-1}) vs. \tit{r-ik'-ud} (\tsc{f-}say\tsc{.ipfv-1}) \sqt{I say}. Before \tit{u} the deletion is obligatory, e.g. \tit{Ø-uq-un}\slash\tit{*w-uq-un} (\tsc{m-}go\tsc{.pfv-pret}) vs. \tit{r-uq-un} (\tsc{f-}go\tsc{.pfv-pret}) \sqt{I went}.

There are two agreement domains for gender agreement, the noun phrase and the clause, which follow two different rules. Within the noun phrase, modifiers agree with the head in gender and number independently of the case marking on the head \refex{ex:agreement number}- \refex{ex:I will be on the straight road agreement} (see \refsec{sec:Noun phrases} for the syntax of noun phrases).
%
\begin{exe}
	\ex	\label{ex:agreement number}
	\begin{xlist}
		\TabPositions{13em}
		\ex	\tit{Ø-uqna admi} \sqt{old person}	\tab	\tit{b-uqna adimte} \sqt{old people}	\label{ex:agreement number@A}
		\ex	\tit{r-uqna xːunul} \sqt{old woman}	\tab	\tit{b-uqna xːunre} \sqt{old women}	\label{ex:agreement number@B}
		\ex	\tit{b-uqna χːʷe} \sqt{old dog}		\tab	\tit{d-uqna χːude} \sqt{old dogs}		\label{ex:agreement number@C}
	\end{xlist}

	\ex	\label{ex:my old grandfather agreement}
	\gll	di-la	Ø-uqna	χatːaj\\
		\tsc{1sg-gen}	\tsc{m-}old	grandfather\\
	\glt	\sqt{my old grandfather}

	\ex	\label{ex:‎‎My old mother was alive agreement}
	\gll	r-uqna	aba	le-r=de	di-la\\
		\tsc{f-}old	mother	exist\tsc{-f=pst}	\tsc{1sg-gen}\\
	\glt	\sqt{‎‎My old mother was alive.} (i.e. existed).

	\ex	\label{ex:I will be on the straight road agreement}
	\gll	b-arx	xːun-ne	k-ercː-an=da\\
		\tsc{n-}straight	way\tsc{-loc}	down-stand\tsc{.ipfv.m-ptcp=1}\\
	\glt	\sqt{I will be on the straight road.} (i.e. I will not behave badly)
\end{exe}

Note that within the noun phrase as well as within the clause, gender agreement with a noun modified by a numeral other than \textit{ca} \sqt{one} is semantically based, i.e. it is plural, although the noun itself does not bear an overt plural suffix. See \refsec{ssec:Semantic agreement and other deviations} below for another example and \refsec{ssec:Lexical, phrasal, and clausal modifiers in noun phrases} for number marking and agreement within the noun phrase.

Within the clause, the agreement controller is most commonly the argument in the absolutive, though it is not necessarily overtly present in the clause. This rule applies independently of polarity, TAM features, and clause types, i.e. it is found with all finite and non-finite verb forms including various nominalized verb forms (participles, masdars). Examples \refexrange{ex:She disappeared like steam@11a}{ex:We did not die@11d} illustrate monovalent predicates agreeing with the S argument. 
%
\begin{exe}

		\ex	\label{ex:She disappeared like steam@11a}
		\gll	it	paˁħ-le	r-itaq-ib\\
			that	steam\tsc{-advz}	\tsc{f-}disappear\tsc{.pfv-pret}\\
		\glt	\sqt{She disappeared like steam.}

		\ex	\label{ex:They did not die@11b}
		\gll	hel-tːi	a-b-ebč'-ib\\
			that\tsc{-pl}	\tsc{neg-hpl-}die\tsc{.pfv-pret}\\
		\glt	\sqt{They (human) did not die.}

		\ex	\label{ex:All roads broke@11c}
		\gll	li<d>il=ra	ka-d-ič-ib	xːun-be\\
			all\tsc{<pl>=add}	\tsc{down-npl}-occur.\tsc{pfv-pret}	way\tsc{-pl}\\
		\glt	\sqt{All roads broke.}

		\ex	\label{ex:We did not die@11d}
		\gll	nušːa	a-d-ebč'-ib=da\\
			\tsc{1pl}	\tsc{neg-1/2pl-}die\tsc{.pfv-pret=1}\\
		\glt	\sqt{We did not die.}
\end{exe}

In \refexrange{ex:S/he ate bread@13a}{ex:He gave pears to the boys@8b} bivalent predicates are presented. Example \refex{ex:S/he ate bread@13a} contains a canonical transitive predicate. The agreement on the verb is controlled by the P argument. Other predicates behaving the same as canonical transitive verbs with respect to agreement are affective predicates with experiencers arguments in the dative or ergative and stimulus arguments in the absolutive case \refex{ex:Well, I must know her} (see also \refsec{sec:Bivalent affective verbs} for more information on bivalent affective predicates). Sentence \refex{ex:She did not touch me@8a} illustrates an extended intransitive predicate whose argument in the absolutive is the agreement controller. In \refex{ex:He gave pears to the boys@8b} a ditransitive predicate is given that agrees with its T argument.
%
\begin{exe}
		\ex	\label{ex:S/he ate bread@13a}
		\gll	it-i-l	t'ult'	b-erkʷ-un\\
			that\tsc{-obl-erg}	bread	\tsc{n-}eat\tsc{.pfv-pret}\\
		\glt	\sqt{S/he ate bread.}

		\ex	\label{ex:Well, I must know her}
		\gll	na	it	du-l	r-alχ-an=q'al\\	
			now	that	\tsc{1sg-erg}	\tsc{f-}know\tsc{.ipfv-ptcp=mod}\\
		\glt	\sqt{Well, I must know her.}

		\ex	\label{ex:She did not touch me@8a}
		\gll	it	dam	kːač a-r-ič-ib\\	
			that	\tsc{1sg.dat}	touch \tsc{neg-f-}occur\tsc{.pfv-pret}\\
		\glt	\sqt{She did not touch me.}

		\ex	\label{ex:He gave pears to the boys@8b}
		\gll	it-i-l	quˁr-be=ra	d-ičː-ib	hel-tːi	durħ-n-aˁ-j\\
			that\tsc{-obl-erg}	pear\tsc{-pl=add}	\tsc{npl-}give\tsc{.pfv-pret}	\tsc{that-pl}	boy\tsc{-pl-obl-dat}\\
		\glt	\sqt{He gave pears to the boys.}
\end{exe}

In the antipassive construction, agreement is also controlled by the absolutive, which is now the agent \refex{ex:‎‎She is telling stories} (see \refsec{sec:Antipassive} for a detailed account).
%
\begin{exe}
	\ex	\label{ex:‎‎She is telling stories}
	\gll	it	χabur-t-a-l	r-ux-ul	ca-r\\
		that	story\tsc{-pl-obl-erg}	\tsc{f-}tell\tsc{.ipfv-icvb}	\tsc{cop-f}\\
	\glt	\sqt{‎‎She is telling stories.}
\end{exe}

Gender agreement with other than absolutive arguments is also attested. It is not very common, but corpus examples can be found. The non-absolutive arguments controlling the agreement are either ergative agents or experiencers in the dative. This phenomenon is discussed in detail in \refsec{ssec:Gender agreement with arguments in other than the absolutive case}. 

In complement constructions in which the complement clause functions as the absolutive argument of the matrix predicate the agreement affix \tit{b} is used in case of local agreement of the matrix predicate with the complement clause \refex{ex:(S/he/they) knew that there were our fish in our river@14a}. This can be interpreted as default agreement, because in Sanzhi predicates that do not govern any argument in the absolutive case and therefore do not have a syntactic agreement controller predominantly take the agreement marker \textit{b} (see below). Alternatively, we can say that the matrix verb agrees with the nominalized complement clause. Nominalization of any linguistic items results in nominals belonging to the neuter gender and therefore the matrix predicate must take \textit{b-}. 
%
\begin{exe}
		\ex	\label{ex:(S/he/they) knew that there were our fish in our river@14a}
		\gll	[nišːa-la	baliqː-e	le-d-ni	nišːa-la	erk'ʷ-li-cːe-d]	b-alχ-ul=de\\
			\tsc{1pl-gen}	fish\tsc{-pl}	exist\tsc{-npl-msd}	\tsc{1pl.obl-gen}	river\tsc{-obl-in-npl}	\tsc{n-}know\tsc{-icvb=pst}\\
		\glt	\sqt{(S/he/they) knew that there were our fish in our river.}
\end{exe}

Sanzhi Dargwa, like many other Dagestanian languages, also has the option for long-distance agreement where the gender/number agreement on the matrix verb is controlled by the absolutive argument of the complement clause. Long-distance agreement occurs rather infrequently in the Sanzhi corpus because there are only few agreeing matrix predicates and the respective complement constructions are not very often used. Therefore, the precise rules specifying its distribution still need to be studied. In \refex{ex:We wanted to cry@14c} the complement clause contains an intransitive predicate whose single argument is suppressed due to co-reference with the overt argument of the main clause. Nevertheless, it controls agreement on both predicates. More examples of long-distance agreement and references to the literature on East Caucasian languages can be found in \refsec{sec:The syntactic properties of complement clauses}.

%
\begin{exe}
		\ex	\label{ex:We wanted to cry@14c}
		\gll	nišːij	d-ikː-ul=de	[d-isː-ij]\\
			\tsc{1pl.dat}	\tsc{1/2pl-}want\tsc{.ipfv-icvb=pst}	\tsc{1/2pl-}cry\tsc{-inf}\\
		\glt	\sqt{We wanted to cry.} (E)
\end{exe}

If the clause does not contain an agreement controller because it is lacking an argument in the absolutive, then mostly the default affix \tit{b} is used:
%
\begin{exe}
	\ex	\label{ex:I miss you@7}
	\gll	dam	a-sa-r	b-et'-ib ca-b\\
		\tsc{1sg.dat}	\tsc{2sg-ante-abl}	\tsc{n-}long.for\tsc{.pfv-pret} \tsc{cop-n}\\
	\glt	\sqt{I miss you.} (E)

	\ex	\label{ex:You are cold@3a}
	\gll	at 	b-uχːar(-re)	ca-b\\
		\tsc{2sg.dat}	\tsc{n-}cold\tsc{(-advz)}	\tsc{cop-n}\\
	\glt	\sqt{You are cold.} (E)
\end{exe}

The same happens with the verb \textit{b-us-} denoting precipitation phenomena (e.g. rain, snow). This predicate governs one single argument marked with the ergative \refex{ex:It is raining@4}. The identical phenomenon is observed in the neighboring Icari Dargwa variety \citep[155]{Sumbatova.Mutalov2003}, but apparently not in Standard Dargwa.
%
\begin{exe}
	\ex	\label{ex:It is raining@4}
	\gll	marka-l	b-us-ul	ca-b\\
		rain\tsc{-erg}	\tsc{n-}rain\tsc{.ipfv-icvb} \tsc{cop-n}\\
	\glt	\sqt{It is raining.}
\end{exe}

Occasionally, not \tit{b-} but \tit{d-} is used as default agreement exponent. This mainly concerns some compound verbal predicates that consist of a bound stem that is not a nominal, and a light verb (\refsec{sec:Bivalent verbs with frozen objects}). For instance, in \refex{ex:I am angry@b} the verb is a compound consisting of the verbal part \tit{b-ulq-} with the meaning \sqt{direct} and a first part \tit{simi}, and the agreement is always \tit{d-}. Another example is the phrase with which one wishes a good day \refex{ex:A good day to you}.
%
\begin{exe}
	\ex	\label{ex:I am angry@b}
	\gll	dam	simi	d-ulq-u\\
		\tsc{1sg.dat}	anger	\tsc{npl-}direct\tsc{.ipfv-prs.3}\\
	\glt	\sqt{I am angry.}
\end{exe}

In addition to verbs also items bearing the essive case and the directional are agreement targets within the clausal domain. All essive cases in Sanzhi Dargwa as well as in other Dargwa varieties are expressed by adding a gender/number suffix to one of the spatial suffixes (\refsec{sec:nouncase}). Thus, in the verbless sentence in \refex{ex:The people were hungry, it was during the years of war@12}, the noun bearing the spatial case suffix \tit{-cːe} in the second clause agrees with the omitted absolutive argument that is identical to the argument in the preceding clause. Both clauses represent copula constructions with an adverbial predicate (first clause) and a nominal predicate (second clause) respectively. Similarly, \refex{ex:‎Back (reversing) from there we went down to the village} shows two spatial adverbs agreeing with the absent absolutive argument.
%
\begin{exe}
	\ex	\label{ex:The people were hungry, it was during the years of war@12}
	\gll	χalq'	kːuš-le=de,	daˁw-i-la	dus-m-a-cːe-b=de\\
		people	hungry\tsc{-advz=pst}	war\tsc{-obl-gen}	year\tsc{-pl-obl-in-hpl=pst}\\
	\glt	\sqt{The people were hungry, during the years of war.}

	\ex	\label{ex:‎Back (reversing) from there we went down to the village}
	\gll	hila-d-a	hekka	gu-d-a	ag-ur=da	šːa\\
		behind\tsc{-1/2pl-dir}	from.there	down\tsc{-1/2pl-dir}	go\tsc{.pfv-pret=1}	village\tsc{.loc}\\
	\glt	\sqt{‎Back (reversing) from there we went down to the village.}
\end{exe}

However, it is possible and occasionally attested in the corpus that gender markers of spatial adverbials show default agreement rather than agreement controlled by the absolutive. For instance, in \refex{ex:‎‎‎Go like that (i.e. in that direction)} the omitted absolutive argument is female, as can be seen from the agreement on the verb, but the directional adverbial exhibits default agreement. Similarly, in \refex{ex:This person needs to be there in the morning at 8 o'clockAA} the agreement controller is the masculine singular noun phrase at the end of the clause, but the adverb in clause-initial position has the neuter  singular suffix. 
%
\begin{exe}
	\ex	\label{ex:‎‎‎Go like that (i.e. in that direction)}
	\gll	hel-itːe-b-a	r-uˁq'-aˁn!\\
		that\tsc{-advz-n-dir}	\tsc{f-}go\tsc{-imp}\\
	\glt	\sqt{‎‎‎Go like that (i.e. in that direction)!} (E)
	
			\ex	\label{ex:This person needs to be there in the morning at 8 o'clockAA}
		\gll	ixtːu-b	čːaˁʡaˁl-li-j	sːaˁʡaˁt	kːaʔal-li-j	w-iχʷ-ij	ʡaˁʁuni-l	ca-w	hel	admi \\
			there.\textsc{up}-\textsc{n}	morning-\textsc{obl}-\textsc{dat} hour	eight-\textsc{obl}-\textsc{dat}	\textsc{m}-be.\textsc{pfv}-\textsc{inf} needed-\textsc{advz}	\textsc{cop-m} that person\\
		\glt	\sqt{That person needs to be there in the morning at 8 o'clock.}	
		
\end{exe}

Another agreement target is the concessive converb of \tit{b-iχʷ-} (\tsc{pfv}) \sqt{be, become, be able, can}, which is used in concessive clauses and, when the verb follows interrogative pronouns, for the formation of free-choice indefinite pronouns (see \refsec{ssec:Free-choice indefinite pronouns}). Since \tit{b-iχʷ-} is a verb with an agreement slot, the indefinite pronouns can, in principle, agree. Mostly they have default agreement, but they can also deviate from this pattern, for instance by being controlled by the absolutive argument. Thus in \refex{ex:(he was) hiding at the house of whomever}, \tit{biχʷarra} could be replaced by \tit{Ø-iχʷarra} which would represent agreement controlled by the omitted absolutive subject. At the present moment I do not have enough data to explain this variation.  
%
\begin{exe}
	\ex	\label{ex:(he was) hiding at the house of whomever}
	\gll	daˁʡaˁna	w-irx-ul	hi-la-k'a	b-iχʷ-ar=ra	qili\\
		secret	\tsc{m-}become\tsc{.ipfv-icvb}	who\tsc{.obl-gen-indef}	\tsc{n-}be\tsc{.pfv-cond.3=add}	home\\
	\glt	\sqt{(He was) hiding at the house of whomever.}
\end{exe}


% - - - - - - - - - - - - - - - - - - - - - - - - - - - - - - - - - - - - - - - - - - - - - - - - - - - - - - - - - - - - - - - - - - - - - - - - - - - - - - - - - - - - - - - - - - - - - - - - - - - - - - - - - - - - - - - - - - - - - - - - - - %

\subsection{Semantic agreement and other peculiarities}
\label{ssec:Semantic agreement and other deviations}

Semantic agreement refers to cases in which the morphosyntactic feature values of the agreement target do not match the formal features of the controller. Instead, the agreement matches some semantic properties of the controller. Many instances of semantic agreement are number or gender mismatches. In Sanzhi Dargwa, such examples are found with gender/number agreement exponents on verbs where the agreement controller is an NP containing numerical quantifiers. A noun modified by a numeral normally does not take a plural suffix, but it requires plural agreement on the verb \refex{ex:I have two wives@15}, \refex{ex:(He) saw three trees}. 
%
\begin{exe}
	\ex	\label{ex:I have two wives@15}
	\gll	di-la	k'ʷel	xːunul	le-b\\
		\tsc{1sg-gen}	two	woman	exist\tsc{-hpl}\\
	\glt	\sqt{I have two wives.}

	\ex	\label{ex:(He) saw three trees}
	\gll	či-d-až-ib	ca-d	ʡaˁbal	kːalkːi\\
		\tsc{spr-npl-}see\tsc{.pfv-pret}	\tsc{cop-npl}	three	tree\\
	\glt	\sqt{(He) saw three trees.}
\end{exe}

A further example of semantic agreement occasionally occurs in fairy tales in which the acting personas are animals. In such cases mostly the inherent gender of the nouns is used, i.e. neuter, but sometimes the referents are treated as if they were human beings and thus masculine agreement affixes appear. In \refex{ex:‎‎‎(The wolf) said, I did not go to school.}, the verb of speech has the neuter singular prefix in accordance with the natural gender of the referent, a wolf. But the verb in the quote shows masculine singular agreement and thus the referent has been humanized.
%
\begin{exe}
	\ex	\label{ex:‎‎‎(The wolf) said, I did not go to school.}
	\gll	``du'',	b-ik'-ul	ca-b	``uškul-le	w-aš-ib-il	akːʷ-a-di	du''\\
		\tsc{1sg}	\tsc{n-}say\tsc{.ipfv-icvb}	\tsc{cop-n}	school\tsc{-loc}	\tsc{m-}go\tsc{-pret-ref}	\tsc{cop.neg-hab.pst-1}	\tsc{1sg}\\
	\glt	\sqt{‎‎‎(The wolf) said, ``I did not go to school.''}
\end{exe}

Agreement with conjoined noun phrases can partially also be treated as semantic agreement (\refsec{ssec:Gender agreement resolution}).

Another deviation that cannot readily be explained as semantic agreement is represented by a special construction for words denoting time spans such as \tit{dus} \sqt{year}, \tit{bac} \sqt{month}, \tit{saˁʡaˁt} \sqt{hour}, or \tit{minut'} \sqt{minute}. These words belong to the neuter gender \refex{ex:‎One year finished (e.g. of my studies)} and all of them except for \tit{bac} have a plural form. However, when they are used to express periods of time with the verb \tit{b-ič-} (\tsc{pfv}) \sqt{occur, be}, then agreement is neuter plural \refex{ex:‎One year passed by since I came here}.
%
\begin{exe}
	\ex	\label{ex:‎One year finished (e.g. of my studies)}
	\gll	ca	dus	či-r-b-it-ag-ur\\
		one	year	\tsc{spr-abl-n-thither}-go\tsc{.pfv-pret}\\
	\glt	\sqt{‎One year finished.} (e.g. of my studies) (E)

	\ex	\label{ex:‎One year passed by since I came here}
	\gll	ca	dus	d-ič-ib	ca-d,	du	sa-jʁ-ib-la\\
		one	year	\tsc{npl-}occur\tsc{.pfv-pret}	\tsc{cop-npl}	\tsc{1sg}	\tsc{hither}-come\tsc{.m.pfv-pret-post}\\
	\glt	\sqt{‎One year passed by since I came here.} (E)
\end{exe}


% - - - - - - - - - - - - - - - - - - - - - - - - - - - - - - - - - - - - - - - - - - - - - - - - - - - - - - - - - - - - - - - - - - - - - - - - - - - - - - - - - - - - - - - - - - - - - - - - - - - - - - - - - - - - - - - - - - - - - - - - - - %

\subsection{Gender/number agreement with conjoined noun phrases}
\label{ssec:Gender agreement resolution}

Agreement with conjoined noun phrases follows two strategies: either the conjoined noun phrase is treated like a noun marked for plural and thus controls plural agreement or there is agreement with the closest conjunct.

The first case can be treated as an instance of semantic agreement since the nouns are not morphologically marked for plural. The rules for this type of agreement with conjoined noun phrases are as follows: two nouns denoting human beings control human plural agreement \refex{ex:‎Then the mother and the father came and looked}, and two nouns denoting animals or objects control neuter plural agreement \refex{ex:‎‎‎The owner took the horse and the donkey and put the load on them}.\footnote{The same rules apply if one or both of the conjuncts are plural nouns.}
%
\begin{exe}
	\ex	\label{ex:‎Then the mother and the father came and looked}
	\gll	heba	er	b-ik'-ul	ca-b	sa-b-eʁ-ib-le	atːa=ra	aba=ra\\
		then	look	\tsc{hpl-}look.at\tsc{.ipfv-icvb}	\tsc{cop-hpl}	\tsc{hither-hpl-}go\tsc{.pfv-pret-cvb}	father\tsc{=add}	mother\tsc{=add}\\
	\glt	\sqt{‎Then the mother and the father came and looked.}

	\ex	\label{ex:‎‎‎The owner took the horse and the donkey and put the load on them}
	\gll	χazajn-ni	d-erč-ib	ca-d	urči=ra	amχa=ra	či-ka-d-išː-ib-le	deχ=ra\\
		owner\tsc{-erg}	\tsc{npl-}lead\tsc{.pfv-pret}	\tsc{cop-npl}	horse\tsc{=add}	donkey\tsc{=add}	\tsc{spr-down}\tsc{-npl}-put.\tsc{pfv-pret-cvb}	load[\tsc{npl}]\tsc{=add}\\
	\glt	\sqt{‎‎‎The owner took the horse and the donkey and put the load on them.}
\end{exe}

When the first and second person singular or plural pronouns are conjoined with nouns the agreement for first and second person plural is used, i.e. \tit{d} \refex{ex:There was not time for you and me to chat.}, \refex{ex:‎Then we invited the neighbors and we sat together}.
%
\begin{exe}
\ex \label{ex:There was not time for you and me to chat.}
\gll u=ra du=ra ʁaj d-uq-ij ja zamana a-b-ič-ib\\
\tsc{2sg=add}	\tsc{1sg=add}	word	\tsc{1/2pl}-go.\tsc{pfv-inf}		or	time	\tsc{neg-n}-occur.\tsc{pfv-pret} \\
\glt	\sqt{There was not time for you and me to chat.} (E)

	\ex	\label{ex:‎Then we invited the neighbors and we sat together}
	\gll	c'il	sa-č-ib-le,	nišːa-la	zunra=ra	nušːa=ra	qːuʁa-l	ka-d-iž-ib-le,	\ldots\\
		then	\tsc{hither}-lead\tsc{.pfv-pret-cvb}	\tsc{1pl-gen}	neighbor\tsc{=add}	\tsc{1pl=add}	beautiful\tsc{-advz}	\tsc{down}\tsc{-1/2pl-}be\tsc{.pfv-pret-cvb}\\
	\glt	\sqt{‎Then after having invited (them), our neighbors and we sat together, \ldots}
\end{exe}

When a noun denoting a human being occurs in conjunction with a noun denoting an animal or object the agreement is human plural:
%
\begin{exe}
	\ex	\label{ex:‎(The Sanzhi man shot) and killed the horsefly and he killed the Xuduc man, and (they both) died}
	\gll	zija=ra	kax-ub	ca-b,	χudec'an=ra	kax-ub	ca-w,	b-ebč'-ib	ca-b\\
		horsefly\tsc{=add}	kill\tsc{.pfv-pret}	\tsc{cop-n}	Khuduc.person\tsc{=add}	kill\tsc{.pfv-pret} \tsc{cop-m}	\tsc{hpl-}die\tsc{.pfv-pret}	\tsc{cop-hpl}\\
	\glt	\sqt{‎(The Sanzhi man shot) and killed the horsefly and he killed the Khuduc man, and (they both) died.}

	\ex	\label{ex:(The boy and the dog) are looking}
	\gll	er	b-ik'-ul	ca-b\\
		look	\tsc{hpl-}look.at\tsc{.ipfv-icvb}	\tsc{cop-hpl}\\
	\glt	\sqt{(The boy and the dog) are looking.}
\end{exe}

Plural agreement is sometimes even found in comitative constructions. There are two ways of expressing comitative roles. One is via the use of the comitative case \refex{ex:Now he is fighting with this wife}, \refex{ex:‎The good boy with this beautiful wife lived until now} and the other is via the use of the reflexive pronoun \refex{ex:He is already walking with this son}, \refex{ex:If he would sit together with his wife well like this, this would be good} (\refsec{sec:Comitative constructions}). In both constructions normally the absolutive argument controls the agreement as the following two sentences show:
%
\begin{exe}
	\ex	\label{ex:Now he is fighting with this wife}
	\gll	hana	hež	xːunul-li-cːella	w-irħ-uˁl	ca-w\\
		now	this	woman\tsc{-obl-comit}	\tsc{m-}fight\tsc{.ipfv-icvb}	\tsc{cop-m}\\
	\glt	\sqt{Now he is fighting with this wife.}

	\ex	\label{ex:He is already walking with this son}
	\gll	uže	heχ	durħuˁ=ra	ca-w=ra	arg-ul	ca-w=nu\\
		already	\tsc{dem.down}	boy\tsc{=add}	\tsc{refl-m=add}	go\tsc{.ipfv-icvb}	\tsc{cop-m=}but\\
	\glt	\sqt{He is already walking with this son.}
\end{exe}

However, there are very few examples in which the comitative phrase is treated as a plural noun phrase and therefore controls plural agreement. Example \refex{ex:‎The good boy with this beautiful wife lived until now} illustrates this for the comitative case, and example \refex{ex:If he would sit together with his wife well like this, this would be good} shows the comitative construction with a reflexive pronoun. In the first example, the human plural agreement could be replaced with masculine singular \textit{w}-. In the second example, the agreement is first/second person plural \textit{d}- since the author of the quote is referring to himself and his wife, thus the sentence is a quote with an omitted matrix clause.
%
\begin{exe}
	\ex	\label{ex:‎The good boy with this beautiful wife lived until now}
	\gll	a	hel	ʡaˁħ	durħuˁ	cin-na	qːuʁa	xːunul-li-cːella er b-iχ-ub ca-b hana-li-j=sat=ra\\
		and	that	good	boy	\tsc{refl.sg-gen}	beautiful	woman\tsc{-obl-comit}	life	\tsc{hpl-}be.\tsc{pfv-pret}	\tsc{cop-hpl}	now\tsc{-obl-dat=}as.much\tsc{=add}\\
	\glt	\sqt{‎The good boy with this beautiful wife lived until now.}

	\ex	{[he probably thinks]}\\	\label{ex:If he would sit together with his wife well like this, this would be good}
	\gll	heχ-itːe xːunul=ra	ca-w=ra	qːuʁa-l	ka-d-iž-ib	d-iχ-utːel,	ʡaˁħ-le	b-určː-i\\
		\tsc{dem.down}\tsc{-advz}	woman\tsc{=add}	\tsc{refl-m=add}	beautiful\tsc{-advz}	\tsc{down-1/2pl-}be\tsc{.pfv-pret}	\tsc{1/2pl-}be\tsc{.pfv-cond.pst}	good\tsc{-advz}	\tsc{n-}find\tsc{.ipfv-hab.pst.3}\\
	\glt	\sqt{If he would sit together with his wife well like this, this would be good.}
\end{exe}

The alternative to plural agreement in conjoined noun phrases is called \textit{closest conjunct agreement}. Closest conjunct agreement has been demonstrated to exist in a number of East Caucasian languages (see e.g. \citealp{Polinsky.etal2009} on Tsez, and \citealp{Chumakina2014} on Archi). It is possible with conjoined noun phrases that follow or precede the verb. In each case, the member of the conjunction that happens to occur closer to the agreement target controls the agreement instead of agreement with the noun phrase as a whole:
%
\begin{exe}
	\ex	\label{ex:‎The boy had a dog and a frog}
	\gll	duˁrħuˁ-la	b-už-ib	ca-b	χːʷe=ra	ʡaˁt'a=ra\\
		boy\tsc{-gen}	\tsc{n-}stay\tsc{-pret}	\tsc{cop-n}	dog\tsc{=add}	frog\tsc{=add}\\
	\glt	\sqt{‎The boy had a dog and a frog.}
\end{exe}

\citet{Polinsky.etal2009} show that in Tsez, agreement with the closest conjunct is only possible when the agreement controller is adjacent to the verb. This is not the case in Sanzhi. Example \refex{ex:‎‎And there I remembered my family and everything} shows that the noun \tit{kulpat} \sqt{family} controls the agreement on the preceding verb (\tsc{hpl}) even though the personal pronoun intervenes.
%
\begin{exe}
	\ex	\label{ex:‎‎And there I remembered my family and everything}
	\gll	heχ-tːu-b	han	b-ič-ib	dam	kulpat=ra,	li<d>il	cik'al	hel-tːi=ra \ldots\\
		\tsc{dem.down}\tsc{-loc-hpl}	remember	\tsc{hpl-}occur\tsc{.pfv-pret}	\tsc{1sg.dat}	family\tsc{=add}	all\tsc{<npl>}	something	that\tsc{-pl=add}\\
	\glt	\sqt{‎‎And there I remembered my family and everything, \ldots}
\end{exe}

In the following two examples, the agreement affix \tit{b} can either be interpreted as neuter and thus as instantiating closest conjunct agreement or as human plural agreement, i.e. semantic agreement with a noun phrase that is treated as a noun in the plural \refex{ex:‎The deer on its horns threw the boy and the dog into the water}, \refex{ex:‎The other shot at the forehead and killed the horsefly and the man}. More specifically, in example \refex{ex:‎The deer on its horns threw the boy and the dog into the water}, the verb shows closest conjunction agreement with the following noun \tit{duˁrħuˁ} \sqt{boy}, and the agreement of the clause-final spatial adverb \tit{b-i} \sqt{into} is ambiguous. Similarly, in \refex{ex:‎The other shot at the forehead and killed the horsefly and the man}, the agreement suffix of the copula \tit{ca-b} is also ambiguous and both noun phrases are equally close to the verb in terms of linear adjacency.
%
\begin{exe}
	\ex	\label{ex:‎The deer on its horns threw the boy and the dog into the water}
	\gll	il	alen-ni	qi-m-a-cːe-r	lak'	w-arq'-ib	ca-w	duˁrħuˁ=ra	χːʷe=ra	hin-ni-cːe	b-i\\
		that	deer\tsc{-erg}	horn\tsc{-pl-obl-in}-\tsc{abl}	throw	\tsc{m-}do\tsc{.pfv-pret}	\tsc{cop-m}	boy\tsc{=add}	dog\tsc{=add}	water\tsc{-obl-}in	\tsc{n/hpl}-in\\
	\glt	\sqt{‎The deer threw the boy and the dog into the water on its horns.}

	\ex	\label{ex:‎The other shot at the forehead and killed the horsefly and the man}
	\gll	zija=ra	kax-ub	ca-b	il	admi=ra\\
		horsefly\tsc{=add}	kill\tsc{.pfv-pret}	\tsc{cop-n/hpl}	that	person\tsc{=add}\\
	\glt	\sqt{(He) killed the horsefly and the man.}
\end{exe}


% - - - - - - - - - - - - - - - - - - - - - - - - - - - - - - - - - - - - - - - - - - - - - - - - - - - - - - - - - - - - - - - - - - - - - - - - - - - - - - - - - - - - - - - - - - - - - - - - - - - - - - - - - - - - - - - - - - - - - - - - - - %

\subsection{Gender agreement with arguments in other than the absolutive case (\textit{Deviant agreement})}
\label{ssec:Gender agreement with arguments in other than the absolutive case}

Surprising for East Caucasian languages is the fact that, in Sanzhi Dargwa, under certain circumstances the agreement in a simple clause can be controlled by arguments not in the absolutive case, but in the ergative or dative. These arguments can be present or absent from the clause. In the following, I will refer to this phenomenon as \textit{deviant gender agreement} or simply \textit{deviant agreement}. The agreement targets for which agreement with the ergative or dative is attested are the standard copula \xxref{ex:when he came home, he was talking.}{ex:She put it on HIS nose} and the existential/locational copulas \refex{ex:telling a storyLOCCOP} when they are used as auxiliaries in periphrastic verb forms and also the exponents of the essive case \refex{ex:In the sovkhoz I worked for 40 years.}. It is mostly found in clauses with a number of analytic verb forms such as the compound present \refex{ex:when he came home, he was talking.} or the resultative \refex{ex:She put it on HIS nose}. 

In my Sanzhi corpus agreement with non-absolutive arguments is not particularly frequent, but there are a few clear examples. The majority contains verbs of speech or cognition, in particular \tit{b-urs-} \sqt{\tsc{n}-tell} \xxref{ex:when he came home, he was talking.}{ex:‎‎He is telling about what he experienced}, but also a few other verbs \refex{ex:She put it on HIS nose}. In all examples \xxref{ex:when he came home, he was talking.}{ex:She put it on HIS nose} the standard copula \textit{ca-b} has an agreement suffix that differs from the agreement prefix of the lexical verb with which the copula forms an analytic verb form.
%
\begin{exe}
	\ex	\label{ex:when he came home, he was talking.}
	\gll	il	sa-sa-jʁ-ib=qːel, χabar	b-urs-ul	ca-w	il-i-l\\
		that	\tsc{ante-hither}-come\tsc{.m.pfv-pret=}when	story	\tsc{n-}tell\tsc{-icvb}	\tsc{cop-m}	that\tsc{-obl-erg}\\
	\glt	\sqt{When he came home, he was talking (telling stories).}

	\ex	\label{ex:‎‎She is telling stories, she says to me}
	\gll	it-i-l	di-cːe	d-urs-ul	ca-r\\
		that\tsc{-obl-erg}	\tsc{1sg-in}	\tsc{npl-}tell\tsc{-icvb}	\tsc{cop-f}\\
	\glt	\sqt{‎‎She tells (stories) to me.}

	\ex	\label{ex:‎‎He is telling about what he experienced}
	\gll	b-urs-ul	ca-w	heχ-i-l	cin-i-j	či-d-ič-ib-t-a-lla	qari=či-d\\
		\tsc{n-}tell\tsc{-icvb}	\tsc{cop-m}	\tsc{dem.down}\tsc{-obl-erg}	\tsc{refl.sg-obl-dat}	\tsc{spr-npl-}occur\tsc{.pfv-pret-dd.pl-obl-gen}	up=on\tsc{-npl}\\
	\glt	\sqt{‎‎He is telling about what he experienced.}

	\ex	{[The wife came and says, Come home!]}\\	\label{ex:She put it on HIS nose}
	\gll	heχ	b-ič-aq-ib	ca-w	qːuˁnq-li-cːe\\
		\tsc{dem.down}	\tsc{n-}occur\tsc{.pfv-caus-pret}	\tsc{cop-m}	nose\tsc{-obl-in}\\
	\glt	\sqt{(He) put it on her nose (i.e. he hit her nose).} 
\end{exe}

There are also a number of examples with dative experiencers that control gender agreement \xxref{ex:Maybe these, he wants to take}{ex:bring those years back now}.
%
\begin{exe}
	\ex	\label{ex:Maybe these, he wants to take} [‎The boy is looking at this, right?]\\
	\gll	ce=jal	il-tːi;	h-asː-ij	b-ikː-ul	ca-w	il-i-j\\
		what\tsc{=indq}	that\tsc{-pl}	\tsc{up}-take\tsc{.pfv-inf}	\tsc{n-}want\tsc{.ipfv-icvb}	\tsc{cop-m}	that\tsc{-obl-dat}	\\
	\glt	\sqt{Maybe these; he wants to take (it).}

	\ex	\label{ex:wants to give him knowledge}
	\gll	ʡaˁq'lu	b-ikː-ar-aj	b-ikː-ul	ca-w\\
		mind	\tsc{n-}give\tsc{.pfv-prs-subj.3}		\tsc{n-}want\tsc{.ipfv-icvb}	\tsc{cop-m}\\
	\glt	\sqt{‎‎‎(He) wants to give him knowledge.}

	\ex	\label{ex:bring those years back now}
	\gll	han	b-irk-ul	ca-w	heχ-i-j\\
		seem	\tsc{n-}occur\tsc{.ipfv-icvb}	\tsc{cop-m}	\tsc{dem.down}\tsc{-obl-dat}\\
	\glt	\sqt{He is thinking\slash imagining.} 
\end{exe}

There are very few corpus examples in which it is a spatial adjunct in the essive case that shows deviant agreement with an argument that is not marked for absolutive case. In example \refex{ex:In the sovkhoz I worked for 40 years.} the ergative first person pronoun is omitted, but it controls the masculine singular agreement on the clause-initial adverbial. More examples can be elicited; in \refex{ex:Isakadi is writing a book in the school.} the lexical verb does not have an agreement prefix, so the agreement mismatch is not immediately obvious, but the absolutive patient \textit{kiniga} `book' is neuter singular and would require the suffix \textit{-b} on the copula in case of non-deviant agreement.

\begin{exe}
	\ex	\label{ex:In the sovkhoz I worked for 40 years.}
	\gll	hež	sawχuz-li-cːe-w	aʁʷ-c'al	dus	ʡaˁči	b-arq'-ib=da\\
			this	sovkhoz-\tsc{obl-in-m}	four-\tsc{ten}	year	work	\tsc{n}-do.\tsc{pfv-pret}=1\\
		\glt	\sqt{In the sovkhoz (I, masc.) worked for 40 years.}
		
			\ex	\label{ex:Isakadi is writing a book in the school.}
	\gll	Isaq'adi-l	uškul-le-w	kiniga	luk'-unne	ca-w\\
		Isakadi-\tsc{erg}	school-\tsc{loc-m}	book	write.\tsc{ipfv-icvb}		\tsc{cop-m}\\
	\glt	\sqt{Isakadi is writing a book in the school.} (i.e. sitting in the school). (E)
\end{exe}

Deviant agreement with the existential copulas can be elicited:

\begin{exe}
	\ex	\label{ex:telling a storyLOCCOP}
	\gll	χabar	b-urs-ul	le-w\slash		te-w\slash		χe-w\slash		k'e-w	il-i-l\\
		story	\tsc{n}-tell-\tsc{icvb}	exist-\tsc{m}\slash 	exist.\tsc{away-m}\slash 	exist.\tsc{down-m}\slash	exist.\tsc{up-m}	that-\tsc{obl-erg}\\
	\glt	\sqt{He is telling a story.} (E)
\end{exe}


Deviant agreement never occurs with agreement exponents that belong to the lexical part of the predicate (lexical root, preverbs), but only with copula-auxiliaries and clausal adjuncts. Furthermore, the controller is in the ergative or dative and functions as a agent or experiencer argument of the predicate. It cannot be in any other case. Ergative agents and dative experiencer arguments of transitive and affective verbs share many subject properties with absolutive arguments of intransitive verbs \citep{Forker2017, Forker2019}. By contrast, arguments that do not function as agents or experiencers and are marked by other cases lack subject properties and cannot function as agreement controllers.

Deviant agreement is unusual for East Caucasian languages, but has been documented for a number of Dargwa varieties, most notably Akusha (Standard) Dargwa \citep{vandenBerg1999, GanenkovForthcoming}, Tanti Dargwa \citep[450\tnd493]{Sumbatova.Lander2014} and Shiri Dargwa \citep{Belyaev2016, Belyaev2017a, Belyaev2017b}. The different authors have put forward various explanations and hypotheses concerning the syntactic and semanto-pragmatic properties of the construction. According to all authors, gender agreement with the ergative argument (but also with the absolutive or dative) is conditioned by information structure. In her account, \citet{vandenBerg1999} states that deviant agreement with ergative agents does not require any specific pragmatic conditions whereas agreement with patients in the absolutive highlights them. She further claims that absolutive patients controlling agreement are topical (\textit{themes} in her terminology). \citet{Sumbatova.Lander2014} refine this analysis and claim that topical arguments independently of their case marking control gender agreement. \citet{Sumbatova2010} and \citet{Sumbatova.Lander2014} write that deviant agreement with ergative agents is frequent in Tanti Dargwa narratives. They further show that deviant agreement can also occur in cleft constructions that express constituent focus.




When discussing sentences with ergative agreement with Sanzhi speakers and eliciting new examples, an effect on the information structure is noticeable. Absolutive agreement is always possible, so it is the deviation from this pattern that requires an explanation. Absolutive agreement is preferred in answers to constituent questions regarding the agent or the patient that have narrow focus \refex{ex:Aminat is tearing out a carrot from under the earth@16a}. By contrast, ergative agreement is readily available when the question is, for example, about the place in which the agent is located \refex{ex:Aminat is tearing out a carrot from under the earth@16b}.
%
\begin{exe}
	\ex	\label{ex:Aminat and the carrots@16}
	\begin{xlist}
		\ex	{[Who is tearing out the carrots?]}\\	\label{ex:Aminat is tearing out a carrot from under the earth@16a}
		\gll	Aminat-li	žit'a	gu-r-ha-b-ilt'-unne ca-b\\
			Aminat\tsc{-erg}	carrot	\tsc{sub-abl-up-n-}tear\tsc{.ipfv-icvb} \tsc{cop-n}\\
		\glt	\sqt{Aminat is tearing out a carrot from under the earth.} (E)

		\ex	{[Where is Aminat?]}\\	\label{ex:Aminat is tearing out a carrot from under the earth@16b}
		\gll	Aminat-li	žit'a	gu-r-ha-b-ilt'-unne ca-r\\
			Aminat\tsc{-erg}	carrot	\tsc{sub-abl-up-n-}tear\tsc{.ipfv-icvb} \tsc{cop-f}\\
		\glt	\sqt{Aminat is tearing out a carrot from under the earth.} (E)
	\end{xlist}
\end{exe}

Constituent order and closeness to the agreement controller also play a role for deviant agreement. In sentence \refex{ex:Rasul remembers what he had done@c} the controller occurs in sentence-initial position whereas the target, the copula, appears clause-finally. With such a constituent order agreement with a dative (or ergative) controller is highly marginal (although available in elicitation as \refex{ex:Aminat is tearing out a carrot from under the earth@16b} proves). It becomes possible when the controller occurs next to the target, more specifically when it is following the target \refex{ex:Rasul remembers what he had done@d}. In fact, in all but one instance of agreement with an ergative or dative argument attested in the Sanzhi corpus the controller immediately follows the copula \refex{ex:Maybe these, he wants to take}, \refex{ex:bring those years back now}. Furthermore, the controllers are expressed by pronouns \refex{ex:‎‎She is telling stories, she says to me}, \refex{ex:bring those years back now}, or absent from the clause \refex{ex:She put it on HIS nose}, \refex{ex:wants to give him knowledge}


\begin{exe}
	\ex	\label{ex:Rasul remembers what he had done@X}
	\begin{xlist}
		\ex	\label{ex:Rasul remembers what he had done@c}
		\gll	Rasul-li-j	cin-ni	d-arq'-ib-te	han d-irčaq-ul ca-d\\
			Rasul\tsc{-obl-dat}	\tsc{refl-erg}	\tsc{npl-}do\tsc{.pfv-pret-dd.pl} 	remember \tsc{npl}-occur.\textsc{ipfv-icvb} \tsc{cop-npl}\\
		\glt	\sqt{Rasul remembers what he had done.} (E)

		\ex	\label{ex:Rasul remembers what he had done@d}
		\gll	cin-ni	d-arq'-ib-te	han d-irčaq-ul ca-w	Rasul-li-j\\
			\tsc{refl-erg}	\tsc{npl-}do\tsc{.pfv-pret-dd.pl} 	remember \tsc{npl-}occur\tsc{.ipfv-icvb} \tsc{cop-m}	Rasul\tsc{-obl-dat}\\
		\glt	\sqt{Rasul remembers what he had done.} (E)
	\end{xlist}
\end{exe}

This seems to point to point to an explanation based on topicality as formulated by \citet{Sumbatova.Lander2014}, i.e., topical items control agreement. And more specifically, deviant agreement is only possible by topical controllers, because (i) pronouns and zero arguments are usually topical, and (ii) the position after the predicate is a frequent position for topical subjects in Sanzhi, and can also be used for contrastive topics (\refsec{sec:Constituent order at the clause level and information structure}). 

Yet, this analysis must be rejected. My Sanzhi data are in accordance with \citet{GanenkovForthcoming}, who notices a number of problems with the ``topic controller hypothesis''. Most importantly, the hypothesis implies that in the majority of transitive clauses the patient must be topical, because it is far more common for the absolutive patient to control the gender agreement than for the ergative agent to control it. Such an assumption seems implausible. Ganenkov further shows that focal arguments or indefinite pronouns that cannot be topical nevertheless control gender agreement. His arguments can be replicated for Sanzhi Dargwa. 

Furthermore, it is not clear for all corpus examples that the controller is really topical. For instance, in \refex{ex:‎‎She takes (the jug) out of my hands (and washes my legs)} the referent of the omitted ergative argument that controls the agreement has not been mentioned in the preceding context, apart from the use of the indefinite pronoun, because the speaker had a specific person in mind, but could not remember her name. Thus, the absent agreement controller in this sentence cannot really be called ``topical''.
%
\begin{exe}
	\ex	{[To someone (= a woman whose name the speaker forgot) I said, well I will wash (my legs)]}\\	\label{ex:‎‎She takes (the jug) out of my hands (and washes my legs)}
	\gll	kʷi-r-sawtː-ul	ca-r\\
		\tsc{in.the.hands-abl-}tear.off\tsc{.ipfv}-\tsc{icvb}	\tsc{cop-f}\\
	\glt	\sqt{‎‎She takes (the jug) out of my hands (and washes my legs).}
\end{exe}

Moreover, most topical agents or experiencers do not control gender agreement, as in \refex{ex:When she was sweeping, she found a walnut.}. This sentence is the first main clause with a transitive predicate in the narrative. The agent argument, which refers to the protagonist of the story, is the sentence topic and has been omitted. It does not control agreement. Instead, the agreement in the main clause is controlled by the newly introduced patient argument, which is not topical under any account of topicality.

\begin{exe}
	\ex	 {[Once upon a time there was a girl called Patima. She was the oldest within her family. Once after the rain (she) went up to sweep in front of the house.]} \\ \label{ex:When she was sweeping, she found a walnut.} 
	\gll	qʷaˁrš	b-irq'-an=qːel,	b-arčː-ib	ca-b	qix\\
		sweep	\tsc{n}-do.\tsc{ipfv-ptcp}=when	\tsc{n}-find.\tsc{pfv-pret}	\tsc{cop-n}	nut\\
	\glt	\sqt{When she was sweeping, she found a walnut.}
\end{exe}

Therefore, the topicality hypothesis as formulated by \citet{Sumbatova.Lander2014} needs to be rejected, and for a final conclusion about the pragmatic functions of deviant agreement more research is needed.

From a syntactic point of view, the sentences discussed in this section lead to the question whether they are really counterexamples to the claim that gender agreement can only be controlled by nouns in the absolutive case. This would not be the case if it were possible to analyze them as biclausal. This means that the copula is the head of the superordinate clause and agrees with a non-overt absolutive argument that is coreferent with the ergative or dative argument in the subordinate clause. Such an analysis would motivate the pragmatic differences between absolutive and ergative agreement \refex{ex:Aminat is tearing out a carrot from under the earth@16a}, \refex{ex:Aminat is tearing out a carrot from under the earth@16b}, and it would also be consistent with the generalization that the prefixes can only agree with the absolutive argument. 

This idea has been proposed by \citet{Sumbatova2010} and \citet{GanenkovForthcoming}. Ganenkov observes that certain characteristics of deviant gender agreement, namely that it is restricted to the copula-auxiliary (as opposed to agreement prefixes of lexical verbs) and thus found only in periphrastic tenses, resemble biabsolutive constructions. In biabsolutive constructions, the agent agrees with the copula-auxiliary and the patient with the lexical verb \refex{ex:As to Murad, he is mowing hay.}. They have been described for many East Caucasian languages (see \citet{Forker2012a} and \citet{Gagliardietal.2014} for recent accounts) including Sanzhi's neighbor Icari \citet[156]{Sumbatova.Mutalov2003}, but are not attested in Sanzhi. In biabsolutive constructions, the agent is generally topicalized whereas the patient is pragmatically demoted and backgrounded. For biabsolutive constructions a biclausal analysis has been proposed \citep{Kazenin1998, Kazenin.Testelec1999, Kazenin2001}: the agent in the higher clause controls agreement on the copula-auxiliary just like other intransitive predicates; the patient is located in the subordinate clause and thus only controls the agreement of the lexical verb \refex{ex:As to Murad, he is mowing hay.}.

\begin{exe}
	\ex	Icari Dargwa \citep[156]{Sumbatova.Mutalov2003} \\\label{ex:As to Murad, he is mowing hay.}
	\gll	murad	[mura	d-utː-a-tːi]	ca-w\\
		Murad	hay(\tsc{npl})	\tsc{npl}-mow.\tsc{ipfv-prog-prog.cvb}	\tsc{cop-m}\\
	\glt	\sqt{As to Murad, he is mowing hay.}
\end{exe}

\citet{GanenkovForthcoming} adopts the biclausal analysis for deviant agreement and poses an unexpressed absolutive argument higher in the clause that is co-referent with the ergative (or dative) argument and controls the agreement on the copula auxiliary. In other words, the initial ergative subject raises to the position of the higher absolutive subject (subject-to-subject raising) and controls gender agreement. By contrast, the expressed ergative argument is located in the subordinate clause as schematized in \refex{ex:Aminat is tearing out a carrot from}. 

\begin{exe}
	\ex	\label{ex:Aminat is tearing out a carrot from}
	\gll	 \textunderscore\raisebox{-.4ex}{\scriptsize i}\emph{\tsc{(abs)}}	[Aminat-li\raisebox{-.4ex}{\scriptsize i}	žit'a	gu-r-ha-b-ilt'-unne]	ca-r\\
		{} Aminat-\tsc{erg}	carrot	\tsc{sub-abl-up-n}-tear.\tsc{ipfv-icvb} \tsc{cop-f} \\
	\glt	\sqt{Aminat is tearing out a carrot from under the earth.}
\end{exe}

Ganenkov's proposal goes back to the suggestion by \citet{Sumbatova2010} to analyze deviant agreement as backward control and thus also as having a bipartite structure. In backward control constructions, the subject-like argument of a complement-taking predicate is expressed in the complement clause and thus receives case marking from the embedded lexical verb. Nevertheless, the matrix verb shows agreement with the embedded controller (see \refex{ex:‎Is Paitu able to dance} in \refsec{ssec:Infinitive and subjunctive} and also \refex{ex:‎The girl will be able to sew the dress}, \refex{ex:‎The girl began to eat the pilau} in \refsec{sec:Argument control in complement constructions}). On the surface this looks like agreement with an oblique argument, but as \citet{Polinsky.Potsdam2002, Polinsky.Potsdam2006} have shown for Tsez, it can be argued that the matrix verb contains a covert controllee in the absolutive case.

However, the biclausal approach seems to be problematic. As \citet{Forker2012a} demonstrated, a synchronic biclausal analysis for many biabsolutive constructions is not tenable. The same can be said for constructions with deviant agreement. As explained above, in natural texts the default position of the agreement controller is after the predicate. Since subordinate clauses cannot be discontinues or split up by constituents from the main clause, we thus would have to claim that the pronoun in \refex{ex:when, he was talking.} occurs to the right of the clausal boundary. However, topical pronouns following the predicate are common and there is no reason to assume that they are extraclausal constituents (e.g. no intonational break). Furthermore, in examples such as \refex{ex:‎‎He is telling about what he experienced} not only the subject pronoun but also the complement would have to be treated as extraclausal. 

\begin{exe}
	\ex	\label{ex:when, he was talking.}
	\gll	[χabar	b-urs-ul]	ca-w	il-i-l\\
		story	\tsc{n-}tell\tsc{-icvb}	\tsc{cop-m}	that\tsc{-obl-erg}\\
	\glt	\sqt{He was talking (telling stories).}
\end{exe}

In sum, neither the functional-pragmatic properties nor the syntactic properties of deviant agreement are settled. It seems that a synchronic biclausal analysis for deviant agreement poses problems (though a diachronic analysis may still be possible). Alternatively, we can argue that a binary opposition of monoclausal vs. biclausal constructions is too limited. We should instead refine our notion of clause by applying the model of Multivariate Typology \citep{Bickel2011, Bickel2015}. This would mean breaking up the notion of clause into a sensible number of variables by means of detailed language-specific studies. In a second step we can then check our data for clusters around potential categories such as \textit{monoclausal} and \textit{biclausal} constructions and determine whether the Sanzhi deviant agreement construction fits into one of these.

A typologically-informed account of deviant agreement needs to not only take into account what we know so far, but also non-verbal agreement controllers, more detailed information about word order and further aspects that have not been investigated yet. To the latter belong referential properties of the agreement controller such as animacy or humanness, since we know that some languages do not allow inanimate agents in biabsolutive constructions \citep{Forker2012a}.


%%%%%%%%%%%%%%%%%%%%%%%%%%%%%%%%%%%%%%%%%%%%%%%%%%%%%%%%%%%%%%%%%%%%%%%%%%%%%%%%

\section{Person agreement}
\label{sec:Person agreement}


% - - - - - - - - - - - - - - - - - - - - - - - - - - - - - - - - - - - - - - - - - - - - - - - - - - - - - - - - - - - - - - - - - - - - - - - - - - - - - - - - - - - - - - - - - - - - - - - - - - - - - - - - - - - - - - - - - - - - - - - - - - %

\subsection{Introduction}
\label{ssec:IntroductionPersonAgreement}

Like all Dargwa varieties, Sanzhi Dargwa has person agreement enclitics and agreement suffixes. When suffixes are used and when enclitics are used depends on the TAM forms of the verbs, which means that all verbs can, in principle, be used with person suffixes and with person enclitics (with the exception of the morphologically defective copula verbs, which can only attach person enclitics). Suffixes and enclitics follow the same agreement rules, but differ in their form and morphosyntactic characteristics. The origins of the Dargwa agreement systems including Sanzhi Dargwa remain opaque. Pronouns and auxiliaries have been proposed as possible sources but there are no reliable proofs \citep[147\tnd158]{Sumbatova2011}.

The form of the agreement suffixes varies depending on the TAM form. There are a number of different sets. They mostly resemble each other because (i) the third person is either unmarked or differs from the other persons in morphological make-up, (ii) the first and third person are not differentiated for number, and (iii) only the second person has two distinct suffixes for the singular and the plural. Thus the person systems are rather reduced, with a clear opposition of speech act participants (first and second person) vs. third person.

The use of the suffixes is restricted to verbs, i.e. only verbs can serve as targets. The most common sets of person suffixes are given in \reftab{tab:Person agreement suffixes in the habitual present and habitual past}, \reftab{tab:Conditional forms}, and \reftab{tab:The optative, imperative, and prohibitive}. Imperative and prohibitive suffixes are given here because of their resemblance with the optative paradigm (imperative) and the habitual present, haitual past and conditional paradigms (prohibitive), which suggests a diachronic relationship.

Person agreement is subject to clause-level conditions because not all verb forms of main clauses have person agreement markers. Certain forms with past time reference (e.g. the past progressive, the evidential past, and the evidential pluperfect) make use of the past enclitic, which is in complementary distribution with the person enclitics. Another factor is finiteness: almost exclusively verb forms in finite main clauses and in conditional clauses can be marked for person agreement. Thus, the masdar, converbs, and participles, when used in subordinate clauses, do not contain agreement markers (see \refex{ex:We wanted to cry@26} below for the subjunctive, which represents the exception to this rule).
%

%
\begin{table}
	\caption{Person agreement suffixes in the habitual present and \protect\mbox{habitual} past}
	\label{tab:Person agreement suffixes in the habitual present and habitual past}
	\small
	\begin{tabularx}{0.66\textwidth}[]{%
		>{\raggedright\arraybackslash}p{10pt}
		>{\centering\arraybackslash\itshape}X
		>{\centering\arraybackslash\itshape}X
		>{\centering\arraybackslash\itshape}X
		>{\centering\arraybackslash\itshape}X}
		
		\lsptoprule
		{}	&	\multicolumn{2}{c}{habitual present (\tsc{ipfv})}	&	\multicolumn{2}{c}{habitual past (\tsc{ipfv})}\\\cmidrule(lr){2-3}\cmidrule(lr){4-5}
		{}	&	\tup{\tsc{sg}}	 &	\tup{\tsc{pl}}		&	\tup{\tsc{sg}}	&	\tup{\tsc{pl}}\\
		\midrule 
		1	&	\multicolumn{2}{c}{-\tup{V}-\textit{d}}			&	\multicolumn{2}{c}{-\textit{a-di}} 			\\
		2	&	-\tup{V}-tːe		&	-\tup{V}-tːa			&	-a-tːe			&	-a-tːa\\
		3	&	\multicolumn{2}{c}{-\textit{u}\slash -\textit{ar}}					&	\multicolumn{2}{c}{-\textit{i(ri)}}			\\
		\lspbottomrule
	\end{tabularx}
\end{table}
%
\begin{table}
	\caption{Person agreement suffixes in conditional forms}
	\label{tab:Conditional forms}
	\small
	\begin{tabularx}{0.97\textwidth}[]{%
		>{\raggedright\arraybackslash}p{10pt}
		>{\centering\arraybackslash\itshape}p{36pt}
		>{\centering\arraybackslash\itshape}p{36pt}
		>{\centering\arraybackslash\itshape}p{36pt}
		>{\centering\arraybackslash\itshape}p{36pt}
		>{\centering\arraybackslash\itshape}X
		>{\centering\arraybackslash\itshape}X}
		
		\lsptoprule
		{}	&	\multicolumn{2}{c}{realis cond. (\tsc{pfv})}	&	\multicolumn{2}{c}{past cond. (\tsc{pfv})}	&	\multicolumn{2}{c}{imperfective cond. (\tsc{ipfv})}\\\cmidrule(lr){2-3}\cmidrule(lr){4-5}\cmidrule(lr){6-7}
		{}	&	\tup{\tsc{sg}}	 &	\tup{\tsc{pl}}	&	\tup{\tsc{sg}}	&	\tup{\tsc{pl}}	&	\tup{\tsc{sg}}	&	\tup{\tsc{pl}}\\
		\midrule
		1	&	\multicolumn{2}{c}{-\tup{V}-\textit{lle}}			&	\multicolumn{2}{c}{-\tup{V}-\textit{tːel}}		&	\multicolumn{2}{c}{-\textit{aχː-a-lle}}		\\
		2	&	-\tup{V}-tːe(l)	&	-\tup{V}-tːal		&	-\tup{V}-tːel		&	-\tup{V}-tːal		&	-aχː-a-t(te)		&	-aχː-a-t(tal)\\
		3	&	\multicolumn{2}{c}{-\textit{ar(re)}\slash -\textit{an}}			&	\multicolumn{2}{c}{-\textit{ar-del}\slash -\textit{an-del}}		&	\multicolumn{2}{c}{-\textit{aχː-a-n(ne)}\slash -\textit{aχː-a-r(re)}} 	\\
		\lspbottomrule
	\end{tabularx}
\end{table}
%
\begin{table}
	\caption{Person agreement in the optative, imperative, and prohibitive}
	\label{tab:The optative, imperative, and prohibitive}
	\small
	\begin{tabularx}{0.80\textwidth}[]{%
		>{\raggedright\arraybackslash}p{10pt}
		>{\centering\arraybackslash\itshape}X
		>{\centering\arraybackslash\itshape}X
		>{\centering\arraybackslash\itshape}X
		>{\centering\arraybackslash\itshape}X
		>{\centering\arraybackslash\itshape}X
		>{\centering\arraybackslash\itshape}p{42pt}}
		
		\lsptoprule
		{}	&	\multicolumn{2}{c}{optative (\tsc{pfv})}	&	\multicolumn{2}{c}{imperative (\tsc{pfv})}	&	\multicolumn{2}{c}{prohibitive (\tsc{ipfv})}\\\cmidrule(lr){2-3}\cmidrule(lr){4-5}\cmidrule(lr){6-7}
		{}	&	\tup{\tsc{sg}}	 &	\tup{\tsc{pl}}	&	\tup{\tsc{sg}}	&	\tup{\tsc{pl}}	&	\tup{\tsc{sg}}	&	\tup{\tsc{pl}}\\
		\midrule
		1	&	\multicolumn{2}{c}{\tit{-ab-a}}			&	\tmd			&	\tmd			&	\tmd			&	\tmd\\
		2	&	-ab-e			&	-ab-a\slash  		&	-a /			&	-aj(a)\slash 		&	-\tup{V}-t(ːa)		&	-\tup{V}-tːaj(a)\\
		{}	&	{}			&	-ab-aj\slash  		&	-e /			&	-ene			&	{}			&	{}\\
		{}	&	{}			&	-ab-aja 		&	-en			&	{}			&	{}			&	{}\\
		3	&	\multicolumn{2}{c}{\tit{-ab}}			&	\tmd			&	\tmd			&	\tmd			&	\tmd\\
		\lspbottomrule
	\end{tabularx}
\end{table}

In the habitual present, the realis conditional, and the past conditional, the person suffix for the first and second person is preceded by a stem augment vowel that is indicated with V in the Tables above. The vowel is either \tit{i} or \tit{u}. The same vowels are also used in the subjunctive and the prohibitive \refex{ex:We wanted to cry@26}, and the same distinction (though without the stem augment vowels) is attested in the imperative. For one-place verbs \tit{u} is the only vowel that is used. For two-place verbs the following distribution is observed:
%
\begin{itemize}
	\item	\tit{-u} with reflexive and reciprocal constructions and agentive third persons
	\item	\tit{-i} with patientive third persons
	\item	\tit{-u} or \tit{-i} in all other cases, i.e. agentive second person with patientive first, and vice versa agentive first person with patientive second person
\end{itemize}
%
%
\begin{table}
	\caption{Stem augment vowels for transitive and two-place affective verbs}
	\label{tab:Stem augment vowels for transitive and two-place affective verbs}
	\small
	\begin{tabularx}{0.53\textwidth}[]{%
		>{\raggedright\arraybackslash}p{30pt}
		>{\centering\arraybackslash\itshape}X
		>{\centering\arraybackslash\itshape}X
		>{\centering\arraybackslash\itshape}X}
		
		\lsptoprule
		{}		&	\upshape1 patient	 &	\upshape2 patient	&	\upshape3 patient\\
		\midrule 
		1 agent	&	-u			&	-i, -u			&	-i\\
		2 agent	&	-i, -u			&	-u			&	-i\\
		3 agent	&	-u			&	-u			&	-u\\
		\lspbottomrule
	\end{tabularx}
\end{table}

This has been summed up in \reftab{tab:Stem augment vowels for transitive and two-place affective verbs}. The stem augment vowels are treated as part of the suffixes. They are not part of the stem. Therefore, they are not separately glossed, but written together with the TAM suffixes. As the Table shows, there is variation when both core arguments are speech act participants (i.e. first and second person). Based on my corpus data and on elicitation I do not have an explanation for the variation and thus my analysis is only preliminary and requires further research before a conclusion can be reached.

In the following, I will briefly illustrate the use of the stem augment vowels. Sentence \refex{ex:‎‎‎I become rich fast} shows the habitual present first person of an intransitive verb (see also \refex{ex:I (masc.) laugh habitual present} below for another intransitive verb with the stem augment vowel \tit{u}).
%
\begin{exe}
	\ex	\label{ex:‎‎‎I become rich fast}
	\gll	ixʷle	dawlači-w	w-irχ-ud\\
		fast	rich\tsc{-m}	\tsc{m-}become\tsc{.ipfv-prs.1}\\
	\glt	\sqt{‎‎‎I (masc.) become rich fast.}
\end{exe}

Examples \refex{ex:if you beat me up} and \refex{ex:if you beat him up} illustrate the realis conditional with a person marker for second singular. In the first sentence, the stem augment is \tit{u} but \tit{i} would also be possible). In \refex{ex:if you beat him up} there is a second person agent acting upon a third person, hence only \tit{i} is allowed. 
%
\begin{exe}
	\ex	\label{ex:if you beat me up}
	\gll	u-l	du	w-it-utːe\\
		\tsc{2sg-erg}	\tsc{1sg}	\tsc{m-}beat.up\tsc{-cond.2sg}\\
	\glt	\sqt{if you beat me up} (E)

	\ex	\label{ex:if you beat him up}
	\gll	u-l	it	w-it-itːe\\
		\tsc{2sg-erg}	that	\tsc{m-}beat.up\tsc{-cond.2sg}\\
	\glt	\sqt{if you beat him up} (E)
\end{exe}

In sentence \refex{ex:I will see him@20a}, the habitual present illustrates a first person experiencer with a third person stimulus with the stem augment \tit{-i} and \refex{ex:S/he will see me@20c} shows the reversed scenario with the stem augment vowel \tit{u}.
%
\begin{exe}
	\ex	\label{ex:habitual present stem augmentation@20}
	\begin{xlist}
		\ex	\label{ex:I will see him@20a}
		\gll	dam	it 	či-w-iž-id\\
			\tsc{1sg.dat}	that	\tsc{spr-m-}see\tsc{.ipfv-prs.1}\\
		\glt	\sqt{I will see him.}

		\ex	\label{ex:S/he will see me@20c}
		\gll	it-i-j	du	či-w-ig-ud\\
			that\tsc{-obl-dat}	\tsc{1sg}	\tsc{spr-m-}see\tsc{.ipfv-prs.1}\\
		\glt	\sqt{S/he will see me (masc.)}
	\end{xlist}
\end{exe}

\reftab{tab:Person agreement enclitics} displays the agreement enclitics. As can be seen in this table, only the second person singular has a unique marker. For the third person there are no person markers. Instead, depending on the time reference of the clause and on the context, the third person is left unmarked, or some other marker appears filling the gap in the paradigm (e.g. the copula \tit{ca-b}, which exhibits gender/number agreement or the suffix \tit{-ne}). Person agreement enclitics are widely used throughout the verbal paradigm, e.g. in the compound present and past, the perfect, the preterite, the future, etc.
%
\begin{table}
	\caption{Person agreement enclitics}
	\label{tab:Person agreement enclitics}
	\small
	\begin{tabularx}{0.30\textwidth}[]{%
		>{\raggedright\arraybackslash}p{10pt}
		>{\centering\arraybackslash\itshape}X
		>{\centering\arraybackslash\itshape}X}
		
		\lsptoprule
		{}	&	\tup{\tsc{sg}}	 &	\tup{\tsc{pl}}\\
		\midrule 
		1	&	\multicolumn{2}{c}{=\textit{da}}\\
		2	&	=de			&	=da\\
		3	&	\tmd			&	\tmd\\
		\lspbottomrule
	\end{tabularx}
\end{table}
%

The person enclitics belong to the predicative particles (\refsec{sec:Predicative particles}). They are normally added to the predicate, but, just as other predicative particles, can also be used to express term focus (also called \textit{constituent focus}). In this case, they are encliticized to the item in focus, which can be an argument or adjunct, such that agreement targets are not only verbs but can be also nominals, adverbs, or other items (\citealp{Kalinina.Sumbatova2007}, \citealp{Sumbatova2013}, \citealp{Forker2016a}).

Person suffixes and person enclitics are subject to the same syntactic alignment rules: S, A, P, and T (i.e. the theme argument of a ditransitive verb) control person agreement. Person agreement is obligatory and it is freely combinable with gender/number agreement because both agreement systems operate independently of each other. Only one argument can control the agreement. The alignment patterns for person agreement among the Dargwa languages vary to a substantial extent (see \citealp{Sumbatova2011} and \citey{Sumbatova2013} for overviews). They are determined by the ranking of absolutive vs. ergative arguments, and in a number of varieties also by person hierarchies. The person hierarchies found are either 2 > 1 > 3 (e.g. Icari, Kajtag, Qunqi, and Xuduc) or 1, 2 > 3 (e.g. Akusha and Standard Dargwa, Chirag). In many varieties the hierarchies are combined with a ranking of grammatical roles: patient argument (absolutive) > agent argument (ergative) is found in Akusha and Standard Dargwa, whereas agent argument (ergative) > patient argument (absolutive) has been documented for Chirag, Kubachi, and Mehweb. In Shiri Dargwa, in contrast to the above mentioned varieties, there is a considerable amount of variation within the speech community, and \citet{Belyaev2013} distinguishes three slightly different alignment systems. A similar conclusion can be drawn for Sanzhi. There is also a certain degree of intra- and inter-speaker variation.


% - - - - - - - - - - - - - - - - - - - - - - - - - - - - - - - - - - - - - - - - - - - - - - - - - - - - - - - - - - - - - - - - - - - - - - - - - - - - - - - - - - - - - - - - - - - - - - - - - - - - - - - - - - - - - - - - - - - - - - - - - - %

\subsection{Person agreement rules}
\label{ssec:Person agreement rules}

In clauses with monovalent predicates, only the S argument serves as controller. Examples of first and second person are given in \xxref{ex:I (masc.) laugh habitual present}{ex:Where are you going compound present} for verbal predicates and \refex{ex:I am hungry agreement rules} for a copula construction.
%
\begin{exe}
		\ex	{Habitual present}\\	\label{ex:I (masc.) laugh habitual present}
		\gll	du 	ħaˁħaˁ 		Ø-ik'-ud\\
			\tsc{1sg} 	laughter	\tsc{i}-say\tsc{.ipfv-1}\\
		\glt	\sqt{I (masc.) laugh.}

		\ex	{Realis conditional}\\	\label{ex:There is much there, if you go there realis conditional}
		\gll	celij 	d-aqil	k'e-d,	či-d-uˁq'-uˁtːal\\
			whole	\tsc{npl-}much	exist\tsc{.up-npl}	\tsc{spr-1/2pl-}go\tsc{.pfv-cond.2pl}\\
		\glt	\sqt{There is much there (i.e. the graveyard is large), if you go there.}

		\ex	{Compound present}\\	\label{ex:Where are you going compound present}
		\gll	čina	arg-ul=de?\\
			where	go\tsc{.ipfv-icvb=2sg}\\
		\glt	\sqt{Where are you going?}

	\ex	\label{ex:I am hungry agreement rules}
	\gll	du	kːuš-le=da\\
		\tsc{1sg}	hungry\tsc{-advz=1}\\
	\glt	\sqt{I am hungry.}
\end{exe}

In the following examples, third person agreement with intransitive predicates is illustrated. The agreement exponent can be a suffix as in the examples of the habitual past in \refex{ex:‎The sound went even to the mountains habitual past}. Example \refex{ex:It is snowing compound present} shows the compound present for which the copula is used for third person agreement (whereas in the third or second person a person enclitic would occur, see \refex{ex:Where are you going compound present}). Other analytic tenses such as the preterite do not make use of the copula for the third person (but employ person markers for the first and second person) \refex{ex:My fire died out preterite}. In the copula construction in \refex{ex:The daughter is similar to her mother} also the copula is used. Finally, verb forms such as the compound present that in declarative main clauses require a copula for the third person omit the copula in questions with interrogative enclitics \refex{ex:Where is s/he going compound present}. This is possible because the  interrogative enclitics belong to the predicative particles, which fulfill copula-functions, among other things (\refsec{sec:Predicative particles}).
%
\begin{exe}
		\ex	{Habitual Past}\\	\label{ex:‎The sound went even to the mountains habitual past}
		\gll	daže	hex-tːi	dubur-t-a-cːe	t'ama	ha-d-aš-iri\\
			even	\tsc{dem.up}\tsc{-pl}	mountain\tsc{-pl-obl-in}	sound(\tsc{npl})	\tsc{up-npl-}go\tsc{-hab.pst.3}\\
		\glt	\sqt{‎The sound went even to the mountains.}
		
				\ex	{Preterite}\\	\label{ex:My fire died out preterite}
		\gll	di-la	c'a	d-iš-aq-un\\
			\tsc{1sg-gen}	fire(\tsc{npl})	\tsc{npl-}die.out\tsc{.pfv-caus-pret}\\
		\glt	\sqt{My fire died out.}
		
			\ex	\label{ex:The daughter is similar to her mother}
	\gll	rursːi	aba-j	miši-l ca-r\\
		girl	mother\tsc{-dat}	similar\tsc{-advz} \tsc{cop-f}\\
	\glt	\sqt{The daughter is similar to her mother.} (E)
		
		\ex	{Compound present}\\	\label{ex:Where is s/he going compound present}
		\gll	čina	it	arg-ul=e?\\
			where	that	go\tsc{.ipfv-icvb=q}\\
		\glt	\sqt{Where is s/he going?}
\end{exe}

The same rule applies to extended intransitive verbs, i.e. verbs that have one argument in the absolutive and another one marked with the dative or a spatial case. Thus in \refex{ex:‎I would watch TV if I were able to see habitual conditional past} and \refex{ex:Are you afraid of your wife compound present}, verb forms with first and second person markers occur; in \refex{ex:‎‎‎The bandits did not touch him preterite}, the preterite is used, which lacks a marker for the third person.
%
\begin{exe}
		\ex	Habitual past, conditional past\\	\label{ex:‎I would watch TV if I were able to see habitual conditional past}
		\gll	tiliwizur-ri-j	er	r-ik'ʷ-a-di,	či-d-ig-ul	r-iχ-utːel\\
			television\tsc{-obl-dat}	look	\tsc{f-}look.at\tsc{.ipfv-hab.pst-1}	\tsc{spr-npl-}see\tsc{.ipfv-icvb}	\tsc{f-}be.able\tsc{.pfv-cond.pst.1sg}\\
		\glt	\sqt{‎I would watch TV if I were able to see.}

		\ex	Compound present\\	\label{ex:Are you afraid of your wife compound present}
		\gll	xːunul-li-sa-r	uruχ	Ø-ik'-ul=de=w?\\
			woman\tsc{-obl-ante-abl}	fear	\tsc{m-}say\tsc{.ipfv-icvb=2sg=q}\\
		\glt	\sqt{Are you afraid of your wife?}

		\ex	Preterite\\	\label{ex:‎‎‎The bandits did not touch him preterite}
		\gll	iltːi	qːačuʁ-e	kːač	a-b-ič-ib	il-i-j\\
			\tsc{3pl}	bandit\tsc{-pl}	touch	\tsc{neg-hpl-}occur\tsc{.pfv-pret}	that\tsc{-obl-dat}\\
		\glt	\sqt{‎‎‎The bandits did not touch him.}
\end{exe}

There are a number of monovalent predicates that lack absolutive arguments and have only dative arguments. In \refsec{General remarks on gender/number agreement} the consequences for gender agreement were discussed. These predicates cannot control person agreement, and instead the third person is always used \refex{ex:I / you / she /	he feel(s) bad there}, \refex{ex:‎‎‎I got cold agreement}. A number of weather predicates only have ergative arguments, and likewise they only exhibit third person agreement \refex{ex:It is snowing compound present}.
%
\begin{exe}
	\ex	\label{ex:I / you / she /	he feel(s) bad there}
	\gll	dam	/	at	/ hel-i-j	wahi-l	ca-b	heχ-tːu-b\\
		\tsc{1sg.dat}	/	\tsc{2sg.dat}	/	that\tsc{-obl-dat}	bad\tsc{-advz}	\tsc{cop-n}	\tsc{dem.down}\tsc{-loc-n}\\
	\glt	\sqt{I\slash you\slash she\slash he feel(s) bad there.} (E)

	\ex	\label{ex:‎‎‎I got cold agreement}
	\gll	dam	b-uχːar	ačː-ib\\
		\tsc{1sg.dat}	\tsc{n-}cold	get\tsc{.pfv-pret}\\
	\glt	\sqt{‎‎‎I got cold.} (E)

	\ex	Compound present\\	\label{ex:It is snowing compound present}
	\gll	duˁħi-l	b-us-ul	ca-b\\
		snow\tsc{-erg}	\tsc{n-}snow\tsc{.ipfv-icvb}	\tsc{cop-n}\\
	\glt	\sqt{It is snowing.} (E)
\end{exe}

There are other monovalent predicates that are compound verbs, and that from a morphological point of view contain petrified nominal arguments which in some cases control gender agreement and in others do not. These behave just like any other monovalent predicate, i.e. the single argument controls the person agreement \refex{ex:‎‎Come on, we will eat future}; see also \refex{ex:I am angry@b} above.
%
\begin{exe}
	\ex	Future\\	\label{ex:‎‎Come on, we will eat future}
	\gll	dawaj		(nušːa)	dum	d-alt-an=da\\
		let's		(\tsc{1pl})	eating	\tsc{1/2pl-}let\tsc{.ipfv-ptcp=1}\\
	\glt	\sqt{‎‎Come on, we will eat.} (modified example)
\end{exe}

In clauses with bivalent verbs that are either genuine transitive verbs or affective verbs both arguments (i.e. agents\slash experiencers, and patients\slash stimuli) can control person agreement, but only one argument at a time. 

In clauses with only third person arguments we find the respective agreement markers for the third person:
%
\begin{exe}
		\ex	Habitual present\\	\label{ex:S/he sees him habitual present}
		\gll	it-i-j	it	či-w-ig-u\\
			that\tsc{-obl-dat}	that	\tsc{spr-m-}see\tsc{.ipfv-prs.3}\\
		\glt	\sqt{S/he sees him.} (E)

		\ex	Resultative\\		\label{ex:The police took him resultative}
		\gll	milic'a-b-a-l	w-erč-ib	ca-w	il\\
			police\tsc{-pl-obl-erg}	\tsc{m-}lead\tsc{.pfv-pret}	\tsc{cop-m}	that\\
		\glt	\sqt{The police took him.}

		\ex	Future\\		\label{ex:Now he will also bring trouble future}
		\gll	na=ra	bala	q'adar	či-sa-d-iqː-an-ne\\
			now\tsc{=add}	misfortune	destiny	\tsc{spr-hither}\tsc{-npl-}carry\tsc{.ipfv-ptcp-fut.3}\\
		\glt	\sqt{Now he will also bring trouble.}
\end{exe}

If we have one third-person argument and one first or second-person argument the latter controls the agreement, independently of the grammatical relation, i.e. these clauses are governed by the person hierarchy 1, 2 > 3. 
%
\begin{exe}
	\ex	Realis conditional 1 > 3\\	\label{ex:‎‎If (I) send my (brother) RC13}
	\gll	di-la	w-at	k-aʁ-ille ...\\
		\tsc{1sg-gen}	\tsc{m-}send		\tsc{down}-do\tsc{.pfv-cond.1.prs}\\
	\glt	\sqt{‎‎If (I) send my (brother) ...}

	\ex	Realis conditional 2 > 3\\	\label{ex:‎‎‎like this, if (you) put this here, in the middle RC23}
	\gll	wot	tak	het	hetːu-b-a	sa-qː-itːel	urkːa ...\\
		well	so	that	there\tsc{-n-dir}	\tsc{hither}-carry\tsc{.pfv-cond.2sg.prs}	between\\
	\glt	\sqt{‎‎‎like this, if (you) put this here, in the middle ...}

	\ex	Habitual present 2 > 3\\	\label{ex:‎‎‎I have a cousin called Mamala Kurban, you know him HP 23}
	\gll	iž	di-la	ucːiq'ar	χe-w,	Mamma-la	Q'urban	b-ik'-ul,	ašːi-j	w-alχ-atːa\\
		this	\tsc{1sg-gen}	cousin	exist.\tsc{down-m}	Mamma\tsc{-gen}	Kurban	\tsc{hpl-}say\tsc{.ipfv-icvb}		\tsc{2pl-dat}		\tsc{m-}know\tsc{.ipfv-prs.2pl}\\
	\glt	\sqt{‎‎‎I have a cousin called Mamala Kurban, you know him.}

	\ex	Habitual past 3 > 1\\	\label{ex:‎‎‎Uncle Shamkhal led me HB31}
	\gll	šːamχal	acːi-l	r-ik-a-di\\
		Shamxal	uncle\tsc{-erg}	\tsc{f-}lead\tsc{.ipfv-hab.pst-1}\\
	\glt	\sqt{‎‎‎Uncle Shamkhal led me (fem.).}

	\ex	Preterite 3 > 2\\		\label{ex:Did the poor man pull you out PT32}
	\gll	tːura	ha-qː-ib=de=w	u	iž	miskin-ni?\\
		outside	\tsc{up}-carry\tsc{.pfv-pret=2sg=q}	\tsc{2sg}	this	poor\tsc{-erg}\\
	\glt	\sqt{Did the poor man pull you out?}
\end{exe}

In clauses with two speech act participants, in principle either participant can control agreement independently of its grammatical role. All four logically possible combinations can be obtained in elicitation with male and female Sanzhi speakers of various ages:
%
\begin{exe}
	\ex	\label{ex:‎‎‎I will keep you (masc.) in my hands}
	\begin{xlist}
		\ex	1 > 2, agent controls agreement\\	\label{ex:‎‎‎I will keep you (masc.) in my hands 12AcA}
		\gll	du-l	u	kʷi	urc-an=da\\
			\tsc{1sg-erg}	\tsc{2sg}	in.the.hands	keep\tsc{.m.ipfv-ptcp=1}\\
		\glt	\sqt{‎‎‎I will keep you (masc.) in my hands.} (E)

		\ex	1  > 2, patient controls agreement\\	\label{ex:‎‎‎I will keep you (masc.) in my hands 12PcA}
		\gll	du-l	u	kʷi	urc-an=de\\
			\tsc{1sg-erg}	\tsc{2sg}	in.the.hands	keep\tsc{.m.ipfv-ptcp=2sg}\\
		\glt	\sqt{‎‎‎I will keep you (masc.) in my hands.} (E)
	\end{xlist}

	\ex	\label{ex:‎‎‎You will keep me (masc.) in my hands}
	\begin{xlist}
			\ex	2  > 1, agent controls agreement\\	\label{ex:‎‎‎You will keep me (masc.) in my hands 21PcA}
		\gll	u-l	du	kʷi	urc-an=de\\
			\tsc{2sg-erg}	\tsc{1sg}	in.the.hands	keep\tsc{.m.ipfv-ptcp=2sg}\\
		\glt	\sqt{‎‎‎You will keep me (masc.) in my hands.} (E)
		
		\ex	2 > 1, patient controls agreement\\	\label{ex:‎‎‎You will keep me (masc.) in my hands 21AcA}
		\gll	u-l	du	kʷi	urc-an=da\\
			\tsc{2sg-erg}	\tsc{1sg}	in.the.hands	keep\tsc{.m.ipfv-ptcp=1}\\
		\glt	\sqt{‎‎‎You will keep me (masc.) in my hands.} (E)


	\end{xlist}
\end{exe}

There is only one example of such a scenario in my corpus \refex{ex:if you save me RC21}, and it shows agreement controlled by a second person agent.
%
\begin{exe}
	\ex	Realis conditional 2 > 1\\	\label{ex:if you save me RC21}
	\gll	du	w-erc-aq-utːe	\\
		\tsc{1sg}	\tsc{m-}save\tsc{.pfv-caus-cond.2sg.prs}\\
	\glt	\sqt{‎(I give you a lot of money) if you save me.}
\end{exe}

It seems that there is a slight tendency in elicitation for speakers to prefer the examples in which the second person controls the agreement, be it a second person agent, patient, experiencer, or stimulus \refex{ex:You see me HP21@21b}. Nevertheless, Sanzhi Dargwa is unlike Icari in having also first person agreement controllers in clauses with only speech act participants \refex{ex:I see you HP12@21a}, \refex{ex:‎We hit you HP12}. The same variation in person alignment has also been attested for Shiri Dargwa in \citet{Belyaev2013}.
%
\begin{exe}
		\ex	Habitual present 2 > 1\\	\label{ex:You see me HP21@21b}
		\gll	at	du	či-w-ig-utːe\\
			\tsc{2sg.dat}	\tsc{1sg}	\tsc{spr-m-}see\tsc{.ipfv-2sg}\\
		\glt	\sqt{You see me.} (E)
		
		\ex	Habitual present 1 > 2\\	\label{ex:I see you HP12@21a}
		\gll	dam	u	či-w-ig-utːe\\
			\tsc{1sg.dat}	\tsc{2sg}	\tsc{spr-m-}see\tsc{.ipfv-2sg}\\
		\glt	\sqt{I see you.} (E)

		\ex	Habitual present 1 > 2\\	\label{ex:‎We hit you HP12}
		\gll	nušːa-l	ušːa	d-uˁrq-itːa\\
			\tsc{1pl-erg}	\tsc{2pl}	\tsc{1/2pl-}hit\tsc{.ipfv-prs.2pl}\\
		\glt	\sqt{‎We hit you.} (E)
\end{exe}

To sum up scenarios with two speech act participants functioning as agents and patients, I can only state that my preliminary analysis did not yield more precise results and that the variation is an interesting problem, which requires further testing.

The alignment patterns, including the described variation, seems to slightly change for predicates with three arguments. As said above, recipients, addressees, beneficiaries, and other arguments that are not agents or patients never control person agreement. In sentences with first person agent arguments and second person patient arguments, both agent and patient can control the agreement. This means we either have hierarchical agreement with 2 > 1 as in the second version of \refex{ex:‎Will you show me to Madina FT21}, or agreement with the agent as in the first version of \refex{ex:‎‎‎I showed you to Madina PT12} and in \refex{ex:(I) will show (you) the tree FT12}. If the agent is a second person pronoun, only this argument can control the agreement \refex{ex:‎Will you show me to Madina FT21}. Agreement controlled by the first person patient argument is ungrammatical. This is in contrast to examples with two-place predicates such as \refex{ex:‎‎‎You will keep me (masc.) in my hands 21AcA} which has a first person patient argument controlling the agreement. At the present moment I do not have any explanation for why the agreement patterns of three-place verbs seem to diverge from those of two-place verbs and the few examples I was able to elicit do not allow me to draw and further conclusions or to develop hypotheses, so this topic must be left for future research.


%
\begin{exe}
		\ex	Preterite 1 > 2\\	\label{ex:‎‎‎I showed you to Madina PT12}
		\gll	du-l	u	Madina-j	či-w-až-aq-ib=da	/	či-w-až-aq-ib=de\\
			\tsc{1sg-erg}	\tsc{2sg}	Madina\tsc{-dat}	\tsc{spr-m-}see\tsc{.pfv-caus-pret=1}\slash\tsc{spr-m-}see\tsc{.pfv-caus-pret=2sg} \\
		\glt	\sqt{‎‎‎I showed you to Madina.} (E)

		\ex	Future 1 > 2\\	\label{ex:(I) will show (you) the tree FT12}
		\gll	hek'	kːalkːi	či-b-až-aq-an=da\\
			\tsc{dem.up}	tree	\tsc{spr-n-}see\tsc{.pfv-caus-ptcp=1}\\
		\glt	\sqt{(I) will show (you) the tree.}

		\ex	Future 2 > 1\\	\label{ex:‎Will you show me to Madina FT21}
		\gll	u-l	du	Madina-j	či-w-iž-aq-an=de=w /	 {*}~či-w-iž-aq-an=da=w?\\
			\tsc{2sg-erg}	\tsc{1sg}	Madina\tsc{-dat}	\tsc{spr-m-}see\tsc{.ipfv-caus-ptcp=2sg=q}	/	{\hphantom{*}}~\tsc{spr-m-}see\tsc{.ipfv-caus-ptcp=1=q}\\
		\glt	\sqt{‎Will you show me to Madina?} (E)
\end{exe}

As soon as a speech act participant co-occurs with a third person agent or patient, it is the speech act participant that controls the agreement \refex{ex:I did not ask you anything PT13}, \refex{ex:‎‎if you give the girl to their son RC23}. In \refex{ex:‎‎if you give the girl to their son RC23} the verb also has a gender/number agreement prefix that is controlled by the absolutive argument. Thus, we can clearly see that person and gender/number agreement function independently. In clauses with only third person agents and patients we find third person agreement, even if we have first or second person recipients \refex{ex:(He) gave me his bag PT33}.
%
\begin{exe}
		\ex	Preterite 1 > 3\\		\label{ex:I did not ask you anything PT13}
		\gll	du-l	a-cːe	cik'al-la	tiladi	a-b-arq'-ib=da\\
			\tsc{1sg-erg}	\tsc{2sg-in}	thing\tsc{-gen}	request	\tsc{neg-n-}do\tsc{.pfv-pret=1}\\
		\glt	\sqt{I did not ask you anything!}

		\ex	Realis conditional 2 > 3\\	\label{ex:‎‎if you give the girl to their son RC23}
		\gll	hetː-a-la	durħuˁ-li-j	hej	rursːi	r-ičː-itːe\\
			those\tsc{-obl-gen}	boy\tsc{-obl-dat}	this	girl	\tsc{f-}give\tsc{.pfv-cond.2sg.prs}\\
		\glt	\sqt{‎‎if you give the girl to their son}

		\ex	Preterite 3 > 3\\		\label{ex:(He) gave me his bag PT33}
		\gll	sumk'a	di-cːe	b-ičː-ib\\
			bag	\tsc{1sg-in}	\tsc{n-}give\tsc{.pfv-pret}\\
		\glt	\sqt{(He) gave me his bag.}
\end{exe}
%


The obligative (\refsec{ssec:Obligative}), the obligative present (\refsec{ssec:Obligative present}) and the experiential I and II (\refsec{ssec:Experiential I and experiential II}) diverge from the TAM forms discussed so far in their agreement rules because they do not make use of any person hierarchy, but person agreement is always controlled by the patient (in clauses with two-place verbs). Thus, example \refex{ex:I cut those reeds1} shows the experiential I with the third person patient serving as agreement controller. The use of the first person enclitic is ungrammatical. Sentence \refex{ex:‎‎‎I was born in 1935 analytic2} from the corpus illustrates the experiential II and does not have an overt agent, but an overt first person patient, which controls the agreement on the verb. All examples also show that the patient also controls the gender marking on the lexical verb (and on the copula if there is any), which is expected and in accordance with the gender agreement rules. Furthermore, the cross-categorical suffix on the lexical verb agrees in number with the patient: a singular patient requires the suffix -\textit{ce} \refex{ex:‎I gave birth to (my son) under a blanket@A} or -\textit{il} \refex{ex:‎‎‎I was born in 1935 analytic2}, \refex{ex:‎I gave birth to (my son) under a blanket@B}; a plural patient requires -\textit{te} \refex{ex:Then you have to find them2}, \refex{ex:I cut those reeds1}, \refex{ex:‎‎‎I have to give the money back}.


%
\begin{exe}
	\ex	\label{ex:I cut those reeds1}
	\gll	itːi	qːamuš		dul		ka-d-ičː-ib-te	ca-d	/	{*}~ka-d-ičː-ib-te=da\\
		those	reed		\tsc{1sg.erg}	\tsc{down-npl-}cut.up\tsc{.pfv-pret-dd.pl}	\tsc{cop-npl}		/	{\hphantom{*}}~\tsc{down-npl-}cut.up\tsc{.pfv-pret-dd.pl=1}\\
	\glt	\sqt{I cut those reeds.} (E)
	
		\ex	\label{ex:‎‎‎I was born in 1935 analytic2}
	\gll	w-arq'-ib-il=da	du	azir-lim	urč'em	darš-lim	ʡaˁb-c'anu	xu-ra-ibil\\
		\tsc{m-}do\tsc{.pfv-pret-ref=1}	\tsc{1sg}	thousand\tsc{-num}	nine	hundred\tsc{-num}	three-\tsc{tens}	five\tsc{-num-ord}\\
	\glt	\sqt{‎‎‎I (masc.) was born in 1935.}
	
		\ex	\label{ex:‎I gave birth to (my son) under a blanket@A}
		\gll	du-l	julʁan-ni-gu-w	w-arq'-ib-ce	ca-w\\
			\tsc{1sg-erg}	blanket\tsc{-obl-sub-m}	\tsc{m-}do\tsc{.pfv-pret-dd.sg} 	\tsc{cop-m}\\
		\glt	\sqt{‎I gave birth to (my son) under a blanket.} [modified corpus example]

\end{exe}


In \refex{ex:‎‎‎I have to give the money back} the modal enclitic \tit{=q'al} functions as a predicative marker such that the copula can be omitted (which is accordance with the general rules for the omission of the copula).
 
 
\begin{exe}

	\ex	\label{ex:‎‎‎I have to give the money back}
	\gll	arc	lukː-an-te=q'al	du-l\\
		money(\tsc{npl})	give\tsc{.ipfv-ptcp-dd.pl=mod}	\tsc{1sg-erg}\\
	\glt	\sqt{‎‎‎I have to give the money back.}
\end{exe}


The four TAM forms are analytic and make use of either the cross-categorical suffixes \tit{-ce} or \tit{-il}, which, among other things, are used for the formation of referential attributes that have the morphosyntactic properties of nominals (e.g. headless relative clauses). These constructions therefore resemble biclausal constructions, but a detailed investigation is needed before any conclusions can be made.



In sum, person agreement in Sanzhi is conditioned by person, by grammatical relations, and by TAM forms. Only agent, experiencer and patient arguments control person agreement, and the relevant hierarchy is 1, 2 > 3. In clauses with two speech act participants either argument can control agreement, even if it seems that there is a small preference for second person controllers because in elicitation speakers seem to accept them more readily. Variation in clauses with speech act participants is also found with respect to the stem augment vowels (\reftab{tab:Stem augment vowels for transitive and two-place affective verbs}), whereas in all other scenarios no variation is allowed.


Person agreement does not interact with polarity. However, the form of the verb and therefore the form of the agreement marker may change, e.g. in a copula clause with a first or second person subject and present time reference, the person enclitics given in \reftab{tab:Person agreement enclitics} are used; if the same clause is negated, the negated forms of the copula to which person suffixes are added occur \refex{ex:I am not hungry@24}. 
%
\begin{exe}
	\ex	\label{ex:I am not hungry@24}
	\gll	du	kːuš-le	akːʷa-di\\
		\tsc{1sg}	hungry\tsc{-advz}	\tsc{cop.neg-1}\\
	\glt	\sqt{I am not hungry.} (E)
\end{exe}

As was mentioned above, non-finite verb forms mostly cannot take person markers. For example, the adverbial clause in \refex{ex:(They) gave them (to me) and (I) went home@25} headed by a converb lacks an agreement marker.  Only the finite verb in the main clause shows person agreement.
%
\begin{exe}
	\ex	\label{ex:(They) gave them (to me) and (I) went home@25}
	\gll	[hel-tːi	d-ičː-ib-le]	qili	sa-ač'-ib=da\\
		that\tsc{-pl}	\tsc{npl-}give\tsc{.pfv-pret-cvb}	home	\tsc{hither}-come\tsc{.pfv-pret=1}\\
	\glt	\sqt{(They) gave them (to me) and (I) went home.}
\end{exe}

The only exceptions are certain complement clauses exhibiting control. They can be headed by an infinitive or alternatively by the subjunctive which has the suffix -\tit{Vtːaj} for the second person and \tit{-anaj}\slash\tit{-araj} for the third person (see \refsec{ssec:The subjunctive (agreeing infinitive)}). There is no suffix for the first person and instead the infinitive is used. Relevant examples are:
%
\begin{exe}
	\ex	\label{ex:We wanted to cry@26}
	\begin{xlist}
		\ex	\label{ex:We wanted to cry@26a}
		\gll	nišːij	b-ikː-ul=de	[d-isː-ij]\\
			\tsc{1pl.dat}	\tsc{n-}want\tsc{.ipfv-cvb=pst}	\tsc{1/2pl-}cry\tsc{-inf}\\
		\glt	\sqt{We wanted to cry.}

		\ex	\label{ex:You wanted to cry@26b}
		\gll	ašːij	b-ikː-ul=de	[d-isː-utːaj	/	d-isː-ij]\\
			\tsc{2pl.dat}	\tsc{n-}want\tsc{.ipfv-cvb=pst}	\tsc{1/2pl-}cry\tsc{-2subj}	/	\tsc{1/2pl-}cry\tsc{-inf}\\
		\glt	\sqt{You wanted to cry.}

		\ex	\label{ex:They wanted to cry@27c}
		\gll	il-tːa-j	b-ikː-ul=de	[b-isː-araj	/	b-isː-ij]\\
			that\tsc{-pl.obl-dat}		\tsc{n-}want\tsc{.ipfv-cvb=pst}	\tsc{hpl-}cry\tsc{-3subj}	/	\tsc{hpl-}cry\tsc{-inf}\\
		\glt	\sqt{They wanted to cry.}
	\end{xlist}
\end{exe}
