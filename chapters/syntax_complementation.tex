\chapter{Complementation}
\label{cpt:Complementation}

Complement clauses are subordinate clauses that function as arguments of verbs. Complement taking predicates can be divided into several semantic subgroups (\refsec{sec:Complement-taking predicates}). Complementation strategies vary according to these subgroups. However, more important for the choice of the formal marking is the semantics of the complement clause (e.g. potential vs. activity vs. fact type) as well as co-reference and control relations between the subject of the matrix predicate and the arguments in the complement clause. Therefore, I will start with a list of complement-taking predicates (\refsec{sec:Complement-taking predicates}). Then I will discuss the semantic types of complement clauses and how the formal strategies are distributed across the semantic types (\refsec{sec:Complementation strategies and their semantics}). Due to their overall frequency in the corpus, reported speech constructions will be treated separately in \refsec{sec:Reported speech constructions}, although they do not exhibit many peculiarities that would distinguish them from other complement constructions. In \refsec{sec:The syntactic properties of complement clauses}, I analyze the syntactic properties of complement constructions and in \refsec{sec:Argument control in complement constructions} I discuss in more detail complement control. 

The chapter closes with a short discussion of constructions that syntactically do not represent complementation, but semantically resemble complement constructions. To these constructions belong parentheticals (\refsec{ssec:Parentheticals}), nominalized relative clauses (\refsec{ssec:Nominalized relative clauses used with emotion and cognition predicates}), and adverbial clauses (\refsec{ssec:Nominalized relative clauses used with emotion and cognition predicates}).

I use square brackets throughout this chapter in order to indicate the complement. Note, however, that the complement is not always syntactically a clause, but can also be a nominalized verb form or an associated clause in case of parenthetical constructions.


%%%%%%%%%%%%%%%%%%%%%%%%%%%%%%%%%%%%%%%%%%%%%%%%%%%%%%%%%%%%%%%%%%%%%%%%%%%%%%%%

\section{Complement-taking predicates}
\label{sec:Complement-taking predicates}

Not all predicates listed in this section are complement-taking predicates in the strict sense that they always require a clausal complement (as an alternative to a nominal argument) in order to build a full grammatical sentence. Copula constructions with adverbs are by themselves independent main clauses, but the clauses that can be added to them formally behave like genuine complement clauses and are therefore included.


% --------------------------------------------------------------------------------------------------------------------------------------------------------------------------------------------------------------------- %

\subsection{Utterance verbs}
\label{ssec:Utterance verbs}

Sanzhi has the following simple verbs of speech:
%
\begin{exe}
	\ex	\label{ex:simple verbs of speech}
	\begin{xlist}
		\ex	\tit{b-ik'ʷ-} (\tsc{ipfv}) \sqt{say, think}
		\ex	\tit{haʔ-} (\tsc{pfv})\slash\tit{herʔ-} (\tsc{ipfv}) \sqt{say}
		\ex	\tit{b-urs-} \sqt{tell}
		\ex	\tit{b-ux-} (\tsc{ipfv}) \sqt{tell}
	\end{xlist}
\end{exe}

More specific verbs are compounds consisting of a first part that can be a noun, an ideophone, or a bound stem, and a following light verb. There are especially many compounds with the noun \tit{ʁaj} \sqt{word, language, talk} (see \refsec{ssec:compoundswithnouns} for more examples); and the simple verbs of speech listed above also occur frequently together with \tit{ʁaj}. Examples include:
%
\begin{exe}
	\ex	\label{ex:compound verbs of speak}
	\begin{xlist}
		\ex	\tit{xar b-eʁ-} (\tsc{pfv})\slash\tit{xar b-irʁ-} (\tsc{ipfv}) \sqt{ask}			
		\ex	\tit{ʁaj (ka-)b-ik'ʷ-} (\tsc{ipfv}) \sqt{quarrel, scold, argue, discuss, talk}
		\ex	\tit{ʁaj b-arq'-} (\tsc{pfv})\slash\tit{ʁaj b-irq'-} (\tsc{ipfv}) \sqt{say, tell}
		\ex	\tit{ʁaj b-ikː-} (\tsc{pfv})\slash\tit{ʁaj lukː-} (\tsc{ipfv}) \sqt{promise}
		\ex	\tit{čːal b-uq-} (\tsc{pfv})\slash\tit{čːal b-ulq-} (\tsc{ipfv}) \sqt{argue, quarrel}
		\ex	\tit{anru b-ikː-} (\tsc{pfv})\slash\tit{anru lukː-} (\tsc{ipfv}) \sqt{command, order}
		\ex	\tit{bursːi b-arq'-} (\tsc{pfv})\slash\tit{bursːi b-irq'-} (\tsc{ipfv}) \sqt{teach}
		\ex	\tit{waˁw b-ik'ʷ- (\tsc{ipfv}); waˁw haʔ-} (\tsc{pfv})\slash\tit{waˁw herʔ-} (\tsc{ipfv}) \sqt{call, cry}
		\ex	\tit{ʁaˁʁ b-ik'ʷ-} (\tsc{ipfv}) \sqt{shout}				
		\ex	\tit{t'irt'ir b-ik'ʷ-} (\tsc{ipfv}) \sqt{chat}
	\end{xlist}
\end{exe}

Not all utterance verbs take complement clauses that are reported speech. Some rather denote actions that involve speech (e.g. \sqt{teach}, \sqt{command, order}) or they denote specific sounds that imitate speech sounds (e.g. \tit{t'irt'ir b-ik'ʷ-} \sqt{chat}). In \refsec{sec:Reported speech constructions}, I will only discuss constructions containing quotes.


% --------------------------------------------------------------------------------------------------------------------------------------------------------------------------------------------------------------------- %

\subsection{Liking and fearing verbs and other verbs denoting emotions and volition}
\label{ssec:Liking and fearing verbs and other verbs denoting emotions and volition}

To this group the following verbs belong: 
%
\begin{exe}
	\ex	\label{ex:verbs of liking of fearing}
	\begin{xlist}
		\ex	\tit{b-ikː-} \sqt{want, like} 
		\ex	\tit{b-ičː-aq-} \sqt{want, like, love} (the causativized variant of \tit{b-ikː-})
		\ex	\tit{razi b-iχʷ-} (\tsc{pfv})\slash\tit{razi b-irχʷ-} (\tsc{ipfv}) \sqt{be happy, agree}
		\ex	\tit{xul b-ik'ʷ-} (\tsc{ipfv}) \sqt{dream, wish, hope}
		\ex	\tit{urk' b-uq-} (\tsc{pfv})\slash\tit{urk' b-ulq-} \sqt{be frightened, astonished, wonder}
		\ex	\tit{tamaša b-arq'-} (\tsc{pfv})\slash\tit{tamaša b-irq'-} (\tsc{ipfv}) \sqt{wonder}
		\ex	\tit{uruχ b-ik'ʷ-} (\tsc{ipfv}) \sqt{get afraid}, \tit{uruχle ca-b} \sqt{be afraid}
		\ex	\tit{uruc b-iχʷ-} (\tsc{pfv})\slash\tit{uruc b-irχʷ-} (\tsc{ipfv}) \sqt{get embarrassed, ashamed}, \tit{uruc ca-b} \sqt{be embarrassed, ashamed}
		\ex	\tit{c'aχ ka-b-icː-} (\tsc{pfv})\slash\tit{c'aχ ka-b-ircː-} (\tsc{ipfv}) \sqt{be embarrassed, ashamed}
		\ex	\tit{pašman b-iχʷ-} (\tsc{pfv})\slash  \tit{pašman b-irχʷ-} (\tsc{ipfv}) \sqt{be sad about, regret}
		\ex	\tit{q'as b-arq'-} (\tsc{pfv})\slash\tit{q'as b-irq'-} (\tsc{ipfv}) \sqt{decide}
		\ex	\tit{sa-b-ag-} (\tsc{pfv})\slash\tit{sa-b-arg-} (\tsc{ipfv}) \sqt{imagine, envisage, see}
	\end{xlist}
\end{exe}

Some of these verbs denote emotions that are cognitively based feelings and that are semantically close to verbs of cognition. Other verbs denoting volition have some semantic overlap with modality.


% --------------------------------------------------------------------------------------------------------------------------------------------------------------------------------------------------------------------- %

\subsection{Cognition predicates}
\label{ssec:Cognition predicates}

Based on their semantics, I will divide the cognition predicates that take complements into three groups:
%
\begin{enumerate}
	\item	verbs of knowledge and acquisition of knowledge
	\item	propositional attitude predicates
	\item	other cognition predicates
\end{enumerate}


% - - - - - - - - - - - - - - - - - - - - - - - - - - - - - - - - - - - - - - - - - - - - - - - - - - - - - - - - - - - - - - - - - - - - - - - - - - - - - - - - - - - - - - - - - - - - - - - - - - - - - - - - - - - - - - - - - - - - - - - - - - %

\subsubsection{Verbs of knowledge and acquisition of knowledge}
\label{sssec:Verbs of knowledge and acquisition of knowledge}

Verbs expressing knowing and the acquisition of knowledge include:
%
\begin{exe}
	\ex	\label{ex:verbs of knowledge}
	\begin{xlist}
		\ex	\tit{b-aχ-} (\tsc{pfv})\slash\tit{b-alχ-} (\tsc{ipfv}) \sqt{get to know, know}
		\ex	\tit{arʁ-} (\tsc{pfv})\slash\tit{irʁ-} (\tsc{ipfv}) \sqt{understand} (can be used together with \tit{urk'i} \sqt{heart})
		\ex	\tit{b-elč'-} (\tsc{pfv})\slash\tit{b-uč'-} (\tsc{ipfv}) \sqt{read} 
	\end{xlist}
\end{exe}

In addition, there is a particle \tit{aχːu} \sqt{I don't know, dunno} that takes complement clauses \refex{ex:I don't know if he is remembering}. It most probably goes back to the verb \sqt{know}. This particle can be used with the first person singular dative pronoun \textit{dam}, but not with any other dative experiencer and not with nominal stimuli, which indicates its status as a particle (in contrast to the full verb, which can be used with arguments of all persons and numbers and also with nominal stimuli arguments). Furthermore, it is also used parenthetically (\refsec{ssec:Parentheticals}).


% - - - - - - - - - - - - - - - - - - - - - - - - - - - - - - - - - - - - - - - - - - - - - - - - - - - - - - - - - - - - - - - - - - - - - - - - - - - - - - - - - - - - - - - - - - - - - - - - - - - - - - - - - - - - - - - - - - - - - - - - - - %

\subsubsection{Propositional attitude predicates}
\label{sssec:Propositional attitude predicates}

These predicates express a kind of propositional attitude toward the truth of the complement.
%
\begin{exe}
	\ex	\label{ex:propositional attitude predicates}
	\begin{xlist}
		\ex	\tit{pikri b-ik'ʷ-} (\tsc{ipfv}), \tit{pikri b-uq-} (\tsc{pfv}), \tit{prikri b-arq'-} (\tsc{pfv}) \sqt{think, worry, give thoughts to}
		\ex	\tit{b-iχː-} \sqt{believe}
		\ex	\tit{b-iχ-(b)-it-ag-} (\tsc{pfv})\slash\tit{b-iχ-(b)-it-arg-} (\tsc{ipfv}) \sqt{believe} (compound verb, containing \textit{b-iχː-} \sqt{believe})
		\ex	\tit{b-iχ-či ag-} (\tsc{pfv})\slash\tit{b-iχ-či arg-} (\tsc{ipfv}) \sqt{believe} (compound verb, containing \textit{b-iχː-} \sqt{believe})
		\ex	\tit{šak b-ik-} (\tsc{pfv})\slash\tit{šak b-irk-} (\tsc{ipfv}) \sqt{guess, suspect, feel}
	\end{xlist}
\end{exe}

In addition, there is a phrase \tit{dila pikri ħaˁsible} (\tsc{1sg.gen} thought following) \sqt{in my mind} that also expresses a propositional attitude, but syntactically represents a parenthetical, not a complement-taking predicate \refex{ex:‎In my mind, he left prison}, \refex{ex:His wife is, in my mind bad like a dog@15}.


% - - - - - - - - - - - - - - - - - - - - - - - - - - - - - - - - - - - - - - - - - - - - - - - - - - - - - - - - - - - - - - - - - - - - - - - - - - - - - - - - - - - - - - - - - - - - - - - - - - - - - - - - - - - - - - - - - - - - - - - - - - %

\subsubsection{Other cognition predicates}
\label{sssec:Other cognition predicates}

These predicates are achievement verbs for positive (e.g. \sqt{remember}) and negative achievement (e.g. \sqt{forget}) in the domain of cognition.
%
\begin{exe}
	\ex	\label{ex:other cognition predicates}
	\begin{xlist}
		\ex	\tit{qum-ert-} (\tsc{pfv})\slash\tit{qum-urt-} (\tsc{ipfv}) \sqt{forget}
		\ex	\tit{han b-ik-}, \tit{b-ičaq-}  (\tsc{pfv})\slash\tit{han b-irk-}, \tit{b-irčaq-} (\tsc{ipfv}) \sqt{remember, seem to, imagine, think} 
		\ex	\tit{han b-el, han ca-b} \sqt{remember} 
		\ex	\tit{urk'i-le sa-b-eʁ-} (\tsc{pfv})\slash\tit{urk'i-le sa-b-irʁ-} (\tsc{ipfv}) \sqt{descend on the heart, remember} 
		\ex	\tit{urk'i-le-b le-b} \sqt{think, have thoughts} (lit. `be on the heart') 
		\ex	\tit{ħaˁsib b-arq'-} (\tsc{pfv})\slash\tit{ħaˁsib b-irq'-} \sqt{test, check}
	\end{xlist}
\end{exe}


% --------------------------------------------------------------------------------------------------------------------------------------------------------------------------------------------------------------------- %

\subsection{Manipulative verbs}
\label{ssec:Manipulative verbs}

Manipulative verbs typically have subjects that differ from the subjects in the complement clause and the semantics of the complement clause is irrealis. To this group belong utterance verbs such as \sqt{command, order} and \sqt{teach} and the basic verb of speech \tit{b-ik'ʷ-} (\tsc{ipfv}) \sqt{say} that is frequently used with a manipulative meaning. Other manipulative verbs are:
%
\begin{exe}
	\ex	\label{ex:manipulative verbs}
	\begin{xlist}
		\ex	\tit{b-at aʁ-} (\tsc{pfv}) \sqt{send}
		\ex	\tit{b-at-} (\tsc{pfv})\slash\tit{b-alt-} (\tsc{ipfv}) \sqt{let, leave}
		\ex	\tit{iχtijar b-ikː-} (\tsc{pfv})\slash\tit{iχtijar lukː-} (\tsc{ipfv}) \sqt{give permission, right}
		\ex	\tit{χajri a-b-iχʷ-} (\tsc{pfv})\slash\tit{χajri a-b-irχʷ-} (\tsc{ipfv}) \sqt{forbid}
	\end{xlist}
\end{exe}


% --------------------------------------------------------------------------------------------------------------------------------------------------------------------------------------------------------------------- %

\subsection{Phasal verbs}
\label{ssec:Phasal verbs}

Sanzhi phasal verbs include:
%
\begin{exe}
	\ex	\label{ex:phasal verbs}
	\begin{xlist}
		\ex	\tit{b-aʔ.ak'-} (\tsc{pfv})\slash\tit{b-aʔ.ik'-} (\tsc{ipfv}) \sqt{begin, start}
		\ex \tit{b-aʔ b-axː-} (\tsc{pfv})\slash\tit{b-aʔ b-irxː-} (\tsc{ipfv}) \sqt{begin, start}
		\ex	\tit{či-ka-b-iħ-} (\tsc{pfv}) \tit{či-ka-b-irħ-} (\tsc{ipfv}) \sqt{begin, start}
		\ex	\tit{taman b-iχʷ-} (\tsc{pfv})\slash\tit{taman b-irχʷ-} (\tsc{ipfv}), \tit{taman b-arq'-} (\tsc{pfv})\slash\tit{taman b-irq'-} (\tsc{ipfv}) \sqt{stop, finish}
	\end{xlist}
\end{exe}

There are two verbs that can express the meaning `continue', \tit{kelgʷ-} (\tsc{pfv}) \sqt{remain, stay, be} and the defective verb \tit{b-el} \sqt{remain, stay}. Both are used in periphrastic verb forms, which are not complement constructions (see \refsec{sec:Verb forms with kelgw- remain} and \refsec{sec:Verb forms with b-el remain, stay}).

% --------------------------------------------------------------------------------------------------------------------------------------------------------------------------------------------------------------------- %

\subsection{Modal predicates}
\label{ssec:Modal predicates}

Modality comprises epistemic modality (likelihood, certainty), deontic modality (necessity, obligation, permission), and ability. In Sanzhi, modality is typically not expressed through modal verbs such as English \tit{must}, \tit{should,} or \tit{may}, but by means of analytic and periphrastic verb forms. The obligative tenses denote obligation in addition to future (\refsec{ssec:Obligative}--\refsec{ssec:Obligative past}). There are a number of periphrastic epistemic modal constructions that express likelihood and certainty (\refsec{ssec:Epistemic modal constructions}, \refsec{sec:Epistemic modality with the auxiliary b-urk find}). In addition, it is possible to use the embedded question marker together with the optional adverb \tit{belki} \sqt{be possible} (\refsec{sec:Predicative particles}). The only complement-taking predicates that express modality convey the meaning of ability or necessity:
%
\begin{exe}
	\ex	\label{ex:modal predicates}
	\begin{xlist}
		\ex	\tit{b-iχʷ-} (\tsc{pfv})\slash\tit{b-irχʷ-} (\tsc{ipfv}) \sqt{can, be able}
		\ex	\tit{ʡaˁʁunil ca-b} \sqt{be needed, necessary}
		\ex	\tit{ħaˁžatle ca-b} \sqt{be needed, necessary}
	\end{xlist}
\end{exe}


% --------------------------------------------------------------------------------------------------------------------------------------------------------------------------------------------------------------------- %

\subsection{Evaluation}
\label{ssec:Evaluation}

Evaluation is expressed by adverbs together with a copula or the verb \tit{ag-} (\tsc{pfv}) \sqt{go}:
%
\begin{exe}
	\ex	\label{ex:evaluation verbs}
	\begin{xlist}
		\ex	\tit{ʡaˁħle ca-b}, \textit{ʡaˁħle ag-} (\tsc{pfv})\slash\textit{ʡaˁħle arg-} (\tsc{ipfv}) \sqt{be good}
		\ex	\tit{wahil ca-b}, \textit{wahil ag-} (\tsc{pfv})\slash\textit{wahil arg-} (\tsc{ipfv}) \sqt{be bad}
	\end{xlist}
\end{exe}


%%%%%%%%%%%%%%%%%%%%%%%%%%%%%%%%%%%%%%%%%%%%%%%%%%%%%%%%%%%%%%%%%%%%%%%%%%%%%%%%

\section{Complementation strategies and their semantics}
\label{sec:Complementation strategies and their semantics}

The following complementation strategies are available in Sanzhi and will be treated in this section.
%
\begin{enumerate}
	\item	major complementation strategies:
	%
	\begin{itemize}
		\item	zero strategy (\refsec{ssec:The zero strategy})
		\item	quotative particles \tit{bik'ul, haʔible} (\refsec{ssec:The quotative particles})
		\item	cross-categorical suffixe \tit{-ce} (\refsec{ssec:The attributive marker -ce (-te)COMPL})
		\item	masdar \tit{-ni} (\refsec{ssec:The masdarComplementation})
		\item	perfective converb \tit{-le} (\refsec{ssec:The preterite converb})
		\item	infinitive \tit{-ij}\slash subjunctive (\refsec{ssec:Infinitive and subjunctive})
		\item	embedded question marker \tit{=el} (\refsec{ssec:The embedded question marker})
	\end{itemize}

	\item	minor complementation strategies:
	%
	\begin{itemize}
		\item	imperfective converb \tit{-ul\slash -unne} (\refsec{ssec:The imperfective converb})
		\item	case marking of participles or masdar (\refsec{ssec:Case marking and postpositions})
		\item the \textit{pretend}-construction (\refsec{ssec:PretendConstruction})
	\end{itemize}
\end{enumerate}

A number of the grammatical markers listen above also occur in other types of subordinate clauses: the perfective and the imperfective converb head adverbial clauses (\refsec{sec:The syntax of adverbial clauses}). The cross-categorical suffix \tit{-ce} occurs in relative clauses (\refcpt{cpt:Relative clauses}). Postpositions and case marking are otherwise used to mark nominal arguments and adjuncts in simple clauses (\refcpt{cpt:postpositions}). Therefore, it is not always easy to tell apart complement constructions from adverbial or relative clauses.

Based on their semantics, we can distinguish four types of complement clauses (\citealp[130]{Hengeveld1989}; \citealp[93]{Dik1997}; \citealp{Dixon2006}):
%
\begin{description}
	\item[potential type:]  refers to the potentiality of the subject of the complement clause becoming involved in an activity
	\item[activity type:] refers to some ongoing activity, relating to its extension in time
	\item[fact type:] refers to the fact that something took place
	\item[speech act type or direct speech type:] refers to a particular speech act
\end{description}

For the linguistic encoding of each semantic type one or more complementation strategies are used (\reftab{tab:Complementation strategies and the semantic types of complements}).  The strategies will be discussed in detail in the following sections.
%
\begin{table}
	\caption{Complementation strategies and the semantic types of complements}
	\label{tab:Complementation strategies and the semantic types of complements}
	\small
	\begin{tabularx}{0.70\textwidth}[]{%
		>{\raggedright\arraybackslash}X
		>{\centering\arraybackslash}p{10pt}
		>{\centering\arraybackslash}p{10pt}
		>{\centering\arraybackslash}p{10pt}
		>{\centering\arraybackslash}p{10pt}}
		
		\lsptoprule
			{}
		&	\rotatebox{90}{potential}
		&	\rotatebox{90}{activity}
		&	\rotatebox{90}{fact}
		&	\rotatebox{90}{speech act~}\\
		\midrule
			zero					&	{}		&	{}		&	{}		&	y\\	   
			quotative particle			&	(y)		&	{}		&	y		&	y\\	   
			attributive 			&	{}		&	y 		&	y		&	y\\	   
			masdar				&	{}		&	{}		&	y		&	{}\\
			perfective converb			&	{}		&	y		&	y		&	{}\\
			infinitive\slash subjunctive		&	y		&	{}		&	{}		&	{}\\				   
			embedded question marker	&	y		&	y		&	{}		&	y\\	   
			imperfective converb		&	y		&	y		&	?	&	{}\\		   
			case marking and postpositions	&	{}		&	y		&	{}		&	{}\\
		\lspbottomrule
	\end{tabularx}
\end{table}



% --------------------------------------------------------------------------------------------------------------------------------------------------------------------------------------------------------------------- %

\subsection{The zero strategy}
\label{ssec:The zero strategy}

No formal marking of the complement clause is a major strategy in reported speech constructions (\refsec{sec:Reported speech constructions}). It is also found, though less commonly, with those emotion and cognition verbs that denote activities that presuppose or imply speech and that have therefore a linguistic component that makes them similar in their behavior to verbs of speech \xxref{ex:‎I also thought, I should go to make a little money}{ex:‎It is called something in Russian, I forgot}. The zero strategy can be viewed as an alternative to the use of the quotative particles since with every verb that allows for the absence of formal encoding of its complement the employment of quotative particles is possible.
%
\begin{exe}
	\ex	\label{ex:‎I also thought, I should go to make a little money}
	\gll	na	dam=ra	han	b-ič-ib	[w-ax-an=da	q'ʷila	arc	d-irq'-an=da]\\
		now	\tsc{1sg.dat=add}	remember	\tsc{n-}occur\tsc{.pfv-pret}	\tsc{m-}go\tsc{-ptcp=1}	a.little	money	\tsc{npl-}do\tsc{.ipfv-ptcp=1}\\
	\glt	\sqt{‎I also thought, I should go to make a little money.}

	\ex	\label{ex:‎He thinks, I need to take these clothes}
	\gll	pikri	Ø-ik'-ul	ca-w	[[hej	paltar	asː-ij]	ʡaˁʁuni-l	ca-d]\\
		thought	\tsc{m-}say\tsc{.ipfv-icvb}	\tsc{cop-m}	this	clothes	take\tsc{.pfv-inf}	needed\tsc{-advz}	\tsc{cop-npl}\\
	\glt	\sqt{‎He thinks, I need to take these clothes.}

		\ex	\label{ex:‎‎‎The poor boy got frightened}
	\gll	urk'	uq-un	il	miskin	[kːurtːa-j	χalq'-la	ʁaj	d-alχ-ul=ew	ce=ja?]\\
		fright	go\tsc{.m.pfv-pret}	that	poor	fox\tsc{-dat}	people\tsc{-gen}	word(\tsc{npl})	\tsc{npl-}know\tsc{.ipfv-icvb=q}	what\tsc{=q}\\
	\glt	\sqt{‎‎‎The poor boy got frightened: Does the fox know the human language or what?}
	
	\ex	\label{ex:‎It is called something in Russian, I forgot}
	\gll	[ʡuˁrus	ʁaj-la	ce=jal	b-ik'-ul	ca-b	it-i-j]	dam	qum.urt-ul	ca-b\\
		Russian	language\tsc{-gen}	what\tsc{=indef}	\tsc{hpl-}say\tsc{.ipfv-icvb}	\tsc{cop-hpl}	that\tsc{-obl-dat}	\tsc{1sg.dat}	forget\tsc{.ipfv}-\tsc{icvb}	\tsc{cop-n}\\
	\glt	\sqt{‎It (i.e. the plant) is called something in Russian, I forgot.}
\end{exe}

Due to the absence of any formal marking, it is alternatively possible to analyze the above examples as juxtaposition of two main clauses without a syntactic link between them, but with a clear semantic relationship, which follows from the meaning of the emotion and cognition verbs and the interpretation of the clauses in brackets as expressing thoughts. In example \refex{ex:‎It is called something in Russian, I forgot} still another approach suggests itself, namely the analysis of the cognitive predicate as parenthetical, which means that this is not a complement construction, but simply an independent sentence followed by another independent sentence that makes a comment on the previous one and functions as a kind of stance marker to inform the hearer that the speaker is unsure about the validity of some of her utterances about plant names. See \refsec{ssec:Parentheticals} for more information about parentheticals.



% --------------------------------------------------------------------------------------------------------------------------------------------------------------------------------------------------------------------- %

\subsection{The quotative particles}
\label{ssec:The quotative particles}

Sanzhi has two quotative particles, \tit{bik'ul} and \tit{haʔible}, which are discussed in more detail in \refsec{sec:Reported speech constructions} in reference to reported speech, because they constitute a major encoding strategy for complements of utterance verbs. Although \tit{bik'ul} and \tit{haʔible} have preserved their verbal properties (e.g. inflectional morphology, gender agreement, position) I will refer to them as `particles' when they occur in addition to matrix verbs of speech in order to differentiate them from the matrix verbs that take complement clauses. They can therefore be called `parentheticals' that do not realize syntactic subordination but pragmatically mark a clause as a speech report.

The particle \tit{haʔible} is far less common than \tit{bik'ul}, and occurs only with verbs of speech (including cases in which they are used as verbs of cognition) and occasionally in purposive clauses \refex{ex:‎‎‎‎I will become a dentist, he probably thinks, I do not know.}. The particle \tit{bik'ul}, in contrast, occurs also in complements of emotion (`be afraid' \refex{ex:She is afraid that he will then beat her up}, `dream' \refex{ex:I dream to win a car in the lottery}) and cognition verbs that denote activities that heavily rely on the (implicit) use of language, most notably verbs meaning \sqt{think} \refex{ex:‎‎‎I thought that he did not write}, \refex{ex:‎The father thinks about whether these clothes are to be washed.}. Verbs of knowledge normally do not mark complement clauses with the quotative particle.
%
\begin{exe}
	\ex	\label{ex:She is afraid that he will then beat her up}
	\gll	[c'il	ca-r	it-an=da	r-ik'-ul]	uruχ-le	ca-r	ik'\\
		then	\tsc{refl-f}	beat.up\tsc{-ptcp=1}	\tsc{f-}say\tsc{.ipfv-icvb}	fear\tsc{-advz}	\tsc{cop-f}	\tsc{dem.up}\\
	\glt	\sqt{She is afraid that he will then beat her up.}

	\ex	\label{ex:I dream to win a car in the lottery}
	\gll	du xul Ø-ik'-ul=da [dam mašin b-irk-an-ne Ø-ik'-ul latereja-le-b]\\
		\tsc{1sg} wish	\tsc{m-}say\tsc{.ipfv-icvb=1}	\tsc{1sg.dat}	car	\tsc{n-}occur\tsc{.icvb-ptcp-fut.3} \tsc{m-}say\tsc{.ipfv-icvb} lottery\tsc{-loc-n}\\
	\glt	\sqt{I dream of winning a car in the lottery.} (E)
\end{exe}

The particle agrees in gender with the subject or subject-like argument of the matrix clause even in those cases in which the matrix predicate takes other cases than the absolutive. For instance, in \refex{ex:‎‎‎I thought that he did not write} the experiencer in the matrix clause is marked by the dative, and the predicate that governs this argument shows local agreement with the complement clause (see \refsec{sec:The syntactic properties of complement clauses} below on the difference between local and long-distance agreement in complement clauses). By contrast, the quotative shows feminine singular agreement because the experiencer has a female referent. Example \refex{ex:‎The father thinks about whether these clothes are to be washed.} shows that even a possessor functioning as experiencer can control gender agreement on the quotative particle. The obvious reason for this behavior is the valency pattern of the verb \tit{b-ik'ʷ-} \sqt{say}, from which the particle originates. It requires an absolutive argument controlling its gender agreement prefix (in addition to the complement clause).
%
\begin{exe}
	\ex	\label{ex:‎‎‎I thought that he did not write}
	\gll	dam	han	b-ič-ib	[a-b-elk'-un-ne	r-ik'-ul]\\
		\tsc{1sg.dat}	seem	\tsc{n-}occur\tsc{.pfv-pret}	\tsc{neg-n-}write\tsc{.pfv-pret-cvb}	\tsc{f-}say\tsc{.ipfv-icvb}\\
	\glt	\sqt{‎‎‎I (fem.) thought that he did not write.}

	\ex	\label{ex:‎The father thinks about whether these clothes are to be washed.}
	\gll	atːa-la	pikri	k'e-b	[hex-tːi	paltar	ic-an-te=jal	Ø-ik'-ul]\\
		father\tsc{-gen}	thought	exist.\tsc{up-n}	\tsc{dem.up}\tsc{-pl}	clothes	wash\tsc{.ipfv-ptcp-dd.pl=indq}	\tsc{m-}say\tsc{.ipfv-icvb}\\
	\glt	\sqt{‎The father thinks about whether these clothes are to be washed.}

	\ex	\label{ex:He also guessed that the people had probably said the truth}
	\gll	il=ra	šak	Ø-ič-ib	ca-w	[itːi	χalq'	b-ik'ʷ-an	mar	b-urkː-ar	Ø-ik'-ul]\\
		that\tsc{=add}	feel	\tsc{m-}occur\tsc{.pfv-pret}	\tsc{cop-m}	\tsc{3pl}	people	\tsc{hpl-}say\tsc{.ipfv-ptcp}	truth	\tsc{n-}find\tsc{.ipfv-prs.3}	\tsc{m-}say\tsc{.ipfv-icvb}\\
	\glt	\sqt{He also guessed that the people had probably said the truth.}
\end{exe}

It might be combined with the modal interrogative suffix (see \refsec{sec:modalinterrogative} for more information) in the complement clause if the complement represents a question which has deontic modality and in which the subject is co-referential with the author of the quote \refex{ex:(He) is probably thinking, Should I go stealing or should I not go?}.
%
\begin{exe}
	\ex	\label{ex:(He) is probably thinking, Should I go stealing or should I not go?}
	\gll	[d-iʡ-ij	uq'-ide=l	a-w-q'-idel	Ø-ik'-ul]		pikri	Ø-ik'-ul=el\\
		\tsc{npl-}steal\tsc{.pfv-inf}	go\tsc{.m.pfv-modq=indq}	\tsc{neg-m-}go\tsc{.pfv-modq}	\tsc{m-}say\tsc{.ipfv-icvb}	thought	\tsc{m-}say\tsc{.ipfv-icvb=indq}\\
	\glt	\sqt{(He) is probably thinking, Should I go stealing or should I not go?}
\end{exe}


In \refex{ex:‎‎‎‎I will become a dentist, he probably thinks, I do not know.} the particle \tit{haʔible} is followed by the verb 
\textit{b-ik'ʷ-} used with the meaning `think'. 

\begin{exe}
	\ex	\label{ex:‎‎‎‎I will become a dentist, he probably thinks, I do not know.}
	\gll	[cul-b-a-la	tuχtur	arg-an=da]	haʔ-ib-le	ik'-ul	a-urkː-ar	aχːu\\
		tooth-\tsc{pl-obl-gen}	doctor	go.\tsc{ipfv-ptcp=1}	say\tsc{.pfv-pret-cvb}	\tsc{neg}-find.\tsc{ipfv-prs.3}	not.know\\
	\glt	\sqt{‎‎‎‎I will become a dentist, he is probably thinking, I do not know.}
\end{exe}

	


% --------------------------------------------------------------------------------------------------------------------------------------------------------------------------------------------------------------------- %

\subsection{The cross-categorical suffix \protect\tit{-ce}}
\label{ssec:The attributive marker -ce (-te)COMPL}

In complement clauses, the cross-categorical suffix -\textit{ce} is added to the preterite participle \xxref{ex:He got to know that (they) ate (them)}{ex:‎‎‎I am happy that you came} or to the modal participle \refex{‎‎‎I am happy that during the next year we will finish}. This suffix is used for the formation of definite descriptions (see \refsec{ssec:The -ce / -te attributive} for detailed accounts of its complex array of functions). When it is suffixed to verbs the verbs take over the function of attributes (i.e. as participles in relative clauses) or as nominalized referentially independent clauses. The latter function is relevant for the occurrence of -\textit{ce} in complement clauses of the fact type. More specifically, the proposition in the complement clause, which is marked by -\textit{ce}, is considered to be true and treated as a fact by the speaker. Therefore, only certain verbs denoting emotions, cognition verbs, as well as evaluative predicates express their complement clauses with the attributive marker. 
%
\begin{exe}
	\ex	\label{ex:He got to know that (they) ate (them)}
	\gll	iž-i-l	b-aχ-ur	ca-b	[d-erk-un-ce]\\
		this\tsc{-obl-erg}	\tsc{n-}know\tsc{.pfv-pret}	\tsc{cop-n}	\tsc{npl-}eat\tsc{.pfv-pret-dd.sg}\\
	\glt	\sqt{He got to know that (they) ate (them).}

	\ex	\label{ex:‎‎She suspected that the wolf had eaten (the sisters)}
	\gll	il	šak	r-ič-ib	ca-r	[bec'-li	b-erkː-un-ce]\\
		that	feel	\tsc{f-}occur\tsc{.pfv-pret}	\tsc{cop-f}	wolf\tsc{-erg}	\tsc{hpl-}eat\tsc{.pfv-pret-dd.sg}\\
	\glt	\sqt{‎‎She suspected that the wolf had eaten (the sisters).}

	\ex	\label{ex:‎‎‎I am happy that you came}
	\gll	du	razi-l=da	[u	sa-r-eʁ-ib-le	/	sa-r-eʁ-ib-ce]\\
		\tsc{1sg}	happy\tsc{-advz=1}	\tsc{2sg}	\tsc{hither}\tsc{-f-}go\tsc{.pfv-pret-cvb}	/	\tsc{hither-f-}go\tsc{.pfv-pret-dd.sg}\\
	\glt	\sqt{‎‎‎I am happy that you came.} (E)
\end{exe}

The meaning of these complement clauses is very close to complement clauses formed with the masdar \refex{ex:It was bad that Rashid did not repair the chair} (\refsec{ssec:The masdarComplementation}) and with the perfective converb \refex{ex:‎‎‎I am happy that you came} (\refsec{ssec:The preterite converb}) and the strategies can usually be replaced by each other.

\begin{exe}
	\ex	\label{ex:It was bad that Rashid did not repair the chair}
	\gll	wahi-l=de [Rašid-li ust'ul ʡaˁħ a-b-arq'-ib-le\slash a-b-arq'-ib-ce\slash a-b-arq'-ni]\\
		bad\tsc{-adv=pst} Rashid\tsc{-erg} chair good \tsc{neg-n-}do\tsc{.pfv-pret-cvb}\slash\tsc{neg-n-}do\tsc{.pfv-pret-dd.sg}\slash\tsc{neg-n-}do\tsc{.pfv-msd}\\
	\glt	\sqt{It was bad that Rashid did not repair the chair.} (E)
\end{exe}

Note that the suffix -\textit{ce} also occurs in nominalized relative clauses that semantically resemble complement clauses of the activity type. These constructions are discussed in \refsec{Other strategies}.

% --------------------------------------------------------------------------------------------------------------------------------------------------------------------------------------------------------------------- %

\subsection{The masdar}
\label{ssec:The masdarComplementation}

The masdar is a deverbal noun that is used not only in complement clauses, but also in other argument and adjunct positions (\refsec{ssec:The masdar}). Complements expressed by the absolutive (i.e. not further case-marked) masdar denote facts. Therefore, basically the same matrix predicates that make use of the cross-categorical suffix -\textit{ce} also allow for the masdar, i.e., cognition predicates, evaluation predicates and emotion predicates if they refer to factual complements that have propositional meaning:
%
\begin{exe}
	\ex	\label{ex:‎‎I know that this will be a sin}
	\gll	[bunah	b-irχʷ-ni]	b-alχ-ul=da\\
		sin	\tsc{n-}become\tsc{.ipfv-msd}	\tsc{n-}know\tsc{.ipfv-icvb=1}\\
	\glt	\sqt{‎‎I know that this will be a sin.}

	\ex	\label{ex:‎‎‎Then the wind understood that he would not be able to take off the coat of this person}
	\gll	heba	č'an-ni	arʁ-ib	[[il	admi-la	walžaʁ	či-r-sa-b-ertː-ij]	a-b-irχʷ-ni]\\
		then	wind\tsc{-erg}	understand\tsc{.pfv-pret}	that	person\tsc{-gen}	coat	\tsc{spr-abl-}\tsc{hither-n-}take\tsc{.pfv-inf}	\tsc{neg-n-}be.able\tsc{.ipfv-msd}\\
	\glt	\sqt{‎‎‎Then the wind understood that he would not be able to take off the coat of this person.}

	\ex	\label{ex:‎‎‎I don't remember that she was Mahammad's wife}
	\gll	[it	Maˁħaˁmmad-la	xːunul	r-iχʷ-ni]	dam	han	b-akːu\\
		that	Mahammad\tsc{-gen}	woman	\tsc{f-}be\tsc{.pfv-msd}	\tsc{1sg.dat}	remember	\tsc{n-}\tsc{cop.neg}\\
	\glt	\sqt{‎‎‎I don't remember that she was Mahammad's wife.}
\end{exe}

As mentioned above and shown in the elicited example \refex{ex:It was bad that Rashid did not repair the chair} shown in the previous section, the masdar is often semantically equivalent to the preterite participle and the cross-categorical suffix.
%



% --------------------------------------------------------------------------------------------------------------------------------------------------------------------------------------------------------------------- %

\subsection{The perfective converb}
\label{ssec:The preterite converb}

The perfective converb is used to form the same types of complement clauses as the cross-categorical suffix -\textit{ce}, that is, fact complements. Thus, in elicitation it is given as an alternative to \tit{-ce} (\refsec{ssec:The attributive marker -ce (-te)COMPL}), and the types of matrix predicates with which it occurs are the same as for that suffix, namely cognition verbs \refex{ex:‎I remember that he came riding on a horse} and emotion verbs \refex{ex:‎‎‎I am happy that you came preterite converb}, and evaluation predicates \refex{ex:‎‎‎He said that it is not good that he did not go to the meeting}, \refex{ex:It was bad that Rashid did not repair the chair}. The complement clauses refer to situations in the past.
%
\begin{exe}
	\ex	\label{ex:‎I remember that he came riding on a horse}
	\gll	[ca-w	urči-j	murtːa-l	ha-jʁ-ib-le]	han	le-w\\
		\tsc{refl-m}	horse\tsc{-dat}	rider\tsc{-advz}	\tsc{up}-come\tsc{.m.pfv-pret-cvb}	remember	exist\tsc{-m}\\
	\glt	\sqt{(‎I) remember that he came riding on a horse.}

	\ex	\label{ex:‎‎‎I am happy that you came preterite converb}
	\gll	du	razi-l=da	[u	sa-r-eʁ-ib-le	/	sa-r-eʁ-ib-ce]\\
		\tsc{1sg}	happy\tsc{-advz=1}	\tsc{2sg}	\tsc{hither-f-}go\tsc{.pfv-pret-cvb}	/ \tsc{hither-f-}go\tsc{.pfv-pret-dd.sg}\\
	\glt	\sqt{‎‎‎I am happy that you (fem.) came.} (E)

	\ex	\label{ex:‎‎‎I regret that I did not stay for the holidays}
	\gll	du	pašman	r-iχ-ub-le=da	[bajram-t-a-j	a-kelg-un-ne]\\
		\tsc{1sg}	sad	\tsc{f-}be\tsc{.pfv-pret-cvb=1}	holiday\tsc{-pl-obl-dat}	\tsc{neg-}remain\tsc{.pfv-pret-cvb}\\
	\glt	\sqt{‎‎‎I (fem.) regretted that I did not stay for the holidays.} (E)


	\ex	\label{ex:‎‎‎He said that it is not good that he did not go to the meeting}
	\gll	``[salam-le	a-s-ač'-ib-le]	ʡaˁħ-le	a-arg-u=q'al,''	Ø-ik'-ul	ca-w\\
		greeting\tsc{-loc}	\tsc{neg-hither}-come\tsc{.pfv-pret-cvb}	good\tsc{-advz}	\tsc{neg-}go\tsc{.ipfv-prs.3=mod}	\tsc{m-}say\tsc{.ipfv-icvb}	\tsc{cop-m}\\
	\glt	\sqt{‎‎‎He said, ``It is not good that I did not go to the meeting.''} (lit. having come to the greetings)
	
\end{exe}

Clauses with the perfective converb also express activity complements when they are used, e.g., with certain emotional predicates \refex{ex:He did not like when one whistled}. Similarly, \refex{ex:‎‎‎I regret that I did not stay for the holidays} could also be translated as `I regretted when I did not stay for the holidays.' 

%
\begin{exe}
	\ex	\label{ex:He did not like when one whistled}
	\gll	it-i-j	[xʷit'	haʔ-ib-le]	a-b-ičː-aq-i\\
		that\tsc{-obl-dat}	whistle	say\tsc{.pfv-pret-cvb}	\tsc{neg-n-}want\tsc{.ipfv-caus-hab.pst}\\
	\glt	\sqt{He did not like when one whistled.}
\end{exe}

This makes clear that some constructions, which at the first glance look like complement clauses formed with the perfective converb, could also be analyzed as adverbial clauses occurring together with a main clause, which contains one of the complement-taking predicates given in \refsec{sec:Complement-taking predicates}. The preterite converb is a regular means of forming adverbial clauses that refer to events and situation occurring prior to or at the same time as the situation referred to in the main clause i.e. `while, when, after, and' (\refsec{sssec:The perfective converb}). In contrast to complement clauses, adverbial clauses do not fulfill argument positions but serve as clausal adjuncts. For some of the examples in this section further research is needed in order to decide if the subordinate clause is a true complement or if it is an adjunct \refex{ex:‎‎‎I am happy that you came preterite converb},\footnote{This example could probably be translated as `You came and I am happy'.} \refex{ex:‎‎‎I regret that I did not stay for the holidays}, \refex{ex:‎‎‎He said that it is not good that he did not go to the meeting}. In example \refex{ex:‎I remember that he came riding on a horse} an adverbial-clause interpretation seems rather unlikely.

Finally, the perfective converb occurs in complements of \sqt{finish} as an alternative to the infinite or subjunctive  \refex{ex:Aminat finished to wash / washing the clothes}, \refex{ex:‎When they finished building, a man appeared}. Such complements are of the activity or the potential type. Example \refex{ex:‎When they finished building, a man appeared} shows the verb `finish', which contains an intransitive lexical verb \textit{b-iχʷ-ij} of which the nominal part \textit{taman} `time' functions as the subject-like argument of this verb. The clause preceding this verb contains a verb bearing the perfective converb suffix just as the complement clause in example \refex{ex:Aminat finished to wash / washing the clothes} and the two verbs `finish' form a pair of which the two members differ with respect to the lexical verbs (intransitive \textit{b-iχʷ-ij} `be, become' vs. transitive \textit{b-arq'-ij} `do, make') (see \refsec{sec:Compound verbs} for many more pairs of verbs of this kind). Thus, it seems reasonable to treat both constructions in \refex{ex:‎When they finished building, a man appeared} and \refex{ex:Aminat finished to wash / washing the clothes} analogously as complement constructions with complements expressed by perfective converbs. However, in  \refex{ex:‎When they finished building, a man appeared} an analysis as adverbial clause construction seems again to be possible. In that case the translation would rather be `Having built (the mill) and the time finished, ...'. Such an analysis cannot be applied to \refex{ex:Aminat finished to wash / washing the clothes}. Further testing of the syntactic properties and whether the interpretation as adverbial clause construction in \refex{ex:‎When they finished building, a man appeared} is in fact possible or necessary or perhaps incorrect must be clarified by future research.
%
\begin{exe}

	\ex	\label{ex:‎When they finished building, a man appeared}
	\gll	na	[b-arq'-ib-le]	taman	b-iχ-ub-le,	ca	ʡaˁχːuˁl	dak'u	uq-un	ca-w\\
		now	\tsc{n-}do\tsc{.pfv-pret-cvb}	end	\tsc{n-}be\tsc{.pfv-pret-cvb}	one	guest	appear	go\tsc{.m.pfv-pret}	\tsc{cop-m}\\
	\glt	\sqt{‎When they finished building, a man appeared.}
	
	\ex	\label{ex:Aminat finished to wash / washing the clothes}
	\gll	Amina-l taman d-irq'-ul ca-d	 [d-irc-ij	/	d-irc-ib-le	paltar]\\
		Amina\tsc{-erg}	end	\tsc{npl-}do\tsc{.ipfv-icvb}	\tsc{cop-npl} \tsc{npl-}wash\tsc{.pfv-inf}	/  \tsc{npl-}wash\tsc{.pfv-pret-cvb} clothes\\
	\glt	\sqt{Aminat is finishing to wash\slash washing the clothes.} (E)
\end{exe}


% --------------------------------------------------------------------------------------------------------------------------------------------------------------------------------------------------------------------- %

\subsection{Infinitive and subjunctive}
\label{ssec:Infinitive and subjunctive}

The infinitive and the subjunctive are very widely used in complement clauses of the potential type that occur with complement control. Complement control means that the subject in the complement clause is obligatorily omitted because it is identical to the subject or another argument (typically the object) of the matrix clause (\refsec{sec:Argument control in complement constructions}), and when the matrix clause is an impersonal construction, as it is the case with evaluative predicates. Note, however, that it is possible to express an overt subject in an infinitival complement clause containing a trivalent verb when the matrix clause has a different subject \refex{ex:Father sent the sheep in order to be slaughtered by the uncle@47c}.

The infinitive can be used with all persons, whereas the subjunctive has the suffix \tit{-tːaj} for the second person and \tit{-araj\slash-anaj} for the third person (see \refsec{ssec:The infinitive} for a more detailed account of the infinitive and \refsec{ssec:The subjunctive (agreeing infinitive)} for the subjunctive). There is no subjunctive suffix for the first person. The subjunctive can always be replaced by the infinitive without any change in the meaning of the sentence.

Emotion and cognition predicates take complement clauses headed by the infinitive or the subjunctive:
%
\begin{exe}
	\ex	\label{ex:‎Not wanting to leave, he left Sanzhi as the very last}
	\gll	a-b-ikː-ul	[gu-r-uq'-aˁnaj],	bah	hila-r	gu-r-ag-ur	Sanži-r\\
		\tsc{neg-n-}want\tsc{.ipfv-icvb}	\tsc{sub-abl-}go\tsc{.m-subj.3}	most	last\tsc{-abl}	\tsc{sub-abl-}go\tsc{.pfv-pret}	Sanzhi\tsc{-abl}\\
	\glt	\sqt{‎Not wanting to leave, he left Sanzhi as the very last.}

	\ex	\label{ex:(I) am ashamed to go to the village, said the father}
	\gll	``[šːi-l-cːe	uˁq'-ij=ra]	c'aχ-le	ca-b,''	Ø-ik'ʷ-ar	atːa\\
		village\tsc{-obl-in}	go\tsc{.m-inf=add}	shame\tsc{-advz}	\tsc{cop-n}	\tsc{m-}say\tsc{.ipfv-prs}	father\\
	\glt	\sqt{``(I) am ashamed to go to the village,'' said the father.}

	\ex	\label{ex:They decided to test their ability}
	\gll	[ču-la	hunar	ħaˁsib	b-arq'-ij]	q'as	b-arq'-ib\\
		\tsc{refl.pl-gen}	ability	test	\tsc{n-}do\tsc{.pfv-inf}	decision	\tsc{n-}do\tsc{.pfv-pret}\\
	\glt	\sqt{They decided to test their ability.}

	\ex	\label{ex:‎He forgot to write and I remained like this}
	\gll	qum.ert-ur-re	[cin-na	b-elk'-anaj],	heχ-itːe	kelg-un=da\\
		forget\tsc{.pfv-pret-cvb}	\tsc{refl.sg-gen}	\tsc{n-}write\tsc{.pfv-subj.3}	\tsc{dem.down}\tsc{-advz}	remain\tsc{.pfv-pret=1}\\
	\glt	\sqt{‎He forgot to write and I remained like this.}
\end{exe}

Due to the nature of manipulative predicates and modal predicates their complements belong to the potential type and they exhibit complement control. Therefore, the use of the infinitive or the subjunctive is the only possible strategy for complementation \xxref{ex:He did not allow (them) to kill him}{ex:‎‎‎What need is there for you, (he) said, to move here and there, boy}. Other complementizers such as the attributive suffix or the masdar are ungrammatical. As example \refex{ex:‎Is Paitu able to dance} demonstrates, modal predicates with infinitival complements allow for backward control: the matrix predicate shows feminine singular agreement because the ergative agent \textit{Paitu} in the complement clause has a feminine singular referent. See \refsec{sec:Argument control in complement constructions} below for more details.
%
\begin{exe}
	\ex	\label{ex:He did not allow (them) to kill him}
	\gll	[hej	kaxʷ-ij]	a-at-ur\\
		this	kill\tsc{.pfv-inf}	\tsc{neg-}let\tsc{.pfv-pret}\\
	\glt	\sqt{He did not allow (them) to kill him.}

	\ex	\label{ex:‎‎‎He (the dead body) remained there, not being allowed to bring him (to Sanzhi)}
	\gll	hel	kelg-un	hel-tːu-w,	a-w-alt-ul	[k-aqː-araj]\\
		that	remain\tsc{.pfv-pret}	that\tsc{-loc-m}	\tsc{neg-m-}let\tsc{.ipfv-icvb}	\tsc{down}-carry\tsc{-subj.3}\\
	\glt	\sqt{‎‎‎He (the dead body) remained there, not being allowed to bring him (to Sanzhi).}

	\ex	\label{ex:‎Is Paitu able to dance}
	\gll	[Pajt'u-l	ʡaˁjar	b-arq'-ij]	r-irχ-u=w?\\
		Paitu\tsc{-erg}	dance	\tsc{n-}do\tsc{.pfv-inf}	\tsc{f-}be.able\tsc{.ipfv-prs.3=q}\\
	\glt	\sqt{‎Is Paitu able to dance?}

	\ex	\label{ex:They themselves were not able to kill them}
	\gll	ca-b=ra	a-b-iχ-ub	[kaxʷ-araj]\\
		\tsc{refl-hpl=add}	\tsc{neg-hpl-}be.able\tsc{.pfv-pret}	kill\tsc{.pfv-subj.3}\\
	\glt	\sqt{They themselves were not able to kill them.}

	\ex	\label{ex:‎‎‎What need is there for you, (he) said, to move here and there, boy}
	\gll	``ce	ħaˁžat-le,''	Ø-ik'ʷ-ar,	``at	[betsat	w-ič-itːaj],	durħuˁ''\\
		what	need\tsc{-advz}	\tsc{m-}say\tsc{.ipfv-prs.3}	\tsc{2sg.dat}		here.there	\tsc{m-}lead\tsc{.ipfv-subj.2}	boy\\
	\glt	\sqt{``‎‎‎What need is there for you,'' (he) said, ``to move here and there, boy?''}
\end{exe}

Phase predicates have complement clauses of the potential or of the activity type. In the first case, they make use of the infinitive and subjunctive. Thus, in elicitation, when translating narratives from Russian or Standard Dargwa, or when telling prepared stories, the complements of \sqt{begin} contain the infinitive or subjunctive \refex{ex:‎The boy and the frog began to search for the frog}, \refex{ex:He began to shout}. Otherwise, the imperfective converb is employed (\refsec{ssec:The imperfective converb}). Similarly, with \sqt{finish} we find either the perfective converb (\refsec{ssec:The preterite converb}) or the infinitive/subjunctive \refex{ex:Ahmed stopped crying}.
%
\begin{exe}
	\ex	\label{ex:‎The boy and the frog began to search for the frog}
	\gll	durħuˁ=ra	kac'i=ra	či-ka-b-iħ-ib	[umc'-anaj	ʡaˁt'a-j]\\
		boy\tsc{=add}	puppy\tsc{=add}	\tsc{spr-down}\tsc{-hpl-}begin\tsc{.pfv-pret}	search\tsc{.ipfv-subj.3}	frog\tsc{-dat}\\
	\glt	\sqt{‎The boy and the puppy began to search for the frog.}

	\ex	\label{ex:He began to shout}
	\gll	[waˁw	Ø-ik'ʷ-ij]	w-aʔ	Ø-išː-ib	ca-w\\
		shout	\tsc{m-}say\tsc{.ipfv-inf}	\tsc{m-}begin	\tsc{m-}become\tsc{.pfv-pret}	\tsc{cop-m}\\
	\glt	\sqt{He began to shout.}

	\ex	\label{ex:Ahmed stopped crying}
	\gll	ʡaˁħmed taman Ø-irχ-ul=de	[w-isː-ij]\\
		Ahmed end	\tsc{m-}be\tsc{.ipfv-icvb=pst}	\tsc{m-}cry\tsc{-inf}\\
	\glt	\sqt{Ahmed stopped crying.} (E)
\end{exe}

Evaluative predicates also employ the infinitive\slash subjunctive if the complement clause has potential semantics:
%
\begin{exe}
	\ex	\label{ex:‎Now my lips are swollen, it is difficult to talk}
	\gll	hana	heštːi	k'unt'-be	d-emtː-un-ne,	[ʁaj	r-ik'ʷ-ij]	wahi-l	ca-b	\\
		now	these	lip\tsc{-pl}	\tsc{npl-}swell\tsc{.pfv-pret-cvb}	word	\tsc{f-}say\tsc{.ipfv-inf}	bad\tsc{-advz}	\tsc{cop-n}\\
	\glt	\sqt{‎Now my lips are swollen, it is difficult to talk.}

	\ex	\label{ex:‎It is good (better) to go (on the ice-covered ground) than from there}
	\gll	c'il=ra	ʡaˁħ-le	ca-b	[w-aš-ij]	čem	hextːu-le-rka	\\
		then\tsc{=add}	good\tsc{-advz}	\tsc{cop-n}	\tsc{m-}go\tsc{-inf}	than	there.\tsc{up-loc-abl}\\
	\glt	\sqt{‎It is good (better) to go (on the ice-covered ground) than from there.}
\end{exe}


% --------------------------------------------------------------------------------------------------------------------------------------------------------------------------------------------------------------------- %

\subsection{The embedded question marker}
\label{ssec:The embedded question marker}

The embedded question enclitic has three allomorphs: \tit{=jal} after vowels, \tit{=el} after consonants, and \tit{=l} after some suffixes ending in /e/. It is used in all types of embedded questions and occurs in complementary distribution with the interrogative enclitics used in independent questions (\refsec{sec:Subordinate questions}). It belongs to the class of predicative particles (together with the other two interrogative enclitics and a few other particles, see \refsec{sec:Predicative particles}). The particle encliticizes to the head of the complement, i.e., the verb, or occasionally to interrogative pronouns. 

Complement clauses with the embedded question marker are of the potential type \refex{ex:I don't know whether they drank the second or not}, \refex{ex:I do not know what to do} and of the activity type \refex{ex:‎The horse said, I do not know when I was born}, but not of the direct speech type, because for direct speech the other two interrogative enclitics have to be used. Matrix predicates that employ the embedded question markers are utterance predicates (see \refsec{ssec:Utterance verbs} for examples) and cognition predicates (\refsec{ssec:Cognition predicates}). It co-occurs with the quotative marker \refex{ex:(He) is probably thinking, Should I go stealing or should I not go?}.

Examples \xxref{ex:I don't know if he is remembering}{ex:Do you know whether Ali will come on Monday?} show embedded polar questions. The matrix predicates are negated or they imply the use of an embedded question such as `know' (\refsec{sec:Subordinate questions}). With affirmative matrix predicates we could alternatively have fact complements (`know that') and consequently other complementation strategies. The matrix clause can be a statement or a question.
%
\begin{exe}
	\ex	\label{ex:I don't know if he is remembering}
	\gll	[han	d-irč-aq-ul=el]	aχːu\\
		remember	\tsc{npl-}occur\tsc{.ipfv-caus-icvb=indq}	not.know\\
	\glt	\sqt{I don't know if (he) is remembering.}

	\ex	\label{ex:‎‎I know whether they will come (or not), but I am not going to tell you.}
	\gll	dam	b-alχ-ad		sa-b-irʁ-u=jal	itːi		(ja=ra	a-sa-b-irʁ-u=jal)		amma	a-cːe	a-b-urs-an=da\\
		\tsc{1sg.dat}	\tsc{n}-know.\tsc{ipfv-prs.1}	\tsc{hither-hpl}-come.\tsc{ipfv-prs.3=indq}	they	or=\tsc{add} \tsc{neg-hither-hpl}-come.\tsc{ipfv-prs=indq}		but	\tsc{2sg-in}	\tsc{neg-n}-tell-\tsc{ptcp}=1\\
	\glt	\sqt{‎‎I know whether they will come (or not), but I am not going to tell you.}
	
	\ex	\label{ex:Do you know whether Ali will come on Monday?}
	\gll	at	b-alχ-atːe=w		ʡaˁli	panedelnik-le-w	s-erʁ-u=jal	\\
		\tsc{2sg.dat}	\tsc{n}-know.\tsc{ipfv-prs.2sg=q}		Ali Monday-\tsc{loc-m}	\tsc{hither}-come.\tsc{m.ipfv-prs.3=indq}\\
	\glt	\sqt{Do you know whether Ali will come on Monday?} (E)
\end{exe}

The following two corpus examples illustrate embedded disjunct polar questions. They have the same structure as the embedded polar questions with the only difference that there is only one embedded clause and not two. 

\begin{exe}
	\ex	\label{ex:I don't know whether they drank the second or not}
	\gll	[k'ʷi	ibil	b-erčː-ib-le=l	a-b-erčː-ib-le=l]	a-b-alχ-ad\\
		two	\tsc{ord}	\tsc{n-}drink\tsc{.pfv-pret-cvb=indq}	\tsc{neg-n-}drink\tsc{.pfv-pret-cvb=indq}	\tsc{neg-n-}know\tsc{.ipfv-prs.1}\\
	\glt	\sqt{I don't know whether (they) drank the second (bottle) or not.}

	\ex	\label{ex:I don't know if it was one month or two months}
	\gll	[ca	bac=de=l,	k'ʷel	bac=de=l]	aχːu	dam\\
		one	moon\tsc{=pst=indq}	two	moon\tsc{=pst=indq}	not.know	\tsc{1sg.dat}\\
	\glt	\sqt{I don't know if it was one month or two months.}
\end{exe}


With embedded content questions the enclitic mostly appears on the verb \xxref{ex:I suspect, says the fox, whose work this was}{ex:‎The horse said, I do not know when I was born}, as it is also common for the interrogative enclitics in independent questions. The matrix clause can be affirmative or negative.
%
\begin{exe}
	\ex	\label{ex:I suspect, says the fox, whose work this was}
	\gll	``hel	šak	b-irk-ul=da,''	b-ik'-ul	ca-b,	``[hi-l	b-arq'-ib=el	hel	ʡaˁči]''\\
		that	feel	\tsc{n-}occur\tsc{.ipfv-icvb=1}	\tsc{n-}say\tsc{.ipfv-icvb}	\tsc{cop-n}	who\tsc{.obl-erg}	\tsc{n-}do\tsc{.pfv-pret=indq}	that	work\\
	\glt	\sqt{``I suspect,'' says the fox, ``whose work this was.''}

	\ex	\label{ex:(He) probably remembers how he beat her up@18}
	\gll	urkː-ar	[hek'	cet'-le	it-ul=el]	han	d-irk-ul\\
		find.\tsc{ipfv-prs}	\tsc{dem.up}	how\tsc{-advz}	beat.up\tsc{-icvb=indq}	remember	\tsc{npl-}occur\tsc{.ipfv-icvb}\\
	\glt	\sqt{(He) probably remembers how he beat her up.}

	\ex	\label{ex:I don't know about what he is thinking@22}
	\gll	aχːu	dam	[ce-lla	qari=či-w	pikri	Ø-ik'-ul=el]\\
		not.know	\tsc{1sg.dat}	what\tsc{-gen}	on.top=on\tsc{-m}	thought	\tsc{m-}say\tsc{.ipfv-icvb=indq}\\
	\glt	\sqt{I don't know about what he is thinking.}

	\ex	\label{ex:‎The horse said, I do not know when I was born}
	\gll	``du-l	a-b-alχ-ad,''	Ø-ik'-ul	ca-w	``[ceqːel	hak'-ub=da=jal]''\\
		\tsc{1sg-erg}	\tsc{neg-n-}know\tsc{.ipfv-prs.1}	\tsc{m-}say\tsc{.ipfv-icvb}	\tsc{cop-m}	when	appear\tsc{.pfv-pret=1=indq}\\
	\glt	\sqt{(‎The horse) said, ``I do not know when I was born.''}
\end{exe}

If the complement does not contain a verb the enclitic appears on the question word \refex{ex:How do you know them}. In utterances with verbs and questions words it is possible to attach the enclitic to the latter \refex{ex:‎I don't know when aunt Zhamilat was bornA}, but the variant with the verbal host is judged as preferable \refex{ex:‎I don't know when aunt Zhamilat was bornB}.
%
\begin{exe}
	\ex	\label{ex:How do you know them}
	\gll	ce	b-alχ-ul=de	[ča-qal=el]?\\
		what	\tsc{hpl-}know\tsc{.ipfv-icvb=2sg}	who\tsc{-assoc=indq}\\
	\glt	\sqt{How do you know them?} (i.e. who they are)

	\ex	\label{ex:‎I don't know when aunt Zhamilat was bornA}
	\gll	[ceqːel=el	hak'-ub-ce	žamilat	azi]	dam	a-b-alχ-ad\\
		when\tsc{=indq}	appear\tsc{.pfv-pret-dd.sg}	Zhamilat	aunt	\tsc{1sg.dat}	\tsc{neg-n-}know\tsc{.ipfv-prs.1}\\
	\glt	\sqt{‎I don't know when aunt Zhamilat was born.} (E)

	\ex	\label{ex:‎I don't know when aunt Zhamilat was bornB}
	\gll	[ceqːel	hak'-ub-ce=jal	žamilat	azi]	dam	a-b-alχ-ad\\
		when	appear\tsc{.pfv-pret-dd.sg=indq}	Zhamilat	aunt	\tsc{1sg.dat}	\tsc{neg-n-}know\tsc{.ipfv-prs.1}\\
	\glt	\sqt{‎I don't know when aunt Zhamilat was born.} (E)
\end{exe}

If the subject of an embedded question is first person, the verb in the complement clause takes the modal interrogative suffix \tit{-ide(l)}, which most probably goes back to a person marker \textit{-id} plus the petrified marker for embedded questions =\textit{el} (\refsec{sec:modalinterrogative}) (\refex{ex:I do not know what to do}). The suffix can occur in combination with the quotative marker (\refsec{ssec:The quotative particles}).
%
\begin{exe}
	\ex	\label{ex:I do not know what to do}
	\gll	[ce	b-arq'-ide=l]	a-b-alχ-ul=da\\
		what	\tsc{n-}do\tsc{.pfv-modq=indq}	\tsc{neg-n-}know\tsc{.ipfv-icvb=1}\\
	\glt	\sqt{I do not know what to do.}
\end{exe}


% --------------------------------------------------------------------------------------------------------------------------------------------------------------------------------------------------------------------- %

\subsection{The imperfective converb}
\label{ssec:The imperfective converb}

The imperfective converb belongs to the minor complementation strategies. It expresses potential or activity complements with the verb \sqt{begin}, for which it represents the most common way of marking complements \refex{ex:‎‎‎The fox began betraying / to betray the devil sister} (alternatively, the infinitive\slash subjunctive is used, see \refsec{ssec:Infinitive and subjunctive}).
%
\begin{exe}
	\ex	\label{ex:‎‎‎The fox began betraying / to betray the devil sister}
	\gll	kːurtːa	b-aʔ	b-išː-ib	[šajt'an	rucːi	r-irʡ-uˁl]\\
		fox	\tsc{n-}begin	\tsc{n-}put\tsc{.pfv-pret}	devil	sister	\tsc{f-}betray\tsc{-icvb}	\\
	\glt	\sqt{‎‎‎The fox began betraying\slash to betray the devil sister.}
\end{exe}

Another possible matrix predicate for complements heading the imperfective converb is the perception verb \sqt{see} whose complement clauses are either of the fact type as the translation in \refex{ex:‎S/he saw that mother milked the cows} suggests or of the activity type \refex{ex:‎I got happy when I saw that Nursijat recovered and is (already) walking around}. 
%
\begin{exe}

	\ex	\label{ex:‎S/he saw that mother milked the cows}
	\gll	it-i-j	či-d-až-ib	[aba-l	q'uˁl-e	icː-ul]\\
		that\tsc{-obl-dat}	\tsc{spr-npl-}see\tsc{.pfv-pret}	mother\tsc{-erg}	cow\tsc{-pl}		milk\tsc{.ipfv-icvb}\\
	\glt	\sqt{‎S/he saw that mother was milking the cows.} (E)
	
	\ex	\label{ex:‎I got happy when I saw that Nursijat recovered and is (already) walking around}
	\gll	[Nursijat	ʡaˁħ	r-iχ-ub-le,	r-ax-ul]	či-r-až-ib-le,	razi	r-iχ-ub=da=q'al\\
		Nursijat	good	\tsc{f-}be\tsc{.pfv-pret-cvb}	\tsc{f-}go\tsc{-icvb}	\tsc{spr-f-}see\tsc{.pfv-pret-cvb}	happy	\tsc{f-}be\tsc{.pfv-pret=1=mod}\\
	\glt	\sqt{‎I (fem.) got happy when I saw that Nursijat recovered and is (already) walking around.}

\end{exe}


% --------------------------------------------------------------------------------------------------------------------------------------------------------------------------------------------------------------------- %

\subsection{The \textit{pretend}-construction}
\label{ssec:PretendConstruction}
The cross-categorical suffix \tit{-il}, which forms referential attributes that can modify nouns (e.g. relative clauses) or occur referentially independent in argument or adjunct positions, can can also take the genitive case to express activity complements in the pretending-construction \xxref{ex:‎‎‎He pretends to sleep (lit. to have laid down)}{ex:‎‎‎Apparently they pretended to make him to put him (to bed) well, took the money that was in his pocket and left} (see \refsec{ssec:The -il attributive} for a summary of all functions of \tit{-il}). Depending on whether \tit{-il} is suffixed to the modal participle \refex{ex:‎‎‎The girl pretends to make the homework} or to the preterite participle \refex{ex:‎‎‎Ruslan pretended to write a book} the complement clauses have past time reference or non-past time reference. Syntactically, the dependent clause does not represent the object argument of the matrix verb \sqt{do, make} because of the genitive case. Nevertheless, we can analyze this construction as a complement construction because there are a number of similar examples with nouns instead of clauses that semantically, but not syntactically, are arguments of the verb \sqt{do, make} despite bearing the genitive case (\refsec{sssec:Genitive}).
%
\begin{exe}

	\ex	\label{ex:‎‎‎He pretends to sleep (lit. to have laid down)}
	\gll	it-i-l	[usː-un-il-la]	či-b-irq'-ul	ca-b\\
		that\tsc{-obl-erg}	lay\tsc{.m.pfv-pret-ref-gen}	\tsc{spr-n-}do\tsc{.ipfv-icvb}	\tsc{cop-n}\\
	\glt	\sqt{‎‎‎He pretends to sleep.} (lit. to have laid down) (E)
	
	\ex	\label{ex:‎‎‎The girl pretends to make the homework}
	\gll	rursːi-l	[durs-re	luk'-an-il-la]	či-b-irq'-ul	ca-b\\
		girl\tsc{-erg}	homework\tsc{-pl}	write\tsc{.ipfv-ptcp-ref-gen}	\tsc{spr-n-}do\tsc{.ipfv-icvb}	\tsc{cop-n}\\
	\glt	\sqt{‎‎‎The girl pretends to do the homework.} (E)


	\ex	\label{ex:‎‎‎Apparently they pretended to make him to put him (to bed) well, took the money that was in his pocket and left}
	\gll	[ʡaˁħ-le k-alt-an-il-la]	či-b-arq'-ib-le,	kisna-d	arc	tːura	h-asː-ib-le, b-ax-ul	b-už-ib	ca-b\\
		good\tsc{-advz}	\tsc{down}-let\tsc{.ipfv-ptcp-ref-gen}	\tsc{spr-n-}do\tsc{.pfv-pret-cvb}		in.pocket\tsc{-npl}	money	outside	\tsc{up}-take\tsc{.pfv-pret-cvb}	\tsc{hpl-}go\tsc{-icvb}	\tsc{hpl-}be\tsc{-pret}	\tsc{cop-hpl}	\\
	\glt	\sqt{‎‎‎Apparently they pretended to put him (to bed) well, took the money that was in his pocket and left.}
\end{exe}

The agreement prefix on the matrix verb can be \tit{b-} or \tit{d-} with no difference in semantics \refex{ex:‎‎‎Ruslan pretended to write a book}. The prefix does not seem to be governed by an agreement controller, because of the two available options one is not attested (namely long distance agreement with the absolutive argument of the complement clause), and the other (local agreement with the entire complement clause) is implausible since it allows only for the \tit{b-} prefix, but not for the \tit{d-} prefix. Furthermore, in constructions without a complement only \tit{d-} is possible \refex{ex:‎Do not pretend}.
%
\begin{exe}
	\ex	\label{ex:‎‎‎Ruslan pretended to write a book}
	\gll	Ruslan-ni	[kiniga	b-elk'-un-il-la]	či-b-arq'-ib	/	či-d-arq'-ib\\
		Ruslan\tsc{-erg}	book	\tsc{n-}write\tsc{.pfv-pret-ptcp-gen}	\tsc{spr-n-}do\tsc{.pfv-pret}\slash\tsc{spr-npl-}do\tsc{.pfv-pret}\\
	\glt	\sqt{‎‎‎Ruslan pretended to write a book.} (E)

	\ex	\label{ex:‎Do not pretend}
	\gll	či-ma-d-irq'-itːa!\\
		\tsc{spr-proh-npl-}do\tsc{.ipfv-proh.sg}\\
	\glt	\sqt{‎Do not pretend!} (E)
\end{exe}



% --------------------------------------------------------------------------------------------------------------------------------------------------------------------------------------------------------------------- %



%%%%%%%%%%%%%%%%%%%%%%%%%%%%%%%%%%%%%%%%%%%%%%%%%%%%%%%%%%%%%%%%%%%%%%%%%%%%%%%%

\section{Reported speech constructions}
\label{sec:Reported speech constructions}


% --------------------------------------------------------------------------------------------------------------------------------------------------------------------------------------------------------------------- %

\subsection{General characteristics of reported speech}
\label{ssec:General characteristics of reported speech}

Reported speech constructions usually contain an utterance verb and a quote. The relationship between the clause containing the verb of speech and the quote can be marked or unmarked. The utterance verb precedes the quote \refex{ex:‎His wife begged him, Do not go}, interrupts it \refex{ex:He says, I do not need these things}, or follows it \refex{ex:‎‎‎I bought 99 goats, said Ali}. Sometimes it is repeated and occurs in more than one position \refex{ex:‎‎The wife says, This is enough, get up}. The quote itself does not bear any specific grammatical marking apart from the optional use of quotative particles to pragmatically mark quotes.

The verb \tit{b-ik'ʷ-} is the most frequently occurring verb of speech that is also used as a quotative particle in reported speech constructions and other complement clauses (see also \refsec{ssec:The quotative particles}). The basic meaning of this verb seems to be \sqt{say}, but it is often used with the meaning \sqt{think}, i.e., expressing mental activities such as thinking, considering, or reflecting. The verb has only an imperfective stem. Its subject argument takes the absolutive case and controls the gender agreement prefix. It is very widely used as a light verb in compounding, as shown by some examples above. The compounds can denote activities related to speech and language such as \tit{pikri b-ik'ʷ-} \sqt{think}, \tit{xul b-ik'ʷ-} \sqt{wish, dream}, \tit{ʁaj b-ik'ʷ-} \sqt{scold}, \tit{ʁumku b-ik'ʷ-} \sqt{swear}, \tit{iχtilat b-ik'ʷ-} \sqt{chat}, etc., but they can also have totally different meanings such as \tit{qus b-ik'ʷ-} \sqt{slide} or \tit{duc' b-ik'ʷ-} \sqt{run} (see \refsec{sec:Compound verbs} for more examples). The verb is used as a matrix verb in reported speech constructions, either in the form of the compound present \refex{ex:He says, I do not need these things} or with the suffix \tit{-ar} for past time reference \refex{ex:‎‎‎I bought 99 goats, said Ali}.
%
\begin{exe}
	\ex	\label{ex:He says, I do not need these things}
	\gll	``dam	ʡaˁʁuni-l	akːu''	Ø-ik'-ul	Ø-ik'-ul ca-w	``hel-tːi	cik'al''\\
		\tsc{1sg.dat}	needed\tsc{-advz}	\tsc{cop.neg}	\tsc{m}-say\tsc{.ipfv-icvb}	\tsc{m}-say\tsc{.ipfv-icvb} be-\tsc{m}	that\tsc{-pl}	something\\
	\glt	\sqt{He says, ``I do not need these things.''}

	\ex	\label{ex:‎‎‎I bought 99 goats, said Ali}
	\gll	``du-l	urč'em-c'anu	urč'em-ra	ečːa	asː-ib=da''	Ø-ik'ʷ-ar	ʡaˁli	\\
		\tsc{1sg-erg}	nine-\tsc{ten}	nine\tsc{-num}	she.goat	buy\tsc{.pfv-pret=1}	\tsc{m-}say\tsc{.ipfv-prs.3}	Ali\\
	\glt	\sqt{``‎‎‎I bought 99 goats,'' said Ali.}

	\ex	\label{ex:‎‎In once place, there are, he says, trees. Whatever may happen, do not look at these trees}
	\gll	``ca	musːa-d	k'e-d''	Ø-ik'-ul	ca-w,	``kːalk-me.	warilla.wari	u	iχ-tː-a-j	er	či-ma-hark'-utːa!''\\
		one	place\tsc{-npl}	exist\tsc{.up-npl}	\tsc{m-}say\tsc{.ipfv-icvb}	\tsc{cop-m}	tree\tsc{-pl}	no.way	\tsc{2sg}	\tsc{dem.down-pl-obl-dat}	look	\tsc{spr-proh-}look\tsc{.ipfv-proh.sg}\\
	\glt	\sqt{\dqt{‎‎In once place, there are,} he says, ``trees. Whatever may happen, do not look at these trees!''}

	\ex	\label{ex:‎‎The wife says, This is enough, get up}
	\gll	hel	xːunul	r-ik'-ul	ca-r	``d-irʁ-an-ne=n,	ha-jcː-e''	r-ik'-ul	ca-r	``gu-r!''\\
		that	woman	\tsc{f-}say\tsc{.ipfv-icvb}	\tsc{cop-f}	\tsc{npl-}be.enough\tsc{.ipfv-ptcp-fut.3=}but	\tsc{up}-get.up\tsc{.m.pfv-imp}	\tsc{f-}say\tsc{.ipfv-icvb}	\tsc{cop-f}	down\tsc{-abl}\\
	\glt	\sqt{‎‎The wife says, ``This is enough, get up!''}
\end{exe}

This verb is also used when mentioning the name of something or somebody or the word for something in another language or dialect \refex{ex:‎‎In our (language) it is called plant of the wound I forgot what it is called in Russian}, e.g. \tit{Saliħaˁt b-ik'-ul} \sqt{(a person) called Salihat}.
%
\begin{exe}
	\ex	\label{ex:‎‎In our (language) it is called plant of the wound I forgot what it is called in Russian}
	\gll	nišːa-la	``daˁqaˁ-lla	q'ar''	b-ik'ʷ-ar.	[ʡuˁrus	ʁaj-la	ce=jal	b-ik'-ul	ca-b	it-i-j] dam	qum.urt-ul	ca-b\\
		\tsc{1pl-gen}	wound\tsc{-gen}	plant	\tsc{hpl-}say\tsc{.ipfv-prs}	Russian word\tsc{-gen}	what\tsc{=indq}	\tsc{hpl-}say\tsc{.ipfv-icvb}	\tsc{cop-hpl}	that\tsc{-obl-dat}	\tsc{1sg.dat}	forget\tsc{.ipfv-icvb}	\tsc{cop-n}\\
	\glt	\sqt{‎‎In our (language) it is called \dqt{plant of the wound.} I forgot what it is called in Russian.}
\end{exe}

The verb has grammaticalized into a quotative particle (see below). Moreover, it can express hearsay evidentiality.

Another very frequent utterance verb is \tit{-ʔ-} (\tsc{pfv})\slash\tit{-erʔ-} (\tsc{ipfv}) \sqt{say}, which is almost always used with the spatial preverb \tit{ha-} \sqt{upwards}, that is \tit{haʔ-}\slash\tit{herʔ-}. This is a transitive verb that marks the subject, i.e. the speaker, with the ergative. It is mainly used in reported speech constructions with past time reference. Besides that it functions as a quotative particle (see below).
%
\begin{exe}
	\ex	\label{ex:‎‎Look!, said the wind}
	\gll	``er	b-erč'-e!''	haʔ-ib	č'an-ni\\
		look	\tsc{n-}look\tsc{.pfv-imp}	say\tsc{.pfv-pret}	wind\tsc{-erg}\\
	\glt	\sqt{``‎‎Look!'' said the wind.}
\end{exe}

The imperfective stem is used, among other things, for meta-comments on how you express what you want to say, which words you use:
%
\begin{exe}
	\ex	\label{ex:Gather did, you should not say in Dargwa}
	\gll	``sabrat		d-arq'-ib''	herʔ-an	akːu=q'al	darkːʷan	ʁaj-la	\\
		gather	\tsc{npl-}do\tsc{.pfv-pret}	say\tsc{.ipfv-ptcp}	\tsc{.cop.neg=mod}	Dargwa	language\tsc{-gen}\\
	\glt	\sqt{``Gather did,'' you should not say in Dargwa.}
\end{exe}

Other common simple utterance verbs are the transitive verbs \tit{b-urs-} and \tit{b-ux-}, which both can be translated with \sqt{tell}, and the transitive verb \tit{xar b-eʁ-} (\tsc{pfv})\slash\tit{xar b-irʁ-} (\tsc{ipfv}) \sqt{ask}, which occur, like all verbs of speech, with and without a quotative particle.
%
\begin{exe}
	\ex	\label{ex:I hit my wife, what should I do, he says, he is telling the stories}
	\gll	``du-l	b-aˁq-ib-le''	Ø-ik'-ul	ca-w	``xːunul-li-j,	ce	b-arq'-ide=l''	Ø-ik'-ul	χabur-t-a-l	ux-ul	ca-w	heχ\\
		\tsc{1sg-erg}	\tsc{n-}hit\tsc{.pfv-pret-cvb}	\tsc{m-}say\tsc{.ipfv-icvb}	\tsc{cop-m}	woman\tsc{-obl-dat}	what	\tsc{n-}do\tsc{.pfv-modq=indq}	\tsc{m-}say\tsc{.ipfv-icvb}	story\tsc{-pl-obl-erg}	tell\tsc{.m.ipfv-icvb}	\tsc{cop-m}	\tsc{dem.down}	\\
	\glt	\sqt{``I hit my wife, what should I do,'' he says; he is telling the stories.}

	\ex	\label{ex:They asked us, where did you come from}
	\gll	itːi=ra	``čina-r	sa-d-eʁ-ib-te=da=j?''	b-ik'-ul	xar.b.eʁ-ib	nišːa-la\\
		\tsc{3pl=add}	where\tsc{-abl}	\tsc{hither-1/2pl-}go\tsc{.pfv-pret-dd.pl=2pl=q}	\tsc{hpl-}say\tsc{.ipfv-icvb}	ask\tsc{.n.pfv-pret}	\tsc{1pl-gen}\\
	\glt	\sqt{They also asked us ``Where did you come from?''}

	\ex	\label{ex:‎His wife begged him, Do not go}
	\gll	xːunul-li	tiladi	b-arq'-ib	ca-b	hel-i-cːe	``ma-ax-utːa!''	r-ik'-ul\\
		woman\tsc{-erg}	request	\tsc{n-}do\tsc{.pfv-pret}	\tsc{cop-n}	that\tsc{-obl-in}	\tsc{proh-}go\tsc{-proh.sg}	\tsc{f-}say\tsc{.ipfv-icvb}\\
	\glt	\sqt{‎His wife begged him ``Do not go!''}
	
	\ex	\label{ex:‎‎Then they asked the wolf, When were you born?}
	\gll	c'il	bec'-li-cːe	xar.b.eʁ-ib	ca-b	``u	ceqːel	hak'-ub=de?''\\
		then	wolf\tsc{-obl-in}	ask\tsc{.n.pfv-pret}	\tsc{cop-n}	\tsc{2sg}	when	appear\tsc{.pfv-pret=2sg}\\
	\glt	\sqt{‎‎Then they asked the wolf ``When were you born?''}
	
\end{exe}

A minor strategy for expressing reported speech is the use of the verb \sqt{begin} \refex{ex:He began when he came home, I will not anymore do things like this} and other non-utterance predicates \refex{ex:‎‎‎The daughter-in-law must have asked, From where do you bring the body}.
%
\begin{exe}
	\ex	\label{ex:He began when he came home, I will not anymore do things like this}
	\gll	w-aʔ	ač'-ib,	qili	sa-jʁ-ib=er	[\ldots]	``du-l	hel=ʁuna	cik'al	imc'a	a-b-irq'-an=da''	Ø-ik'-ul\\
		\tsc{m-}begin	come\tsc{.pfv-pret}	home	\tsc{hither}-come\tsc{.m.pfv-pret=}when	{}	\tsc{1sg-erg}	this\tsc{=eq} something	anymore	\tsc{neg-n-}do\tsc{.ipfv-ptcp=1}	\tsc{m-}say\tsc{.ipfv-icvb}\\
	\glt	\sqt{He began when he came home, [\ldots] ``I will not do things like this anymore.''}

	\ex	\label{ex:‎‎‎The daughter-in-law must have asked, From where do you bring the body}
	\gll	``čina-r	sa-k-ul=de?''	r-ik'-ul	r-irχʷ-an=de	het	durħuˁ-la	xːunul	\\
		where\tsc{-abl}	\tsc{hither}-lead\tsc{.pfv-icvb=2sg}	\tsc{f-}say\tsc{.ipfv-icvb}	\tsc{f-}become\tsc{.ipfv-ptcp=pst}	that	boy\tsc{-gen}	woman\\
	\glt	\sqt{‎‎‎The daughter-in-law must have asked ``From where do you bring the body?''}
\end{exe}

Finally, the topic of a conversation can be expressed by using the postposition \tit{qari=či-b} \sqt{on.top=on\tsc{-n}} together with a complement clause bearing the genitive case suffix (see \refsec{ssec:postposition qari} for an example).


% --------------------------------------------------------------------------------------------------------------------------------------------------------------------------------------------------------------------- %

\subsection{Formal marking in reported speech constructions}
\label{ssec:Formal marking in reported speech constructions}

The distinction between direct and indirect speech as we know it from European languages cannot be applied to Sanzhi because it relies on deictic shift, but in Sanzhi the original speaker's deictic frame is usually retained. Sanzhi does not have any special verb forms or sequences of tense. The only formal marking that is available for reported speech are quotative particles occurring at the end of the quote, and very occasionally simple reflexive pronouns. These quotative particles are also used with other matrix verbs that are not utterance verbs (\refsec{ssec:The quotative particles}), and their use is mostly optional. In fact, unmarked quotes are as common as quotes marked by quotative particles.

Sanzhi has two quotative particles for reported speech that transparently derive from the two most frequently used verbs of speech. The first is \tit{b-ik'-ul}, the imperfective converb of \tit{b-ik'ʷ-}. The second is \tit{haʔ-ib-le}, the perfective converb of \tit{haʔ-}. Both are, in fact, formally indistinguishable from the respective converbs. They have not undergone any phonological reduction so far, and the gender prefix of \tit{b-ik'-ul} follows the same agreement rules as the matrix verb of speech from which it is derived. Therefore, it is often impossible to say whether a certain occurrence of them represents the use as a matrix verb or a quotative particle. If the quotative particles co-occur with framing verbs in a matrix clause we can be sure that we are dealing with the quotative-particle use \refex{ex:He says, I do not need these things}, \refex{ex:They asked the fox, When were you born?, ...}, \refex{ex:‎‎‎It is necessary that he must come, at 8 he must be there, tell him this, I said}. Sometimes it looks as if the quotative particles alone can mark an utterance as a quote \refex{ex:Are you afraid of your wife?, they say and}. Such an analysis naturally suggests itself if we remember that the marker of embedded questions can also be used without a matrix clause in epistemic modal constructions (\refsec{sec:Subordinate questions}) \citep{ForkerLTSanzhi}. However, despite the relative frequency of examples such as \refex{ex:Are you afraid of your wife?, they say and} used in what looks like an independent utterance, these clauses are dependent clauses that cannot occur on their own (see \refsec{ssec:Adverbial clauses as independent utterances} for a discussion of the apparent use of converbs in what seem to be main clauses).
%
\begin{exe}
	\ex	\label{ex:Are you afraid of your wife?, they say and}
	\gll	``xːunul-li-sa-r	uruχ	Ø-ik'-ul=de=w?''	b-ik'-ul,~\ldots\\
		woman\tsc{-obl-ante-abl}	fear	\tsc{m-}say\tsc{.ipfv-icvb=2sg=q}	\tsc{hpl-}say\tsc{.ipfv-icvb}\\
	\glt	\sqt{``Are you afraid of your wife?'' they say and \ldots}
\end{exe}

In elicitation, \tit{b-ik'-ul} can apparently not be used when the matrix verb of speech occurs in the preterite and instead the particle \tit{haʔible} is employed. Thus, if we replace \tit{haʔible} with \tit{ik'-ul} in \refex{ex:‎‎‎Ramazan gave me his word, I'll help, he said}, then the sentence is rejected by Sanzhi speakers. However, in the Sanzhi corpus one can find examples of matrix verbs of speech in the preterite used together with \tit{b-ik'-ul} \refex{ex:They asked us, where did you come from}, \refex{ex:‎His wife begged him, Do not go}.

The quotative particle \tit{haʔible} is only rarely used and therefore ambiguous examples are harder to find \refex{ex:They asked the fox, When were you born?, ...}, \refex{ex:‎‎‎It is necessary that he must come, at 8 he must be there, tell him this, I said}. In \refex{ex:‎‎‎It is necessary that he must come, at 8 he must be there, tell him this, I said} it could either be analyzed as a quotative particle that follows the first part of the quote or as matrix verb that heads the preceding complement clause.
%
\begin{exe}
	\ex	\label{ex:They asked the fox, When were you born?, ...}
	\gll	kːurtːa-cːe	xar.b.eʁ-ib	ca-b	``ceqːel	hak'-ub=de?''	haʔ-ib-le\\
		fox\tsc{-in}	ask\tsc{.n.pfv-pret}	\tsc{cop-n}	when	appear\tsc{.pfv-pret=2sg}	say\tsc{.pfv-pret-cvb}\\
	\glt	\sqt{They asked the fox ``When were you born?'' \ldots}

	\ex	\label{ex:‎‎‎It is necessary that he must come, at 8 he must be there, tell him this, I said}
	\gll	``sːaˁʡaˁt	kːaʔal-le-w	hextːu	či-ha-jʁ-ij	ʡaˁʁuni-l	ca-b''	haʔ-ib-le	``b-urs-a''	haʔ-ib=da	``u-l!''\\
		hour	eight\tsc{-loc-m}	there.\tsc{up}	\tsc{spr-up}-come\tsc{.m.pfv-inf}	needed\tsc{-advz}	\tsc{cop-n}	say\tsc{.pfv-pret-cvb}	\tsc{n-}tell\tsc{-imp}	say\tsc{.pfv-pret=1}	\tsc{2sg-erg}\\
	\glt	\sqt{``‎‎‎It is necessary that he must come, at 8 he must be there. Tell him this!'' I said.}
\end{exe}

In elicitiation, the quotative particle \tit{haʔible} occurs when the matrix clause has past time reference because it developed from the perfective converb construction that is derived from the preterite participle \refex{ex:‎‎‎Ramazan gave me his word, I'll help, he said}. The quote together with \tit{haʔible} looks exactly like an adverbial clause that follows the main clause and into which a complement is embedded.
%
\begin{exe}
	\ex	\label{ex:‎‎‎Ramazan gave me his word, I'll help, he said}
	\gll	Ramazan	ʁaj	b-ičː-ib	``dam	kumek	b-irq'-an=da''	haʔ-ib-le\\
		Ramazan	word	\tsc{n-}give\tsc{.pfv-pret}	\tsc{1sg.dat}	help	\tsc{n-}do\tsc{.ipfv-ptcp=1}	say\tsc{.pfv-pret-cvb}\\
	\glt	\sqt{‎‎‎Ramazan gave me his word ``I'll help.''} (E)
\end{exe}

The same converb is used with the meaning \sqt{because, in order to} to express reasons or purpose clauses. The expression of reason or cause is shown in \refex{ex:Because he beat up his family, because the boy was in the arms (of the mother), they led him away}. It might have developed from an adverbial construction in which \tit{haʔible} functions as a verb of speech and the converb clause, which precedes \tit{haʔible}, represents a quote that explains or provides reasons for the situation referred to in the main clause. In other words, \refex{ex:Because he beat up his family, because the boy was in the arms (of the mother), they led him away} could alternatively be translated as `After (they) said that he beat up his family and (they) said that the boy was in the arms (of the mother), they led him away.'

%
\begin{exe}
	\ex	\label{ex:Because he beat up his family, because the boy was in the arms (of the mother), they led him away}
	\gll	kulpat	b-it-ib-le	haʔ-ib-le,	nik'a-ce	kʷi-lle	naˁq-li-cːe-w	haʔ-ib-le,	w-erč-ib-le\\
		family	\tsc{hpl-}beat.up\tsc{-pret-cvb}	say\tsc{.pfv-pret-cvb}	small\tsc{-attr}	in.the.hands\tsc{-advz}	arm\tsc{-obl-in}\tsc{-m}	say\tsc{.pfv-pret-cvb}	\tsc{m-}lead\tsc{.pfv-pret-cvb}\\
	\glt	\sqt{Because he beat up his family, because the boy was in the arms (of the mother), they led him away.}
\end{exe}

In elicitation, the quotative particle \tit{bik'ul} is not used when \tit{b-ik'ʷ-} is the matrix verb. But this restriction has purely stylistic reasons and is only apparent. In the corpus, counter-examples can readily be found.

The use of quotative markers together with the infinitive in purpose clauses with the meaning \sqt{in order to} has been noted in a number of other East Caucasian languages such as Ingush, Godoberi, Hinuq, Tsez, and probably also Tsakhur \citep{Forker2016c}. For this construction, it is plausible to assume that it goes back to a reported speech construction with \tit{haʔible} originally functioning as the framing verb to a quote which might have contained another verb with volitional semantics. In other words, \refex{ex:They probably sit down in order to drink here, I do not know} might have developed from a construction like \sqt{They said, we want to drink.}.
%
\begin{exe}
	\ex	\label{ex:They probably sit down in order to drink here, I do not know}
	\gll	deč-li	b-učː-ij	haʔ-ib-le	ka-b-iž-ib-te	b-iχʷ-ij	heštːu,	aχːu	dam\\
		drinking\tsc{-erg}	\tsc{hpl-}drink\tsc{.ipfv-inf}	say\tsc{.pfv-pret-cvb}	\tsc{down-hpl-}be\tsc{.pfv-pret-dd.pl} \tsc{hpl-}be\tsc{.pfv-inf}	here	not.know	\tsc{1sg.dat}\\
	\glt	\sqt{They probably sit down in order to drink here, I do not know.}
\end{exe}

If the quote is an utterance with non-declarative mood, be it a command or a question, then the mood markers such as the imperative suffix \refex{ex:‎‎‎It is necessary that he must come, at 8 he must be there, tell him this, I said} or the enclitics for content questions \refex{ex:They asked us, where did you come from} and polar questions \refex{ex:Are you afraid of your wife?, they say and} are normally kept and can co-occur with the quotative markers. Otherwise, it is possible to use the special enclitic for embedded questions that does not co-occur with the interrogative enclitics for independently used questions, but can co-occur with the quotative markers (see \refsec{ssec:The embedded question marker} above and \refsec{sec:Subordinate questions}). This enclitic is added to the head of the interrogative clause or to the item in focus. In embedded disjunctive polar questions such as \refex{ex:‎‎‎There we will ask if he had a head or not} it encliticizes to each member of the disjunction. The embedded question marker does normally not occur in independent clauses (except for when it is used to express epistemic modality). Therefore, the complement clauses containing it are marked as dependent, although they have at their disposal the full range of TAM forms as well as person agreement.
%
\begin{exe}
	\ex	\label{ex:‎‎‎There we will ask if he had a head or not}
	\gll	xar	b-irʁ-an=da	[bek'	le-b=de=l	b-akːʷ-i=jal]\\
		ask	\tsc{n-}ask\tsc{.ipfv-ptcp=1}	head	exist\tsc{-n=pst=indq}	\tsc{n-}\tsc{cop.neg-hab.pst=indq}\\
	\glt	\sqt{We will ask if he had a head or not.}
\end{exe}

The only further peculiarity that reported speech construction show, and which they share with other subordinate clauses, most notably other complement clauses, is the use of reflexive pronouns as logophors (see \citealp{ForkerSubmittedb} for a detailed account of logophoric reflexives and other properties of non-direct speech constructions in Sanzhi). When the author of the quote, which must be third person, is identical to an argument or adjunct in the quote, the reflexive pronoun can be used instead of the first person pronoun \refex{ex:‎‎‎By God, he said that he himself had bought 10 bottles of vodka; and apparently he had brought them with him}. The use of demonstrative pronouns is impossible since they would express disjoint reference with the author of the quote.
%
\begin{exe}
	\ex	\label{ex:‎‎‎By God, he said that he himself had bought 10 bottles of vodka; and apparently he had brought them with him}
	\gll	wallah		Ø-ik'ʷ-ar	wec'al	ʡaˁraˁq'i-la	šuša	Ø-ik'ʷ-ar	cin-ni	asː-ib-le,	d-alli	h-aqː-ib-te	d-už-ib	ca-d	\\
		by.God \tsc{m-}say\tsc{.ipfv-prs.3}	ten	vodka\tsc{-gen}	bottle	\tsc{m-}say\tsc{.ipfv-prs.3}	\tsc{refl.sg-erg}	buy\tsc{.pfv-pret-cvb}	\tsc{npl-}together	\tsc{up}-carry\tsc{-pret-dd.pl} 	\tsc{npl-}be\tsc{-pret}	\tsc{cop-npl}\\
	\glt	\sqt{‎‎‎By God, he said that he himself had bought 10 bottles of vodka; and apparently he had brought them with him.}
\end{exe}

The use of personal pronouns is also possible. The first person pronoun is employed when the referent in the quote is identical to the author \refex{ex:I hit my wife, what should I do, he says, he is telling the stories} and the second person pronoun is used when the referent is identical to the addressee \refex{ex:‎‎Then they asked the wolf, When were you born?}. Furthermore, the agreement on the verb in clauses with subject-like arguments expressed by reflexive pronouns is not third person, as would be expected when a reflexive pronoun occurs, but first person \refex{ex:‎(He said,) The things that I have done, I will repair (make better), I will be on the right road}.
%
\begin{exe}
	\ex	\label{ex:‎(He said,) The things that I have done, I will repair (make better), I will be on the right road}
	\gll	``cin-ni	d-arq'-ib-te	cik'al,	ʡaˁħ	d-irq'-an=da,	b-arx	xːun-ne	k-ercː-an=da''	haʔ-ib-le\\
		\tsc{refl.sg-erg}	\tsc{npl-}do\tsc{.pfv-pret-dd.pl} something	good	\tsc{npl-}do\tsc{.ipfv-ptcp=1}	\tsc{n-}direct	road\tsc{-loc}	\tsc{down}-stand\tsc{.ipfv-ptcp=1}	say\tsc{.pfv-pret-cvb}\\
	\glt	\sqt{‎(He said,) ``The things that I have done, I will repair (make better), I will be on the right road.''}
\end{exe}

The use of reflexive pronouns in quotes referring to overtly expressed speakers that are first person or second person pronouns is ungrammatical:
%
\begin{exe}
	\ex	\label{ex:‎‎‎I said that I am sick}
	\gll	du-l	haʔ-ib=da	``du	/	{*} ca-w	ʡaˁrkːa-l=da''\\
		\tsc{1sg-erg}	say\tsc{.pfv-pret=1}		\tsc{1sg}	/	{} \tsc{refl.sg-m}	ill\tsc{-advz=1}\\
	\glt	\sqt{‎‎‎I said ``I am sick.''} (E)

	\ex	\label{ex:‎You said that you were rich}
	\gll	u-l	haʔ-ib=de	[u	/	{*} ca-w	dawla-či-w-ce	Ø-iχʷ-ni]\\
		\tsc{2sg-erg}	say\tsc{.pfv-pret=2sg}	\tsc{2sg}	/	{ } \tsc{refl.sg-m}	wealth\tsc{-adjvz-m-dd.sg}	\tsc{m-}be\tsc{.pfv-msd}\\
	\glt	\sqt{‎You said that you were rich.} (E)
\end{exe}

With respect to the position of the quote in relation to the utterance predicate we can state that there are four options available:
%
\begin{enumerate}
	\item	predicate - quote \refex{ex:‎‎‎I said that I am sick}
	\item	quote - predicate \refex{ex:‎‎Look!, said the wind}
	\item	quote - predicate - quote \refex{ex:I said, He does not have the time to go there}
	\item	matrix clause constituent(s) - quote - predicate \refex{ex:They asked us, where did you come from}
\end{enumerate}

The first and the second option prevail among the examples from the Sanzhi corpus that have been presented in this section. Instances of a matrix utterance verb followed by the quote can be found in \refex{ex:‎‎The wife says, This is enough, get up}, \refex{ex:‎‎Then they asked the wolf, When were you born?}, \refex{ex:‎His wife begged him, Do not go}, and \refex{ex:‎‎‎I said that I am sick}, and the reverse order is attested in \refex{ex:He says, I do not need these things}, \refex{ex:‎‎‎I bought 99 goats, said Ali}, and \refex{ex:‎‎Look!, said the wind}. The third option means that the quote is interrupted by the verb of speech. The constituent that follows is typically a focused item that is newly introduced \refex{ex:‎‎In once place, there are, he says, trees. Whatever may happen, do not look at these trees} or, more frequently, a contrastive topic that is stressed and emphasized \refex{ex:‎‎‎It is necessary that he must come, at 8 he must be there, tell him this, I said}, \refex{ex:I said, He does not have the time to go there}. This type of constituent order is unattested for all other kinds of complement clauses that have been discussed in the previous sections and only found with reported speech.
%
\begin{exe}
	\ex	\label{ex:I said, He does not have the time to go there}
	\gll	``hextːu	uq'-ij	zamana	b-akːu''	haʔ-ib=da 	``heχ-i-la''\\
		there.\tsc{up}	go\tsc{.m.pfv-inf}	time	\tsc{n-cop.neg}	say\tsc{.pfv-pret=1}	\tsc{dem.down-obl-gen}\\
	\glt	\sqt{I said ``He does not have the time to go there.''}
\end{exe}

The position of the quotative particles is mostly at the right edge of the quote, which can easily be explained by their origin. Since they are transparently derived from converbs, they occupy the most common position of converbs in adverbial clauses, that is, the final position (see \refsec{sec:The syntax of adverbial clauses} for the constituent order in adverbial clauses). However, occasionally one finds examples in which the quotative particle occurs within the quote as in the following sentence \refex{ex:He is telling that there is something there}. Example \refex{ex:He is telling that there is something there} can be analyzed in analogy to \refex{ex:I said, He does not have the time to go there} with the only difference being that in \refex{ex:I said, He does not have the time to go there} the matrix predicate separates the contrastive topic from the rest of the quote, whereas in \refex{ex:He is telling that there is something there} it is the quotative particle that is followed by the contrastive topic.
%
\begin{exe}
	\ex	\label{ex:He is telling that there is something there}
	\gll	hek'-i-l	b-urs-ul	ca-b	``ce=jal	te-d''	Ø-ik'-ul	``hetːu-d''\\
		\tsc{dem.up-obl-erg}	\tsc{n-}tell\tsc{.pfv-icvb}	\tsc{cop-n}	what\tsc{=indef}	exist.\tsc{away-npl}		\tsc{m-}say\tsc{.ipfv}-\tsc{icvb}	there-\tsc{npl}\\
	\glt	\sqt{He is telling that there is something there (i.e. to steal).}
\end{exe}


%%%%%%%%%%%%%%%%%%%%%%%%%%%%%%%%%%%%%%%%%%%%%%%%%%%%%%%%%%%%%%%%%%%%%%%%%%%%%%%%

\section{The syntactic properties of complement clauses}
\label{sec:The syntactic properties of complement clauses}

All complement taking predicates in Sanzhi that have been analyzed so far occur in the position of objects, i.e., patients or stimuli. So far I did not find complement-taking predicates for which the complement clause is required to function as a subject-like argument.

Complement clauses show many overlaps in their structure with the other types of subordinate clauses (relative clauses, adverbial clauses), as the following paragraphs will make clear.
The argument structure of complement clauses is like that of main clauses: all arguments are retained and adjuncts can be freely expressed. Verbs in complement clauses retain the distinction between imperfective and perfective aspect because this is expressed through the stem and there are no restrictions on negation or on word formation, i.e., all types of derived or compound verbs can be used \refex{ex:‎‎‎Then the wind understood that he would not be able to take off the coat of this person}.

The number of verbal categories expressed depends on the complementation strategy. Zero-marked complements and those bearing the embedded interrogative enclitic or containing quotative particles express the same number of categories as main clauses, i.e., person marking, TAM marking, and illocutionary force marking are fully retained. For all other strategies (converbs, cross-categorical suffixes \tit{-ce} and \tit{-il}, infinitive, subjunctive, masdar, case marking, and postpositions) the number of categories expressed in the complement clause is smaller than in the main clause. For instance, marking for illocutionary force and for person is excluded (except for the subjunctive with its rudimentary person paradigm). Tense marking is largely impossible except for the opposition preterite participle vs. modal participle, which functions as a basic distinction between past time reference and everything else as a number of examples in another section of this chapter show (\refsec{ssec:Case marking and postpositions}) and the following elicited minimal pair illustrates:
%
\begin{exe}
	\ex	\label{ex:‎I know that he will come / came}
	\gll	dam	b-alχ-a-d	[it	s-erʁ-an-ce	/	sa-jʁ-ib-ce]\\
		\tsc{1sg.dat}	\tsc{n-}know\tsc{.ipfv-hab.pst-1}	that	\tsc{hither}-come\tsc{.m.ipfv-ptcp-dd.sg}	/	\tsc{hither}-come\tsc{.m.pfv-pret-dd.sg}\\
	\glt	\sqt{‎I know that he will come\slash came.} (E)
\end{exe}

The constituent order in complement clauses is more frequently verb final than in main clauses, but this is not a strict requirement, e.g. \refex{ex:‎The boy and the frog began to search for the frog}. In order to make some preliminary generalizations with respect to the position of the complement clause, I counted all non-elicited complement constructions in this section whose structure is unambiguous and which do not represent reported speech (see \refsec{ssec:Formal marking in reported speech constructions} above for the position of the quote in reported speech constructions). The total number is 54, among which one half has the order \textit{matrix verb-complement}, and the other half has the reverse order. Within this data, there is a very small tendency to have the order \textit{verb-complement} the longer the complement is, but this needs further research. It is rare for the complement clauses to be center-embedded into the matrix clause, but two sentences in this section belong to this category, e.g. \refex{ex:He did not like when one whistled}.

What concerns co-reference across the complement and the main clause, Sanzhi has complement control constructions with obligatory subject omission in the complement clause if the latter is headed by the infinitive (or subjunctive). For the details see \refsec{sec:Argument control in complement constructions} below. In case of co-referential arguments, the overt argument normally occurs in the matrix clause (e.g. \refex{ex:(I) am ashamed to go to the village, said the father} among many others). Occasionally, one can find examples that might look like they are contradicting this claim \refex{ex:‎‎‎One time, guys, we had for some reason to go to say condolences}. The matrix clause in \refex{ex:‎‎‎One time, guys, we had for some reason to go to say condolences} contains an adverbial \textit{ʡaˁʁunil} `necessarily, needed', and if we assume that there is an absent argument in this clause that shares the reference with the subject in the complement, then this argument bears the semantic role of a beneficiary or some other role similar to an ethical dative. In other words, it is not a subject or subject-like argument.
%
\begin{exe}
	\ex	\label{ex:‎‎‎One time, guys, we had for some reason to go to say condolences}
	\gll	ca	zamana,	durħ-ne,	[nušːa	cellij	ʡaˁlħaˁm-li-j	d-uˁq'-ij]	ʡaˁʁuni-l=de\\
		one	time		boy\tsc{-pl}	\tsc{1pl}	why	condolence\tsc{-obl-dat}	\tsc{1/2pl-}go\tsc{.pfv-inf}	needed\tsc{-advz=pst}\\
	\glt	\sqt{‎‎‎One time, guys, we had for some reason to go to offer condolences.}
\end{exe}

In a contrastive context, in which arguments are compared to each other, it is possible to add a subject to a infinitival clause in a control construction:
%
\begin{exe}
	\ex	\label{ex:‎I want to make the tart than another person}
	\gll	dam	b-ikː-ul=da	[du-l	tort	b-arq'-ij	cara-lli-ja-r]\\
		\tsc{1sg.dat}	\tsc{n-}want\tsc{.ipfv-icvb=1}	\tsc{1sg-erg}	tart \tsc{n-}do\tsc{.pfv-inf}	other\tsc{-obl-loc-abl}\\
	\glt	\sqt{‎I want to make the tart (rather) than another person (making the tart).} (E)
\end{exe}

Co-reference between third person arguments, most notably between the subject in the matrix clause and any argument or adjunct in the complement clause, is expressed by the use of reflexive pronouns. For example, the omitted subject in \refex{ex:He is sitting and remembering how (they) beat him up} shares the referent with the goal argument in the complement, which is encoded by the reflexive pronoun in the dative case. Other instances can be found in \refex{ex:He remembered what he had done attributive markers}, in which the agent in the complement is co-referential, and in \refex{ex:‎‎and thought of what he had said and done}, in which the possessor is co-referential.
%
\begin{exe}
	\ex	\label{ex:He is sitting and remembering how (they) beat him up}
	\gll	[cini-j	d-aˁq-ib-te=ra]	han	d-irk-ul,	ca-w=ra	ka-jž-ib	ca-w\\
		\tsc{refl.sg.obl-dat} \tsc{npl-}hit\tsc{.pfv-pret-dd.pl=add}	remember	\tsc{npl-}occur\tsc{.ipfv-icvb}	\tsc{refl-m=add}	\tsc{down}-be\tsc{.m.pfv-pret}	\tsc{cop-m}\\
	\glt	\sqt{He is sitting and remembering how (they) beat him up.}
\end{exe}

Multiple embeddings are possible though rare in natural texts. Relevant instances are \refex{ex:‎‎‎Then the wind understood that he would not be able to take off the coat of this person}, which is a translation from Russian, and \refex{ex:‎I thought that she cannot walk}. Both examples have \sqt{be able} in the complement in the middle and therefore an infinitive in the most deeply embedded clause. Other examples in this section illustrate reported speech that it is itself complex containing a complement clause \refex{ex:(I) am ashamed to go to the village, said the father}. Example \refex{ex:‎‎‎I did not know that you know that Abdul was at our place} has been elicited and shows that multiple embedding is allowed with complementation strategies other than the infinitive.
%
\begin{exe}
	\ex	\label{ex:‎‎‎I did not know that you know that Abdul was at our place}
	\gll	dam	a-b-alχ-ul=de	[at	b-alχ-an-ce	[nišːa-la	qili-w	ʡaˁbdul	le-w-ce]]\\
		\tsc{1sg.dat}	\tsc{neg-n-}know\tsc{.ipfv-icvb=pst}	\tsc{2sg.dat}	\tsc{n-}know\tsc{.ipfv-ptcp-dd.sg}	\tsc{1pl-gen}	home\tsc{-m}	Abdul	exist-\tsc{m-dd.sg}\\
	\glt	\sqt{‎‎‎I did not know that you know that Abdul was at our place.} (E)
\end{exe}

Matrix predicates that have agreement prefixes and non-absolutive arguments usually exhibit local agreement in which the matrix verb agrees with the complement clause as a whole and therefore has the prefix \tit{b-} (neuter singular) \refex{ex:‎I thought that she cannot walk}. This prefix can also be considered to be the default prefix when there is no agreement controller (see \refsec{sec:Gender/number agreement}).
%
\begin{exe}
	\ex	\label{ex:‎I thought that she cannot walk}
	\gll	dam	han	b-ič-ib	[[r-aš-ij]	r-irχ-ul	akːu	r-ik'-ul]\\
		\tsc{1sg.dat}	seem	\tsc{n-}occur\tsc{.pfv-pret}	\tsc{f-}go\tsc{-inf} \tsc{f-}be.able\tsc{.ipfv-icvb}	\tsc{cop.neg}	\tsc{f-}say\tsc{.ipfv-icvb}\\
	\glt	\sqt{‎I thought that she cannot walk.}
\end{exe}

Sanzhi has, in principle, long-distance agreement in gender and number between the matrix predicate and the absolutive argument in the complement clause \refex{ex:You wanted to cry1}, \refex{ex:Aminat finished to wash / washing the clothes}, \refex{ex:‎I got happy when I saw that Nursijat recovered and is (already) walking around} (see also \refsec{General remarks on gender/number agreement}). But in contrast to other East Caucasian languages in which this is a relative common construction (e.g. Tsezic languages, see \citealp{Polinsky.Potsdam2001} and \citealp{Polinsky2003} for Tsezic, and \citealp[628\tnd639]{Forker2013a} for Hinuq and further references), long-distance agreement is almost unattested in the Sanzhi Dargwa corpus. 
%
\begin{exe}
	\ex	\label{ex:You wanted to cry1}
	\gll	ašːij	b-ikː-ul=de	[d-isːu-tːaj	/	d-isː-ij]\\
		\tsc{2pl.dat}	\tsc{n-}want\tsc{.ipfv-icvb=pst}	\tsc{1/2pl-}cry\tsc{-subj.2}	/	\tsc{1/2pl-}cry\tsc{-inf}\\
	\glt	\sqt{You wanted to cry.} (E)
\end{exe}

As in other varieties of Dargwa \citep{Serdobolskaya2010}, only a few complement-taking predicates allow for long-distance agreement, most notably \sqt{want} \refex{ex:You wanted to cry1}, \sqt{know} \refex{ex:‎Which food do you know to cook}, \sqt{finish} \refex{ex:Aminat finished to wash / washing the clothes}, and \sqt{see} \refex{ex:‎I got happy when I saw that Nursijat recovered and is (already) walking around}, \refex{ex:‎S/he saw that mother milked the cows}. The complement clauses must be of the potential type or of the activity type and can only contain the infinitive\slash subjunctive \refex{ex:You wanted to cry1}, the imperfective converb \refex{ex:‎S/he saw that mother milked the cows}, or the perfective converb \refex{ex:Aminat finished to wash / washing the clothes}. 
%
\begin{exe}
	\ex	\label{ex:‎Which food do you know to cook}
	\gll	[ceʁuna	χurejg	d-arq'-ij]	b-alχ-atːe	/	d-alχ-atːe	at?\\
		which	food(\tsc{npl})	\tsc{npl-}do\tsc{.pfv-inf}	\tsc{n-}know\tsc{.ipfv-prs.2}	/	\tsc{npl-}know\tsc{.ipfv-prs.2}	\tsc{2sg.dat}\\
	\glt	\sqt{‎Which food do you know how to cook?} (E)
\end{exe}

Serdobolskaya \citeyearpar{Serdobolskaya2010, Serdobolskaya2009} argues that in Xuduc and Qunqi Dargwa long-distance agreement can be analyzed as clause reduction (clause union) that shares many properties with raising constructions in other languages. She shows that complement constructions with embedded subjunctives/infinitives or converbal clauses have some monoclausal properties. This seems to be true for Sanzhi as well. For instance, arguments of embedded infinitival clauses can easily occur in a clause-final position that can hardly belong to the embedded clause \refex{ex:The father and the mother want to give medicine of the son, to the son.}.

\begin{exe}
	\ex	\label{ex:The father and the mother want to give medicine of the son, to the son.}
	\gll	atːa-j=ra	aba-j=ra	[darman	b-arq'-ij]	b-ikː-ul	ca-b	durħuˁ-la,		durħuˁ-li-j\\
		father-\tsc{dat=add}	mother-\tsc{dat=add}	medicine	\tsc{n}-do.\tsc{pfv-inf}	\tsc{n}-want.\tsc{ipfv-icvb}	be-\tsc{n}	boy-\tsc{gen}	boy-\tsc{obl-dat}\\
	\glt	\sqt{The father and the mother want to give medicine of the son, to the son.} (the speaker corrected herself)
\end{exe}

The pragmatic effect of long-distance agreement is sometimes described as highlighting the argument that serves as agreement controller, but before being able to make more specific claims about its impact on information structure in Sanzhi Dargwa more research is needed. 


%%%%%%%%%%%%%%%%%%%%%%%%%%%%%%%%%%%%%%%%%%%%%%%%%%%%%%%%%%%%%%%%%%%%%%%%%%%%%%%%

\section{Argument control in complement constructions}
\label{sec:Argument control in complement constructions}

Complement constructions in Sanzhi show heterogeneous behavior with respect to control of the obligatorily omitted argument. Complements of the verb \tit{b-aʔašː-} \sqt{begin} can be headed by the imperfective converb (\refsec{ssec:The imperfective converb}) or by the infinitive\slash subjunctive (\refsec{ssec:Infinitive and subjunctive}). The controller, i.e., the one who begins something, must be in the absolutive. The controllee can be the single argument of an intransitive verb or the most prominent argument of a two-place verb as the following examples show:
%
\begin{exe}
	\ex	\label{ex:Madina, Murad, snakes, teacher@42}
	\begin{xlist}
		\ex	\label{ex:Madina began to laugh@42a}
		\gll	Madina	r-aʔ.ašː-ib	[ \_	ħaˁħaˁ	r-ik'-ul]\\
			Madina	\tsc{f-}begin\tsc{-pret}	{} \tsc{abs}		laughter	\tsc{f-}say\tsc{.ipfv-icvb}\\
		\glt	\sqt{Madina began to laugh.} (controllee = S) (E)

		\ex	\label{ex:Murad began to kill snakes@42b}
		\gll	Murad	w-aʔ.ašː-ib	[ \_	maˁlʡuˁn-te	kerx-ul]\\
			Murad	\tsc{m-}begin\tsc{-pret}	{} \tsc{erg}	snake\tsc{-pl}	kill\tsc{-icvb}\\
		\glt	\sqt{Murad began to kill snakes.} (controllee = A) (E)

		\ex	\label{ex:Murad began to understand the teacher@42c}
		\gll	Murad	w-aʔ.ašː-ib	[ \_	maˁʡaˁlim	čirʁ-ij]\\
			Murad	\tsc{m-}begin\tsc{-pret}	{} \tsc{dat}	teacher	understand\tsc{-inf}\\
		\glt	\sqt{Murad began to understand the teacher.} (controllee = EXP) (E)
	\end{xlist}
\end{exe}

The controllee can never be the second argument of a two-place verb such as the patient \refex{ex:The snakes began to be killed by Murad@43a} or the stimulus \refex{ex:The teacher began to be understood by Murad@43bA}. 
%
\begin{exe}
	\ex	\label{ex:Snakes, Murad, teacher ungrammatical@43}
	\begin{xlist}
		\ex	\label{ex:The snakes began to be killed by Murad@43a}
		\gll	{*}	maˁlʡuˁn-te$_{i}$	d-aʔ.ašː-ib	[Murad-li	\_$_{i}$	kerx-ul]\\
			{}	snake\tsc{-pl}	\tsc{npl-}begin\tsc{-pret}	Murad\tsc{-erg}	\tsc{abs}	kill\tsc{-icvb}\\
		\glt	(Intended meaning: \sqt{The snakes began to be killed by Murad}.) (E)

		\ex	\label{ex:The teacher began to be understood by Murad@43bA}
		\gll	{*}	maˁʡaˁlim$_{i}$	w-aʔ.ašː-ib	[Murad-li-j	\_$_{i}$	čirʁ-ij]\\
			{}	teacher	\tsc{m-}begin\tsc{-pret}	Murad\tsc{-obl-dat}	\tsc{abs}	understand\tsc{-inf}\\
		\glt	(Intended meaning: \sqt{The teacher began to be understood by Murad}.) (E)
	\end{xlist}
\end{exe}

But if we look at bivalent complement-taking predicates, the situation is slightly different. With the matrix verb \tit{b-ikː-} \sqt{want} the complement clause contains either an infinitive or a subjunctive. The controllee can be the subject of an intransitive verb \refex{ex:‎Not wanting to leave, he left Sanzhi as the very last}. But it can also be any of the arguments of a two-place verb (e.g. agent or patient), depending on the verb form in the complement clause. With subject-like controllees the embedded verb takes the infinitive suffix \refex{ex:Murad did not want to push Ali@44}, \refex{ex:‎I wanted to give the book to Khadizhat}.
%
\begin{exe}
	\ex	\label{ex:Murad did not want to push Ali@44}
	\gll	Murad-li-j$_{i}$	a-b-ikː-ul=de	[ \_$_{i}$	ʡaˁli	qːurt	w-arq'-ij]\\
		Murad\tsc{-obl-dat}	\tsc{neg-n-}want\tsc{-icvb=pst}	{} \tsc{erg}	Ali	push	\tsc{m-}do\tsc{.pfv-inf}\\
	\glt	\sqt{Murad did not want to push Ali.} (controllee = A) (E)
\end{exe}

However, if the controllee is the second argument of a two-place predicate, then the verb form in the complement clause cannot be the infinitive, but must be the perfective converb \refex{ex:Murad wants Madina to see him@45a}, \refex{ex:He wants that I write the letter@45b}. The infinitive can only occur when the experiencer of \sqt{want} is controlling a subject-like argument in the complement clause. In \refex{ex:He wants that I write the letter@45b} both verbs have different arguments, and the embedded verb cannot bear the infinitive suffix.
%
\begin{exe}
	\ex	\label{ex:Murad, Madina, letter@45}
	\begin{xlist}
		\ex	\label{ex:Murad wants Madina to see him@45a}
		\gll	Murad-li-j$_{i}$	b-ikː-ul ca-b	[Madina-j	\_$_{i}$	či-w-až-ib-le]\\
			Murad\tsc{-obl-dat}	\tsc{n-}want\tsc{-cvb} \tsc{cop-n}	 Madina\tsc{-dat}	\tsc{abs}	\tsc{spr-m-}see\tsc{.pfv-pret-cvb}\\
		\glt	\sqt{Murad$_{i}$ wants Madina to see him$_{i}$.} (controllee = P) (E)

		\ex	\label{ex:He wants that I write the letter@45b}
		\gll	it-i-j	b-ikː-ul ca-b	[du-l	kaʁar 	b-elk'-un-ne	/ 	{*}	b-elk'-ij]\\
			that\tsc{-obl-dat}	\tsc{n-}want\tsc{.ipfv-icvb} \tsc{cop-n}	\tsc{1sg-erg}	letter	\tsc{n-}write\tsc{.pfv-pret-cvb}	/	{} \tsc{n-}write\tsc{.pfv-inf}\\
		\glt	\sqt{He wants that I write the letter.} (E)
	\end{xlist}
\end{exe}

The same phenomenon is observed with another complement-taking predicate, \tit{uruχle cab-} \sqt{fear}. If the controllee is a subject-like argument, the complement clause is headed by an infinitive \refex{ex:Khamis fears to see the bear@46a}. Otherwise the attributive suffix \tit{-ce} is employed \refex{ex:Khamis fears that Madina pushes her@46b}, \refex{ex:Ali feared that Madina would not recognize / know him@46c}.
%
\begin{exe}
	\ex	\label{ex:Khamis, bears, Madina, Ali@46}
	\begin{xlist}
		\ex	\label{ex:Khamis fears to see the bear@46a}
		\gll	χamis$_{i}$	uruχ-le ca-r	[ \_$_{i}$	sːika	či-b-až-ij]\\
			Khamis	fear\tsc{-advz} \tsc{cop-f}	{} \tsc{dat}	bear	\tsc{spr-n-}see\tsc{.pfv-inf}\\
		\glt	\sqt{Khamis fears to see the bear.} (controllee = A) (E)

		\ex	\label{ex:Khamis fears that Madina pushes her@46b}
		\gll	χamis$_{i}$	uruχ-le ca-r	[Madina-l	\_$_{i}$	qːurt	r-irq'-an-ce]\\
			Khamis	fear\tsc{-advz} \tsc{cop-f}	Madina\tsc{-erg}	\tsc{abs}	push	\tsc{f-}do\tsc{.ipfv-ptcp-dd.sg}\\
		\glt	\sqt{Khamis fears that Madina pushes her.} (controllee = P) (E)

		\ex	\label{ex:Ali feared that Madina would not recognize / know him@46c}
		\gll	ʡaˁli$_{i}$	uruχ-le=de	[Madina-j	\_$_{i}$	a-w-aχ-ur-ce]\\
			Ali	fear\tsc{-advz=pst}	Madina\tsc{-dat}	\tsc{abs}	\tsc{neg-m-}know\tsc{.pfv-pret-dd.sg}\\
		\glt	\sqt{Ali feared that Madina would not recognize\slash know him.}  (controllee~=~P) (E)
	\end{xlist}
\end{exe}

It seems that with trivalent matrix verbs there is no such difference between the treatment of subject controllees on the one hand and object controllees on the other hand. Both types are allowed and the embedded verb forms are identical \xxref{ex:Father sent the daughter to the university to study@47a}{ex:Father sent the sheep in order to be slaughtered by the uncle@47c}. 
%
\begin{exe}
	
		\ex	\label{ex:Father sent the daughter to the university to study@47a}
		\gll	atːa-l	rursːi$_{i}$	uniwersitet-le	[ \_$_{i}$	r-uč'-ij]	r-ataʁ-ib\\
			father\tsc{-erg}	girl	university\tsc{-loc}	{} \tsc{abs}	\tsc{f-}learn\tsc{-inf}	\tsc{f-}let\tsc{.pfv-pret}\\
		\glt	\sqt{Father sent the daughter to the university to study.} (controllee = S) (E)

		\ex	\label{ex:Mother sent the son to cut firewood@47b}
		\gll	aba-l	durħuˁ$_{i}$	w-ataʁ-ib	[ \_$_{i}$	urcul	d-alʁ-ij]\\
			mother\tsc{-erg}	boy	\tsc{m-}let\tsc{.pfv-pret}	{} \tsc{erg} 	wood	\tsc{npl-}cut\tsc{.pfv-inf}\\
		\glt	\sqt{Mother sent the son to cut firewood.} (controllee = A) (E)

		\ex	\label{ex:Father sent the sheep in order to be slaughtered by the uncle@47c}
		\gll	atːa-l	macːa$_{i}$	b-ataʁ-ib	[acːi-l	\_$_{i}$	b-elχʷ-ij	/	b-elχʷ-anaj]\\
			father\tsc{-erg}	sheep	\tsc{n-}let\tsc{.pfv-pret}	uncle\tsc{-erg}	\tsc{abs}	\tsc{n-}slaughter\tsc{.pfv-inf}	/ \tsc{n-}slaughter\tsc{.pfv-subj.3}\\
		\glt	\sqt{Father sent the sheep in order to be slaughtered by the uncle.} (controllee = P) (E)
	
\end{exe}

However, this again can be interpreted as a difference in the treatment of subject-like vs. object-like arguments, but now regarding the controller, not the controllee. If the controller is the subject, then the verb form in the complement clause depends on whether the controllee is the object or also the subject. If the controller is the object, then, in contrast, no such difference in the verb form is noticed. To sum up, in complement control we have some indication of an S/A pivot. There are no clause level conditions and at least for the tested complement-taking predicates no difference in the treatment of embedded predicates could be observed. The predicate class of the embedded verb is possibly a decisive feature that needs to be studied in more detail in the future since for other East Caucasian languages it has been observed that intransitive, canonical transitive, and affective verbs are treated differently in some complement constructions  \citep{Kibrik2003}.

Finally, I will briefly discuss backward control. This notion refers to complement constructions in which the overt controller appears in the embedded clause, and thus its case is assigned by the embedded verb. Nevertheless, the matrix verb shows agreement with the controller. On the surface these constructions look as if the verb is agreeing with a non-absolutive argument. But instead it is argued that the matrix verb contains a covert controllee in the absolutive case that is co-referential with the overt nominal in the non-absolutive case. Backward control is found in other East Caucasian languages, see, e.g. \citea{Polinsky.Potsdam2002} (\citey{Polinsky.Potsdam2002, Polinsky.Potsdam2006}) on Tsez, and \citet{Serdobolskaya2010} on Qunqi Dargwa, and is typically restricted to a few modal and phasal predicates.

In Sanzhi Dargwa, there are two verbs that allow for backward control, \tit{-b-iχʷ-} (\tsc{pfv})\slash\tit{b-irχʷ-} (\tsc{ipfv}) \sqt{can, be able} and \tit{b-aʔ axː-} (\tsc{pfv}), \tit{b-aʔ b-išː-} (\tsc{pfv})\slash\tit{b-aʔ b-irxː-} \sqt{begin, start}. The verb \sqt{can, be able} is far more readily available. In the Sanzhi corpus, backward control is only attested with the verb \sqt{can, be able}, but can be obtained with \sqt{begin, start} in elicitation. In standard forward control constructions, the two verbs require subject-like arguments in the absolutive case that control gender (and person) agreement just like intransitive verbs:
%
\begin{exe}
	\ex	\label{ex:‎The girl will be able to sew the dress}
	\gll	rursːi	[kːurtːi	b-arχ-ij]	r-irχʷ-an-ne\\
		girl	dress	\tsc{n-}sew\tsc{.pfv-inf}	\tsc{f-}be.able\tsc{.ipfv-ptcp-fut.3}\\
	\glt	\sqt{‎The girl will be able to sew the dress.} (E)

	\ex	\label{ex:‎The girl began to eat the pilau}
	\gll	rursːi	[palaw	b-uk-unne]	r-aʔ	r-išː-ib	/	r-aʔ.ašː-ib\\
		girl	pilaw	\tsc{n-}eat\tsc{.ipfv-icvb} \tsc{f-}begin	\tsc{f-}become\tsc{.pfv-pret}	/ \tsc{f-}begin\tsc{.pfv-pret}\\
	\glt	\sqt{‎The girl began to eat the pilaw.} (E)
\end{exe}

In backward control constructions the clauses contain subject-like arguments in the ergative case that has been assigned by the embedded verb. This means that in both \refex{ex:‎The girl will be able to sew the dress} and \refex{ex:‎The girl began to eat the pilau} the subject argument bears the ergative case because the embedded verbs are transitive.
%
\begin{exe}
	\ex	\label{ex:‎The girl will be able to sew the dress}
	\gll	[rursːi-l	kːurtːi	b-arχ-ij]	r-irχʷ-an-ne\\
		girl\tsc{-erg}	dress	\tsc{n-}sew\tsc{.pfv-inf}	\tsc{f-}be.able\tsc{.ipfv-ptcp-fut.3}\\
	\glt	\sqt{‎The girl will be able to sew the dress.} (E)

	\ex	\label{ex:‎The girl began to eat the pilau}
	\gll	[rursːi-l	palaw	b-uk-unne]	r-aʔ	r-išː-ib	/	r-aʔ.ašː-ib\\
		girl\tsc{-erg} pilau	\tsc{n-}eat\tsc{.ipfv-icvb} \tsc{f-}begin	\tsc{f-}become\tsc{.pfv-pret}	/ \tsc{f-}begin\tsc{.pfv-pret}\\
	\glt	\sqt{‎The girl began to eat the pilau.} (E)
\end{exe}

Backward control is only available with embedded transitive verbs. Affective verbs do not allow for this construction. There are two cases that look like apparent exceptions. In example \refex{ex:‎One beautiful young man fell in love with a young girl}, the experiencer argument occurs in the dative, assigned by the affective verb \tit{b-ikː-} \sqt{want, like, love} appearing as imperfective converb, whereas the finite verb is \tit{haq-}, which usually translates as \sqt{manage, be enough}. Thus, one might suspect that \tit{haq-} functions as a matrix complement-taking verb into which a complement clause headed by \tit{b-ikː-} has been embedded together with both the experiencer and the stimulus argument. However, \tit{b-ikː-ul haq-} rather functions as a lexicalized periphrastic predicate and the construction is monoclausal. The verb \tit{b-ikː-ul} cannot be replaced by any other verb and the semantics of the periphrastic predicate is not transparently composed of the semantics of the individual predicates.
%
\begin{exe}
	\ex	\label{ex:‎One beautiful young man fell in love with a young girl}
	\gll	ca	qːuʁa	žahil	durħuˁ-li-j	r-ikː-ul	haq-ib	ca-r	žahil	rursːi\\
		one	beautiful	young	boy\tsc{-obl-dat}	\tsc{f-}want\tsc{.ipfv-icvb}	manage\tsc{.pfv-pret}	\tsc{cop-f}	young	girl\\
	\glt	\sqt{‎One beautiful young man fell in love with a young girl.} 
\end{exe}

The second apparent exception is the use of affective predicates that are usually bivalent as monovalent predicates. This is possible with \sqt{see}, which then has the meaning \sqt{be\slash become visible} and \sqt{hear}, which then means \sqt{be\slash become audible}. Thus, the dative nominals in the following two examples can be left out, such that we end up with intransitive constructions. The agreement on the verb \sqt{begin} is controlled by the absolutive arguments, not by the dative experiencers, which is an unambiguous indication that we do not deal with backward control with an embedded bivalent affective verb, but with an intransitive complement clause in a construction to which a dative adjunct has been added.
%
\begin{exe}
	\ex	\label{ex:‎The song began to be audible}
	\gll	dalaj	[t'am	b-iq'-ul]	b-aʔ	b-išː-ib\\
		song	sound	\tsc{n-}hear\tsc{.ipfv-icvb}	\tsc{n-}begin	\tsc{n-}become\tsc{.pfv-pret}\\
	\glt	\sqt{‎The song began to be audible.} (E)

	\ex	\label{ex:‎The girl began to hear the song}
	\gll	rursːi-j	dalaj	t'am	b-iq'-ul	b-aʔ	b-išː-ib\\
		girl\tsc{-dat}	song	sound	\tsc{n-}hear\tsc{.ipfv-icvb}	\tsc{n-}begin	\tsc{n-}become\tsc{.pfv-pret}\\
	\glt	\sqt{‎The girl began to hear the song.} OR \sqt{The song began to be audible to the girl.} (E)

	\ex	\label{ex:‎The mountains started to be visible}
	\gll	dubur-te	[či-d-ig-ul]	d-aʔ.ašː-ib\\
		mountain\tsc{-pl}	\tsc{spr-npl-}see\tsc{.ipfv-icvb}	\tsc{npl-}begin\tsc{.pfv-pret}\\
	\glt	\sqt{‎The mountains started to be visible.} (E)

	\ex	\label{ex:‎I began to see the mountains}
	\gll	dam	dubur-te	či-d-ig-ul	d-aʔ.ašː-ib\\
		\tsc{1sg.dat}	mountain\tsc{-pl}	\tsc{spr-npl-}see\tsc{.ipfv-icvb}	\tsc{npl-}begin\tsc{.pfv-pret}\\
	\glt	\sqt{‎I began to see the mountains.} OR \sqt{The mountains started to be visible to me.} (E)
\end{exe}

If the constructions were truly biclausal, we would expect restrictions on the constituent order, since items of one clause should normally not be allowed to appear within the other clause. This is precisely what we find with \sqt{begin, start}. In a backward control construction, the ergative argument must occur within the complement clause; it cannot be positioned clause-initially if it is followed by the matrix predicate. This is only possible in forward control since then the argument is governed by the matrix predicate:
%
\begin{exe}
	\ex	\label{ex:‎I began to read the book}
	\gll	du	r-aʔ	r-išː-ib=da	[kiniga	b-elč'-ij]\\
		\tsc{1sg}	\tsc{f-}begin	\tsc{f-}become\tsc{.pfv-pret=1}	book	\tsc{n-}read\tsc{.pfv-inf}\\
	\glt	\sqt{‎I began to read the book.}

	\ex	\label{ex:I began to read the book ungrammatical}
	\gll	{*} 	du-l	r-aʔ	r-išː-ib=da	[kiniga	b-elč'-ij]\\
		{}	\tsc{1sg-erg}	\tsc{f-}begin	\tsc{f-}become\tsc{.pfv-pret=1}	book	\tsc{n-}read\tsc{.pfv-inf}\\
	\glt	‎ (Intended meaning: \sqt{I began to read the book.})
\end{exe}

For the verb \sqt{be, become, happen, can, be able} the data are not so clear. Some examples show a very flexible word order, which points towards a monoclausal analysis with a periphrastic predicate:
%
\begin{exe}
	\ex	\label{ex:‎The girl will be able to sew the dress}
	\gll	rursːi-l	r-irχʷ-an-ne kːurtːi	b-arχ-ij \\
		girl\tsc{-erg}	\tsc{f-}be.able\tsc{.ipfv-ptcp-fut.3}	dress	\tsc{n-}sew\tsc{.pfv-inf}\\
	\glt	\sqt{‎The girl will be able to sew the dress.} (E)

	\ex	\label{ex:‎The girl will be able to sew the dress}
	\gll	kːurtːi	b-arχ-ij	r-irχʷ-an-ne	rursːi-l\\
		dress	\tsc{n-}sew\tsc{.pfv-inf}	\tsc{f-}be.able\tsc{.ipfv-ptcp-fut.3}	girl\tsc{-erg}\\
	\glt	\sqt{‎The girl will be able to sew the dress.} (E)
\end{exe}

Other examples have been rejected by speakers. Further research is needed to provide a more detailed account of the properties of backward control in Sanzhi.



%%%%%%%%%%%%%%%%%%%%%%%%%%%%%%%%%%%%%%%%%%%%%%%%%%%%%%%%%%%%%%%%%%%%%%%%%%%%%%%%
\section{Constructions that semantically resemble complement clauses}
\label{Constructions that semantically resemble complement clauses}

% --------------------------------------------------------------------------------------------------------------------------------------------------------------------------------------------------------------------- %

\subsection{Parentheticals}
\label{ssec:Parentheticals}

There are three particles and phrases that refer to cognitive activities and are used in a way that resembles complement clauses with cognition predicates. The first is the frequently used phrase \textit{... pikri ħaˁsible} \sqt{according to the thoughts of (somebody)}, which consists of a possessor followed by the two borrowed items \textit{pikri} `thought' and the postposition \textit{ħaˁsib-le} (test-\tsc{advz}) `according to'. The entire construction is a kind of calque that partially consist of loans and partially of Sanzhi morphemes. For the clauses with which the phrase occurs zero marking is the only possible usage option \refex{ex:‎In my mind, he left prison}, \refex{ex:His wife is, in my mind bad like a dog@15}; the use of quotative particles or other complementation markers together with the phrase is ungrammatical. In contrast to the common positions of complement clauses as either following, preceding or occasionally being center-embedded into the matrix clause (see \refsec{sec:The syntactic properties of complement clauses} below), the phrase frequently occurs in the middle of the complement. Taking all these peculiarities together, the phrase has to be characterized as a parenthetical. 
%
\begin{exe}
	\ex	\label{ex:‎In my mind, he left prison}
	\gll	hej,	di-la	pikri	ħaˁsible,	tusnaq-le-r	tːura	uq-un	ca-w	hež\\
		this	\tsc{1sg-gen}	thought following	prison\tsc{-loc-abl}	outside	go\tsc{.m.pfv-pret}	\tsc{cop-m}	this\\
	\glt	\sqt{‎In my mind, he left prison.}

	\ex	\label{ex:His wife is, in my mind bad like a dog@15}
	\gll	hež-i-la	xːunul,	di-la	pikri	ħaˁsible,	χʷe=ʁuna	wahi-ce	ca-r \\
		this\tsc{-obl-gen}	woman	\tsc{1sg-gen}	thought	following	dog\tsc{=eq}	bad\tsc{-dd.sg}	\tsc{cop-f}\\
	\glt	\sqt{His wife is, in my mind, bad like a dog.}
\end{exe}

Note that other complement-taking predicates can also occur in the middle of the complement clause, in which case additional overt marking of the complement by means of particles is forbidden. For instance, in \refex{ex:His wife is, in my mind bad like a dog@15} we can replace \tit{dila pikri ħaˁsible} by \tit{dam han birkul cab} \sqt{seems to me}. This complement-taking predicate normally requires the use of the quotative particle when it occurs before the complement clause.

The phrase \textit{... pikri ħaˁsible} is sometimes replaced by its Russian equivalent \tit{po-moemu}, which is used in the same manner \refex{ex:In my mind, the father beat up the mother@23}. 
%
\begin{exe}
	\ex	\label{ex:In my mind, the father beat up the mother@23}
	\gll	pomoemu,	atːa-l	aba	r-it-ib	ca-r\\
		my.opinion	father\tsc{-erg}	mother	\tsc{f-}beat.up\tsc{-pret}	\tsc{cop-f}\\
	\glt	\sqt{In my mind, the father beat up the mother.}
\end{exe}

Moreover, the particle \tit{aχːu} \sqt{I don't know, dunno} also occasionally occurs as a parenthetical. In \refex{ex:But he is screaming, I don't know, he also rose up his arms@21} there is not only no formal sign of subordination, but not even a clear semantic relationship between \tit{aχːu} and the surrounding clauses, so this is an example of its parenthetical use. However, in the majority of examples the clause accompanying the particle is marked by the embedded question marker =\textit{el} (\refsec{ssec:The embedded question marker}).
%
\begin{exe}
	\ex	\label{ex:But he is screaming, I don't know, he also rose up his arms@21}
	\gll	amma	ʁaˁʁ Ø-ik'-ul ca-w	ik',	aχːu,	nuˁq-be	aq d-arq'-ib	ca-d	ik'-i-l=ra\\
		but	scream \tsc{m-}say\tsc{.ipfv-icvb} \tsc{cop-m}	\tsc{dem.up}	not.know	arm\tsc{-pl}		high \tsc{npl-}do\tsc{.pfv-pret}	\tsc{cop-npl}	\tsc{dem.up}\tsc{-obl-erg=add}\\
	\glt	\sqt{But he is screaming, I don't know, he also rose up his arms.}
\end{exe}



% --------------------------------------------------------------------------------------------------------------------------------------------------------------------------------------------------------------------- %

\subsection{Nominalized relative clauses resembling complement constructions}
\label{ssec:Nominalized relative clauses used with emotion and cognition predicates}
There are a variety of constructions with the predicates listen in \refsec{sec:Complement-taking predicates} above as `complement-taking predicates', which are syntactically not complement clauses. They function as a core argument of a higher clause and have the internal constituent structure of a clause. But they are usually shorter than real complement clauses and contain only one core argument in addition to the verb. Most importantly, they do not refer to propositions, but to entities such as persons, events, etc. They are, therefore, not complement clauses but headless relative clauses, which have been nominalized. Headless relative clauses can be formed with (i) the cross-categorical suffix -\textit{ce} (plural -\textit{te}), the cross-categorical suffix -\textit{il}, and the modal participle -\textit{an}. In this section, I will only discuss headless relative clauses that occur together with complement-taking predicates and show how they differ from true complementation. For general information about headless relative clauses see \refsec{sec:Headless relative clauses}.

As has been described in \refsec{ssec:The attributive marker -ce (-te)COMPL} above, the suffix -\textit{ce} marks complements of the fact-type \refex{‎‎‎I am happy that during the next year we will finish}. In this function, -\textit{ce} can never be replaced by  -\textit{te}, which otherwise function as the plural equivalent of -\textit{ce}. Thus, replacing -\textit{ce} by -\textit{te} in \refex{‎‎‎I am happy that during the next year we will finish} would result in an ungrammatical sentence.

\begin{exe}
		\ex	\label{‎‎‎I am happy that during the next year we will finish}
	\gll	du-l		razi-l=da		[c'il	dus	[nušːa-l	basːejn	b-arq'-ib-le]		ha-b-urχː-an-ce]\\
		\tsc{1sg-erg}	happy\tsc{-advz=1}	then	year	\tsc{1pl-erg}	pool	\tsc{n-}do\tsc{.pfv-pret-cvb}	\tsc{up-n-}finish\tsc{.ipfv-ptcp-dd.sg} \\
	\glt	\sqt{‎‎‎I am happy that during the next year we will finish building the pool.} (E)
\end{exe}
	
By contrast, when the suffix occurs in headless relative clauses, the use of the plural suffix \textit{te} is possible when the referent of the nominalized clause is plural \refex{ex:He remembered what he had done attributive markers}, \refex{ex:‎‎and thought of what he had said and done}. In \refex{ex:‎‎and thought of what he had said and done} the two nominalized relative clauses are conjoined by means of the additive enclitic =\textit{ra}, which is regularly used to conjoin noun phrases (\refsec{sec:Coordination of noun phrases and other phrases}). Examples \refex{ex:He remembered what he had done attributive markers} and \refex{ex:‎‎and thought of what he had said and done} share with genuine complement clauses their occurrence in the argument position of verbs of cognition and their ability to preserve their internal argument structure. For instance, in \refex{ex:He remembered what he had done attributive markers} the agent of the embedded verb is expressed by means of a reflexive pronoun in the ergative case, which is in accordance with the usual case frame required by transitive verbs. However, the embedded clauses are propositions (e.g. they cannot be expressed through a \textit{that}-clause in English). This is particularly clear in  \refex{ex:‎‎and thought of what he had said and done} because in this example the nominalized clauses are, first of all, conjoined like ordinary noun phrases, and second, modified by a possessor, which refers to the agent, and by a quantifier. In a complement clause the agent of the embedded verb is not expressed by a possessor in the genitive, but by a nominal in the ergative case. 


%
\begin{exe}
	\ex	\label{ex:He remembered what he had done attributive markers}
	\gll	[cin-ni	d-arq'-ib-te]	han	d-irč-aq-ul	ca-w	uže\\
		\tsc{refl.sg-erg}	\tsc{npl-}do\tsc{.pfv-pret-dd.pl} remember	\tsc{npl-}occur\tsc{.ipfv-caus-icvb}	\tsc{cop-m}	already\\
	\glt	\sqt{He remembered what he had done.}

	\ex	\label{ex:‎‎and thought of what he had said and done}
	\gll	pikri	uq-un-ne	[cin-na	li<d>il	[d-urs-an-te=ra]	[d-irq'-an-te=ra]] ...\\
		thought	go\tsc{.m.pfv-pret-cvb}	\tsc{refl.sg-gen}	all\tsc{<npl>}	\tsc{npl-}tell\tsc{-ptcp-dd.pl=add}	\tsc{npl-}do\tsc{.ipfv-ptcp-dd.pl=add}\\
	\glt	\sqt{‎‎and thought of what he had said and done ...}
\end{exe}

The same kind of reasoning applies to nominalized clauses with the cross-categorical suffix -\textit{il}. The following two sentences show a minimal pair. The first example illustrates the headless relative clause, which refers to an animal \refex{ex:He got to know who ate (them)@B}. The second clause is a complement construction with a clausal complement \refex{ex:He got to know that (they) ate (them)@A}. 
%
\begin{exe}
	\ex	\label{ex:He learned that they ate them}
	\begin{xlist}
			\ex	\label{ex:He got to know who ate (them)@B}
		\gll	iž-i-j	b-aχ-ur	ca-b	[d-erk-un-il]\\
			this\tsc{-obl-dat}	\tsc{n-}know\tsc{.pfv-pret}	\tsc{cop-n}	\tsc{npl-}eat\tsc{.pfv-pret-ref}\\
		\glt	\sqt{He got to know the one (animal) who ate (them).} (E)
		
		\ex	\label{ex:He got to know that (they) ate (them)@A}
		\gll	iž-i-j	b-aχ-ur	ca-b	[d-erk-ni]\\
			this\tsc{-obl-dat}	\tsc{n-}know\tsc{.pfv-pret}	\tsc{cop-n}	\tsc{npl-}eat\tsc{.pfv-msd}\\
		\glt	\sqt{He got to know that (they) ate (them).} (E)
	\end{xlist}
\end{exe}

The use of nominalized clauses with the -\textit{il} suffix is ungrammatical in constructions that require clausal complements:

\begin{exe}

	\ex	\label{ex:‎‎‎I am happy that during the next year we will finish building the pool}
	\gll	{*} du-l		razi-l=da		[c'il	dus	[nušːa-l	basːejn	b-arq'-ib-le]	ha-b-urχː-an-il]\\
		{} \tsc{1sg-erg}	happy\tsc{-advz=1}	then	year	\tsc{1pl-erg}	pool	\tsc{n-}do\tsc{.pfv-pret-cvb}	\tsc{up-n-}finish\tsc{.ipfv-ptcp-ref} \\
	\glt	(Intended meaning: \sqt{‎‎‎I am happy that during the next year we will finish building the pool.}) (E)
\end{exe}

As has been shown in \refex{ex:‎‎and thought of what he had said and done} for nominalized clauses with -\textit{ce}, nominalized clauses can also be conjoined:

%
\begin{exe}
	\ex	\label{ex:‎I forget what I say and what I do}
	\gll	na	[[b-urs-ib-il=ra]	[b-arq'-ib-il=ra]]	qum.urt-u	dam\\
		now	\tsc{n-}tell\tsc{-pret-ref=add}	\tsc{n-}do\tsc{.pfv-pret-ref=add}	forget\tsc{.ipfv-prs.3}	\tsc{1sg.dat}\\
	\glt	\sqt{‎I forget what I say and what I do.}
\end{exe}


The modal participle regularly occurs in headless relative clauses \refsec{sec:Headless relative clauses}. When the main clause contains a complement-taking predicate as in \refex{ex:‎Brother finished to study / studying / the studies}, the structure seems to be ambiguous between an interpretation as a complement clause of the activity type (`studying') and a nominalized verb that functions as action noun (`the studies'). This type of construction requires future research in order to be able to decide whether the construction is, in fact, ambiguous, or whether we can exclude one of the two potential analyses.
 	
	%
\begin{exe}	
	\ex	\label{ex:‎Brother finished to study / studying / the studies}
	\gll	ucːi-l	taman	b-arq'-ib	[b-uč'-an]\\
		brother\tsc{-erg}	end	\tsc{n-}do\tsc{.pfv-pret}	\tsc{n-}learn\tsc{.ipfv-ptcp}\\
	\glt	\sqt{‎Brother finished studying\slash the studies.} (E)
\end{exe}


In elicitation, I also obtained example \refex{ex:‎I wanted to give the book to Khadizhat}. This looks like a complement clauses of the potential type with the modal participle. At the same time this sentence instantiates a constituent focus construction with a floating predicative particle (the past enclitic =\textit{de}, see \refsec{ssec:Contrastive focus and floating predicative particles} for more information). 

%
\begin{exe}
	\ex	\label{ex:‎I wanted to give the book to Khadizhat}
	\gll	dam	b-ikː-an	[χadižat-li-j=de	kiniga	lukː-an]\\
		\tsc{1sg.dat}	\tsc{n-}want\tsc{.ipfv-ptcp}	Khadizhat\tsc{-obl-dat=pst}	book	give\tsc{.ipfv-ptcp}\\
	\glt	\sqt{‎I wanted to give the book to KHADIZHAT.} or \sqt{It was Khadizhat to whom I wanted to give the book.} (E)
	\end{exe}

Interestingly, it is impossible to position the predicative particle =\textit{de} on its usual host, which would be the verb in the main clause \refex{ex:‎I wanted to give the book to Khadizhat1}. This is only allowed if we simultaneously replace the modal participle with the infinitive, which is the default marker for complement clauses with potential meaning \refex{ex:‎I wanted to give the book to Khadizhat2}. At the same time the use of the infinitive instead of the modal participle in \refex{ex:‎I wanted to give the book to Khadizhat} is ungrammatical because the constituent focus construction requires the use of participles.

%
\begin{exe}
		\ex	\label{ex:‎I wanted to give the book to Khadizhat1}
	\gll	* dam	b-ikː-an=de	[χadižat-li-j	kiniga	lukː-an]\\
		{} \tsc{1sg.dat}	\tsc{n-}want\tsc{.ipfv-ptcp=pst}	Khadizhat\tsc{-obl-dat}	book	give\tsc{.ipfv-ptcp}\\
	\glt	(Intended meaning: \sqt{‎I wanted to give the book to Khadizhat.}) (E)
	
			\ex	\label{ex:‎I wanted to give the book to Khadizhat2}
	\gll	dam	b-ikː-an=de	[χadižat-li-j	kiniga	lukː-ij]\\
		 \tsc{1sg.dat}	\tsc{n-}want\tsc{.ipfv-ptcp=de}	Khadizhat\tsc{-obl-dat}	book	give\tsc{.ipfv-inf}\\
	\glt	\sqt{‎I wanted to give the book to Khadizhat.} (E)
\end{exe}

The cross-categorical suffixes \tit{-il} and -\textit{ce} (as well as the masdar) can take case suffixes. Occasionally these nominalized verbs can occur in the argument position of complement-taking predicates. For examples, the verb `believe' regularly requires the dative case on its goal argument. In sentence \refex{ex:‎Nobody believed what had happened to me, there up on the coachwork} the goal argument is a nominalized clause with its own arguments and adjuncts.

%
\begin{exe}
	\ex	\label{ex:‎Nobody believed what had happened to me, there up on the coachwork}
	\gll	ča-k'al	w-iχčit	ag-ur-il	akːʷ-i	[du	ce	Ø-iχʷ-ni-li-j	hek'	kuzaw-le-w=či-w]\\
		who\tsc{-indef}	\tsc{m-}believe	go\tsc{.pfv-pret-ref}	\tsc{cop.neg-hab.pst}	\tsc{1sg}	what	\tsc{m-}be\tsc{.pfv-msd-obl-dat}	\tsc{dem.up}	coachwork\tsc{-loc-m=}on\tsc{-m}\\
	\glt	\sqt{‎Nobody believed in what had happened to me there up on the coachwork (on the car).}
\end{exe}

For the utterance verbs and cognition verbs whose complement clauses denote speech acts or similar types of activities that require the use of language it is possible to use the postposition \tit{qari-či-b} followed by a participial clause with the appropriate case marking to denote the topic of the speech act (\refsec{ssec:postposition qari}) or the topic of the cognitive act \refex{ex:He thinks about what he had done}. Syntactically this construction is not a relative clause but a nominalized case-marked verb, which is governed by a postposition.  
%
\begin{exe}
	\ex	\label{ex:He thinks about what he had done}
	\gll	pikri	Ø-ik'-ul	ca-w	[can	d-arq'-ib-t-a-la	qari=či-b]\\
		thought	\tsc{m-}say\tsc{.ipfv-icvb}	\tsc{cop-m}	meet	\tsc{npl-}do\tsc{.pfv-pret-pl-obl-gen}	on.top=on\tsc{-n}\\
	\glt	\sqt{He thinks about what he had done.}
\end{exe}

% --------------------------------------------------------------------------------------------------------------------------------------------------------------------------------------------------------------------- %
\subsection{Adverbial clauses used with emotion and cognition predicates}
\label{ssec:Nominalized relative clauses used with emotion and cognition predicates}
The possibility of an analysis adverbial clauses instead of a complement constructions has been amply discussed in \refsec{ssec:The preterite converb} for the perfective converb. Sentence \refex{ex:‎He is probably remembering when he was beaten} shows another example of a construction for which, however, only the adverbial-clause interpretation is available because the modal participle is followed by temporal/causal enclitic \tit{=qːel} \sqt{when, while, because}, which regularly occurs in adverbial clauses (\refsec{sec:enclitic =qella}). Precisely to what extent these constructions are used, as well as their semantic and morphosyntactic properties, needs to be clarified by future research.


\begin{exe}
	\ex	\label{ex:‎He is probably remembering when he was beaten}
	\gll	han	d-irk-ul	d-urkː-ar	[heχ	cin-ni	it-an=qːel]\\
		remember	\tsc{npl-}occur\tsc{.ipfv-icvb}	\tsc{npl-}find\tsc{.ipfv-prs.3}	\tsc{dem.down}	\tsc{refl.sg-erg}	beat.up\tsc{-ptcp=}when\\
	\glt	\sqt{‎He is probably remembering when he was beaten.}
\end{exe}

%\begin{exe}
%	\ex	\label{ex:}
%	\gll	\\
%		\\
%	\glt	\sqt{}
%\end{exe}
%
%\begin{exe}
%	\ex	\label{ex:}
%	\begin{xlist}
%		\ex	\label{ex:}
%		\gll	\\
%			\\
%		\glt	\sqt{}
%
%		\ex	\label{ex:}
%		\gll	\\
%			\\
%		\glt	\sqt{}
%	\end{xlist}
%\end{exe}
