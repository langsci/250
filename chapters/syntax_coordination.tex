\chapter{Coordination}
\label{cpt:Coordination}

This chapter describes the \isi{coordination} of phrases (\refsec{sec:Coordination of noun phrases and other phrases}) and clauses (\refsec{ssec:Conjunctive coordination of clauses}), including adversative (\refsec{ssec:Adversative coordination of clauses}) and disjunctive \isi{coordination} (\refsec{ssec:Disjunctive coordination of clauses}).
%%%%%%%%%%%%%%%%%%%%%%%%%%%%%%%%%%%%%%%%%%%%%%%%%%%%%%%%%%%%%%%%%%%%%%%%%%%%%%%%

\section{Coordination of noun phrases and other phrases}
\label{sec:Coordination of noun phrases and other phrases}

For the \isi{coordination} of noun phrases the \isi{additive enclitic} \tit{=ra} is used (\refsec{ssec:The additive enclitic}). It is encliticized to the head of every member of the \isi{conjunction}, which can consist of more than two noun phrases \refex{ex:‎‎‎To Kajakent, from here up downwards we travelled, Abdukhalik, Isakadi and me, for our matters}.
%
\begin{exe}
	\ex	\label{ex:‎‎‎To Kajakent, from here up downwards we travelled, Abdukhalik, Isakadi and me, for our matters}
	\gll	qːajakent-li-j,	hej-ka	gu-d-a	ag-ur=da	ʡaˁbdulχaliq'=ra	Isaq'adi=ra	du=ra	nišːa-la	qːulluqː-a-j\\
		Kajakent\tsc{-obl-dat}	this\tsc{-abl}	down\tsc{-1/2pl-dir}	go\tsc{.pfv-pret=1}	Abdulkhalik\tsc{=add} Isakadi\tsc{=add}	\tsc{1sg=add}	\tsc{1pl-gen}	matter\tsc{-obl-dat}\\
	\glt	\sqt{‎‎‎To Kajakent, from here up downwards we traveled, Abdukhalik, Isakadi and me, for our matters.}
\end{exe}

The \isi{enclitic} can also coordinate other types of phrases or modifiers within a \isi{noun phrase}. In \refex{ex:‎‎‎as much as being embarrassed and anxious in front of people} two adverbials are conjoined, in \refex{ex:‎Now I forget what I say and what I do} two \isi{participles} and in \refex{ex:‎Such a story happened to Tawlu and him} two extraposed genitives:
%
\begin{exe}
	\ex	\label{ex:‎‎‎as much as being embarrassed and anxious in front of people}
	\gll	hel=sat	χalq'-li-sa-r	uruc-le=ra		uruχ-le=ra\\
		that=as.much	people\tsc{-obl-ante-abl}	embarrassed\tsc{-advz=add}	fear\tsc{-advz=add}\\
	\glt	\sqt{‎‎‎as much as (being) embarrassed and anxious in front of people}

	\ex	\label{ex:‎Now I forget what I say and what I do}
	\gll	na	b-urs-ib-il=ra	b-arq'-ib-il=ra	qum.urt-u	dam\\
		now	\tsc{n-}tell\tsc{-pret-ref=add} \tsc{n-}do\tsc{.pfv-pret-ref=add}	forget\tsc{.ipfv-prs.3}	\tsc{1sg.dat}\\
	\glt	\sqt{‎Now I forget what I say and what I do.}

	\ex	\label{ex:‎Such a story happened to Tawlu and him}
	\gll	hel=ʁuna	t'amahama	ag-ur-te=de	Tawlu-la=ra	heχ-i-la=ra\\
		that\tsc{=eq}	story	go\tsc{.pfv-pret-dd.pl=pst}	Tawlu\tsc{-gen=add}	\tsc{dem.down}\tsc{-obl-gen=add}\\
	\glt	\sqt{‎Such a story happened to Tawlu and him.}
\end{exe}

The \isi{additive enclitic} is used in \isi{comitative} constructions formed with \isi{reflexive pronouns} (\refsec{sec:Comitative constructions}). Syntactically, they have the structure of coordinated noun phrases (NP\tit{=ra} \tsc{refl}\tit{=ra}), e.g. \refex{ex:with his family}.
%
\begin{exe}
	\ex	\label{ex:with his family}
	\gll	kulpat=ra ca-w=ra\\
		family\tsc{=add}	\tsc{refl-m=add}\\
	\glt	\sqt{with his family}
\end{exe}

Occasionally, \isi{noun phrase} \isi{conjunctions} or \isi{conjunctions} of other phrases occur without overt marking by simple \isi{juxtaposition} of the phrases \xxref{ex:There were mother and son}{ex:‎They were hungry and poor}. With respect to \isi{nouns} there is only a very limited \isi{number} of kinship terms – such as the ones in the example \refex{ex:There were mother and son} – that allow for \isi{coordination} by \isi{juxtaposition}. Sanzhi also has other coordinators that are loans, namely \tit{wa} \sqt{and}, and the Russian \isi{conjunctions} \tit{i} \sqt{and}, \tit{a} \sqt{and, but}, but these loan words are only used in clause \isi{coordination} (\refsec{sec:Coordination of clauses other phrases}). That the two \isi{nouns} in \refex{ex:There were mother and son} are coordinated is not only clear from the meaning of the clause and their \isi{juxtaposition}, but also from the \isi{gender}/\isi{number} agreement on the verb, human plural (\tsc{hpl}), which is used for conjoint noun phrases with human referents.
%
\begin{exe}
	\ex	\label{ex:There were mother and son}
	\gll	aba	durħuˁ	b-irχʷ-iri\\
		mother	boy	\tsc{hpl-}be\tsc{.ipfv-hab.pst}\\
	\glt	\sqt{There were mother and son.}

	\ex	\label{ex:‎They were hungry and poor}
	\gll	kːuš-le	ʁarib-le=de\\
		hungry\tsc{-advz}	poor.fellow\tsc{-advz=pst}\\
	\glt	\sqt{(‎They) were hungry and poor.}
\end{exe}

It is possible to form disjunctive noun phrases, either by means of the disjunction \tit{ja} (or \tit{ja} bearing the \isi{additive enclitic}) \refex{ex:their pumpkins or watermelons} or by means of the polar question marker \tit{=w} \refex{ex:‎a trolleybus or a bus} that is used in disjunctive polar \isi{questions} (see \refsec{sec:Simple polar questions and disjunctive polar questions} for details on disjunction encoded by the polar interrogative \isi{enclitic} and more examples).
%
\begin{exe}
	\ex	\label{ex:their pumpkins or watermelons}
	\gll	qːabuʁ-e	ja=ra	qːalpuz-e	ču-la\\
		pumpkin\tsc{-pl}	or\tsc{=add}	watermelon\tsc{-pl}	\tsc{refl.pl-gen}\\
	\glt	\sqt{their pumpkins or watermelons}

	\ex	\label{ex:‎a trolleybus or a bus}
	\gll	tralejbus=uw		awtobus=uw	\\
		trolleybus\tsc{=q}	bus\tsc{=q}\\
	\glt	\sqt{‎a trolleybus or a bus}
\end{exe}

Coordinated noun phrases are semantically and syntactically plural and therefore trigger plural agreement \refex{ex:There were mother and son}. For agreement resolution with coordinated noun phrases, see \refsec{ssec:Gender agreement resolution}.


%%%%%%%%%%%%%%%%%%%%%%%%%%%%%%%%%%%%%%%%%%%%%%%%%%%%%%%%%%%%%%%%%%%%%%%%%%%%%%%%

\section{Coordination of clauses}
\label{sec:Coordination of clauses other phrases}

\subsection{General remarks on the conjunctive coordination of clauses}
\label{ssec:General remarks on the conjunctive coordination of clauses}


Sanzhi Dargwa, like many East Caucasian languages, does not have native words or special syntactic strategies for the \isi{coordination} of independent main clauses, except for simple \isi{juxtaposition}. Instead, the main way of combining clauses such that they are semantically equivalent to coordinated clauses in European languages is the use of simple converbs, predominantly of the preterite converb \refex{ex:‎We added milk, prepared flat breads and ate them}. Those clauses sometimes contain the \isi{additive enclitic} \tit{=ra}, which is used for the \isi{coordination} of phrases (\refsec{sec:Coordination of noun phrases and other phrases}), on a constituent such as the object or in some other position. Nevertheless, with respect to their morphosyntactic properties, these constructions do not represent \isi{coordination} in the strict sense, since they contain dependent clauses and they show some other properties of subordination.
%
\begin{exe}
	\ex	\label{ex:‎We added milk, prepared flat breads and ate them}
	\gll	[nejg=ra	dabawit	d-arq'-ib-le]	[č'irič'an	b-arq'-ib-le]	č'irič'an,	hel-tːi	d-uk-a-di	nušːa-l\\
		milk\tsc{=add} add	\tsc{npl-}do\tsc{.pfv-pret-cvb}	flat.bread	\tsc{n-}do\tsc{.pfv-pret-cvb}	flat.bread	that\tsc{-pl}	\tsc{npl-}eat\tsc{.ipfv-hab.pst-1}	\tsc{1pl-erg}\\
	\glt	\sqt{‎We added milk, prepared flat breads and ate them.}
\end{exe}

Juxtaposition of clauses is illustrated in \xxref{ex:We fed 28 animals, and we also fed two good (lit. big) breeding bulls.}{ex:‎They bought (the medicine) at home and I did (the cure)}. Again in the second sentence in \refex{ex:We fed 28 animals, and we also fed two good (lit. big) breeding bulls.} we find the \isi{additive} \tit{=ra} encliticized to the object, which emphasizes the semantic relationship between the two sentences, but does not function as a syntactic means of clause \isi{conjunction}. The sentence in \refex{ex:‎They bought (the medicine) at home and I did (the cure)} shows that simple \isi{juxtaposition} is also possible.
%
\begin{exe}
	\ex	\label{ex:We fed 28 animals, and we also fed two good (lit. big) breeding bulls.}
	\gll	ʁanu	kːaʔ-ra	mas	d-alχː-a-di.	k'ʷel	buʁa=ra d-alχː-a-di,	χːula-te	žins-la.\\
		twenty	eight\tsc{-num}	animal	\tsc{npl-}feed\tsc{.ipfv-hab.pst-1} two	bull\tsc{=add}	\tsc{npl-}feed\tsc{.ipfv-hab.pst-1}	big\tsc{-dd.pl} 	lineage\tsc{-gen}\\
	\glt	\sqt{We fed 28 animals. And we also fed two good (lit. big) breeding bulls (of purebreed ancestry).}

	\ex	\label{ex:‎They bought (the medicine) at home and I did (the cure)}
	\gll	qili-b	asː-ib.	b-arq'-ib=da.\\
		home\tsc{-n}	buy\tsc{.pfv-pret}	\tsc{n-}do\tsc{.pfv-pret=1}\\
	\glt	\sqt{‎They bought (the medicine) at home. I did (the medication).} % not idiomatic!
\end{exe}

Coordinated \isi{copula} clauses are normally only juxtaposed, and the \isi{copula} item occurs only once in the first clause. This is possible even in those examples in which the two \isi{copula} subjects do not share person/\isi{number} values:
%
\begin{exe}
	\ex	\label{ex:‎‎She is Tabasaran, I (am) Dargwa}
	\gll	it	tabasran	ca-r,		du	darkːʷan\\
		that	Tabasaran	\tsc{cop-f}	\tsc{1sg}	Dargwa\\
	\glt	\sqt{‎‎She is Tabasaran, I (am) Dargwa.} (E)

	\ex	\label{ex:‎‎I am Dargwa, you (are) Kumyk}
	\gll	du	darkːʷan=da		u	žaˁndar\\
		\tsc{1sg}	Dargwa\tsc{=1}		\tsc{2sg}	Kumyk\\
	\glt	\sqt{‎‎I am Dargwa, you (are) Kumyk.} (E)
\end{exe}

Sanzhi has a set of \isi{conjunctions} ultimately borrowed from Arabic and Persian of which \tit{wa} \sqt{and}, \tit{amma(ki)} \sqt{but}, and \tit{ja(ra)} \sqt{or} are used for the \isi{coordination} of main clauses (for a full list see \refsec{sec:Conjunctions}). Moreover, it has borrowed the same \isi{conjunctions} again from Russian: \tit{i} \sqt{and}, \tit{a} \sqt{and, but}, \tit{no} \sqt{but}, and \tit{ili} \sqt{or}. The Russian \isi{conjunctions} are far more frequently used than the older borrowings. In particular \textit{wa} is almost absent from the corpus (see below for the \isi{number} of occurrences).


% --------------------------------------------------------------------------------------------------------------------------------------------------------------------------------------------------------------------- %

\subsection{Conjunctive coordination of clauses}
\label{ssec:Conjunctive coordination of clauses}

The Arabic loan \tit{wa}, although commonly used in written Standard Dargwa for clause \isi{conjunction} \citep{vandenBerg2004}, is not widespread in the other varieties such as Icari, Ashti, or Sanzhi. In the Sanzhi corpus it is only attested in one text that is a translation from Russian. It is a monosyndetic medial \isi{conjunction} normally occurring between two main clauses \refex{ex:‎I will blow at him, I will take his coat off, said the wind and began to blow}. However, since its use is so rare and it can have been acquired only via formal education in Standard Dargwa, Sanzhi speakers do not fully adapt to the manner in which \tit{wa} is used in the Standard. Thus \refex{ex:‎‎‎He took off his coat, put it together well, and tied it to his horse's saddle.} shows the \isi{conjunction} used between a preterite converb clause and a main clause.
%
\begin{exe}
	\ex	\label{ex:‎I will blow at him, I will take his coat off, said the wind and began to blow}
	\gll	[``du	či-b-uq-un-ne	hel-i-j,	hel-i-la	walžaʁ	či-r-sa-jsː-an=da,''	b-urs-ib	č'an-ni]	wa	[uf	b-ik'-ul	b-aʔ	axː-ib]\\
		\tsc{1sg}	\tsc{spr-n-}go\tsc{.pfv-pret-cvb}	that\tsc{-obl-dat}	that\tsc{-obl-gen}	coat	\tsc{spr-abl-hither}-tear\tsc{.ipfv-ptcp=1}	\tsc{n-}tell\tsc{-pret}	wind\tsc{-erg}	and	blow	\tsc{n-}say\tsc{.ipfv-icvb}	\tsc{n-}begin	put\tsc{.pfv-pret}\\
	\glt	\sqt{``‎I will blow at him, I will take his coat off,'' said the wind and began to blow.}

	\ex	\label{ex:‎‎‎He took off his coat, put it together well, and tied it to his horse's saddle.}
	\gll	[či-r-ix-ub-le	cin-na	walžaʁ,	qːuʁa-l	ka-b-arkː-ur-re]	wa	[urči-la	žilixʷa-la	hila	hitːi	b-iχ-un]\\
		\tsc{spr-abl-}remove\tsc{.pfv-pret-cvb}	\tsc{refl.sg-gen}	coat	beautiful\tsc{-advz}	\tsc{down-n-}wrap\tsc{.pfv-pret-cvb}	and	horse\tsc{-gen}	saddle\tsc{-gen}	behind	behind	\tsc{n-}tie\tsc{.pfv-pret}\\
	\glt	\sqt{‎‎‎He took off his coat, put it together well, and tied it to his horse's saddle.}
\end{exe}

The Russian \isi{conjunction} \tit{i} is far more frequently used than \tit{wa}, predominantly in translations from Russian \refex{ex:‎‎‎Once a boy found a frog and brought it home}, but also occasionally in natural discourse \refex{ex:‎I am doing my house work and get a pension.}. In addition, it occurs as clause-initial conjunctional adverb \sqt{and then} \refex{ex:‎And they stir him up or they calm him down} that connects longer stretches of discourse (see \refsec{sec:Conjunctions} for examples). The total \isi{number} of occurrences of \tit{i} in the Sanzhi corpus is 45, whereas  \tit{wa} appears only three times in one and the same text, which had been translated from Russian into Sanzhi and intended to represent a non-colloquial, written text.
%
\begin{exe}
	\ex	\label{ex:‎‎‎Once a boy found a frog and brought it home}
	\gll	[caj-na	durħuˁ-l	b-arčː-ib	ʡaˁt'a]	i	[sa-qː-ib	qili]\\
		one-\tsc{time}	boy\tsc{-erg}	\tsc{n-}find\tsc{.pfv-pret} frog	and	\tsc{hither}-carry\tsc{-pret}	home\\
	\glt	\sqt{‎‎‎Once a boy found a frog and brought it home.}

	\ex	\label{ex:‎I am doing my house work and get a pension.}
	\gll	[di-la	xazajstweni	ʡaˁči	d-irq'-ul=da]	i	[pensija	ha-jsː-ul=da]\\
		\tsc{1sg-gen}	household	work	\tsc{npl-}do\tsc{.ipfv-icvb=1}	and	pension	\tsc{up}-take\tsc{-icvb=1}\\
	\glt	\sqt{‎I do my house work and receive retirement pay.}
\end{exe}


% --------------------------------------------------------------------------------------------------------------------------------------------------------------------------------------------------------------------- %

\subsection{Adversative coordination of clauses}
\label{ssec:Adversative coordination of clauses}

The Arabic \isi{conjunction} \tit{amma(ki)} \sqt{but} is very rarely employed as a genuine \isi{conjunction} for independent clauses \xxref{ex:‎‎‎He thought that the Kumyk man would come out (of the pit), but a fox came out}{ex:‎‎‎There is this medical cow-parsnip, but if these (plants) get on (the skin), it is bad}. In the majority of instances, it appears as an adversative adverbial in clause-initial \refex{ex:‎But for 600, for 600, that is not bad at all minor} or clause-final position \refex{ex:‎But isn't this similar to a prison minor} (for both cases see \refsec{sec:Conjunctions}). Though in this function it also has the adversative semantics, it rather connects larger episodes of texts that are contrasted to each other.
%
\begin{exe}
	\ex	\label{ex:‎‎‎He thought that the Kumyk man would come out (of the pit), but a fox came out}
	\gll	[il-i-j	han	b-ič-ib	tːura	h-erʁ-an-ne	žaˁndar]	amma	[tːura	ha-b-eʁ-ib kːurtːa]\\
		that\tsc{-obl-dat}	seem	\tsc{n-}occur\tsc{.pfv-pret}	outside \tsc{up}-come\tsc{.m-ptcp-fut.3}	Kumyk		but	outside \tsc{up}\tsc{-n-}come\tsc{.pfv-pret}	fox\\
	\glt	\sqt{‎‎‎He thought that the Kumyk man would come out (of the pit), but a fox came out.}

	\ex	\label{ex:‎‎‎There is this medical cow-parsnip, but if these (plants) get on (the skin), it is bad}
	\gll	[birikːʷa=ra	darman-na	d-irχʷ-ar]	amma	[itːi	či-d-ig-ar	wahi-te	ca-d]\\
		cow-parsnip\tsc{=add}	medicine\tsc{-gen}	\tsc{npl-}become\tsc{.ipfv-prs.3}	but	those	\tsc{spr-npl-}be\tsc{.pfv-cond.3}	evil\tsc{-dd.pl} \tsc{cop-npl}\\
	\glt	\sqt{‎‎‎There is this medical cow-parsnip, but if these (plants) get on (the skin), it is bad.}
\end{exe}

The same functional range is found with the Russian \isi{conjunction} \tit{a} \sqt{and, but}. It predominantly occurs as sentence-initial marker of topic switch (\refsec{sec:Conjunctions}), but there are also examples of adversative \isi{coordination} of clauses \refex{ex:‎There all the people who came want to leave, but there are no cars to leave}. The second adversative \isi{coordination} from Russian, \tit{no} \sqt{but}, differs in its semantics from \tit{a} because its meaning is more specific. In the Sanzhi corpus there is only one instance of \tit{no} occurring between an independent clause and an ellipsis \refex{ex:‎‎‎The Latvian people were harmful, they were bad people, but order, cleanness is great among them}.
%
\begin{exe}
	\ex	\label{ex:‎There all the people who came want to leave, but there are no cars to leave}
	\gll	[hextːu-b	šːatːir	sa-b-ač'-ib-te	li<b>il=ra	χalq'	b-ikː-ul	ca-b	gu-r-b-uq'-aˁnaj]		a	[gu-r-b-uq'-ij	mašin-te	d-akːu]\\
		there.\tsc{up-hpl}	visit	\tsc{hither-hpl-}come\tsc{.pfv-pret-dd.pl} 	all\tsc{<hpl>=add}	people	\tsc{hpl-}want\tsc{.ipfv-icvb}	\tsc{cop-hpl}	\tsc{sub-abl-hpl-}go\tsc{-subj.3}	but	\tsc{down}\tsc{-abl-hpl-}go\tsc{-inf}	car\tsc{-pl}	\tsc{npl-}\tsc{cop.neg}\\
	\glt	\sqt{‎There all the people who came want to leave, but there are no cars to leave.}

	\ex	\label{ex:‎‎‎The Latvian people were harmful, they were bad people, but order, cleanness is great among them}
	\gll	[latiši	wredni	χalq'=de,	wahi	χalq'=de]	no	[parjadok,	amzu-dex	χʷal-le	heχ-tː-a-lla]\\
		Latvian	harmful	people\tsc{=pst}	evil	people\tsc{=pst}	but	order	clean\tsc{-nmlz}	big\tsc{-advz}	\tsc{dem.down}\tsc{-pl-obl-gen}\\
	\glt	\sqt{‎‎‎The Latvian people were harmful, they were bad people, but order, cleanliness (was) great among them.}
\end{exe}


% --------------------------------------------------------------------------------------------------------------------------------------------------------------------------------------------------------------------- %

\subsection{Disjunctive coordination of clauses}
\label{ssec:Disjunctive coordination of clauses}

The \isi{particle} \tit{ja(ra)} (\tit{jara} is composed of \tit{ja} and the \isi{additive} \tit{=ra}) is ultimately a loan from Persian. It used as a bisyndetic \isi{particle} in clause initial position. In affirmative clauses it means \sqt{either \ldots\ or} and in negative clauses it means \sqt{neither \ldots\ nor}. Although the use in affirmative clauses can be obtained in elicitation, all corpus examples show negative clauses. Sentence \refex{ex:‎‎‎Since at least five or six years, he says, no man did fell down from the nut (tree) nor broke the hand or broke a leg} illustrates that more than two clauses can be coordinated.
%
\begin{exe}
	\ex	\label{ex:‎‎‎I did not see neither how they slaughtered (the sheep) nor did I see how they put the pot on the fire (in order to cook it)}
	\gll	[ja	luχ-unne	či-a-b-až-ib=da]	[ja	ħaˁšak	či-ha-b-irxː-ul	či-a-b-až-ib=da]\\
		or	slaughter\tsc{.ipfv-icvb}	\tsc{spr-neg-n-}see\tsc{.pfv-pret=1}	or	pot	\tsc{spr-up-n-}put\tsc{.ipfv-icvb}	\tsc{spr-neg-n-}see\tsc{.pfv-pret=1}\\
	\glt	\sqt{‎‎‎Neither did I see how they slaughtered (the sheep) nor did I see how they put the pot on the fire (in order to cook it).}

	\ex	\label{ex:‎‎‎Since at least five or six years, he says, no man did fell down from the nut (tree) nor broke the hand or broke a leg}
	\gll	[ca	arrah	ja	xujal	urekːal	dus,	Ø-ik'-ul	ca-w,	ca	arrah	admi	ja	qix-le-r	či-r-a-ka-jč-ib]	[ja	naˁq	a-b-aˁč-un]	[ja	t'uˁ	a-b-aˁč-un]\\
		one	at.least	or	five	six	year	\tsc{m-}say\tsc{.ipfv-icvb}	\tsc{cop-m}	one	at.least	person	or	nut\tsc{-loc-abl}	\tsc{spr-abl-neg-down}-occur\tsc{.pfv-pret}	or	hand	\tsc{neg-n-}crush\tsc{.pfv-pret}	or	leg	\tsc{neg-n-}crush\tsc{.pfv-pret}\\
	\glt	\sqt{For at least five or six years, he says, no man has fallen down from the nut (tree), nor broken a hand or broken a leg.}
%	\glt	\sqt{‎‎‎Since at least five or six years, he says, no man did fell down from the nut (tree) nor broke the hand or broke a leg.}
\end{exe}

For the disjunctive \isi{coordination} of affirmative clauses the Russian disjunction \tit{ili} is used, which occurs between the members of the disjunction or in clause-initial position \refex{ex:‎Or it must probably be this, or when he is set free, or when is taken (into prison)}.
%
\begin{exe}
	\ex	\label{ex:‎‎‎Did they kill them or did they take them}
	\gll	[kax-ub-le=w]	ili	[b-uč-ib-le=w]?\\
		kill\tsc{.pfv-pret-cvb=q}	or	\tsc{n-}gather\tsc{.ipfv-pret-cvb=q}\\
	\glt	\sqt{‎‎‎Did they kill them or did they take them?}

	\ex	\label{ex:‎And they stir him up or they calm him down}
	\gll	[i	heχ-tː-a-l	heχ	či-ha-jʁ-ul	ca-w]	ili [heχ	parʁat	Ø-irq'-ul ca-w]\\
		and	\tsc{dem.down}\tsc{-pl-obl-erg}	\tsc{dem.down}	\tsc{spr-up-}drive\tsc{.pfv-icvb}	\tsc{cop-m}	or	\tsc{dem.down}	quiet	\tsc{m-}do\tsc{.ipfv-icvb}	\tsc{cop-m}\\
	\glt	\sqt{‎And they stir him up or they calm him down.}

	\ex	\label{ex:‎Or it must probably be this, or when he is set free, or when is taken (into prison)}
	\gll	[ili	hež	b-iχʷ-ij	ʡaˁʁuni-l	ca-b]	[ili	hež w-at	tːura	iʁ-ul]	[ili	uk-ul]\\
		or	this	\tsc{n-}be\tsc{.pfv-inf}	needed\tsc{-advz}	\tsc{cop-n}	or	this	\tsc{m-}free outside	come\tsc{.ipfv-icvb}	or	gather\tsc{.m.ipfv-icvb}\\
	\glt	\sqt{‎Or it must probably be this, or when he is set free, or when is taken (into prison).}
\end{exe}
