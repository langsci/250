\chapter{Interrogative clauses}
\label{cpt:Interrogative clauses}

Interrogative clauses are marked by interrogative enclitics and by rising intonation, but the latter is not always particularly salient. The interrogative enclitics belong to the class of predicative \is{particle}particles (\refsec{sec:Predicative particles}). This means that interrogative enclitics can co-occur with certain non-finite verb forms in analytic tenses, and they turn the verb plus \isi{enclitic} complex into a finite verb form used in main clauses. Thus, in many \isi{questions} there is no \isi{copula}, auxiliary or other predicative \isi{particle} (person \isi{enclitic}, past \isi{enclitic}), but only a non-finite lexical verb and an interrogative \isi{enclitic}, but the clause is nevertheless a full grammatical question. This chapter covers

\begin{itemize}
	\item	polar \isi{questions} and disjunctive polar \isi{questions} (\refsec{sec:Simple polar questions and disjunctive polar questions})
	\item	content \isi{questions} (\refsec{sec:Content questions})
	\item	\is{tag question}tag questions (\refsec{sec:Tag questions})
	\item	embedded \isi{questions} (\refsec{sec:Subordinate questions})
\end{itemize}


%%%%%%%%%%%%%%%%%%%%%%%%%%%%%%%%%%%%%%%%%%%%%%%%%%%%%%%%%%%%%%%%%%%%%%%%%%%%%%%%

\section{Simple polar questions and disjunctive polar questions}
\label{sec:Simple polar questions and disjunctive polar questions}

Polar \isi{questions} are marked by means of an \isi{enclitic} that has four allomorphs: \tit{=w} after vowels, and \tit{=uw} after \isi{consonants}, with additional restrictions. The allomorph \tit{=ew} occurs after the \isi{imperfective converb} suffix \tit{-ul} and the allomorph \tit{=aw} after the first person suffix \textit{-id} and after certain words ending in /aw/, e.g. \textit{qːanaw} `ditch'. Its use is obligatory. It is encliticized to the predicate, i.e., to the verb if there is a verb \refex{ex:Is this similar to him}, \refex{ex:Is this place called khanhara? Yes}, otherwise to the nominal predicate which bears a person or past \isi{enclitic} \refex{ex:‎He also belonged to them? (i.e. was their relative)}. Elliptic polar \isi{questions} can consist of nominals or other items without an accompanying verb or other auxiliary \refex{ex:‎The bandits?}. 

\begin{exe}
	\ex	\label{ex:Is this similar to him}
	\gll	miši-l	ca-w=uw	iχ	iχ-i-j?\\
		similar\tsc{-advz}	\tsc{cop-m=q}	\tsc{dem.down}	\tsc{dem.down-obl-dat}\\
	\glt	\sqt{Is this similar to him?}

	\ex	\label{ex:Is this place called khanhara? Yes}
	\gll	hel	musːa-j	χaˁnhara	b-ik'-u=w?	\\
		that	place\tsc{-dat}	khanhara	\tsc{hpl-}say\tsc{.ipfv-prs=q}	\\
	\glt	\sqt{Is this place called khanhara? Yes.}

	\ex	\label{ex:‎He also belonged to them? (i.e. was their relative)}
	\gll	hel-t-a-lla=de=w	il=ra?	\\
		that\tsc{-pl-obl-gen=pst=q}	that\tsc{=add}\\
	\glt	\sqt{‎He also belonged to them?} (i.e. was their relative)

	\ex	\label{ex:‎The bandits?}[Why did they not kill Osman?]\\
		\gll	qːačuʁ-e=w?	\\
			bandit\tsc{-pl=q}	\\
		\glt	\sqt{‎The bandits?}

\end{exe}

It cannot be added to another constituent if the clause contains a verb:

\begin{exe}
	\ex	\label{ex:Do you have bread}
	\begin{xlist}
		\ex	\label{ex:Do you have bread@A}
		\gll	t'ult'	le-b=uw	ala?\\
			bread	exist\tsc{-n=q}	\tsc{2sg.gen}\\
		\glt	\sqt{Do you have bread?} (E)
	
		\ex	\tit{* t'ult'uw leb ala?} \label{ex:Do you have bread@B}
	\end{xlist}
\end{exe}

Very frequently polar \isi{questions} are followed by the interrogative pronoun \tit{ce=ja} \refex{ex:‎‎‎Did the movie start or what}, which functions as a kind of hesitation marker or expresses doubts on the part of the speaker similar to a phrase such as \sqt{I wonder}.

\begin{exe}
	\ex	\label{ex:‎‎‎Did the movie start or what}
	\gll	kinu	b-aʔ	ač'-ib-le=w	ce=ja?	\\
		film	\tsc{n-}begin	come\tsc{.pfv.pret-cvb=q}	what\tsc{=q}\\
	\glt	\sqt{‎‎‎Did the movie start or what?}
\end{exe}

Examples \xxref{ex:‎‎‎Did the movie start or what}{ex:Did (the cow) die? he asked. No, I said}, \refex{ex:Is he going to Derbent} illustrate that the interrogative enclitics are sufficient to guarantee the \isi{finiteness} of the clause even when no \isi{copula}, person \isi{enclitic} or past \isi{enclitic} is present. In \refex{ex:‎‎‎Did the movie start or what} the lexical verb appears as \isi{perfective converb}, and in \refex{ex:‎Is she able to walk} and \refex{ex:Is he going to Derbent} as \isi{imperfective converb}. In \refex{ex:Was he himself Kumyk? No, Dargi} the interrogative \isi{enclitic} follows the past \isi{enclitic}, which cannot be left out. 

There are three basic ways of answering polar \isi{questions}: (i) the full verb with or without arguments and adjuncts can be used \refex{ex:‎Is she able to walk}; (ii) only the appropriate \isi{copula} or auxiliary occurs \refex{ex:Did (the cow) die? he asked. No, I said}, or (iii) only the \is{particle}particles \textit{e} \sqt{yes} or \tit{a}, \tit{aʔa} \sqt{no} are used. However, the use of only these \is{particle}particles is not very common as Sanzhi speakers told me and as the data from my own corpus show. It is also possible to combine the strategies.

\begin{exe}

	\ex	\label{ex:‎Is she able to walk}
	\gll	r-aš-ij	r-irχ-ul=ew? e,	e,	r-irχ-ul	ca-r\\
		\tsc{f-}go\tsc{-inf}	\tsc{f-}be.able\tsc{.ipfv-icvb=q} yes	yes	\tsc{f-}be.able\tsc{.ipfv-icvb}	\tsc{cop-f}\\
	\glt	\sqt{‎Is she able to walk? Yes, yes, she is able.}

	\ex	\label{ex:Did (the cow) die? he asked. No, I said}
	\gll	``b-ebč'-ib-le=w?'' Ø-ik'ʷ-ar.	``akːu,''	haʔ-ib=da\\
		\tsc{n-}die\tsc{.pfv-pret-cvb=q}	\tsc{m-}say\tsc{.ipfv-prs} \tsc{cop.neg} say\tsc{.pfv-pret=1}\\
	\glt	\sqt{``Did (the cow) die?'' he asked. ``No,'' I said.}
	
	\ex	\label{ex:Was he himself Kumyk? No, Dargi}
	\gll	ca-w	qːumuqlan=de=w?	aʔa,	darkːʷan=de\\
		\tsc{refl-m}	Kumyk\tsc{=pst=q} 	no	Dargwa\tsc{=pst}\\
	\glt	\sqt{Was he himself Kumyk? No, Dargi.}
\end{exe}

If the question is headed by a non-finite verb form without a \isi{copula}, only the negative \isi{copula} can stand alone in an answer. If the answer is affirmative either the \isi{particle} \textit{e} \sqt{yes} or the whole predicate including the lexical verb must be used. The \isi{copula} alone cannot serve as a grammatical answer \refex{ex:He is going / not going}.

\begin{exe}
	\ex	\label{ex:Is he going to Derbent}
	\gll	it	w-ax-ul=ew	Derbent-le?\\
		that	\tsc{m-}go\tsc{.ipfv-icvb=q}	Derbent\tsc{-loc}\\
	\glt	\sqt{Is he going to Derbent?} (E)

	\ex	\label{ex:He is going / not going}
	\gll	w-ax-ul		ca-w	/	(w-ax-ul)		akːu		/	*~ca-w\\
		\tsc{m-}go\tsc{.ipfv-icvb}	\tsc{cop-m}	/	\tsc{m-}go\tsc{.ipfv-icvb}	\tsc{cop.neg}	/	{\hphantom{*}}~\tsc{cop-m}\\
	\glt	\sqt{He is going\slash not going.} (E)
\end{exe}

If the question already contains the affirmative \isi{copula}, then the \isi{copula} alone is enough for making up a complete answer \refex{ex:Is he going to Derbent? Yes, he is}. However, in an \isi{interrogative clause} such as the one in \refex{ex:Is he going to Derbent? Yes, he is} normally no \isi{copula} is used because the interrogative \isi{particle} is sufficient (compare examples \refex{ex:‎Is she able to walk}, \refex{ex:Is he going to Derbent? Yes, he is} above).

\begin{exe}
	\ex	\label{ex:Is he going to Derbent? Yes, he is}
	\gll	it	w-ax-ul	ca-w=uw	Derbent-le?	ca-w\\
		that	\tsc{m-}go\tsc{.ipfv-icvb}	\tsc{cop-m=q}	Derbent\tsc{-loc}	\tsc{cop-m}\\
	\glt	\sqt{Is he going to Derbent? Yes, he is.} (E)
\end{exe}

Answers to negative polar \isi{questions} normally contain a predicate because the \is{particle}particles alone could lead to misunderstanding \refex{ex:Did you not also spend the night there? We stayed one night}, \refex{ex:Was the river not big? Now it was big}. However, negative polar \isi{questions} are rare and most of them are rather rhetorical \isi{questions} to which no real answer is expected, but they express surprise or disbelief on part of the speaker \refex{ex:‎Don't the Sanzhi people know at least that (story) to tell}. 

\begin{exe}
	\ex	\label{ex:Did you not also spend the night there? We stayed one night}
	\begin{xlist}
		\ex	\label{ex:Did you not also spend the night there? We stayed one night@A}
		\gll	daˁrχaˁ=ra	heχtːu-d	a-d-už-ib=da=w?\\
			evening\tsc{=add}	there\tsc{.down-1/2pl}	\tsc{neg-1/2pl-}be\tsc{-pret=2pl=q}\\
		\glt	\sqt{Did you not also spend the night there?}

		\ex	\label{ex:Did you not also spend the night there? We stayed one night@B}
		\gll	ca	dučːi	d-už-ib=da\\
			one	night	\tsc{1/2.pl-}be\tsc{-pret=1}\\
		\glt	\sqt{We stayed one night.}
	\end{xlist}

	\ex	\label{ex:Was the river not big? Now it was big}
	\gll	erk'ʷ	χʷal-le=kːʷi=w? 	hana	χʷal-le=de\\
		river	big\tsc{-advz=neg.pst=q} 	now	big\tsc{-advz=pst}\\
	\glt	\sqt{Was the river not big? Now it was big.}

	\ex	\label{ex:‎Don't the Sanzhi people know at least that (story) to tell}
	\gll	hel	arrah	a-b-alχ-ul=ew	c'il	heštːi	sunglan-t-a-l	b-urs-araj?\\
		that	at.least	\tsc{neg-n-}know\tsc{.ipfv-icvb=q}	then	these	Sanzhi\tsc{-pl-obl-erg}	\tsc{n-}tell\tsc{-subj.3}	\\
	\glt	\sqt{‎Don't the Sanzhi people know at least that (story) to tell?}
\end{exe}

Polar \is{interrogative clause}interrogative clauses are mostly verb-final \refex{ex:Is this place called khanhara? Yes}, \refex{ex:Did you not also spend the night there? We stayed one night@A}, but it is also possible to find examples with verbs occurring in other positions \refex{ex:Do you have bread@A}, \refex{ex:Is he going to Derbent? Yes, he is}, \refex{ex:Are you going home because of what she said}.

\begin{exe}
	\ex	\label{ex:Are you going home because of what she said}
	\gll	hel-i-la	ʁaj-li-j	qili	arg-ul=de=w	u?	\\
		that\tsc{-obl-gen}	word\tsc{-obl-dat}	home	go\tsc{.ipfv-icvb=2sg=q}	\tsc{2sg}	\\
	\glt	\sqt{Are you going home because of what she said?}
\end{exe}

Polar interrogatives are frequently combined with a following phrase that bears the marker for embedded \isi{questions} \refex{ex:Was it wheat or something else} since that marker is also used for expressing epistemic uncertainty (\refsec{sec:Subordinate questions}).

\begin{exe}
	\ex	\label{ex:Was it wheat or something else}
	\gll	ač'i=de=w	ce	ca-d=de=l?\\
		wheat\tsc{=pst=q}	what	\tsc{cop-npl=pst=indq}\\
	\glt	\sqt{Was it wheat or something else.}
\end{exe}

In disjunctive polar \isi{questions} the interrogative \isi{enclitic} appears on the predicate in each member of the disjunction:

\begin{exe}
	\ex	\label{ex:Is he in a prison or in the army}
	\gll	tusnaq-le-w=uw	iž	ʡaˁrmija-le-w=uw?\\
		prison\tsc{-loc-m=q}	this	army\tsc{-loc-m=q}\\
	\glt	\sqt{Is he in a prison or in the army?}

	\ex	\label{ex:‎They say to the wife, Did your husband have a head or not}
	\gll	``iž	ala	sub-la	bek'	le-b-il=de=w'',	b-ik'-ul	ca<b>i,	``akːʷ-ar-il=de=w?''	\\
		this	\tsc{2sg.gen}	husband\tsc{-gen}	head	exist\tsc{-n-ref=pst=q}	\tsc{hpl-}say\tsc{.ipfv-icvb}	\tsc{cop<hpl>}	\tsc{cop.neg-prs-ref=pst=q}\\
	\glt	\sqt{‎They say to the wife, ``Did your husband have a head or not?''}
\end{exe}

The same construction can also be used as an assertive disjunction without any interrogative illocutionary force \refex{ex:This is similar to the sun or the moon}. A similar multifunctionality of interrogative \is{particle}particles covering polar and content \isi{questions} as well as in declarative disjunctions and some other contexts is attested in a \isi{number} of other East Caucasian languages (e.g. Hinuq, see \citeb{Forker2013c}) as well as unrelated languages such as Japanese and Malayalam (\citealp[2]{Slade2011}, \citeb{Uegaki2018}). 

\begin{exe}
	\ex	\label{ex:This is similar to the sun or the moon}
	\gll	hek'	bari-li-j=uw	bac-li-j=uw	miši-l	ca-b\\
		\tsc{dem.up}	sun\tsc{-obl-dat=q}	moon\tsc{-obl-dat=q}	similar\tsc{-advz}	\tsc{cop-n}\\
	\glt	\sqt{This is similar to the sun or the moon.}
\end{exe}


%%%%%%%%%%%%%%%%%%%%%%%%%%%%%%%%%%%%%%%%%%%%%%%%%%%%%%%%%%%%%%%%%%%%%%%%%%%%%%%%

\section{Content questions}
\label{sec:Content questions}

Content \isi{questions} contain interrogative pronouns (see \refsec{sec:Interrogative pronouns} for a list of the pronouns) and an \isi{enclitic} that has two allomorphs. It is \tit{=e} after \isi{consonants} and \tit{=ja} after vowels. The use of the interrogative \isi{enclitic} is obligatory. The only two contexts where its use is prohibited are \isi{questions} containing the second person \isi{enclitic} \tit{=de} or the past \isi{enclitic} \tit{=de}, which both end with /e/.\footnote{One can argue that the \isi{enclitic} \tit{=e} is used in such cases, but because its phonological form is identical to the vowel in the past \isi{enclitic} and the second person \isi{enclitic}, its presence is not noticeable. However, in many instances a sequence of identical vowels leads to long vowels (\refsec{ssec:Sequences of identical vowels}), which is not the case for the respective \is{interrogative clause}interrogative clauses. The long vowel [eː] does not occur very often, but it is attested. Thus, it is reasonable to assume that in \is{interrogative clause}interrogative clauses an underlying long vowel [eː] has been shorted, but for a conclusive argumentation more research is needed.} It is usually added to the predicate (verbal or nominal), unless there is no verb or other predicates, then it is added to the question word.

The \isi{constituent order} in \isi{questions} is such that topical items mostly precede the interrogative pronoun and the verb appears in clause-final position.

\begin{exe}
	\ex	\label{ex:‎Why did we send a man without a head there into (the cave)}
	\gll	bek'	akːʷ-ar	admi	celij	w-i-ha-w-q-aq-un=da=ja	nušːa-li?\\
		head	\tsc{cop.neg-prs.3}	person	why	\tsc{m-in-up}\tsc{-m}-go.\tsc{pfv-caus-pret=1=q}	\tsc{1pl-erg}\\
	\glt	\sqt{‎Why did we send a man without a head there into (the cave)?}

	\ex	\label{ex:‎‎‎How can you get to know them, mother}
	\gll	il-tːi	cet'-le	b-aχ-ij	b-irχ-u=ja,	aba?\\
		that-\tsc{pl}	how\tsc{-advz}	\tsc{hpl-}know\tsc{.pfv-inf}	\tsc{hpl-}be.able\tsc{.ipfv-prs=q}	mother\\
	\glt	\sqt{‎‎‎How can you get to know them, mother?}

	\ex	\label{ex:What does Jusup's son do}
	\gll	Jusup-la	durħuˁ-l	ce	b-irq'-u=ja?\\
		Jusup\tsc{-gen}	boy\tsc{-erg}	what	\tsc{n-}do\tsc{.ipfv-prs=q}\\
	\glt	\sqt{What does Jusup's son do?}
\end{exe}

Often \isi{genitive} phrases are split up if the head is part of the interrogative phrase and the \isi{genitive} then follows the host of the interrogative \isi{enclitic} \refex{ex:‎What other dishes do we have}. Such extraposed genitives are not restricted to \isi{questions} but also frequently found in assertions (\refsec{ssec:Extraposed genitives}). 

\begin{exe}
	\ex	\label{ex:‎What other dishes do we have}
	\gll	cara	ce	χurejg	d-irχ-u=ja	nišːa-la?\\
		other	what	food	\tsc{npl-}be\tsc{.ipfv-prs=q}	\tsc{1pl-gen}\\
	\glt	\sqt{‎What other dishes do we have?}
\end{exe}

If there is no verb then the respective predicate or even the interrogative pronoun can end up in clause-final position bearing the interrogative \isi{enclitic}.

\begin{exe}
	\ex	\label{ex:‎This where should it be}
	\gll	hej	čina-b	b-iχʷ-ij	ʡaˁʁuni-ce=ja,	hež?\\
		this	where\tsc{-n}	\tsc{n-}be\tsc{.pfv-inf}	needed\tsc{-dd.sg=q}	this\\
	\glt	\sqt{‎This where should it be, this?}

	\ex	\label{ex:‎Where is this (picture) of the empty bottles}
	\gll	d-ac'	šuš-n-a-la	cek'u	čina-b=e?\\
		\tsc{npl-}empty	bottle\tsc{-pl-obl-gen}	whatchamacallit	where\tsc{-n=q}\\
	\glt	\sqt{‎Where is this (picture) of the empty bottles?}
\end{exe}

It is also possible to put the interrogative pronoun in clause-initial position, but this is far less common \xxref{ex:‎[He does not leave me alone], what should I do with him}{ex:Why will he not greet us}. Equally possible and slightly more common is the occurrence of material following the verb.

\begin{exe}
	\ex	\label{ex:‎[He does not leave me alone], what should I do with him}
	\gll	ce	b-arq'-ida=ja	it-i-j	du-l?\\
		what	\tsc{n-}do\tsc{.pfv-modq=q}	that\tsc{-obl-dat}	\tsc{1sg-erg}\\
	\glt	\sqt{‎[He does not leave me alone], what should I do with him?}

	\ex	\label{ex:‎‎‎Then from where do you cross the river}
	\gll	c'il	čina-r	d-ax-utːa=ja	ušːa	erk'ʷ-la itille?\\
		then	where\tsc{-abl}	\tsc{1/2pl-}go\tsc{-prs.2pl=q}	\tsc{2pl}	river\tsc{-gen}	further\\
	\glt	\sqt{‎‎‎Then from where do you cross the river?} (lit. \sqt{go to the other side of the river})

	\ex	\label{ex:‎Where are these now? (these = people)}
	\gll	čina-b=e	hana	heštːi?\\
		where\tsc{-hpl=q}	now	these	\\
	\glt	\sqt{‎Where are these now? (these = people)}

	\ex	\label{ex:Why will he not greet us}
	\gll	cellij	it-i-l	salam	a-lukː-an=e	nišːi-cːe?\\
		why	that\tsc{-obl-erg}	greeting	\tsc{neg-}give\tsc{.ipfv-ptcp=q}		\tsc{1pl-in}\\
	\glt	\sqt{Why will he not greet us?}
\end{exe}

The only \isi{constituent order} that is strictly forbidden is for the interrogative pronoun to follow the verb \refex{ex:What did Madina buy ungrammatical} or to follow the constituent that bears the person \isi{enclitic} or past \isi{enclitic} \refex{ex:Who are you ungrammatical}. In the latter examples possible orders are \tit{u ča=de?} and \tit{ča u=de?}:

\begin{exe}
	\ex[*]{	\label{ex:What did Madina buy ungrammatical}
	\gll	 Madina-l	asː-ib=e ce?\\
		 Madina\tsc{-erg}	buy\tsc{.pfv-pret=q}	what\\
	\glt	(Intended meaning: \sqt{What did Madina buy?}) (E)}

	\ex[*]{	\label{ex:Who are you ungrammatical}
	\gll	 u=de ča?\\
		 \tsc{2sg=2sg}	who\\
	\glt	(Intended meaning: \sqt{Who are you}) (E)}
\end{exe}

As in polar \isi{questions} the lexical verb can appear in non-finite forms \refex{ex:Why will he not greet us} and the interrogative \isi{particle} can take the place of the \isi{copula} \refex{ex:‎Where are these now? (these = people)}. Except for verbs, any constituent can be questioned, be it an argument or an adjunct, e.g. \isi{absolutive} \refex{ex:‎What other dishes do we have}, \isi{ergative}, spatial adjunct \refex{ex:‎Where are these now? (these = people)}, manner adjunct \refex{ex:‎‎‎How can you get to know them, mother}, causal adjunct \refex{ex:Why will he not greet us}. Genitive modifiers or other modifiers of \isi{nouns} can also be questioned (see \refsec{sec:Interrogative pronouns} for examples). It is also possible to question constituents of subordinate clauses \refex{ex:‎Who did you say was the head (of the kolkhoz)}, \refex{ex:The ones who were thrown out, who did you (masc.) say was this} or of postpositional phrases \refex{ex:About whom were they talking}.

\begin{exe}
	\ex	\label{ex:‎Who did you say was the head (of the kolkhoz)}
	\gll	a	[presedatel	ča 	ca-w=de]	Ø-ik'ʷ-a-tːe?\\
		and	head	who	\tsc{cop-m=pst}	\tsc{m-}say\tsc{.ipfv-hab.pst-2sg}\\
	\glt	\sqt{‎Who did you (masc.) say was the head (of the kolkhoz)?}

	\ex	\label{ex:The ones who were thrown out, who did you (masc.) say was this}
	\gll	[t'ut'u	b-arq'-ib-te,	ča-qal]	Ø-ik'ʷ-a-tːe?\\
		throw.out	\tsc{hpl-}do\tsc{.pfv-pret-dd.pl} 	who\tsc{-assoc}	\tsc{m-}say\tsc{.ipfv-hab.pst-2sg}\\
	\glt	\sqt{The ones who were thrown out, who did you (masc.) say was this?}

	\ex	\label{ex:About whom were they talking}
	\gll	hila	qari=či-b	itːi	ʁaj	ka-b-ik'-ul=de?\\
		who.\tsc{gen}	up=on\tsc{-n}	those	word	\tsc{down-hpl-}say\tsc{.ipfv-icvb=pst}\\
	\glt	\sqt{About whom were they talking?} (E)
\end{exe}

Interrogative clauses can contain more than one interrogative pronoun \refex{ex:Who did make which present}. The order of the interrogative pronouns in \refex{ex:‎‎‎And this is what medicine for what} can also be switched around.

\begin{exe}
	\ex	\label{ex:Who did make which present}
	\gll	hi-l ce padarit b-arq'-ib=e?\\
		who\tsc{.obl-erg}	what	make.present	\tsc{n-}do\tsc{.pfv-pret=q}\\
	\glt	\sqt{Who made which present?} (E)

	\ex	\label{ex:‎‎‎And this is what medicine for what}
	\gll	a	it	ce	darman=e	cellij?\\
		and	that	what	medicine\tsc{=q}	what\tsc{.dat}\\
	\glt	\sqt{‎‎‎And this is what medicine for what?}
\end{exe}

Answers to content \isi{questions} can either consist of only the focus\is{focus} part \refex{ex:‎And Kak Husen, who is he with respect to uncle Abdukhalik? The Father} or they can be whole sentences. In the answers that are full sentences the nominal part of the constituent that constitutes the focus\is{focus} can be absent (i.e. in \refex{ex:‎‎We do not know, they say} the inflection on the verb is enough to convey the meaning of the first person pronoun that represents the answer to the question). In \refex{ex:‎‎‎Who should give this money (back)? ‎‎I have to give the money back} the item in focus\is{focus} follows the verb which is rather unexpected if one embraces the position that the focus\is{focus} position in East Caucasian is immediately before the verb.

\begin{exe}
	\ex	\label{ex:‎‎‎Who knows where he is}
	\gll	čina-w=el	hi-j	b-alχ-ul=e	il?\\
		where\tsc{-m=indq}	who\tsc{.obl-dat}	\tsc{n-}know\tsc{.ipfv-icvb=q}	that\\
	\glt	\sqt{‎‎‎Who knows where he is?}

	\ex	\label{ex:‎‎We do not know, they say}
	\gll	``a-b-alχ-ad''	b-ik'-ul\\
		\tsc{neg-n-}know\tsc{.ipfv-prs.1}	\tsc{hpl-}say\tsc{.ipfv-icvb}\\
	\glt	\sqt{``‎‎We do not know,'' they say, \ldots}

	\ex	\label{ex:‎‎‎Who should give this money (back)? ‎‎I have to give the money back}
	\gll	hek'-tːi	arc	hi-l	lukː-an-te=ja?		arc	lukː-an-te=q'al	du-l\\
		\tsc{dem.up}\tsc{-pl}	money	who\tsc{.obl-erg}	give\tsc{.ipfv-ptcp-dd.pl=q}	money	give\tsc{.ipfv-ptcp-dd.pl=mod}	\tsc{1sg-erg}\\
	\glt	\sqt{‎‎‎Who should give this money (back)? ‎‎I have to give the money back.}

	\ex	\label{ex:‎And Kak Husen, who is he with respect to uncle Abdukhalik? The Father}
	\gll	a	qːaq	ħuˁsen	hana	heχ	ʡaˁbdulχaliq'	acːi-lla	ča=ja? 		atːa\\
		and	back	Husen	now	\tsc{dem.down}	Abdulkhalik	uncle\tsc{-gen}	who\tsc{=q}	 father\\
	\glt	\sqt{‎And Kak Husen, who is he with respect to uncle Abdukhalik? The Father.}
\end{exe}


%%%%%%%%%%%%%%%%%%%%%%%%%%%%%%%%%%%%%%%%%%%%%%%%%%%%%%%%%%%%%%%%%%%%%%%%%%%%%%%%

\section{Tag questions}
\label{sec:Tag questions}

Tag \isi{questions} are very common in Sanzhi and occur after assertions. The form of the tag depends on the polarity of the assertion because it reverses the polarity. In the vast majority of corpus examples \is{tag question}tag questions follow clauses with affirmative clauses and the tags are formed by means of the negative \isi{present tense} \isi{copula} to which the polar question marker \tit{=w} is attached, i.e. \tit{akːu=w?} (lit. \sqt{Is it not?}) \refex{ex:About what had happened he is talking, right? Yes}, \refex{ex:This is the company (group of friends), right}. This \isi{copula} can even be used when the assertion contains a verb with past time reference \refex{ex:‎In my opinion, he was imprisoned and then came back from there, right}. However, in such a case the negative \isi{past tense} \isi{copula} can also be used \refex{ex:‎We did not eat boars, right}. Note that the \isi{copula} in \refex{ex:About what had happened he is talking, right? Yes} that functions as short answer to the \isi{tag question} has the default neuter agreement although the \isi{copula} in the assertion before the \isi{tag question} is inflected for masculine singular.

\begin{exe}
	\ex	\label{ex:About what had happened he is talking, right? Yes}
	\gll	ag-ur	d-iχ-ub-t-a-lla	qari=či-b 	ʁaj	Ø-ik'-ul	ca-w	heχ,	akːu=w?	ca-b\\
		go\tsc{.pfv-pret}	\tsc{npl-}be\tsc{.pfv-pret-pl-obl-gen}	at.top=on\tsc{-n}	word	\tsc{m-}say\tsc{.ipfv-icvb}	\tsc{cop-m}	\tsc{dem.down}	\tsc{cop.neg=q}	\tsc{cop-n}\\
	\glt	\sqt{About what had happened he is talking, right? Yes.}

	\ex	\label{ex:This is the company (group of friends), right}
	\gll	ij	kampanija	ca-b,	akːu=w?\\
		this	company	\tsc{cop-hpl}	\tsc{cop.neg=q}\\
	\glt	\sqt{This is the company (group of friends), right?}

	\ex	\label{ex:‎In my opinion, he was imprisoned and then came back from there, right}
	\gll	di-la	pikri	ħaˁsible	ka-jž-ib-le	het-ka	čar	Ø-iχ-ub	ca-w,	akːu=w?\\
		\tsc{1sg-gen}	thought	following	\tsc{down}-be\tsc{.m.pfv-pret-cvb}	that-\tsc{down}	back \tsc{m-}be\tsc{.pfv-pret}	\tsc{cop-m}	\tsc{cop.neg=q}\\
	\glt	\sqt{‎In my opinion, he was imprisoned and then came back from there, right?}

	\ex	\label{ex:‎We did not eat boars, right}
	\gll	žaq'-ne	a-d-uk-i	nušːa-l,	akːʷ-i=w?\\
		boar\tsc{-pl}	\tsc{neg-npl-}eat\tsc{.ipfv-hab.pst}	\tsc{1pl-erg}		\tsc{cop.neg-hab.pst=q}\\
	\glt	\sqt{‎We did not eat boars, right?}
\end{exe}

The tense of the \isi{copula} in the \isi{tag question} reflects the tense of the verb (\isi{copula} or other) or the tense of the \isi{copula} if it is an analytic inflection form in the assertion preceding the tag. For instance, in \refex{ex:‎We did not eat boars, right} the verb in the assertion appears in the \isi{habitual past} and the \isi{copula} in the tag also appears in the \isi{habitual past}.

When the assertion is negative the tag marker normally has positive polarity as, e.g., the \isi{copula} \tit{ca-b} \refex{ex:(These) are not fish, are they? By God, I don't know} or the verb in \refex{ex:And these friends, what they are saying, only by means of the picture, (one) cannot know, can one? One cannot know, yes}, which is simply the \isi{negation} of the predicate in the assertion.\footnote{Note that this is not the case in example \refex{ex:‎We did not eat boars, right}, and I do not have an explanation for this example.}

\begin{exe}
	\ex	\label{ex:(These) are not fish, are they? By God, I don't know}
	\gll	baliqː-e	akːu=q'al,	ca-d=uw? 	aχːu	wallah\\
		fish\tsc{-pl}	\tsc{cop.neg=mod}	\tsc{cop-npl=q} 	not.know	by.God\\
	\glt	\sqt{(These) are not fish, are they? By God, I don't know.}

	\ex	\label{ex:And these friends, what they are saying, only by means of the picture, (one) cannot know, can one? One cannot know, yes}
	\gll	[a	iš-tːi	juldašː-e	ce	b-ik'-ul=el]	tolko	hel sːurrat	ħaˁsible	b-aχ-ij	a-w-irχʷ-ar,	w-irχʷ-an-ne=w? 		e-rχʷ-an-ne,	e\\
		but	that\tsc{-pl}	friend\tsc{-pl}	what	\tsc{n-}say\tsc{.ipfv-icvb=indq}	only	that picture	following	\tsc{n-}know\tsc{.pfv-inf}	\tsc{neg-m-}be.able\tsc{.ipfv-prs} \tsc{m-}be.able\tsc{.ipfv-ptcp-fut.3=q}	\tsc{neg-}be.able\tsc{.ipfv-ptcp-fut.3}	yes\\
	\glt	\sqt{And these friends, what they are saying, only by means of the picture, (one) cannot know, can one? One cannot know, yes.}
\end{exe}


%%%%%%%%%%%%%%%%%%%%%%%%%%%%%%%%%%%%%%%%%%%%%%%%%%%%%%%%%%%%%%%%%%%%%%%%%%%%%%%%

\section{Subordinate questions}
\label{sec:Subordinate questions}

Subordinate \isi{questions} are marked with an \isi{enclitic} that has three allomorphs: (i) \tit{=jal} after vowels, (ii) \tit{=el} after \isi{consonants}, and (iii) \tit{=l} after some suffixes ending in /e/, in particular the \isi{perfective converb} suffix. The \isi{enclitic} occurs in all types of embedded interrogatives, i.e. content \isi{questions}, polar and disjunctive polar \isi{questions}. It is also used as a complementizer with verbs of speech and cognition (\refsec{ssec:The embedded question marker}), and for the formation of specific \is{indefinite pronoun}indefinite pronouns (\refsec{ssec:Specific indefinite pronouns}). The rules for the placement of the \isi{enclitic} are the same as for normal \isi{questions}. This means that in embedded polar \isi{questions} the marker occurs on the verb if there is any \refex{ex:We went to see if (the cartridge) had struck}, and otherwise on the non-verbal predicate \refex{ex:I don't know whether these are watermelons}.

\begin{exe}

	\ex	\label{ex:We went to see if (the cartridge) had struck}
	\gll	ag-ur=da	[ačː-ib=el]	či-d-až-ij\\
		go\tsc{.pfv-pret=1}	get\tsc{.pfv-pret=indq}	\tsc{spr-1/2pl-}see\tsc{.pfv-inf}\\
	\glt	\sqt{We went to see if (the cartridge) had struck.}

	\ex	\label{ex:I don't know whether these are watermelons}
	\gll	a	[iš-tːi	qːalpuz-e=jal]	aχːu	heštːi\\
		and	this\tsc{-pl}	watermelon\tsc{-pl=indq}	not.know	these\\
	\glt	\sqt{I don't know whether these are watermelons.}

\end{exe}

In embedded disjunctive polar \isi{questions} it occurs on all members of the disjunction:

\begin{exe}
	\ex	\label{ex:He ran and ran, looked around, wondering whether his sister is there or not}
	\gll	duc'	Ø-ik'-ul,	duc'	Ø-ik'-ul,	er=itːi	sark'-ul,		[le-r=el	rucːi	r-akːu=jal]	Ø-ik'-ul	hel\\
		run	\tsc{m-}\tsc{aux.ipfv-icvb}	run	\tsc{m-}\tsc{aux.ipfv-icvb}	look=after	inspect\tsc{.ipfv-icvb}	exist\tsc{-f=indq}	sister	\tsc{f-cop.neg=indq}	\tsc{m-}say\tsc{.ipfv-icvb}	that\\
	\glt	\sqt{He ran and ran, looked around, wondering whether his sister is there or not.}

	\ex	\label{ex:Is this water or wine, I do not know}
	\gll	[hin	ca-d=el	iχ-tːi	čaˁʁir=el]	b-alχ-an	w-akːu\\
		water	\tsc{cop-npl=indq}	\tsc{dem.down-pl}	wine\tsc{=indq}	\tsc{n-}know\tsc{.ipfv-ptcp}	\tsc{m}-\tsc{cop.neg}\\
	\glt	\sqt{Is this water or wine, (I) do not know.}
\end{exe}

In embedded content \isi{questions} the \isi{enclitic} appears on the verbal or non-verbal predicate:

\begin{exe}
	\ex	\label{ex:He is thinking, sitting, about what must be done}
	\gll	pikri	Ø-ik'-ul	ka-jž-ib	ca-w	heχ	[ce		b-arq'-ij	ħaˁžat-le=jal]	Ø-ik'-ul\\
		thought	\tsc{m-}say\tsc{.ipfv-icvb}	\tsc{down}-remain\tsc{.m.pfv-pret}	\tsc{cop-m}	\tsc{dem.down}	what	\tsc{n-}do\tsc{.pfv-inf}	need\tsc{-advz=indq}	\tsc{m-}say\tsc{.ipfv-icvb}\\
	\glt	\sqt{He is thinking, sitting, about what must be done.}

	\ex	\label{ex:‎‎‎We are thinking where to meet Abdulkhalik}
	\gll	na	[ʡaˁbdulχaliq'	čina-w	suk	sa-w-irk-u=jal]	pikri	d-ik'-ul=da	nušːa	\\
		now	Abdulkhalik	where\tsc{-m}	meet	\tsc{hither-m-}occur\tsc{.ipfv-prs=indq}	thought	\tsc{1/2pl}-say.\tsc{ipfv-icvb=1}	\tsc{1pl}\\
	\glt	\sqt{‎‎‎We are thinking where to meet Abdulkhalik.}

	\ex	\label{ex:‎Well we knew which difficulties we had seen}
	\gll	nu	[ceʁuna	qːihin-dexː-e	či-d-až-ib=da=jal]	nišːi-j	b-alχ-ul=de\\
		well	which	difficult\tsc{-nmlz-pl}	\tsc{spr-npl-}see\tsc{.pfv-pret=1=indq}	\tsc{1pl-dat}	\tsc{n-}know\tsc{.ipfv-icvb=pst}\\
	\glt	\sqt{‎Well, we knew which difficulties we had seen.}
\end{exe}

The Sanzhi \isi{enclitic} is not only used in embedded \isi{questions}, but also in the protasis of realis and irrealis \is{conditional clause}conditional clauses \refex{ex:If the woman would have stolen, they would/should have imprisoned her} (\refsec{cpt:conditionalconcessiveclauses}). The apodosis of irrealis conditionals can be omitted, in which case the construction expresses wishes similar to an \isi{optative} \refex{ex:If they would go! They bore (me).}.

\begin{exe}
	\ex	\label{ex:If the woman would have stolen, they would/should have imprisoned her}
	\gll	r-ilʡ-uˁnne	r-iχʷ-ar=de=l,	xːunul r-i-ka-jʁ-an=de=q'al\\
		\tsc{f}-steal.\tsc{ipfv-icvb}	\tsc{f}-be.\tsc{pfv-cond.3=pst=indq}	woman \tsc{f-in-down}-drive.\tsc{pfv=ptcp=pst=mod} \\
	\glt	\sqt{If the woman would have stolen, they would/should have imprisoned her.}
		
	\ex	\label{ex:If they would go! They bore (me).}
	\gll	b-uˁq'-aˁn-de=l!	či-b-b-et'-ib	ca-b\\
		\tsc{hpl}-go-\tsc{ptcp=pst=indq}	\tsc{spr-hpl-hpl}-bore.\tsc{pfv-pret}	\tsc{cop-hpl} \\
	\glt	\sqt{If they would go! They bore (me).} (E)
\end{exe}


The marker for embedded \isi{questions} has also evolved into a general marker of epistemic uncertainty occurring in rhetorical \isi{questions} for which the speaker does not expect an answer \refex{ex:How should this be said, oh, Allah!A}.

\begin{exe}
	\ex	\label{ex:How should this be said, oh, Allah!A}
	\gll	cet'le	herʔ-an-ne=l,	aj	Allah!\\
		how	say\tsc{.ipfv-ptcp-fut.3=indq}	oh	Allah\\
	\glt	\sqt{How should this be said, oh, Allah!} (i.e. \sqt{How should I say this?})
\end{exe}

And more importantly, it is used when speakers are expressing thoughts whose truth they do not vouch for and when they advance hypotheses, i.e. for the expression of epistemic modality, more specifically, epistemic uncertainty. In such cases there is no matrix \isi{complement-taking predicate} \refex{ex:They are making a trial or what}, \refex{ex:Then he went to prison or so / or what}, but optionally the borrowed adverb \textit{belki} may occur \refex{ex:‎Maybe this is when he came from prison}. Since the marker for embedded \isi{questions} belongs, just like the other interrogative enclitics, to the class of predicative \is{particle}particles, it can be used with a converb, but the resulting clause is finite (\refsec{sec:Predicative particles}). Such examples might be interpreted as ``\isi{insubordination}'' (\citealp{Evans2007, EvansWatanabe2016}, see also \citealp{Mithun2008}) because a clause with a dependency marker is used as an independent sentence. 

\begin{exe}
	\ex	\label{ex:They are making a trial or what}
	\gll	heš-tː-a-l	sud	b-irq'-ul=el\\
		this\tsc{-pl-obl-erg}	trial	\tsc{n-}do\tsc{.ipfv-icvb=indq}\\
	\glt	\sqt{They are probably making a trial.}

	\ex	\label{ex:Then he went to prison or so / or what}
	\gll	c'il	abuχar	tusnaq-le	w-ič-ib=el\\
		then	then	prison\tsc{-loc}	\tsc{m-}occur\tsc{.pfv-pret=indq}\\
	\glt	\sqt{Then he probably went to prison.}

	\ex	\label{ex:‎Maybe this is when he came from prison}
	\gll	belki	ij	tusnaq-le-r	sa-jʁ-ib-il	ca-w=el\\
		it.is.possible	this	prison-\tsc{spr-abl}	\tsc{hither}-come\tsc{.m.pfv-pret-ref}	\tsc{cop-m=indq}\\
	\glt	\sqt{‎Maybe this is when he came from prison.}
\end{exe}

This use has been conventionalized in the frequently occurring phrase \tit{ce ca-b=el} (\tit{ce ca-d=el}) \sqt{what it might be, whatever} lit. \sqt{what \tsc{cop-n-indq}} \refex{ex:He came out to reconcile them or so or there is something else}.

\begin{exe}
	\ex	\label{ex:He came out to reconcile them or so or there is something else}
	\gll	masliʡaˁt	b-arq'-ij	saˁ-q'-un-ne=jal	cara	ce	ca-b=el\\
		reconciliation	\tsc{n-}do\tsc{.pfv-inf}	\tsc{hither}-go\tsc{.pfv-pret-cvb=indq}	other	what	\tsc{cop-n=indq}\\
	\glt	\sqt{He came out to reconcile them or so or there is something else.}
\end{exe}

\citet[367]{Evans2007}, who introduces the term ``\isi{insubordination}'', defines it as ``the conventionalized main clause use of what, on prima facie grounds, appear to be formally subordinate clauses.'' He explains the diachronic development of \isi{insubordination} by means of four steps:
\begin{enumerate}
\item subordinate construction with subordinate morphosyntax
\item ellipsis of main clause
\item restriction of interpretation of ellipsed material
\item reanalysis (conventionalized main clause use of formally subordinate clause)
\end{enumerate}
Evans' four-step model provides a very plausible path of the diachronic change: the Sanzhi \isi{particle} underwent all four stages, but at the same time preserved its function as complementizer. Thus, the syntactic development can be summarized as \refex{ex:complementizer}, accompanied by the semantic extension schematized in \refex{ex:complementizer2}.

\begin{exe}
	\ex	 \label{ex:complementizer}
		syntax: complementizer > to predicative \isi{particle} with a functional range similar to copula-auxiliaries 
\end{exe}

\begin{exe}
	\ex	 \label{ex:complementizer2}
	semantics: marker of embedded \isi{questions} > epistemic modality (more specifically, uncertainty)
\end{exe}

Examples for the first step are \xxref{ex:We went to see if (the cartridge) had struck}{ex:‎Well we knew which difficulties we had seen}. Embedded interrogatives occur with matrix verbs that denote cognitive activities (which can include perception verbs such as ‘see'). Embedded interrogatives are often of irrealis modality and therefore not asserted as factual or actual events or situations \refex{ex:I don't know whether these are watermelons}. During the third step the interpretation was restricted from various possible main clauses to an omitted main clause as general as ‘It is probable that X'. Language-internally, the reanalysis (step 4) might have been supported by the presence of the other predicative \is{particle}particles. This means that by analogy with the person markers, the past \isi{enclitic} or the modal \isi{particle} the \isi{embedded question marker} received its syntactic ability to express \isi{finiteness} of clauses.

However, there is one general problem with the \isi{insubordination} analysis of the \isi{enclitic} \textit{=(e)l}\slash\textit{=(j)al}. In principle, it is possible that the diachronic development occurred in the reverse order, i.e. that the \isi{particle} was originally a marker of epistemic modality that subsequently came be to be used in embedded interrogatives due to its epistemic modal meaning. Because we lack data of older stages of Sanzhi Dargwa, this question cannot be resolved with certainty.

Insubordination of the type just described is also found in other East Caucasian languages. Examples in case are irrealis markers in the Tsezic languages Bezhta, Hunzib, and Hinuq, the potential \isi{infinitive} in Bagvalal, and the potentialis (i.e. the \isi{infinitive}) in Tsakhur (see \citet{ComrieForkerKhalilova2016} for examples and references). The study by \citet{Kalinina2011} provides many examples of insubordinated exclamative utterances in Agul, Archi, Avar, Bagvalal, and Bezhta (East Caucasian) as well as Adyghe (West Caucasian). Of the surveyed languages only Agul employs an irrealis \isi{conditional} form in embedded \isi{questions} and exclamatives in a similar fashion as the Sanzhi example \refex{ex:If they would go! They bore (me).}.






%%%%%%%%%%%%%%%%%%%%%%%%%%%%%%%%%%%%%%%%%%%%%%%%%%%%%%%%%%%%%%%%%%%%%%%%%%%%%%%%

\section{Other uses of questions}
\label{sec:Other uses of questions}

Some \isi{questions} are normally used as greetings. Thus female addressees are not greeted with the Arabic phrase \tit{As-salam ʡaˁlaykum}, but with the sentences below; the first variant is for single addressees \refex{ex:Are you sitting? I am sitting@A}, the second for situations when the addressee is more than one person \refex{ex:Are you sitting? I am sitting@C}. The answer is again a clause \refex{ex:Are you sitting? I am sitting@B}.

\begin{exe}
	\ex	\label{ex:Are you sitting? I am sitting}
	\begin{xlist}
		\ex	\label{ex:Are you sitting? I am sitting@A}
		\gll	ka-r-iž-ib-le=de=w!	\\
			\tsc{down-f-}be\tsc{.pfv-pret-cvb=2sg=q}		\\
		\glt	\sqt{Hello!} (lit. \sqt{Are you sitting?}) (singular addressee)

		\ex	\label{ex:Are you sitting? I am sitting@C}
		\gll		ka-d-iž-ib-le=da=w!\\
			 \tsc{down-1/2pl-}be\tsc{.pfv-pret-cvb=2pl=q}	\\
		\glt	\sqt{Hello!} (lit. \sqt{Are you sitting?}) (plural addressee)
		
		\ex	\label{ex:Are you sitting? I am sitting@B}
		\gll	ka-r-iž-ib-le=da\\
			\tsc{down-f-}be\tsc{.pfv-pret-cvb=1}\\
		\glt	\sqt{Hello!} (lit. \sqt{I am sitting.})
	\end{xlist}
\end{exe}

For greeting people who have just got up in the morning, the sentences in \refex{ex:Good morning} are used. The addressee can be female or male.

\begin{exe}
	\ex	\label{ex:Good morning}
	\gll	r-alh-un=de=w!	/	w-alh-un=de=w!\\
		\tsc{f-}wake.up\tsc{.pfv-pret=2sg=q}	/	\tsc{m-}wake.up\tsc{.pfv-pret=2sg=q}\\
	\glt	\sqt{Good morning!} (female\slash male addressee)
\end{exe}

When Sanzhi speakers meet during the day they might ask when greeting each other:

\begin{exe}
	\ex	\label{ex:What (work) have you done?}
	\gll	ce	ʡaˁči	b-arq'-ib=de?\\
		what	work	\tsc{n-}do\tsc{.pfv-pret=2sg}\\
	\glt	\sqt{What (work) have you done?}
\end{exe}

This question can be taken literally, but it is also a general question of the type ``How do you do?''

%\begin{exe}
%	\ex	\label{ex:}
%	\gll	\\
%		\\
%	\glt	\sqt{}
%\end{exe}
