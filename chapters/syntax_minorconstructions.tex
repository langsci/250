\chapter{Minor constructions}
\label{cpt:Minor constructions}


%%%%%%%%%%%%%%%%%%%%%%%%%%%%%%%%%%%%%%%%%%%%%%%%%%%%%%%%%%%%%%%%%%%%%%%%%%%%%%%%

\section{Comparative constructions}
\label{sec:Comparative constructions}

In \isi{comparative constructions} two or more items are examined in order to note similarities and differences in degree between them \citep[787]{Dixon2008}. Inequality between two items is expressed by means of one of the \isi{spatial cases} (\refsec{sssec:spr-lative -le/-ja/-a, spr-essive -le-b/-ja-b/-a-b and spr-ablative -le-r/-ja-r/-a-r}). In superlative constructions, \isi{degree adverbs} occur. Equative constructions and the expression of similarity are realized by means of several \isi{particles} (\refsec{sec:Equative constructions and the expression of similarity}).

In Sanzhi \isi{comparative constructions} we find a comparee, the \isi{standard of comparison}, and the parameter of comparison. The \isi{standard of comparison} is marked with the \tsc{loc}-\isi{ablative} case that has the suffixes \tit{-ler(ka)}, \tit{-ar(ka)} or \tit{-jar(ka)} (\refsec{sssec:spr-lative -le/-ja/-a, spr-essive -le-b/-ja-b/-a-b and spr-ablative -le-r/-ja-r/-a-r}). It is cross-linguistically common to mark the \isi{standard of comparison} with an \isi{ablative} (or locative) case \citep[791]{Dixon2008}, and East Caucasian languages including Dargwa varieties nicely confirm this tendency. Neither the comparee nor the parameter of comparison bears any special marking. Consequently, if the \isi{standard of comparison} were to be omitted, the construction would be a \isi{simple clause} and not a \isi{comparative construction}. Most commonly the standard precedes the comparee. The parameter is a gradable adjective or adverb that occurs in its plain form without any additional index (as, e.g., English \tit{more}).
%
\begin{exe}
	\ex	\label{ex:‎Bahmud was smarter than Bahamma}
	\gll	Baħaˁmma-ja-rka	Baˁħmud	šːustri=de	\\
		Bahamma\tsc{-loc-abl}	Bahmud	smart\tsc{=pst}\\
	\glt	\sqt{‎Bahmud was smarter than Bahamma.}

	\ex	\label{ex:‎‎There were those older than father}
	\gll	atːa-ja-r	χːula-te=ra	b-irχʷ-i\\
		father\tsc{-loc-abl}	big\tsc{-dd.pl=add} \tsc{hpl-}be\tsc{.ipfv-hab.pst.3}\\
	\glt	\sqt{‎‎There were those older than father.}

	\ex	\label{ex:‎‎It (bread) is better (when made) of barley than of wheat.}
	\gll	ij	ač'i-lla-ja-rka	[\ldots]	muqi-lla=ra	ʡaˁħ-ce	b-irχ-u	\\
		this	wheat\tsc{-gen-loc-abl}	{} barley\tsc{-gen=add}	good\tsc{-dd.sg}	\tsc{n-}become\tsc{.ipfv-prs.3}\\
	\glt	\sqt{‎‎It (bread) is better (when made) of barley than of wheat.}

	\ex	\label{ex:‎I (will) make a big barrow (maχ) earlier than you}
	\gll	u-le-rka	sala-r	du-l	maχ	χːula-ce	b-arq'-ij\\
		\tsc{2sg-loc-abl}	front\tsc{-abl}	\tsc{1sg-erg}	barrow	big\tsc{-dd.sg}	\tsc{n-}do\tsc{.pfv-inf}	\\
	\glt	\sqt{‎I (will) make a big barrow (\tit{maχ}) earlier than you.}

	\ex	\label{ex:‎‎‎The animals had apparently more conscience than our rich (people)}
	\gll	žaniwar-t-a-lla	χʷal-le	jaˁħ=ra		namus=ra	b-už-ib	ca-b	nišːa-la	dawla-či-b-t-a-lla-ja-r	\\
		animal\tsc{-pl-obl-gen}	big\tsc{-advz}	conscience\tsc{=add}	conscience\tsc{=add}	\tsc{n-}stay\tsc{-pret}	\tsc{cop-n}	\tsc{1pl-gen}	wealth\tsc{-adjvz-hpl-pl-obl-gen-loc-abl}\\
	\glt	\sqt{‎‎‎The animals had apparently more conscience than our rich (people).} (lit. \sqt{their conscience was bigger})
\end{exe}

Superlative constructions contain a comparee, a \isi{standard of comparison}, and a parameter. They basically have the same structure as the constructions described so far in this section. The only differences are the case marking of the standard, which is now the \tsc{in}-lative, and the additional \isi{degree adverb} modifying the parameter. The \isi{standard of comparison} can be omitted if it is inferable from the context
%
\begin{exe}
	\ex	\label{ex:‎‎‎Among all as the most beautiful (country) seemed to me Latvia}
	\gll	li<b>il-li-cːe-rka	bah	qːuʁa-ce	dune	ka-b-icː-ur-il	dam	dejstwitelno	Latwija=de	\\
		all\tsc{<n>-obl-in-abl}	most	beautiful\tsc{-dd.sg}	world	\tsc{down-n-}stand\tsc{.pfv-pret-ref}	\tsc{1sg.dat}	really	Latvia\tsc{=pst}	\\
	\glt	\sqt{‎‎‎Among all as the most beautiful (country) seemed to me Latvia.}

	\ex	\label{ex:‎‎‎She was the oldest within her family.}
	\gll	il	kulpat-li-cːe-r	bah	χːula-ce	r-už-ib	ca-r\\
		that	family\tsc{-obl-in-abl}	most	big\tsc{-dd.sg}	\tsc{f-}be\tsc{-pret}	\tsc{cop-f}\\
	\glt	\sqt{‎‎‎She was the oldest within her family.}

	\ex	\label{ex:‎‎‎The worst (place) was Azerbajan}
	\gll	bah	wahi-ce	ʡaˁzirbažan=de\\
		most	bad\tsc{-dd.sg}	Azerbajan\tsc{=pst}\\
	\glt	\sqt{‎‎‎The worst (place) was Azerbajan.}
\end{exe}


%%%%%%%%%%%%%%%%%%%%%%%%%%%%%%%%%%%%%%%%%%%%%%%%%%%%%%%%%%%%%%%%%%%%%%%%%%%%%%%%

\section{Equative constructions and the expression of similarity}
\label{sec:Equative constructions and the expression of similarity}

For \isi{equative constructions} and the expression of similarity Sanzhi has two \isi{particles}, \tit{ʁuna} and \tit{daˁʡle} \sqt{as, like}, and the adjective \tit{miši} \sqt{similar}.

The \isi{particle} \tit{ʁuna} \sqt{as, like} immediately follows the parameter of comparison that it has scope over like, e.g., focus-sensitive \isi{particles}. The parameter can be a pronoun, an adjective \refex{ex:He is like a young man}, an adverbial \refex{ex:There were not such customs as here}, or a noun \refex{ex:This was like a mug}. Very often it is simply a \isi{demonstrative pronoun}, and the combination of demonstrative and equative \isi{particle} means \sqt{like this, such} \refex{ex:They were so good people}. Depending on the parameter, the \isi{particle} thus appears, e.g., within noun phrases \refex{ex:He is like a young man} or in the position of adverbial modifiers.
%
\begin{exe}
	\ex	\label{ex:He is like a young man}
	\gll	žahil	ʁuna	admi	ca-w	heχ\\
		young	\tsc{eq}	person	\tsc{cop-m}	\tsc{dem.down}\\
	\glt	\sqt{He is like a young man.}

	\ex	\label{ex:There were not such customs as here}
	\gll	heštːu-d	ʁuna	ʡaˁdat-urme	akːʷ-i\\
		here\tsc{-npl}	\tsc{eq}	custom\tsc{-pl}	\tsc{cop.neg-hab.pst.3}\\
	\glt	\sqt{There were not such customs as here.}

	\ex	\label{ex:They were so good people}
	\gll	hel	ʁuna	ʡaˁħ	χalq'	b-irχʷ-iri\\
		that	\tsc{eq}	good	people	\tsc{hpl-}become\tsc{.ipfv-hab.pst.3}\\
	\glt	\sqt{They were good people like that.}
\end{exe}

It can also occur as a predicate in a \isi{copula} clause without a head noun and it can be nominalized by suffixing -\textit{b} (unclear origin) and the \isi{cross-categorical suffix} in the plural form -\textit{te} (\tit{ʁunabte}).
%
\begin{exe}
	\ex	\label{ex:This was like a mug}
	\gll	heχ	kuruškːa	ʁuna	b-irχʷ-i\\
		\tsc{dem.down}	mug \tsc{eq}	\tsc{n-}be\tsc{.ipfv-hab.pst.3}\\
	\glt	\sqt{This was like a mug.}
\end{exe}

The \isi{particle} \tit{daˁʡle} \sqt{as, like}, which diachronically seems to be an adverbial derived with the adverbializing suffix -\textit{le}, has a meaning very similar but not identical to \tit{ʁuna}. It indicates only that some situation or some item resembles another situation or item. Both \isi{particles} slightly differ in their distribution. The \isi{particle} \tit{daˁʡle} follows the parameter of comparison over which it has scope. As with \tit{ʁuna}, the parameter can be expressed by \isi{nouns} \refex{ex:‎This looks like kiwi, similar to kiwi or so}, adverbials \refex{ex:At that (time) there were no minibuses like now}, or \isi{adjectives} \refex{ex:This woman looks even like old.}. But in contrast to \tit{ʁuna}, \tit{daˁʡle} is most frequently used in non-finite clauses headed by \isi{participles} \refex{ex:as I said} or the \isi{infinitive} \refex{ex:He is keeping his jar as if it fell down}. 
%
\begin{exe}
	\ex	\label{ex:‎This looks like kiwi, similar to kiwi or so}
	\gll	kiwi	daˁʡle	χe-d	heχ-tːi,	kiwi	ʁuna	cik'al\\
		kiwi	as	exist.\tsc{down-npl}	\tsc{dem.down}\tsc{-pl}	kiwi \tsc{eq}	something\\
	\glt	\sqt{‎This looks like kiwi, something similar to kiwi.}

	\ex	\label{ex:At that (time) there were no minibuses like now}
	\gll	it=qːella	hana	daˁʡle	maršrutka-be	a-d-irχʷ-i=q'al\\
		that=when	now	as	minibus\tsc{-pl}	\tsc{neg-npl-}be\tsc{.ipfv-hab.pst.3=mod}\\
	\glt	\sqt{At that (time) there were no minibuses like now.}

	\ex	\label{ex:This woman looks even like old.}
	\gll	heχ	xːunul	bulan	r-uqna-ce	daˁʡle	či-r-ig-ul	ca-r\\
		\tsc{dem.down}	woman	even	\tsc{f-}old\tsc{-dd}	as	\tsc{spr-f-}see\tsc{.ipfv-icvb}	\tsc{cop-f}\\
	\glt	\sqt{This woman even looks like she is old.} 

	\ex	\label{ex:as I said}
	\gll	du-l	haʔ-ib daˁʡle\\
		\tsc{1sg-erg}	say\tsc{.pfv-pret}	as\\
	\glt	\sqt{as I said}
	
		\ex	\label{ex:He is keeping his jar as if it fell down}
	\gll	qaˁjqaˁj	b-uc-ib ca-b	a-ka-b-ič-ij	daˁʡle\\
		jaw	\tsc{n-}catch\tsc{.pfv-pret} \tsc{cop-n}	\tsc{neg-down}\tsc{-n-}occur\tsc{.pfv-inf}	as\\
	\glt	\sqt{He is keeping his jar as if it fell down.} (lit. \sqt{like not to fall down})
\end{exe}

Finally, the adjective \tit{miši} \sqt{similar} assigns the \isi{dative} case to its complement that represents the \isi{standard of comparison} \refex{ex:This is not similar to a prison}. In \isi{copula} clauses, in which it is used in the \isi{copula} complement, the adverbializing suffix \tit{-le} is added, as it regularly happens with adjectival stems in \isi{copula} construction.
%
\begin{exe}
	\ex	\label{ex:This is not similar to a prison}
	\gll	tusnaq-li-j	miši-l	akːu\\
		prison\tsc{-obl-dat}	similar\tsc{-advz}	\tsc{cop.neg}\\
	\glt	\sqt{This is not similar to a prison.}
\end{exe}

The differences between the three \isi{comparative constructions} lie mostly in their morphosyntactic behavior, with an additional semantic distinction between \tit{ʁuna} and \tit{daˁʡle} on the one side, and \tit{mišil} on the other \refex{ex:‎‎‎(He) is like a young man. (i.e. He seems to be young, he looks young or behaves as if he were young)}, \refex{ex:‎‎‎(He) is similar to a young man}. The \isi{particles} \tit{ʁuna} and \tit{daˁʡle} have the distribution of focus-sensitive \isi{particles} and can therefore occur within certain types of phrases as, e.g., noun phrases, but do not assign case to the items they scope over, in contrast to the case-assigning adjective \tit{miši}.
%
\begin{exe}
	\ex	\label{ex:‎‎‎(He) is like a young man. (i.e. He seems to be young, he looks young or behaves as if he were young)}
	\gll	žahil	admi	ʁuna	/	daˁʡle	ca-w\\
		young	person	\tsc{eq}	/	as	\tsc{cop-m}\\
	\glt	\sqt{‎‎‎(He) is like a young man. (i.e. He seems to be young, he looks young or behaves as if he were young)}

	\ex	\label{ex:‎‎‎(He) is similar to a young man}
	\gll	žahil	admi-li-j	miši-l	ca-w\\
		young	person\tsc{-obl-dat}	similar\tsc{-advz}	\tsc{cop-m}\\
	\glt	\sqt{‎‎‎(He) is similar to a young man.}
\end{exe}


%%%%%%%%%%%%%%%%%%%%%%%%%%%%%%%%%%%%%%%%%%%%%%%%%%%%%%%%%%%%%%%%%%%%%%%%%%%%%%%%

\section{Comitative constructions}
\label{sec:Comitative constructions}

Sanzhi has two ways of expressing \isi{comitative} meaning: case marking (in combination with optional postpositions) and a construction involving the use of \isi{reflexive pronouns}.

The cases used are the \isi{comitative} case (\tit{-cːella}, \refsec{sssec:Comitative}) or, more rarely, the \tsc{in}-\isi{ablative} case (\refsec{sssec:in-lative -cːe, in-essive -cːe-b, and in-ablative -cːe-r}). They can occur together with the postposition \tit{b-alli} (\refsec{ssec:postposition balli}) or the postposition\slash adverb \tit{canille} (\refsec{ssec:postposition canille}). These constructions can be used with animate and inanimate \isi{nouns}. In the latter case they can express instruments \refex{ex:having made a hole with a fork}.
%
\begin{exe}
	\ex	\label{ex:having made a hole with a fork}
	\gll	č'ala-cːella	ʡaˁmi	ka-b-at-ur-re\\
		fork\tsc{-comit}	hole	\tsc{down-n-}let\tsc{.pfv-pret-cvb}\\
	\glt	\sqt{having made a hole with a fork}

	\ex	\label{ex:Well, he was together with him at that time, right}
	\gll	nu,	iž-i-cːella	canille=qːel	il	hel	zamana,	akː-u=w?\\
		well	this\tsc{-obl-comit}	together=\tsc{when}	that	that	time	\tsc{cop.neg-prs.3=q}\\
	\glt	\sqt{Well, he was together with him at that time, right?}
\end{exe}

There does not seem to be a clear semantic difference between \tit{b-alli} and \tit{canille} \refex{ex:‎The lid should be together with the pot}. The two items can only be distinguished by means of their morphosyntactic behavior, because \tit{b-alli} agrees in \isi{gender} with the argument in the \isi{absolutive} \refex{ex:‎The lid should be together with the pot} and it always implies a complement even when the complement is not overtly expressed. For instance, \refex{ex:‎Madina and I came together (with somebody else / with other people)} entails that there were other people with whom we came, whereas in \refex{ex:‎Madina and I came together} there is no such implication and \tit{canille} only functions as an adverb that expresses the fact that Madina and the speaker came together:
%
\begin{exe}
	\ex	\label{ex:‎The lid should be together with the pot}
	\gll	burta	ħaˁšak-li-cːella	b-alli	/	canille	b-irχʷ-an	ca-b\\
		lid	pot\tsc{-obl-comit}	\tsc{n-}together	/	together	\tsc{n-}be\tsc{.ipfv-ptcp}	\tsc{cop-n}\\
	\glt	\sqt{‎The lid should be together with the pot.} (E)

	\ex	\label{ex:‎Madina and I came together (with somebody else / with other people)}
	\gll	Madina=ra	du=ra	d-alli	ag-ur=da\\
		Madina\tsc{=add}	\tsc{1sg=add}	\tsc{1/2pl-}together	go\tsc{.pfv-pret=1}\\
	\glt	\sqt{‎Madina and I came together (with somebody else\slash with other people).} (E)

	\ex	\label{ex:‎Madina and I came together}
	\gll	Madina=ra	du=ra	canille	ag-ur=da\\
		Madina\tsc{=add}	\tsc{1sg=add}	together	go\tsc{.pfv-pret=1}\\
	\glt	\sqt{‎Madina and I came together.} (E)
\end{exe}

The second construction is the use of a \isi{reflexive pronoun} in what looks like a \isi{coordination} of noun phrases. This construction has been described for Standard Dargwa by \citet{vandenBerg2004}. The structure is [Y\tit{=ra} X\tit{=ra}] \sqt{X with Y}. X refers to an animate (usually human) entity and is formally expressed through the \isi{reflexive pronoun}. Y is a nominal that can be animate or inanimate and takes case suffixes. It can be a common noun, a pronoun, a personal name or any other type of \isi{noun phrase}. Both X and Y are marked with the \isi{additive} \tit{=ra} and are often adjacent to each other, which makes them look like a coordinated \isi{noun phrase}. However, the argument referred to by the reflexive can be expressed independently. Furthermore, the coordinated \isi{noun phrase} usually does not take an argument position in the clause. It is rather one of the individual members that functions as argument. For instance, in \refex{ex:‎‎With a bucket of water he is standing} the pronoun \tit{heχ} that is following the \isi{comitative} phrase represents the subject of the following verb as the agreement on the verb shows (masculine singular).
%
\begin{exe}
	\ex	\label{ex:‎‎With a bucket of water he is standing}
	\gll	[hin-na	badra=ra	ca-w=ra]	heχ	ka-jcː-ur	ca-w\\
		water\tsc{-gen}	bucket\tsc{=add}	\tsc{refl-m=add}	\tsc{dem.down}	\tsc{down}-stand\tsc{.m.pfv-pret}	\tsc{cop-m}\\
	\glt	\sqt{‎‎With a bucket of water he is standing.}

	\ex	\label{ex:He ran away with his dog}
	\gll	sa-r-uq-un	ca-w	χːʷe=ra	ca-w=ra\\
		\tsc{hither-abl}-go.\tsc{pfv-pret}	\tsc{cop-m}	dog\tsc{=add}	\tsc{refl-m=add}\\
	\glt	\sqt{He ran away with his dog.}
\end{exe}

Example \refex{ex:‎Everybody is with a bottle in their hands. (lit. There is one bottle each in everybody's (hand), and they also.)} shows that the two items bearing the \isi{additive enclitic} \tit{=ra} can be separated by other material. The agreement on the \isi{existential copula} is controlled by the first \isi{noun phrase} \tit{ca ca šuša,} which is semantically plural and functions as the \isi{copula} subject of the existential\slash \isi{locational copula} \textit{χe-d}, and the \isi{reflexive pronoun} appears in a kind of right-dislocated position, such that it is syntactically not part of the subject constituent. 
%
\begin{exe}
	\ex	\label{ex:‎Everybody is with a bottle in their hands. (lit. There is one bottle each in everybody's (hand), and they also.)}
	\gll	harkil-li-cːe-d	ca	ca	šuša=ra	χe-d	ca-b=ra\\
		every-\tsc{obl-in-npl}	one	one bottle=\tsc{add}	exist.\tsc{down-npl}	\tsc{refl-hpl=add}	\\
	\glt	\sqt{‎Everybody is with a bottle in their hands.} (lit. \sqt{There is one bottle each in everybody's (hand), and they also.})
\end{exe}
%

It is also possible to elicit examples in which the semantically coordinated items function as a coordinated \isi{noun phrase}. The coordinated \isi{noun phrase} controls plural agreement on \isi{intransitive verbs} if it functions as subject \refex{ex:‎He and (his) sister were standing there}. However, masculine singular would also be possible in this type of construction as \refex{ex:He ran away with his dog} shows. In example \refex{ex:Musa sang a song together with his sister} the two coordinated items are marked for the \isi{ergative} case. Again the coordinated \isi{noun phrase} rather looks like an adjunct in the clause in which \tit{Musal} is the agentive argument.
%
\begin{exe}
	\ex	\label{ex:‎He and (his) sister were standing there}
	\gll	rucːi=ra	ca-w=ra	ka-b-icː-ur	ca-b\\
		sister\tsc{=add}	\tsc{refl-m=add}	\tsc{down}\tsc{-hpl-}stand\tsc{.pfv-pret}	\tsc{cop-hpl}\\
	\glt	\sqt{‎He and (his) sister were standing there.} (E)

	\ex	\label{ex:Musa sang a song together with his sister}
	\gll	rucːi-li=ra	cin-ni=ra	Musa-l	dalaj	b-uč'-un	ca-b\\
		sister\tsc{-erg=add}	\tsc{refl.sg-erg=add}	Musa\tsc{-erg}	song	\tsc{n-}sing\tsc{.ipfv-pret}	\tsc{cop-n}\\
	\glt	\sqt{Musa sang a song together with his sister.}
\end{exe}

The construction has probably evolved from the emphatic use of \isi{reflexive pronouns} (\refsec{ssec:Emphatic reflexive use}) in combination with the \isi{additive} meaning of the \isi{enclitic} \tit{=ra}. Thus, in \refex{ex:There one woman grabbed his backpack, took it, and went away with it} the two parts occur in independent clauses that follow each other as arguments of their respective verbs. The \isi{reflexive pronoun} in the second clause doubles an omitted subject argument and conveys the emphatic meaning \sqt{she herself}. The whole construction can be rephrased as \sqt{both the backpack and she herself} and has a \isi{comitative} reading (\sqt{she went away with the backpack}) that has to be inferred from the structure.
%
\begin{exe}
	\ex	\label{ex:There one woman grabbed his backpack, took it, and went away with it}
	\gll	ca	xːunul-li	χːap	b-arq'-ib-le	hil-i-la	wešimišuk'=ra b-erqː-ib	ca-b,	ca-r=ra	ag-ur	ca-r\\
		one	woman\tsc{-erg}	grab	\tsc{n-}do\tsc{.pfv-pret-cvb}	that\tsc{-obl-gen}	backpack\tsc{=add} \tsc{n-}carry\tsc{.pfv-pret}	\tsc{cop-n}	\tsc{refl-f=add}	go\tsc{.pfv-pret}	\tsc{cop-f}\\
	\glt	\sqt{There one woman grabbed his backpack, took it, and went away with it.} (lit. \sqt{She also took [the backpack], and [she herself also] went away.})
\end{exe}


%%%%%%%%%%%%%%%%%%%%%%%%%%%%%%%%%%%%%%%%%%%%%%%%%%%%%%%%%%%%%%%%%%%%%%%%%%%%%%%%

\section{Possession}
\label{sec:Possession}

Possession is either expressed by cases or by means of the \tit{b-ah} construction. In the first case the possessor is marked with the \isi{genitive} case and most commonly preceding the possessed item \refex{ex:I had mother and father}, but other positions are available, too \refex{ex:‎Is your sack there, Ashura} (see \refsec{sec:Noun phrases} on noun phrases and \refsec{sec:Constituent order at the phrase level} on the \isi{constituent order} of phrases). There is no grammaticalized distinction between alienable and inalienable \isi{possession}. Clauses expressing \isi{possession} are \isi{copula} clauses (\refsec{sec:copulaclauses}) containing locational copulas (\refsec{sec:Locational copulae}).
%
\begin{exe}
	\ex	\label{ex:I had mother and father}
	\gll	di-la	atːa	aba	le-b=de\\
		\tsc{1sg-gen}	father	mother	exist\tsc{-hpl=pst}\\
	\glt	\sqt{I had mother and father.}

	\ex	\label{ex:‎Is your sack there, Ashura}
	\gll	qːap	χe-b=uw,	wa	Ašura,		ala?\\
		sack	exist.\tsc{down-n=q}	hey	Ashura		\tsc{2sg.gen}\\
	\glt	\sqt{‎Is your sack there, Ashura?}
\end{exe}

If the respective item is not permanently possessed but only temporarily in the custody of the possessor, the \tsc{in}-essive case is used \refex{ex:‎‎He has a bottle with a drink} (\refsec{sssec:in-lative -cːe, in-essive -cːe-b, and in-ablative -cːe-r}).
%
\begin{exe}
	\ex	\label{ex:‎‎He has a bottle with a drink}
	\gll	hež-i-cːe-b	šuša	ca-b	deč-la\\
		this\tsc{-obl-in-n}	bottle	\tsc{cop-n}	drinking\tsc{-gen}\\
	\glt	\sqt{‎‎He has a bottle with a drink.}
\end{exe}

The second way of expressing \isi{possession} is the \tit{b-ah} construction. The noun \tit{b-ah} means \sqt{owner} (plural \tit{b-ahin-te}, also translates as \sqt{parents}). It is one of the few \isi{nouns} that have a \isi{gender} prefix expressing the \isi{gender} of the owner. The possessed item appears in the \isi{genitive} with \tit{b-ah} as the head noun of the \isi{genitive} phrase. If the possessor is overt it occurs after \tit{b-ah}. Both noun phrases together form an appositive phrase (\refsec{sec:Noun phrases}). The possessed items in this construction are normally inanimate objects \xxref{ex:There was one with a tail}{ex:‎On the upper side you smear water, egg, whatever you have}. Often they refer to clothes \refex{ex:‎All three had hats} or body parts \refex{ex:There was one with a tail}, \refex{ex:the devil with one eye} that are used to characterize and identify the owner. From this noun the adjective-like item \tit{wahwalla} \sqt{own, everybody's own} with frozen \isi{gender} agreement has been derived.
%
\begin{exe}
	\ex	\label{ex:There was one with a tail}
	\gll	č'imi-la	b-ah	b-irχʷ-i\\
		tail\tsc{-gen}	\tsc{n-}owner	\tsc{n-}be\tsc{.ipfv-hab.pst.3}\\
	\glt	\sqt{There was one with a tail.}

	\ex	\label{ex:the devil with one eye}
	\gll	ca	ul-la	b-ah	šajt'an\\
		one	eye\tsc{-gen}	\tsc{n-}owner	devil\\
	\glt	\sqt{the devil with one eye}

	\ex	\label{ex:‎All three had hats}
	\gll	šljaˁp'a-la	b-ahin-te=de	ʡaˁbal-ra\\
		hat\tsc{-gen}	\tsc{hpl-}owner\tsc{-pl=pst}	three\tsc{-num}\\
	\glt	\sqt{‎All three had hats.}

	\ex	\label{ex:‎On the upper side you smear water, egg, whatever you have}
	\gll	qar=či-b	sa-b-ik-u	[\ldots] hin-ni,	duqu-l, le-b-il	ʁuna	w-ah-la	cik'al-li\\
		up=on\tsc{-n}	\tsc{hither-n-}smear\tsc{.ipfv-prs} {}	water\tsc{-erg}	egg\tsc{-erg}	exist\tsc{-n-ref}	\tsc{eq}	\tsc{m-}owner\tsc{-gen}	thing\tsc{-erg}\\
	\glt	\sqt{‎On the upper side you smear [\ldots] water, egg, whatever you have.}
\end{exe}
