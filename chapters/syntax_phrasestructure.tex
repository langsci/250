\chapter{Noun phrases and postpositional phrases}
\label{cpt:Phrase structure}

This chapter addresses the properties of noun phrases (\refsec{sec:Noun phrases}) and postpositional phrases (\refsec{sec:Postpositional phrases}) including their \isi{constituent order}. Nominal modifiers that occur outside the \isi{noun phrase} (i.e. so-called ``\isi{floating modifiers}'') are only briefly discussed (\refsec{ssec:Floating modifiers}). For a detailed treatment of \isi{floating modifiers} see \refsec{ssec:Extraposed genitives} and \refsec{ssec:Extraposed \isi{adjectives}, postpositional phrases, and relative clauses}.

%%%%%%%%%%%%%%%%%%%%%%%%%%%%%%%%%%%%%%%%%%%%%%%%%%%%%%%%%%%%%%%%%%%%%%%%%%%%%%%%

\section{Noun phrases}
\label{sec:Noun phrases}


% --------------------------------------------------------------------------------------------------------------------------------------------------------------------------------------------------------------------- %

\subsection{Introduction}
\label{ssec:IntroductionNP}

The \isi{noun phrase} (NP) with an overt head noun minimally consists of a nominal head that can be optionally modified. Nominals that occur as heads of NPs are all sorts of pronouns, common \isi{nouns}, personal names, or nominalized items. Noun phrases can be coordinated (\refsec{sec:Coordination of noun phrases and other phrases}), and it is possible to have noun phrases with nominalized modifiers instead of head \isi{nouns} (\refsec{ssec:Headless noun phrases and nominalizations}). Noun phrases are head-final and thus modifiers precede the head. Usually, \isi{nouns} are most prone to be modified, but occasionally other nominals can also take modifiers. Noun phrases admit the following types of modifiers:
%
\begin{itemize}
	\item	lexical modifiers such as demonstrative and possessive pronouns, \isi{adjectives}, numerals and other quantifiers, and \isi{nouns} (appositive \isi{nouns}, \isi{nouns} marked by the \isi{genitive})
	\item	spatial modifiers in the essive case (e.g. \isi{nouns} with \isi{spatial case} marking or postpositional phrases)
	\item	\isi{relative clauses} and purpose clauses
\end{itemize}
%

Noun phrases occur in core argument and adjunct position (e.g. as instruments or temporal adjuncts). They can also be used as predicates in \isi{copula} clauses (\refsec{sec:copulaclauses}) and as complements in postpositional phrases (\refsec{sec:Postpositional phrases}).

Within the \isi{noun phrase}, there is \isi{gender} and \isi{number} agreement. Targets for \isi{gender}\slash \isi{number} agreement are a \isi{number} of vowel-initial \isi{adjectives} \refex{ex:a/one long branch@1} and adjectival quantifiers that have agreement affixes \refex{ex:Three boys came and gathered all the pears@6}, \refex{ex:‎They had many young boys who had come to help}, any items bearing essive cases \refex{ex:another little girl on a bike@4b}, \refex{ex:the place in front of his house@7a}, and \isi{participles} of verbs with \isi{gender} prefixes \refex{ex:that dead woman, may she rest in peace@9c}. Number agreement without \isi{gender} agreement is found with \isi{demonstrative pronouns} \refex{ex:and those two neighbors@8b}, and with modifiers that have the \isi{cross-categorical suffix} -\textit{ce} (singular) \refex{ex:It snowed a lot@4a} vs. -\textit{te} (plural) \refex{three old cowsNP} (\refsec{ssec:The -ce / -te attributive}). There is no case agreement between modifiers and the head noun within the \isi{noun phrase}, and case suffixes can only occur on the head noun.
%
\begin{exe}
	\ex	\label{ex:a/one long branch@1}
	\gll	ca	b-uqen	q'aˁli\\
		one	\tsc{n-}long	branch\\
	\glt	\sqt{a/one long branch}
\end{exe}


% --------------------------------------------------------------------------------------------------------------------------------------------------------------------------------------------------------------------- %

\subsection{Lexical, phrasal, and clausal modifiers in noun phrases}
\label{ssec:Lexical, phrasal, and clausal modifiers in noun phrases}

Sanzhi does not have a special class of articles. Instead, \isi{demonstrative pronouns} (\refsec{sec:Demonstrative pronouns}) and the numeral \tit{ca} \sqt{one} can be used in the function of definite and indefinite articles respectively, but often their interpretation is ambiguous between definite article and \isi{demonstrative pronoun}, or indefinite article and numeral \refex{ex:There with them was also a / one young boy@2}.
%
\begin{exe}
	\ex	\label{ex:There with them was also a / one young boy@2}
	\gll	hel-tː-a-lla	hel-tːu-w	le-w=de	ca	žahil	durħuˁ\\
		that\tsc{-pl-obl-gen}	that\tsc{-loc-m}	exist\tsc{-m=pst}	one	young	boy\\
	\glt	\sqt{There with them was also a\slash one young boy.} 
\end{exe}

Nouns frequently occur without the numeral \tit{ca} or a \isi{demonstrative pronoun} and receive an indefinite or definite interpretation from the context. Personal names can take \isi{demonstrative pronouns} when they occur as topical noun phrases \refex{ex:Whose was that Razhab@3}, but normally they occur without demonstratives.
%
\begin{exe}
	\ex	\label{ex:Whose was that Razhab@3}
	\gll	hila=de	il	Ražab?\\
		whose\tsc{=pst}	that	Razhab\\
	\glt	\sqt{Whose was that Razhab?} (i.e. from which family)
\end{exe}

There are no special possessive pronouns. Personal pronouns (first and second person), \isi{demonstrative pronouns} (third person) or \isi{reflexive pronouns} (third person) marked by the \isi{genitive} are used instead \refex{ex:the place in front of his house@7a}, \refex{ex:drinks like our alcoholic homebrew}. Most \isi{adjectives} distinguish between a short bare form and a long form with the suffix \tit{-ce} (plural \tit{-te}) (\refsec{sec:adjmorphclasses}). The use of the suffix is obligatory for \isi{adjectives} in predicative function and for attributive \isi{adjectives} that do not occur in their canonical prenominal position (see \refsec{ssec:The structure and order of constituents within the noun phrase} below). Adjectives used as attributes to \isi{nouns} can occur with \refex{ex:It snowed a lot@4a} or without the suffix \refex{ex:another little girl on a bike@4b}, the omission of the suffix being far more frequent than its presence.
%
\begin{exe}

		\ex	\label{ex:It snowed a lot@4a}
		\gll	χːula-ce	duˁħi	b-irq'-iri\\
			big\tsc{-dd.sg}	snow	\tsc{n-}do\tsc{.ipfv-hab.pst}\\
		\glt	\sqt{It snowed a lot.}
	
		\ex	\label{ex:another little girl on a bike@4b}
		\gll	cara	welisipjed-li-cːe-r	nik'a	rursːi\\
			other	bike\tsc{-obl-in-f}	small	girl\\
		\glt	\sqt{another little girl on a bike}

\end{exe}

Nouns modified by numerals are not marked for plural (as are \isi{nouns} modified by the interrogative word \tit{čum} \sqt{how many}), although they trigger plural agreement on \isi{demonstrative pronouns} \refex{ex:and those two neighbors@8b}, \isi{adjectives} \refex{three old cowsNP}, and also within the clause, i.e. on verbs, postpositions or adverbs \refex{ex:There were two women of their troops called Kadiashra and Bataj@5}. This means that not only semantically, but also syntactically, the \isi{noun phrase} is plural. Modifying \isi{adjectives} in noun phrases can occur in the stem form or with the \isi{cross-categorical suffix}, which has a singular form -\textit{ce} and a plural form -\textit{te} (\refsec{ssec:The -ce / -te attributive}). In noun phrases with numerals as modifiers, the plural form must be used when the noun has plural reference \refex{three old cowsNP}
%
\begin{exe}
	\ex	\label{ex:There were two women of their troops called Kadiashra and Bataj@5}
	\gll	il-tːu-b	[Q'adiʡaˁšra=ra	b-ik'-ul]	[Bat'aj=ra	b-ik'-ul]		k'ʷel	xːunul	b-irχ-i	il-tːa-lla	atrjad-la\\
		that\tsc{-loc-hpl}	Kadiashra\tsc{=add}	\tsc{hpl-}say\tsc{.ipfv-icvb}	Bataj\tsc{=add}	\tsc{hpl-}say\tsc{.ipfv-icvb}		two	woman	\tsc{hpl-}be\tsc{.ipfv-hab.pst}	that\tsc{-obl.pl-gen}	troop\tsc{-gen}\\
	\glt	\sqt{There were two women of their troops called Kadiashra and Bataj.}
	
	\ex	\label{three old cowsNP}
	\gll ʡaˁbal	d-uqna(-te)	q'ʷal\\
	three	\tsc{npl}-old-(\tsc{dd.pl}) 	cow\\
	\glt	\sqt{three old cows} (E)
\end{exe}


Younger speakers occasionally use the plural suffix on the noun in noun phrases with numerals as it is done in Russian \refex{ex:Three boys came and gathered all the pears@6}. In Sanzhi noun phrases that contain quantifiers such as \tit{b-aqil} \sqt{much, many} \refex{ex:‎They had many young boys who had come to help}, the noun has also to be marked for plural.
%
\begin{exe}
	\ex	\label{ex:Three boys came and gathered all the pears@6}
	\gll	ʡaˁbal	durħ-ne	sa-b-eʁ-ib-le,	quˁr-be	li<d>il		d-alc'-un\\
		three	boy\tsc{-pl}	\tsc{hither-hpl-}go\tsc{.pfv-pret-cvb}	pear\tsc{-pl}	all\tsc{<npl>}	\tsc{npl-}gather\tsc{.pfv-pret}\\
	\glt	\sqt{Three boys came and gathered all the pears.}
\end{exe}



Nominal modifiers in NPs can be appositions, \isi{nouns} bearing the suffixes \tit{-il} or \tit{-ce}, and \isi{nouns} marked for \isi{spatial cases} \refex{ex:the place in front of his house@7a} or for the \isi{genitive} case \refex{ex:There were two women of their troops called Kadiashra and Bataj@5}. If plural \isi{nouns} bear the \isi{genitive} they can have a non-specific interpretation, not referring to a specific possessor but restricting the meaning of the head noun to a certain type \refex{ex:a big beautiful apple tree@7b}. 
%
\begin{exe}
		\ex	\label{ex:the place in front of his house@7a}
		\gll	[cin-na	qal-li-sa-b]	musːa\\
			\tsc{refl.sg-gen}	house\tsc{-obl-ante-n}	place\\
		\glt	\sqt{the place in front of his house}
	
		\ex	\label{ex:a big beautiful apple tree@7b}
		\gll	χːula	qːuʁa	hinc-b-a-lla	kːalkːi\\
			big 	beautiful	apple\tsc{-pl-obl-gen}	tree\\
		\glt	\sqt{a big beautiful apple tree}
\end{exe}

Appositions consist of two (or more) \isi{nouns} with the same referents immediately following each other. As indicated by their modifiers and case marking, appositional phrases behave like a single \isi{noun phrase}. They most frequently consist of a proper name and a kinship term, namely \tit{acːi} \sqt{uncle} \refex{ex:Uncle Shamkhal guided me@8a} or \tit{azi} \sqt{aunt} or of \isi{nouns} denoting different types of roles such as social roles (e.g. \tit{zunra} \sqt{neighbor}, \tit{saldat} \sqt{soldier}, \tit{tuχtur} \sqt{doctor}, \tit{busurman} \sqt{Muslim}, \tit{jatim} \sqt{orphan}) \refex{ex:and those two neighbors@8b} or \isi{gender} roles (\tit{xːunul} \sqt{woman}, \tit{murgul} \sqt{man}). The role-denoting \isi{nouns} modify more general terms such as \tit{admi} \sqt{person, man}, \tit{insan} \sqt{person} \refex{ex:and those two neighbors@8b}, and some other \isi{nouns}. Appositions not involving proper names resemble compounds \refex{ex:and those two neighbors@8b} since the meaning of the second \isi{nouns} is restricted through the meaning of the preceding noun.
%
\begin{exe}
		\ex	\label{ex:Uncle Shamkhal guided me@8a}
		\gll	šːamχal	acːi-l	r-ik-a-di\\
			Shamxal	uncle\tsc{-erg}	\tsc{f-}lead\tsc{.ipfv-hab.pst-1}\\
		\glt	\sqt{Uncle Shamkhal guided me (fem.).}
	
		\ex	\label{ex:and those two neighbors@8b}
		\gll	hel-tːi	k'ʷel=ra	zunra	admi=ra\\
			that\tsc{-pl}	two\tsc{=add}	neighbor	person\tsc{=add}\\
		\glt	\sqt{and those two neighbors}
	
\end{exe}

Furthermore, the nominals cannot be separated, their order is rather fixed, and only the second nominal is marked for case \refex{ex:I say to Doctor Mahammad}
%
\begin{exe}
	\ex	\label{ex:I say to Doctor Mahammad}
	\gll	du	tuχtur	Maˁħaˁmmad-li-cːe	Ø-ik'-ul=da ...\\
		\tsc{1sg}	doctor		Mahammad\tsc{-obl-in}	\tsc{m-}say\tsc{.ipfv-icvb=1}\\
	\glt	\sqt{I say to Doctor Mahammad ...}
\end{exe}

Other nominals occurring in appositive phrases are \isi{reflexive pronouns} that function as emphatic reflexives (\refsec{ssec:Emphatic reflexive use}) or pronouns with quantifiers, e.g. \tit{nušːa lidil} \sqt{we all}.

The \isi{noun phrase} can contain an equative expression that either contains the adjective \tit{miši}, which governs the \isi{dative} \refex{ex:meat similar to human (flesh)}, or the \isi{particle} \tit{ʁuna} \refex{ex:a road like through hapraqu@9a}, \refex{ex:drinks like our alcoholic homebrew} that changes to \tit{ʁunab} when the suffix \tit{-ce} is added.
%
\begin{exe}

		\ex	\label{ex:meat similar to human (flesh)}
		\gll	[admi-li-j	miši]	dig\\
			person\tsc{-obl-dat}	similar	meat\\
		\glt	\sqt{meat similar to human (flesh)}
		
		
		\ex	\label{ex:a road like through hapraqu@9a}
		\gll	ħaˁpra.qu-m-a-ja-r	či-b-a	ʁuna	xːun\\
			Hapra.field\tsc{-pl-obl-loc-abl}	on\tsc{-n-dir}	\tsc{eq}	road\\
		\glt	\sqt{a road like through Hapra-field} (place name)

	
		\ex	\label{ex:drinks like our alcoholic homebrew}
		\gll	nišːa-la	ʁuna	mukːatːa-lla	aruš-la	napitka-be\\
			\tsc{1pl-gen}	\tsc{eq}	alcoholic.drink\tsc{-gen}	home.brew\tsc{-gen}	drink\tsc{-pl} \\
		\glt	\sqt{drinks like our alcoholic homebrew}

\end{exe}

Phrasal and clausal modifiers of noun phrases are postpositional phrases \refex{ex:a road like through hapraqu@9a}, \refex{ex:the bag in the hands of the woman@9b}, \refex{ex:the golden coin between the rocks}, \isi{relative clauses} \refex{ex:that dead woman, may she rest in peace@9c}, \refex{ex:There is our place, called the midday summit, the praying summit@10a} (see also \refcpt{cpt:Relative clauses}), purpose clauses \refex{ex:time to think} and other clauses \refex{ex:that dead woman, may she rest in peace@9c}. Although not obligatory, the suffix \tit{-il}, which is used to form nominal modifiers from various parts of speech (\refsec{ssec:The -il attributive}), has been added to the postpositional phrase in \refex{ex:the bag in the hands of the woman@9b}. The suffix can be omitted as example \refex{ex:a road like through hapraqu@9a} shows, but then the postpositional phrase can either function as a modifier of the noun or as a clausal modifier. Sentence \refex{ex:that dead woman, may she rest in peace@9c} illustrates the use of a commemoration formula that is headed by a verb in the unmarked \isi{optative}. The unmarked \isi{optative} can be nominalized and take further case markers. Therefore, the \isi{optative} clause can be interpreted as a clausal modifier to the noun \tit{xːunul} \sqt{woman}. But it can also be interpreted as a parenthesis that is not syntactically related to the noun. The \isi{optative} clause is followed by a short \isi{relative clause} consisting only of a \isi{participle}.
%
\begin{exe}

		\ex	\label{ex:the bag in the hands of the woman@9b}
		\gll	[xːunul-li-cːe-b	kʷi-b-il]	sumk'a\\
			woman\tsc{-obl-in-n}	in.the.hands\tsc{-n-ref}	bag\\
		\glt	\sqt{the bag in the hands of the woman}
	
		\ex	\label{ex:the golden coin between the rocks}
		\gll	[qič'-me	urkːa-b]	ašrapi\\
			rock\tsc{-pl}	between\tsc{-n}	golden.coin\\
		\glt	\sqt{the golden coin between the rocks}
	
		\ex	\label{ex:time to think}
		\gll	[pikri	Ø-ik'ʷ-ij]	zamana\\
			thought	\tsc{m-}say\tsc{.ipfv-inf}	time\\
		\glt	\sqt{time to think}
	
		\ex	\label{ex:that dead woman, may she rest in peace@9c}
		\gll	it	[ʡaˁpa	b-arq'	cin-na]	[r-ebč'-ib-il]	xːunul\\
			that	commemoration	\tsc{n-}do\tsc{.pfv}	\tsc{refl.sg-gen}	\tsc{f-}die\tsc{.pfv-pret-ref}	woman\\
		\glt	\sqt{that dead woman, may she rest in peace}

\end{exe}


% --------------------------------------------------------------------------------------------------------------------------------------------------------------------------------------------------------------------- %

\subsection{The structure and order of constituents within the noun phrase}
\label{ssec:The structure and order of constituents within the noun phrase}

Noun phrases can be complex consisting of several modifiers, but in natural texts three or more modifiers are not very common. Because the modifiers themselves can be complex, the actual \isi{number} of words in noun phrases might easily reach five or more. The usual order of modifiers is displayed in \refex{ex:Modifiers and their order in the NP}, which shows that the noun occupies the right-most position in the \isi{noun phrase}.
%
\begin{exe}
	\ex	 demonstrative\slash \isi{genitive}	\newline\hspace*{1.5em}	numeral / \isi{quantifier}	\newline\hspace*{3.0em}	phrase or clause	\newline\hspace*{4.5em}	adjective	\newline\hspace*{6.0em}	demonstrative / \isi{genitive} / \isi{quantifier}	\newline\hspace*{7.5em}	appositive noun	\newline\hspace*{9.0em}	\textbf{head} 	\label{ex:Modifiers and their order in the NP}
\end{exe}

\citet[654]{Testelec1998a} has proposed for other East Caucasian, particularly Avar-Andic and Tsezic languages, that the order of modifiers in the \isi{noun phrase} reflects \dqt{the degree of their contribution to the identification of the NP's referent.} If this generalization is taken to express a tendency, rather than a strict rule, it can be applied to the Sanzhi \isi{noun phrase} as well. For example, \isi{genitive} modifiers and \isi{demonstrative pronouns} demonstrate this tendency since the former, typically closer to the head noun than the latter (see \refex{ex:your husband's head@10b} above), make a larger contribution to identification of referents, although the reverse order is possible \refex{ex:their this story@11a}.

The modifiers can be divided into two groups: (i) quantifiers, demonstratives, and genitives, which specify the quantity, definiteness, and referentiality of the \isi{noun phrase} and thus anchor it in the discourse, and (ii) \isi{adjectives}, nominals, phrases, or clauses, which denote qualities and provide further information about the properties of the referent. The two groups are not only distinguished by their semantics, but also by their position within the \isi{noun phrase}. Members of the first group, i.e. quantifiers, demonstratives, and genitives, can occur in two different positions as \refex{ex:Modifiers and their order in the NP} shows: either in phrase-initial position or (almost) immediately before the head noun. Furthermore, they can float off from the head noun and occur outside the \isi{noun phrase}. This will be discussed in detail below.

The examples in \xxref{ex:There is our place, called the midday summit, the praying summit@10a}{ex:all people who had come for a visit@11b} illustrate complex noun phrases. Relative clauses are given in square brackets. More examples in this chapter illustrate other constellations of nominal modifiers in complex noun phrases: numeral + adjective \refex{ex:a/one long branch@1}, demonstrative + numeral + noun \refex{ex:and those two neighbors@8b}, adjective + adjective + \isi{genitive} \refex{ex:a big beautiful apple tree@7b}, and \isi{relative clause} + \isi{relative clause} + numeral \refex{ex:There were two women of their troops called Kadiashra and Bataj@5}.
%
\begin{exe}
		\ex	\isi{relative clause} + \isi{relative clause} + \isi{genitive} pronoun\\	\label{ex:There is our place, called the midday summit, the praying summit@10a}
		\gll	[debʁalla	b-irq'-an]	[arilla	muza	b-ik'-ul]	nišːa-lla	musːa	k'e-b\\
			prayer	\tsc{n-}do\tsc{.ipfv-ptcp}	during.day	summit	\tsc{hpl-}say\tsc{.ipfv-icvb}	\tsc{1pl-gen}	place	exist.\tsc{up-n}\\
		\glt	\sqt{There is our place, called the midday summit, the praying summit.}

		\ex	demonstrative + \isi{genitive} pronoun + \isi{genitive} noun\\	\label{ex:your husband's head@10b}
		\gll	iž	ala	sub-la	bek'\\
			this	\tsc{2sg.gen}	husband\tsc{-gen}	head\\
		\glt	\sqt{your husband's head}

		\ex	demonstrative + \isi{relative clause} + \isi{genitive} pronoun\\	\label{ex:that our fellow villager called Osban}
		\gll	hel	[ʡuˁsban	b-ik'ʷ-an]	nišːa-lla	šːan\\
			that	Osban	\tsc{hpl-}say\tsc{.ipfv-ptcp}	\tsc{1pl-gen}	fellow.villager\\
		\glt	\sqt{that fellow villager of ours called Osban}


		\ex	demonstrative + adjective\\				\label{ex:with these other boys}
		\gll	heštːi	cara	durħ-n-a-cːella\\
			these	other	boy\tsc{-pl-obl-comit}\\
		\glt	\sqt{with these other boys}
		
				\ex	\isi{relative clause} + \isi{quantifier}\\		\label{ex:all people who had come for a visit@11b}
		\gll	[šːatːir	sa-b-ač'-ib-te]	li<b>il=ra	χalq'\\
			visit	\tsc{hither-hpl-}come\tsc{.pfv-pret-dd.pl} 	all\tsc{<hpl>=add}	people\\
		\glt	\sqt{all people who had come for a visit}

\end{exe}

Short \isi{adjectives} are subject to a positional restriction: they can be separated from the head noun only by other \isi{adjectives} (short or long ones, which bear the attributive suffix -\textit{ce}) \refex{ex:‎They had many young boys who had come to help} or by appositive \isi{nouns} \refex{ex:the sick doctor Mahammad}. All other modifiers need to precede short \isi{adjectives} \refex{ex:‎They had many young boys who had come to help}, every other order being ungrammatical \refex{ex:my new car ungrammatical}.
%
\begin{exe}
	\ex	\label{ex:the sick doctor Mahammad}
	\gll	ʡaˁrkːa	tuχtur	Maˁħaˁmmad\\
		sick doctor	Mahammad\\
	\glt	\sqt{the sick doctor Mahammad}

	\ex	\label{ex:my new car ungrammatical}
	\gll	* jangi di-la mašin\\
		{} new	\tsc{1sg-gen}	car\\
	\glt	(Intended meaning: \sqt{my new car}) (E)
\end{exe}


There are two positions in which genitives, especially \isi{genitive} pronouns, occur \refex{ex:Modifiers and their order in the NP}. They are either placed in phrase-initial position \refex{ex:their this story@11a} or, more frequently, directly preceding the head \refex{ex:There is our place, called the midday summit, the praying summit@10a}. As mentioned above, the common order of \isi{demonstrative pronouns} and \isi{genitive} pronouns is for the demonstrative to precede the \isi{genitive} \refex{ex:your husband's head@10b}, but the reverse order is also attested \refex{ex:their this story@11a}.
%
\begin{exe}
		\ex	\isi{genitive} pronoun + demonstrative\\	\label{ex:their this story@11a}
		\gll	hež-tː-a-la	hej	χabar\\
			this\tsc{-pl-obl-gen}	this	story\\
		\glt	\sqt{this story of theirs}
\end{exe}

There can be scope differences associated with certain constituent orders. For instance, the \isi{genitive} pronoun in \refex{ex:‎‎‎my sister who sells bread} can scope over the entire \isi{noun phrase} or it can be restricted to the immediately following noun. Similarly, the interpretations of \refex{ex:‎‎‎my three books ‎‎‎three books of mine@A} and \refex{ex:‎‎‎my three books ‎‎‎three books of mine@B} differ slightly.
%
\begin{exe}
	\ex	\label{ex:‎‎‎my sister who sells bread}
	\gll	di-la	t'ult'-e	d-irc-an	rucːi\\
		\tsc{1sg-gen}	bread\tsc{-pl}	\tsc{npl-}sell\tsc{.ipfv-ptcp}	sister\\
	\glt	\sqt{‎‎‎my sister who sells bread} OR \sqt{the sister who sells my bread} (E)

	\ex	\label{ex:ex:‎‎‎my three books ‎‎‎three books of mine}
	\begin{xlist}
		\ex	\label{ex:‎‎‎my three books ‎‎‎three books of mine@A}
		\gll	di-la	ʡaˁbal	kiniga\\
			\tsc{1sg-gen}	three	book\\
		\glt	\sqt{‎‎‎my three books} (E)
	
		\ex	\label{ex:‎‎‎my three books ‎‎‎three books of mine@B}
		\gll	ʡaˁbal	di-la	kiniga\\
			three	\tsc{1sg-gen}	book\\
		\glt	\sqt{‎‎‎three books of mine} (E)
	\end{xlist}
\end{exe}

If the \isi{genitive} is a genuine possessor rather than a \isi{genitive} which denotes the material and can be retrieved from the context, the head noun is frequently omitted. The second \isi{genitive} in \refex{ex:‎‎The villagers have milled their wheat; the people from other villages have milled (it)} lacks an overt head. Headless genitive-marked nominals can also, just like other modifiers in noun phrases, be nominalized and take case suffixes (\refsec{ssec:Headless noun phrases and nominalizations}).
%
\begin{exe}
	\ex	\label{ex:‎‎The villagers have milled their wheat; the people from other villages have milled (it)}
	\gll	šːan-t-a-lla	deq'a	d-elq'-un	ca<d>i;	tːura	šːan-t-a-lla	d-elq'-un	ca<d>i\\
		fellow.villager\tsc{-pl-obl-gen}	grain	\tsc{npl-}grind\tsc{.pfv-pret}	\tsc{cop<npl>}	outside	fellow.villager\tsc{-pl-obl-gen}	\tsc{npl-}grind\tsc{.pfv-pret}	\tsc{cop<npl>}\\
	\glt	\sqt{‎‎The villagers' grain has been ground; (The grain) of the people from other villages has been ground.}
\end{exe}

In content interrogative \isi{questions} with the meaning \sqt{which other X}, the order is adjective-first (\tit{cara} \sqt{other}), followed by the interrogative pronoun \tit{ce} \sqt{what}, which belongs to the modifiers of the first group \refex{ex:What other food of ours exists@12}:
%
\begin{exe}
	\ex	\label{ex:What other food of ours exists@12}
	\gll	cara	ce	χurejg	d-irχ-u=ja	nišːa-lla?\\
		other	what	food	\tsc{npl}-be.\tsc{ipfv-prs=q}	\tsc{1pl-gen}\\
	\glt	\sqt{What other food of ours exists?}
\end{exe}

% --------------------------------------------------------------------------------------------------------------------------------------------------------------------------------------------------------------------- %

\subsection{Floating modifiers}
\label{ssec:Floating modifiers}
In principle, all modifiers except for \isi{demonstrative pronouns} can float off to positions outside the \isi{noun phrase} (i.e. they can be extraposed). We find \isi{genitive} pronouns \refex{ex:What other food of ours exists@12} and \isi{genitive} \isi{nouns} \refex{ex:The people's respect also finished} as well as quantifiers \refex{ex:and all the people from the other villages down there}, \isi{adjectives} \refex{ex:There is shampoo for children, expensive@19b} and \isi{relative clauses} \refex{ex:‎This also and this also is probably the man who hit the woman on the jaw2} outside the \isi{noun phrase}. However, the extent to which floating is possible and which positions in the clause are common or available for \isi{floating modifiers} depends on the type of modifier. The greatest freedom is enjoyed by floating genitives because they do not require any special marking when they are extraposed. Floating quantifiers require overt case marking, and \isi{floating modifiers} of the second group (nominals, \isi{adjectives}, \isi{relative clauses}, etc.) need special additional marking (in addition to case). 

Floating modifiers are semantic modifiers of \isi{nouns}, but do not occur within the corresponding \isi{noun phrase}; they are separated from the \isi{noun phrase} by other constituents. In the following, the morphosyntactic properties of different types of \isi{floating modifiers} will be discussed in more detail. The information-structural properties are mainly discussed in \refsec{ssec:Extraposed genitives} and \refsec{ssec:Extraposed \isi{adjectives}, postpositional phrases, and relative clauses}.

The most common modifier that occurs detached from the noun is the \isi{genitive}, and this has been noticed for other East Caucasian languages, the first detailed analysis of floating genitives being made by \citet{Creissels2013} for Akhvakh. Floating genitives often follow the noun and occur at the end of the clause after the verb \refex{ex:There were two women of their troops called Kadiashra and Bataj@5}, \refex{ex:The people's respect also finished}, but sometimes the reverse order is found in which case the \isi{genitive} precedes a clause-final noun \refex{ex:‎We Sanzhi people have a story}. In the latter example it seems that it is the head noun which has been extraposed while the two genitives simply remain in their canonical position. 

%
\begin{exe}
	\ex	\label{ex:The people's respect also finished}
	\gll	lamusː-e=ra	ha-d-erχː-ur	ca-d	χalq'-la\\
		respect\tsc{-pl=add}	\tsc{up-npl-}fulfill\tsc{.pfv-pret}	\tsc{cop-npl}	people\tsc{-gen}\\
	\glt	\sqt{The people's respect also finished. (i.e. people do not show respect any more.)}

	\ex	\label{ex:‎We Sanzhi people have a story}
	\gll	nišːa-lla	sunglan-t-a-lla	le-b	χabar\\
		\tsc{1pl-gen}	Sanzhi\tsc{-pl-obl-gen}	exist\tsc{-n}	story\\
	\glt	\sqt{‎We Sanzhi people have a story.}
\end{exe}


Example \refex{ex:‎They had many young boys who had come to help} shows a relatively complex \isi{noun phrase} that functions as the subject of the \isi{existential copula} clause. The \isi{genitive} pronoun and the \isi{quantifier} following the \isi{copula} are semantically associated with the \isi{noun phrase}, but have been dislocated to the right of the \isi{copula}. A possible explanation for this might be that the \isi{noun phrase} would be otherwise quite complex and difficult to interpret. More examples of floating genitives and a detailed discussion of their information-structural interpretation can be found in \refsec{ssec:Extraposed genitives}. 
%
\begin{exe}
	\ex	\label{ex:‎They had many young boys who had come to help}
	\gll	[hel	kumek-le	ha-b-ač'-ib	žahil]	durħ-ne	le-b=de hel-tː-a-lla	b-aqil\\
		that	help\tsc{-loc}	\tsc{up-hpl-}come\tsc{.pfv-pret}	young	boy\tsc{-pl}	exist\tsc{-hpl=pst}	that\tsc{-pl-obl-gen}	\tsc{hpl-}much\\
	\glt	\sqt{‎They had many young boys who had come to help.}
\end{exe}

Some corpus examples of floating quantifiers can also be found: \refex{ex:Three boys came and gathered all the pears@6}, \xxref{ex:‎They had many young boys who had come to help}{ex:and all the people from the other villages down there}. Example \refex{ex:‎‎‎Put one apple (there), do not put two (apples)} illustrates that postnominal modifiers can be interpreted contrastively, in particular in elicited, context-free sentences, but as \refex{ex:and all the people from the other villages down there} shows, a contrastive reading is not obligatory. 

\begin{exe}
		\ex	\label{ex:‎‎‎Put one apple (there), do not put two (apples)}
	\gll	hinci	ca	ka-b-ixː-a,	k'ʷel	ma-ka-d-iršː-it!\\
		apple	one	down\tsc{-n-}put\tsc{.pfv-imp}	two	\tsc{proh-}\tsc{down-npl-}put\tsc{.ipfv-proh.sg}\\
	\glt	\sqt{‎‎‎Put one apple (there), do not put two (apples)!} (E)
	
	\ex	\label{ex:and all the people from the other villages down there}
	\gll	cara-r	heχtːu	šːi-la	tːura-b-te	χalq'	li<b>il=ra\\
		other\tsc{-abl}	there.\tsc{down}	village\tsc{-gen}	outside\tsc{-hpl-dd.pl} 	people	all\tsc{<hpl>=add}\\
	\glt	\sqt{and all the people from the other villages down there}
	
\end{exe}

Quantifier floating will be illustrated through the use of the \isi{quantifier} \sqt{all}, which contains a \isi{gender}/\isi{number} infix agreeing with the noun it modifies. If the modified noun bears the \isi{absolutive}, the \isi{quantifier} can occur in various positions other than the canonical position before the noun \xxref{ex:Yesterday all girls went to school@B}{ex:Yesterday all girls went to school@D}. 


%
\begin{exe}
	\ex	\label{ex:Yesterday all girls went to school}
	\begin{xlist}
		\ex	\label{ex:Yesterday all girls went to school@A}
		\gll	sːa	li<b>il	rurs-be	ag-ur	uškul-le\\
			yesterday	all\tsc{<hpl>}	girl\tsc{-pl}	go\tsc{.pfv-pret}	school\tsc{-loc}\\
		\glt	\sqt{Yesterday all girls went to school.}

		\ex	\tit{sːa rursbe libil agur uškulle}	\label{ex:Yesterday all girls went to school@B}

		\ex	\tit{sːa rursbe agur libil uškulle}	\label{ex:Yesterday all girls went to school@C}

		\ex	\tit{sːa rursbe agur uškulle libil}	\label{ex:Yesterday all girls went to school@D}

	\end{xlist}
\end{exe}

If the modified noun occupies the A function and has non-\isi{absolutive} case marking, then the \isi{quantifier} can, when it bears the same case suffix as the noun it modifies, only occur in positions other than the canonical position preceding the verb. Otherwise the sentence becomes ungrammatical \refex{ex:All girls used to make carpets@14c}. The reason is that in a \isi{noun phrase}, case marking can only occur once, namely on the head noun. If the \isi{quantifier} appears directly before the noun, it is part of the \isi{noun phrase} and can not be case-marked.

%
\begin{exe}
	\ex	\label{ex:All girls used to make carpets@14}
	\begin{xlist}
		\ex	\label{ex:All girls used to make carpets@14a}
		\gll	li<b>il	rurs-b-a-l	t'ams-ne	d-irq'-i\\
			all\tsc{<hpl>}	girl\tsc{-pl-obl-erg}	carpet\tsc{-pl}	\tsc{npl-}do\tsc{.ipfv-hab.pst}\\
		\glt	\sqt{All girls used to make carpets.}
	
		\ex	\label{ex:All girls used to make carpets@14b}
		\gll	rurs-b-a-l	li<b>il-li	t'ams-ne	d-irq'-i\\
			girl\tsc{-pl-obl-erg}	all\tsc{<hpl>-erg}	carpet\tsc{-pl}	\tsc{npl-}do\tsc{.ipfv-hab.pst}\\
		\glt	\sqt{All girls used to make carpets.}
		
\ex 	\tit{* libilli	rursbal	t'amsne	dirq'i}	\label{ex:All girls used to make carpets@14c}
		
	\end{xlist}
\end{exe}

For other grammatical relations, the restrictions are more severe. Quantifiers of addressee arguments floating away from their canonical position are rarely acceptable even if they bear the same case marking as the noun they modify. Sentences such as \refex{ex:Mother told the stories to all women@15} are marginal. It is possible to make a short break before the \isi{quantifier}, which is then interpreted as right dislocated, and the translation is \sqt{Mother told the stories to the women, to all (of them).}
%
\begin{exe}
	\ex	\label{ex:Mother told the stories to all women@15}
	\gll	?? 	aba-l	χabur-te	xːun-r-a-cːe	li<b>il-li-cːe	d-urs-i\\	
		{}	mother\tsc{-erg}	story\tsc{-pl}	woman\tsc{-pl-obl-in}	all\tsc{<hpl>-obl-in}	\tsc{npl-}tell\tsc{.pfv-hab.pst}\\
	\glt	\sqt{Mother told the stories to all women.}
\end{exe}

Floating quantifiers are occasionally found in texts, but all examples have head \isi{nouns} in the \isi{absolutive} case \refex{ex:Three boys came and gathered all the pears@6}, \refex{ex:‎They had many young boys who had come to help}. 

In contrast to genitives and quantifiers, modifiers of the second group, that is, \isi{adjectives}, postpositional phrases, or \isi{relative clauses}, need special marking when they occur in a position outside the \isi{noun phrase}, either immediately following it or further extraposed to the right. There are only a few corpus examples of \isi{floating modifiers} in a position before and at the same time outside a \isi{noun phrase} \refex{ex:‎We Sanzhi people have a story} (see also \refsec{ssec:Extraposed \isi{adjectives}, postpositional phrases, and relative clauses}).

There are two markers: the suffix \tit{-ce} (plural -\textit{te}) (\refsec{ssec:The -ce / -te attributive}) and the suffix -\textit{il} (\refsec{ssec:The -il attributive}). These markers optionally occur on modifiers within the \isi{noun phrase} as \refex{ex:It snowed a lot@4a}, \refex{three old cowsNP}, \refex{ex:that dead woman, may she rest in peace@9c} and \refex{ex:all people who had come for a visit@11b} show, but for extraposed modifiers the use is obligatory. Both markers are \isi{cross-categorical suffixes} and are used for the formation of referential attributes, which morphosyntactically behave like nominals. Their use is similar, but -\textit{ce} and -\textit{il} can only express singular referents, whereas -\textit{te} requires plural referents. The suffix \tit{-ce} is used with \isi{adjectives} \refex{ex:There is shampoo for children, expensive@19b}, spatial modifiers in the essive case (=postpositional phrases), and occasionally \isi{relative clauses}; the suffix \tit{-il} is mainly used with \isi{relative clauses} \refex{ex:‎This also and this also is probably the man who hit the woman on the jaw2}. The same two markers are employed when no head noun is present in the clause and the items, which would otherwise be used as modifiers, are nominalized and take over the function of \isi{nouns} (\refsec{ssec:Headless noun phrases and nominalizations}). Furthermore, adjectival roots are obligatorily marked with -\textit{ce} when they are employed in predicative function (\refsec{sec:adjmorphclasses}).

%
\begin{exe}
		\ex	\label{ex:There is shampoo for children, expensive@19b}
		\gll	detski	šampun	le-b	durqa-ce\\
			children's	shampoo	exist\tsc{-n}	expensive\tsc{-dd.sg}\\
		\glt	\sqt{There is shampoo for children, expensive.}

	\ex	\label{ex:‎This also and this also is probably the man who hit the woman on the jaw2}
	\gll	iž=ra	het=ra,	het	ʡaˁχːuˁl	Ø-iχʷ-ij	[xːunul-la	qajqaj-li-cːe	b-aˁq-ib-il]\\
		this\tsc{=add}	that\tsc{=add}	that	guest	\tsc{m-}be\tsc{.pfv-inf}	woman\tsc{-gen}	jaw\tsc{-obl-in}	\tsc{n-}hit\tsc{.pfv-pret-ref}\\
	\glt	\sqt{‎This also and this also is probably the man who hit the woman on the jaw.}
\end{exe}

As was mentioned for postnominal quantifiers above, all corpus examples of \isi{floating modifiers} occur in clauses in which the noun to which they semantically belong appears in the \isi{absolutive} case. This means that it does not have overt case marking. If the noun is case-marked, it is not sufficient to add the special marking in form of the suffix -\textit{ce} or -\textit{il} to the modifier \refex{ex:He went to a good doctor, he did not go to a bad one ungrammatical minor2}, but the modifier must also take case marking identical to the case of the noun \refex{ex:He went to a doctor who is good; he did not go to a bad one minor2}.

\begin{exe}
	\ex	\label{ex:He went to a good doctor, he did not go to a bad one ungrammatical minor2}
	\gll	{*}	it	sa-jʁ-ib	tuχtur-ri-šːu ʡaˁħ-ce,	wahi-ce-lli-šːu	a-ag-ur\\
		{}	that	\tsc{hither}-come\tsc{.m.pfv-pret}	doctor\tsc{-obl-ad} good\tsc{-dd.sg}	bad\tsc{-dd.sg-obl-ad}	\tsc{neg-}go\tsc{.pfv-pret}\\
	\glt	(Intended meaning: \sqt{He went to a good doctor, he did not go to a bad one.}) (E)

	\ex	\label{ex:He went to a doctor who is good; he did not go to a bad one minor2}
	\gll	it	sa-jʁ-ib	tuχtur-ri-šːu ʡaˁħ-ce-lli-šːu,	wahi-ce-lli-šːu	a-ag-ur\\
		that	\tsc{hither}-come\tsc{.m.pfv-pret}	doctor\tsc{-obl-ad} good\tsc{-dd.sg-obl-ad}	bad\tsc{-dd.sg-obl-ad}	\tsc{neg-}go\tsc{.pfv-pret}\\
	\glt	\sqt{He went to a doctor who is good; he did not go to a bad one.} (E)
\end{exe}

Case marking of modifiers that occur within the \isi{noun phrase} is ungrammatical (see \refsec{sssec:Analyzing the suffix -ce and its cognates in other Dargwa languages} for an example).

\citet{Kazenin2009} analyzes \isi{floating modifiers} in the East Caucasian language Lak and in the unrelated languages Circassian and Nogai. He distinguishes between simple splits in which the nominal modifier is fronted and thus occurs in a position to the left of the \isi{noun phrase}, and inverted splits when the modifiers follow the \isi{noun phrase}. Inverted splits always require nominalization of the modifier independently of case marking; simple splits require nominalization whenever the head noun of the \isi{noun phrase} and the modifier are overtly case-marked, which concerns all cases except for the \isi{absolutive}. He concludes that nominalized modifiers do not form a single constituent with the \isi{noun phrase}, which is in line with the analysis adopted in this grammar.

Floating adjectival modifiers have restrictive semantics and imply the existence of another, contrasting referent. For instance, the shampoo in \refex{ex:There is shampoo for children, expensive@19b} is identified because the modifying adjective restricts the interpretation to expensive shampoos, which might at the same time be contrasted with cheap ones although this remains open within the context of the utterance. However, not every extraposed modifier is contrastive. A discussion of the pragmatic functions of \isi{floating modifiers} and more examples can be found in \refsec{ssec:Extraposed \isi{adjectives}, postpositional phrases, and relative clauses}.


% --------------------------------------------------------------------------------------------------------------------------------------------------------------------------------------------------------------------- %

\subsection{Nominalized modifiers used as head nouns}
\label{ssec:Headless noun phrases and nominalizations}

When the noun is absent, the item occurring in the rightmost position in the \isi{noun phrase} undergoes nominalization and functions as the head noun. In principle, all modifiers listed in \refsec{ssec:IntroductionNP} can be nominalized and then used as heads of a \isi{noun phrase} in argument position or as adjunct. Depending on their use in the clause, they take further case suffixes. 
 
Demonstrative pronouns, numerals, and other quantifiers do not require additional morphology when used as \isi{nouns} (except for \isi{oblique stem} marking and case suffixes if required by their function in the clause). Adjectives \refex{ex:Does an older (person) know it better, (or a young guy)}, pronouns \refex{ex:(I) could profit from yours}, and \isi{nouns} marked for the \isi{genitive} \refex{ex:‎‎‎Ours should not be let. (i.e. our people should not be allowed to sell our land)} or \isi{spatial cases}, postpositional phrases, and \isi{relative clauses} \refex{ex:Who wants to pours sugar (on the dish)}, and other clauses take either the suffix \tit{-ce} (\tit{-te}) or the suffix \tit{-il}. Thus, \refex{ex:(I) could profit from yours} shows a nominalized \isi{genitive} pronoun being suffixed with \tit{-ce} and thus able to take over the function of the S argument of the verb \tit{b-ič-ij}. Contrast this with \refex{ex:‎‎The villagers have milled their wheat; the people from other villages have milled (it)}, which illustrated the \isi{genitive} whose head noun has been omitted. Nominalized modifiers take case suffixes and often also \isi{oblique stem} markers just like any other nominals \refex{ex:Does an older (person) know it better, (or a young guy)}, \refex{ex:‎‎‎Ours should not be let. (i.e. our people should not be allowed to sell our land)}.
%
\begin{exe}

	\ex	\label{ex:Does an older (person) know it better, (or a young guy)}
	\gll	χːula-ce-li-j	ʡaˁħ-le	ʡaˁq'lu	b-alχ-u=w?\\
		big\tsc{-dd.sg-obl-dat}	good-\tsc{advz}	mind	\tsc{n-}know\tsc{.ipfv-prs.3=q}\\
	\glt	\sqt{Does an older (person) know it better (or a young guy)?}

	\ex	\label{ex:(I) could profit from yours}
	\gll	ala-ce	b-ič-ib-le	χajri	b-irχ-i\\
		\tsc{2sg.gen-dd.sg}	\tsc{n-}occur\tsc{.pfv-pret-cvb}	benefit	\tsc{n-}be.able\tsc{.ipfv-hab.pst}\\
	\glt	\sqt{(I) could profit from yours. (i.e. from your milk)}
	
	\ex	\label{ex:‎‎‎Ours should not be let. (i.e. our people should not be allowed to sell our land)}
	\gll	nišːa-la-t-a-l=q'ar	a-b-at-ij	ħaˁžat-le	ca-b\\
		\tsc{1pl-gen-pl-obl-erg=mod}	\tsc{neg-n-}let\tsc{.pfv-inf}	need\tsc{-advz}	\tsc{cop-n}\\
	\glt	\sqt{It is necessary to not let our (people to sell our land).}

	\ex	\label{ex:Who wants to pours sugar (on the dish)}
	\gll	b-ikː-an-il-li	pisuk'	či-k-erx-u\\
		\tsc{n-}want\tsc{.ipfv-ptcp-ref-erg}	sugar	\tsc{spr-down}-pour\tsc{.ipfv-prs}\\
	\glt	\sqt{The one who wants pours sugar (on the dish).}
\end{exe}

However, headless \isi{relative clauses} in which the verb bears the \isi{modal participle} suffix \tit{-an} and that function as \isi{absolutive} arguments without any further case marking, frequently occur without \tit{-ce} or \tit{-il} (see \refsec{sec:Headless relative clauses} for examples).

Nominalized modifiers can themselves be modified. In \refex{ex:Buy a ripe red one minor2}, a nominalized adjective is modified by a \isi{participle}, and in \refex{ex:The one who is standing says2} the \isi{relative clause}, which consists of only one \isi{participle}, is preceded by a \isi{demonstrative pronoun}.


\begin{exe}
	\ex	\label{ex:Buy a ripe red one minor2}
	\gll	asː-a	[b-iq'-ur(-ce)]	it'in-ce!\\
		buy\tsc{.pfv-imp}	\tsc{n-}ripen\tsc{-pret(dd.sg)}	red\tsc{-dd.sg}\\
	\glt	\sqt{Buy a/the red one that is ripe!} (E)
	
		\ex	\label{ex:The one who is standing says2}
	\gll	hej	ka-jcː-ur-il	Ø-ik'-ul	ca-w, ...\\
		this	\tsc{down}-get.up.\tsc{m.pfv-pret-ref}	\tsc{m-}say\tsc{.ipfv-icvb}	\tsc{cop-m} \\
	\glt	\sqt{The one who is standing says, ...''}
	
	
\end{exe}



%%%%%%%%%%%%%%%%%%%%%%%%%%%%%%%%%%%%%%%%%%%%%%%%%%%%%%%%%%%%%%%%%%%%%%%%%%%%%%%%

\section{Postpositional phrases}
\label{sec:Postpositional phrases}

Postpositional phrases consist of a postposition that is preceded by a \isi{noun phrase}. The \isi{noun phrase} can be complex containing modifiers, quantifiers or determiners \refex{ex:on one high precipice}. Postpositions govern various cases, most notably the \isi{genitive} and a few \isi{spatial cases} \refex{ex:on one high precipice}, \refex{ex:‎Now he is crying in prison}. Non-\isi{spatial postpositions} mostly govern the \isi{absolutive} case. See \refcpt{cpt:postpositions} for more examples of postpositions and postpositional phrases.
%
\begin{exe}
	\ex	\label{ex:on one high precipice}
	\gll	ca	aq	dahag-le-b či-b\\
		one	high	precipice\tsc{-loc-n} on\tsc{-n}\\
	\glt	\sqt{on one high precipice}

	\ex	\label{ex:‎Now he is crying in prison}
	\gll	na	w-isː-ul	ca-w	tusnaq-le-w	w-i-w\\
		now	\tsc{m-}cry\tsc{-icvb}	\tsc{cop-m}	prison\tsc{-loc-m}	\tsc{m-}in\tsc{-m}\\
	\glt	\sqt{‎Now he is crying in prison.}
\end{exe}

Postpositional phrases are always head-final, so it is ungrammatical for a postposition to precede the noun. However, all \isi{spatial postpositions} also occur as adverbials and/or spatial \isi{preverbs} without any additional morphology. Thus, it is not always easy to identify which part of speech a relevant item belongs to. For instance, the postposition \tit{b-i-b} \sqt{in, inside} governs, among other cases, the \tsc{loc}-essive, \tsc{loc}-lative or \tsc{loc}-\isi{ablative} (\refsec{ssec:postposition bi}), as shown in \refex{ex:‎Now he is crying in prison}. However, there is also a spatial \isi{preverb} \tit{b-i} and an adverbial \tit{b-i} with the same meaning. In example \refex{ex:He stayed in prison}, what looks like a stranded postposition \tit{w-i(-w)} is instead the \isi{spatial adverb}. Example \refex{ex:I will go inside} illustrates the preverbal use. No case marked \isi{noun phrase} is preceding \tit{w-i}.
%
\begin{exe}
	\ex	\label{ex:He stayed in prison}
	\gll	w-i-w kelg-un	hel	tusnaq-le-w\\
		\tsc{m-}in\tsc{-m} remain\tsc{.pfv-pret}	that	prison\tsc{-loc-m}\\
	\glt	\sqt{He stayed in prison.}

	\ex	\label{ex:I will go inside}
	\gll	du	w-i-ha-lqʷ-an=da\\
		\tsc{1sg}	\tsc{m-in-up}-direct\tsc{.ipfv-ptcp=1}\\
	\glt	\sqt{I will go inside.}
\end{exe}
