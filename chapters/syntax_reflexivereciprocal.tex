\chapter{Reflexive and reciprocal constructions}
\label{cpt:Reflexive and reciprocal constructions}

This chapter treats the syntax of reflexive and reciprocal constructions. The morphological paradigms of reflexive and reciprocal pronouns are given in \refsec{sec:Reflexive pronouns} and \refsec{sec:Reciprocal pronouns}.


%%%%%%%%%%%%%%%%%%%%%%%%%%%%%%%%%%%%%%%%%%%%%%%%%%%%%%%%%%%%%%%%%%%%%%%%%%%%%%%%

\section{Reflexive constructions}
\label{sec:Reflexive constructions}

Sanzhi has morphologically simple and complex reflexive pronouns. The pronouns are \tit{ca-w}\slash\tit{ca-r}\slash\tit{ca-b} in the singular (oblique stem \tit{cin-}) and \tit{ca-b}\slash\tit{ca-d} (oblique stem \tit{ču-}) in the plural. The relevant case paradigms are given in \refsec{sec:Reflexive pronouns}. Their functions are summarized in \reftab{tab:Functions of simple and complex reflexive pronouns}. There are two types of complex reflexive pronouns, which always consist of reduplication of the simple reflexive pronoun. Both types contain one reflexive that bears case marking according to the function of the reflexive pronoun in its clause and often appears as the second part of the complex reflexive pronoun. The first part either copies the case marking of the controlling noun phrase (`case copying') or it invariably bears the genitive case (`complex genitive reflexive'). The functional range of both types is roughly identical, but there are three restrictions that all concern reflexives marked by the genitive case. First of all, complex genitive reflexives do not have a form of the genitive case, because this would lead to two identical pronouns used together, which is ungrammatical (see \reftab{tab:Complex reflexive pronouns} in \refsec{sec:Reflexive pronouns}). Therefore, in the respective constructions only case-copying complex reflexives ca be used (e.g. local reflexivization and reciprocal constructions). Second, in emphatic reflexivization the complex reflexives are morphologically rather a mixture of genitive reflexive and case-copying reflexive, because they consist of a first part in the genitive and a second part that has the same case as the nominal to which the emphatic belongs (\refsec{ssec:Emphatic reflexive use}). Third, as pause fillers only genitive forms of simple reflexives occur; all other case forms cannot be used.
%
\begin{table}
	\caption{Functions of simple and complex reflexive pronouns}
	\label{tab:Functions of simple and complex reflexive pronouns}
	\small
	\begin{tabularx}{0.95\textwidth}[]{%
		>{\raggedright\arraybackslash}X
		>{\centering\arraybackslash}p{36pt}
		>{\centering\arraybackslash}p{36pt}}
		
		\lsptoprule
		{}									&	simple		& 	complex\\
		\midrule
		local reflexivization (\refsec{ssec:Local reflexivization})							&	y		&	y\\  
		reciprocalization (\refsec{sec:Reciprocal constructionss})							&	n		&	y\\
		emphatic reflexivization	(\refsec{ssec:Emphatic reflexive use})					&	y		&	y\\
		long-distance reflexivization (including logophoric contexts) (\refsec{ssec:Long-distance reflexivization})	&	y		&	n\\
		%co-reference across clausal boundaries				&	y		&	n\\
		pause fillers (\refsec{sec:Pause fillers, address particles, exclamatives, and interjections})						&	y		&	n\\
		comitative constructions (\refsec{sec:Comitative constructions})				&	y		&	n\\
		\lspbottomrule
	\end{tabularx}
\end{table}

In this chapter, local, non-local, and emphatic reflexivization are treated, as well as co-reference across clausal boundaries, and reciprocalization. For the other functions see the respective sections (references to them are given in \reftab{tab:Functions of simple and complex reflexive pronouns}).

In all types of local reflexive constructions, the pronouns are only used with third person referents. For reflexivization of first and second person the personal pronouns are used. In long-distance reflexivization with logophoric function simple reflexive pronouns are occasionally used to refer to speech act participants (i.e. first or second person referents).


% --------------------------------------------------------------------------------------------------------------------------------------------------------------------------------------------------------------------- %

\subsection{Local reflexivization}
\label{ssec:Local reflexivization}

In local reflexive constructions, pronouns are bound by an antecedent within the same clause. With first and second person the normal personal pronouns occur:
%
\begin{exe}
	\ex	\label{ex:Do not kill yourself}
	\gll	u-l	u	ma-kerx-utːa!\\
		\tsc{2sg-erg}	\tsc{2sg}	\tsc{proh-}kill\tsc{.ipfv-proh.sg}\\
	\glt	\sqt{Do not kill yourself!} (E)

	\ex	\label{ex:‎We love us}
	\gll	nišːi-j	nušːa	d-ičː-aq-ud\\
		\tsc{1pl-dat}	\tsc{1pl}	\tsc{1/2pl-}want\tsc{.ipfv-caus-1.prs}\\
	\glt	\sqt{‎We love ourselves.} (E) 
\end{exe}

Only for third person reflexivization the reflexive pronouns are used. Almost all corpus examples contain simple reflexive pronouns, but \refex{ex:She does not make herself a glass of tea like you@10} shows a complex reflexive in the function of beneficiary.
%
\begin{exe}
	\ex	\label{ex:(The boy) gave them one pear each, one he kept for himself@1}
	\gll	ca	ca	il-tːa-j	d-ičː-ib,	ca	cin-i-j	b-at-ur\\
		one	one	that-\tsc{pl.obl-dat}	\tsc{npl-}give\tsc{.pfv-pret}	one	\tsc{refl.sg-obl-dat}	\tsc{n-}let\tsc{.pfv-pret}\\
	\glt	\sqt{(The boy) gave them one (pear) each, one he kept for himself.}

	\ex	\label{ex:She does not make herself a glass of tea like you@10} [talking about the sister of one of the speakers]\\
	\gll	u-l	daˁʡle	čaˁj-la	istikan	a-b-irq'-u	cin-ni cin-i-j\\
		\tsc{2sg-erg}	as	tea\tsc{-gen}	glass	\tsc{neg-n-}do\tsc{.ipfv-prs}	\tsc{refl.sg-erg}	\tsc{refl.sg-obl-dat}\\
	\glt	 \sqt{She does not make herself a glass of tea like you.}
\end{exe}

Reflexive pronouns are in complementary distribution with personal or demonstrative pronouns \refex{ex:S/he buys himself something}, \refex{ex:S/he buys him/her something}. The c-command requirement holds, i.e., a possessor cannot control a reflexive pronoun \refex{ex:Madina's mother sees herself@11a}.
%
\begin{exe}
	\ex	\label{ex:S/he buys himself something}
	\gll	it-i-l	cin-i-j	cik'al	isː-ul	ca-b\\
		that\tsc{-obl-erg}	\tsc{refl.sg-obl-dat}	something	buy\tsc{.ipfv-icvb	}	\tsc{cop-n}\\
	\glt	\sqt{S/he buys herself/himself something.} (E)

	\ex	\label{ex:S/he buys him/her something}
	\gll	it-i-l	it-i-j	cik'al	isː-ul	ca-b\\
		that\tsc{-obl-erg}	\tsc{dem-obl-dat}	something	buy\tsc{.ipfv-icvb}		\tsc{cop-n}\\
	\glt	\sqt{S/he buys him/her something.} (no co-reference) (E)

	\ex	\label{ex:Madina's mother sees herself@11a}
	\gll	Madina-la	aba	cinij	ca-r	či-r-ig-ul ca-r\\
		Madina\tsc{-gen}	mother	\tsc{refl.dat}	\tsc{refl-f}	\tsc{spr-f}-see\tsc{.ipfv-icvb} \tsc{cop-f}\\
	\glt	\sqt{Madina's$_{i}$ mother$_{j}$ sees herself$_{*i/j}$.} (E)
\end{exe}

The complex reflexive pronouns must be locally bound \refex{ex:Madina wants that the mother sees herself@11b} and therefore cannot have an antecedent in another clause. The example in \refex{ex:Madina wants that the mother sees herself@11b} is fully grammatical if the pronoun is bound by the noun \tit{aba} \sqt{mother}, which occurs in the same clause.
%
\begin{exe}
	\ex	\label{ex:Madina wants that the mother sees herself@11b}
	\gll	Madina-j	b-ikː-ul ca-b	[aba	cinij	ca-r	či-r-až-ib-le]\\
		Madina\tsc{-dat}	\tsc{n-}want\tsc{.ipfv-icvb} \tsc{cop-n}	mother	\tsc{refl.dat}	\tsc{refl-f}	\tsc{spr-f}-see\tsc{.pfv-pret-cvb}\\
	\glt	\sqt{Madina$_{i}$ wants that the mother$_{j}$ sees herself$_{*i/j}$.} (E)
\end{exe}

The simple pronouns can occur as arguments and adjuncts of various types, e.g. patients, experiencers \refex{ex:Madina's mother sees herself@11a}, stimuli \refex{ex:Madina likes / wants / loves herself. OR Madina likes / wants / loves her}, beneficiaries \refex{ex:S/he buys himself something}, goals \refex{ex:Rashid is looking at himselfA}, possessors \refex{ex:‎‎‎The wolf gave him also his hair}, and complements of postpositions \refex{ex:Madina is talking about herself}. 
%
\begin{exe}
	\ex	\label{ex:Rashid is looking at himselfA}
	\gll	Rašid	(ca-w)	cin-i-j	er=či	w-ik'-ul	ca-w\\
		Rashid	(\tsc{refl-m})	\tsc{refl.sg-obl-dat}	look=on	\tsc{m-}say\tsc{.ipfv-icvb}	\tsc{cop-m}\\
	\glt	\sqt{Rashid is looking at himself.} (E)

	\ex	\label{ex:‎‎‎The wolf gave him also his hair}
	\gll	bec'-li=ra	d-ičː-ib	hel-i-j	cin-na	ʁiz-be\\
		wolf\tsc{-erg=add}	\tsc{npl-}give\tsc{.pfv-pret}	that\tsc{-obl-dat}	\tsc{refl.sg-gen}	hair\tsc{-pl}\\
	\glt	\sqt{‎‎‎The wolf gave him also his hair.}

	\ex	\label{ex:Madina is talking about herself}
	\gll	Madina	(ca-r)	cin-na	qari=či-r	ʁaj	r-ik'-ul	ca-r\\
		Madina	\tsc{refl-f}	\tsc{refl.sg-gen}	at.top=on\tsc{-f}	word	\tsc{f-}say\tsc{.ipfv-icvb}	\tsc{cop-f}\\
	\glt	\sqt{Madina is talking about herself.} (E)
\end{exe}

The same is true for the complex reflexive pronoun except for the possessor function and the use in postpositional phrases. The complex genitive can never be used as possessor, and the case-copy pronoun is judged as marginal or interpreted as an emphatic reflexive and thus not as a part of a complex reflexive pronoun. Thus, the preferred and entirely acceptable reading of \refex{ex:Rashid loves his mother} is \sqt{Rashid himself loves his mother}. In the possessor function normally the simple reflexive pronouns are used \refex{ex:‎‎‎The wolf gave him also his hair}. The same point is illustrated in example \refex{ex:Rashid himself / alone built his house}: the reflexive bearing the ergative case functions as emphatic particle whereas the genitive reflexive occupies the possessor position.
%
\begin{exe}
	\ex	\label{ex:Rashid loves his mother}
	\gll	{??}	Rašid-li-j	cin-i-j	cin-na	aba	r-ičː-aq-u\\
		{}	Rashid\tsc{-obl-dat}	\tsc{refl.sg-obl-dat}	\tsc{refl.sg-gen}	mother	\tsc{f-}love\tsc{-caus-prs.3}\\
	\glt	\sqt{Rashid loves his mother.} (E)

	\ex	\label{ex:Rashid himself / alone built his house}
	\gll	Rašid-li	cin-ni	cin-na	qal	b-arq'-ib\\
		Rashid\tsc{-erg}	\tsc{refl.sg-erg}	\tsc{refl.sg-gen}	house	\tsc{n-}do\tsc{.pfv-pret}\\
	\glt	\sqt{Rashid himself\slash alone built his house.} (E)
\end{exe}

Similarly, a complex reflexive within a postpositional phrase is judged as possible but less felicitous than a simple pronoun unless the first part can function as emphatic reflexive \refex{ex:Madina is talking about herself}.

The simple and the reflexive pronouns are interpreted as bound variables that can be bound by indefinite noun phrases \refex{ex:Every boy loves his mother}, \refex{ex:Every boy saw himself@25}.
%
\begin{exe}
	\ex	\label{ex:Every boy loves his mother}
	\gll	har	durħuˁ-j	cin-na	aba	r-ičː-aq-u\\
		every	boy\tsc{-dat}	\tsc{refl-gen}	mother	\tsc{f-}love\tsc{-caus-3.prs}\\
	\glt	\sqt{Every boy loves his mother.} (E)

	\ex	\label{ex:Every boy saw himself@25}
	\gll	har	durħuˁ-j	[cin-na	ca-w]	či-w-až-ib\\
		every	boy\tsc{-dat}	\tsc{refl-gen}	\tsc{refl-m}	\tsc{spr-m-}see\tsc{.pfv-pret}\\
	\glt	\sqt{Every boy saw himself.} (E)
\end{exe}
%
The internal order of the two parts of the complex reflexive pronoun that exhibits case copying is free to some degree. Thus, in example \refex{ex:Rashid is looking at himself} the two pronouns can be switched around and they can also be positioned before the controller.
%
\begin{exe}
	\ex	\label{ex:Rashid is looking at himself}
	\begin{xlist}
		\ex	\label{ex:Rashid is looking at himself@A}
		\gll	Rašid	[ca-w	cin-i-j]	er=či	w-ik'-ul	ca-w\\
			Rashid	\tsc{refl-m}	\tsc{refl.sg-obl-dat}	look=on	\tsc{m-}say\tsc{.ipfv-icvb}	\tsc{cop-m}\\
		\glt	\sqt{Rashid is looking at himself.} (E)

		\ex	\tit{Rašid [cinij caw] erči wik'ul caw}	\label{ex:Rashid is looking at himself@B}

		\ex	\tit{[caw cinij] Rašid erči wik'ul caw}	\label{ex:Rashid is looking at himself@C}
	\end{xlist}
\end{exe}

This is not possible in the ergative construction. In contrast to the extended intransitive construction in \refex{ex:Rashid is looking at himself}, the complex reflexive cannot precede the controlling noun \refex{ex:himself Rashid is praising ungrammatical@29a}. Due to time constraints I did not systematically test verbs from different valency classes and their use with complex reflexives in local reflexivization, and this topic must be left to future research. 
%
\begin{exe} 
	\ex	\label{ex:himself Rashid is praising ungrammatical@29a}
	\gll {*} [cin-ni ca-w] Rasul-li gap w-irq'-ul ca-w	\\
	{}	\tsc{refl.sg-erg} \tsc{refl-m}	Rasul\tsc{-erg}		praise \tsc{m}-do.\tsc{ipfv-icvb}	\tsc{cop-m}\\
	\glt	(Intended meaning: \sqt{Rasul is praising himself.}) (E)
\end{exe}

We can speculate a bit about the origin of complex reflexives. It is possible to elicit examples in which it seems that the pronoun can be split up \refex{ex:Rashid will you please stop staring at the mirror@A}, \refex{ex:Rashid will you please stop staring at the mirror@B}. In these examples, the two parts of the case-copying reflexive pronoun are independent of each other and do not form one constituent. The part that copies the case functions as emphatic reflexive, which is co-referent with the controlling noun phrase, and enforces the reflexive interpretation. The second part is a simple reflexive pronoun. It is likely that such sentences represent the diachronic source of the case-copying complex reflexive constructions.\footnote{I am grateful to an anonymous reviewer for this suggestion. Note that the root of the absolutive reflexive pronouns, \textit{ca}-, is very likely a cognate of the standard copula \textit{ca-b}.} In the following examples \refex{ex:Rashid will you please stop staring at the mirror} the emphatic reflexive is given in bold. The other reflexive pronoun functions as goal argument of the predicate \sqt{look at} in a standard local reflexive construction.
%
\begin{exe}
	\ex	\label{ex:Rashid will you please stop staring at the mirror}
	\begin{xlist}
		\ex	\tit{cinij Rašid caw erči wik'ul caw}		\label{ex:Rashid will you please stop staring at the mirror@A}

		\ex	\tit{caw Rašid cinij erči wik'ul caw}		\label{ex:Rashid will you please stop staring at the mirror@B}
	\end{xlist}
\end{exe}

As \refex{ex:Rashid will you please stop staring at the mirror@C} shows, the emphatic reflexive cannot be separated from the noun it accompanies by an intervening verb, which is a general rule that applies to all emphatic reflexives.
%
\begin{exe}
	\ex	\tit{{*} Rašid cinij erči wik'ul caw caw}		\label{ex:Rashid will you please stop staring at the mirror@C}
\end{exe}


The situation is different in case of the complex genitive reflexive which consists of a genitive reflexive pronoun and a second reflexive pronoun that takes the appropriate case-marking, because they do not allow for varying orders of the two pronouns. The reason for the restriction is probably a grammaticalization path, which differs from that of the case-copying complex reflexives. Genitive reflexives are not used in the emphatic reflexive function, but only as pause fillers, and I do not want to suggest that their function as pause fillers forms the basis of the complex genitive, although I lack an alternative hypothesis. In the complex genitive reflexive construction, the internal order of the genitive reflexive pronoun cannot be changed \refex{ex:Rasul is praising himself ungrammatical@29c}. If we switch the order around to \tit{cinna caw}, \refex{ex:Rasul is praising himself ungrammatical@29c} becomes grammatical.
%
\begin{exe}
	\ex	\label{ex:Rasul is praising himself ungrammatical@29c}
	\gll	{*} 	Rasul-li [ca-w cin-na] gap w-irq'-ul	ca-w\\
		{}	Rasul\tsc{-erg}	\tsc{refl-m}	\tsc{refl.sg-gen}	praise \tsc{m}-do.\tsc{ipfv-icvb}	\tsc{cop-m}\\
	\glt	(Intended meaning: \sqt{Rasul is praising himself.}) (E)
\end{exe}

Within a ditransitive construction the direct \refex{ex:Patimat showed Rashid to himself on the picture@12a} or the indirect object \refex{ex:Patimat showed to Arsen himself on the picture@12b} can function as binder though simple reflexive pronouns would be preferred in such examples.
%
\begin{exe}

		\ex	\label{ex:Patimat showed Rashid to himself on the picture@12a}
		\gll	Patʼimat-li	Rašid$_{i}$	surraticːe-w	cin-na	cinij$_{i}$	či-w-iž-aq-ul=de\\
			Patimat\tsc{-erg}	Rashid	picture.\tsc{in-m}	\tsc{refl-gen}	\tsc{refl.dat}	\tsc{spr-m}-see.\tsc{ipfv-caus-icvb=pst}\\
		\glt	\sqt{Patimat was showing Rashid$_{i}$ to himself$_{i}$ on the picture.} (E)

		\ex	\label{ex:Patimat showed to Arsen himself on the picture@12b}
		\gll	Patʼimat-li	či-w-iž-aq-ul=de	Arsen-ni-ji$_{i}$	surrat-le-w	či-w$_{i}$	cinij	ca-w\\
			Patimat\tsc{-erg}	\tsc{spr-m}-see.\tsc{ipfv-caus-icvb=pst}	Arsen\tsc{-obl-dat}	picture\tsc{-loc-m}	on\tsc{-m}		\tsc{refl.dat} \tsc{refl-m} \\
		\glt	\sqt{Patimat was showing to Arsen$_{i}$ himself$_{i}$ on the picture.} (E)

\end{exe}

Most notably, the case marking of the antecedent and the reflexive pronoun can swap. This phenomenon, which is cross-linguistically rare, has been observed in a number of East Caucasian languages, among them Sanzhi Dargwa and other Dargwa varieties (see \citealp{Forker2014} for a typological study). It is constrained by three interacting factors: morphological complexity of the pronouns, constituent order, and valency class of the predicate.

For morphologically simple reflexive pronouns, case swapping is generally unavailable. For instance, if in clauses with transitive or affective verbs the cases are distributed such that the controlling noun bears the case marking of the agent or experiencer (ergative or dative) and the reflexive appears in the absolutive, then a local reflexive \refex{ex:Every boy loves his mother}, \refex{ex:Every boy saw himself@25} and a non-reflexive reading are possible \refex{ex:Madina likes / wants / loves herself. OR Madina likes / wants / loves her}. In the non-reflexive reading, the pronoun refers to a contextually salient referent that, for instance, had been mentioned in the preceding discourse:
%
\begin{exe}
	\ex	\label{ex:Madina likes / wants / loves herself. OR Madina likes / wants / loves her}
	\gll	Madina-j ca-r	r-ikː-ul	ca-r\\
		Madina\tsc{-dat}	\tsc{refl-f}	\tsc{f-}want\tsc{.ipfv-icvb}	\tsc{cop-f}\\
	\glt	\sqt{Madina likes\slash wants\slash loves herself.} OR \sqt{Madina likes\slash wants\slash loves her.}~(E)
\end{exe}

If we swap the case marking, only the non-reflexive reading remains. With swapped case marking, it is more natural to position the pronoun in the dative case before the noun in the absolutive \refex{ex:She likes / wants / loves Madina}, although the reversed order is also possible.
%
\begin{exe}
	\ex	\label{ex:She likes / wants / loves Madina}
	\gll	cin-i-j	Madina r-ikː-ul	ca-r\\
		\tsc{refl.sg-obl-dat}	Madina \tsc{f-}want\tsc{.ipfv-icvb}	\tsc{cop-f}\\
	\glt	\sqt{She likes\slash wants\slash loves Madina.} (E)
\end{exe}

With complex reflexive pronouns, affective as well as transitive predicates exhibit a reversal of case marking \xxref{ex:Rasul is praising himself@13}{ex:Rasul sees himself@13}, but all other positions including co-arguments of extended intransitive verbs are excluded \refex{ex:Rasul is believing in himself@13}. With transitive and affective predicates the distribution of the case marking in reflexive constructions is free, i.e. either the controller or the pronoun take the ergative or the dative case suffix \xxref{ex:Rasul is praising himself@13}{ex:Rasul sees himself@13}.
%
\begin{exe}
	\ex	\label{ex:Rasul is praising himself@13}
	\begin{xlist}
		\ex	\label{ex:Rasul is praising himself@13a}
		\gll	Rasul-li	cin-ni	ca-w	/	cin-na	ca-w	gap w-irq'-ul ca-w\\
			Rasul\tsc{-erg}	\tsc{refl-erg}	\tsc{refl-m}	/	\tsc{refl-gen}	\tsc{refl-m}	praise \tsc{m}-do.\tsc{ipfv-icvb} \tsc{cop-m}\\
		\glt	\sqt{Rasul is praising himself.} (E)

		\ex	\label{ex:ex:Rasul is praising himself@13b}
		\gll	Rasul	ca-w	cin-ni	/	cin-na	cin-ni	gap w-irq'-ul ca-w\\
			Rasul	\tsc{refl-m}	\tsc{refl-erg}	/	\tsc{refl-gen}	\tsc{refl-erg}	praise \tsc{m}-do.\tsc{ipfv-icvb} \tsc{cop-m}\\
		\glt	\sqt{Rasul is praising himself.} (E)
	\end{xlist}

	\ex	\label{ex:Rasul sees himself@13}
	\begin{xlist}
		\ex	\label{ex:Rasul sees himself@13c}
		\gll	Rasul-li-j	cinij	ca-w	/	cin-na	ca-w	čiːg-ul ca-w\\
			Rasul\tsc{-obl-dat}	\tsc{refl.dat}	\tsc{refl-m}	/	\tsc{refl-gen}	\tsc{refl-m}	see\tsc{.m-cvb} \tsc{cop-m}\\
		\glt	\sqt{Rasul sees himself.} (E)

		\ex	\label{ex:Rasul sees himself@13d}
		\gll	Rasul	cinij	ca-w	/	cin-na	cinij	čiːg-ul ca-w\\
			Rasul	\tsc{refl.dat}	\tsc{refl-m}	/	\tsc{refl-gen}	\tsc{refl.dat}	praise\tsc{.m-cvb} \tsc{cop-m}\\
		\glt	\sqt{Rasul sees himself.} (E)
	\end{xlist}
\end{exe}

The reversal of the marking is forbidden for coarguments of extended intransitive verbs that are marked with spatial cases. For example, ‘believe' requires an argument in the absolutive and another argument in the dative. The controller of the reflexive pronoun must be in the absolutive; any change in the marking leads to ungrammaticality \refex{ex:Rasul is believing in himself@13}. 

\begin{exe}
	\ex	\label{ex:Rasul is believing in himself@13}
	\begin{xlist}
		\ex	\label{ex:Rasul is believing in himself@13A}
		\gll	Šamil	či-w-w-irχː-ul ca-w	cin-na	cinij\\
			Shamil	\tsc{spr-m-m}-believe.\tsc{m.ipfv-icvb} \tsc{cop-m} \tsc{refl-gen}	\tsc{refl.dat}\\
		\glt	\sqt{Shamil believes in himself.} (E)

		\ex	\label{ex:Rasul is believing in himself@13B}
		\gll {*} Šamil-li-j	či-w-w-irχː-ul ca-w	cin-na	ca-w	\\
		{}	Shamil-\tsc{obl-dat}	\tsc{spr-m-m}-believe.\tsc{m.ipfv-icvb} \tsc{cop-m} \tsc{refl-gen}	\tsc{refl-m}\\
		\glt	 (Intended meaning: \sqt{Shamil believes in himself.}) (E)
	\end{xlist}
\end{exe}


Out of context, there is no semantic or pragmatic difference between reflexive constructions with standard case marking and reflexive constructions with reversed marking. This stands in contrast to other East Caucasian languages for which such differences have been reported (see \citealp{Forker2014} for references). There are some restrictions on the word order, concerning both standard reflexive constructions (see above) and those with reversed marking, but this topic requires further research.
%
%\begin{exe}
	%\ex	\tit{* cin-ni gap w-irq'-ul ca-w Rasul ca-w}	\label{ex:himself Rasul is praising ungrammatical@29b}	
		%\glt	(Intended meaning: \sqt{Rasul is praising himself.}) (E)
%\end{exe}

To sum up the discussion on complex reflexive pronouns in Sanzhi we can state that these anaphoras can function as A (agents and experiencers) as the examples in \refex{{ex:ex:Rasul is praising himself@13b}} and \refex{ex:Rasul sees himself@13d} have shown. The constructions illustrated in both sentences are plain reflexive constructions with basic transitive and affective verbs. Under certain circumstances the complex reflexive pronouns can also precede their antecedents \refex{{ex:Rashid is looking at himself@C}}, but this topic requires more testing. Since the complex reflexive pronouns are not personal or demonstrative pronouns, but must be locally bound, the Sanzhi data look like a violation of a commonly assumed constraint, namely that anaphors cannot fulfill the grammatical role of subjects. However, such an argumentation crucially depends on the notion of subject and whether it can be applied to Sanzhi Dargwa. \citet{Forker2014} argues that Sanzhi does not have subjects in exactly the same sense of how this term is applied to European languages, and that therefore the examples in \refex{ex:ex:Rasul is praising himself@13b}, \refex{ex:Rasul sees himself@13d} do not represent a violation of the subject anaphora constraint.


% --------------------------------------------------------------------------------------------------------------------------------------------------------------------------------------------------------------------- %

\subsection{Emphatic reflexive and intensifying uses}
\label{ssec:Emphatic reflexive use}

Simple and complex genitive reflexives serve emphatic and intensifying functions, in which case they are not bound by a co-referential argument in the clause but simply occur together with a nominal co-constituent. Complex reflexive pronouns that copy the case of the controller cannot be used in this function. The nominal co-constituent can be a pronoun \refex{ex:They themselves were carrying the sacks and when they were filled@15a} or a noun \refex{ex:Neither Xurija herself comes (to me) nor do I go (to her)@3c}, or it can be absent \refex{ex:‎‎She did not eat the nut herself, but threw it through the chimney into the house@3a}. In the Sanzhi corpus most emphatic reflexives occur without the nominal co-constituent, but if they co-occur, then the reflexive follows the nominal, which is cross-linguistically the most common position for emphatic reflexives. Furthermore, the simple reflexives are preferred over the complex reflexives in this function. The simple and complex genitive reflexives can only function as emphatic reflexives with third person co-constituents. As is the case with local reflexivization, emphatic reflexivization of first and second persons is done with first and second person pronouns. These pronouns occur in the genitive case and are usually used without the pronominal co-constituents (see the examples below).

\citet{König.Gast2006} list the following four functions of emphatic reflexives:
%
\begin{enumerate}
	\item	adnominal \tit{(X, not Y or X, in contrast to Y)}
	\item	adverbial-exclusive \tit{(alone, without help)} 
	\item	adverbial-inclusive \tit{(also, too)} 
	\item	attributive \tit{(own)}
\end{enumerate}

Reflexive pronouns in Sanzhi predominantly occur in the first function, in which two situations or two referents are contrasted with each other. This can be done by means of parallel structures in which the items follow each other and are explicitly contrasted. In \refex{ex:Neither Xurija herself comes (to me) nor do I go (to her)@3c} the speaker contrasts herself with a friend called Hurija. In example \refex{ex:‎Although there were no brothers (relatives) and sisters@6b} the contrast is expressed in a parallel structure that is marked by means of the additive enclitic =\textit{ra} on both members, the speaker herself and the friends.
%
\begin{exe}
	\ex	\label{ex:Neither Xurija herself comes (to me) nor do I go (to her)@3c}
	\gll	ħurija	ja	ca-r	ha-r-ax-ul	akːu	ja	du	r-ax-ul	akːʷa-di\\
		Hurija	or	\tsc{refl-f}	\tsc{up-f-}go\tsc{-icvb}	\tsc{cop.neg}	or	\tsc{1sg}	\tsc{f-}go\tsc{-icvb}	\tsc{cop.neg-1}\\
	\glt	\sqt{Neither Hurija herself comes (to me) nor do I go (to her).}

	\ex	\label{ex:‎Although there were no brothers (relatives) and sisters@6b}
	\gll	tuχum-te	ruc-be	akːu=xːar,	w-alχ-an	ucːi	urš-b-a-l,	du-l=ra usal-dex	a-b-irq'-id	mas-la	arc-la, ču-l=ra	a-b-irq'-ul	er	r-arq'-ib=da\\
		relative\tsc{-pl}	sister\tsc{-pl}	be.\tsc{neg=conc}	\tsc{m-}know\tsc{.ipfv-ptcp}	brother	guy\tsc{-pl-obl-erg}	\tsc{1sg-erg=add}	weak\tsc{-nmlz}	\tsc{neg-n-}do\tsc{.ipfv-1}	wealth\tsc{-gen}	silver\tsc{-gen}	\tsc{refl.pl-erg=add}	\tsc{neg-n-}do\tsc{.ipfv-icvb}	life	\tsc{f-}do\tsc{.pfv-pret=1}\\
	\glt	\sqt{‎Although there were no brothers (relatives) and sisters, my friends (known brothers), the guys, I also did not show a lack of money, things (i.e. supported them), and (they) themselves also did not, so I have lived (my life).}
\end{exe}

In most cases the contrast is rather indirect and resembles topic switch constructions in which the sentence topic switches from one sentence to the next \refex{ex:He says, Hello, Asja Iwanowna@3b}, \refex{ex:‎‎Itself (the fox itself) is sitting and cleaning its face@3d}. As examples \refex{ex:They themselves were carrying the sacks and when they were filled@15a}, \refex{ex:Moreover, (the Icari people) themselves were prepared for our coming@15b} show, among the complex reflexives only genitive reflexives occur in the emphatic reflexive function. 
%
\begin{exe}
	\ex	\label{ex:He says, Hello, Asja Iwanowna@3b} [He says, ``Hello, Asja Iwanowna.'']\\
	\gll	ca-r	ka-r-icː-ur	ca-r	er či-ka-r-ik'-ul	heχ-i-j\\
		\tsc{refl-f}	\tsc{down-f-}stand\tsc{.pfv-pret}	be\tsc{-f}	look \tsc{spr-down}\tsc{-f}-look\tsc{.ipfv-icvb}	\tsc{3sg.down-obl-dat}\\
	\glt	\sqt{‎‎(She) herself is standing and looking at him.}

	\ex	\label{ex:‎‎Itself (the fox itself) is sitting and cleaning its face@3d} [The fox brought a lot of animals to the poor farmer. The farmer came home and was wondering, looked at them.]\\
	\gll	ca-b	ka-b-iž-ib-le	daˁʡ	amzu	d-irq'-ul	ca-d\\
		\tsc{refl-n}	\tsc{down-n-}be\tsc{.pfv-pret-cvb}	face	clean	\tsc{npl-}do\tsc{.ipfv-icvb}	\tsc{cop-npl}\\
	\glt	\sqt{(The fox) itself is sitting and cleaning its face.}

	\ex	\label{ex:They themselves were carrying the sacks and when they were filled@15a}
	\gll	hel-tː-a-li	ču-la	ču-l	d-iqː-ul,	hetːi	qːup-re	d-ic'-ib-le ...\\
		that\tsc{-pl-obl-erg}	\tsc{refl.pl-gen}	\tsc{refl.pl-erg}	\tsc{npl-}carry\tsc{.ipfv-icvb}	those	sack\tsc{-pl}	\tsc{npl-}fill\tsc{.pfv-pret-cvb}\\
	\glt	\sqt{They themselves were carrying the sacks and when they were filled ...}

	\ex	\label{ex:Moreover, (the Icari people) themselves were prepared for our coming@15b}
	\gll	tem.bolee	nišːa-la	priezd-li-j=ra	ču-la	ca-b=ra	padgatuwleni=de\\
		moreover	\tsc{1pl-gen}	arrival\tsc{-obl-dat=add}	\tsc{refl.pl-gen}	\tsc{refl-hpl=add}	prepared\tsc{=pst}\\
	\glt	\sqt{Moreover, (the Icari people) themselves were prepared for our coming.}
\end{exe}
	
	
The second function of emphatic reflexives, the adverbial-exclusive function \sqt{alone, without help}, is also attested for Sanzhi. Example \refex{ex:He went completely on his own (alone)@15c} illustrates this function with a complex genitive reflexive. Example \refex{} comes from a fairy tale and here the reflexive can be interpreted as adverbial-exclusive and/or as adnominal-contrastive.
%
\begin{exe}
	\ex	\label{ex:He went completely on his own (alone)@15c}
	\gll	absalut'na	cin-na	ca-w	w-aš-i\\
		absolutely	\tsc{refl.sg-gen}	\tsc{refl-m}	\tsc{m-}go\tsc{-hab.pst}\\
	\glt	\sqt{(He) went completely on his own (alone).}
	
	\ex	\label{ex:‎‎She did not eat the nut herself, but threw it through the chimney into the house@3a} [When she was sweeping, she found a walnut.]\\
	\gll	cin-ni	a-b-erk-un-ne,	turba-le-r	lak' 	b-i-ka-b-arq'-ib	ca-b	qili	hel	qix\\
		\tsc{refl.sg-erg}	\tsc{neg-n-}eat\tsc{.pfv-pret-cvb}	chimney\tsc{-loc-abl}	throw	\tsc{n-in-down-n-}do\tsc{.pfv-pret}	\tsc{cop-n}	home	that	nut\\
	\glt	\sqt{(‎‎She) did not eat the nut herself, but threw it through the chimney into the house.}
\end{exe}

The third function (adverbial-inclusive \sqt{also, too}) is not common in East Caucasian languages, including Sanzhi, because the languages have additive enclitics that already serve this function (\refsec{ssec:The additive enclitic}). The fourth function (attributive \sqt{own}) is covered by pronouns in the genitive case, i.e., by personal pronouns for first and second person and by simple reflexive pronouns for the third person. 

The first and second person genitive pronouns when used as intensifiers slightly differ in their morphosyntactic properties from the reflexive pronouns. First of all, they normally do not occur together with the pronominal co-constituent \refex{ex:I was a little boy@BREFL}. It would not be natural to add the pronoun in the absolutive to this sentence in order to overtly fill the position of the copula subject \refex{ex:I was a little boy@BREFL2}, and there are no examples of this kind in the corpus.
%

\begin{exe}
	\ex	\label{ex:ex:I was a little boy@BREFLIST}
	\begin{xlist}
	\ex	\label{ex:I was a little boy@BREFL} [He was a relative) of those people up there. I don't know of whose family he was.] \\
	\gll	nik'a	durħuˁ=de	di-la\\
		small	boy\tsc{=pst}	\tsc{1sg-gen}\\
	\glt	\sqt{(I) myself was a little boy.}
	
		\ex	\label{ex:I was a little boy@BREFL2} 
	\gll	{??} du nik'a	durħuˁ=de	di-la\\
		{} \tsc{1sg} small	boy\tsc{=pst}	\tsc{1sg-gen}\\
	\glt	 \sqt{(I) myself was a little boy.} (E)
		\end{xlist}
\end{exe}

Second, in all corpus examples, which are only a small handful, the genitive pronoun appears at the right boundary of the clause after the verb \refex{ex:I was a little boy@BREFL}, \refex{You can leave on your own, she says, without worries.}, \refex{They asked us, where did you come from?}. This position is typical for contrastive topics (\refsec{ssec:Declarative clauses}). But as the example in \refex{ex:(You) yourself go awayREFL} proves, this position is not obligatory. As to the function, the first and second person genitive pronouns used as intensifiers fulfill the first function as adnomial intensifiers \refex{ex:I was a little boy@BREFL}, \refex{ex:(You) yourself go awayREFL} and the second as adverbial-exclusive particles \refex{You can leave on your own, she says, without worries.}, \refex{They asked us, where did you come from?}.

\begin{exe}
		\ex	\label{You can leave on your own, she says, without worries.} [‎‎At our place, nobody steals, she said.]\\
		\gll	``uˁq'-en,''	r-ik'ʷ-ar,	``bahla-l	ala!''\\
			go.\tsc{m-imp}	\tsc{f}-say.\tsc{ipfv-prs.3}	slow-\tsc{advz}	\tsc{2sg.gen} \\
		\glt	\sqt{``You (can) leave,'' she says, ``without worries!''} (lit. ``Go slowly yourself!'' she says.)
		
			\ex	\label{They asked us, where did you come from?} [We left the Sanzhi. We went to Shari.]\\
		\gll	itːi=ra	``čina-r	sa-d-eʁ-ib-te=da=j?''	b-ik'-ul	xar	b-eʁ-ib,	``nišːa-la''\\
			\tsc{3pl=add}	where-\tsc{abl}	\tsc{hither-1/2pl}-go.\tsc{pfv-pret-dd.pl=2pl=q}	\tsc{hpl}-say.\tsc{ipfv-icvb}	ask	\tsc{n-aux.pfv-pret}	\tsc{1pl-gen}\\
		\glt	\sqt{They also asked us, ``Where did you come from?''} 
\
			\ex	\label{ex:(You) yourself go awayREFL}
		\gll	ala	r-uˁq'-aˁn!\\
			2\textsc{sg}.\textsc{gen}	\textsc{f}-go-\textsc{imp}\\
		\glt	\sqt{(You) yourself go away!} (I do not go.) (E)
\end{exe}


% --------------------------------------------------------------------------------------------------------------------------------------------------------------------------------------------------------------------- %

\subsection{Long-distance reflexivization}
\label{ssec:Long-distance reflexivization}

The simple reflexive pronouns are also bound across clausal boundaries. This means that they can occur in various types of subordinate clauses with the antecedent belonging to the main clause. Such a usage is impossible for complex reflexives. Sentence \refex{ex:At that the friend gave to Ali the gifts@5} illustrates long-distance reflexivization with a relative clause. In \refex{ex:When the man was in prison he remembered a lot, how the police beat} we find the reflexive pronoun in a complement clause.
%
\begin{exe}
	\ex	\label{ex:At that the friend gave to Ali the gifts@5}
	\gll	il=qːel	juldaš-li	[juldašː-a-l	cin-i-j	sa-qː-ib-te]		xunul-be	ʡaˁli-j	d-ičː-ib\\
		that=when	friend\tsc{-erg}	friend\tsc{.pl-obl-erg}	\tsc{refl.sg-obl-dat}	\tsc{hither}-carry\tsc{.pfv-pret-dd.pl} 		gift\tsc{-pl}	Ali\tsc{-dat}	\tsc{npl-}give\tsc{.pfv-pret}\\
	\glt	\sqt{At that the friend gave to Ali the gifts that his friend had brought to him (= to the friend).}	
\end{exe}

The vast majority of instances of long-distance reflexivization are logophoric contexts. This includes longer stretches of discourse that are framed by verbs of speech and cognition. Normally it is the author of the quote that serves as the antecedent of the reflexive:
%
\begin{exe}

		\ex	\label{ex:He is telling his wife and his son what had happened to him@4a}
		\gll	xːunul-li-cːe=ra	durħuˁ-cːe=ra	χabar	b-urs-ul	ca-b	[cin-ni-j	ag-ur-il-la]\\
			woman\tsc{-obl-in=add}	boy\tsc{-in=add}	story	\tsc{n-}tell\tsc{.pfv-icvb}	\tsc{cop-n}	\tsc{refl.sg-obl-dat}	go\tsc{.pfv-pret-ref-gen}\\
		\glt	\sqt{He is telling his wife and his son what had happened to him.}	

		\ex	\label{ex:When the man was in prison he remembered a lot, how the police beat}
		\gll	tusnaq-le-w=qːella	hek'-i-j	d-aqil	cik'al	han	či-sa-d-irk-ul=de	hel	admi-li-j	[cet'le	milic'a-b-a-l	ca-w	w-it-ib-ce=de=l,	cet'le	cin-ni	xːunul	it-ul	kelg-un-ce=de=l]\\
			prison\tsc{-loc-m=}when	\tsc{3sg-obl-dat}	\tsc{npl-}much	something	remember	\tsc{spr-hither}\tsc{-npl-}occur\tsc{.ipfv-icvb=pst}	that	person\tsc{-obl-dat}	how	police\tsc{-pl-obl-erg}	\tsc{refl-m}	\tsc{m-}beat.up\tsc{-pret-dd.sg=pst=indq}	how	\tsc{refl.sg-erg}	woman	beat.up\tsc{-icvb}	remain\tsc{-pret-dd.sg=pst=indq}\\
		\glt	\sqt{When the man was in prison he remembered a lot, how the police beat him up, how he beat up his wife.}	

	\ex	\label{ex:‎‎Once the sun and the evil wind argued about who is stronger}
	\gll	ca	zamana	bari=ra	wahi-ce	č'an=ra	čːal	d-uq-un	[kutːi	ču-cːe-rka	c'aq'-ce=de=l]\\
		one	time	sun\tsc{=add}	evil\tsc{-dd.sg}	wind\tsc{=add}	argument(\tsc{npl})	\tsc{npl-}go\tsc{.pfv-pret}	which	\tsc{refl.pl-in-abl}	mighty-\tsc{dd.sg=pst=indir.q}\\
	\glt	\sqt{‎‎Once the sun and the evil wind argued about who is stronger.}
\end{exe}

It is also possible that the speaker of the quote is included in the group of referents of the reflexive pronoun, i.e., pronoun and antecedent are not identical in their denotation, but the domain of reference of the reflexive is larger \refex{ex:‎‎‎He told what had happened to them}.
%
\begin{exe}
	\ex	\label{ex:‎‎‎He told what had happened to them}
	\gll	hel-i-l	b-urs-ib	ca-b	[ču-j	ag-ur-il]\\
		that\tsc{-obl-erg}	\tsc{n-}tell\tsc{-pret}	\tsc{cop-n}	\tsc{refl.pl-dat}	go\tsc{.pfv-pret-ref}\\
	\glt	\sqt{‎‎‎He told what had happened to them.}
\end{exe}

Reflexive pronouns are also used in non-logophoric contexts of co-reference across clauses and across sentences. They can occur when the speaker wants to refer to a topical referent in the discourse that is not necessarily used as an argument in the preceding clause, but simply a salient discourse topic at the moment of utterance. This is the case with optative phrases used to commemorate dead relatives and friends or acquaintances \refex{ex:It turned out that the Russian woman had looked}.
%
\begin{exe}
	\ex	\label{ex:It turned out that the Russian woman had looked}
	\gll	hel	ʡuˁrus	xːunul	er	r-ik'-ul	r-už-ib-le;	alžana	b-ikː-ab	cin-i-j\\
		that	Russian	woman	look	\tsc{f-}look.at\tsc{.ipfv-icvb}	\tsc{f-}be\tsc{-pret-cvb}	heaven	\tsc{n-}give\tsc{.pfv-opt.3}	\tsc{refl.sg-obl-dat}\\
	\glt	\sqt{It turned out that the Russian woman had looked (at the events), may heaven be given to her.} 
\end{exe}

Discourse topics expressed with reflexive pronouns are also found outside of optative phrases \refex{ex:‎Like this there is a little food prepared and served for her}. In \refex{ex:therefore he gave them pears, to the boys}, it seems that the speaker used first the reflexive because he assumed that the referent of the pronoun would be topical enough to be interpretable, but then he changed his mind and added the full noun phrase as an afterthought in order to reassure the reference of the pronoun.
%
\begin{exe}
	\ex	{[talking about the sister of one of the speakers]}\\	\label{ex:‎Like this there is a little food prepared and served for her}
	\gll	itwaj	cin-i-j	kam-le	χurejg-e	χe-d	ħaˁdur-re	ka-d-išː-ib-te\\
		like.this	\tsc{refl.sg-obl-dat}	little\tsc{-advz}	food\tsc{-pl}	exist.\tsc{down-npl}	ready\tsc{-advz}	\tsc{down-npl-}put\tsc{.pfv-pret-dd.pl}\\
	\glt	\sqt{‎Like there is little food for her prepared and served (lit. put down).} (i.e. as if she wasn't getting enough food)

	\ex	{[Talking about a group of three boys and another boy. One boy from the group brought the other boy back his hat, which he had lost.]}\\	\label{ex:therefore he gave them pears, to the boys}
	\gll	il	bahanne	it-i-l	ču-j	quˁr-be=ra	d-ičː-ib	hel-tːi	duˁrħ-n-aˁ-j\\
		that	therefore	that\tsc{-obl-erg}	\tsc{refl.pl-dat}	pear\tsc{-pl=add}	\tsc{npl-}give\tsc{.pfv-pret}	that\tsc{-pl}	boy\tsc{-pl-obl-dat}\\
	\glt	\sqt{Therefore he (= the other boy) gave them pears, to those boys.}
\end{exe}

In cases of discourse topics the referent of a reflexive pronoun can even be inanimate \refex{ex:‎[‎‎There is a tall summit.] It is called the midday summit}.
%
\begin{exe}
	\ex	\label{ex:‎[‎‎There is a tall summit.] It is called the midday summit} ‎[‎‎There is a tall summit.] \\
	\gll	muza	arilla	muza	b-ik'ʷ-ar	cin-i-j=ra\\
		summit	during.day	summit	\tsc{hpl-}say\tsc{.ipfv-prs}	\tsc{refl.sg-obl-dat=add}\\
	\glt	\sqt{It is called the midday summit.}
\end{exe}


%%%%%%%%%%%%%%%%%%%%%%%%%%%%%%%%%%%%%%%%%%%%%%%%%%%%%%%%%%%%%%%%%%%%%%%%%%%%%%%%

\section{Reciprocal constructions}
\label{sec:Reciprocal constructionss}

Reciprocal constructions are built either with plural reflexive pronouns \refex{ex:Madina and Patimat see each other@14a} or with specialized pronouns that make use of the numeral \tit{ca} \sqt{one}. There are three types of reciprocal pronouns whose paradigms are given in \refsec{sec:Reciprocal pronouns}. All pronouns occurring in reciprocal constructions are morphologically complex with the same patterns that the morphologically complex reflexive pronouns exhibit.

Reciprocal pronouns are always clause-bound. They occur in various argument and adjunct positions and are controlled by a suitable plural antecedent that can be omitted. In the following examples they function as patient \refex{ex:‎Murad and Rashid hit each other@A}, as experiencer or stimulus \refex{ex:‎They respected each other, loved each other}, as beneficiary, as addressee \refex{ex:‎‎They talk to each other what they would do}, as goal \refex{ex:They look at each other@A}, and as genitive possessor fulfilling the semantic role of an experiencer \refex{ex:‎The main point is that they understand each other, the couple}, \refex{ex:There no bad feelings between each other}. 
%
\begin{exe}
	\ex	\label{ex:‎They respected each other, loved each other}
	\gll	ca-lla	ca	ħuˁrmat	b-irq'-ul=de,	calli-j	ca	b-ikː-ul=de\\
		one\tsc{-gen}	one	respect	\tsc{n-}do\tsc{.ipfv-icvb=pst}	one\tsc{.obl-dat}	one	\tsc{hpl-}want\tsc{.ipfv-icvb=pst}\\
	\glt	\sqt{‎They respected each other, loved each other.}

	\ex	\label{ex:‎‎They talk to each other what they would do}
	\gll	[b-arq'-ib=el]	cal-li	calli-cːe	b-urs-ul	ca-b\\
		\tsc{n-}do\tsc{.pfv-pret=indq}	one\tsc{.obl-erg}	one\tsc{.obl-in}	\tsc{n-}tell\tsc{-icvb}	\tsc{cop-n}\\
	\glt	\sqt{‎‎They are talking to each other what they would do.}

	\ex	\label{ex:‎The main point is that they understand each other, the couple}
	\gll	glawni	cal-li	ca-lla	urk'i	arʁ-ib	ca-d	heχ-tːi	sub-xːunul-li\\
		main	one\tsc{.obl-erg}	one\tsc{-gen}	heart	understand\tsc{.pfv-pret}	\tsc{cop-npl}	\tsc{dem.down}\tsc{-pl}	husband-woman\tsc{-erg}\\
	\glt	\sqt{‎The main point is that they understand each other, the couple.}

	\ex	\label{ex:There no bad feelings between each other}
	\gll	ca-lla	ca-lla	urk'i	hitːi-dex	b-akːu\\
		one\tsc{-gen}	one\tsc{-gen}	heart	behind\tsc{-nmlz}	\tsc{n-}\tsc{cop.neg}\\
	\glt	\sqt{There are no bad feelings between each other.} 
\end{exe}

In all the above corpus examples the first part of the reciprocal pronoun copies the case of the antecedent, which is absent from the clause, and the second part takes the case marking appropriate to its role in the clause. It is also possible, just like with complex reflexive pronouns, to mark the first part invariably with the genitive \refex{ex:There no bad feelings between each other}, \refex{ex:Murad and Rashid push each other.GEN}.

\begin{exe}
	\ex	\label{ex:Murad and Rashid push each other.GEN}
	\gll	Murad=ra Rašid=ra ca-lla calli-j qːurt b-ik'-ul ca-b\\
	Murad=\tsc{add} Rashid=\tsc{add} one\tsc{-gen}	one\tsc{.obl-dat}	push \tsc{hpl-aux-icvb} \tsc{cop-hpl}\\
	\glt	\sqt{Murad and Rashid are pushing each other.}
\end{exe}


Other variants of reciprocal constructions involve the plural reflexive pronouns \refex{ex:Madina and Patimat see each other@14a} and the group numeral form of \tit{ca} \sqt{one}, which is \tit{ca-b-a} \refex{ex:(They) love each other@14b} (\refsec{sec:groupnumerals}). The latter item means \sqt{the ones, some} and therefore \refex{ex:(They) love each other@14b} has, in addition to the reciprocal interpretation, another reading in which one person loves another one, who in turn, loves a third person, and so on, such that there are no reciprocal feelings of love between any of the involved persons.
%
\begin{exe}
		\ex	\label{ex:Madina and Patimat see each other@14a}
		\gll	Madina-j=ra	Pat'imat-li-j=ra	čula	ca-b 	či-b-ig-ul	ca-b\\
			Madina\tsc{-dat=add}	Patimat\tsc{-obl-dat=add}	\tsc{refl.pl.gen}	\tsc{refl-hpl}	\tsc{spr-hpl-}see\tsc{.ipfv-icvb}	\tsc{cop-hpl}\\
		\glt	\sqt{Madina and Patimat see each other.} (E)

		\ex	\label{ex:(They) love each other@14b}
		\gll	ca-b-a-li-j	ca-b-a	b-ičː-aq-u\\
			one\tsc{-hpl-group-obl-dat}	one\tsc{-hpl-group}	\tsc{hpl-}like\tsc{.ipfv-caus-prs.3}\\
		\glt	\sqt{(They\slash Some) love each other.} (E)

\end{exe}

Reciprocal pronouns can also be marked with spatial cases \refex{ex:‎‎They talk to each other what they would do} or be governed by postpositions \refex{ex:and putting them in order one after the other}, \refex{ex:Madina and Ashura talk about each other}.
%
\begin{exe}
	\ex	\label{ex:and putting them in order one after the other}
	\gll	i	po.porjadku	ka-d-irxː-ul	hel-tːi	calli-hara	ca	hitːille\\
		and	in.order	\tsc{down-npl}-put.\tsc{ipfv-icvb}	that\tsc{-pl}	one\tsc{.obl-post}	one	on.back\\
	\glt	\sqt{and putting them in order one after the other}

	\ex	\label{ex:Madina and Ashura talk about each other}
	\gll	Madina=ra	ʡaˁšura=ra	ca	ca-lla	qari=či-b	ʁaj	ka-b-ik'-ul	ca-b\\
		Madina\tsc{=add}	Ashura\tsc{=add}	one	one\tsc{-gen}	on.top=on\tsc{-hpl}	word	\tsc{down-hpl}-say.\tsc{ipfv-icvb}	\tsc{cop-hpl}\\
	\glt	\sqt{Madina and Ashura talk about each other.} (E)
\end{exe}

Syntactically, reciprocal constructions show the same properties as local reflexivization. The c-command
requirement holds. Therefore, possessors cannot bind reciprocal pronouns. For instance, in \refex{ex:Murad and Patimat's parents love each other.} the conjoined possessor noun phrase cannot serve as an antecedent for the reciprocal pronoun, but only the head of the genitive phrase can. The pronouns are interpreted as bound variables and can thus be controlled by non-specific noun phrases \refex{ex:All boys know each other}.
%
\begin{exe}
	\ex	\label{ex:Murad and Patimat's parents love each other.}
	\gll	Murad-la=ra Pat'imat-la=ra bahinte calli-j ca b-ičː-aq-u\\
		Murad\tsc{-gen=add}	Patimat\tsc{-gen=add}	parents	one\tsc{.obl-dat}	one	\tsc{hpl-}love\tsc{.ipfv-caus-prs.3}\\
	\glt	\sqt{Murad and Patimat's parents love each other.} (E)

	\ex	\label{ex:All boys know each other}
	\gll	li<b>il	durħ-n-aˁ-j	calli-j	ca	b-alχ-u\\
		all\tsc{<hpl>}	 boy\tsc{-pl-obl-dat}	one\tsc{.obl-dat}	one	\tsc{hpl-}know\tsc{.ipfv-prs.3}\\
	\glt	\sqt{All boys know each other.} (E)
\end{exe}

As has been shown for complex reflexive pronouns above, the reciprocal pronouns can also occur in the position of the ergative agent controlled by an antecedent that fulfills the role of the absolutive patient. 
%
\begin{exe}
	\ex	\label{ex:‎Murad and Rashid hit each other}
	\begin{xlist}
		\ex	\label{ex:‎Murad and Rashid hit each other@A}
		\gll	Murad-li=ra	Rašid-li=ra	cal-li	ca	b-aˁq-ib\\
			Murad\tsc{-erg=add}	Rashid\tsc{-erg=add}	one\tsc{.obl-erg}	one	\tsc{n-}hit\tsc{.pfv-pret}\\
		\glt	\sqt{‎Murad and Rashid hit each other.} (E)
	
		\ex	\label{ex:‎Murad and Rashid hit each other@B}
		\gll	Murad=ra	Rašid=ra	cal-li	ca	b-aˁq-ib\\
			Murad\tsc{=add}	Rashid\tsc{=add}	one\tsc{.obl-erg}	one	\tsc{n-}hit\tsc{.pfv-pret}\\
		\glt	\sqt{‎Murad and Rashid hit each other.} (E)
	\end{xlist}
\end{exe}

Similarly, experiencers can be expressed by reciprocal pronouns that are bound by absolutive stimuli. In other words, case marking can swap from the standard distribution to the reverse non-standard distribution. Note that in \refex{ex:Musa and Murad know each other} this does not lead to any change in the form of the reciprocal pronoun because this is the case-copying variant and the two cases involved are the same independently of which case appears on the antecendent.
%
\begin{exe}
	\ex	\label{ex:Musa and Murad know each other}
	\begin{xlist}
		\ex	\label{ex:Musa and Murad know each other@A}
		\gll	Musa-j=ra	Murad-li-j=ra	calli-j	ca	b-alχ-u\\
			Musa\tsc{-dat=add}	Murad\tsc{-obl-dat=add}	one\tsc{.obl-dat}	one	\tsc{hpl-}know\tsc{.ipfv-prs.3}\\
		\glt	\sqt{Musa and Murad know each other.} (E)

		\ex	\label{ex:Musa and Murad know each other@B}
		\gll	Musa=ra	Murad=ra	calli-j	ca	b-alχ-u\\
			Musa\tsc{=add}	Murad\tsc{=add}	one\tsc{.obl-dat}	one	\tsc{hpl-}know\tsc{.ipfv-prs.3}\\
		\glt	\sqt{Musa and Murad know each other.} (E)
	\end{xlist}
\end{exe}

In fact, it seems that the reversed case marking pattern is sometimes preferred with affective constructions. Thus, the standard case marking has been rejected or judged as very marginal for a similar clause with the same type of reciprocal pronoun \refex{ex:Madina and Patimat see each other uncertain@14}.
%
\begin{exe}
	\ex	\label{ex:Madina and Patimat see each other uncertain@14}
	\gll	{??}	Madina-j=ra	Pat'imat-li-j=ra	calli-j	ca	či-b-ig-ul	ca-b\\
		{}	Madina\tsc{-dat=add}	Patimat\tsc{-obl-dat=add}	one\tsc{.obl-dat}	one	\tsc{spr-hpl-}see\tsc{.ipfv-icvb}	\tsc{cop-hpl}\\
	\glt	(Intended meaning: \sqt{Madina and Patimat see each other.}) (E)
\end{exe}

Transitive verbs and affective verbs are the only valency types that permit the cases to be switched around. As we have noticed for complex reflexive pronouns, swapping of case marking is ungrammatical for extended intransitive verbs \refex{ex:They look at each other uncertain@B}.
%
\begin{exe}
	\ex	\label{ex:They look at each other}
	\begin{xlist}
		\ex	\label{ex:They look at each other@A}
		\gll	itːi	calli-j ca er či-b-ik'-u\\
			\tsc{3pl}	one\tsc{.obl-dat}	one	look	\tsc{spr-hpl-}say\tsc{.ipfv-prs.3}\\
		\glt	\sqt{They look at each other.} (E)

		\ex	\label{ex:They look at each other uncertain@B}
		\gll	{*} itː-a-j	calli-j ca er či-b-ik'-u\\
			{} \tsc{3pl-obl-dat}	one\tsc{.obl-dat}	one	look	\tsc{spr-hpl-}say\tsc{.ipfv-prs.3}\\
		\glt	(Intended meaning: \sqt{They look at each other.}) (E)
	\end{xlist}
\end{exe}

Again, there is some freedom concerning the word order both with the standard case marking pattern and when the cases have been switched around. Nevertheless, there are word orders that are forbidden, most notably when the pronoun is split apart and the part that copies the case precedes its antecedent from which the case has been copied. More generally, complex reciprocal pronouns, just like complex reflexive pronouns, cannot be split into two parts, and none of the individual parts could be interpreted as fulfilling another function (e.g. as emphatic particle or intensifier or as pause filler). Thus, they must occur next to each other as one constituent.
%
\begin{exe}
	\ex	standard case marking	\label{ex:Madina and Dinara (regularly) praised each other}
	\begin{xlist}
		\ex	\label{ex:Madina and Dinara (regularly) praised each other@A}
		\gll	Madina-l=ra	Dinara-l=ra	[cal-li	ca]	gap	b-irq'-i\\
			Madina\tsc{-erg=add}	Dinara\tsc{-erg=add}	one\tsc{.obl-erg}	one	praise	\tsc{hpl-}do\tsc{.ipfv-hab.pst.3}\\
		\glt	\sqt{Madina and Dinara (regularly) praised each other.} (E)

		\ex	\tit{[calli ca] Madinalra Dinaralra gap birq'i}	\label{ex:Madina and Dinara (regularly) praised each other@B}

		\ex	\tit{*calli Madinalra Dinaralra gap birq'i ca}	\label{ex:Madina and Dinara (regularly) praised each other ungrammatical@C}
	\end{xlist}
\end{exe}

Note that again the case marking of the reciprocal pronouns is identical for the standard patterns as well as for the reversed pattern as is obvious when comparing examples above with the following sentences \refex{ex:Madina and Dinara (regularly) praised each other2}.
%
\begin{exe}
	\ex	reversed case marking	\label{ex:Madina and Dinara (regularly) praised each other2}
	\begin{xlist}
		\ex	\label{ex:Madina and Dinara (regularly) praised each other@A2}
		\gll	Madina=ra	Dinara=ra	cal-li	ca	gap	b-irq'-i\\
			Madina\tsc{=add}	Dinara\tsc{=add}	one\tsc{.obl-erg}	one	praise	\tsc{hpl-}do\tsc{.ipfv-hab.pst.3}\\
		\glt	\sqt{Madina and Dinara (regularly) praised each other.} (E)

		\ex	\tit{Madinara Dinarara [ca calli] gap birq'i}	\label{ex:Madina and Dinara (regularly) praised each other@B2}

		\ex	\tit{[ca calli] Madinara Dinarara gap birq'i}	\label{ex:Madina and Dinara (regularly) praised each other@C2}
	\end{xlist}
\end{exe}

Finally, reciprocal pronouns can only have antecedents within the same clause. For instance, in \refex{ex:‎The parents want that Patimat and Murad help each other} the pronoun is bound by the conjoined noun phrase \sqt{Patimat and Murad} and cannot be controlled by the compound noun \tit{atːa-aba} \sqt{parents} in the higher clause. The pronoun consists of a part in the dative in accordance with its function in the clause, and a first part that either copies the case of the controller \refex{ex:‎The parents want that Patimat and Murad help each other@A} or occurs in the absolutive \refex{ex:‎The parents want that Patimat and Murad help each other@B}.
%
\begin{exe}
	\ex	\label{ex:‎The parents want that Patimat and Murad help each other}
	\begin{xlist}
		\ex	\label{ex:‎The parents want that Patimat and Murad help each other@A}
		\gll	atːa	aba-j	b-ikː-ul	ca-b	[Pat'imat-li=ra	Murad-li=ra	cal-li	calli-j	kumek	b-arq'-ib-le]\\
			father	mother\tsc{-dat}	\tsc{n-}want\tsc{.ipfv-icvb}	\tsc{cop-n}	Patimat\tsc{-erg=add}	Murad\tsc{-erg=add}	one\tsc{.obl-erg}	one\tsc{.obl-dat}	help	\tsc{n-}do\tsc{.pfv-pret-cvb}\\
		\glt	\sqt{‎The parents want that Patimat and Murad help each other.}

		\ex	\label{ex:‎The parents want that Patimat and Murad help each other@B}
		\gll	atːa	aba-j	b-ikː-ul	ca-b	[Pat'imat-li=ra	Murad-li=ra	ca	calli-j	kumek	b-arq'-ib-le]\\
			father	mother\tsc{-dat}	\tsc{n-}want\tsc{.ipfv-icvb}	\tsc{cop-n}	Patimat\tsc{-erg=add}	Murad\tsc{-erg=add}	one	one\tsc{.obl-dat}	help	\tsc{n-}do\tsc{.pfv-pret-cvb}\\
		\glt	\sqt{‎The parents want that Patimat and Murad help each other.}
	\end{xlist}
\end{exe}
