\chapter{Relative clauses}
\label{cpt:Relative clauses}


%%%%%%%%%%%%%%%%%%%%%%%%%%%%%%%%%%%%%%%%%%%%%%%%%%%%%%%%%%%%%%%%%%%%%%%%%%%%%%%%

\section{Introduction}
\label{sec:relativeclausesintro}

Sanzhi Dargwa uses \isi{participles} for the formation of \isi{relative clauses}. Like other modifiers, \isi{relative clauses} normally precede the head. There are two simple \isi{participles} that consist of a suffix added to the verbal stem, and complex \isi{participles} that make use of additional suffixes. The simple \isi{participles} are the \isi{preterite participle} (\refsec{sssec:The preterite participle}), which is identical in form to the preterite, and the \isi{modal participle} -\textit{an} (\refsec{sssec:The modal participle -an}). For verbs whose stems exhibit the aspectual distinction, the \isi{preterite participle} is almost exclusively used with perfective stems, whereas the \isi{modal participle} occurs only with imperfective stems. The complex \isi{participles} consist of the simple \isi{participles} plus the \isi{cross-categorical suffixes} \tit{-ce} or suffix \tit{-il} (\refsec{sssec:The attributive markers -il and -ce / -te in combination with the participles}). Furthermore, Sanzhi has a \isi{locative participle} that is used when the head of the \isi{relative clause} denotes a location (\refsec{sssec:The locative participle}).

The simple \isi{participles} and the complex \isi{participles} derived from them express temporal relations. The \isi{modal participle} occurs in \isi{relative clauses} with non-past time reference (e.g. present, future, habitual) \refex{ex:The girl who is laughing is my sister}; the \isi{preterite participle} occurs in \isi{relative clauses} with past time reference \refex{ex:The girl who was laughing is my sister}.
%
\begin{exe}
	\ex	\label{ex:The girl who is laughing is my sister}
	\gll	[ħaˁħaˁ	r-ik'ʷ-an]	rursːi	di-la	rucːi	ca-r\\
		laughter	\tsc{f-}say\tsc{.ipfv-ptcp}	girl	\tsc{1sg-gen}	sister	\tsc{cop-f}\\
	\glt	\sqt{The girl who is laughing is my sister.} (E)
	
	\ex	\label{ex:The girl who was laughing is my sister}
	\gll	[ħaˁħaˁ	r-ik'-ub]	rursːi	di-la	rucːi	ca-r\\
		laughter	\tsc{f-}say\tsc{.ipfv-pret}	girl	\tsc{1sg-gen}	sister	\tsc{cop-f}\\
	\glt	\sqt{The girl who was laughing is my sister.} (E)

\end{exe}

Since the \isi{preterite participle} is identical to the preterite itself, in a \isi{number} of cases two interpretations are possible: a main clause that precedes another clause with argument sharing between the two clauses or a \isi{relative clause}:
%
\begin{exe}
	\ex	
	\gll	raχːaz	b-ertː-ib	χːʷe	sa-r-b-uq-un\\
		chain	\tsc{n-}tear\tsc{.pfv-pret}	dog	\tsc{ante-abl-n-}go\tsc{.pfv-pret}\\
	\glt	\sqt{‎‎‎The dog tore off the chain and left.} OR \sqt{The dog who tore off the chain left.} (E)
\end{exe}

The suffixes \tit{-ce} (plural \tit{-te}) and \tit{-il} are used to form attributes that can denote referents. This means that items that bear these suffixes can be used as modifiers in noun phrases but also as predicates or as nominals. With respect to \isi{relative clauses}, they are used whenever the \isi{relative clause} occurs in a position that diverges from its canonical prenominal position. See \refsec{sec:Other syntactic properties of relative clauses} below for a discussion.

Moreover, purpose clauses with nominal heads are structurally similar to \isi{relative clauses}, but must contain an \isi{infinitive} (or \isi{subjunctive}) and can also be marked with \textit{-ce} \refex{ex:There is the thought (i.e. I have the thought) to go home}, \refex{ex:‎He is also there (something) to eat that has been placed in front of him for him}. Participles are not allowed if the clause has a purposive meaning.
%
\begin{exe}
	\ex	\label{ex:There is the thought (i.e. I have the thought) to go home}
	\gll	[du	qili	uq'-ij]	pikri	le-b\\
		\tsc{1sg}	home	go\tsc{.m-inf}	thought	exist\tsc{-n} \\
	\glt	\sqt{There is the thought to go home.} (i.e. I have the thought) (E)

	\ex	\label{ex:‎He is also there (something) to eat that has been placed in front of him for him}
	\gll	sa-ka-b-išː-ib-le,	[cini-j	b-erkʷ-ij-ce]=ra	χe-b	ca-w=ra\\
		\tsc{ante-down}\tsc{-n-}put\tsc{.pfv-pret-cvb}	\tsc{refl.sg.obl-dat}	\tsc{n-}eat\tsc{.pfv-inf-dd.sg=add}	exist.\tsc{down-n}	\tsc{refl-m=add}\\
	\glt	\sqt{‎He is there and (something) to eat for him, which has been placed in front (of him).}
\end{exe}

In the following, I will first show which positions can be relativized (\refsec{sec:Positions that can be relativized}), then analyze further semantic and syntactic properties of \isi{relative clauses} (\refsec{sec:Other syntactic properties of relative clauses}), and then briefly discuss headless \isi{relative clauses} (\refsec{sec:Headless relative clauses}). 


%%%%%%%%%%%%%%%%%%%%%%%%%%%%%%%%%%%%%%%%%%%%%%%%%%%%%%%%%%%%%%%%%%%%%%%%%%%%%%%%

\section{Positions that can be relativized}
\label{sec:Positions that can be relativized}

The following examples illustrate the positions that can be relativized. They are labeled with the semantic roles and the cases that nominals in that role bear in a main clause.
%
\begin{enumerate}
	\item	single argument of an \isi{intransitive verb} (\isi{absolutive})
	%
	\begin{exe}
		\ex	\label{ex:the boys who went slowly}
		\gll	il-tːi	[bahla-l	ag-ur]	durħ-ne\\
			that-\tsc{pl}	slow\tsc{-advz}	go\tsc{.pfv-pret}	boy\tsc{-pl}\\
		\glt	\sqt{the boys who went slowly}

		\ex	\label{ex:‎‎‎In the water that boiled in the pot, she died}
		\gll	[ħaˁšukː-a-r	rurq-aˁn]	hin-ni-cːe-r	il	r-ebč'-ib	ca-r	\\
			pot\tsc{-loc-f}	boil\tsc{-ptcp}	water\tsc{-obl-in}-\tsc{f}	that	\tsc{f-}die\tsc{.pfv-pret}	\tsc{cop-f}	\\
		\glt	\sqt{‎‎‎In the water that boiled in the pot, she died.}
	
		\ex	\label{ex:‎Yesterday I went to the daughter (baby girl) of the woman who gave birth}
		\gll	sːa	hextːu	[r-emkː-un]	xːunul-la	rursːi-šːu	ag-ur-re=de\\
			yesterday	there.\tsc{up}	\tsc{f-}give.birth\tsc{.pfv-pret}	woman\tsc{-gen}	girl\tsc{-ad}	go\tsc{.pfv-pret-cvb=pst}\\
		\glt	\sqt{‎Yesterday I went to the daughter (baby girl) of the woman who gave birth.}
	\end{exe}

	\item	\isi{agent} of \isi{transitive verb} (\isi{ergative})
	%
	\begin{exe}
		\ex	\label{ex:There are those, the men who not take away their hands from their wives (i.e. who beat them constantly)}
		\gll	ca	ca	le-b	[xːun-re-ja-r	naˁq	či-r-a-ha-jsː-an]	murg-le\\
			one	one	exist\tsc{-hpl} woman\tsc{-pl-loc-abl}	hand	\tsc{spr-abl-neg-up}-take\tsc{.pfv-ptcp}	man\tsc{-pl}\\
		\glt	\sqt{There are those, the men who do not take away their hands from their wives.} (i.e. who beat them constantly)

		\ex	\label{ex:the washing machine (lit. the machine that washes clothes)}
		\gll	hež	[paltar	ic-an]	mašin\\
			this	clothes	wash\tsc{.ipfv-ptcp}	machine\\
		\glt	\sqt{the washing machine} (lit. the machine that washes clothes)
	\end{exe}

	\item	\isi{experiencer} of bivalent \isi{affective verb} (\isi{dative} or \isi{ergative})
	%
	\begin{exe}
		\ex	\label{ex:the girl who knows me}
		\gll	[du	w-alχ-an]	rursːi\\
			\tsc{1sg}	\tsc{m-}know\tsc{.ipfv-ptcp}	girl\\
		\glt	\sqt{the girl who knows me (masc.)} (E)

		\ex	\label{ex:‎They look like husband and wife who very well understand each other}
		\gll	[ħaˁq'-le	qːuʁa-l	ca-l-li	ca	urk'i	ha-b-eʁ-ib]	sub-xːunul-li-j	miši-l	ca-b	heštːi\\
			very\tsc{-advz}	beautiful\tsc{-advz}	one\tsc{-obl-erg}	one	heart	\tsc{up-n-}go\tsc{.pfv-pret}	husband-woman\tsc{-obl-dat}	similar\tsc{-advz}	\tsc{cop-hpl}	these\\
		\glt	\sqt{‎They look like husband and wife who very well understand each other.}
		
				\ex	\label{ex:to those people who you do not love}
		\gll	il-tːi	[a-jkː-an]	admi-li-j\\
			that\tsc{-pl}	\tsc{neg-}want\tsc{.m.ipfv-ptcp}	person\tsc{-obl-dat}\\
		\glt	\sqt{to those people who do not love you (masc.)}
		
	\end{exe}

	\item	\isi{agent} in the \isi{antipassive} construction (\isi{absolutive})
	%
	\begin{exe}
		\ex	\label{ex:‎They are (hard)-working people}
		\gll	[ʡaˁči-l	b-irq'-an]	adim-te	ca-b	hel-tːi	\\
			work\tsc{-erg}	\tsc{hpl-}do\tsc{.ipfv-ptcp}	person\tsc{-pl}	\tsc{cop-hpl}	that\tsc{-pl}	\\
		\glt	\sqt{‎They are (hard)-working people.}

		\ex	\label{ex:the friends who are drinking}
		\gll	[deč-li	b-učː-an]	juldašː-e\\
			drinking\tsc{-erg}	\tsc{hpl-}drink\tsc{.ipfv-ptcp}	friend\tsc{-pl}\\
		\glt	\sqt{the friends who are drinking}
	\end{exe}

	\item	\isi{patient} of transitive (\isi{absolutive})
	%
	\begin{exe}
		\ex	
		\gll	[iž-i-l	d-alc'-un]	q'ar	le-d=de\\
			this\tsc{-obl-erg}	\tsc{npl-}gather\tsc{.pfv-pret}	plant\tsc{npl}	exist\tsc{-pl=pst}\\
		\glt	\sqt{‎‎There were plants that she gathered.}

		\ex	\label{ex:‎‎‎a knife that was not necessary to be sharpened}
		\gll	[umqːa-l	b-irq'-an]	dis	\\
			whetstone\tsc{-erg}	\tsc{n-}do\tsc{.pfv-ptcp}	knife	\\
		\glt	\sqt{‎‎‎a knife that has been sharpened} (E) 
	\end{exe}

	\item	\isi{theme} of ditransitive verb (\isi{absolutive})
	%
	\begin{exe}
		\ex	\label{ex:the food that is given to their dogs}
		\gll	[ču-la	χu-d-a-j	lukː-an]	χurejg\\
			\tsc{refl.pl-gen}	dog\tsc{-pl-obl-dat}	give\tsc{.ipfv-ptcp}	food\\
		\glt	\sqt{the food that is given to their dogs}
	\end{exe}

	\item	\isi{stimulus} of bivalent \isi{affective verb} (\isi{absolutive})
	%
	\begin{exe}

		\ex	\label{ex:in a town that you do not know}
		\gll	[a-b-alχ-an]	šahar-ri-cːe-w\\
			\tsc{neg-n-}know\tsc{.ipfv-ptcp}	town\tsc{-obl-in-m}\\
		\glt	\sqt{in a town that (you) do not know} (E)
	\end{exe}

	\item	\isi{recipient} of ditransitive verb (\isi{dative})
	%
	\begin{exe}
		\ex	\label{ex:the boy to whom I gave the book}
		\gll	[du-l	kiniga	b-ičː-ib]	durħuˁ\\
			\tsc{1sg-erg}	book	\tsc{n-}give\tsc{.pfv-pret}	boy\\
		\glt	\sqt{the boy to whom I gave the book} (E)
	\end{exe}

	\item	\isi{goal} of extended \isi{intransitive verb} (\isi{dative})
	%
	\begin{exe}
		\ex	\label{ex:the girl at whom I looked}
		\gll	[du	er=či	w-erč'-ib-il]	rursːi\\
			\tsc{1sg}	look=on	\tsc{m-}look\tsc{.pfv-pret-ref}	girl\\
		\glt	\sqt{the girl at whom I looked} (E)

	\end{exe}

	\item	beneficiary (\isi{dative})
	%
	\begin{exe}
		\ex	\label{ex:The son for whom father built the house got happy}
		\gll	[atːa-l	qal	b-arq'-ib-il]	durħuˁ	razi	Ø-iχ-ub\\
			father\tsc{-erg}	house	\tsc{n-}do\tsc{.pfv-pret-ref}	boy	happy	\tsc{m-}become\tsc{.pfv-pret}\\
		\glt	\sqt{The son for whom father built the house got happy.} (E)

	\end{exe}

	\item	spatial location (essive cases)
	%
	\begin{exe}
		\ex	\label{ex:‎‎‎One time we went there through the canyon where the bear had died}
		\gll	caj-na	arg-ul	le-d=da	[sːika	b-ebč'-ib]	qːatːa-r	či-d-a\\
			one-\tsc{time}	go\tsc{.ipfv-icvb}	exist\tsc{-1/2pl=1}	bear	\tsc{n-}die\tsc{.pfv-pret}	canyon\tsc{.loc-abl}	on\tsc{-1/2pl-dir}\\
		\glt	\sqt{‎‎‎One time we went there through the canyon where the bear had died.}

	\ex	\label{ex:‎‎‎The boys went to the place where cows are slaughteredA}
	\gll	durħ-ne	ag-ur	[q'ʷal	luχ-na]	musːa\\
		boy\tsc{-pl}	go\tsc{.pfv-pret}	cow	cut\tsc{.ipfv-ptcp.loc}	place.\tsc{spr}\\
	\glt	\sqt{‎‎‎The boys went to the place where cows are slaughtered.} (this refers to a specific place in Sanzhi)
	
		\ex	\label{ex:‎across the place where you go to the pakh-pakh (place name)}
		\gll	[paˁχ.paˁχ-le	či-w-ax-an]	musːa-r\\
			pakh.pakh\tsc{-loc}	\tsc{spr-m-}go\tsc{-ptcp}	place\tsc{.loc-abl}	\\
		\glt	\sqt{‎across the place where (you) go to the Pakh-Pakh (place name)}
	\end{exe}



	\item	spatial \isi{goal} (lative cases)
	%
	\begin{exe}
		\ex	\label{ex:‎You did not find a place that I did not reach.}
		\gll	[du	a-r-it-eʁ-ib]	musːa	b-a-b-určː-i\\
			\tsc{1sg}	\tsc{neg-f-thither}-go\tsc{.pfv-pret}	place	\tsc{n-neg-n-}find\tsc{.ipfv-hab.pst.3}\\
		\glt	\sqt{‎You did not find a place that I (fem.) did not reach.} (E)
	\end{exe}

	\item	spatial source (\isi{ablative} cases)
	%
	\begin{exe}
		\ex	\label{ex:the town from which I came is far away}
		\gll	[du	sa-jʁ-ib]	šahar	haraq-le	ca-b\\
			\tsc{1sg}	\tsc{hither}-come\tsc{.m.pfv-pret}	town	far\tsc{-advz}	\tsc{cop-n}\\
		\glt	\sqt{the town from which I (masc.) came is far away.} (E)
	\end{exe}

	\item	source of emotion (\tsc{in}-\isi{ablative})
	%
	\begin{exe}
		\ex	\label{ex:the boy that I am afraid of}
		\gll	[du	uruχ	Ø-ik'ʷ-an]	durħuˁ\\
			\tsc{1sg}	fear	\tsc{m-}say\tsc{.ipfv-ptcp}	boy\\
		\glt	\sqt{the boy that I am afraid of} (E)
	\end{exe}

	\item	cause\slash source (\tsc{in}-\isi{ablative})
	%
	\begin{exe}
		\ex	\label{ex:Give me the choclates from which the people do not get fat}
		\gll	d-iqː-a	dam	[χalq'	c'erx	a-b-irχʷ-an]	q'ampit'-e!\\
			\tsc{npl-}carry\tsc{.ipfv-imp}	\tsc{1sg.dat}	people	fat	\tsc{neg-hpl-}become\tsc{.ipfv-ptcp}	chocolates\tsc{-pl}\\
		\glt	\sqt{Give me the chocolates from which the people do not get fat!} (E)
	\end{exe}

	\item	topic of conversation or addressee (\tsc{in}-\isi{ablative} for addressee or complement of postposition for topic of conversation)
	%
	\begin{exe}
		\ex	\label{ex:The woman about whom Madina talked is my sister}
		\gll	[Madina-l	χabar	b-urs-ib-il]	xːunul	di-la	rucːi	ca-r\\
			Madina\tsc{-erg}	story	\tsc{n-}tell\tsc{-pret-ref}	woman	\tsc{1sg-gen}	sister	\tsc{cop-f}\\
		\glt	\sqt{The woman about whom Madina talked is my sister.} OR \sqt{The woman to whom Madina talked is my sister.} (E)
				
		\ex	\label{ex:‎‎‎the woman whom Madina scolds}
		\gll	[Aminat	ʁaj=či-r	ka-r-ik'ʷ-an]	xːunul\\
			Aminat	word=on\tsc{-abl}	\tsc{down-f-}say\tsc{.ipfv-ptcp}	woman\\
		\glt	\sqt{‎‎‎the woman about Aminat talks (gossips)} (E) 
		
		
	\end{exe}

	\item	\isi{comitative}
	%
	\begin{exe}
		\ex	\label{ex:the boys with whom I play}
		\gll	[du	ħaˁz-t-a-l	Ø-irq'-an]	durħ-ne\\
			\tsc{1sg}	game\tsc{-pl-obl-erg}	\tsc{m-}do\tsc{.ipfv-ptcp}	boy\tsc{-pl}\\
		\glt	\sqt{the boys with whom I (masc.) play} (E)
	\end{exe}

	\item	possessor (\isi{genitive})\footnote{If in \refex{ex:the woman whose husband died is crying} the simple instead of the complex \isi{participle} is used, the first clause is interpreted as a main clause preceding another main clause (\sqt{The husband died. The wife is crying.}).}
	%
	\begin{exe}
		\ex	\label{ex:the woman whose husband died is crying}
		\gll	[sub	w-ebč'-ib-il]	xːunul	r-isː-ul	ca-r\\
			husband	\tsc{m-}die\tsc{.pfv-pret-ref}	woman	\tsc{f-}cry\tsc{-icvb}	\tsc{cop-f}\\
		\glt	\sqt{the woman whose husband died is crying} (E)
	\end{exe}

	\item	instrument (\isi{comitative} or \isi{ergative})
	%
	\begin{exe}
		\ex	\label{ex:The knife with which I cut the bread is blunt}
		\gll	[du-l	t'ult'	ka-b-ičː-ib-il]	dus	qːut'a-l	ca-b\\
			\tsc{1sg-erg}	bread	\tsc{down-n-}cut.up\tsc{.pfv-pret-ref}	knife	blunt\tsc{-advz}	\tsc{cop-n}\\
		\glt	\sqt{The knife with which I cut the bread is blunt.} (E)
	\end{exe}	
\end{enumerate}

Complements of postposition heading \isi{relative clauses} are not found in the corpus and somewhat hard to elicit, but \refex{ex:The house in front of which the boy is standing is ours} shows an example. Another one is \refex{ex:The woman about whom Madina talked is my sister} above in the interpretation in which the head of the \isi{relative clause} is the topic of conversation (which is normally expressed by a postposition, see \refsec{ssec:postposition qari}).
%
\begin{exe}
	\ex	\label{ex:The house in front of which the boy is standing is ours}
	\gll	[durħuˁ	sala	sa-ka-jcː-ur-il]	qal	nišːa-lla	ca-b\\
		boy	front	\tsc{hither-down}-stand\tsc{.m.pfv-pret-ref}	house	\tsc{1pl-gen}	\tsc{cop-n}\\
	\glt	\sqt{The house in front of which the boy is standing is ours.} (E)
\end{exe}

More complex constructions are also possible. For instance, the argument of a complement clause can function as the head of a \isi{relative clause} \refex{ex:the book that I (fem.) started to read}. Similarly, arguments of \isi{adverbial clauses} can be extracted in order to serve as heads of \isi{relative clauses} \refex{ex:the boy who broke the glass after I gave it to him}. We can have multiple \isi{relative clauses} embedded into each other \refex{ex:the boy who know the girl who lives in Terkeme}. 
%
\begin{exe}
	\ex	\label{ex:the book that I (fem.) started to read}
	\gll	[du	[b-elč'-ij]	r-aʔ	r-išː-ib-il]	kiniga\\
		\tsc{1sg}	\tsc{n-}read\tsc{.pfv-inf}	\tsc{f-}begin	\tsc{f-}put\tsc{.pfv-pret-ref}	book\\
	\glt	\sqt{the book that I (fem.) started to read} (E)

	\ex	\label{ex:the boy who broke the glass after I gave it to him}
	\gll	[[du-l	istikan	b-ičː-ib-le]	b-elq'-aq-un-il]	durħuˁ\\
		\tsc{1sg-erg}	glass	\tsc{n-}give\tsc{.pfv-pret-cvb}	\tsc{n-}break\tsc{.pfv-caus-pret-ref}	boy\\
	\glt	\sqt{the boy who broke the glass after I gave it to him} (E)

	\ex	\label{ex:the boy who know the girl who lives in Terkeme}
	\gll	[[Tarkama-le-r	er	r-irχʷ-an]	rursːi	r-alχ-an]	durħuˁ\\
		Terkeme\tsc{-loc-f}	life	\tsc{f-}be\tsc{.ipfv-ptcp}	girl	\tsc{f-}know\tsc{.ipfv-ptcp}	boy\\
	\glt	\sqt{the boy who knows the girl who lives in Terkeme} (E)
\end{exe}

It is possible to find examples in which the head noun does not bear any syntactic relation to the \isi{relative clause}, i.e., in which it is impossible to argue that the \isi{relative clause} contains a gap from the extracted head. This is widespread in case of head \isi{nouns} with a very broad semantics such as \tit{zamana} \sqt{time} but there are also sentences with other head \isi{nouns}. For instance, \refex{ex:the boy called Mahammadhazhi} illustrates a common construction that explicates the name of a person. The verb \tit{b-ik'ʷ-} `say, call', that is used in the \isi{relative clause}, normally requires an \isi{absolutive} argument that denotes its subject. However, in \refex{ex:the boy called Mahammadhazhi} the subject is absent because it is an impersonal construction, retrievable only from the context and from the fact that the human plural \isi{gender} prefix is used. The \isi{gender} \isi{agreement prefix} is frozen and cannot be replaced by any other prefix. Instead of a complement clause, which is usually used together with the verb \tit{b-ik'ʷ-} \sqt{say, call}, the name \tit{Maħaˁmmadħaˁži} occurs. The head noun \textit{durħuˁ} `boy' does not fulfill an argument or adjunct role in the \isi{relative clause}. This issue is discussed further in the following \refsec{sec:Other syntactic properties of relative clauses}. Example \refex{ex:There was this game of ours, the hide-and-seek} is similar in that the head noun \textit{ħaˁz} `game' is also not in a syntactic relationship with the verb in the \isi{relative clause} `hide'.
%
\begin{exe}
	\ex	\label{ex:the boy called Mahammadhazhi}
	\gll	[Maħaˁmmadħaˁži	b-ik'ʷ-an]	durħuˁ=ra\\
		Mahammadhazhi	\tsc{hpl-}say\tsc{.ipfv-ptcp}	boy\tsc{=add}\\
	\glt	\sqt{the boy called Mahammadhazhi}

	\ex	\label{ex:There was this game of ours, the hide-and-seek}
	\gll	[daˁʡaˁna	b-irx-an]	ħaˁz	b-irχ-i	nišːa-la\\
		secret	\tsc{hpl-}become\tsc{.ipfv-ptcp}	game	\tsc{n-}be\tsc{.ipfv-hab.pst}	\tsc{1pl-gen}\\
	\glt	\sqt{There was this game of ours, the hide-and-seek.} (lit. the game where one had to hide)
\end{exe}


%%%%%%%%%%%%%%%%%%%%%%%%%%%%%%%%%%%%%%%%%%%%%%%%%%%%%%%%%%%%%%%%%%%%%%%%%%%%%%%%

\section{Other syntactic properties of relative clauses}
\label{sec:Other syntactic properties of relative clauses}

Relative clauses can have a nominal head or be headless (see \refsec{sec:Headless relative clauses} below). The head is normally a common noun, but it can also be a personal pronoun, an \isi{indefinite pronoun}, a \isi{demonstrative pronoun}, or a personal name. Thus, \isi{relative clauses} can be restrictive or non-restrictive without any difference in their morphosyntactic form.
%
\begin{enumerate}
	\item	head is a third person pronoun\slash \isi{demonstrative pronoun}
	%
	\begin{exe}
		\ex	\label{ex:‎Well, you will say what you see}
		\gll	nu	[ašːi-j	či-b-ig-an]	hel	b-urs-an	ca-b\\
			well	\tsc{2pl-dat} \tsc{spr-n-}see\tsc{.ipfv-ptcp}	that	\tsc{n-}tell\tsc{-ptcp}	\tsc{cop-n}\\
		\glt	\sqt{‎Well, you (pl.) will say what you see.}

		\ex	\label{ex:‎Are not these who wrestle}
		\gll	heštːi	akːu=w	heštːi	[b-iħ-ib-te]?\\
			these	\tsc{cop.neg=q}	these	\tsc{hpl-}wrestle\tsc{.pfv-pret-dd.pl}\\
		\glt	\sqt{‎Are these not the ones who wrestle?}
	\end{exe}

	\item	head is a personal pronoun
	%
	\begin{exe}
		\ex	\label{ex:I who went to Russia made money}
		\gll	[Rassija-le	ag-ur-il]	du-l	arc d-irq'-ul=de\\
			Russia\tsc{-loc}	go\tsc{.pfv-pret-ref}	\tsc{1sg-erg}	money	\tsc{npl-}do\tsc{.ipfv-icvb=pst}\\
		\glt	\sqt{I, who went to Russia, made money.} (i.e. earned money) (E)
	\end{exe}

	\item	head is an interrogative pronoun (used as \isi{indefinite pronoun})
	%
	\begin{exe}
		\ex	\label{ex:‎The ones who were thrown out, who did you say was this}
		\gll	[t'ut'u	b-arq'-ib-te]	ča-qal	Ø-ik'ʷ-a-tːe?\\
			throw.out	\tsc{hpl-}do\tsc{.pfv-pret-dd.pl} who\tsc{-assoc}	\tsc{m-}say\tsc{.ipfv-hab.pst-2sg}\\
		\glt	\sqt{‎The ones who were thrown out, who did you say this was?}
	\end{exe}

	\item	head is an \isi{indefinite pronoun} \refex{ex:I will not beat anybody who told the truth}, including pronouns used as \isi{nouns} with a light semantics \refex{ex:‎‎‎Now there are other things that appeared (lit. went out)}
	%
	\begin{exe}
		\ex	\label{ex:I will not beat anybody who told the truth}
		\gll	du-l	[mar	haʔ-ib-il]	ča-k'al	a-jt-an=da\\
			\tsc{1sg-erg}	truth	say\tsc{.pfv-pret-ref}	who\tsc{-indef}	\tsc{neg-}beat.up\tsc{-ptcp=1}\\
		\glt	\sqt{I will not beat anybody who told the truth.} (E)
	\end{exe}

	\item	head is a personal name
	%
	\begin{exe}
		\ex	\label{ex:My brother Zamir, who married last year, lives in Ogni}
		\gll	[irig	xːunul	ka-r-iž-ib-il]	di-la	ucːi	Zamir	er	Ø-irχ-u	Agni-le-w\\
			last.year	woman	\tsc{down-f-}be\tsc{.pfv-pret-ref}	\tsc{1sg-gen}	brother	Zamir	life	\tsc{m-}be\tsc{.ipfv-prs}	Ogni-\tsc{loc-m}\\
		\glt	\sqt{My brother Zamir, who married last year, lives in Ogni.} (E)
	\end{exe}
\end{enumerate}

The verbal categories expressed in \isi{relative clauses} are fewer than those expressed in main clauses. Due to the \isi{participles} employed, the expression of tense is possible to a certain degree \refex{ex:The girl who is laughing is my sister}, \refex{ex:The girl who was laughing is my sister} and \isi{negation} is available \refex{ex:I saw the details that I did not see for 25 years}, but \isi{person agreement} and the marking of illocutionary force are excluded. 

Within the \isi{noun phrase}, to which a \isi{relative clause} belongs, the \isi{relative clauses} can be preceded and followed by other nominal modifiers such as \isi{demonstrative pronouns} \refex{ex:‎Are not these who wrestle}, possessive pronouns \refex{ex:My brother Zamir, who married last year, lives in Ogni}, quantifiers, \isi{adjectives}, and others. See \refsec{sec:Noun phrases} for further information on the structure of noun phrases. They can be modified by adverbials just like \isi{adjectives} \refex{ex:‎a very fitting (= good) place}.
%
\begin{exe}
	\ex	\label{ex:‎a very fitting (= good) place}
	\gll	[c'aq'-le	b-al	b-ič-ib]	musːa	\\
		very\tsc{-advz}	\tsc{n-}fit	\tsc{n-}occur\tsc{.pfv-pret}	place\\
	\glt	\sqt{‎a very fitting (= good) place}
\end{exe}

In elicitation, Sanzhi speakers report that there is no difference in the meaning between a \isi{relative clause} with a simple \isi{participle} and a \isi{relative clause} with a \isi{participle} to which one of the the suffixes \tit{-ce} or \tit{-il} is added. Relative clauses with complex \isi{participles} in the default position preceding the nominal head are relatively rare in the Sanzhi corpus. The following three sentences \refexrange{ex:‎How do we call in our language the back that hangs behind his back?}{ex:The one that was in the hand had been put down} illustrate the use of both suffixes with the \isi{preterite participle}. In \refex{ex:‎‎‎Now there are other things that appeared (lit. went out)} the speaker is talking about the tools with which Sanzhi women used to cook and compares them with new appliances. Example \refex{ex:The one that was in the hand had been put down} shows that the \isi{cross-categorical suffixes} are also added to other parts of speech. In this sentence, -\textit{il} appears on the \isi{preterite participle} which because of the suffix acquires nominal properties can can function as referent with the meaning `the one that has been put down'. The second appearance on the noun that is inflected for the \tsc{in}-essive case makes a referent with the meaning `the one that is in the hand' out of a spatial adverbial
%
\begin{exe}
	\ex	\label{ex:‎How do we call in our language the back that hangs behind his back?}
	\gll	ce	b-ik'-u=ja	nišːa-la	ʁaj-la,	hež	[hitːi	kemq-un-il]	q'ucur=uw?\\
		what	\tsc{n-}say\tsc{.ipfv-prs=q}	\tsc{1pl-gen}	language\tsc{-gen}	this	behind	hang\tsc{-pret-ref}		bag\tsc{=q}\\
	\glt	\sqt{‎How do we call in our language the bag that hangs behind his back? (i.e. a backpack)}

	\ex	\label{ex:‎‎‎Now there are other things that appeared (lit. went out)}
	\gll	hana	[tːura	d-uq-un-te]	cara	cik'al	le-d\\
		now	outside	\tsc{npl-}go\tsc{.pfv-pret-dd.pl} 	other	something	exist\tsc{-npl}\\
	\glt	\sqt{‎‎‎Now there are other things that appeared (lit. went out).}
	
	\ex	\label{ex:The one that was in the hand had been put down}
\gll	iž	[ka-b-išː-ib-il]	naˁq-li-cːe-b-il=de\\
	this	\tsc{down}\tsc{-n-}put\tsc{.pfv-pret-ref}	hand\tsc{-obl-in-n-ref=pst}\\
	\glt	\sqt{The one had been put down was the one the one in the hand.}
\end{exe}


The head noun in the vast majority of corpus examples follows the \isi{relative clause}, but other positions are possible, too. Whenever the \isi{relative clause} occurs after the head or separated from the head (preceding it or following it), the \isi{cross-categorical suffixes} or the \isi{modal participle} need to be employed and it can be argued that the \isi{relative clause} is morphosyntactically not part of the \isi{noun phrase} anymore. Note that such examples are not particularly frequent and most of the following examples stem from staged narrations and poems. Sentence \refex{ex:‎‎‎The snake that sat in a pit together with the rich Ismail gave it to me} comes from the translation of a fairy tale from Standard Dargwa\slash Russian.  Russian has postnominal \isi{relative clauses} and the Russian \isi{word order} has simply been copied. Similarly, \refex{ex:the brothers who came home with empty handsSyntax} and \refex{ex:‎‎‎I love the girl who makes carpets} are translations from Russian. Example \refex{ex:They found this bride who had just married} and \refex{ex:We tell you the story like the story that we put together, how we got there} come from spontaneous narratives, and \refex{ex:the one that turned out my fire that was burning for so many years} is part of a poem.
%
\begin{enumerate}
	\item	\isi{relative clause} following the head noun 
	%
	\begin{exe}
		\ex	\label{ex:‎‎‎The snake that sat in a pit together with the rich Ismail gave it to me}
		\gll	dam	b-ičː-ib	iž	maˁlʡuˁn-ni	[ca	kur-re	ka-b-iž-ib-il	dawla-či-w	Ismaˁʔil-li-cːella]\\
			\tsc{1sg.dat}	\tsc{n-}give\tsc{.pfv-pret}	this	snake\tsc{-erg}	one	pit\tsc{-loc}	\tsc{down-n-}be\tsc{.pfv-pret-ref}	wealth\tsc{-adjvz-m}	Ismail\tsc{-obl-comit}\\
		\glt	\sqt{‎‎‎The snake that sat in a pit together with the rich Ismail gave it to me.}
	\end{exe}

	\item	\isi{relative clause} following the head noun
	%
	\begin{exe}
		\ex	\label{ex:the brothers who came home with empty handsSyntax}
		\gll	uc-be	[čar	b-iχ-ub-te	d-ac'	nuˁq-b-a-cːella]\\
			brother\tsc{-pl}	back	\tsc{hpl-}be\tsc{.pfv-pret-dd.pl} \tsc{npl-}empty	hand\tsc{-pl-obl-comit}\\
		\glt	\sqt{the brothers who came back with empty hands}
	\end{exe}
		
	\item	\isi{relative clause} following the head noun and separated from it
	%
	\begin{exe}
		\ex	\label{ex:‎‎‎I love the girl who makes carpets}
		\gll	dam	rursːi	r-ikː-ul=da	[t'ams-ne	d-irq'-an]\\
			\tsc{1sg.dat}	girl	\tsc{f-}want\tsc{.ipfv-icvb=1}	carpet\tsc{-pl}	\tsc{npl-}do\tsc{.ipfv-ptcp}\\
		\glt	\sqt{‎‎‎I love the girl who makes carpets.} (E)
	\end{exe}
	
	
	\item	\isi{relative clause} following the head noun and separated from it
	%
	\begin{exe}
		\ex	\label{ex:They found this bride who had just married}
		\gll	il	c'ikuri	r-arčː-ib	ca<r>i,		[hana	xadi	ka-r-iž-ib-il]\\
			that	bride	\tsc{f-}find\tsc{.pfv-pret}	\tsc{cop<f>}	now	married	\tsc{down-f-}be\tsc{.pfv-pret-ref}\\
		\glt	\sqt{They found this bride who had just married}
	\end{exe}

		\item	\isi{relative clause} in canonical positions and \isi{relative clause} preceding the head noun
	%
	\begin{exe}
		\ex	\label{ex:We tell you the story like the story that we put together, how we got there}
	\gll	[nušːa-l	hež	b-arq'-ib-il]	χabar,	[hextːu	ag-ur-il]	a-cːe	b-urs-ul=da	χabar	daˁʡle\\
		\tsc{1pl-erg}	this	\tsc{n-}do\tsc{.pfv-pret-ref}	story	there.\tsc{up}	go\tsc{.pfv-pret-ref}	\tsc{2sg-in}	\tsc{n-}tell\tsc{-icvb=1}	story	as\\
	\glt	\sqt{We tell you the story like the story that we put together, how we got there.}
		\end{exe}
		
			
	\item	\isi{relative clause} preceding the head noun and headless \isi{relative clause}
	%
	\begin{exe}
		\ex	\label{ex:the one that turned out my fire that was burning for so many years}
		\gll	[[čum-k'al=ra	dus	halk-un-te]	di-la	c'a	d-iš-aq-un-ce]\\
			how.many\tsc{-indef=add}	year	catch.fire\tsc{.pfv-pret-dd.pl} \tsc{1sg-gen}	fire	\tsc{npl-}die.out\tsc{.pfv-caus-pret-dd.sg}\\
		\glt	\sqt{the one that turned out my fire that was burning for so many years}
	\end{exe}
	
\end{enumerate}

Relative clauses are verb-final with very few exceptions, which can be obtained in elicitation or when translating stories from Russian. This property distinguishes them from other subordinate clauses that can more easily place the verb in positions other than the right edge of the clause. For instance, as just mentioned, \refex{ex:the brothers who came home with empty handsSyntax} is part of a fairy tale that has been translated from Russian into Sanzhi. The \isi{relative clause} not only follows the head, since this is the standard \isi{constituent order} in Russian, but also contains a \isi{comitative} phrase after the verb, so that the verb does not end up in the final position.

Within the \isi{relative clause} the head is usually indicated by a gap. As is typical for East Caucasian languages, Sanzhi does not have relative pronouns. However, \isi{reflexive pronouns} can, in principle, be used to express co-reference between an argument in the main clause and another argument or adjunct in the \isi{relative clause}. In the corpus such sentences are not particularly frequent, but some examples can be found. In \refex{ex:He is probably telling the story about what he had experienced} the \isi{reflexive pronoun} is a \isi{goal} or \isi{experiencer} that is coreferent with the omitted \isi{agent} of the main clause. The \isi{participle} in the \isi{relative clause} is case-marked with the \isi{genitive} because it expresses the topic of a conversation. In \refex{ex:the mistakes that were caused by him} the \isi{reflexive pronoun} encodes the causer, and in \refex{ex:(he is remembering) the things that he had done} the \isi{agent}. Example \refex{ex:the food that is given to their dogs} above shows a \isi{reflexive pronoun} functioning as possessor within the \isi{relative clause}. In all unambiguous examples these pronouns are co-referential with the omitted subject argument in the main clause, and the head functions as an object in that main clause.
%
\begin{exe}
	\ex	\label{ex:He is probably telling the story about what he had experienced}
	\gll	[cin-i-j	či-d-sa-d-ač'-ib-t-a-lla]	χabar	b-urs-ul	urkː-ar	\\
		\tsc{refl.sg-obl-dat}	\tsc{spr-npl-hither}\tsc{-npl}-come.\tsc{pfv-pret-dd.pl-obl-gen}	story	\tsc{n-}tell\tsc{-icvb}	find\tsc{.m.ipfv-prs}\\
	\glt	\sqt{He is probably telling the story about what he had experienced.}

	\ex	\label{ex:the mistakes that were caused by him}
	\gll	[cin-ni-cːe-r	ka-d-ič-ib]	χat'a\\
		\tsc{refl.sg-obl-in-abl}	\tsc{down-npl-}occur\tsc{.pfv-pret}	mistake\\
	\glt	\sqt{the mistakes that were caused by him}

	\ex	\label{ex:(he is remembering) the things that he had done}
	\gll	[cin-ni	d-arq'-ib-te]	cik'al\\
		\tsc{refl.sg-erg}	\tsc{npl-}do\tsc{.pfv-pret-dd.pl} 	something\\
	\glt	\sqt{the things that he had done}
\end{exe}

Examples in which the nominal head itself is expressed by a reflexive in the \isi{relative clause} were judged as not very well-formed sentences by Sanzhi speakers:
%
\begin{exe}
	\ex	\label{ex:The girl whose cat died is crying}
	\gll	{?}	[cin-na	kːaˁta	b-ebč'-ib-il]	rursːi	r-isː-ul	ca-r\\
		{}	\tsc{refl.sg-gen}	cat	\tsc{n-}die\tsc{.pfv-pret-ref}	girl	\tsc{f-}cry\tsc{-icvb}	\tsc{cop-f}\\
	\glt	\sqt{The girl whose cat died is crying.} (E)

	\ex	\label{ex:The friend with whom I came goes to Derbent tomorrow}
	\gll	{??}	[du	cin-i-cːella	sa-jʁ-ib-il]	juldaš	\\
		{}	\tsc{1sg}	\tsc{refl.sg-obl-comit}	\tsc{hither}-come\tsc{.pfv-pret-ref}	friend	\\
	\glt	\sqt{the friend with whom I (masc.) came} (E)
\end{exe}

Relative clauses with semantically empty or light head \isi{nouns} can be found, and in most cases it is the noun \tit{zamana}, which takes over this function. These clauses have been grammaticalized into \isi{adverbial clauses} expressing temporal simultaneity \refex{ex:at the time (when they) were listening carefully}, \refex{ex:at the time (when) you send the water from the aperture} (\refsec{sec:constructions with zamana}). Relative clauses with \tit{musːa} \sqt{place} as head can be interpreted in a similar fashion as \isi{adverbial clauses} referring to the location of an event \refex{ex:‎‎‎The boys went to the place where cows are slaughteredA}, \refex{ex:‎across the place where you go to the pakh-pakh (place name)}.
%
\begin{exe}
	\ex	\label{ex:at the time (when they) were listening carefully}
	\gll	[ʡaˁħ-ʡaˁħ-le	gu-lik'-an]	zamana\\
		good-good\tsc{-advz}	\tsc{down}-listen\tsc{-ptcp}	time\\
	\glt	\sqt{at the time (when they) were listening carefully}

	\ex	\label{ex:at the time (when) you send the water from the aperture}
	\gll	[ʁaˁni-le-r	gu-d-a	hin	d-at-aʁ-ib]	zamana\\
		aperture\tsc{-loc-abl}	\tsc{down}\tsc{-npl-dir}	water	\tsc{npl-}send-do\tsc{.pfv-pret}	time\\
	\glt	\sqt{at the time (when) you send the water from the aperture}
\end{exe}

In general, \isi{relative clauses} in Sanzhi Dargwa are part of a larger family of constructions that can be classified as `noun-modifying clause constructions.' They include, apart from genuine \isi{relative clauses} in which the head has a position in the \isi{relative clause}, also constructions with \sqt{light nouns}  such as \textit{zamana} `time' \refex{ex:at the time (when they) were listening carefully}, \refex{ex:at the time (when) you send the water from the aperture} and other sentential complements of \isi{nouns}. Sentences \refex{ex:the boy called Mahammadhazhi}, \refex{ex:There was this game of ours, the hide-and-seek} above already showed that the same formal means that are employed to formulate \isi{relative clauses} are also used when there is no syntactic relationship between the head noun and the preceding noun. In such cases the hearer is expected to establish the semantic link between the noun and the clause that modifies the noun on the basis of the context and of general knowledge. The sentences in \xxref{ex:the story that Mulla Nasredin sold a donkey}{ex:the smell of baking bread} provide more examples of such sentential modifiers. Such versatility of the \isi{relative clause} construction is typical for East Caucasian languages and has been repeatedly discussed in the literature (\citealp{Daniel.Lander2008, Daniel.Lander2010}; \citealp{Comrie.Forker.Khalilova2017}).
%
\begin{exe}
	\ex	\label{ex:the story that Mulla Nasredin sold a donkey}
	\gll	[Malla Nasretːin-ni amχa	b-ic-ib]	χabar\\
		Mullah	Nasredin\tsc{-erg}	donkey	\tsc{n-}sell\tsc{.pfv-pret}	story\\
	\glt	\sqt{the story that Mullah Nasredin sold a donkey} (E)

	\ex	\label{ex:the mastery of making bracelets}
	\gll	[qulexa	b-irq'-an]	usta-dex\\
		bracelet	\tsc{n-}do\tsc{.ipfv-ptcp}	master\tsc{-nmlz}\\
	\glt	\sqt{the mastery of making bracelets} (E)

	\ex	\label{ex:the tradition of giving alms}
	\gll	[sadaq'a	lukː-an	/		luk-ni-la]	ʡaˁdat\\
		alms	give\tsc{.ipfv-ptcp}	/	give\tsc{.ipfv-msd-gen}	custom\\
	\glt	\sqt{the tradition of giving alms} (E)

	\ex	\label{ex:the smell of baking bread}
	\gll	[t'ult'	b-uc'-an]	t'em\\
		bread	\tsc{n-}bake\tsc{.ipfv-ptcp}	smell\\
	\glt	\sqt{the smell of baking bread} (E)
\end{exe}

Instead of \isi{relative clauses} it is also possible to have a nominalized clause with the \isi{masdar} suffix that is marked for the \isi{genitive} \refex{ex:the mastery of making bracelets}, \refex{ex:the story how we went somewhere for condolence}. Such constructions are semantically equivalent to the noun-modifying construction above \refex{ex:He is probably telling the story about what he had experienced}, \refex{ex:the story that Mulla Nasredin sold a donkey}. 
%
\begin{exe}
	\ex	\label{ex:the story how we went somewhere for condolence}
	\gll	[čina-k'u	ʡaˁlħaˁm-le	d-uˁq'-ni-la]	χabar\\
		where\tsc{-indef}	condolence\tsc{-loc}	\tsc{1/2pl-}go\tsc{-msd-gen}	story\\
	\glt	\sqt{the story how we went somewhere for condolences}
\end{exe}


%%%%%%%%%%%%%%%%%%%%%%%%%%%%%%%%%%%%%%%%%%%%%%%%%%%%%%%%%%%%%%%%%%%%%%%%%%%%%%%%

\section{Headless relative clauses}
\label{sec:Headless relative clauses}

Headless \isi{relative clauses} can be formed in four different ways: (i) with the \isi{modal participle} (\tit{-an}) \xxref{ex:Do what you want}{ex:There is even nobody who is talking.}, (ii) by using the \isi{locative participle} (see \refsec{sssec:The locative participle} for further information and examples), (iii) by attaching the \isi{cross-categorical suffixes} \tit{-il} \refex{ex:The one that was in the hand had been put down}, \refex{ex:The one who is standing says} or \tit{-ce} (plural -\textit{te}) to the preterite or the \isi{modal participle} \refex{ex:the one that turned out my fire that was burning for so many years}, \refex{ex:There are those that they put there (lit. from above let down their own people)}, and (iv) occasionally by means of the nominalized \isi{optative} \refex{ex:Ah, the ones who have one beloved (i.e. the owners), who have apparently killed my son, may they die}. The types differ with respect to their function and morphosyntactic properties. Headless \isi{relative clauses} with the \isi{modal participle} can only be used when the nominalized \isi{relative clause} takes over the function of an \isi{absolutive} argument and therefore does not require further case marking. For instance, in \refex{ex:Do what you want} the \isi{relative clause} functions as P argument, and in \refex{ex:She is the one who I know, from Kala-Kureish} and \refex{ex:There is even nobody who is talking.} as \isi{copula} predicate. 
%
\begin{exe}
	\ex	\label{ex:Do what you want}
	\gll	[at	b-ikː-an]	b-arq'-a!\\
		\tsc{2sg.dat}	\tsc{n-}want\tsc{.ipfv-ptcp}	\tsc{n-}do\tsc{.pfv-imp}\\
	\glt	\sqt{Do what you want!} (E)

	\ex	\label{ex:She is the one who I know, from Kala-Kureish}
	\gll	it	[r-alχ-an]	ca-r,	urc'mucːan\\
		that	\tsc{f-}know\tsc{.ipfv-ptcp}	\tsc{cop-f}	Kala-Kureish.person\\
	\glt	\sqt{She is the one who (I) know, from Kala-Kureish.}

	\ex	\label{ex:There is even nobody who is talking.}
	\gll	ja	[ʁaj	Ø-ik'ʷ-an]	w-akːu\\
		even	word	\tsc{m-}say\tsc{.ipfv-ptcp}	\tsc{m-}\tsc{cop.neg}\\
	\glt	\sqt{There is not even anybody who is talking. }
\end{exe}

Relative clauses with the \isi{locative participle} can only express spatial meaning, and the \isi{locative participle} can be marked with directional case suffixes (essive, \isi{ablative}), but not with any other cases.

The use of the \isi{cross-categorical suffixes} \tit{-il} and -\textit{ce} is a major strategy for the formation of headless \isi{relative clauses}. The two suffixes are used in a variety of contexts and their overall function can be roughly described as forming referential attributes\slash definite descriptions. Items marked with the suffixes acquire the morphosyntactic properties of \isi{nouns} (see \refsec{ssec:The -ce / -te attributive} and \refsec{ssec:The -il attributive} for detailed accounts). There is a functional distribution between the two suffixes. Both suffixes are used when the headless \isi{relative clause} denotes a singular referent and when it is used without any further case marking, i.e., when it is an argument in the \isi{absolutive} case in the main clause  \refex{ex:the one that turned out my fire that was burning for so many years}, \refex{ex:The one who is standing says}, \refex{ex:‎The one where they are sitting and drinking, I put this (picture) somewhere}, but the suffix -\textit{il} is more common. Note that the headless relative in \refex{ex:‎The one where they are sitting and drinking, I put this (picture) somewhere} contains a further \isi{adverbial clause} that is embedded into the \isi{relative clause}. I could find only a handful examples of headless \isi{relative clauses} bearing \tit{-ce} in the corpus. The example in \refex{ex:the one that turned out my fire that was burning for so many years} comes from a poem.  

%
\begin{exe}
	\ex	\label{ex:The one who is standing says}
	\gll	hej	[ka-jcː-ur-il]	Ø-ik'-ul	ca-w, ``ħaˁsan	ʡaˁli	hel-itːe,''	Ø-ik'-ul	ca-w,	``uruχ-le=de=w?''\\
		this	\tsc{down}-get.up.\tsc{pfv-pret-ref}	\tsc{m-}say\tsc{.ipfv-icvb}	\tsc{cop-m} Hasan	Ali	that\tsc{-advz}	\tsc{m}-say\tsc{.ipfv-icvb}	\tsc{cop-m}	fear\tsc{-advz=2sg=q}\\
	\glt	\sqt{The one who is standing says, ``Like Hassan Ali,'' he says, ``Are you afraid?''}

	\ex	\label{ex:‎The one where they are sitting and drinking, I put this (picture) somewhere}
	\gll	[[b-učː-ul]	ka-b-iž-ib-il]	ka-b-išː-ib=da	heltːu	čina-del\\
		\tsc{hpl-}drink\tsc{.ipfv-icvb}	\tsc{down-hpl-}be\tsc{.pfv-pret-ref}	\tsc{down-n-}put\tsc{.pfv-pret=1}	there where\tsc{-indef}\\
	\glt	\sqt{‎The one (picture) where they are sitting and drinking, I put this (picture) somewhere.}
\end{exe}

When the referent of a nominalized \isi{relative clause} is plural, only the suffix -\textit{te} is allowed \refex{ex:Except from those who were there somewhere (i.e. hid themselves), there are no remaining ones}, \refex{ex:There are those that they put there (lit. from above let down their own people)}. 
%
\begin{exe}
	\ex	\label{ex:Except from those who were there somewhere (i.e. hid themselves), there are no remaining ones}
	\gll	[kelg-un-te]	akːʷ-ar\\
		remain\tsc{.pfv-pret-dd.pl} \tsc{cop.neg-prs}\\
	\glt	\sqt{except for the remaining ones}

	\ex	\label{ex:There are those that they put there (lit. from above let down their own people)}
	\gll	[hetː-a-li	itːa-lla	či-ka-b-at-ur-te]	le-b=q'al\\
		those\tsc{-obl-erg}	those\tsc{.obl-gen}	\tsc{spr-down}\tsc{-hpl-}let\tsc{.pfv-pret-dd.pl} exist\tsc{-hpl=mod}\\
	\glt	\sqt{There are those that they put there (lit. from above let down their own people).}
\end{exe}

When the headless \isi{relative clauses} take case markers, the suffix -\textit{il} is used for reference in the singular and \tit{-te} (in its \isi{oblique stem} form \tit{-ta}) for reference in the plural. Examples with singular referents are not very common in the corpus \refex{ex:He did not even look at his savior.} (see \refex{ex:the son of (the one) who took away our ploughshare} in \refsec{ssec:The -il attributive} for one more instance). Example \refex{ex:‎The one who was wrestling hit the police officer}, in which the nominalized \isi{relative clause} functions as \isi{agent}, has been elicited. When the \isi{dative} case is added, the resulting clauses can have the semantics of \isi{adverbial clauses} expressing causes (due to the meaning of the \isi{dative} case). One example is \refex{ex:He became sad, in my opinion, he got very sad, because of what he did minor} in \refsec{sssec:The attributive markers -il and -ce / -te in combination with the participles}. 
%
\begin{exe}
	\ex	\label{ex:He did not even look at his savior.}
	\gll	ca-w	w-erc-aq-ur-il-li-j	er=či=ra	a-w-erč'-ib	\\
		\tsc{refl-m}	\tsc{m}-save.\tsc{pfv-caus-pret-ref-obl-dat}	look=on=\tsc{add}	\tsc{neg-m}-look.\tsc{pfv-pret}	\\
	\glt	\sqt{He did not even look at his savior.}

	\ex	\label{ex:‎The one who was wrestling hit the police officer}
	\gll	w-iħ-ib-il-li	milic'a-j	b-aˁq-ib\\
		\tsc{m-}wrestle\tsc{.pfv-pret-ref-erg}	police\tsc{-dat}	\tsc{n-}hit\tsc{.pfv-pret}\\
	\glt	\sqt{‎The one who was wrestling hit the police officer.} (E)
\end{exe}


Examples of headless \isi{relative clauses} with plural referents and further case markers are comparatively frequent in the Sanzhi corpus. As \refex{ex:‎The ones who are drinking, what good things do they do, they do not do anything (good)}, \refex{ex:these, hm, with whom (he) himself was drinking} show, the nominalized \isi{relative clauses} can occur in various argument and adjunct positions in the main clause.
%
\begin{exe}
	\ex	\label{ex:‎The ones who are drinking, what good things do they do, they do not do anything (good)}
	\gll	heštːi	[deč-li	b-učː-an-t-a-l]	cik'al=č'u	ʡaˁħ-dex,	iš-tː-a-l	ce	b-irq'-u=ja,		cik'al=č'u	a-b-irq'-u\\
		these	drinking\tsc{-erg}	\tsc{n-}drink\tsc{.ipfv-ptcp-pl-obl-erg}	thing\tsc{=emph}	good\tsc{-nmlz}	this\tsc{-pl-obl-erg}	what	\tsc{n-}do\tsc{.ipfv-prs.3=q}	thing\tsc{=emph}	\tsc{neg-n-}do\tsc{.ipfv-prs.3}\\
	\glt	\sqt{‎The ones who are drinking, what good things do they do, they do not do anything (good).}

	\ex	\label{ex:these, hm, with whom (he) himself was drinking}
	\gll	hel-tːi	cinna	hetːi	[ca-w	učː-ib-t-a-cːella]\\
		that\tsc{-pl}	pause.filler	those	\tsc{refl-m}	drink\tsc{.m.ipfv-pret-pl-obl-comit}\\
	\glt	\sqt{these, hm, with whom (he) himself was drinking}
\end{exe}

The suffix \tit{-ce} (but not -\textit{il} or -\textit{te}) is also used as a nominalized verb form taking over an argument position in a clause with a \isi{complement-taking predicate}. This means that -\textit{ce} functions as a complementizer in complement clauses of the fact-type (see \refsec{ssec:The attributive marker -ce (-te)COMPL}). In some cases the nominalized clause, which occurs together with a \isi{complement-taking predicate}, does not express a proposition, but refers to an entity such as a human being or an event or to abstract entities such as thoughts, wishes, etc. In that case the nominalized verb does not function as a complement, but as a headless \isi{relative clause} (\refsec{ssec:Nominalized relative clauses resembling complement constructions}).

In addition to the just discussed types of nominalized \isi{relative clauses}, Sanzhi has a nominalized \isi{optative} that functions like a headless \isi{relative clause} in the sense that it can take over arguments or adjunct positions in the clause and can be inflected. It preserves the semantics of the \isi{optative} (\refsec{sec:optative}). In example \refex{ex:Ah, the ones who have one beloved (i.e. the owners), who have apparently killed my son, may they die}, the nominalized verb \tit{w-ebk'-} \sqt{die} is inflected for the \isi{ergative} because it functions as the \isi{agent} of the verb \tit{kax-} \sqt{kill}.
%
\begin{exe}
	\ex	\label{ex:Ah, the ones who have one beloved (i.e. the owners), who have apparently killed my son, may they die}
	\gll	ah,	w-ah	w-ebk'-ar-t-a-l	di-la	durħuˁ	kax-ub-le	už-ib-le=q'al\\
		ah	\tsc{m-}owner	\tsc{m-}die\tsc{.pfv-opt-pl-obl-erg}	\tsc{1sg-gen}	boy	kill\tsc{.pfv-pret-cvb}	be\tsc{.m-pret-cvb=mod}\\
	\glt	\sqt{Ah, may the ones die who have (a beloved one), since they apparently killed my son!}
\end{exe}
