\chapter{Simple clauses including copula clauses and grammatical relations}
\label{cpt:Simpleclauses}

This chapter addresses the basic structural properties of simple main clauses, including declarative, interrogative, \isi{imperative} and \isi{optative} clauses that express different illocutionary acts, namely assertions, \isi{questions} and directives. These three clause types contrast mainly by their morphological marking (verbal suffixes, predicative \is{particle}particles), rather than by differences in \isi{constituent order} or intonation. Simple clauses can be divided into simple verbal clauses with verbs other than the \isi{copula} (\refsec{sec:Simple clauses headed by verbs other than copulae}) and \isi{copula} clauses (\refsec{sec:copulaclauses}). This division cross-cuts the division into declarative, interrogative, \isi{imperative} and \isi{optative} clauses. Grammatical relations and the notion of subjecthood are analyzed in \refsec{sec:Grammatical relations}. Word order at the clausal level and information-structural properties of \is{simple clause}simple clauses and other types of clauses are analyzed in \refcpt{cpt:Constituent order and information structure}. 


%%%%%%%%%%%%%%%%%%%%%%%%%%%%%%%%%%%%%%%%%%%%%%%%%%%%%%%%%%%%%%%%%%%%%%%%%%%%%%%%

\section{Simple clauses headed by verbs other than copulas}
\label{sec:Simple clauses headed by verbs other than copulae}

This section discusses \is{simple clause}simple clauses with verbal predicates, which can be declarative, interrogative, \isi{imperative} and \isi{optative} clauses. The main focus of the section is on declarative clauses. The other clause types will be briefly mentioned at the end of the section. The \isi{constituent order} in \is{simple clause}simple clauses is discussed in \refsec{sec:Constituent order at the clause level and information structure}.

Verbal clauses obligatorily consist of a verbal predicate. All other constituents (arguments and adjuncts) can be left out. By contrast, declarative \isi{copula} clauses can be formed with non-verbal predicates and without finite verb forms if they contain predicative \is{particle}particles (\refsec{sec:copulaclauses}).  

In general, the \isi{number} of arguments, their semantic roles as well as their case-marking, depend on the \is{valency class}valency classes of the verbs and on whether further valency-changing operations such as \isi{antipassive} or causativization have been applied. Valency patterns of predicates can be divided into one-place (monovalent), two-place (bivalent) and three-place (trivalent) patterns (see \refsec{sec:Verb valency classes} for a detailed overview). I will follow the macrorole approach as presented in \citet{Bickel2011} and \citet{Bickel.etal2015} and distinguish between bivalent predicates between an A argument (the argument with the most agentive properties) and a P argument (the argument with the least agentive or most patientive properties). Three-place verbs have, in addition to the A argument, a G argument and a T argument. The G (goal-like) argument is more stationary than the T (theme-like) argument, and, in contrast to T, might be receiving an experience or being exposed to an experience. The label S will be used for the single argument of \is{intransitive verb}intransitive verbs and for \isi{absolutive} arguments of extended \is{intransitive verb}intransitive verbs.

Cases used to encode the arguments are \isi{absolutive}, \isi{ergative}, \isi{dative}, and to a lesser extent, \isi{genitive} and \is{spatial case}spatial cases such as the \tsc{ante}-\isi{ablative} or the \tsc{in}-lative. Gender agreement is controlled in most cases by the \isi{absolutive} argument of the clause (\refsec{sec:Gender/number agreement}). However, certain verb forms allow the \isi{ergative} or the \isi{dative} argument as controller (\refsec{ssec:Gender agreement with arguments in other than the absolutive case}), while other clauses lack \isi{absolutive} arguments and resort to default agreement (\refsec{General remarks on gender/number agreement}). Person agreement, which works  independently of \isi{gender} agreement, follows the hierarchy 1, 2 > 3, whereby only S, A, P, and T arguments can function as controllers. From the person hierarchy follows that speech act participants control the agreement if the clause has any speech act participants. In clauses with only first and second person arguments, either person can control agreement independently of their \isi{grammatical role}, although there might be a small tendency for second person arguments to outrank first person arguments (see \refsec{sec:Person agreement} for more information).

Declarative verbal clauses obligatorily contain finite verb forms, which exhibit the full range of TAM marking and \isi{gender} and \isi{person agreement}. Negation is expressed through a prefix or, if the verb form includes the \isi{copula} as auxiliary, by means of the negative \isi{copula} (\refsec{sec:Negation}). Arguments and adjuncts can be pronouns or full noun phrases \refex{ex:He did not kill the fox}, \refex{ex:The fox gave him his hair}. Arguments that can be interpreted through the context are often omitted \refex{ex:She is crying.Why}. Adjuncts can be instruments, companions or express temporal, spatial or other circumstances (\refsec{ssec:Functionsofgrammaticalcases}, \refsec{ssec:Functions of semantic cases}). The following examples illustrate simple verbal declarative clauses with verbs of the major \is{valency class}valency classes: intransitive, extended intransitive, affective, transitive, and extended transitive. 

In\is{transitive verb}transitive verbs (\refsec{sec:Intransitive verbs}) have one single argument (S) in the \isi{absolutive} case controlling \isi{gender}, \isi{number} and \isi{person agreement} \refex{ex:She is crying.Why}.

\begin{exe}
	\ex	\label{ex:She is crying.Why}
	\gll	cellij r-isː-ul=de?	\\
		why \tsc{f}-cry-\tsc{icvb=2sg} 	\\
	\glt	\sqt{Why are you (fem.) crying?}
\end{exe}

Extended \is{intransitive verb}intransitive verbs are bivalent and have an S argument in the \isi{absolutive} and a further argument either in the \isi{dative}, \tsc{in}-lative, \tsc{ante}-\isi{ablative}, or in other \is{spatial case}spatial cases \refex{ex:The wolf looked at the fox} (see \refsec{sec:Extended intransitive verbs} for further examples).

\begin{exe}
	\ex	\label{ex:The wolf looked at the fox}
	\gll	bec'	kːurtːa-j er	b-erč'-ib-le, \ldots	\\
		wolf	fox\tsc{-dat} look	\tsc{n-}look\tsc{.pfv-pret-cvb}	\\
	\glt	\sqt{The wolf looked at the fox, \ldots}
\end{exe}

Bivalent \is{affective verb}affective verbs (\refsec{sec:Bivalent affective verbs}) have a dative-marked \isi{experiencer} (A) and a \isi{stimulus} argument in the \isi{absolutive} case (P) \refex{ex:‎‎‎The little brother experienced many difficulties}.

\begin{exe}
	\ex	\label{ex:‎‎‎The little brother experienced many difficulties}
	\gll	d-aqil	qːihin-dex	či-d-ič-ib	nik'a	ucːi-j\\
		\tsc{npl-}much	difficult\tsc{-nmlz}	\tsc{spr-npl-}occur\tsc{.pfv-pret}	small	brother\tsc{-dat}\\
	\glt	\sqt{‎‎‎The little brother experienced many difficulties.}
\end{exe}

Transitive verbs (\refsec{sec:Transitive verbs}) require an \isi{agent} in the \isi{ergative} (A) and a \isi{patient} in the \isi{absolutive} (P) \refex{ex:He did not kill the fox}. 

\begin{exe}
	\ex	\label{ex:He did not kill the fox}
	\gll	hel-i-l	kːurtːa a-kax-ub\\
		that\tsc{-obl-erg}	fox \tsc{neg-}kill\tsc{.pfv-pret}\\
	\glt	\sqt{He did not kill the fox.}
\end{exe}

Extended \is{transitive verb}transitive verbs (\refsec{sec:Extended transitive verbs and ditransitive verbs}) add a further G argument marked by the \isi{dative} or by \is{spatial case}spatial cases to the \isi{ergative} A and the \isi{absolutive} T \refex{ex:The fox gave him his hair}.

\begin{exe}
	\ex	\label{ex:The fox gave him his hair}
	\gll	kːurtːa-l	b-ičː-ib	hel-i-j	cin-na	ʁez\\
		fox\tsc{-erg}	\tsc{n-}give\tsc{.pfv-pret}	that\tsc{-obl-dat}	\tsc{refl.sg-gen}	hair\\
	\glt	\sqt{The fox gave him his hair.}
\end{exe}

More detailed information on these and other \is{valency class}valency classes can be found in \refsec{sec:Verb valency classes}.

Interrogative clauses are marked by interrogative enclitics, which also belong to the class of predicative \is{particle}particles. They are often but not always signaled by rising intonation. They usually contain finite verb forms, but it is also possible to encode \is{interrogative clause}interrogative clauses with certain non-finite verb forms and interrogative enclitics. See \refcpt{cpt:Interrogative clauses} for more information on their morphosyntactic properties and \refsec{ssec:Information structure in interrogative clauses and question-answer pairs} for the \isi{constituent order}, which is largely identical to the \isi{constituent order} of declarative clauses (except for differences that are due to the \isi{information structure}). Sanzhi has another type of \isi{interrogative clause} with modal semantics. This type, which I call the \textit{modal interrogative} contrasts to all other \is{interrogative clause}interrogative clauses since it makes use of a special verbal suffix. This suffix exists only for first person subject-like arguments (\refsec{sec:modalinterrogative}) and does not express tense. The \isi{modal interrogative} is thus more like \isi{imperative} and \isi{optative} clauses than \is{interrogative clause}interrogative clauses. 

Imperative, \isi{prohibitive} and \isi{optative} clauses contain verbs inflected with suffixes which do not express tense. Imperative and \isi{prohibitive} clauses are restricted in their use to second person subject-like arguments, which can be overtly expressed as in declarative or \is{interrogative clause}interrogative clauses (\refsec{sec:imperative} and \refsec{sec:prohibitive}). The \isi{optative} does not show person restrictions, but cannot be formed from imperfective verb stems and is thus constrained in its aspectual value (\refsec{sec:optative}).



%%%%%%%%%%%%%%%%%%%%%%%%%%%%%%%%%%%%%%%%%%%%%%%%%%%%%%%%%%%%%%%%%%%%%%%%%%%%%%%%

\section{Morphosyntactic properties of copula clauses}
\label{sec:copulaclauses}

Copula clauses are constructions prototypically containing a \isi{copula}, a \isi{copula} predicate and a \isi{copula} subject. The \isi{copula} can, but does not have to be, a verb. Genuine copulas are considered to be semantically empty \citep[5]{Pustet2005}. According to this definition, Sanzhi Dargwa \tit{ca-b} can be considered a \isi{copula} as well as the person enclitics \tit{=da} and \tit{=de} and the past \isi{enclitic} \tit{=de}. In addition, Sanzhi has four locational copulas and a \isi{number} of auxiliaries that also head \isi{copula} clauses. \reftab{tab:Items functioning as copulae} provides an overview of all items used in \isi{copula} constructions with references to the sections in which more information about the morphology of the items is given. In the following section I will refer to all items in \reftab{tab:Items functioning as copulae} as copulas whenever they are used in \isi{copula} constructions. The items listed in the last part of the table are auxiliaries that are not entirely `semantically empty'. They are included in the table and discussed in this section because they are used to express TAM forms of \isi{copula} clauses for which the copulas and predicative \is{particle}particles cannot be used since they lack those forms. Some of the items in \reftab{tab:Items functioning as copulae} regularly combine in \isi{copula} clauses. (I) The predicative \is{particle}particles can occur as enclitics on the \isi{copula} \textit{ca-b} (see below for examples), but not on the negative \isi{copula}. They can also occur on the locational copulas (\refsec{ssec:Location, existence, and possession}) and on the other auxiliaries because they are a regular part of certain TAM forms. (II) The \isi{copula} \textit{ca-b} and its negative counterpart are also used in a few TAM forms and thus can combine with most auxiliaries given in \reftab{tab:Items functioning as copulae} \refex{ex:The hips will be a strong medicine}. (III) The same applies to the locational copulas, which are also occasionally used in certain periphrastic TAM clauses and therefore can combine with the auxiliaries in \isi{copula} clauses (\refsec{sec:Verb forms with locational copulae}).

Which \isi{copula} item is chosen depends on the person of the \isi{copula} subject, on the meaning of the \isi{copula} construction, on the temporal reference of the clause, and on further categories such as mood, epistemic modality, illocutionary force, and clause type. The \isi{copula} \tit{ca-b}, the locational copulas, and the verb \tit{b-el} are defective, and in the case of \tit{ca-b}, also suppletive under \isi{negation}. This means that they form only a very small set of verb forms compared with the forms available for regular verbs (including the other auxiliaries listed in \reftab{tab:Items functioning as copulae}). 

\begin{table}
	\caption{Items functioning as copulas}
	\label{tab:Items functioning as copulae}
	\small
	\begin{tabularx}{0.98\textwidth}[]{%
		>{\raggedright\arraybackslash\itshape}p{55pt}
		>{\raggedright\arraybackslash}X}
		
		\lsptoprule
			\multicolumn{2}{l}{Copulas (\refsec{sec:The copula})}\\
		\midrule
 			ca-b, ca<b>i				&	third person, \isi{present tense}, habitual\\
			akː-u				&	third person, \isi{negation}, \isi{present tense}, habitual\\
			b-akː-u			&	third person, \isi{negation}, existence\slash location\\
		\midrule
			\multicolumn{2}{l}{Predicative \is{particle}particles (\refsec{sec:Predicative particles})}\\\midrule
		
			=da				&	first person (singular and plural), second person plural;\\
			{}				&	~~\isi{present tense}, habitual\\
			=de				&	second person singular; \isi{present tense}, habitual\\
			=de				&	all persons, \isi{past tense}\\\midrule		
			\multicolumn{2}{l}{Locational copulas (\refsec{sec:Locational copulae})}\\\midrule
		
			le-b				&	existence\slash location close to speaker and hearer or undifferentiated\\
			te-b				&	existence\slash location away from the deictic center (speaker)\\
			k'e-b				&	existence\slash location above the deictic center (speaker) 	\\
			χe-b				&	existence\slash location below the deictic center (speaker) \\\midrule		
			\multicolumn{2}{l}{Other auxiliaries used in \isi{copula} constructions (\refsec{sec:Other verbs used in copula-functions and as auxiliaries})}\\\midrule		
			b-el				&	\isit{past tense} (\sqt{remain, stay})\\
			\mbox{b-irχʷ- \emph{(\tsc{ipfv}) /}}	&	subordinate clauses, future tense, non-indicative mood\\
			~~b-iχʷ-~\emph{(\tsc{pfv)}}	&	~~(\sqt{be, become, happen, can})\\
			\mbox{b-irk- \emph{(\tsc{ipfv}) /}}	&	future tense, non-indicative mood (\sqt{be, occur, get, receive})\\
			~~b-ik- \emph{(\tsc{pfv})}\\
			b-ug-  		&	indirect evidentiality (\sqt{be, stay, remain})\\
			b-urkː- \emph{(\tsc{ipfv})} & epistemic modality (\sqt{find}) \\
		\lspbottomrule
	\end{tabularx}
\end{table}

Under certain circumstances, one of the three constituents can be absent. Copula clauses without a subject are, for instance, weather constructions \refex{ex:It used to be very cold in the winter} or other impersonal clauses \refex{ex:‎It is difficult, it is bad} (\refsec{ssec:Copula clauses without a subject}). The \isi{copula} predicate is present in most types of \isi{copula} clauses, but often lacking in \isi{existential copula} clauses \refex{ex:There is nothing more} or when \isi{possession} is expressed \refex{ex:I have a house}. The \isi{copula} can only be omitted in \isi{copula} clauses with third person subjects and predicative \is{particle}particles (\refsec{ssec:Copula clauses without a copula}).

The \isi{copula} subject occurs in the \isi{absolutive} case and controls agreement. It may be any kind of \isi{noun phrase} (common noun, proper name, pronoun, etc.) or nominalized clause. The \isi{copula} predicate, which bears the \isi{absolutive} case or an overt case marker, may be a \isi{noun phrase}, but it may also be an adjective, an adverbial, a postpositional phrase, or a nominalized clause; this partially depends on the \isi{copula} item. Sentence \refex{ex:‎I was a little boy@A} illustrates a \isi{copula} construction with the first person pronoun in the \isi{absolutive} as \isi{copula} subject and a noun with its adjectival modifier as predicate. The past \isi{enclitic} \tit{=de} fulfills the \isi{copula} function.

\begin{exe}
	\ex	\label{ex:‎I was a little boy@A}
	\gll	nik'a	durħuˁ=de	du\\
		small	boy\tsc{=pst}	\tsc{1sg}\\
	\glt	\sqt{‎I was a little boy.}
\end{exe}

The \isi{gender}/\isi{number} and \isi{person agreement} is always controlled by the \isi{copula} subject, not by the predicate \refex{ex:‎‎‎That boy is a monster}, \refex{ex:Murad is a / the master@A},  \refex{ex:Murad is here},  \refex{ex:‎I am milkmaid of the SOVKHOZ, not of the kolkhoz}, but not all copulas have \isi{gender} agreement slots, for example the negative \isi{copula} lacks an \isi{agreement prefix}. In \isi{copula} clauses that lack arguments in the \isi{absolutive} case, the \isi{copula} bears the default agreement suffix -\textit{b} and there is invariable third \isi{person agreement} (see \refsec{ssec:Copula clauses without a subject} below for more details). This mainly concerns \isi{experiential} constructions with affective predicates and experiencers in the \isi{dative} case \refex{ex:I feel(s) bad there}. One item used as a \isi{copula}, the past \isi{enclitic} =\textit{de}, does not encode person \refex{ex:‎I was a little boy@A} (see also \refsec{sec:Simple clauses headed by verbs other than copulae} below for the general structure of \is{simple clause}simple clauses with other verbs than the \isi{copula}).

\begin{exe}
	\ex	\label{ex:‎‎‎That boy is a monster}
	\gll	het	durħuˁ	aždaha	ca-w	/ *ca-b\\
		that	boy	monster	\tsc{cop-m}	/ \tsc{cop-n}\\
	\glt	\sqt{‎‎‎That boy is a monster.} (E)
	
	\ex	\label{ex:I feel(s) bad there}
	\gll	dam	wahi-l	ca-b	heχ-tːu-b\\
		\tsc{1sg.dat}	bad\tsc{-advz}	\tsc{cop-n}	\tsc{dem.down}\tsc{-loc-n}\\
	\glt	\sqt{I feel bad there.} (E)	
	
\end{exe}


In \isi{copula} clauses, where both the subject and the predicate are in the \isi{absolutive} and are identical in \isi{gender} \refex{ex:Murad is a / the master}, and in \isi{copula} clauses, where the \isi{copula} function is taken over by an item that does not exhibit agreement \refex{ex:‎I was a little boy@A}, it is impossible to determine the \isi{copula} subject by means of agreement. But based on general information-structural properties, we can assume that the topical \isi{noun phrase} is usually the subject, which in many cases is a pronoun \refex{ex:‎I was a little boy@A}, \refex{ex:Yes, they are one family}, \refex{ex:This is Zapir}. In a similar fashion, \isi{constituent order} can be indicative. 

The person enclitics, as well as the past \isi{enclitic}, can optionally be accompanied by the \isi{copula} (\tit{ca-b}), and in such constructions the \isi{copula} always serves as the host for the \isi{enclitic} \refex{ex:‎‎‎Are you his wife}. 


\begin{exe}
	\ex	\label{ex:‎‎‎Are you his wife}
	\gll	iž-i-la	xːunul	ca-r=de=w?\\
		this\tsc{-obl-gen}	woman	\tsc{cop-f=2sg=q}\\
	\glt	\sqt{‎‎‎Are you his wife?}
\end{exe}

Furthermore, the \isi{copula} and the person \isi{enclitic} or past \isi{enclitic} need to occur on the same host; they cannot be separated \refex{ex:It is me who is your sister.}. If in the example below the \isi{copula} at the end is omitted, the clause becomes grammatical with the meaning given in brackets below. 

\begin{exe}
	\ex	\label{ex:It is me who is your sister.}
	\gll	{*} du=da	ala	rucːi	ca-r\\
		{} \tsc{1sg=1}	\tsc{2sg.gen}	sister	\tsc{cop-f}\\
	\glt	(Intended meaning: \sqt{It is me who is your sister.}) (E)
\end{exe}

In contrast to the neighboring Dargwa variety of Icari \citep[138]{Sumbatova.Mutalov2003}, the \isi{copula} and the interrogative enclitics can co-occur in Sanzhi \refex{ex:Is this similar to himSyntax}. According to Nina Sumbatova (p.c.), Icari is the only Dargwa variety found so far where the \isi{copula} \textit{ca-b} excludes all other predicative \is{particle}particles; in contrast, the situation that we have in Sanzhi Dargwa is common and attested in many other Dargwa varieties (e.g. Akusha). 

\begin{exe}
	\ex	\label{ex:Is this similar to himSyntax}
	\gll	miši-l	ca-w=uw	iχ	iχ-i-j?\\
		similar\tsc{-advz}	\tsc{cop-m=q}	\tsc{dem.down}	\tsc{dem.down}\tsc{-obl-dat}\\
	\glt	\sqt{Is this similar to him?}
\end{exe}

In line with other clause types, the \isi{copula} most frequently occurs in clause-final position. Subjects predominantly precede the predicate, such that we can assume that the first \isi{absolutive} constituent is the \isi{copula} subject and the second one, which is often the host of enclitical \isi{copula} items, functions as the predicate in clauses with two \isi{absolutive} constituents \refex{ex:Murad is a / the master}, \refex{ex:‎‎‎Madina was a Dargwa girl@A}, \refex{ex:Yes, they are one family}. The standard third person \isi{copula} \tit{ca-b} is comparably strict in its requirement to occur in clause-final position in elicited assertions \refex{ex:Murad is a / the master}, though it might be followed by additional demonstratives that refer to the same item as the subject, and in \isi{questions} from the corpus we also find \isi{copula} subjects and predicates following \textit{ca-b} \refex{ex:Is this similar to himSyntax}. In principle, \tit{ca-b} can occur on its own and make up a full clause, such that phonological dependency is ruled out as an explanation for the ungrammaticality of \refex{ex:Murad is a / the master@B}. 


\begin{exe}
	\ex	\label{ex:Murad is a / the master}
	\begin{xlist}
		\ex	\label{ex:Murad is a / the master@A}
		\gll	Murad	ustːa ca-w\\
			Murad master \tsc{cop-m}\\
		\glt	\sqt{Murad is a\slash the master.} (E)
	
		\ex[*]{\tit{ca-w Murad ustːa}} \label{ex:Murad is a / the master@B}
		\ex[*]{\tit{Murad ca-w ustːa}} \label{ex:Murad is a / the master@C}
	\end{xlist}
\end{exe}

For other \isi{copula} items, it is easier to find utterances with other than clause-final position of the \isi{copula}, in particular when the predicative \is{particle}particles are used in \isi{copula} function \refex{ex:‎I was a little boy@A}. Locational copulas can also occur in clause-initial position:

\begin{exe}
	\ex	\label{ex:Murad is here}
	\begin{xlist}
		\ex	\label{ex:Murad is here@A}
		\gll	ištːu-w	Murad	le-w\\
			here\tsc{-m}	Murad	exist\tsc{-m}\\
		\glt	\sqt{Murad is here.} (E)
	
		\ex	\tit{le-w ištːu-w Murad}	\label{ex:Murad is here@B}
	\end{xlist}
\end{exe}

When used in \isi{copula} clauses, the predicative \is{particle}particles can only occur on the head of the phrase that functions as the subject \refex{ex:‎The good one was a singer, not the bad one} or on the head of the phrase that functions as the \isi{copula} predicate \refex{ex:‎The good one was a singer, not the bad one}, \refex{ex:‎‎‎Madina was a Dargwa girl@A}. Alternatively, they in case of coordinated constituents on the leftmost member of the \isi{coordination} \refex{ex:You are a liar, a drinking dog, and a thief}. They cannot be encliticized to any constituent that modifies the head \refex{ex:‎‎‎Madina was a Dargwa girl@B}. 

\begin{exe}
	\ex	\label{ex:‎The good one was a singer, not the bad one}
	\gll	[it	ʡaˁħ-ce]=de	dalaj-či,	cara	wahi-ce=de\\
		that	good\tsc{-dd.sg=pst}	song\tsc{-nmlz}	other	bad\tsc{-dd.sg=pst}\\
	\glt	\sqt{‎The good one was a singer, not the bad one.} (lit. the other was the bad one) (E)

	\ex	\label{ex:‎‎‎Madina was a Dargwa girl}
	\begin{xlist}
		\ex	\label{ex:‎‎‎Madina was a Dargwa girl@A}
		\gll	Madina	[darkːʷan	rursːi]=de\\
			Madina	Dargwa	girl\tsc{=pst}\\
		\glt	\sqt{‎‎‎Madina was a Dargwa girl.} (E)

		\ex[*]{\tit{Madina	[darkːʷan=de	rursːi]}} \label{ex:‎‎‎Madina was a Dargwa girl@B}
	\end{xlist}
\end{exe}

The only exception to this rule that I found so far are \isi{genitive} modifiers: they can host predicative \is{particle}particles in \isi{copula} clauses even if they do not occupy the functions of \isi{copula} subject or \isi{copula} predicate. The \isi{genitive} noun in \refex{ex:‎I am milkmaid of the SOVKHOZ, not of the kolkhoz} modifies the following \isi{copula} predicate. This is only possible in term focus\is{focus} constructions in which the host of the \isi{enclitic} is the focal part of the clause. Furthermore, in term focus\is{focus} constructions, the predicative \is{particle}particles can, in principle, also be attached to adverbial modifiers such as spatial adverbials, but the properties of these constructions require further research.

\begin{exe}
	\ex	\label{ex:‎I am milkmaid of the SOVKHOZ, not of the kolkhoz}
	\gll	du	sawχuz-la=da	dajark'a,	kalχuz-la	akːʷa-di\\
		\tsc{1sg}	sovkhoz\tsc{-gen=1}	milkmaid	kolkhoz\tsc{-gen}	\tsc{cop.neg-1}\\
	\glt	\sqt{‎I am milkmaid of the SOVKHOZ, not of the kolkhoz.} (E)
\end{exe}

In \refex{ex:You are a liar, a drinking dog, and a thief}, the nominal predicate is a coordinated \isi{noun phrase} that consists of three members, each bearing the \isi{additive enclitic} as required for nominal \isi{coordination} (\refsec{sec:Coordination of noun phrases and other phrases}). The person \isi{enclitic} follows the last member of the nominal predicate.

\begin{exe}
	\ex	\label{ex:You are a liar, a drinking dog, and a thief}
	\gll	u	[sːunkuˁq'=ra,	deč-la	χʷe=ra,	bilʡuˁt'=ra]=de\\
		\tsc{2sg}	liar\tsc{=add}	drinking\tsc{-gen}	dog\tsc{=add}	thief\tsc{=add=2sg}\\
	\glt	\sqt{You are a liar, a drinking dog, and a thief!}
\end{exe}

Copula constructions in Sanzhi can express identity, group membership, attribution, \isi{possession}, benefaction, and also location and existence (see, e.g. \citealp{Curnow2000}, \citealp[159\tnd188]{Dixon2010}). 
% --------------------------------------------------------------------------------------------------------------------------------------------------------------------------------------------------------------------- %

\subsection{Copula constructions expressing identity, group membership and attribution}
\label{ssec:Copula constructions expressing identity, group membership and attribution}

In \isi{copula} constructions that express identity, group membership, and attribution, all items displayed in \reftab{tab:Items functioning as copulae} except for the locational copulas are used. Predicates are mostly nominals, \isi{adjectives}, or adverbials. Sentences \refex{ex:Yes, they are one family} and \refex{ex:This is Zapir} show \isi{copula} constructions expressing identity, including deictic identificational clauses, and group membership. 

\begin{exe}
	\ex	\label{ex:Yes, they are one family}
	\gll	e,	heχ-tːi	ca	kulpat	ca-b\\
		yes	\tsc{dem.down}\tsc{-pl}	one	family	\tsc{cop-hpl}	\\
	\glt	\sqt{Yes, they are one family.}

	\ex	\label{ex:This is Zapir}
	\gll	hež	/	hej	Keno	ca-w\\
		this	/	this	Keno	\tsc{cop-m}\\
	\glt	\sqt{This is Keno.} (E)
\end{exe}

If the subject is first or second person and the temporal reference is present time or there is no temporal reference because of habituality, then the person enclitics are used \refex{ex:You are a liar, a drinking dog, and a thief}, \refex{ex:‎I was the son of Muslim parents; I am a Muslim}. If the clause has past time reference, the past \isi{enclitic} occurs \refex{ex:‎We three were from Daghestan, and there were two Russian guys}.

\begin{exe}
	\ex	\label{ex:‎I was the son of Muslim parents; I am a Muslim}
	\gll	busurman	atːa.aba-la	durħuˁ=de; du	busurman	insan=da\\
		Muslim father.mother\tsc{-gen}	boy\tsc{=pst}	\tsc{1sg}	Muslim	person\tsc{=1}		\\
	\glt	\sqt{‎I was the son of Muslim parents; I am a Muslim.}

	\ex	\label{ex:‎We three were from Daghestan, and there were two Russian guys}
	\gll	ʡaˁbal	nušːa	daʁistan-na=de,	k'ʷel	ʡuˁrus	duˁrħuˁ=de\\
		three	\tsc{1pl}	Dagestan\tsc{-gen=pst}	two	Russian	boy\tsc{=pst}\\
	\glt	\sqt{‎We three were from Dagestan, and there were two Russian guys.}
\end{exe}


Adjectives distinguish a short form from a long form. The long form contains the \isi{cross-categorical suffix} \tit{-ce} (plural \tit{-te}) (\refsec{ssec:The -ce / -te attributive}). The short form is reserved for the attributive usage within noun phrases (in addition to \isi{compounding}) \refex{ex:‎‎‎She was an old woman}; the long form is required for the predicative use \refex{ex:One is like old, one like young}. 

\begin{exe}
	\ex	\label{ex:‎‎‎She was an old woman}
	\gll	hel	r-uqna	xːunul=de	hel\\
		that	\tsc{f-}old	woman\tsc{=pst}	that\\
	\glt	\sqt{‎‎‎She was an old woman.}

	\ex	\label{ex:One is like old, one like young}
	\gll	ca	r-uqna-ce	ʁuna	ca-r	iχ,	ca	žahil-ce	ʁuna\\
		one	\tsc{f-}old\tsc{-dd.sg}	\tsc{eq}	\tsc{cop-f}	\tsc{dem.down}	one	young\tsc{-dd.sg}		\tsc{eq}\\
	\glt	\sqt{One (fem.) is like old, one like young.}
\end{exe}

Adverbials can also serve as predicates in \isi{copula} clauses. Example \refex{ex:‎This means, how we put (the pictures) down is correct} shows a nominalized participial clause in the subject position and an adverb as predicate of the \isi{copula} clause.

\begin{exe}
	\ex	\label{ex:‎This means, how we put (the pictures) down is correct}
	\gll	hej	značit,	hana	[nušːa-l	ka-b-išː-ib-il]	b-arx-le	ca-b\\
		this	thus	now	\tsc{1pl-erg}	\tsc{down-n-}put\tsc{.pfv-pret-ref}	\tsc{n-}correct\tsc{-advz}	\tsc{cop-n}\\
	\glt	\sqt{‎This means, how we put (the pictures) down is correct.}
\end{exe}

Often either the adverb or the adjective can be used as predicates in a \isi{copula} clause. This leads to a slight difference in meaning that can be illustrated with the following examples \refex{ex:This year is good} and \refex{ex:In / during this year it is good}. In case of an adjectival predicate, the \isi{copula} clause refers to the quality denoted by the adjective that is ascribed to the referent of the subject \refex{ex:This year is good}. By contrast, if an adverb is used, it is the situation denoted by the \isi{copula} clause that is attributed a quality and not the subject referent \refex{ex:In / during this year it is good}. Furthermore, in \refex{ex:This year is good} the \isi{noun phrase} \textit{iž dus} functions as the subject, whereas in \refex{ex:In / during this year it is good} it is a temporal adjunct and there is no overt subject.

\begin{exe}

	\ex	\label{ex:This year is good}
	\gll	iž	dus	ʡaˁħ-ce	ca<b>i\\
		this	year	good\tsc{-dd.sg}	\tsc{cop<n>}\\
	\glt	\sqt{This year is good.} (E)
	
	
	\ex	\label{ex:In / during this year it is good}
	\gll	iž	dus	ʡaˁħ-le	ca<b>i\\
		this	year	good\tsc{-advz}	\tsc{cop<n>}\\
	\glt	\sqt{In\slash during this year it is good.} (E)

\end{exe}

The predicate can also be any kind of pronoun, for example a personal pronoun in the \isi{genitive} \refex{ex:‎‎This world was not yours}, a \isi{demonstrative pronoun}, or a personal pronoun in the \isi{absolutive} \refex{ex:‎‎‎This is not me, that is me}, \refex{ex:That is me}. For instance, example \refex{ex:‎‎This world was not yours} demonstrates a situation in which the referent of the \isi{copula} subject does not belong to the referent of the predicate, which is a \isi{genitive} pronoun. It is also possible to add the \isi{cross-categorical suffix} -\textit{ce} to the \isi{genitive} pronoun (\textit{ala-ce}).

\begin{exe}
	\ex	\label{ex:‎‎This world was not yours}
	\gll	iž	dune	ala	akːʷ-i\\
		this	world	\tsc{2sg.gen}	\tsc{cop.neg-hab.pst}\\
	\glt	\sqt{‎‎This world was not yours.}
	
	\ex	\label{ex:‎‎‎This is not me, that is me}
	\gll	iž	akːʷa-di	du,	it=da	du\\
		this	\tsc{cop.neg-1}	\tsc{1sg}	that\tsc{=1}	\tsc{1sg}\\
	\glt	\sqt{‎‎‎This is not me, that is me.}

	\ex	\label{ex:That is me}
	\gll	it	du=da\\
		that	\tsc{1sg=1}\\
	\glt	\sqt{That is me.}
\end{exe}

Negation is expressed by means of the negative \isi{copula} \textit{akːʷ-}, which does not have a \isi{gender} prefix, but inflects for person and tense \refex{ex:‎‎This world was not yours}, \refex{ex:‎‎‎This is not me, that is me}. As was mentioned in the introduction (\refsec{sec:copulaclauses}), the \isi{copula} \textit{ca-b}, the person enclitics, and the past \isi{enclitic} can only express a restricted range of TAME forms. For instance, the imperfective verb \tit{b-urkː-} has the meaning \sqt{find}, and is also used as an auxiliary with the epistemic meaning \sqt{probably, be possible}. The latter use includes \isi{copula} clauses \refex{That is probably their field.}. More examples of \isi{copula} constructions with the verbs labeled ``other auxiliaries'' in \reftab{tab:Items functioning as copulae} are given in \refsec{sec:Other verbs used in copula-functions and as auxiliaries}.

 \begin{exe}
 \ex	\label{That is probably their field.}
\gll ču-la	qu	b-urkː-ar	hel	\\
\tsc{refl.pl-gen}	field	\tsc{n-aux.ipfv-prs.3}	that	\\
 \glt	\sqt{That is probably their field.}
\end{exe}

Note, however, that there are clauses with other auxiliaries that superficially look like \isi{copula} clauses, but represent intransitive clauses This is because the auxiliary functions as an \isi{intransitive verb} and not as a \isi{copula}. For instance, the sentence in \refex{ex:peopleusedtobealive} contains an S argument in the \isi{absolutive} case, \textit{χalq'} `people'. The item following it, \textit{mic'ir} `alive', is a short adjectival stem which cannot be used as adjectival predicate in \isi{copula} clauses. As mentioned above, short forms of \isi{adjectives} only occur as attributes within noun phrases or as part of \is{compound verb}compound verbs. The complex  \textit{mic'ir	b-irχʷ-} has to be treated as one compound predicate with the meaning `be\slash become alive', and therefore the construction does not contain a \isi{copula} verb.
 
 \begin{exe}
	\ex	\label{ex:peopleusedtobealive}
	\gll	χalq'	mic'ir	b-irχʷ-i\\
		people	alive	\tsc{hpl-}be\tsc{.ipfv-hab.pst.3}\\
	\glt	\sqt{The people were alive.}
\end{exe}


% --------------------------------------------------------------------------------------------------------------------------------------------------------------------------------------------------------------------- %

\subsection{Location, existence, and possession}
\label{ssec:Location, existence, and possession}

Location and existence are generally expressed by specialized locational copulas \refex{ex:We were five guys}, \refex{ex:There is nothing more}, \refex{ex:Omar's Khalil is in Kaspisk}, and in negative clauses by the negated \isi{copula} with the \isi{gender} prefix \refex{ex:There is no intention to go back}, which can optionally be preceded by a \isi{locational copula} \refex{ex:There is nothing more} (\reftab{tab:Items functioning as copulae}). If the negated \isi{copula} does not have the \isi{gender} prefix, it cannot express location or existence but only identity, group membership or attribution \refex{ex:‎‎This world was not yours}, \refex{ex:‎‎‎This is not me, that is me}. 

\begin{exe}
	\ex	\label{ex:We were five guys}
	\gll	xujal	duˁrħuˁ	le-d=de	nušːa\\
		five	boy	exist\tsc{-1/2pl=pst}	\tsc{1pl}\\
	\glt	\sqt{We were five guys.}
	

	\ex	\label{ex:There is no intention to go back}
	\gll	c'il	čar	∅-iχʷ-ij	q'ast	b-akː-u\\
		then	back	\tsc{m-}be\tsc{.pfv-inf}	intention	\tsc{n-}\tsc{cop.neg-prs.3}\\
	\glt	\sqt{There is no intention to go back.} (i.e. I do not have the intention).

	\ex	\label{ex:There is nothing more}
	\gll	cik'al	χe-b-akː-u\\
		something	exist.\tsc{down-n-}\tsc{cop.neg-prs.3}\\
	\glt	\sqt{There is nothing more.}

\end{exe}

Sentences expressing location contain spatial adverbials such as adverbs or nominals bearing \is{spatial case}spatial cases, which can precede or follow the locational copulas \refex{ex:Omar's Khalil is in Kaspisk}, \refex{ex:‎They were under the mulberry tree, from that side}. The standard \isi{copula} \textit{ca-b} can also be used \refex{ex:‎He is in prison}, although locational copulas are normally preferred. 

\begin{exe}
	\ex	\label{ex:Omar's Khalil is in Kaspisk}
	\gll	χalil-la	ʡuˁmer	b-ik'-ul	te-w	Kaspisk-le-w\\
		Khalil\tsc{-gen}	Omar	\tsc{hpl-}say\tsc{.ipfv-icvb}	exist\tsc{.away-m}	Kaspisk\tsc{-loc-m}\\
	\glt	\sqt{Omar's Khalil is in Kaspisk.}

	\ex	\label{ex:‎They were under the mulberry tree, from that side}
	\gll	tut-la	kːalkːi-l-gu-b	le-b=de	hek'	šːal-le-rka\\
		mulberry\tsc{-gen}	tree\tsc{-obl-sub-hpl}	exist\tsc{-hpl=pst}	\tsc{dem.up}	side\tsc{-loc-abl}\\
	\glt	\sqt{‎They were under the mulberry tree, from that side.}

\end{exe}

The person enclitics \refex{ex:‎Khadizhat, where are you? I am in Sanzhi} and the past \isi{enclitic} \refex{ex:‎We three were from Daghestan, and there were two Russian guys} can also be used in locational or existential clauses. In addition, the location copulas can attach the person enclitics:

\begin{exe}
	\ex	\label{ex:‎Khadizhat, where are you? I am in Sanzhi}
	\gll	χadižat,	čina-r=de	u?	du	Sanži-r=da	/	Sanži-r	le-r=da\\
		Khadizhat	where\tsc{-f=2sg}	\tsc{2sg}	\tsc{1sg}	Sanzhi\tsc{-f=1}	/ Sanzhi\tsc{-f} exist\tsc{-f=1}\\
	\glt	\sqt{‎Khadizhat, where are you? I am in Sanzhi.} (E)
\end{exe}

For \isi{locational copula} clauses the standard \isi{copula} \textit{ca-b} can also be used \refex{ex:‎He is in prison}, although locational copulas are normally preferred. The exact distribution of existential/locational copulas vs. the standard \isi{copula} \textit{ca-b} needs to be determined by future research. 

 \begin{exe}
	\ex	\label{ex:‎He is in prison}
	\gll	hež	tusnaq-le-w	ca-w\\
		this	prison\tsc{-loc-m}	\tsc{cop-m}\\
	\glt	\sqt{‎He is in prison.}
\end{exe}

 
The expression of \isi{possession} implies the existence of the possessed item. This means that when talking about any types of \isi{possession} that one has, be it objects or relatives, the locational copulas are used. In the unmarked case this is \tit{le-b} \refex{ex:‎(I) have ten children; they are seven girls and three boys}. The possessor occurs in the \isi{genitive} case.

\begin{exe}
	\ex	\label{ex:‎(I) have ten children; they are seven girls and three boys}
	\gll	durħ-ne	le-b	wec'al,	weral	rursːi	ca<b>i	ʡaˁbal	durħuˁ\\
		boy\tsc{-pl}	exist\tsc{-hpl}	ten	seven	girl	\tsc{cop<hpl>}	three	boy\\
	\glt	\sqt{‎(I) have ten children; they are seven girls and three boys.}
\end{exe}

The following minimal pair illustrates the difference between the two types of copulas. The first sentence in \refex{ex:That is my house} requires an identificational interpretation. It can, for instance, be used when showing and identifying the house. The more literal translation of the second sentence \refex{ex:I have a house} would be \sqt{With\slash at me there is a house.} or \sqt{My house exists.} If the \isi{genitive} pronoun is a predicate instead \refex{ex:‎‎This world was not yours} or if other semantic components play a role, the other copulas are used. 

\begin{exe}
	\ex	\label{ex:That is my house}
	\gll	het	di-la	qal	ca-b\\
		that \tsc{1sg-gen}	house	\tsc{cop-n}\\
	\glt	\sqt{That is my house.} (E)

	\ex	\label{ex:I have a house}
	\gll	di-la	qal	le-b\\
		\tsc{1sg-gen}	house	exist\tsc{-n}\\
	\glt	\sqt{I have a house.} (E)
\end{exe}


Less common ways of constructing locational and \isi{existential copula} clauses or \isi{copula} clauses expressing \isi{possession} are available by means of the other auxiliaries given in \reftab{tab:Items functioning as copulae}. The example in \refex{ex:‎‎‎Once upon a time there were three boys} represents the traditional opening formula for fairy tales and is thus not a normal existential clause. The verb \textit{b-ug-} can express indirect evidential semantics, which is often found in fairy tales. The verb \textit{b-iχʷ-} (\tsc{pfv})\slash\textit{b-irχʷ-} (\tsc{ipfv}) \sqt{be, become, happen, can} is used, among other things, to express epistemic modal constructions including different subtypes of \isi{copula} clauses with a modal meaning \refex{This (picture) must be here.}. The sentence in \refex{ex:‎We had another thing in the mountains} expresses not only past time reference, but also habituality and therefore also contains the auxiliary \tit{b-irχʷ-}, because neither the standard \isi{copula} \textit{ca-b} nor the past \isi{enclitic} =\textit{de} can express this specific combination of temporal and aspectual meanings.

\begin{exe}
		\ex	\label{ex:‎‎‎Once upon a time there were three boys}
	\gll	b-už-ib	ca-b,	b-už-ib-le=kːu	ʡaˁbal	durħuˁ\\
		\tsc{hpl-}stay\tsc{-pret}	\tsc{cop-hpl}	\tsc{hpl-}be\tsc{-pret-cvb=}\tsc{cop.neg}	three	boy\\
	\glt	\sqt{‎‎‎Once upon a time there were three boys.} (lit. \sqt{there were, there were not})
	
	 \ex	\label{This (picture) must be here.}	
\gll hež	heštːu-b	b-irχʷ-an	ca-b \\
this	here-\tsc{n}	\tsc{n}-be.\tsc{ipfv-ptcp}	\tsc{cop-n} \\
 \glt	\sqt{This (picture) must be here.}
 
 	\ex	\label{ex:‎We had another thing in the mountains}
	\gll	a	c'il	ca	ca=ra	b-irχʷ-i	nišːa-la	dubur-ri-cːe-b\\
		and	then	one	one\tsc{=add}	\tsc{n-}be\tsc{.ipfv-hab.pst}	\tsc{1pl-gen}	mountain\tsc{-obl-in-n}\\
	\glt	\sqt{‎We used to have something else in the mountains.}
 \end{exe}


% --------------------------------------------------------------------------------------------------------------------------------------------------------------------------------------------------------------------- %

\subsection{Copula clauses without a subject}
\label{ssec:Copula clauses without a subject}

Copula clauses with temporal or spatial adverbials can occur without an overt \isi{copula} subject. They only contain a \isi{copula} predicate:

\begin{exe}
	\ex	\label{ex:It used to be very cold in the winter}
	\gll	ganilla	c'aq'-le	b-uχːar-re	b-irχʷ-iri\\
		in.winter	strong\tsc{-advz}	\tsc{n-}cold\tsc{-advz}	\tsc{n-}become\tsc{.ipfv-hab.pst}\\
	\glt	\sqt{It used to be very cold in the winter.}

	\ex	\label{ex:‎It was in the mountains.}
	\gll	dubur-t-a-cːe-b=de\\
		mountain\tsc{-pl-obl-in-n=pst}\\
	\glt	\sqt{‎It was in the mountains.} (E)
\end{exe}

More generally, \isi{copula} clauses with predicates expressed by \isi{manner adverbs} do not require a subject, but can be impersonal \refex{ex:‎It is difficult, it is bad}, \refex{ex:‎When it is warm (i.e. in warm places) the houses are built like this}. The \isi{gender} agreement affix in such clauses is invariably \tit{b}, since this is the default agreement affix (\refsec{sec:Gender/number agreement}). It is possible to add a \isi{dative} argument fulfilling the semantic role of \isi{experiencer} or beneficiary \refex{ex:I am well}.

\begin{exe}
	\ex	\label{ex:‎It is difficult, it is bad}
	\gll	qihin-ne	ca-b,	wahi-l	ca-b	\\
		difficult\tsc{-advz}	\tsc{cop-n}	bad\tsc{-advz}	\tsc{cop-n}	\\
	\glt	\sqt{‎It is difficult, it is bad.}

	\ex	\label{ex:‎When it is warm (i.e. in warm places) the houses are built like this}
	\gll	guna=qːel	ca-b,	hel-itːe	daˁʡle	b-arq'-ib	qal\\
		warm=when	\tsc{cop-n}	that\tsc{-advz}	as	\tsc{n-}do\tsc{.pfv-pret}	house	\\
	\glt	\sqt{‎When it is warm (i.e. in warm places) the houses are built like this.}

	\ex	\label{ex:I am well}
	\gll	dam ʡaˁħ-le ca-b\\
		\tsc{1sg}	good\tsc{-adv}	\tsc{cop-n}\\
	\glt	\sqt{I am well.}
\end{exe}


% --------------------------------------------------------------------------------------------------------------------------------------------------------------------------------------------------------------------- %

\subsection{Copula clauses without a copula}
\label{ssec:Copula clauses without a copula}

Copula clauses obligatorily require a \isi{copula} item (\reftab{tab:Items functioning as copulae}), otherwise they are ungrammatical: 

\begin{exe}
	\ex[*]{	\label{ex:I am a master}
	\gll	 du	ustːa\\
			\tsc{1sg}	master\\
	\glt	\sqt{(Intended meaning: I am a master.)} (E)}

	\ex[*]{	\label{ex:This, who is it}
	\gll		ij,	ča	iž?\\
			this	who	this \\
	\glt	\sqt{(Intended meaning: This, who is it?)} (E)}
\end{exe}

In \isi{copula} constructions that have third person subjects and present time reference or habitual meaning, the \isi{copula} can be omitted when one of the pragmatic predicative \is{particle}particles is used. This can be either one of the three interrogative enclitics if the \isi{copula} clause is a question (polar question, \isi{content question}, embedded question) (\refcpt{cpt:Interrogative clauses}), or the \isi{modal enclitic} \tit{=q'al} (\refsec{ssec:The enclitic =q'al}). This is possible because the \isi{modal enclitic} and the interrogative enclitics belong, just like the person enclitics =\textit{da} and =\textit{de} and the past \isi{enclitic} =\textit{de}, to the predicative \is{particle}particles that can head finite clauses (\refsec{sec:Predicative particles}). The following examples show the use of \tit{=q'al} \refex{ex:‎That is a Russian word}, a \isi{content question} \refex{ex:Where are your witnesses} and a polar question \refex{ex:Is this the father}.

\begin{exe}
	\ex	\label{ex:‎That is a Russian word}
	\gll	ʡuˁrus	ʁaj-la=q'al	il\\
		Russian	word\tsc{-gen=mod}	that	\\
	\glt	\sqt{‎That is a Russian word.}

	\ex	\label{ex:Where are your witnesses}
	\gll	čina-b=e	ala	biq'ru-me?\\
		where\tsc{-hpl=q}	\tsc{2sg.gen}	witness\tsc{-pl}\\
	\glt	\sqt{Where are your witnesses?}

	\ex	\label{ex:Is this the father}
	\gll	hež	atːa=w	iž?\\
		this	father\tsc{=q}	this\\
	\glt	\sqt{Is this the father?}
\end{exe}

Similarly, in \isi{copula} clauses that function as embedded \isi{questions} or assertions expressing epistemic modality (uncertainty), the \isi{embedded question marker} is used \refex{ex:‎Show what herbs these are} and the \isi{copula} is absent. Sentence \refex{ex:Now it is probably two years (that have passed by)} does not show an embedded question, but an epistemic uncertainty construction (\refsec{sec:Subordinate questions}).

\begin{exe}
	\ex	\label{ex:‎Show what herbs these are}
	\gll	či-d-až-aq-a	hari	[ce	q'ar=el]!\\
		\tsc{spr-npl-}see\tsc{.pfv-caus-imp}	let's	what	herbs\tsc{=indq}\\
	\glt	\sqt{‎Show what herbs these are!}

	\ex	\label{ex:Now it is probably two years (that have passed by)}
	\gll	hana	k'ʷel	dus=el\\
		now	two	year\tsc{=indq}\\
	\glt	\sqt{Now it is probably two years (that have passed by).}
\end{exe}

The sole use of a pragmatic predicative \isi{particle} is impossible if the \isi{copula} subject is first or second person. In such cases, the predicative person marker needs to occur before the question \isi{enclitic} and cannot be omitted:

\begin{exe}
	\ex	\label{ex:Are you afraid}
	\gll	uruχ-le=de=w?\\
		fear\tsc{-advz=2sg=q}\\
	\glt	\sqt{Are you afraid?}
\end{exe}

When two \isi{copula} clauses are coordinated, the \isi{copula} can be omitted in one of the clauses, usually the second clause \refex{ex:‎‎‎I am good, you (are) bad} (see \refsec{sec:Coordination of clauses other phrases} for one more example).

\begin{exe}
	\ex	\label{ex:‎‎‎I am good, you (are) bad}
	\gll	du	ʡaˁħ-ce=da,	u	wahi(=de)\\
		\tsc{1sg}	good\tsc{-dd.sg=1}	\tsc{2sg}	bad(\tsc{=2sg})\\
	\glt	\sqt{‎‎‎I am good, you (are) bad.} (E)
\end{exe}


%%%%%%%%%%%%%%%%%%%%%%%%%%%%%%%%%%%%%%%%%%%%%%%%%%%%%%%%%%%%%%%%%%%%%%%%%%%%%%%%


\section{Grammatical relations}
\label{sec:Grammatical relations}

There are only a few studies on grammatical relations in Dargwa varieties so far by Nina Sumbatova \citep{Sumbatova2014, Sumbatova2017}, but there is a considerable amount of literature on grammatical relations in other East Caucasian languages, and there are works from a comparative perspective that include Dargwa (see \citealp{Forker2017} for a recent overview). Case studies of individual languages are often centered on the question whether the investigated language(s) is only morphologically \isi{ergative}, or whether it also shows indications of syntactic ergativity (cf. \citealp{Nichols1980}, \citealp{Crisp1983}, \citealp{Comrie.etal2011}). The majority of scholars state that ergativity is mostly restricted to morphology. \citet{Kibrik1985, Kibrik1997, Kibrik2003} concludes that East Caucasian languages belong to the so-called ``role-dominated'' languages \citep[123]{Foley.vanValin1984} in which the marking of arguments is semantically motivated. 

In this section, I will briefly discuss the constructions (or variables) displayed in \reftab{tab:Grammatical relations in Sanzhi Dargwa}. The section does not present and discuss the data, but contains only cross-references to the relevant sections in this grammar that contain data for most of the constructions given in the table. See \citealp{Forker2019} for data and analysis of more constructions such as complement control or \isi{quantifier} floating and a detailed account of grammatical relations in Sanzhi. I will not analyze \isi{word order} because \isi{word order} on the clausal level strongly depends on the \isi{information structure} and not on grammatical relations (\refsec{sec:Constituent order at the clause level and information structure}). 


\begin{table}
	\caption{Grammatical relations in Sanzhi Dargwa}
	\label{tab:Grammatical relations in Sanzhi Dargwa}
	\small
	\begin{tabularx}{0.98\textwidth}[]{%
		>{\raggedright\arraybackslash}p{90pt}
		>{\raggedright\arraybackslash}X
		>{\raggedright\arraybackslash}p{100pt}}
		
		\lsptoprule
			Construction			&		Grammatical relations	&		Constraints\\
		\midrule
			Person agreement
		&	S=A=P
		&	TAM forms, person hierarchy 1, 2>3\\[2mm]
   
			Gender/\isi{number}\newline\hspace*{0.5em}agreement
		&	S=P vs. A (but mostly only for S and P in the \isi{absolutive})
		&	case (predominantly \isi{absolutive})\\[2mm]
		
			Case
		&	S=P vs. A (but this depends on the predicate class)
		&	case-defined predicate class (\reftab{tab:Valency classes}), clause type\\[2mm]

			Imperative
		&	S=A vs. P\footnote{But the evidence for \is{affective verb}affective verbs is inconsistent because \isi{imperative} formation of \is{affective verb}affective verbs is often impossible for semantic reasons.}
		&	semantic predicate class\\[2mm]
  
			Complement control
		&	S=A vs. P
		&	{}\\[1mm]
  
			Reflexivization,\newline\hspace*{0.5em}Reciprocalization
		&	S=A=P for \isi{experiential} verbs and for default \is{transitive verb}transitive verbs with complex reflexive\slash \is{reciprocal pronoun}reciprocal pronouns; S=A otherwise
		&	case-defined predicate class\\[2mm]
   
			\mbox{Conjunction reduction}
		&	tendency for  S=A vs. P
		&	no known constraints\\[2mm]
   
			Relativization
		&	not sensitive to grammatical relations 
		&	{}\\[2mm]
 
			Antipassive
		&	not sensitive to grammatical relations 
		&	case-defined predicate class and verb semantics, TAM form\\[2mm]
  
			Causativization
		&	S=A vs. P
		&	{}\\[2mm]
	   
			Quantifier floating
		&	S=P vs. A (\refsec{ssec:Floating modifiers})
		&	case (only \isi{absolutive})\\
		\lspbottomrule
	\end{tabularx}
\end{table}

We can identify three alignment types in Sanzhi Dargwa: \isi{ergative} alignment, accusative alignment, and neutral alignment. Additionally, there are a \isi{number} of constructions in Sanzhi that are not sensitive to grammatical relations. The most important constraint is the case-defined predicate class, that is, the distinction between canonical transitive, affective, extended intransitive, and other verbs. These \is{valency class}valency classes of verbs are defined on the basis of case assignment patterns to the arguments, and not so much on the basis of the meaning of the predicates. In other words, cases have a high semantic load and the choice of one case suffix over the other largely depends on the semantic contribution of the cases. Thus, Sanzhi Dargwa confirms once more the fact that the semantic impact of cases for East Caucasian languages should not be underestimated.

Ergative alignment, labeled as S=P vs. A in \reftab{tab:Grammatical relations in Sanzhi Dargwa}, is basically found in the morphology, namely in the \isi{gender} agreement and the case marking. There is a large \isi{number} of bivalent and \is{trivalent verb}trivalent verbs that assign \isi{ergative} case to their A argument, although not all bivalent and \is{trivalent verb}trivalent verbs belong to this class. Additionally, there are even more verbs whose S and P arguments trigger \isi{gender}/\isi{number} agreement because the arguments bear the \isi{absolutive} case. Outside the realm of morphology there are almost no indications for ergativity, apart perhaps from \isi{quantifier} floating (\refsec{ssec:Floating modifiers}) and causativization (\refsec{sec:Causativization}). Instead, accusative alignment (symbolized with S=A vs. P), neutral alignment (S=A=P) and no alignment (no \is{grammatical role}grammatical roles identifiable) are found. Person agreement and reflexivization\slash reciprocalization are neutral since S, A, P and T are not distinguishable. They only behave differently from G, but this is not relevant for the determination of \is{grammatical role}grammatical roles. All four macroroles S, A, P and T can control \isi{person agreement} (\refsec{sec:Person agreement}) or reflexive and \is{reciprocal pronoun}reciprocal pronouns (\refcpt{cpt:Reflexive and reciprocal constructions}) and thus we have neutral alignment. In contrast, relativization largely depends on pragmatics and a suitable context and is not sensitive to grammatical relations because a large variety of positions (S, A, P, G, T, other) can be relativized. Accusativity is found with imperatives because both S and A can be subjects in \isi{imperative} clauses, but not P or any other position (\refsec{sec:imperative}). This is not surprising and frequently found in \isi{ergative} as well as in accusative languages, and some authors do not consider imperatives to represent suitable test constructions for establishing \is{grammatical role}grammatical roles, e.g. \citet[131]{Dixon1994}. Furthermore, complement control (\refsec{sec:Argument control in complement constructions}) and \isi{conjunction} reduction in clauses with the preterite converb show some accusative traits because S and A are always suitable controllers of arguments in complement clauses or converbal clauses, but P is largely excluded. Similarly, causativization can be analyzed as distinguishing between S/A on the one side and P on the other side because it is never the P or the T that is affected when bivalent or trivalent predicates are causativized (\refsec{sec:Causativization}). P arguments remain unchanged (because P and T essentially have the same morphosyntactic properties), whereas S changes to P, and A changes to G under causativization, such that causativization can perhaps be taken as a further indicator of an S/A pivot. The \isi{antipassive} is not a suitable test construction because its application is restricted to the class of \is{transitive verb}transitive verbs, excluding all other \is{valency class}valency classes, such that we cannot check how S would be treated.

To sum up, there is no justification for establishing a category of \isi{ergative} subject that would comprise S and P, and thus Sanzhi Dargwa is only morphologically \isi{ergative}. This claim is not surprising but supports what has been previously stated for the East Caucasian languages. The only indications for syntactic accusativity are complement constructions and causativization, which is not enough for establishing a category of subject comprising S and A as we know it from European languages. However, simple \is{reflexive construction}reflexive constructions and \isi{imperative} could be viewed as further, though weaker indications for singling out S and A in contrast to P. At the basis of textual frequency even \isi{person agreement} shows a tendency to occur predominantly with S and A controllers in natural texts because P arguments that are second person are relatively rare. 

In this grammar, I use the term ``subject-like'' or even sometimes ``subject'' in order to refer to S and A arguments, whereas P arguments are called ``object-like'' or ``object''. This terminology has been chosen for reasons of convenience and familiarity. It has to be viewed against the background of the discussion of \is{grammatical role}grammatical roles in Sanzhi as given in this Section.


%\begin{exe}
%	\ex	\label{ex:}
%	\gll	\\
%		\\
%	\glt	\sqt{}
%\end{exe}
