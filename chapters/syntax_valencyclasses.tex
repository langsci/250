\chapter{Valency classes and modification of valency patterns}
\label{cpt:Verb valency classes}


%%%%%%%%%%%%%%%%%%%%%%%%%%%%%%%%%%%%%%%%%%%%%%%%%%%%%%%%%%%%%%%%%%%%%%%%%%%%%%%%
\section{Valency classes}
\label{sec:Verb valency classes}

\subsection{Introduction}
\label{sec:valencyclassesintro}

Valency classes cross-cut the morphological classes of verbs. This means that the morphological classes (underived verb stems with or without preverbs, derived verbs, compound verbs, see  \refsec{sec:Overview about the general morphological structure of verbs} and \refcpt{cpt:verbformation}) distribute over the valency classes with probably a preference for the simple underived verbs to occur in the intransitive, the transitive and to a somewhat lesser extent the affective valency class.

I will categorize verbs into valency classes according to two main criteria: (i) the number of arguments and (ii) the case marking of the subject-like argument. By `subject-like argument' I refer to the argument of the simple clause that has the most subject properties as opposed to all other arguments (see \refsec{sec:Grammatical relations} for more details). Subject-like arguments are marked with one of the three cases absolutive, ergative or dative. I use the terms \sqt{one-place} or \sqt{monovalent}, \sqt{two-place} or \sqt{bivalent}, and \sqt{three-place} or \sqt{trivalent} for referring to the number of semantic arguments required by the verbs. The basic valency classes and the case marking of the subject-like argument are summarized in \reftab{tab:Valency classes}.



\begin{table}
	\caption{Valency classes and case marking of subject-like arguments}
	\label{tab:Valency classes}
	\small
	\begin{tabularx}{0.98\textwidth}[]{%
		>{\raggedright\arraybackslash}p{52pt}
		>{\raggedright\arraybackslash}X
		>{\raggedright\arraybackslash}X
		>{\raggedright\arraybackslash}p{60pt}}
		
		\lsptoprule
		{} & \multicolumn{3}{c}{subject-like argument}\\
			\# valency		&	absolutive				&	dative			&	ergative\\
		\midrule
			monovalent	&	intransitive (\refsec{sec:Intransitive verbs})		&	monovalent affective (\refsec{sec:Monovalent affective verbs and exceptional monovalent constructions})	&	one verb \refex{ex:precipitation}\\	   
			bivalent		&	extended intransitive (\refsec{sec:Extended intransitive verbs})		&	bivalent affective (\refsec{sec:Bivalent affective verbs})		&	transitive (\refsec{sec:Transitive verbs})\\
			trivalent	&	\#	&		\#		&	ditransitive (\refsec{sec:Extended transitive verbs and ditransitive verbs})\\ 
		\lspbottomrule
	\end{tabularx}
\end{table}

The term \sqt{extended intransitive} refers to two-place predicates that, in addition to an argument in the absolutive, have a further argument in the dative or another case; \sqt{extended transitive} verbs are three-place verbs that besides having two arguments bearing the cases that are also used for transitive verbs, have an additional argument marked with the dative or in some other way \citet[122\tnd123]{Dixon1994}. Thus, extended transitive verbs are ditransitive verbs. Furthermore, I use the term \sqt{affective predicates} for a clear-cut class of mostly experiential predicates that express the experiencer argument in the dative and the stimulus argument, if there is one, in the absolutive. Affective verbs typically form their own valency class in East Caucasian (see, e.g. \citealp{Comrie.vandenBerg2006}; \citealp{Ganenkov2006}; \citealp{Comrie.Forker.KhalilovaInPressb}). One might hypothesize that they belong to the class of extended intransitive verbs. However, if one applies the commonly used test for subjecthood to the extended intransitive verbs and the affective verbs it immediately becomes clear that with the former class it is the absolutive argument that exhibits most subject properties whereas with the latter class it is the dative argument. For more information on grammatical relations in Sanzhi see \refsec{sec:Grammatical relations} and \citealp{Forker2019}.

\reftab{tab:Major valency classes} provides an overview of the major valency classes discussed in this chapter; some minor classes are not listed, but discussed below. All verbs in the table and in the following subsections are presented in the order imperfective\slash perfective if they have two stems. Otherwise the single stem that is unspecified for aspect is given. In the table and in this chapter as well as elsewhere in the grammar I will use the following letters as mnemonics for macro roles (see \citet{Bickel2011} and \citet{Bickel.etal2015}): 
%
\begin{itemize}
	\item S = single argument of an intransitive predicate or absolutive argument of an extended intransitive predicate
	\item A = the argument with the most agentive properties of a bivalent or trivalent predicate (except for extended intransitive predicates, for which S is used)
	\item P = the argument with the least agentive or most patientive properties of a bivalent predicate
	\item G = goal-like argument of a trivalent predicate (e.g. recipient)
	\item T = more stationary theme-like argument of a trivalent predicate 
	\end{itemize}
	%
Subject-like arguments are of the type S or A. Note that S occurs with monovalent and bivalent verbs, which might seem slightly unusual. My reason for using the label S in this way is case marking, because all arguments falling under this label are marked by the absolutive case, which leads to a range of common morphosyntactic properties. For more details on grammatical relations see \refsec{sec:Grammatical relations}.

%
\begin{table}
	\caption{Major valency classes}
	\label{tab:Major valency classes}
	\small
	\begin{tabularx}{0.98\textwidth}[]{%
		>{\raggedright\arraybackslash}p{90pt}
		>{\raggedright\arraybackslash}p{50pt}
		>{\raggedright\arraybackslash}X}
		
		\lsptoprule
			case marking		&	number~of		&	predicates\\
			patterns		&	arguments		&	and examples\\
		\midrule
		\multicolumn{3}{c}{\tbf{\tsc{monovalent predicates}}}\\
 			\multicolumn{3}{l}{\tbf{intransitive (absolutive)}}\\ 
			S-\tsc{abs}
		&	1
		&	\tit{b-ubk'-\slash b-ebk'-} \sqt{die}; \tit{či-r-ha-b-ulq-\slash či-r-ha-b-uq-} \sqt{vomit} \refex{ex:My arms got tired}, \refex{ex:If it was (the time) to die, I would have died long ago}\\[2mm]

 			\multicolumn{3}{l}{\tbf{monovalent affective verbs (dative)}}\\
			S-\tsc{dat}
		&	1
		&	\tit{ʡaˁħ-le ca-b} \sqt{feel good, be well}; \tit{c'aχ ka-b-ircː-\slash c'aχ ka-b-icː-} \sqt{feel ashamed} \refex{ex:I feel bad there},  \refex{ex:At night I had no sleep}\\[2mm]
				   	\midrule
				   	\multicolumn{3}{c}{\tbf{\tsc{bivalent predicates}}}\\
			\multicolumn{3}{l}{\tbf{extended intransitive (absolutive + dative\slash spatial case)}}\\	   
			S-\tsc{abs}, P-\tsc{dat}
		&	2
		&	\tit{kːač b-irk-}\slash\tit{kːač b-ik-} \sqt{touch}, \tit{gu-lik'-} \sqt{listen to} \refex{ex:I am listening to her/his song}\\	   
			S-\tsc{abs}, \hspace*{0.5em}P-\tsc{in-lative}/-\tsc{dat}
		&	2
		&	\tit{b-ik'ʷ-} \sqt{talk to}; \tit{xʷit' b-ik'ʷ-} \sqt{whistle at} \refex{ex:The boy began to whistle and to cry to the dogs in the village}, \refex{ex:The grandfather is calling me@27a}\\
			S-\tsc{abs}, \hspace*{0.5em}P-\tsc{ante-ablative}
		&	2
		&	\tit{uruχ b-irχʷ-\slash uruχ b-iχʷ-} \sqt{become/be afraid of}; \tit{uruc b-irχʷ-\slash uruc b-iχʷ-} \sqt{be/become ashamed\slash embarrassed of} \refex{ex:Are you afraid of your wife? he says.}\\[2mm]
   
			\multicolumn{3}{l}{\tbf{bivalent affective verbs (dative + absolutive\slash other)}}\\
			A-\tsc{dat/erg}, P-\tsc{abs}
		&	2
		&	\tit{či-b-ig-\slash či-b-ag-} \sqt{see}; \tit{b-irʁ-\slash b-arʁ-} \sqt{understand} \refex{ex:You (pl.) bored me, I said},  \refex{ex:I know a good place}\\
			A-\tsc{dat}, \hspace*{0.5em}P-\tsc{ante-ablative}
		&	2
		&	\tit{c'aχ-le ca-b} \sqt{to feel/be ashamed in front of}; \tit{b-irt'-\slash b-et'-} \sqt{long for, miss} \refex{ex:Hey, Kurban, I am ashamed in front of you}\\[2mm]

			\multicolumn{3}{l}{\tbf{transitive (ergative + absolutive)}}\\
			A-\tsc{erg}, P-\tsc{abs}
		&	2
		&	\tit{b-irc-\slash b-ic-} \sqt{sell}; \tit{b-urχ-\slash b-arχ-} \sqt{sew} \refex{ex:Did these people beat you up},  \refex{ex:The boy stopped him}\\[2mm]

			\multicolumn{3}{l}{\tbf{other bivalent verbs (ergative + dative)}}\\
			A-\tsc{erg}, P-\tsc{dat}
		&	2
		&	\tit{b-aˁq-\slash b-uˁrq-} \sqt{hit}; \tit{zaˁnʁ d-aˁq-\slash zaˁnʁ d-uˁrq-} \sqt{phone} \refex{ex:Brother called you}\\[2mm]
	\midrule
	\multicolumn{3}{c}{\tbf{\tsc{trivalent predicates}}}\\
			\multicolumn{3}{l}{\tbf{extended transitive (ergative + absolutive + other)}}\\
			A$_{ditr}$-\tsc{erg}, T-\tsc{abs}, \hspace*{0.5em}G-\tsc{dat}/-\tsc{in-lative}
		&	3
		&	\tit{lukː-\slash b-ikː-} \sqt{give}; \tit{či-b-iž-aq-\slash či-b-až-aq-} \sqt{show}; \tit{haʔ-\slash herʔ-} \sqt{say, tell}; \tit{b-urs-} \sqt{say, tell}; \tit{xar b-irʁ-\slash xar b-eʁ-} \refex{ex:The snake gave it to me}, \refex{ex:He hit with the fist on the jaw of his wife}, \refex{ex:Why did he not tell us that the water does not flow, they said}\\ 
		\lspbottomrule
	\end{tabularx}
\end{table}


As \reftab{tab:Valency classes} shows, monovalent verbs have three possibilities for marking their single argument. The majority of the monovalent verbs assign the absolutive case to the single argument (\refsec{sec:Intransitive verbs}), though dative or, in case of one verb, ergative are also possible (\refsec{sec:Monovalent affective verbs and exceptional monovalent constructions}).
%


\reftab{tab:Case-marking of arguments of bivalent predicates} summarizes the case-marking patterns available in constructions with bivalent predicates, because they are the largest and most heterogeneous group. The columns represent the possible cases for subject-like arguments, which can be absolutive (S) or ergative (A), or dative (A). The rows display the possible cases for P arguments (absolutive, dative, genitive, spatial cases, ergative). As the table shows, the absolutive case is the most versatile case that can be combined with all other cases and encodes S, A or P, but the ergative is also quite flexible. 
%
\begin{table}
	\caption{Case-marking of arguments in constructions with bivalent predicates}
	\label{tab:Case-marking of arguments of bivalent predicates}
	\small
	\begin{tabularx}{0.98\textwidth}[]{%
		>{\raggedright\arraybackslash}p{45pt}
		>{\raggedright\arraybackslash}p{75pt}
		>{\raggedright\arraybackslash}X
		>{\raggedright\arraybackslash}X}
		
		\lsptoprule
			{}		&	absolutive	S		&	ergative A				&	dative A\\
		\midrule
			absolutive P	&	y  \refex{ex:The people turned into stones}, \refex{ex:I will be 75 years old}		&	\mbox{y (transitive)}	&	y (affective)\\
			dative P		&	y (extended intr.)		&	y					&	\#\\
			genitive P	&	y	\refex{ex:The hips will be a strong medicine}			&	y					&	\#\\
			spatial	P	&	y (extended intr.)		&	y					&	y\\
			ergative P	&	y (antipassive)	&	\#						&	\#\\	 
		\lspbottomrule
	\end{tabularx}
\end{table}

Note that \reftab{tab:Case-marking of arguments of bivalent predicates} conflates basic valency classes for bivalent predicates (i.e. extended intransitive, transitive, affective) with a number of other special constructions, which are available for some predicates of the basic valency classes (antipassive, constructions with absolutive, dative or genitive Ps).  

The first row in \reftab{tab:Case-marking of arguments of bivalent predicates} lists all constructions that consist of an S argument in the absolutive and a further P argument. These clauses with extended intransitive predicates (\refsec{sec:Extended intransitive verbs}), but also antipassives (\refsec{sec:Antipassive}), and two minor constructions with absolutive and genitive P arguments are described in \refsec{sec:Extended intransitive verbs}. In all clauses with bivalent verbs and absolutive S arguments it is the S that controls gender agreement. Gender agreement with any other arguments is ungrammatical.

The second row in \reftab{tab:Case-marking of arguments of bivalent predicates} contains all constructions with ergative A arguments and P arguments with various cases. First of all, the P argument can have the absolutive case (standard transitive verbs including causativized intransitive verbs, and, in certain TAM forms, affective verbs, \refsec{sec:Transitive verbs}, \refsec{sec:Causativization}, \refsec{sec:Bivalent affective verbs}). A few bivalent verbs with an ergative agent (A) require a goal or beneficiary argument marked with the dative or \tsc{in}-lative, or even an experiencer in the genitive \refex{ex:I am hugging you}, which represents the P argument. These verbs commonly have lexicalized direct objects in the absolutive case that are invariable parts of the compound verb and therefore do not count as arguments (\refsec{sec:Bivalent verbs with frozen objects}).

Third, bivalent verbs with A arguments taking the dative are, as mentioned above, mostly affective verbs that have a P (stimulus) in the absolutive. 


%%%%%%%%%%%%%%%%%%%%%%%%%%%%%%%%%%%%%%%%%%%%%%%%%%%%%%%%%%%%%%%%%%%%%%%%%%%%%%%%

\subsection{Intransitive verbs}
\label{sec:Intransitive verbs}

Intransitive verbs are one-place verbs. The single argument occurs in the absolutive and controls the gender/number agreement and the person agreement on the verb. Example verbs are given in \refex{ex:intransitiveverbs} and examples sentences in \refex{ex:My arms got tired}-\refex{ex:Our (people) were fasting like nowadays1}. 
%
\begin{exe}
	\ex	\label{ex:intransitiveverbs}
	\begin{xlist}
		\ex	\tit{ʡaˁlħ-}\slash\tit{ʡaˁħ-} \sqt{fly}
		\ex	\tit{b-ubk'-}\slash\tit{b-ebk'-} \sqt{die}
		\ex	\tit{b-ilš-}\slash\tit{b-iš-} \sqt{die out, extinguish}
		\ex	\tit{ka-b-ilsː-}\slash\tit{ka-b-isː-} \sqt{lie down, sleep}
		\ex	\tit{ʡiˁbħ-}\slash\tit{ʡaˁbħ-} \sqt{get tired}
		\ex	\tit{rurt-}\slash\tit{b-ert-} \sqt{curdle, solidify}
		\ex	\tit{či-r-ha-b-ulq-}\slash\tit{či-r-ha-b-uq-} \sqt{vomit} 
		\ex	\tit{luqː-\slash b-elqː-} \sqt{be, become full, fed up}
		\ex	\tit{t'aš b-ircː-\slash t'aš b-icː-} \sqt{stop}
		\ex	\tit{uruc b-ik'ʷ-; uruc b-irχʷ-\slash uruc b-iχʷ-} \sqt{be/become embarrassed, ashamed}	
	\end{xlist}

	\ex	\label{ex:My arms got tired}
	\gll	nuˁq-be	ʡaˁbħ-ib ca<d>i\\
		arm\tsc{.obl-pl}	get.tired\tsc{.pfv-pret} be\tsc{<npl>}\\
	\glt	\sqt{My arms got tired.}

	\ex	\label{ex:If it was (the time) to die, I would have died long ago}
	\gll	ažal	d-iχʷ-ar-del,	dawnu	r-ubk'-a-di\\
		death	\tsc{npl-}be\tsc{.pfv-prs-cond.pst}	long.ago	\tsc{f-}die\tsc{.ipfv-hab-1}\\
	\glt	\sqt{If it was (the time) to die, I (fem.) would have died long ago.}

	\ex	\label{ex:I got fed up}
	\gll	w-elqː-un-ne=da\\
		\tsc{m-}sate\tsc{.pfv-pret-cvb=1}\\
	\glt	\sqt{I (masc.) got fed up.}

\end{exe}

There are a variety of intransitive verbs that are compounds and contain a nominal part (\refsec{ssec:compoundswithnouns}). The nominal part, however, does not function as argument of the verb. It most frequently appears in the absolutive case \refex{ex:Our (people) were fasting like nowadays1}, but the genitive case is also possible \refex{ex:He built a house and married.}, or the \tsc{loc}-lative or spatial postpositions/adverbials. Note that in \refex{ex:Our (people) were fasting like nowadays1} the absolutive argument that controls the agreement has been omitted and only its genitive modifier appears in the clause.

\begin{exe}
	\ex	\label{ex:intransitiveverbsCompoundVerbs}
	\begin{xlist}
		\ex	\tit{dum b-urc-\slash dum b-uc-} \sqt{fast}
		\ex	\tit{čːal b-ik'ʷ-; čːal b-ulq-\slash čːal b-uq-} \sqt{argue, quarrel}
		\ex	\tit{sːiħ b-ik'ʷ-} \sqt{breath}
		\ex \textit{qal-la b-iχʷ-ij\slash qal-la ka-b-iž-ij} \sqt{get married} 
		\ex \textit{waˁʡda-la b-iχʷ-ij} \sqt{negotiate, conspire}
	\end{xlist}
	
		\ex	\label{ex:Our (people) were fasting like nowadays1}
	\gll	nišːa-la	dum	b-urc-ul=q'al	hana	daˁʡle\\
		\tsc{1pl-gen}	fasting	\tsc{hpl-}keep\tsc{.ipfv-icvb=mod}	now	as\\
	\glt	\sqt{Our (people) were fasting like nowadays.}
	
			\ex \label{ex:He built a house and married.}
		\gll b-arq'-ib-le qal, qal-la ka-jž-ib ca-w\\
		\tsc{n}-make.\tsc{pfv-pret-cvb} house house-\tsc{gen} \tsc{down}-be.\tsc{pfv.m-pret} \tsc{cop-m}\\
		\glt \sqt{He built a house and married.}
		
	\end{exe}

%%%%%%%%%%%%%%%%%%%%%%%%%%%%%%%%%%%%%%%%%%%%%%%%%%%%%%%%%%%%%%%%%%%%%%%%%%%%%%%%

\subsection{Monovalent affective verbs and exceptional monovalent constructions}
\label{sec:Monovalent affective verbs and exceptional monovalent constructions}

Sanzhi Dargwa has a few constructions with monovalent predicates and a single argument fulfilling the role of a dative-marked experiencer \refex{ex:affective verbsMonovalent}. Such constructions can be copula constructions with adverbs \refex{ex:I feel bad there} or contain compound verbs \refex{ex:At night I had no sleep}. Gender/number agreement is frozen (prefixes \tit{b-} or occasionally \tit{d-}) and the person agreement is invariably third person.
%
\begin{exe}
	\ex	\label{ex:affective verbsMonovalent}
	\begin{xlist}
		\ex	\tit{b-uχːar-(le) ca-b} \sqt{be cold}
		\ex	\tit{wahi-l ca-b} \sqt{feel bad}
		\ex	\tit{ʡaˁħ-le ca-b} \sqt{feel good, be well}	
		\ex	\tit{buruš hitːi sa-b-irk-\slash buruš hitːi sa-b-ik-} \sqt{sleep, fall asleep}	
		\ex	\tit{beža hitːi d-irk-\slash beža hitːi d-ik-} \sqt{catch a cold}
	\end{xlist}

	\ex	\label{ex:I feel bad there}
	\gll	dam	wahi-l	ca-b	heχ-tːu-b\\
		\tsc{1sg.dat}	bad\tsc{-advz}	\tsc{cop-n}	\tsc{dem.down}\tsc{-loc-n}\\
	\glt	\sqt{I feel bad there.}

	\ex	\label{ex:At night I had no sleep}
	\gll	dučːi	dam	buruš	hitːi	a-sa-b-ič-ib\\
		night	\tsc{1sg.dat}	mattress	behind	\tsc{neg-hither}\tsc{-n-}occur\tsc{.pfv-pret}\\
	\glt	\sqt{At night I had no sleep.} (E)
\end{exe}

The affective verbs \sqt{see} and \sqt{hear} can also be used as monovalent verbs with the meaning \sqt{be/become visible, show off} and \sqt{be/become audible} (see \refsec{sec:Argument control in complement constructions} for two example sentences and \refsec{sec:Bivalent affective verbs} below for a discussion).

There is a special predicate denoting weather phenomena that has one single argument marked with the ergative \refex{ex:precipitation}, \refex{ex:It is raining}. The verb does not have an aspectual distinction and always shows neuter gender and third person agreement. The same phenomenon is observed in the neighboring Icari Dargwa variety \citep[155]{Sumbatova.Mutalov2003}, but apparently not in Standard Dargwa.
%
\begin{exe}
	\ex	\label{ex:precipitation}
	\begin{xlist}
		\ex	\tit{marka-l b-us-} \sqt{rain}
		\ex	\tit{mig-li b-us-} \sqt{hail}
		\ex	\tit{duˁħi-l b-us-} \sqt{snow}
	\end{xlist}

	\ex	\label{ex:It is raining}
	\gll	marka-l	b-us-ul	ca-b\\
		rain\tsc{-erg}	\tsc{n-}rain\tsc{-icvb}	\tsc{cop-n}\\
	\glt	\sqt{It is raining.} 
\end{exe}


%%%%%%%%%%%%%%%%%%%%%%%%%%%%%%%%%%%%%%%%%%%%%%%%%%%%%%%%%%%%%%%%%%%%%%%%%%%%%%%%

\subsection{Extended intransitive verbs and other constructions with bivalent predicates and absolutive S arguments}
\label{sec:Extended intransitive verbs}
The major class of predicates falling into this category are extended intransitive verbs. They are bivalent and have one S argument in the absolutive and another argument in the dative or in a spatial case. Gender/number as well as person agreement on extended intransitive verbs is always controlled by the absolutive argument.

The largest number of extended intransitive verbs have a goal argument in the dative \refex{ex:extended intransitive verbs}. Example sentences are provided in \refex{ex:The bandits did not touch him}-\refex{ex:‎‎‎The people fought us}. For the verb \tit{xadi ag-\slash argʷ-} \sqt{marry} it is always the nominal referring to the woman that occurs in the absolutive while the dative argument denotes the man \refex{ex:Then did Kursum (fem.) marry Husen (masc.)}. Some of the verbs given in \refex{ex:extended intransitive verbs} can be used as one-place verbs with reciprocal meaning, e.g. \tit{qaˁb lus b-ilk-\slash qaˁb lus b-ik-} \sqt{embrace each other, hug each other}.
%
\begin{exe}
	\ex	\label{ex:extended intransitive verbs}
	\begin{xlist}
		\ex	\tit{q'uc' b-irχʷ-\slash q'uc' b-iχʷ-} \sqt{be offended by}
		\ex	\tit{hitːi ka-b-ig-} \sqt{wait for}
		\ex	\tit{kːač b-irk-\slash kːač b-ik-} \sqt{touch}
		\ex	\tit{gu-lik'-} \sqt{listen to}
		\ex	\tit{xʷit' b-ik'ʷ-} \sqt{whistle at}
		\ex	\tit{waˁw b-ik'ʷ-} \sqt{shout at, call, cry}
		\ex	\tit{b-iχči(t) argʷ-\slash b-iχči(t) ag-} \sqt{believe}
		\ex	\tit{xadi argʷ-\slash xadi-ag-}\slash\sqt{marry} 
		\ex	\tit{er b-ik'ʷ-} \sqt{look at}
		\ex	\tit{paˁq b-ik'ʷ-} \sqt{hit at, strike}
		\ex	\tit{b-urʁ-} \sqt{shoot}
		\ex	\tit{qaˁb lus b-ilk-\slash qaˁb lus b-ik-} \sqt{embrace, hug}
	\end{xlist}

	\ex	\label{ex:The bandits did not touch him}
	\gll	il-tːi	qːačuʁ-e	kːač	a-b-ič-ib	il-i-j\\
		that\tsc{-pl}	bandit\tsc{-pl}	touch	\tsc{neg-hpl-}occur\tsc{.pfv-pret}	that\tsc{-obl-dat}\\
	\glt	\sqt{The bandits did not touch him.}

	\ex	\label{ex:I am listening to her/his song}
	\gll	du 	gu.lik'-unne=da 	it-i-la 	dalaj-li-j\\
		\tsc{1sg}	listen\tsc{.ipfv-icvb=1}	\tsc{3sg-obl-gen}	song\tsc{-obl-dat}\\
	\glt	\sqt{I am listening to her/his song.} (E)

	\ex	\label{ex:I (fem.) do not believe (in) you@30a}
	\gll	du	at	r-iχči a-argu-d\\
		\tsc{1sg}	\tsc{2sg.dat}	\tsc{f-}believe \tsc{neg}-go.tsc{ipfv-1}\\
	\glt	\sqt{I (fem.) do not believe (in) you.} (E)

	\ex	\label{ex:The boy began to whistle and to cry to the dogs in the village}
	\gll	durħuˁ	w-aʔ.ašː-ib ca-w	xʷit'	Ø-ik'-ul,	waˁw	Ø-ik'ʷ-ij	šːi-l-cːe-d	χu-d-a-j\\
		boy	\tsc{m-}begin\tsc{.pfv-pret} \tsc{cop-m}	whistle	\tsc{m-}say\tsc{.ipfv-icvb}	shout	\tsc{m-}say\tsc{.ipfv-inf}	village\tsc{-obl-in-npl}	dog\tsc{-pl-obl-dat}\\
	\glt	\sqt{The boy began to whistle and to shout at the dogs in the village.}

	\ex	\label{ex:Then did Kursum (fem.) marry Husen (masc.)}
	\gll	c'il	qːaq	ħuˁsen-ni-j	xadi	ag-ur-il=de=w	Kursum?\\
		then	Kak	Husen\tsc{-obl-dat}	married	go\tsc{.pfv-pret-ref=pst=q}	Kursum\\
	\glt	\sqt{Then did Kursum (fem.) marry Kak (lit. 'back') Husen (masc.)?}

	\ex	\label{ex:‎‎‎The people fought us}
	\gll	itːi	χalq'	nišːi-j	b-urʁ-ul	ca-b\\
		\tsc{3pl}	people	\tsc{1pl-dat}	\tsc{hpl-}shoot\tsc{-icvb}	\tsc{cop-hpl}\\
	\glt	\sqt{‎‎‎The people shot at us.}
\end{exe}

Verbs of speech and verbs with similar meanings may mark their addressee argument with the dative \refex{ex:The boy began to whistle and to cry to the dogs in the village}, but much more common is the use of the \tsc{in}-lative \refex{ex:The grandfather is calling me@27a}, \refex{ex:He was the one who was talking to the doctors}. There is only one extended intransitive verb of speech, \tit{b-ik'ʷ-} \sqt{say}, which is, however, also widely used in compound verbs \refex{ex:verbs of speech}.
%
\begin{exe}
	\ex	\label{ex:verbs of speech}
	\begin{xlist}
		\ex	\tit{xʷit' b-ik'ʷ-} \sqt{whistle at}
		\ex	\tit{waˁw b-ik'ʷ-} \sqt{shout at, call, cry}
		\ex	\tit{ʁaˁʁ b-ik'ʷ-} \sqt{scream, yell}
		\ex	\tit{t'irt'ir b-ik'ʷ-} \sqt{chat}
		\ex	\tit{ʁaj (ka-)b-ik'ʷ-} \sqt{quarrel, scold, argue, discuss, talk}
		\ex	\tit{paˁq b-ik'ʷ-} \sqt{hit at, strike}
	\end{xlist}

	\ex	\label{ex:The grandfather is calling me@27a}
	\gll	di-cːe	waˁw	Ø-ik'-ul	ca-w	hel	χatːaj\\
		\tsc{1sg-in}	call	\tsc{m-}say\tsc{.ipfv-icvb}	\tsc{cop-m}	that	grandfather\\
	\glt	\sqt{The grandfather is calling me.}

	\ex	\label{ex:He was the one who was talking to the doctors}
	\gll	c'il	hel-tːi	tuχtur-t-a-cːe	ʁaj	Ø-ik'ʷ-an	Ø-irχʷ-iri	il\\
		then	that\tsc{-pl}	doctor\tsc{-pl-obl-in}	word	\tsc{m-}say\tsc{.ipfv-ptcp} \tsc{m-}be\tsc{.ipfv-hab.pst.3}	that\\
	\glt	\sqt{He was the one who was talking to the doctors.} 
\end{exe}

There are a number of compound verbs and copula constructions with experiential semantics that belong to the extended intransitive class and mark the second argument with the \tsc{ante}-ablative \refex{ex:experiential compound verbs}, \refex{ex:Are you afraid of your wife? he says.} or take a clausal complement \refex{ex:We were embarrassed to talk a lot.} (\refcpt{cpt:Complementation}).
%
\begin{exe}
	\ex	\label{ex:experiential compound verbs}
	\begin{xlist}
		\ex	\tit{uruχ-le ca-b, uruχ b-ik'ʷ-} \sqt{be afraid of, fear}	
		\ex	\tit{uruχ b-irχʷ-\slash uruχ b-iχʷ-} \sqt{be/become afraid of, fear}
		\ex	\tit{uruc ca-b, uruc b-ik'ʷ-} \sqt{be ashamed of, be embarrassed of}
		\ex	\tit{uruc b-irχʷ-\slash uruc b-iχʷ-} \sqt{be/become ashamed of, be/become embarrassed of}
	\end{xlist}

	\ex	\label{ex:Are you afraid of your wife? he says.}
	\gll	``xːunul-li-sa-r	uruχ	Ø-ik'-ul=de=w,''	Ø-ik'-ul	ca-w,	``u?''\\
		woman\tsc{-obl-ante-abl}	fear	\tsc{m-}\tsc{aux.ipfv-icvb=2sg=q} \tsc{m-}say\tsc{.ipfv-icvb}	\tsc{cop-m}	\tsc{2sg}\\
	\glt	\sqt{``Are you afraid of your wife?'' he says.}

	\ex	\label{ex:We were embarrassed to talk a lot.}
	\gll χːʷalle	ʁaj	d-ik'ʷ-ij=ra	uruc d-ik'-ul=de\\
		much	word	\tsc{1/2pl}-say.\tsc{ipfv-inf=add}	embarrassed \tsc{1/2pl-aux.ipfv-icvb=pst}\\
		\glt	\sqt{We were embarrassed to talk a lot.}
\end{exe}


	
		
Extended intransitive verbs expressing location, position, or movement combine with various spatial cases. Which spatial case is used depends on the semantics of the spatial reference point (i.e. the ground) and on the type of localization or motion (e.g. in or on a reference point, movement to a goal or from a source). The most common spatial series employed in these functions are the \tsc{spr}-series \refex{ex:A horsefly sat down on the forehead of one (man)}, the \tsc{in}-series \refex{ex:(Apparently) they (=the animals) lived in the mountains}, and the \tsc{ad}-series. Many more examples can be found in \refsec{ssec:Functions of semantic cases} on the spatial cases.
%
\begin{exe}
	\ex	\label{ex:A horsefly sat down on the forehead of one (man)}
	\gll	ca-la	antːa-le	či-ka-b-iž-ib	ca-b	zija\\
		one\tsc{-gen}	forehead\tsc{-loc}	\tsc{spr-down}\tsc{-n-}sit\tsc{.pfv-pret}	\tsc{cop-n}	horsefly\\
	\glt	\sqt{A horsefly sat down on the forehead of one (man).}

	\ex	\label{ex:(Apparently) they (=the animals) lived in the mountains}
	\gll	il-tːi	dubur-t-a-cːe-d	er	d-irχ-ul	d-už-ib	ca-d\\
		that\tsc{-pl}	mountain\tsc{-pl-obl-in-npl}	life	\tsc{npl-}become\tsc{.ipfv-icvb}	\tsc{npl-}be\tsc{-pret} \tsc{cop-npl}\\
	\glt	\sqt{(Apparently) they (= the animals) lived in the mountains.}
\end{exe}

Other constructions with an absolutive S and a bivalent predicate are instantiated by verbs that assign absolutive or genitive to the P argument, depending on the meaning of the construction. The combination absolutive S plus absolutive P is rare \refex{ex:The people turned into stones}, \refex{ex:I will be 75 years old}. These clauses syntactically strongly resemble copula clauses but make use of verbs that express meanings other than the simple copula meaning. The verb agrees in person, number and gender with the subject-like argument, which can be distinguished from the second argument in the absolutive by reference to prominence properties such as animacy and person.
%
\begin{exe}
	\ex	\label{ex:The people turned into stones}
	\gll	χalq'	qːarq-ne	arž-i\\
		people	stone\tsc{-pl}	go\tsc{.ipfv-hab.pst.3}\\
	\glt	\sqt{The people turned into stones.}

	\ex	\label{ex:I will be 75 years old}
	\gll	du	wer-c'anu	xu-ra	dus	r-irχʷ-an=da\\
		\tsc{1sg}	seven-\tsc{ten}	five\tsc{-num} year	\tsc{f-}be\tsc{.ipfv-ptcp=1}\\
	\glt	\sqt{I (fem.) will be 75 years old.}
\end{exe}

Example \refex{ex:The hips will be a strong medicine} illustrates a clause with two arguments that also resembles copula clauses. The S argument in the absolutive case functions as subject-like argument (e.g. it controls agreement on the verb). The nominal bearing the genitive is not a possessor of an omitted head noun, but an argument of the verb. Note that it is possible to replace the genitive by the absolutive with no salient change in the meaning of the clause.
%
\begin{exe}
	\ex	\label{ex:The hips will be a strong medicine}
	\gll	gacbe	c'aq'	darman-na	d-irχʷ-an-te	ca-d\\
		hips	strong	medicine\tsc{-gen}	\tsc{npl-}become\tsc{.ipfv-ptcp-dd.pl}	\tsc{cop-npl}\\
	\glt	\sqt{The hips will be a strong medicine.} (E)
\end{exe}


%%%%%%%%%%%%%%%%%%%%%%%%%%%%%%%%%%%%%%%%%%%%%%%%%%%%%%%%%%%%%%%%%%%%%%%%%%%%%%%%

\subsection{Transitive verbs}
\label{sec:Transitive verbs}

Simple transitive verbs have two arguments: one is marked with the ergative and functions in a subject-like manner, and the other one bears the absolutive. Gender/number agreement is triggered by the absolutive argument. Person agreement can be controlled by the ergative or the absolutive argument and mostly follows the hierarchy 1,2 > 3 (see \refsec{sec:Person agreement} on person agreement for more details and examples). Transitive verbs can be simple underived verbs as the six first verbs in \refex{ex:transitive verbs} and the examples in \refex{ex:Did these people beat you up}, \refex{ex:‎‎I did not steal the stone. I did not kill anyone}, verbs containing various preverbs, compounds containing transitive light verbs \refex{ex:The boy stopped him}, \refex{ex:‎‎They taught me to smoke marihuana}, or causativized intransitive verbs \refex{ex:This makes the mill spin around}, \refex{ex:‎Marijam was growing cucumbers}. 
%
\begin{exe}
	\ex	\label{ex:transitive verbs}
	\begin{xlist}
		\ex	\tit{b-isː-\slash b-asː-} \sqt{take, buy}
		\ex	\tit{b-irq'-\slash b-arq'-} \sqt{do, make}
		\ex	\tit{b-irc-\slash b-ic-} \sqt{sell}
		\ex	\tit{b-urχ-\slash b-arχ-}	\sqt{sew}
		\ex	\tit{b-alš-\slash b-aš-} \sqt{knead}
		\ex	\tit{b-it-} \sqt{beat up}
		\ex	\tit{ač b-irq'-\slash ač b-arq'-} \sqt{open}
		\ex	\tit{qaˁš k-aʁ-} \sqt{cut into pieces}
		\ex	\tit{aq (či-ha-)b-irq'-\slash aq (či-ha-)b-arq'-} \sqt{lift up, take up}	
		\ex	\tit{t'aš b-ircː-aq-\slash t'aš b-icː-aq-; t'aš-aʁ-} \sqt{stop}
		\ex	\tit{bursːi b-arq'-\slash bursːi b-irq'-} \sqt{teach}
	\end{xlist}

	\ex	\label{ex:Did these people beat you up}
	\gll	u	r-it-ib-il=de=w	heštːi	χalq'-li?\\
		\tsc{2sg}	\tsc{f-}beat.up\tsc{-pret-ref=2sg=q}	these	people\tsc{-erg}\\
	\glt	\sqt{Did these people beat you (fem.) up?}

	\ex	\label{ex:The boy stopped him}
	\gll	durħuˁ-l	t'aš	aʁ-ib	it\\
		boy\tsc{-erg}	stop	do\tsc{-pret}	that\\
	\glt	\sqt{The boy stopped him.}

	\ex	\label{ex:‎‎I did not steal the stone. I did not kill anyone}
	\gll	du-l	a-b-iʡ-uˁn=da	qːarqːa.	ča-k'al	du-l	a-kax-ub=da\\
		\tsc{1sg-erg} \tsc{neg-n-}steal\tsc{.pfv-pret=1} stone	who\tsc{-indef}	\tsc{1sg-erg}	\tsc{neg-}kill\tsc{.pfv-pret=1}\\
	\glt	\sqt{‎‎I did not steal the stone. I did not kill anyone.}
\end{exe}


%%%%%%%%%%%%%%%%%%%%%%%%%%%%%%%%%%%%%%%%%%%%%%%%%%%%%%%%%%%%%%%%%%%%%%%%%%%%%%%%

\subsection{Extended transitive verbs (i.e. ditransitive verbs)}
\label{sec:Extended transitive verbs and ditransitive verbs}

Extended transitive verbs have three arguments bearing the ergative, the absolutive and a further case. They follow the same agreement rules as simple transitive verbs. This means that the absolutive argument triggers the gender/number agreement. Person agreement is controlled by the absolutive or the ergative argument, but never by the third argument (reciepient, addressee, etc.). It normally follows the person hierarchy 1, 2 > 3. The extended transitive verbs in \refex{ex:extended di-transitive verbs} all have dative arguments in addition to the ergative and absolutive arguments.
%
\begin{exe}
	\ex	\label{ex:extended di-transitive verbs}
	\begin{xlist}
		\ex	\tit{lukː-\slash b-ikː-} \sqt{give}
		\ex	\tit{qar b-irq'-\slash qar b-arq'-} \sqt{charge, entrust with}
		\ex	\tit{xadi lukː-\slash xadi b-ikː-} \sqt{marry off}
		\ex	\tit{či-b-iž-aq-\slash či-b-až-aq-} \sqt{show}
	\end{xlist}

	\ex	\label{ex:The snake gave it to me}
	\gll	dam	b-ičː-ib	iž	maˁlʡuˁn-ni\\
		\tsc{1sg.dat}	\tsc{n-}give\tsc{.pfv-pret}	this	snake\tsc{-erg}\\
	\glt	\sqt{The snake gave it to me.}

	\ex	\label{ex:My brother married me off to a fellow villager}
	\gll	šːan-ni-j	xadi	r-ičː-ib=da	di-la	ucːi-l\\
		fellow.villager\tsc{-obl-dat} married	\tsc{f-}give\tsc{.pfv-pret=1}	\tsc{1sg-gen} brother\tsc{-erg}\\
	\glt	\sqt{My brother married (me) off to a fellow villager.}
\end{exe}

To this group belong a number of verbs expressing violent physical contact \refex{ex:transitive contact verbs}. These verbs have an absolutive argument denoting the instrument of the action \refex{ex:He hit with the fist on the jaw of his wife}. The instrument is usually omitted such that we are left with two arguments, the ergative agent and the goal that takes the dative or the \tsc{in}-lative \refex{ex:He hit his wife}. The valency frame is typical for this semantic type of verbs and has been described for other East Caucasian languages (\citealp[332\tnd334]{Khalilova2009}; \citealp[476]{Forker2013a}). 
%
\begin{exe}
	\ex	\label{ex:transitive contact verbs}
	\begin{xlist}
		\ex	\tit{b-uˁrq-\slash b-aˁq-} \sqt{beat, hit}
		\ex	\tit{b-urh-\slash b-erh-} \sqt{knock, strike, bang}
		\ex	\tit{irx-\slash ixʷ-} \sqt{throw at, shoot}
	\end{xlist}

	\ex	\label{ex:He hit with the fist on the jaw of his wife}
	\gll	xːunul-la	qajqaj-t-a-cːe	q'uˁš	b-aˁq-ib	ca-b\\
		woman\tsc{-gen}	jaw\tsc{-pl-obl-in}	fist	\tsc{n-}hit\tsc{.pfv-pret}	\tsc{cop-n}\\
	\glt	\sqt{He hit with the fist on the jaw of his wife.}

	\ex	\label{ex:He hit his wife}
	\gll	xːunul-li-j	b-aˁq-ib	ca-b	hel-i-l\\
		woman\tsc{-obl-dat}	\tsc{n-}hit\tsc{.pfv-pret}	\tsc{cop-n}	that\tsc{-obl-erg}\\
	\glt	\sqt{He hit his wife.}

	\ex	\label{ex:when the other shot into the forehead}
	\gll	itilil-li	ix-ub-le	tupang	antːa-l-cːe	\ldots\\
		other\tsc{-erg}	throw\tsc{.pfv-pret-cvb}	weapon	forehead\tsc{-obl-in}\\
	\glt	\sqt{when the other shot into the forehead \ldots}
\end{exe}

However, the verb \tit{b-erh-} (\tsc{pfv}) \sqt{knock, strike, bang} takes only instruments in the ergative or the comitative case that do not control the agreement, such that the resulting clauses lack absolutive arguments \refex{ex:‎‎‎Ali knocked / beat off the giant with the hand}. The agreement trigger is not overtly present in the clause and cannot be retrieved by speakers. The difference in gender agreement goes hand in hand with a difference in the meaning of the clauses: when the neuter singular prefix \tit{b-} is used the event occurred only once; when the neuter plural suffix is used the knocking-event occurred repeatedly such that the meaning is rather \sqt{beat off}.
%
\begin{exe}
	\ex	\label{ex:‎‎‎Ali knocked / beat off the giant with the hand}
	\gll	ʡaˁli-l	weliχan-ni-j	naˁq-li	/	naˁq-li-cːella b-erh-ib	/	d-erh-ib\\
		Ali\tsc{-erg}	giant\tsc{-obl-dat}	hand\tsc{-erg} / hand\tsc{-obl-comit}	\tsc{n-}strike\tsc{.pfv-pret}	/	\tsc{npl-}strike\tsc{.pfv-pret}\\
	\glt	\sqt{‎‎‎Ali knocked\slash beat off the giant with the hand.} (E)
\end{exe}

Another group of extended transitive verbs are verbs of speech that take an addressee argument in the \tsc{in}-lative \refex{ex:verbs of speech addressees}-\refex{ex:I asked Isakadi}. Their absolutive argument is either a clause \refex{ex:Why did he not tell us that the water does not flow, they said}, \refex{ex:I asked Isakadi}, or a noun that refers to the speech event such as \textit{χabar} `story' \refex{ex:He tells the story}.

%
\begin{exe}
	\ex	\label{ex:verbs of speech addressees}
	\begin{xlist}
		\ex	\tit{ha-ʔ-\slash h-erʔ-} \sqt{say, tell}
		\ex	\tit{b-urs-\slash b-ux-} \sqt{say, tell}		
		\ex	\tit{xar b-irʁ-\slash xar b-eʁ-} \sqt{ask}
	\end{xlist}

	\ex	\label{ex:Why did he not tell us that the water does not flow, they said}
	\gll	``il-i-l=ra	cellij	a-b-urs-ib=el	nišːi-cːe	hin	a-ka-d-ax-u,''	haʔ-ib-le ...\\
		that\tsc{-obl-erg=add}	why	\tsc{neg-n-}tell\tsc{.pfv-pret=indq}	\tsc{1pl-in}	water	\tsc{neg-}down\tsc{-npl-}go\tsc{.ipfv-prs}	say\tsc{.pfv-pret-cvb}\\
	\glt	\sqt{``Why did he not tell us that the water does not flow,'' they said ...}

	\ex	\label{ex:I asked Isakadi}
	\gll	du-l	Isaq'adi-cːe	xar.b.eʁ-ib=da, ...\\
		\tsc{1sg-erg}	Isakadi\tsc{-in}	ask\tsc{.n.pfv-pret=1}\\
	\glt	\sqt{I asked Isakadi, ...}
\end{exe}

Verbs denoting movement and positioning of objects or animate entities combine with various spatial cases, e.g. the \tsc{ad}-lative or the \tsc{loc}-lative.
%
\begin{exe}
	\ex	\label{ex:verbs movement position}
	\begin{xlist}
		\ex	\tit{b-at-(it)-aʁ-} \sqt{send}
		\ex	\tit{sa-b-ik-\slash s-ak-, h-ak-, k-ak-}  \sqt{bring, lead}
		\ex	\tit{b-uk-\slash b-erč-} \sqt{take, collect, bring}
		\ex	\tit{(či-)(ka-)b-irxː-\slash (či-)(ka-)b-ixː-} \sqt{put}
	\end{xlist}

	\ex	\label{ex:You sent me to the doctor}
	\gll	hek'	tuχtur-ri-šːu	u-l	r-at.aʁ-ib=de\\
		\tsc{dem.up}	doctor\tsc{-obl-ad}	\tsc{2sg-erg}	\tsc{f-}send\tsc{.pfv-pret=2sg}\\
	\glt	\sqt{You sent (me, fem.) to the doctor.}
\end{exe}


%%%%%%%%%%%%%%%%%%%%%%%%%%%%%%%%%%%%%%%%%%%%%%%%%%%%%%%%%%%%%%%%%%%%%%%%%%%%%%%%

\subsection{Bivalent verbs with lexicalized objects and other rare constructions with bivalent verbs}
\label{sec:Bivalent verbs with frozen objects}

There are a couple of verbs with ergative subject-like arguments that have fixed, lexicalized objects that control the gender/number agreement on the verb, but cannot be exchanged with other nominals because they are constitutive for the meaning of the predicate \refex{ex:Bivalent verbs with frozen objectsValency}. For such verbs it is impossible to add another nominal in the absolutive, because the object-like position is already occupied by the lexicalized object. However, it is not always clear whether the lexicalized object controls the gender agreement on the verb or whether the gender agreement is frozen. From a semantic point of view, the verbs can be analyzed as bivalent or trivalent. The bivalent verbs have additional arguments in the dative \refex{ex:Brother called you} or \tsc{in}-essive \refex{ex:His wife begged him, Don't go} or occasionally in other cases \refex{ex:I am hugging you}. Verbs of speech preferably make use of the \tsc{in}-essive whereas other verbs mostly employ the dative. Some of these verbs take additional complement clauses.
%
\begin{exe}
	\ex	\label{ex:Bivalent verbs with frozen objectsValency}
	\begin{xlist}
		\ex	\tit{zaˁnʁ d-uˁrq-\slash zaˁnʁ d-aˁq-; telepun d-aˁq-\slash telepun d-uˁrq-} \sqt{call on the phone}
		\ex	\tit{kumek b-irq'-\slash kumek b-arq'-} \sqt{help} 	
		\ex	\tit{tamaša b-arq'-\slash  tamaša b-irq'-} \sqt{wonder}
		\ex	\tit{urk'ec'i b-irq'-\slash urk'ec'i b-arq'-; urk'ec'i či-d-uq-\slash urk'ec'i či-d-ulq-} \sqt{pity, feel sorry for}
		\ex	\tit{ʁaj b-irq'-\slash ʁaj b-arq'-} \sqt{talk, tell, speak}
		\ex	\tit{tiladi b-irq'-\slash tiladi b-arq'-} \sqt{beg, request}
		\ex	\tit{anru lukː-\slash anru b-ikː-} \sqt{command}
		\ex	\tit{ʁaj lukː-\slash ʁaj b-ikː-} \sqt{promise}
		\ex	\tit{qaˁb (sa)-b-urc-\slash qaˁb (sa)-b-urc-} \sqt{embrace, hug}
	\end{xlist}

	\ex	\label{ex:Brother called you}
	\gll	ucːi-l	at	zaˁnʁ	d-aˁq-ib\\
		brother\tsc{-erg}	\tsc{2sg.dat}	ring	\tsc{npl-}hit\tsc{.pfv-pret}\\
	\glt	\sqt{Brother called you.} (E)

	\ex	\label{ex:His wife begged him, Don't go}
	\gll	xːunul-li	tiladi	b-arq'-ib ca-b	hel-i-cːe,	ma-Ø-ax-utːa!	r-ik'-ul\\
		woman\tsc{-erg}	request	\tsc{n-}do\tsc{.pfv-pret} \tsc{cop-n}	that\tsc{-obl-in}	\tsc{proh-m-}go\tsc{.ipfv-proh.sg}	\tsc{f-}say\tsc{.ipfv-icvb}\\
	\glt	\sqt{His wife begged him: ``Don't go!''}

	\ex	\label{ex:I am hugging you}
	\gll	du-l	ala	qaˁb	b-urc-ul=da\\
		\tsc{1sg-erg}	\tsc{2sg.gen}	neck	\tsc{n-}keep\tsc{.ipfv-icvb=1}\\
	\glt	\sqt{I am hugging you.} (E)

	\ex	\label{ex:I did not promise you everything}
	\gll	du-l at	ci-k'al-la	ʁaj	a-b-ičː-ib=da\\
		\tsc{1sg-erg}	\tsc{2sg.dat}	what\tsc{-indef-gen}	word	\tsc{neg-n-}give\tsc{.pfv-pret=1}\\
	\glt	\sqt{I did not promise you anything.}
\end{exe}

Some of the frozen objects occur in more than one construction. For instance, \tit{urk'ec'i}, when combined with a verb, can occur in the ergative construction with an agentive experiencer \refex{ex:I pity the convicted}, in a construction with a dative experiencer \refex{ex:I pity Madina's sons@b}, and together with an experiencer in the genitive \refex{ex:I pity Madina's sons@c}. The stimulus is always a goal or beneficiary-like argument and therefore takes the dative. 
%
\begin{exe}
	\ex	\label{ex:I pity the convicted}
	\gll	du-l	ʡaˁjib-kar-t-a-j	urk'ec'i	b-irq'-id\\
		\tsc{1sg-erg}	guilt\tsc{-nmlz-pl-obl-dat}	pity	\tsc{n-}do\tsc{.ipfv-prs.1}\\
	\glt	\sqt{I pity the convicted.}

\end{exe}

Note that Sanzhi also has a range of compound verbs with nouns marked by the genitive or by spatial postpositions. However, the nouns used in such constructions are not lexicalized objects but nominal parts of compound verbs. For examples see \refsec{ssec:compoundswithnouns}.

%%%%%%%%%%%%%%%%%%%%%%%%%%%%%%%%%%%%%%%%%%%%%%%%%%%%%%%%%%%%%%%%%%%%%%%%%%%%%%%%

\subsection{Bivalent affective verbs}
\label{sec:Bivalent affective verbs}
Bivalent affective verbs are a relatively small class of two-place verbs with epxeriential\slash affective semantics \refex{ex:bivalent affective verbs}. They express unintentional and uncontrollable perception, emotion, volition, cognitive activities and other non-agentive events and situations. 

%
\begin{exe}
	\ex	\label{ex:bivalent affective verbs}
	\begin{xlist}
		\ex	\tit{či-b-ig-\slash či-b-ag-} \sqt{see}
		\ex	\tit{t'am b-iq'-\slash t'am b-aq'-} \sqt{hear}
		\ex	\tit{b-ikː-} \sqt{want, like, love}
		\ex	\tit{b-irʁ-\slash b-arʁ-} \sqt{understand}
		\ex	\tit{qum urt-\slash qum ert-} \sqt{forget}
		\ex	\tit{b-alχ-\slash b-aχ-} \sqt{know}
		\ex	\tit{b-urkː-\slash b-arkː-} \sqt{find}
		\ex	\tit{han b-irk-\slash han b-irk-} \sqt{remember, seem}
		\ex	\tit{b-ičː-aq-} \sqt{like, love}
		\ex	\tit{či-b-b-irt'-\slash či-b-b-et'-} \sqt{be bored}
		\ex	\tit{b-irt'-\slash b-et'-} \sqt{be bored}
	\end{xlist}
\end{exe}

Most bivalent affective verbs have an experiencer argument in the dative and a stimulus argument in the absolutive. They follow the same agreement rules as transitive verbs, i.e. gender/number agreement with the absolutive argument and person agreement is ruled by the hierarchy 1, 2 > 3 \refex{ex:I did not forget anythingA}, \refex{ex:I know a good place}, and/or it is the experiencer that controls the agreement \refex{ex:I did not forget anythingA}, \refex{ex:I know a good place}, or it is invariably third person \refex{ex:I forgot everythingA}, \refex{ex:‎‎I am angry with Sanzhiat}. 


\begin{exe}
	\ex	\label{ex:I did not forget anythingA}
	\gll	dam	qum.a.art-id	cik'al\\
		\tsc{1sg.dat}	forget\tsc{.ipfv.neg-hab.pst.1}	anything\\
	\glt	\sqt{I did not forget anything.}

	\ex	\label{ex:I know a good place}
	\gll	dam	ʡaˁħ	musːa	b-alχ-ad\\
		\tsc{1sg.dat}	good	place	\tsc{n-}know\tsc{.ipfv-prs.1}\\
	\glt	\sqt{I know a good place.}
	
\end{exe}


When inflected for some tenses such as the habitual past, the compound present \refex{ex:I do not see this person, whether it is female or male} or the future \refex{ex:‎‎I will never forget you}, \refex{ex:I know her, this Salihat} and in some types of subordinate clauses certain affective predicates allow for the experiencer to bear the ergative instead of the dative case. The ergative alignment pattern is more common in other Dargwa varieties such as Icari Dargwa, and has been investigated from a diachronic perspective in \citet{Ganenkov2013}. In Sanzhi Dargwa it is less common and the precise conditions that allow for ergative experiencers still need further investigation. In any case it follows the agreement rules are the same as for dative experiencers, e.g. in \refex{ex:I do not see this person, whether it is female or male} person agreement is controlled by the ergative experiencer and in \refex{ex:‎‎I will never forget you} by the absolutive stimulus. 
%
\begin{exe}
	\ex	\label{ex:I do not see this person, whether it is female or male}
	\gll	či-w-ig-ul	akːʷa-di	du-l	heχ	admi,	xːunul=el	murgul=el\\
		\tsc{spr-m-}see\tsc{.ipfv-icvb}	\tsc{cop.neg-1}	\tsc{1sg-erg}	\tsc{dem.down}	person		woman\tsc{=indq}		masculine\tsc{=indq}\\
	\glt	\sqt{I do not see this person (on the picture), whether it is female or male.}
	
			\ex	\label{ex:‎‎I will never forget you}
	\gll	u	du-l	nikagda	qum.a.art-an=de\\
		\tsc{2sg}	\tsc{1sg-erg}	never	forget\tsc{.ipfv.neg-ptcp=2sg}\\
	\glt	\sqt{‎‎I will never forget you.}
	
	\ex	\label{ex:I know her, this Salihat}
	\gll	na	it	du-l	r-alχ-an=q'al	het	Saliħaˁt\\
		now	that	\tsc{1sg-erg}	\tsc{f-}know\tsc{.ipfv-ptcp=prt}	that	Salikhat\\
	\glt	\sqt{I know her, this Salihat.}

\end{exe}



It is not always possible to determine if a specific examples follows the person hierarchy or if it is the experiencer, who controls the agreement (which can also be formulated as semantic role hierarchy: experiencer > stimulus). For instance, in \refex{ex:You (pl.) bored me, I said} the person agreement enclitic on the verb =\textit{da} expresses first person singular or plural and second person plural agreement, such that it could be either the experiencer (in accordance with the experiencer controlling agreement independently of person) or the stimulus (in accordance with the hierarchy) that functions as controller. Similarly, in both \refex{ex:I did not forget anything} and \refex{ex:I know a good place} a first person experiencer triggers the agreement suffix, which can be explained by the person hierarchy or by the semantic role hierarchy). 


\begin{exe}
	\ex	\label{ex:You (pl.) bored me, I said}
	\gll	či-d-d-et'-ib-le=da,	haʔ-ib=da,	nušːa	dam\\
		\tsc{spr-1/2pl-1/2pl-}bore\tsc{.pfv-pret-cvb=1/2pl}	say\tsc{.pfv-pret=1}	\tsc{1pl}	\tsc{1sg.dat}\\
	\glt	\sqt{``You (pl.) bored me,'' I said.}
\end{exe}


In general, experiencer verbs seem to allow for a higher degree of variation concerning person agreement than transitive verbs. This includes the fact that under certain circumstances the person agreement is third person although the clause contains a first or second person dative pronoun in the semantic role of experiencer. For instance, with the verb `forget' both person agreement enclitics and third person agreement are found in the Sanzhi corpus, but third person agreement prevails. Thus, in \refex{ex:‎‎I will never forget you} we find second person singular controlled by the stimulus and \refex{ex:I did not forget anythingA} the verb agrees with the experiencer in the dative (first person singular). By contrast, in \refex{ex:I forgot everythingA} and \refex{ex:I forget (it), I do not remember it.} the agreement is third person instead of the expected first person agreement. 
%

\begin{exe}
	\ex	\label{ex:I forgot everythingA}
	\gll	qum.ert-ur-re	ca-d	dam	cik'al\\
		forget\tsc{.pfv-pret-cvb}	\tsc{cop-npl}	\tsc{1sg.dat}	thing\\
	\glt	\sqt{I have forgotten everything.}
	
		\ex	\label{ex:I forget (it), I do not remember it.}
	\gll	qum.urt-u dam, han kalg-unne akːu\\
		forget\tsc{.ipfv-prs.3}	\tsc{1sg.dat}	remember remain.\tsc{ipfv-icvb} \tsc{cop-neg}\\
	\glt	\sqt{I forget (it), I do not remember it.}
	
\end{exe}

As the following minimal pair shows, the variation that the verb `forget' shows between person agreement and invariably third person does not imply any differences in meaning and is not tied to certain TAM forms (as it is the case for ergative experiencers, which are only available for a restricted number of TAM forms, but for all affective verbs). The variation includes forms with person suffixes and forms with person enclitics alike. One and the same tense form can show variation as the following two examples of the preterite demonstrate. The first sentence \refex{ex:I (masc.) forgot my own village.A} shows person agreement with the first person experiencer whereas the second sentence \refex{ex:I (masc.) forgot my own village.B} has a verb form that corresponds to the third person preterite (i.e. no person enclitic, no copula). 

\begin{exe}
	\ex	\label{ex:I (masc.) forgot my own village.A}
	\gll	dam qum.ert-ur=da w-ah-la šːi\\
	\tsc{1sg.dat} 	forget\tsc{.pfv-pret=1} \tsc{m}-owner-\tsc{gen} village\\
	\glt	\sqt{I (masc.) forgot my own village.} (E)

	\ex	\label{ex:I (masc.) forgot my own village.B}
	\gll	dam qum.ert-ur w-ah-la šːi\\
		\tsc{1sg.dat} 	forget\tsc{.pfv-pret} \tsc{m}-owner-\tsc{gen} village\\
	\glt	\sqt{I (masc.) forgot my own village.} (E)
\end{exe}

Based on the data collected so far I am not able to explain the variation by means of linguistic or extralinguistic factors. Another, more general question concerns the nature of the third person forms in \refex{ex:I forgot everythingA}, \refex{ex:I forget (it), I do not remember it.}, \refex{ex:I (masc.) forgot my own village.B}, and other affective verbs below, for which three different hypotheses could be suggested. First, we can perhaps analyze it as third person agreement controlled by the absolutive patient that overrules the agreement hierarchies stated above. It would then follow the ergative pattern analogously to the ergative agreement attested in certain TAM forms and discussed in \refsec{ssec:Person agreement rules}. Alternatively, we can claim that we deal with `suspended person agreement' in the sense that the verb shows the default person agreement form, namely third person, but this form does not underlie control but shows actually the lack of an agreement controller. 

A third alternative would be to suggest that the verbs in \refex{ex:I forgot everythingA} and \refex{ex:I forget (it), I do not remember it.} are one-place verbs and the dative pronouns are not genuine arguments of the verb but something like adjuncts and can therefore not control the agreement.\footnote{If this approach can be corroborated by further research, then the discussed verbs and examples have to be classified as monovalent affective verbs. For the sake of the argumentation and because I am unable to draw a conclusion at the present moment I prefer to leave this part of the section where it is.} This argumentation could be supported by the fact that even the verbs `see' and `hear', which are normally used as two-place affective verbs can be used as one-place bivalent verbs with the meanings `be visible' and `be audible'. In that case normally the dative experiencer can be omitted. Thus, \refex{ex:‎I began to see the mountains2} can be used with a dative pronoun, in which case two translations are possible \sqt{‎I began to see the mountains.} or \sqt{The mountains started to be visible to me.} If the pronoun is omitted, then the only translation is \sqt{The mountains started to be visible.}

\begin{exe}
	\ex	\label{ex:‎I began to see the mountains2}
	\gll	(dam)	dubur-te	či-d-ig-ul	d-aʔ	ašː-ib\\
		\tsc{1sg.dat}	mountain\tsc{-pl}	\tsc{spr-npl-}see\tsc{.ipfv-icvb}	\tsc{npl-}begin	begin\tsc{.pfv-pret}\\
	\glt	\sqt{‎I began to see the mountains.} OR \sqt{The mountains started to be visible to me.} (E)
\end{exe}


Similarly, the verb \sqt{remember} is a compound verb in which the verbal part consists of the otherwise intransitive light verb \tit{b-ik-} \sqt{occur}. I found only third person agreement in all corpus examples as well as in elicitation, which suggests that the dative experiencer is syntactically not an argument but an adjunct such as a goal \refex{ex:There I also remembered my family remember}. 
%
\begin{exe}
	\ex	\label{ex:There I also remembered my family remember}
	\gll	heχ-tːu-b	han	b-ič-ib	dam	kulpat=ra\\
		\tsc{dem.down}\tsc{-loc-hpl}	remember	\tsc{hpl-}occur\tsc{.pfv-pret}	\tsc{1sg.dat}	family\tsc{=add}\\
	\glt	\sqt{There I also remembered my family.}
\end{exe}


There are some more predicates that can be classified as two-place affective predicates because they come with two semantic roles, an experiencer and a stimulus, but which differ from the predicates discussed so far in this section. First of all, there are two copula constructions with adverbials that mean `needed' \refex{ex:copula constructions with absolutive stimulus}. In these constructions the absolutive stimulus functions as copula subject and thus person and gender agreement controller \refex{ex:I need your help.AFF}, or alternatively complement clauses can be used. The dative can be classified as copula predicate and its use is optional. The predicates therefore behave in the same way as what has been said above about `see' and `hear', i.e., they can be used as monovalent predicates without an experiencer in impersonal constructions or as bivalent affective verbs.

\begin{exe}
	\ex	\label{ex:copula constructions with absolutive stimulus}
	\begin{xlist}
		\ex	\tit{ʡaˁʁuni-l ca-b} \sqt{need, be necessary}\footnote{This predicate can also occur with an experiencer argument in the absolutive that takes over the role as copula subject and controls agreement. See example \refex{ex:This person needs to be there in the morning at 8 o'clockAA} in \refsec{General remarks on gender/number agreement}.}
		\ex	\tit{ħaˁžat-le ca-b} \sqt{need, be necessary}
	\end{xlist}
	
	
	\ex	\label{ex:I need your help.AFF}
	\gll	dam	ħaˁžat-le	ca-b	ala	kumek\\
		\tsc{1sg.dat}	need-\tsc{advz}	\tsc{cop-n}	\tsc{2sg.gen}	help\\
	\glt	\sqt{I need your help.} (E)
\end{exe}


There are two bivalent affective verbs that do not show person or gender agreement, but invariable third person forms and the default gender agreement prefix \textit{d}- \refex{ex:PITYangry}.\footnote{The same lexical item \textit{urk'ec'i} `pity' is used in another semantically very similar predicate together with the lexical verb `do, make'. In that construction the lexical verb has the agreement prefix \textit{b}- for neuter singular \refex{ex:I pity the convicted}. Therefore, the agreement prefix \textit{d}- in \refex{ex:PITYangry} cannot be controlled by the items preceding the verbs but must be a default prefix. In general, both \textit{b}- and \textit{d}- function as default agreement exponents in a number of different constructions (\refsec{General remarks on gender/number agreement}).} The verb \tit{simi d-uq-\slash simi d-ulq-} \sqt{be angry}, already mentioned in \refsec{sec:Extended intransitive verbs}, is a one-place verb that can be changed into a two-place verb with a further experiencer/goal argument in the dative by adding the spatial preverb \tit{či-} to it \refex{ex:‎‎I am angry with Sanzhiat}. This experiencer/goal argument can never control person agreement (i.e. first person agreement in the examples below is ungrammatical) and thus the person agreement is always third person. The identical lexical verb with the same preverb con also occur in a compound with \textit{urk'ec'i} `pity' with exactly the same morphosyntactic properties \refex{ex:I pity Madina's sons@b}. Note that in \refex{ex:I pity Madina's sons@c} the dative pronoun has been replaced by a genitive possessor that now encodes the semantic role of experiencer. This examples is an indication that the dative pronouns in the other examples \refex{ex:‎‎I am angry with Sanzhiat} and \refex{ex:I pity Madina's sons@b} are not arguments but adjuncts, perhaps comparable to external possessor that can be expressed in the dative or in the genitive. 
%


\begin{exe} 
\ex	\label{ex:PITYangry}
\begin{xlist}
		\ex \tit{ d-uq-\slash simi d-ulq-} \sqt{be angry}
		\ex \tit{urk'ec'i d-uq-\slash urk'ec'i d-ulq-} \sqt{pity}
\end{xlist}

	\ex	\label{ex:‎‎I am angry with Sanzhiat}
	\gll	dam	simi	či-d-ulq-u	Sanžijat-li-j\\
		\tsc{1sg.dat}	anger	\tsc{spr-npl-}direct\tsc{.ipfv-prs}	Sanzhiat\tsc{-obl-dat}\\
	\glt	\sqt{‎‎I am angry with Sanzhiat.} (E)
	
			\ex	\label{ex:I pity Madina's sons@b}
		\gll	dam	urk'ec'i	či-d-ulq-u	Madina-la	durħ-n-a-j\\
			\tsc{1sg.dat}	pity	\tsc{spr-npl-}direct\tsc{.ipfv-prs} Madina\tsc{-gen}	boy\tsc{-pl-obl-dat}\\
		\glt	\sqt{I pity Madina's sons.} (E)

		\ex	\label{ex:I pity Madina's sons@c}
		\gll	di-la	urk'ec'i	či-d-ulq-u	Madina-la	durħ-n-a-j\\
			\tsc{1sg-gen}	pity	\tsc{spr-npl-}direct\tsc{.ipfv-prs} Madina\tsc{-gen}	boy\tsc{-pl-obl-dat}\\
		\glt	\sqt{I pity Madina's sons.} (lit. My pity id directed onto Madina's son.) (E)	
\end{exe}

Finally there are a few constructions with dative experiencers and a source-like or cause-like stimulus arguments that bear the \tsc{ante}-ablative \refex{embarrassed constructions}. This is the same case that is used by some monovalent experiential verbs for marking the source/cause-like arguments \refex{ex:Are you afraid of your wife? he says.}. In these constructions there is again invariable third person agreement and default neuter singular gender agreement that is frozen and not controlled by any of the constituents \refex{ex:Hey, Kurban, I am ashamed in front of you}-\refex{ex:I miss you}.

\begin{exe} 
\ex \label{embarrassed constructions}
	\begin{xlist}
		\ex	\tit{c'aχ ka-b-ircː-\slash c'aχ ka-b-icː; c'aχ-le ca-b} \sqt{feel ashamed, be/become embarrassed} 
		\ex	\tit{c'aχ-le ca-b} \sqt{be ashamed by }
		\ex	\tit{b-irt'-\slash b-et'-} \sqt{long for}
	\end{xlist}
	\end{exe}
	
	\begin{exe}	
		\ex	\label{ex:Hey, Kurban, I am ashamed in front of you}
	\gll	jaʁari	Q'urban,	dam	a-sa-rka	c'aχ-le=ra	ca-b=q'al=nu\\
		\tsc{prt}	Kurban	\tsc{1sg.dat}	\tsc{2sg-ante-abl}	shame\tsc{-advz=add}	\tsc{cop-n=mod=prt}\\
	\glt	\sqt{Hey, Kurban, I am ashamed because of you.}

	\ex	\label{ex:I got embarrassed in front of Ali}
	\gll	dam	Keno-sa-rka	c'aχ	ka-b-ircː-ur\\
		\tsc{1sg.dat}	Keno\tsc{-ante-abl}	shame	\tsc{down-n-}stand\tsc{.ipfv-pret}\\
	\glt	\sqt{I got embarrassed in front of Keno.} (E)
	
	\ex	\label{ex:I miss you}
	\gll	dam	a-sa-r	b-et'-ib ca-b\\
		\tsc{1sg.dat}	\tsc{2sg-ante-abl}	\tsc{n-}long.for\tsc{.pfv-pret} \tsc{cop-n}\\
	\glt	\sqt{I miss you.} (E)
\end{exe}



%%%%%%%%%%%%%%%%%%%%%%%%%%%%%%%%%%%%%%%%%%%%%%%%%%%%%%%%%%%%%%%%%%%%%%%%%%%%%%%%

\subsection{Labile verbs}
\label{sec:Labile verbs}

Sanzhi Dargwa has a number of labile verbs that can be used as intransitive verbs (with the corresponding morphology) or as transitive verbs. Because in Sanzhi arguments can be omitted and are often omitted if their reference is clear from the context, at times it can be difficult to identify labile verbs. Furthermore, occasionally transitive verbs occur in impersonal constructions without arguments that could syntactically be defined as subjects or semantically identified as agents \refex{ex:It used to snow a lotA}, \refex{ex:He was not shaking like this (i.e. not able to move)}. Outside of the constructions shown in \refex{ex:It used to snow a lotA} and \refex{ex:He was not shaking like this (i.e. not able to move)}, the verb \tit{b-irq'-} (\tsc{ipfv})\slash\tit{b-arq'-} (\tsc{pfv}) is transitive and I therefore do not include it in the list of labile verbs.
%
\begin{exe}
	\ex	\label{ex:It used to snow a lotA}
	\gll	χːula-ce	duˁħi	b-irq'-iri\\
		big\tsc{-dd.sg}	snow	\tsc{n-}do\tsc{.ipfv-hab.pst.3}\\
	\glt	\sqt{It used to snow a lot.}

	\ex	\label{ex:He was not shaking like this (i.e. not able to move)}
	\gll	hež-itːe	hak'	w-irq'-ul	akːʷ-i\\
		this\tsc{-advz}	shake	\tsc{m-}do\tsc{.ipfv-icvb}	\tsc{cop.neg-hab.pst}\\
	\glt	\sqt{He was not shaking like this (i.e. he was not able to move).}
\end{exe}

Sanzhi Dargwa makes use of different suffixes for the imperative of many intransitive and transitive verbs, and the stem augment vowels in the prohibitive and the habitual present also differ according to transitivity. Thus, the verbal morphology provides decisive clues for deciding whether a verb is used intransitively or transitively.

The majority of the labile verbs are S=P-labile, preserving the argument with the patientive semantic role \refex{ex:S=P labile verbs}. The first example sentence in \refex{ex:The barn burnt} shows the intransitive use, and the second one in \refex{ex:The people burn the house with fire} illustrates the transitive use.
%
\begin{exe}
	\ex	\label{ex:S=P labile verbs}
	\begin{xlist}
		\ex \tit{(b-)ikːʷ-} \sqt{burn} (not specified for aspect)
		\ex	\tit{b-elq'-} (\tsc{pfv})\slash\tit{luq'-} (\tsc{ipfv}) \sqt{break, shatter, smash}
		\ex	\tit{b-erc'-} (\tsc{pfv})\slash\tit{b-uc'-} (\tsc{ipfv}) \sqt{fry, roast, bake}
		\ex	\tit{b-ic'-} (\tsc{pfv})\slash\tit{b-irc'-} (\tsc{ipfv}) \sqt{fill}
		\ex	\tit{b-aˁč} (\tsc{pfv})\slash\tit{b-aˁlč} (\tsc{ipfv}) \sqt{squeeze, break, crush, crack, trample}
	\end{xlist}
\end{exe}

%
\begin{exe}
	\ex	\label{ex:The barn burnt}
	\gll	daˁrqʷ	b-ikː-ub\\
		barn	\tsc{n-}burn\tsc{-pret}\\
	\glt	\sqt{The barn burnt.} (E)

	\ex	\label{ex:The people burn the house with fire}
	\gll	χalq'-li	qal	c'a-l	b-ikː-ul	ca-b\\
		people\tsc{-erg}	house	fire\tsc{-erg}	\tsc{n-}burn-\tsc{icvb}	\tsc{cop-n}\\
	\glt	\sqt{The people burn the house with fire.} (E)
\end{exe}

The prohibitive of the intransitive clause is given in \refex{ex:‎‎‎Barn, do not burn}, and the prohibitive of the transitive can be found in \refex{ex:‎Eh boy, do not burn the house}.
%
\begin{exe}
	\ex	\label{ex:‎‎‎Barn, do not burn}
	\gll	daˁrqʷ,	ma-jkː-ut!\\
		barn	\tsc{proh-}burn\tsc{-proh.sg}\\
	\glt	\sqt{‎‎‎Barn, do not burn!} (E)

	\ex	\label{ex:‎Eh boy, do not burn the house}
	\gll	ej	durħuˁ,		qal	ma-jkːʷ-it!\\
		eh	boy	house	\tsc{proh-}burn\tsc{-proh.sg}\\
	\glt	\sqt{‎Eh boy, do not burn the house!} (E)
\end{exe}


I found a few S=A labile verbs that preserve the subject-like argument, namely:
%
\begin{exe}
	\ex	\label{ex:S=A labile verbs}
	\begin{xlist}
		\ex	\tit{b-elč'-} (\tsc{pfv})\slash\tit{b-uč'-} (\tsc{ipfv}) \sqt{read, learn, study, sing}, \refex{ex:labile intransitive}, \refex{ex:labile transitive}
		\ex	\tit{b-erkʷ-} (\tsc{pfv})\slash\tit{b-uk-} (\tsc{ipfv}) \sqt{eat}
		\ex	\tit{b-erčː-} (\tsc{pfv})\slash\tit{b-učː-} (\tsc{ipfv}) \sqt{drink, smoke, consume}
		\ex	\tit{b-arq'-} (\tsc{pfv})\slash\tit{b-irq'-} (\tsc{ipfv}) \sqt{do, make, be busy}
	\end{xlist}
\end{exe}

Translational equivalents of \sqt{read} are also labile in a number of other East Caucasian languages (e.g. in Icari Dargwa, \citealp[154\tnd155]{Sumbatova.Mutalov2003}, and in Hinuq, \citealp[492]{Forker2013a}). Note that lability surfaces only with the imperfective aspect of those verbs that can be used intransitively or transitively. This means that the perfective stems always occur in transitive constructions. 
%
\begin{exe}
	\ex	intransitive construction	\label{ex:labile intransitive}
	\begin{xlist}
		\ex	\label{ex:‎He is reading / studying}
		\gll	uč'-un	ca-w	hež\\
			read\tsc{.m.ipfv-pret}	\tsc{cop-m}	this\\
		\glt	\sqt{‎He is reading\slash studying.} (E)

		\ex	\label{ex:study read intransitive}
		\gll	uč'-en	/	ma-wč'-ut!\\
			learn\tsc{.m.ipfv-imp}	/	\tsc{proh-}learn\tsc{.m.ipfv-proh.sg}\\
		\glt	\sqt{Study!} OR \sqt{Read!}\slash\sqt{Do not study!} OR \sqt{Do not read!} (E)
	\end{xlist}

	\ex	transitive construction	\label{ex:labile transitive}
	\begin{xlist}
		\ex	\label{ex:I will read the newspaper.}
		\gll	du-l	kːazat b-uč'-an=da\\
			\tsc{1sg-erg}	newspaper \tsc{n-}read\tsc{.ipfv-ptcp=1}\\
		\glt	\sqt{I will read the newspaper.} (E)

		\ex	\label{ex:Read the book! / Do no read the book}
		\gll	kiniga	b-elč'-en	/	kiniga	ma-b-uč'-it!\\
			book	\tsc{n-}read\tsc{.pfv-imp}	/	book	\tsc{proh-n-}read\tsc{.ipfv-proh.sg}\\
		\glt	\sqt{Read the book!\slash Do not read the book!} (E)
	\end{xlist}
\end{exe}

Furthermore, an optional P argument can be added in the intransitive use. This argument needs to be semantically plural and indefinite, and is marked with the ergative case \refex{ex:Now Sanijat reads books}. This construction is called \sqt{antipassive} in Dargwa languages and treated in more detail in \refsec{sec:Antipassive}. 
%
\begin{exe}
	\ex	\label{ex:Now Sanijat reads books}
	\gll	hana Sanijat kiniga-b-a-l	r-uč'-unne	ca-r\\
		now	Sanijat	book\tsc{-pl-obl-erg}	\tsc{f-}read\tsc{.ipfv-icvb}	\tsc{cop-r}\\
	\glt	\sqt{Now Sanijat reads books.} (E)
\end{exe}

%%%%%%%%%%%%%%%%%%%%%%%%%%%%%%%%%%%%%%%

\section{Modification of valency patterns}
\label{cpt:Modification of valency patterns}

Sanzhi Dargwa has two major means of modifying the valency patterns of verbs, the antipassive as a detransitivizing operation (\refsec{sec:Antipassive}), and the causative as an argument-increasing operation (\refsec{sec:Causativization}). There are no potential or involuntary agent constructions, which in other East Caucasian languages are used to detransitive verbs. There is also no biabsolutive construction, which would allow for the use of two absolutive arguments with a transitive verb.


%%%%%%%%%%%%%%%%%%%%%%%%%%%%%%%%%%%%%%%%%%%%%%%%%%%%%%%%%%%%%%%%%%%%%%%%%%%%%%%%

\subsection{Antipassive}
\label{sec:Antipassive}

Sanzhi Dargwa has an antipassive that is formed by reversing the case marking of A and P in a clause with a canonical transitive predicate \refexrange{ex:S/he sews a dress@63a}{ex:She is a dressmaker@63b}. Since both A and P are obligatorily arguments in the antipassive construction, it is not an argument-decreasing operation, although the A argument is frequently covert in examples from natural texts. The verb remains unmarked, but the gender/number agreement on the verb changes. Due to the lack of formal marking on the verb the antipassive in Sanzhi is not a typical antipassive from a typological perspective \citep{Polinsky2005}. 
%
\begin{exe}

		\ex	ergative construction \label{ex:S/he sews a dress@63a}\\
		\gll	it-i-l	kːurtːi	b-urχ-u\\
			that\tsc{-obl-erg}	dress	\tsc{n-}sew\tsc{.ipfv-prs.3}\\
		\glt	\sqt{S/he sews a dress.} (E) 

		\ex	antipassive construction \label{ex:She is a dressmaker@63b}\\
		\gll	it	kurtːi-l	r-urχ-u \\
			that	dress\tsc{-erg}	\tsc{f-}sew\tsc{.ipfv-prs.3}\\
		\glt	\sqt{She is a dressmaker.} OR \sqt{She habitually sews dresses.} (E)

\end{exe}

Apart from being restricted to only one predicate class, namely canonical transitive verbs, the antipassive is additionally constrained in other ways:
%
\begin{enumerate}
	\item	Only the A argument can be omitted. In texts it is frequently omitted as the examples in \refex{ex:Frying fish, eating fruits we stayed}, \refex{ex:‎‎They taught me to smoke marihuana} show. The overt presence of the P argument is obligatory in order to have an antipassive construction, and it is usually the best indicator of the antipassive because the gender agreement affixes do not unambiguously indicate the controller.

	\item	It is largely (if not fully) restricted to imperfective verb stems and consequently to those tenses that are available for verbs with imperfective stems such as, for instance, the compound present, the compound past, the habitual present, the habitual past, the future forms and the obligative forms. Other tenses, e.g. the preterite or the resultative, cannot be used for antipassive constructions because they are basically formed from the perfective stems. It can also occur in subordinate clauses if the respective clause types allow for verb forms based on stems with imperfective aspect. For instance, \refex{ex:‎When they played with the frog, the boy and his dog got tired and played down to sleep} and \refex{ex:Frying fish, eating fruits we stayed} show adverbial clauses with antipassive constructions, and \refex{ex:the drinking friends} shows a relative clause. In \refex{ex:Frying fish, eating fruits we stayed} the verb in the main clause is intransitive. Due to the antipassive constructions in the preceding sentences the subject that is shared in all three clauses would be in the absolutive case if it would occur overtly. At the first glance, one might think that the antipassive has been used in order to make argument sharing across the three clauses possible, but this is not the case. There are (almost) no syntactic restrictions on co-reference and shared argument between adverbial and main clauses \refex{ex:Frying fish, eating fruits we stayed}. Therefore, the use of standard transitive constructions with ergative subjects would be equally grammatical in the two adverbial clauses. In other words, the antipassive is not needed for pivot modulation. On the contrary, it is used for purely semantic reasons. 
	%
	\begin{exe}
		
		\ex	\label{ex:Frying fish, eating fruits we stayed}
		\gll	[baliqː-a-l	d-ucː-ul	hel-itːe]	[c'idex-li	d-uk-unne]	d-už-ib-da\\
			fish\tsc{-obl-erg}	\tsc{1/2pl-}bake\tsc{.ipfv-icvb}	this\tsc{-attr}	fruit\tsc{-erg}	\tsc{1/2pl-}eat\tsc{.ipfv-icvb}	\tsc{1/2.pl-}be\tsc{-pret-1}\\
		\glt	\sqt{Frying fish, eating fruits we stayed.}
	
		\ex	\label{ex:‎‎They taught me to smoke marihuana}
		\gll	hel-tː-a-l	bursːi	w-arq'-ib=da	[qːama-l	učː-ij]\\
			that\tsc{-pl-obl-erg}	teach	\tsc{m-}do\tsc{.pfv-pret=1}	hemp\tsc{-erg}	consume\tsc{.m.ipfv-inf}\\
		\glt	\sqt{‎‎They taught me to smoke marihuana.}
	
	\ex	\label{ex:‎When they played with the frog, the boy and his dog got tired and played down to sleep}
		\gll	ħaˁz-t-a-l	b-irq'-ib=qːel,	ʡaˁt'a-cːe-r	durħuˁ=ra	kac'i=ra b-arcː-ur-re,	ka-b-isː-un	b-usː-anaj\\
			game\tsc{-pl-obl-erg}	\tsc{hpl-}do\tsc{.ipfv-pret=}when	frog\tsc{-in-abl}	boy\tsc{=add}	puppy\tsc{=add}	\tsc{hpl-}get.tired\tsc{.pfv-pret-cvb}	\tsc{down-hpl-}sleep\tsc{.pfv-pret}	\tsc{hpl-}sleep\tsc{.pfv-subj.3}\\
		\glt	\sqt{‎When they played with the frog, the boy and his dog got tired and lay down to sleep.}
		
		\ex	\label{ex:the drinking friends}
		\gll	[deč-li	b-učː-an]	juldašː-e\\
			drinking\tsc{-erg}	\tsc{hpl-}drink\tsc{.ipfv-ptcp}	friend\tsc{-pl}\\
		\glt	\sqt{the drinking friends}
	\end{exe}

	\item	Not all transitive verbs allow for the antipassive construction. The majority of antipassive clauses in the Sanzhi corpus contain either of the three verbs \tit{b-uk-} \sqt{eat} \refex{ex:I (fem.) am eating bread}, \tit{b-učː-} \sqt{drink, consume, smoke} \refex{ex:the drinking friends}, and \tit{b-irq'-} \sqt{do, make, be busy} \refex{ex:‎When they played with the frog, the boy and his dog got tired and played down to sleep}, \refex{ex:Grandfather used to work}, but a few more are also attested \refex{ex:She reads (i.e. sings) nasheeds@varB}. Typical verbs for which the antipassive is not available are verbs for which it is unclear what the result of the action that they denote would be \refex{ex:The car is pushing Rashid@64b}.\footnote{An anonymous reviewer pointed out that the unavialability of the antipassive reading in \refex{ex:The car is pushing Rashid@64b} might also be due to the fact that the antipassive expresses repeated or habitual situations. This is possible and more research is needed to give a conclusive answer for why this example cannot be interpreted as antipassive.} 
	%
	\begin{exe}
		\ex	\label{ex:Rashid get away from the car@64}
		\begin{xlist}
			\ex	ergative construction \label{ex:Rashid is pushing the car@64a} \\
			\gll	Rašid-li	mašin	qːurt	b-irq'-ul ca-b\\
				Rashid\tsc{-erg}	car	push	\tsc{n-}do\tsc{.ipfv-icvb} \tsc{cop-n}\\
			\glt	\sqt{Rashid is pushing a/the car.} (E)
	
			\ex	ergative construction \label{ex:The car is pushing Rashid@64b} \\
			\gll	Rašid	mašin-ni	qːurt	Ø-irq'-ul ca-w\\
				Rashid	car\tsc{-erg}	push	\tsc{m-}do\tsc{.ipfv-icvb} \tsc{cop-m}\\
			\glt	\sqt{A/the car is pushing Rashid.} (NOT: \sqt{Rashid is pushing a/the car.})~(E)
		\end{xlist}
	\end{exe}

	\item	The antipassive is not available with first or second person patients. There are no person restrictions on the agent \refex{ex:I (fem.) am eating bread}, \refex{ex:You do not drink water}, but the patient must be third person.
	%
	\begin{exe}
		\ex	\label{ex:I (fem.) am eating bread}
		\gll	du	t'ult'-li	r-uk-un-ne=da\\
			\tsc{1sg}	bread\tsc{-erg}	\tsc{f-}eat\tsc{.ipfv-pret-cvb=1}\\
		\glt	\sqt{I (fem.) am eating bread.} (E)
	\end{exe}

	\item	There are animacy restrictions: it is impossible for A and P to be both animate or both inanimate. The last two constraints are not really syntactic in nature since the resulting clauses are grammatical. However, the meaning would not be what is intended. If we switch the case marking of A and P in \refex{ex:The monster is eating me} the outcome is simply a normal clause in which the roles of A and P have been reversed \refex{ex:I am eating the monster.}. 
	%
	\begin{exe}
		\ex	\label{ex:The monster and me are eating each other}
		\begin{xlist}
			\ex	\label{ex:The monster is eating me}
			\gll	aždaha-l	du	Ø-ukː-unne=da\\
				monster\tsc{-erg}	\tsc{1sg}	\tsc{m-}eat\tsc{.ipfv-icvb=1}\\
			\glt	\sqt{The monster is eating me.} (E)
	
			\ex	\label{ex:I am eating the monster.}
			\gll	du-l	aždaha	b-ukː-unne=da\\
				\tsc{1sg-erg}	monster	\tsc{n-}eat\tsc{.ipfv-icvb=1}\\
			\glt	\sqt{I am eating the monster.} (NOT: \sqt{The monster is eating me.}) (E)
		\end{xlist}
	\end{exe}
\end{enumerate}

Syntactically, the antipassive is a detransitivizing operation. The main proof for this is, of course, that the A argument occurs in the absolutive case and controls the gender agreement, whereas the P argument takes the ergative case. The functional range of the ergative comprises not only the expression of agents, but also of other semantic roles with a more peripheral status (adjuncts), most notably instruments (\refsec{sssec:Ergative}). The ergative P of the antipassive largely fits into this range. Furthermore, the distinction between suffixes for intransitive and for transitive verbs that is made in the imperative and in the prohibitive shows that verbs in the antipassive construction are detransitivized. Thus, the prohibitive suffixes for intransitive verbs are \tit{-ut} (\tsc{sg})\slash\tit{-utːaja} (\tsc{pl}) with the stem augment vowel \tit{u}, whereas the transitive verbs have \tit{-it} (\tsc{sg})\slash\tit{-itːaja} (\tsc{pl}) with the stem augment \textit{i} (\refsec{sec:prohibitive}). The antipassive construction requires the same prohibitive suffix as intransitive verbs \refex{ex:antipassive drink water}, which is ungrammatical in the ergative construction \refex{ex: ergative drink water}.
%
\begin{exe}
	\ex	antipassive construction	\label{ex:antipassive drink water}
	\begin{xlist}
		\ex	\label{ex:You do not drink water}
		\gll	ušːa hin-ni	ma-d-učː-utːaja!\\
			\tsc{2pl}	water\tsc{-erg}	\tsc{proh-1/2pl-}drink\tsc{.ipfv-proh.pl}\\
		\glt	\sqt{You do not drink water (regularly)!} (E)
	
		\ex	\label{ex:Do (always) not eat meat!}
		\gll	dig-li ma-w-k-ut!\\
			meat\tsc{-erg} \tsc{proh-m-}eat\tsc{.ipfv-proh.sg}\\
		\glt	\sqt{Do not (always) eat meat!} (said to a man) (E)
	\end{xlist}
	
	\ex	ergative construction	\label{ex: ergative drink water}
	\begin{xlist}
		\ex	\label{ex:You do not drink the water}
		\gll	ušːa-l		hin ma-d-učː-itːaja!\\
			\tsc{2pl-erg}	water \tsc{proh-npl-}drink\tsc{.ipfv-proh.pl}\\
		\glt	\sqt{You do not drink the water!} (E)
	
		\ex	\label{ex:Do not eat the meat!}
		\gll	dig ma-b-uk-it!\\
			meat	\tsc{proh-n-}eat\tsc{.ipfv-proh.sg}\\
		\glt	\sqt{Do not eat the meat!} (E)
	\end{xlist}
\end{exe}

The major problem in the analysis of antipassive constructions concerns the closeness to S=A labile verbs that can be used intransitively and transitively, thereby preserving the agent argument (\refsec{sec:Labile verbs}). For instance, the imperfective stem of the verb \tit{b-elč'-} (\tsc{pfv})\slash\tit{b-uč'-} (\tsc{ipfv}) \sqt{read, learn, study, sing} can be used in an intransitive construction. When adding the ergative adjunct \tit{student-li} (student\tsc{-erg}) to \refex{ex:She reads / studies} the translation is unambiguously \sqt{she studies (at a university as a student)}. The same verb can be used in a transitive construction with an ergative agent and an absolutive patient \refex{ex:S/he reads (i.e. sings) nasheeds@varA}.
%
\begin{exe}
	\ex	\label{ex:She reads / studies}
	\gll	it	r-uč'-unne	ca-r\\
		\tsc{dem}	\tsc{f-}read\tsc{.ipfv-icvb}	\tsc{cop-f}\\
	\glt	\sqt{She reads\slash studies.} (E)

	\ex	\label{ex:S/he reads (i.e. sings) nasheeds@varA}
	\gll	it-i-l	turk-me	d-uč'-unne	ca-d\\
		\tsc{dem-obl-erg}	nasheed\tsc{-pl}	\tsc{npl-}read\tsc{.ipfv-icvb}	\tsc{cop-npl}\\
	\glt	\sqt{S/he reads (i.e. sings) nasheeds.} (E)
\end{exe}

In the antipassive construction, to the intransitive clause in \refex{ex:She reads / studies} a P argument in the plural marked with the ergative case is added \refex{ex:She reads (i.e. sings) nasheeds@varB}. The presence of the P argument is the only difference between the two sentences \refex{ex:She reads / studies} and \refex{ex:She reads (i.e. sings) nasheeds@varB}. Thus, instead of speaking of an antipassive construction we can also say that Sanzhi has a number of S=A labile verbs that are used intransitively with an optional nominal in the ergative that has syntactically rather the status of an adjunct.
%
\begin{exe}
	\ex	\label{ex:She reads (i.e. sings) nasheeds@varB}
	\gll	it	turk-m-a-l	r-uč'-unne	ca-r\\
		that	nasheed\tsc{-pl-obl-erg}	\tsc{f-}read\tsc{.ipfv-icvb}	\tsc{cop-f}\\
	\glt	\sqt{She reads (i.e. sings) nasheeds.} (E)
\end{exe}

The verb \tit{b-irq'-} (\tsc{ipfv})\slash\tit{b-arq'-} (\tsc{pfv}) \sqt{do, make, be busy} belongs to the verbs that frequently occur in antipassive constructions \refex{ex:‎When they played with the frog, the boy and his dog got tired and played down to sleep} and can also be used intransitively without any P argument \refex{ex:She is busy}. For this verb, there is a further possibility of use in weather constructions in which there is no A argument \refex{ex:It used to snow a lotA}. The latter construction thus resembles S=P-labile verbs (\refsec{sec:Labile verbs}).
%
\begin{exe}
	\ex	\label{ex:She is busy}
	\gll	it	r-irq'-ul	ca-r\\
		\tsc{dem}	\tsc{f-}do\tsc{.ipfv-icvb}	\tsc{cop-f}\\
	\glt	\sqt{She is busy.} (E)
\end{exe}

In sum, we can divide verbs in Sanzhi into three classes:
%
\begin{enumerate}
	\item	the class of verbs that do not allow for the antipassive construction at all as exemplified by \refex{ex:Rashid get away from the car@64} above
	\item	the class of S=A labile verbs that allow for transitive and intransitive use with or without a patient such as \tit{b-irq'-} (\tsc{ipfv}) (\tsc{pfv}) \sqt{do, make, be busy} and \tit{b-uč'-} (\tsc{ipfv}) \sqt{read, learn, study, sing}; if an ergative patient is present we can speak of the antipassive construction \refex{ex:‎When they played with the frog, the boy and his dog got tired and played down to sleep}, \refex{ex:She reads (i.e. sings) nasheeds@varB}
	\item	the class of verbs that form an antipassive with an obligatory P argument that can never be omitted; the verb \tit{b-ux-} (\tsc{ipfv}) \sqt{tell} belongs to the latter class since in clauses such as \refex{ex:He tells many stories} the patient needs to occur overtly
\end{enumerate}

The use of antipassives is semantically rather than syntactically motivated. It has habitual semantics, which is typical for antipassives in general and antipassives in East Caucasian languages in particular \refex{ex:She is a dressmaker@63b} (see, e.g., \citealp{vandenBergManuscript}, \citealp{Tatevosov2011}, \citealp{ComrieEtAlXXXX}). Most notably, in all corpus examples the P argument is indefinite and usually in the plural or it has the meaning of a mass noun. Morphologically singular P arguments are only allowed if they can have mass noun readings. The P argument does not refer to a particular, specified object, but is semantically demoted. The sentences refer to repeatedly or habitually occurring actions. For instance, in \refex{ex:Grandfather used to work} the speaker was talking about the life of her grandfather and how he used to be, which types of work he used to do.
%
\begin{exe}
	\ex	\label{ex:Grandfather used to work}
	\gll	χatːaj	ʡaˁči-l	w-irq'-i, \ldots\\
		grandfather	work\tsc{-erg}	\tsc{m-}do\tsc{.ipfv-hab.pst}\\
	\glt	\sqt{Grandfather used to work, [as a builder, as \ldots]}

	\ex	\label{ex:‎The bandits went upwards, singing anasheed in Arabic}
	\gll	c'il	ag-ur	lak [\ldots]	turk-m-a-l	b-uč'-unne	hetːi	ʡaˁrab-la	ʁaj-li		illallah	b-ik'-ul\\
		then	go\tsc{.pfv-pret}	up	{}	nasheed\tsc{-pl-obl-erg}	\tsc{hpl-}sing\tsc{.ipfv-icvb}	those	Arabic\tsc{-gen}	word\tsc{-erg}	Illallah		\tsc{hpl-}say\tsc{.ipfv-icvb}\\
	\glt	\sqt{(‎The bandits) went upwards [\ldots], singing a nasheed in Arabic, \dqt{Illallah}}

	\ex	\label{ex:They were apparently praying}
	\gll	debʁul-m-a-l	b-irq'-ul	b-už-ib-le=de\\
		prayer\tsc{-pl-obl-erg}	\tsc{hpl-}do\tsc{.ipfv-icvb}	\tsc{hpl-}be\tsc{-pret-cvb=pst}\\
	\glt	\sqt{They were apparently praying.}

	\ex	\label{ex:‎Saying I drive that, here, he was writing letters to us}
	\gll	``hel	b-ik-ul=da,''	Ø-ik'-ul,	``heštːu,''		nišːi-j	kaʁur-t-a-l	luk'-unne\\
		that	\tsc{n-}lead\tsc{.ipfv-icvb=1}	\tsc{m-}say\tsc{.ipfv-icvb}	here	\tsc{1pl-dat}	letter\tsc{-pl-obl-erg}	write\tsc{.ipfv-icvb}\\
	\glt	\sqt{‎Saying \dqt{I drive that, here}, he was writing letters to us.}

	\ex	\label{ex:‎They were dancing surprisingly beautifully, not like our (women), (and they were) also elderly women}
	\gll	ʡaˁžib	qːuʁa	ʡaˁjur-t-a-l	b-irq'-ul=de,	nišːa-la-te	daˁʡle	akːʷ-ar,	χːula	xːun-re=ra\\
		surprising	beautiful	dance\tsc{-pl-obl-erg}	\tsc{hpl-}do\tsc{.ipfv-icvb=pst}	\tsc{1pl-gen-dd.pl} 	as	\tsc{cop.neg-prs.3}	big	woman\tsc{-pl=add}\\
	\glt	\sqt{‎They were dancing surprisingly beautifully, not like our (women), also the elderly women.}
\end{exe}

By contrast, the P argument in the ergative construction can have a definite interpretation, referring to specific object. Thus, compare \refex{ex:He tells the story} in which the subject referent is telling a specific story\footnote{This is clear from the context of the example. Without a context the same sentence could also be translated as `He tells a story.'} to \refex{ex:He tells many stories}, which refers to the action of storytelling without specifying the stories further, but could rather be a characterization of the person as a story-teller.\footnote{The two verb in the examples represent two distinct lexemes, which are partially in complementary distribution because of their aspectual properties. The verb in \refex{ex:He tells many stories} is used as the imperfective counterpart of the verb \textit{b-urs-ij}, which occurs in \refex{ex:He tells the story}. It is morphologically defective because it can only be inflected for the imperfective converb and the modal participle, whereas \textit{b-urs-ij} can be inflected for all verb forms and is aspectually neuter. The exact relationship between the two verbs requires further investigation.} 
%
\begin{exe}
	\ex	ergative construction \label{ex:He tells the story} \\
	\gll	hež-i-l	χabar	b-urs-ul	ca-b\\
		this\tsc{-obl-erg}	story	\tsc{n-}tell\tsc{-icvb}	\tsc{cop-n}\\
	\glt	\sqt{He tells the story.}

	\ex	antipassive construction \label{ex:He tells many stories} \\
	\gll	hež	χabur-t-a-l	ux-ul	ca-w\\
		this	story\tsc{-pl-obl-erg}	tell\tsc{.m.ipfv-icvb}	\tsc{cop-m}\\
	\glt	\sqt{He tells stories.}
\end{exe}


%%%%%%%%%%%%%%%%%%%%%%%%%%%%%%%%%%%%%%%%%%%%%%%%%%%%%%%%%%%%%%%%%%%%%%%%%%%%%%%%

\subsection{Causativization}
\label{sec:Causativization}

Sanzhi has a very productive derivational process for the formation of causativized predicates by means of the suffix \tit{-aq}. The derived causativized verbs behave like any other underived verbs, i.e., there are no differences in the range of verbal forms and constructions in which they may appear. The suffixation of \tit{-aq} does not have any impact on the aspectual properties of the verb, such that the differences between imperfective verbs and perfective verbs are preserved. In addition, there are other formal means for making causative constructions such as auxiliary change.

Causative constructions are very widespread among the East Caucasian languages, though not all languages have dedicated derivational suffixes. In Sanzhi Dargwa, causativization normally applies only once to the verbal stem, but in elicitation it can also be added twice. When it is added to the verb, usually the number of arguments of the verb is augmented by one. This means that a monovalent verb becomes bivalent whereby S changes to P and a second argument, the ergative A in the role of the causer is introduced. 
%
\begin{exe}
	\ex	\label{ex:This mill spins around} intransitive \\
	\gll	heχ	urχːab	lus	b-ik'-u\\
		\tsc{dem.down}	mill	around	\tsc{n-}move\tsc{.ipfv-prs.3}\\
	\glt	\sqt{This mill spins around.} (E)


	\ex	\label{ex:This makes the mill spin around} intransitive \\
	\gll	heχ-i-l	heχ	urχːab	lus	b-ik'-aq-u\\
		\tsc{dem.down}\tsc{-obl-erg}	\tsc{dem.}down	mill	around	\tsc{n-}move\tsc{.ipfv-caus-prs.3}\\
	\glt	\sqt{This makes the mill spin around.}

	\ex	\label{ex:‎‎Something like this grows in Sanzhi} transitive \\
	\gll	Sanži-b	b-ik'-u=w	ij=ʁuna?\\
		Sanzhi\tsc{-n}	\tsc{n-}grow\tsc{.ipfv-prs.3=q}	this\tsc{=eq}\\
	\glt	\sqt{Does something like this grows in Sanzhi?}
	
	\ex	\label{ex:‎Marijam was growing cucumbers} transitive \\
	\gll	Marijam-li	χijal-te	d-ač'-aq-ib\\
		Marijam\tsc{-erg}	cucumber\tsc{-pl}	\tsc{npl-}grow\tsc{.pfv-caus-pret}\\
	\glt	\sqt{‎Marijam was growing cucumbers.} (E)
\end{exe}

Similarly, after causativization the S argument of bivalent extended intransitive predicates \refex{ex:The people turned into stones2} becomes P and thus does not change its case marking, the second argument also remains unchanged and a third argument, the causer in the form of an ergative A is added \refex{ex:‎‎‎Irbihin turned the khinkal that was boiling in the pot into frogs}. 
%
\begin{exe}

	\ex	\label{ex:The people turned into stones2} extended intransitive\\
	\gll	χalq'	qːarq-ne	arž-i\\
		people	stone\tsc{-pl}	go\tsc{.ipfv-hab.pst.3}\\
	\glt	\sqt{The people turned into stones.} (E)

	\ex	\label{ex:‎‎‎Irbihin turned the khinkal that was boiling in the pot into frogs} extended transitive\\
	\gll	hek'	ħaˁšukː-a-d	rurčː-an	χːink'-e	ʡaˁt'-ne	arž-aq-i	heχ	Irbihin-ni\\
		\tsc{dem.up}	pot\tsc{-loc-npl}	boil\tsc{-ptcp}	khinkal\tsc{-pl}	frog\tsc{-pl}	go\tsc{.ipfv-caus-hab.pst.3}	\tsc{dem.down}	Irbihin\tsc{-erg}\\
	\glt	\sqt{‎‎‎Irbihin turned the khinkal that was boiling in the pot into frogs.} (E)
\end{exe}

Bivalent transitive predicates become trivalent extended transitive predicates when they are causativized, and the former As become G whereas Ps are unaffected \refex{ex:Madina, mother, and the porridge}. The G argument, that is, the causee, must be marked with the \tsc{in}-lative case. This case is frequently used in valency patterns of various predicates for semantic roles such as addressee, goal or beneficiary, which explains its use in causative constructions. The causee has semantic properties close to these roles since it is the argument, at which the action is directed and that might profit from it. 
%
\begin{exe}
	\ex	\label{ex:Madina, mother, and the porridge}
	\begin{xlist}
		\ex	\label{ex:Madina was eating porridge} transitive \\
		\gll	Madina-l	kaš	b-uk-unne=de\\
			Madina\tsc{-erg}	porridge	\tsc{n-}eat\tsc{.ipfv-icvb=pst}\\
		\glt	\sqt{Madina was eating porridge.} (E)

		\ex	\label{ex:Mother made Madina eat porridge} extended transitive \\
		\gll	aba-l	Madina-cːe	kaš	b-erk-aq-un\\
			mother\tsc{-erg}	Madina\tsc{-in}	porridge	\tsc{n-}eat\tsc{.pfv-caus-pret}\\
		\glt	\sqt{Mother made Madina eat porridge.} (E)
	\end{xlist}
\end{exe}




In the Sanzhi corpus, causativized transitive verbs are rather rare. Sentences \refex{ex:‎‎If it is like this, I will also make him record that other (story)} and \refex{ex:to make the boys drink} show two instances. Many corpus examples of causative constructions have intransitive base verbs such as \refex{ex:This makes the mill spin around} and \refex{ex:‎Marijam was growing cucumbers} above, but causativized affective verbs also occur frequently \refex{ex:(that) doctor Mahammad makes the people find (him)}.
%
\begin{exe}
	\ex	\label{ex:‎‎If it is like this, I will also make him record that other (story)}
	\gll	heχ	cara	zapisat	b-irq'-aq-an=da	du-l	il-i-cːe\\
		\tsc{dem.down}	other	record	\tsc{n-}do\tsc{.ipfv-caus-ptcp=1}	\tsc{1sg-erg} that\tsc{-obl-in}\\
	\glt	\sqt{‎‎If it is like this, I will also make him record that other (story).}

	\ex	\label{ex:to make the boys drink}
	\gll	heχ-tːi	durħ-n-a-cːe	b-erčː-aq-araj\\
		\tsc{dem.down}\tsc{-pl}	boy\tsc{-pl-obl-in}	\tsc{n-}drink\tsc{.pfv-caus-subj.3}\\
	\glt	\sqt{to make the boys drink}
\end{exe}

With bivalent experiential predicates there are two possibilities: either one argument is added or the number of arguments is preserved. In the first case, the experiencer (the former A) becomes G without changing its case marking, but an additional A is added to the clause because the derived verb is trivalent \refex{ex:Patima showed Madina a new dress@59b}. 
%
\begin{exe}
	\ex	\label{ex:Madina, Patima, and the dress@59}
	\begin{xlist}
		\ex	\label{ex:Madina saw a new dress@59a}
		\gll	Madina-j	jangi	kːurtːi	či-b-až-ib\\
			Madina\tsc{-dat}	new	dress	\tsc{spr-n-}see\tsc{.pfv-pret}\\
		\glt	\sqt{Madina saw a new dress.} (E)

		\ex	\label{ex:Patima showed Madina a new dress@59b}
		\gll	Pat'ima-l	Madina-j	jangi	kːurtːi	či-b-iž-aq-ib\\
			Patima\tsc{-erg}	Madina\tsc{-dat}	new	dress	\tsc{spr-n-}see\tsc{.ipfv-caus-pret}\\
		\glt	\sqt{Patima showed Madina a new dress.} (E)
	\end{xlist}
\end{exe}

The same option is available for the causative of \sqt{know}, which translates as \sqt{tell, inform, make know} \refex{ex:‎He thanked the sun}. It is also possible for the experiencer argument to change its case marking from dative to \tsc{in}-lative because the latter case is regularly used for addressees with verbs of speech, but also for causees of causativized transitive and extended transitive verbs \refex{ex:Allah, tell me what happened}, \refex{ex:(that) doctor Mahammad makes the people find (him)}.
%
\begin{exe}
	\ex	\label{ex:‎He thanked the sun}
	\gll	il-i-l	bari-li-j	barkalla	b-aχ-aq-ur\\
		that\tsc{-obl-erg}	sun\tsc{-obl-dat}	thanks	\tsc{n-}know\tsc{.pfv-caus-pret}\\
	\glt	\sqt{‎He thanked the sun.}

	\ex	\label{ex:Allah, tell me what happened}
	\gll	[ce	ag-ur=el],	Allah,	b-aχ-aq-a=kːʷa	di-cːe!\\
		what	go\tsc{.pfv-pret=indq}	Allah	\tsc{n-}know\tsc{.pfv-caus-imp=prt}	\tsc{1sg-in}\\
	\glt	\sqt{Allah, tell me what happened.}

	\ex	\label{ex:(that) doctor Mahammad makes the people find (him)}
	\gll	tuχtur	Maˁħaˁmmad-li	χalq'-li-cːe	w-arčː-aq-ij\\
		doctor	Mahammad\tsc{-erg}	people\tsc{-obl-in}	\tsc{m-}find\tsc{.pfv-caus-inf}\\
	\glt	\sqt{(that) doctor Mahammad makes the people find (him)}
\end{exe}

The second option for affective verbs is not to have any change in the argument structure of the predicate such that both grammatical relations (A and P) as well as semantic roles remain unaltered. Only the semantics of the predicate slightly changes when the verb is causativized \refexrange{ex:Mother likes/wants her son@60a}{ex:Murad loved Madina@60b} and acquires a more agentive reading. This becomes especially obvious when the ergative instead of the dative is used to encode the experiencer of a causativized affective predicate. Verbs that choose this strategy are \tit{b-ikː-} \sqt{want, like}, \tit{b- arkː-} (\tsc{pfv}) \sqt{find}, and \textit{han} \tit{d-irk-} (\tsc{ipfv}) \sqt{remember} (>\tit{han d-irč-aq-}).
%
\begin{exe}
	\ex	\label{ex:Mother, son; Murad, Madina@60}
	\begin{xlist}
		\ex	\label{ex:Mother likes/wants her son@60a}
		\gll	aba-j	durħuˁ	w-ikː-u\\
			mother\tsc{-dat}	boy	\tsc{m-}want\tsc{.ipfv-prs.3}\\
		\glt	\sqt{Mother likes/wants her son.} (E)

		\ex	\label{ex:Murad loved Madina@60b}
		\gll	Murad-li-j	Madina	r-ičː-aq-ib\\
			Murad\tsc{-obl-dat}	Madina	\tsc{f-}want\tsc{.ipfv-caus-pret}\\
		\glt	\sqt{Murad loved Madina.} (E)
	\end{xlist}

	\ex	\label{ex:My cousin loves official appointments very much}
	\gll	d-aq	ħaˁkim-dex	d-ičː-aq-u	di-la	ucːiq'ar-li-j\\
		\tsc{npl-}much	official\tsc{-nmlz}	\tsc{npl-}want\tsc{.ipfv-caus-prs.3}	\tsc{1sg-gen}	cousin-\tsc{obl-dat}\\
	\glt	\sqt{My cousin loves official appointments very much.}
\end{exe}

If trivalent predicates are causativized, then A becomes the causee with the appropriate case suffix (\tsc{in}-lative) and a new causer in the ergative is added to the clause \refex{ex:Father made him show me the way@61}. Since the verb \tit{b-ikː-} (\tsc{pfv})\slash\tit{lukː-} (\tsc{ipfv}) \sqt{give} assigns not only the dative case to the recipient, but alternatively also the \tsc{in}-lative, it is possible to have two arguments with the same case marking in a clause with the causativized verb \sqt{give} \refex{ex:Father made him give me the book}. Due to the identical case marking such clauses are ambiguous.
%
\begin{exe}
	\ex	\label{ex:Father made him show me the way@61}
	\gll	atːa-l	it-i-cːe	dam	xːun	či-b-až-aq-aq-ib\\
		father\tsc{-erg}	\tsc{3sg-obl-in}	\tsc{1sg.dat}	way	\tsc{spr-n-}see\tsc{.pfv-caus-caus-pret}\\
	\glt	\sqt{Father made him show me the way.} (E)

	\ex	\label{ex:Father made him give me the book}
	\gll	atːa-l	di-cːe	it-i-cːe	kiniga	b-ičː-aq-ib\\
		father\tsc{-erg}	\tsc{1sg-in}	\tsc{3sg-obl-in}	book	\tsc{n-}give\tsc{.pfv-caus-pret}\\
	\glt	\sqt{‎‎‎Father made me give him the book.} OR \sqt{Father made him give me the book.} (E)
\end{exe}

In sum, if an additional argument is added by means of causativization, it is always a causer marked with the ergative, independently of the valency class of the base predicate. Because the causer takes the subject position, the original subject (S or A) is demoted into a non-subject position (S > P, A > G), taking over the highest free position on the hierarchy of grammatical relations. For S this is the direct object position (P); for A this is the indirect object position (G) since the direct object position (P/T) is already occupied. It is never P or T that is affected when bivalent or trivalent predicates are causativized such that causativization can perhaps be taken as a weak indicator of an accusative pivot (see the discussion of grammatical roles in \refsec{sec:Grammatical relations}).

Double causativization seems to be possible, as \refex{ex:Father made him show me the way@61} shows, and can lead to the addition of two arguments (i.e. the two-place verb \sqt{see} becomes a four place verb). However, it can also be used for emphasis only such that the second causativization does not result in the addition of a second argument \refex{ex:Ali made the dogs fight UNCERTAIN INTERPRETATION}. In the corpus I found only one example of this \refex{ex:‎He hit her and (we) were made to fight, he says}. The precise properties of double causative constructions are hard to determine because speakers have divergent intuitions about the acceptability and meaning of elicited examples and the only corpus example \refex{ex:‎He hit her and (we) were made to fight, he says} is difficult to understand and to judge, even within its context.
%
\begin{exe}
	\ex	\label{ex:The dogs fought}
	\gll	χːu-de d-iħ-ib\\
		dog\tsc{-pl}	\tsc{npl-}wrestle\tsc{.pfv-pret}\\
	\glt	\sqt{The dogs fought.} (E)

	\ex	\label{ex:Ali made the dogs fight}
	\gll	ʡaˁli-l χːu-de d-iħ-aˁq-ib\\
		Ali\tsc{-erg} dog\tsc{-pl}	\tsc{npl-}wrestle\tsc{.pfv-caus-pret}\\
	\glt	\sqt{Ali made the dogs fight.} (E)

	\ex	\label{ex:Ali made the dogs fight UNCERTAIN INTERPRETATION}
	\gll	ʡaˁli-l χːu-de d-iħ-aˁq-aˁq-ib\\
		Ali\tsc{-erg} dog\tsc{-pl}	\tsc{npl-}wrestle\tsc{.pfv-caus-caus-pret}\\
	\glt	\sqt{Ali made the dogs fight.}  (E)

	\ex	\label{ex:‎He hit her and (we) were made to fight, he says}
	\gll	hek'-i-l	b-aˁq-ib-le,	d-iħ-aˁq-aˁq-ib=da	Ø-ik'-ul\\
		\tsc{dem.up}\tsc{-obl-erg}	\tsc{n-}strike\tsc{.pfv-pret-cvb}	\tsc{1/2pl-}wrestle\tsc{.pfv-caus-caus-pret=1}	\tsc{m-}say\tsc{.ipfv-icvb}\\
	\glt	\sqt{‎He hit her, and ``(we) were made to fight'', he says.}
\end{exe}

The meaning of causative constructions can be described as the expression of \dqt{a causal relation between two events, one of which is believed by the speaker to be caused by the other} \citep{Kulikov2011}. Depending on the semantics of the predicate and on the context, the meaning of the causative construction can be close to force (\sqt{make do X}, \sqt{cause to X}), but it can also be \sqt{soft causation}, i.e., asking, requesting or begging \refex{ex:‎We will ask for telling the story}, or sometimes even quite idiosyncratic and unpredictable. Thus, the meaning of the causativized intransitive verb \tit{b-ucː-} \sqt{work} is \sqt{support, sustain} \refex{ex:He feeds his family (he sustains his family working)} in addition to the expected causative meaning of \sqt{make work}.
%
\begin{exe}
	\ex	\label{ex:‎We will ask for telling the story}
	\gll	χabar	b-urs-aq-an=da\\
		story	\tsc{n-}tell\tsc{-caus-ptcp=1}\\
	\glt	\sqt{‎We will ask for a story to be told.}

	\ex	\label{ex:‎They are also for those who want to buy them}
	\gll	d-ikː-an-il-li-j	asː-aq-ij	iχ-tːi=ra	χe-d\\
		\tsc{npl-}want\tsc{.ipfv-ref-ptcp-obl-dat}	buy\tsc{.pfv-caus-inf} \tsc{dem.down}\tsc{-pl=add} exist.\tsc{down-npl}\\
	\glt	\sqt{‎They are also for those who want to buy them.} (lit. \sqt{They are to make buy those who want them.})

	\ex	\label{ex:‎‎‎(they were) building, working for three days}
	\gll	b-irq'-ul	k'ʷel	ʡaˁbal	bar	b-ucː-ib-le\\
		\tsc{n-}do\tsc{.ipfv-icvb}	two	three	day	\tsc{hpl-}work\tsc{-pret-cvb}\\
	\glt	\sqt{‎‎‎(they were) building, working for three days}

	\ex	\label{ex:He feeds his family (he sustains his family working)}
	\gll	hel-i-l	cin-na	kulpat	b-ucː-aq-ul ca-b\\
		that\tsc{-obl-erg}	\tsc{refl.sg-gen}	family	\tsc{hpl-}work\tsc{-caus-icvb} \tsc{cop-hpl}\\
	\glt	\sqt{He feeds his family (he sustains his family working).} OR \sqt{He makes his family work.}
\end{exe}

Another way of forming causative constructions is by means of transitive light verbs. This operation is applied with compound verbs that contain intransitive light verbs (\refsec{sec:Formation of causative verbs}).



%\begin{exe}
%	\ex	\label{ex:}
%	\gll	\\
%		\\
%	\glt	\sqt{}
%\end{exe}
%
%\begin{exe}
%	\ex	\label{ex:}
%	\begin{xlist}
%		\ex	\label{ex:}
%		\gll	\\
%			\\
%		\glt	\sqt{}
%
%		\ex	\label{ex:}
%		\gll	\\
%			\\
%		\glt	\sqt{}
%	\end{xlist}
%\end{exe}





%\begin{exe}
%	\ex	\label{ex:}
%	\gll	\\
%		\\
%	\glt	\sqt{}
%\end{exe}
%
%\begin{exe}
%	\ex	\label{ex:}
%	\begin{xlist}
%		\ex	\label{ex:}
%		\gll	\\
%			\\
%		\glt	\sqt{}
%
%		\ex	\label{ex:}
%		\gll	\\
%			\\
%		\glt	\sqt{}
%	\end{xlist}
%\end{exe}
