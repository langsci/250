\chapter{Analytic verb forms}
\label{cpt:Analytic verb forms}

All verb forms consisting of a lexical verb bearing a participial or converbal suffix (and possible other suffixes) followed by a person \isi{enclitic}, the past \isi{enclitic}, the \isi{copula} \tit{ca-b}, or the suffix -\textit{ne} are called ``\is{analytic verb form}analytic verb forms'' and described in this chapter. When the standard \isi{copula} is replaced by locational copulas or other auxiliaries, the resulting verb forms will be called ``periphrastic'', and they are separately treated in \refcpt{cpt:Periphrastic verb forms}. The division between analytic and \is{periphrastic verb form}periphrastic verb forms is mainly based on differences in morphology, semantics, and frequency of use. Among the morphologically complex verb forms, \is{analytic verb form}analytic verb forms are the core verb forms because they are basic in terms of the semantics and pragmatics of the inflectional element that accompanies the lexical verb. This element (person \isi{enclitic}, past \isi{enclitic}, standard \isi{copula}, suffix -\textit{ne}) expresses basic verbal categories such as tense, person, \isi{number}, and \isi{gender}.\footnote{Agreement rules and agreement exponents, i.e., \isi{gender} affixes, person suffixes and person enclitics, are separately treated in \refcpt{cpt:Agreement} and therefore not discussed in this chapter.} The lexical verb conveys aspectual and modal meaning. By contrast, in \is{periphrastic verb form}periphrastic verb forms the accompanying auxiliary has additional modal, locational, evidential or aspectual meanings that contribute to the meaning of the complex predicate, which is therefore more specific. Furthermore, the accompanying auxiliary verbs of \is{periphrastic verb form}periphrastic verb forms are also used as full lexical verbs, but not as semantically empty copulas in \isi{copula} clauses. The latter use is only attested for person enclitics, the past \isi{enclitic} and the standard \isi{copula}. Because of their more general meaning most \is{analytic verb form}analytic verb forms occur far more frequently in texts than the \is{periphrastic verb form}periphrastic verb forms with their more specific meaning.     

The \is{analytic verb form}analytic verb forms can be divided into two main groups: forms based on the imperfective stem (\refsec{sec:Forms based on the imperfective stem}) and forms based on the preterite (\refsec{sec:Forms based on the preterite}). The former convey mainly present time or future time reference (and an imperfective past), whereas the latter almost exclusively convey past time reference.


%%%%%%%%%%%%%%%%%%%%%%%%%%%%%%%%%%%%%%%%%%%%%%%%%%%%%%%%%%%%%%%%%%%%%%%%%%%%%%%%

\section{Forms based on the imperfective stem}
\label{sec:Forms based on the imperfective stem}

The TAM forms that can be obtained from the imperfective stem can be divided into two groups, depending on whether the lexical verb bears the \isi{imperfective converb} suffix or the \isi{modal participle} \tit{-an} (\reftab{tab:Analytic verb forms based on the imperfective stem}). The second group has a modal meaning due to the semantics of the \isi{participle}. All forms make use of person enclitics/\isi{copula} \tit{ca-b} for present or future time reference and the past \isi{enclitic} \tit{=de} for past time reference. The following subsections treat all \is{analytic verb form}analytic verb forms based on the imperfective stem according to the order in the table.
%

\begin{table}
	\caption{Analytic verb forms based on the imperfective stem}
	\label{tab:Analytic verb forms based on the imperfective stem}
	\small
	\begin{tabularx}{0.98\textwidth}[]{%
		>{\raggedright\arraybackslash}p{85pt}
		>{\raggedright\arraybackslash}X
		>{\raggedright\arraybackslash}X}
		
		\lsptoprule
			label of TAM form	&	lexical verb & inflection\\
		\midrule
			\multicolumn{3}{l}{{non-modal forms that employ the imperfective converb}}\\\midrule
			~~compound present	&	imperfective					&	+ person enclitics/\isit{copula}\\
			~~compound past 		&	\hspace*{8pt}converb				&	+ past \isit{enclitic} \tit{=de}\\\midrule
			\multicolumn{3}{l}{{modal forms that employ the \isi{participle} -\textit{an}}}\\\midrule
			~~future 			&			&	+ person enclitics/\tit{-ne}\\
			~~future in the past	&	\isit{participle} \tit{-an}							&	+ past \isit{enclitic} \tit{=de}\\
			~~obligative			&	{}							&	+ \isit{copula}\\
			~~obligative present	&	\isit{participle} \tit{-an}							&	 		+ person enclitics/\isit{copula}\\
			~~obligative past 		&	\hspace*{8pt}+ \tit{-ce\slash-te}						&							+ past \tit{=de}\\
		\lspbottomrule
	\end{tabularx}
\end{table}


% --------------------------------------------------------------------------------------------------------------------------------------------------------------------------------------------------------------------- %

\subsection{Compound present}
\label{ssec:Compound present}

The compound present is obtained by adding the imperfective con\-verb \tit{-ul}\slash\mbox{\tit{-un(ne)}} to the verbal stem,\footnote{The \isi{imperfective converb} is, at least diachronically, related to the cross-categorical \isi{adverbializer} -\textit{le} (\refsec{ssec:The adverbializer -le}), and thus also to the \isi{perfective converb}. However, in order to facilitate understanding I treat the converbs and the \isi{adverbializer} as separate items.} which is in turn followed by the person enclitics (first and second person) or by the \isi{copula} \tit{ca-b} (third person).
%
\begin{table}
	\caption{Some exemplary paradigms of the compound present}
	\label{tab:Some exemplary paradigms of the compound present}
	\small
	\begin{tabularx}{0.80\textwidth}[]{%
		>{\raggedright\arraybackslash}p{12pt}
		>{\raggedright\arraybackslash\itshape}X
		>{\raggedright\arraybackslash\itshape}X
		>{\raggedright\arraybackslash\itshape}X
		>{\raggedright\arraybackslash\itshape}X}
		
		\lsptoprule
			{}	&	\multicolumn{2}{c}{\sqt{eat}}	&	\multicolumn{2}{c}{\sqt{do}}\\\cmidrule(lr){2-3}\cmidrule(lr){4-5}
			{}	&	\multicolumn{1}{c}{singular}
				&	\multicolumn{1}{c}{plural}
				&	\multicolumn{1}{c}{singular}
				&	\multicolumn{1}{c}{plural}\\
		\midrule
			1	&	b-uk-un=da	&	b-uk-un=da	&	b-irq'-ul=da	&	b-irq'-ul=da\\
			2	&	b-uk-un=de	&	b-uk-un=da	&	b-irq'-ul=de	&	b-irq'-ul=da\\
			3	&	b-uk-un ca-b	&	b-uk-un ca-b	&	b-irq'-ul ca-b	&	b-irq'-ul ca-b\\
		\lspbottomrule
	\end{tabularx}
\end{table}
%
The compound present is the default tense for conveying present time reference. It covers various imperfective meanings such as progressive, habitual, or continuative.
%
\begin{enumerate}
	\item	Progressive: actions and events that are happening at the moment of speech. In this function, it can also be used with stative verbs.
	%
	\begin{exe}
		\ex	\label{ex:Now I am telling a story}
		\gll	hana	du-l	b-urs-ul=da	χabar\\
			now	\tsc{1sg-erg}	\tsc{n-}tell\tsc{-icvb=1}	story\\
		\glt	\sqt{Now I am telling a story.}

		\ex	\label{ex:Now these (games), many things slowly occur (on my mind)}
		\gll	na	il-tːi	bahla-bahla	d-aqil	cik'al	han	d-irk-ul	ca-d	na\\
			now	that\tsc{-pl}	slow-slow	\tsc{npl-}much	thing	remember	\tsc{npl-}occur\tsc{.ipfv-icvb}	\tsc{cop-npl}	now\\
		\glt	\sqt{Now these (games), many things are slowly occurring (to my mind).}
	\end{exe}

	\item	Habitual: describing general characteristic actions or what people do over an extended period of time, descriptions of games, etc. This use of the compound present strongly resembles the \isi{habitual present} (\refsec{sec:vis-habitualpresent}).
	%
	\begin{exe}
		\ex	\label{ex:The others are hiding analytic}
		\gll	cara-te	daˁʡaˁn	b-irxː-ul	ca-b\\
			other\tsc{-dd.pl} 	secret	\tsc{hpl-}put\tsc{.ipfv-icvb}	\tsc{cop-hpl}\\
		\glt	\sqt{The others hide.} (in a game of hide-and-seek)

		\ex	\label{ex:What do they call it (a plant) in Russian, I forget it analytic}
		\gll	ʡuˁrus	ʁaj-la	ce=jal	b-ik'-ul ca-b	it-i-j,	dam	qum.urt-ul ca-b\\
			Russian	language\tsc{-gen}	what\tsc{=indq}	\tsc{hpl-}say\tsc{.ipfv-icvb}	be\tsc{-hpl}	that\tsc{-obl-dat}	\tsc{1sg.dat}	forget\tsc{.ipfv-icvb}	\tsc{cop-n}\\
		\glt	\sqt{What do they call it in Russian, I forget it.}
	\end{exe}

	\item	Historical present: continuative actions in narrations about the past.
	%
	\begin{exe}
		\ex	\label{ex:He is looking around in the village, nobody is there analytic}
		\gll	šːi-l-cːe-w	er	Ø-ik'-ul	ca-w,	ča-k'al	caʔarrah	admi	w-akːu\\
			village\tsc{-obl-in-m}	look	\tsc{m-}look.at\tsc{.ipfv-icvb}	\tsc{cop-m}	who\tsc{-indef}	not.one	person	\tsc{m-cop}\tsc{.neg}\\
		\glt	\sqt{He is looking around in the village, nobody is there.}
	\end{exe}
\end{enumerate}

Negation can be expressed either through the negative prefix \textit{a-} or by means of the negative auxiliary. In the former case, which represents the rarer variant, the \isi{negation} suffix is simply added to the lexical verb \refex{ex:I do not know what to do analytic}. In the latter case, which is far more common, the negative \isi{copula} \tit{akːʷa-} \sqt{be not} (\tsc{cop.neg}) is used \refex{ex:You tell only my words analytic}; it is inflected for person, but not for \isi{gender} (see \refsec{sec:The copula} for the paradigm of the negative \isi{copula}).
%
\begin{exe}
	\ex	\label{ex:I do not know what to do analytic}
	\gll	ce	b-arq'-idel	a-b-alχ-ul=da\\
		what	\tsc{n-}do\tsc{.pfv-modq}	\tsc{neg-n-}know\tsc{.ipfv-icvb=1}\\
	\glt	\sqt{I do not know what to do.}

	\ex	\label{ex:You tell only my words analytic}
	\gll	di-la	ʁaj	b-urs-ul	akːʷa-tːe	u-l\\
		\tsc{1sg-gen}	word	\tsc{n-}tell\tsc{.pfv-icvb}	\tsc{cop.neg-2sg}	\tsc{2sg-erg}\\
	\glt	\sqt{You tell only my words.}
\end{exe}

In \isi{questions} with a third-person \isi{agreement controller}, the \isi{copula} is replaced by the respective interrogative \isi{enclitic} \refex{ex:What is he doing analytic}. In such contexts, the interrogative \isi{enclitic} acts as a predicative \isi{particle} and takes over the role of the \isi{copula} (\refsec{sec:Predicative particles}).
%
\begin{exe}
	\ex	\label{ex:What is he doing analytic}
	\gll	hež-i-l	ce	b-irq'-ul=e?\\
		this\tsc{-obl-erg}	what	\tsc{n-}do\tsc{.ipfv-icvb=q}\\
	\glt	\sqt{What is he doing?}
\end{exe}

The compound present can also be formed by means of existential copulas instead of the normal \isi{copula}, which leads to a slight change in the meaning (\refsec{sec:Locational copulae}).


% --------------------------------------------------------------------------------------------------------------------------------------------------------------------------------------------------------------------- %

\subsection{Compound past}
\label{ssec:Compound past}

The compound past is formed by encliticizing the past marker \tit{=de} to the imperfective stem that bears the \isi{imperfective converb} suffix. Its semantics corresponds to the semantics of the compound present, but now we have past time reference.
%
\begin{enumerate}
	\item	Progressive/continuative: ongoing actions and states that continuously obtained in the past, situations of long duration of which the endpoint (and the beginning) is not important.
	%
	\begin{exe}
		\ex	\label{ex:In the mountains there was a man gathering pears analytic}
		\gll	dubur-t-a-cːe-b	ca	admi-l	quˁr-be	luc'-unne=de\\
			mountain\tsc{-pl-obl-in}\tsc{-n}	one	person\tsc{-erg}	pear\tsc{-pl}	gather\tsc{.ipfv-icvb=pst}\\
		\glt	\sqt{In the mountains there was a man gathering pears.}
	\end{exe}

	\item	Habitual: general characteristics of situations and habits of people.
	%
	\begin{exe}
		\ex	\label{ex:He helped the people characteristics of the grandfather}
		\gll	χalq'-li-j	kumek	b-irq'-ul=de\\
			people\tsc{-obl-dat}	help	\tsc{n-}do\tsc{.ipfv-icvb=pst}\\
		\glt	\sqt{He helped the people.} (character trait of the grandfather)

		\ex	{[description of a game]}\\\label{ex:put wooden like this, we made a circle}
		\gll	urcul-la		hež-itːe	ka-b-irxː-ul=de	ca	krug	b-irq'-ul=de	\\
			wood\tsc{-gen}		this\tsc{-advz}	\tsc{down-n-}put\tsc{.ipfv-icvb=pst}	one	circle	\tsc{n-}do\tsc{.ipfv-icvb=pst}\\
		\glt	\sqt{(We) put wooden (sticks?) like this, we made a circle.}
	
		\ex	\label{ex:(In contrast to now, at that time my hand) worked, I was writing}
		\gll	b-ucː-ul	b-el=de,		luk'-unne=de	\\
			\tsc{n-}work\tsc{-icvb}	\tsc{n-}remain\tsc{.pfv=pst}		write\tsc{.ipfv-icvb=pst}\\
		\glt	\sqt{(In contrast to now, at that time my hand) worked, I wrote.}
	\end{exe}
\end{enumerate}

For the \isi{negation} there are again two options: prefixation of \tit{a-} \refex{ex:‎Couldn't the bandits go to the graveyard} and use of the negative \isi{copula} inflected for the \isi{past tense}, with the latter option being more frequent \refex{ex:The car did not shake}.
%
\begin{exe}
	\ex	\label{ex:‎Couldn't the bandits go to the graveyard}
	\gll	c'il	qːačuʁ-e	χːuˁrba-cːe	b-uˁq'-ij	a-b-irχ-ul=de=w?\\
		then	bandit\tsc{-pl}	graveyard\tsc{-in}	\tsc{hpl-}go\tsc{-inf}	\tsc{neg-hpl-}be.able\tsc{.ipfv-icvb=pst=q}\\
	\glt	\sqt{‎Couldn't the bandits go to the graveyard?}

	\ex	\label{ex:The car did not shake}
	\gll	hak'	b-ulq-unne	akːʷ-i	mašina\\
		shake	\tsc{n-}direct\tsc{.ipfv-icvb}	\tsc{cop.neg-hab.pst}	car\\
	\glt	\sqt{The car did not shake.}
\end{exe}


% --------------------------------------------------------------------------------------------------------------------------------------------------------------------------------------------------------------------- %

\subsection{Future}
\label{ssec:Future analytic}

The future is formed by adding the person enclitics to the lexical verb that bears the \isi{participle} \tit{-an}. In the third person, the suffix \tit{-ne} is used (\reftab{tab:Some exemplary paradigms of the future}). 
%
\begin{table}
	\caption{Some exemplary paradigms of the future}
	\label{tab:Some exemplary paradigms of the future}
	\small
	\begin{tabularx}{0.80\textwidth}[]{%
		>{\raggedright\arraybackslash}p{12pt}
		>{\raggedright\arraybackslash\itshape}X
		>{\raggedright\arraybackslash\itshape}X
		>{\raggedright\arraybackslash\itshape}X
		>{\raggedright\arraybackslash\itshape}X}
		
		\lsptoprule
			{}	&	\multicolumn{2}{c}{\sqt{eat}}	&	\multicolumn{2}{c}{\sqt{do}}\\\cmidrule(lr){2-3}\cmidrule(lr){4-5}
			{}	&	\multicolumn{1}{c}{singular}
				&	\multicolumn{1}{c}{plural}
				&	\multicolumn{1}{c}{singular}
				&	\multicolumn{1}{c}{plural}\\
		\midrule
			1	&	b-uk-an=da	&	b-uk-an=da	&	b-irq'-an=da	&	b-irq'-an=da\\
			2	&	b-uk-an=de	&	b-uk-an=da	&	b-irq'-an=de	&	b-irq'-an=da\\
			3	&	b-uk-an-ne	&	b-uk-an-ne	&	b-irq'-an-ne	&	b-irq'-an-ne\\
		\lspbottomrule
	\end{tabularx}
\end{table}

Its semantic range is:
%
\begin{enumerate}
	\item	Future: future time reference, predictions of future situations, potential situations.
	%
	\begin{exe}
		\ex	\label{ex:You will be able to spend the 40 days (without me)}
		\gll	aʁʷc'alla	d-arq'-ij	d-irχ-an=da	ušːa-l\\
			forty.days	\tsc{1/2.pl-}do\tsc{.pfv-inf}	\tsc{1/2.pl-}be.able\tsc{.ipfv-oblg=2pl}	\tsc{2pl-erg}\\
		\glt	\sqt{You (pl.) will be able to spend the 40 days (without me).}
	
		\ex	\label{ex:Now (she) will eat you, too analytic}
		\gll	hana	u=ra	Ø-ukː-an=de\\
			now	\tsc{2sg=add}	\tsc{m-}eat\tsc{.ipfv-ptcp=2sg}\\
		\glt	\sqt{Now (she) will eat you (masc.), too.}
	\end{exe}

	\item	Modal meaning: expression of obligation.
	%
	\begin{exe}
		\ex	\label{ex:If there was not the work with the animals, what work would/should there be}
		\gll	hel-tːi	ħaˁjwan-qːačːa-la	akːʷ-ar, cara	ce ʡaˁči b-irχʷ-an-ne	hextːu-b	b-i-b?\\
			that\tsc{-pl.abs}	animal-calf\tsc{-gen}	\tsc{cop.neg-prs}	other	what	work	\tsc{n-}be\tsc{.ipfv-ptcp-fut.3}	there.\tsc{up-n} 	\tsc{n-}in\tsc{-n}\\
		\glt	\sqt{If there was not the work with the animals, what work would/should there be? }

		\ex	\label{ex:What else should be said analytic}
		\gll	cara	ce	Ø-ik'ʷ-an-ne?\\
			other	what	\tsc{m-}say\tsc{.ipfv-ptcp-fut.3}\\
		\glt	\sqt{What else should be said?}

		\ex	\label{ex:Until you take the stick out, you have to stand here}
		\gll	heχ	u-l	dirxːa	gu-r-b-uqː-ij=sat,	k-ercː-an=de 	heštːu\\
			\tsc{dem.down}	\tsc{2sg-erg}	stick	\tsc{sub-abl}-\tsc{n-}take.out\tsc{.ipfv-inf=}as.much	\tsc{down}-stand\tsc{.ipfv-ptcp=2sg}	here\\
		\glt	\sqt{Until you take the stick out, you have to stand here.}
	\end{exe}
\end{enumerate}

Negation is expressed by means of the prefix \tit{a-}:
%
\begin{exe}
	\ex	\label{ex:I will not sell them, I said analytic}
	\gll	a-d-irc-an=da	haʔ-ib=da\\
		\tsc{neg-npl-}sell\tsc{.ipfv-ptcp=1}	say\tsc{.pfv-pret=1}\\
	\glt	\sqt{I will not sell them, I said.}
\end{exe}


% --------------------------------------------------------------------------------------------------------------------------------------------------------------------------------------------------------------------- %

\subsection{Future in the past}
\label{ssec:Future in the past}

The future in the past is formed by adding the past \isi{enclitic} \textit{=de} to the \isi{participle} \tit{-an}. It expresses irrealis modality, referring to situations and actions that should have taken place or performed in the past \xxref{ex:‎Her story (i.e. autobiography) should be written down}{ex:You should have not left your body, even if the end of the world comes}. It is also used in the \isi{counterfactual} apodosis of irrealis \is{conditional clause}conditional clauses \refex{ex:I would have read the book if you (masc.) would have seen it} (\refsec{sec:The syntax of conditional clauses}). The negative prefix is used for \isi{negation} \refex{ex:You should have not left your body, even if the end of the world comes}.
%
\begin{exe}
	\ex	\label{ex:‎Her story (i.e. autobiography) should be written down}
	\gll	wat'	hek'-i-la	istorija	luk'-an=de\\
		well	\tsc{dem.up}\tsc{-obl-gen}	story	write\tsc{.ipfv-ptcp=pst}\\
	\glt	\sqt{‎Her story (i.e. autobiography) should have been written down.}

	\ex	\label{ex:They would/should have done something for me}
	\gll	ču-l	b-irq'-an=de	dam	kːakːjuta \\
		\tsc{refl.pl-erg}	\tsc{n-}do\tsc{.ipfv-ptcp=pst}	\tsc{1sg.dat}	something\\
	\glt	\sqt{They would/should have done something for me.}

	\ex	\label{ex:You should have not left your body, even if the end of the world comes}
	\gll	žan-ni-cːe-r	a-r-ulq-an=de,		ka-d-icː-ar=ra	q'ijama\\
		body\tsc{-obl-in-abl}	\tsc{neg-f-}direct\tsc{.ipfv-ptcp=pst}		\tsc{down-npl-}stand\tsc{.pfv-cond.3=add}	end.of.world\\
	\glt	\sqt{You (fem.) should not have left your body, even if the end of the world comes.}

	\ex	\label{ex:I would have read the book if you (masc.) would have seen it}
	\gll	du-l	kiniga	b-uč'-an=de,	raχle	či-d-ig-ul	Ø-iχ-utːel\\
		\tsc{1sg-erg}	book	\tsc{n-}read\tsc{.ipfv-ptcp=pst}	if	\tsc{spr-npl-}see\tsc{.ipfv-icvb}	\tsc{m-}be\tsc{.pfv-cond.pst}\\
	\glt	\sqt{I would have read the book if I (masc.) had seen it.} (E)
\end{exe}


% --------------------------------------------------------------------------------------------------------------------------------------------------------------------------------------------------------------------- %

\subsection{Obligative}
\label{ssec:Obligative}

The obligative is formally and functionally closely related to the future, but it makes use of the \isi{copula} for all third persons instead of person enclitics. The meaning is usually modal referring to needs and obligations, close to deontic necessity. For \isi{negation} the \isi{copula} \tit{ca-b} is replaced by the negative \isi{copula} \tit{akːu} \refex{ex:I have / do not have to eat bread}, \refex{ex:The elder uncle has to go; the younger aunt should not go}.
%
\begin{exe}
	\ex	\label{ex:Well, he says, this needs to be done}
	\gll	``wat'',	Ø-ik'-ul	ca-w,	``heχ	b-irq'-an	ca-b''	\\
		well	\tsc{m-}say\tsc{.ipfv-icvb}	\tsc{cop-m}	\tsc{dem.down}	\tsc{n-}do\tsc{.ipfv-ptcp}	\tsc{cop-n}\\
	\glt	\sqt{``Well,'' he says, ``this needs to be done.''}


	\ex	\label{ex:It must be a strong medicine}
	\gll	c'aq'	darman-na	b-irχʷ-an	ca-b\\
		strong		medicine-\tsc{gen}	\tsc{n-}become\tsc{.ipfv-ptcp}	\tsc{cop-n}\\
	\glt	\sqt{It must be a strong medicine.}


	\ex	\label{ex:I have / do not have to eat bread}
	\gll	du-l t'ult' b-uk-an	ca-b	/	akː-u\\
		\tsc{1sg-erg}	bread	\tsc{n-}eat\tsc{.ipfv-ptcp}	\tsc{cop-n} /	\tsc{cop.neg-prs}\\
	\glt	\sqt{I have\slash do not have to eat bread.} (E)
	
	\ex	\label{ex:The elder uncle has to go; the younger aunt should not go}
	\gll	χːula	acːi	k-erʁ-an	ca-w,	nik'a	azi	ka-r-irʁ-an	akː-u\\
		big	uncle	\tsc{down}-come\tsc{-ptcp}	\tsc{cop-m}	small	aunt	\tsc{down-f-}come\tsc{.ipfv-ptcp}	\tsc{cop.neg-prs}\\
	\glt	\sqt{The elder uncle has to go; the younger aunt should not go.}
\end{exe}


% --------------------------------------------------------------------------------------------------------------------------------------------------------------------------------------------------------------------- %

\subsection{Obligative present}
\label{ssec:Obligative present}

The obligative present strongly resembles the future and the obligative. With both forms it shares the meaning, this means that, the obligative present expresses future and/or obligation. The only formal difference is the additional use of the cross-categorical suffixe -\textit{ce} (plural -\textit{te}), which is added to the \isi{participle} before the person marker is encliticized.
%
\begin{exe}
	\ex	\label{ex:‎Where will you (pl) go he says}
	\gll	``čina	d-ax-an-te=da=jal,''	Ø-ik'-ul	ca-w	\\
		where	\tsc{1/2pl-}go\tsc{.ipfv-ptcp-dd.pl=2=indq}	\tsc{m-}say\tsc{.ipfv-icvb}	\tsc{cop-m}	\\
	\glt	\sqt{``‎Where will you (pl.) go?'' he says.}

	\ex	\label{ex:I will go / have to go analytic}
	\gll	du	w-ax-an-ce=da\\
		\tsc{1sg}	\tsc{m-}go\tsc{.ipfv-ptcp-dd.sg=1}\\
	\glt	\sqt{I (masc.) will go\slash have to go.} (E)
\end{exe}

Note that in \refex{ex:I will go / have to go analytic}, although the singular form \tit{-an-ce} is available in elicitation, it is not attested in the corpus and it seems that for the singular the future (\refsec{ssec:Future analytic}) (or the obligative, \refsec{ssec:Obligative}) is preferred. All corpus examples contain the plural suffix \tit{-te}. The reason for this might be the general distribution of the \is{cross-categorical suffix}cross-categorical suffixes -\textit{ce} /-\textit{te} and -\textit{il}. For plural referents only -\textit{te} is available \refex{ex:‎Where will you (pl) go he says}. In the singular, in principle -\textit{ce} and -\textit{il} compete, but in natural texts the use of -\textit{il} is clearly preferred and only a few examples of -\textit{ce} can be found in the corpus (see \refsec{ssec:The -ce / -te attributive} and \refsec{ssec:The -il attributive} for detailed analyses of the suffixes). The obligative present (and the obligative past) cannot be formed by means of the suffix -\textit{il} for reasons that are not clear to me. Therefore, the more natural way of forming a singular from a  \refex{ex:‎Where will you (pl) go he says} is to use the future form \textit{du w-ax-an=da} instead of \refex{ex:I will go / have to go analytic}.

In the third person the \isi{copula} is used \refex{ex:He will go / have to go E}. However, the \isi{copula} can also be employed with first and second person, in which case the meaning of obligation is dominant \refex{ex:Then you have to find them analytic}, \refex{ex:Our flour we should / (will) grind, he says analytic}. Negation is expressed by means of the negative \isi{copula} \refex{ex:‎To a family such misfortune should not happen analytic}.
%
\begin{exe}
	\ex	\label{ex:He will go / have to go E}
	\gll	it	w-ax-an-ce	ca-w\\
		\tsc{dem}	\tsc{m-}go\tsc{.ipfv-ptcp-dd.sg}	\tsc{cop-m}\\
	\glt	\sqt{He will go\slash have to go.} (E)

	\ex	\label{ex:Then you have to find them analytic}
	\gll	c'il	u-l	b-urkː-an-te	ca-b	hel-tːi\\
		then	\tsc{2sg-erg}	\tsc{hpl-}find\tsc{.ipfv-ptcp-dd.pl} 	\tsc{cop-hpl}	that\tsc{-pl}\\
	\glt	\sqt{Then you have to find them.}

	\ex	\label{ex:Our flour we should / (will) grind, he says analytic}
	\gll	nišːa-la	bet'u	luq'-an-te ca-d	Ø-ik'-ul	ca-w\\
		\tsc{1pl-gen}	flour	grind\tsc{.ipfv-ptcp-dd.pl} 	\tsc{cop-npl}	\tsc{m-}say\tsc{.ipfv-icvb}	\tsc{cop-m}\\
	\glt	\sqt{Our flour we should\slash (will) grind, he says.} OR \sqt{Our flour should be ground, he says.}

	\ex	\label{ex:‎To a family such misfortune should not happen analytic}
	\gll	kulpat-li-j	heχ-tːi	bala	či-ka-jʁ-an-te=kːu\\
		family\tsc{-obl-dat}	\tsc{dem.down}\tsc{-pl}	misfortune	\tsc{spr-down}-come\tsc{.ipfv-ptcp-dd.pl=neg}\\
	\glt	\sqt{‎To a family such misfortune should not happen.}
\end{exe}


% --------------------------------------------------------------------------------------------------------------------------------------------------------------------------------------------------------------------- %

\subsection{Obligative past}
\label{ssec:Obligative past}

The obligative past is formed by replacing the person \isi{enclitic} or \isi{copula} of the obligative present with the past \isi{enclitic}. It refers to obligations that obtained in the past and that were or were not fulfilled \xxref{ex:Then we had to be back in Icari analytic}{ex:‎‎These (pictures) should have been come in the right order}.
%
\begin{exe}
	\ex	\label{ex:Then we had to be back in Icari analytic}
	\gll	c'il	uc'ari	čar	d-irχʷ-an-te=de	nušːa\\
		then	Icari	back	\tsc{1/2pl-}become\tsc{.ipfv-ptcp-dd.pl=pst}	\tsc{1pl}\\
	\glt	\sqt{Then we had to be back in Icari.}

	\ex	\label{ex:(You) should have told your thoughts analytic}
	\gll	hel-tːi	ala	pikri=ra	herʔ-an-te=de\\
		that\tsc{-pl}	\tsc{2sg.gen}	thought\tsc{=add}		say\tsc{.ipfv-ptcp-dd.pl=pst}\\
	\glt	\sqt{(You) should have told your thoughts.}

	\ex	\label{ex:‎‎These (pictures) should have been come in the right order}
	\gll	heštːi	padrjad	sa-d-aš-aq-an-te=de\\
		these	in.order	\tsc{hither-npl-}go\tsc{.ipfv-caus-ptcp-dd.pl=pst}\\
	\glt	\sqt{‎‎These (pictures) should have been come in the right order.}
\end{exe}

The obligative present and past forms can also have a non-modal and non-future reading when they are instead interpreted like headless \is{relative clause}relative clauses and the person \isi{enclitic}, \isi{copula} or past marker makes up its own \isi{copula} clause \refex{ex:There were the drinking ones; these (are) the working ones}. Thus, in the first part of this sentence the \isi{participle} has been nominalized by means of the \isi{cross-categorical suffix} -\textit{te}, which corresponds to a headless \isi{relative clause} (`the drinking ones'). This nominalized clause functions as subject in an an \isi{existential copula} in which the encliticized past marker =\textit{de} serves as an \isi{existential copula}. The nominalized clause does not have modal or future semantics. The second part has a similar meaning, but the \isi{copula} is missing such that we have only the nominalized clause, which is more complex. It also contains a \isi{demonstrative pronoun} and an adjunct in the \isi{ergative} that serves as \isi{direct object} because the nominalized clause is an \isi{antipassive} construction (\refsec{sec:Antipassive}).
%
\begin{exe}
	\ex	\label{ex:There were the drinking ones; these (are) the working ones}
	\gll	itːu-b	b-učː-an-te=de;	iš-tːi	ʡaˁči-l	b-irq'-an-te ...\\
		there\tsc{-hpl}	\tsc{hpl-}drink\tsc{.ipfv-ptcp-dd.pl=pst}	this\tsc{-pl}	work\tsc{-erg}	\tsc{hpl-}do\tsc{.ipfv-ptcp-dd.pl}\\
	\glt	\sqt{There were the drinking ones; these working ones ...}
\end{exe}


%%%%%%%%%%%%%%%%%%%%%%%%%%%%%%%%%%%%%%%%%%%%%%%%%%%%%%%%%%%%%%%%%%%%%%%%%%%%%%%%

\section{Forms based on the preterite}
\label{sec:Forms based on the preterite}

The preterite is the most important verbal suffix in Sanzhi not just because it is extremely common in terms of token frequency and used as the base for a wide range of TAM forms (\reftab{tab:Forms based on the preterite}), but also because it is the major indicator for verbal inflection classes.

Sanzhi has the standard Dargwa inventory of preterite suffixes: \tit{-ib, -ub, -un,} and \tit{-ur}. The suffix \tit{-ib} is the most frequently used preterite suffix and thus sometimes treated as the default variant (e.g. \citeb{Daniel2015}), and \tit{-ub} is analyzed as a phonologically predictable allomorph that is in complementary distribution with \tit{-ib} and occurs only with labialized stems \citep{BelyaevInPreparation}. The latter is, in fact, the case in Sanzhi, but the complementary distribution still needs further investigation (see \refsec{sec:The structure underived verbal stems} for lists of verbs and their preterite suffixes). The suffix \tit{-un} is the second most frequent suffix after \tit{-ib}. It occurs with verbs that have \tit{l} in the perfective stem and/or an imperfective stem with initial \tit{l}, labialized root \isi{consonants}, or \tit{l} in the imperfective stem. The suffix \tit{-ur} is the least frequent one (though probably more common than \tit{-ub}), and its occurrence cannot be predicted.

In principle, many verbs can inflect the imperfective as well as the perfective stem for the preterite, but not all verbs have this possibility (e.g. \tit{b-alχ-} \sqt{\tsc{n-}know\tsc{.ipfv}} cannot take any of the preterite suffixes). There are only very few corpus examples of imperfective verb stems bearing the preterite suffix, all occurring in the preterite. All other forms can be elicited, but speakers do not seem to have clear intuitions about the meanings and context of use of these forms and translations suggest that the forms are not truly part of verbal paradigms. For instance, \isi{experiential} forms are normally translated with \isi{cleft} constructions, suggesting a biclausal structure.  Therefore, almost all verb forms based on the imperfective stem have been given in parenthesis in \reftab{tab:Forms based on the preterite} and \reftab{tab:Exemplary paradigms based on the preterite for the verb do make}. They will not be further  discussed here.

\reftab{tab:Forms based on the preterite} displays the verb forms that can be obtained from the preterite and \reftab{tab:Exemplary paradigms based on the preterite for the verb do make} provides one exemplary verb. The paradigm shows a clear symmetrical structure. All verb forms make use of one of the three available means for obtaining regular \is{analytic verb form}analytic verb forms: person enclitics, the \isi{copula} \tit{ca-b} or the past \isi{enclitic}. The preterite is the base form and the preterite \isi{resultative} is a kind of minor variant derived from it. From the preterite, three types of verb forms are built. Each of them follows the same pattern: the lexical verb occurs in the preterite form and takes one of three further suffixes. These suffixes are the \isi{perfective converb} suffix, or one of the two \is{cross-categorical suffix}cross-categorical suffixes \textit{-ce} and \tit{-il}. The suffixes are followed by person enclitics (first and second person) or the \isi{copula} (third person), which yields one type of forms. The temporal reference of this type of verb forms can be further shifted to the past by means of the past \isi{enclitic} =\textit{de}. 

At least diachronically all three suffixes used after the preterite belong to the same class of \is{cross-categorical suffix}cross-categorical suffixes that are added to words of different lexical classes and form either referential attributes with the syntactic properties of \isi{nouns} (\textit{-ce} and \tit{-il}) or adverbials (\refsec{ssec:The adverbializer -le}). Thus, the \isi{perfective converb} suffix is identical with the \isi{adverbializer} -\textit{le}. But for the sake of clarity and readability of the grammar I will gloss it as \isi{perfective converb} suffix and treat it as a separate item (the same was done for the \isi{imperfective converb}). The syntactic properties of the \is{cross-categorical suffix}cross-categorical suffixes are preserved in the verb forms containing them: the \isi{perfective converb} is functionally equivalent to adverbials when it is used without the person enclitics, past \isi{enclitic} or \isi{copula}; the verb forms with the suffixes -\textit{ce} and -\textit{il} (\isi{experiential} and \isi{experiential past} forms) are functionally equivalent to nominalized \is{participle}participles that form \is{relative clause}relative clauses.


%
\begin{table}
	\caption{Forms based on the preterite}
	\label{tab:Forms based on the preterite}
	\small
	\begin{tabularx}{0.98\textwidth}[]{%
		>{\raggedright\arraybackslash}p{80pt}
		>{\raggedright\arraybackslash}X
		>{\raggedright\arraybackslash}X}
		
		\lsptoprule
			{}				&	imperfective stem					&	perfective stem\\
		\midrule
			~~preterite			&	preterite + person 				&	preterite + person\\
			{}				&	\hspace*{9pt}{\isit{enclitic}\slash zero}			&	\hspace*{9pt}{\isit{enclitic}\slash zero}\\
			~~\isit{resultative} 		&	preterite + \isit{copula}				&	preterite + \isit{copula}\\
			~~~(only 3$^{rd}$ person)\\\midrule
			\multicolumn{3}{l}{{preterite + \isi{perfective converb}  \tit{-le} + X}}\\\midrule
			~~perfect 			&	(preterite + converb \tit{-le}				&	preterite + converb \tit{-le}\\
			{}				&	\hspace*{2.98pt}~+ person enclitics\slash \isit{copula})	&	~+ person enclitics\slash \isit{copula}\\
			~~past perfect		&	(preterite + converb \tit{-le}			&	preterite + converb \tit{-le}\\
			~~~(\isit{pluperfect})		&	\hspace*{2.98pt}~+ past \isit{enclitic} \textit{=de})			&	~+ past \isit{enclitic} \textit{=de}\\\midrule
			\multicolumn{3}{l}{{preterite + cross-categorical suffix \tit{-ce/-te} + X}}\\\midrule
			~~\isit{experiential} I		&	(preterite + \tit{-ce/-te}			&	preterite + \tit{-ce/-te}\\
			{}				&	\hspace*{2.98pt}~+ person enclitics\slash \isit{copula})	&	~+ person enclitics\slash \isit{copula}\\
			~~\isit{experiential past} I	&	(preterite + \tit{-ce/-te} + past)		&	preterite + \tit{-ce/-te} + past\\\midrule
			\multicolumn{3}{l}{{preterite + cross-categorical suffix \tit{-il} + X}}\\\midrule
			~~\isit{experiential} II		&	(preterite + \tit{-il}				&	preterite + \tit{-il}\\
			{}				&	\hspace*{2.98pt}~+ person enclitics\slash \isit{copula})	&	~+ person enclitics\slash \isit{copula}\\
			~~\isit{experiential}~past~II	&	(preterite + \tit{-il} + past)			&	preterite + \tit{-il} + past\\
		\lspbottomrule
	\end{tabularx}
\end{table}

\begin{table}
	\caption{Exemplary paradigms based on the preterite for the verb \sqt{do, make}}
	\label{tab:Exemplary paradigms based on the preterite for the verb do make}
	\small
	\begin{tabularx}{0.98\textwidth}[]{%
		>{\raggedright\arraybackslash}p{80pt}
		>{\raggedright\arraybackslash\itshape}X
		>{\raggedright\arraybackslash\itshape}X}
		
		\lsptoprule
			{}				&	\upshape imperfective stem			&	\upshape perfective stem\\
		\midrule
			~~preterite			&	b-irq'-ib\slash =da\slash =de 				&	b-arq'-ib\slash =da\slash =de \\
			~~\isit{resultative} 		&	b-irq'-ib ca-b						&	b-arq'-ib ca-b\\
			~~~(only 3$^{rd}$ person)\\\midrule
			\multicolumn{3}{l}{{preterite + \isi{perfective converb}  \tit{-le} + X}}\\\midrule
			~~perfect 			&	(b-irq'-ib-le=da\slash =de\slash  ca-b)			&	b-arq'-ib-le=da\slash =de\slash  ca-b\\
			~~past perfect		&	(b-irq'-ib-le=de)					&	b-arq'-ib-le=de\\
			~~~(\isi{pluperfect})\\\midrule
			\multicolumn{3}{l}{{preterite + cross-categorical suffix \tit{-ce/-te} + X}}\\\midrule
			~~\isit{experiential} I		&	(b-irq'-ib-ce=da\slash =de\slash  ca-b)			&	b-arq'-ib-ce=da\slash =de\slash  ca-b\\
			~~\isit{experiential past} I	&	(b-irq'-ib-ce=de)					&	b-arq'-ib-ce=de\\\midrule
			\multicolumn{3}{l}{{preterite + cross-categorical suffix \tit{-il} + X}}\\\midrule
			~~\isit{experiential} II		&	(b-irq'-ib-il=da\slash =de\slash  ca-b)			&	b-arq'-ib-il=da\slash =de\slash  ca-b\\
			~~\isit{experiential}~past~II	&	(b-irq'-ib-il=de)					&	b-arq'-ib-il=de\\
		\lspbottomrule
	\end{tabularx}
\end{table}


% --------------------------------------------------------------------------------------------------------------------------------------------------------------------------------------------------------------------- %

\subsection{The imperfective preterite and imperfective preterite resultative}
\label{ssec:The imperfective preterite and imperfective preterite resultative}

The imperfective preterite is formed from the preterite stem of imperfective verbs to which person enclitics for first and second persons are added; for the third person no markers are used. In addition to the \isi{ergative} and the affective construction, the imperfect also allows for the \isi{antipassive} construction \refex{ex:We (repeatedly) worked in the garden}. The imperfective preterite expresses past time reference in combination with \isi{imperfective aspect}. It can be negated by means of the prefix \tit{a-}. The imperfective preterite is barely attested in the Sanzhi corpus \refex{ex:Although I said words like a bird, you did not understand them}; examples \xxref{ex:I saw you (repeatedly) from the window of the hospital}{ex:We (repeatedly) worked in the garden} have been elicited. The imperfective preterite \isi{resultative}, which is restricted to third \isi{person agreement} controllers, has been obtained only through elicitation \refex{ex:My grandfather (apparently) built houses analytic}.
%
\begin{exe}
	\ex	\label{ex:I saw you (repeatedly) from the window of the hospital}
	\gll	dam	u	balnic’a-la	ʡaˁme-r	či-w-iž-ib=da\\
		\tsc{1sg.dat}	\tsc{2sg}	hospital\tsc{-gen}	window\tsc{.loc-abl}	\tsc{spr-m-}see\tsc{.ipfv-pret=1}\\
	\glt	\sqt{I saw you (repeatedly) from the window of the hospital.} (E)

	\ex	\label{ex:S/he sold apples analytic}
	\gll	hinc-be	d-irc-ib	it-i-l\\
		apple\tsc{-pl}	\tsc{npl-}sell\tsc{.ipfv-pret}	that\tsc{-obl-erg}\\
	\glt	\sqt{S/he traded with apples.} OR \sqt{S/he sold apples.} (E)

	\ex	\label{ex:We (repeatedly) worked in the garden}
	\gll	agarad-m-a-ja-d	ʡaˁči-l	d-irq'-ib=da	nušːa\\
		garden\tsc{-pl-obl-loc-1/2.pl}	work\tsc{-erg}	\tsc{1/2.pl-}do\tsc{.ipfv-pret=1}	\tsc{1pl}\\
	\glt	\sqt{We (repeatedly) worked in the garden.} (E)

	\ex	\label{ex:Although I said words like a bird, you did not understand them}
	\gll	čaˁkʷa-la	ʁunab-te	ʁaj	d-urs-ib=xːar,	a-jrʁ-ib=de	at\\
		bird\tsc{-gen}	\tsc{eq-dd.pl} 	word	\tsc{npl}-tell.\tsc{pfv-pret=conc}	\tsc{neg}-understand.\tsc{ipfv-pret=2sg}	\tsc{2sg.dat}\\
	\glt	\sqt{Although I said words like a bird, you did not understand them.} (modified corpus example)
	
	\ex	\label{ex:My grandfather (apparently) built houses analytic}
	\gll	di-la	χatːaj-li	qul-be	d-irq'-ib	ca-d\\
		\tsc{1sg-gen}	grandfather\tsc{-erg}	house\tsc{-pl}	\tsc{npl-}do\tsc{.ipfv-pret}	\tsc{cop-npl}\\
	\glt	\sqt{My grandfather (apparently) built houses.} (E)

\end{exe}

Examples \refex{ex:Uh, first he drank here analytic} and \refex{ex:He ate until he was full analytic} are from the corpus and show \isi{antipassive} constructions. In \refex{ex:Uh, first he drank here analytic}, the demoted \isi{agent} is expressed (clause-final pronoun). The demoted patients, which would have been in the \isi{ergative} case, are left unexpressed in both examples. See \refsec{sec:Antipassive} for \isi{antipassive} constructions.
%
\begin{exe}
	\ex	\label{ex:Uh, first he drank here analytic}
	\gll	ha,	bahsar-ka	heštːu-w	učː-ib	ca-w	iž\\
		uh	first\tsc{-abl}	here\tsc{-m}	 drink\tsc{.ipfv.m-pret}	\tsc{cop-m}	this\\
	\glt	\sqt{Uh, first he drank here.}

	\ex	\label{ex:He ate until he was full analytic}
	\gll	w-elqː-ij=sat	uk-un	ca-w\\
		\tsc{m-}sate\tsc{.pfv-inf=}until	eat\tsc{.ipfv.m-pret}	\tsc{cop-m}\\
	\glt	\sqt{He ate until he was full.} OR \sqt{He ate until he is full.}
\end{exe}


% --------------------------------------------------------------------------------------------------------------------------------------------------------------------------------------------------------------------- %

\subsection{The preterite}
\label{ssec:The preterite}

The preterite is the default \isi{past tense} with respect to form and function. It is formed from the perfective stem by adding the preterite suffix and for first and second persons the person enclitics; the third person does not have additional marking. It conveys past time reference and is very frequent in the Sanzhi corpus, especially in autobiographical narratives \refex{ex:Three years I remained, I was not one single hour at the guardhouse} and in daily conversations when speakers report about past events \refex{ex:We did not gather plants in Sanzhi}. However, it can also occur in traditional narratives \refex{ex:I helped. I pulled him out}, \refex{ex:Once upon a time the sun and the evil wind argued analytic} and in other narratives about the past that are not related to the personal experience of the speaker \refex{ex:Nobody found out what colour this is}.
%
\begin{exe}

	\ex	\label{ex:Three years I remained, I was not one single hour at the guardhouse}
	\gll	ʡaˁbal	dus	kelg-un=da,		du	gaupaχt-le	a-ka-jč-ib=da	ca	sːaˁʡaˁt\\
		three	year	remain\tsc{.pfv-pret=1}		\tsc{1sg}	guardhouse\tsc{-loc}	\tsc{neg-down}-occur\tsc{.pfv-pret=1}	one	hour\\
	\glt	\sqt{Three years I remained, I was not one single hour at the guardhouse.}

\ex	\label{ex:I helped. I pulled him out}
	\gll	du-l	kumek	b-arq'-ib=da,	tːura-h-aqː-ib=da\\
		\tsc{1sg-erg}	help	\tsc{n-}do\tsc{.pfv-pret=1}	\tsc{out-up}-take.out\tsc{.pfv-pret=1}\\
	\glt	\sqt{I helped. I pulled him out.}
	
	\ex	\label{ex:Once upon a time the sun and the evil wind argued analytic}
	\gll	ca	zamana	bari=ra	wahi-ce	č'an=ra	čːal	d-uq-un\\
		one	time	sun\tsc{=add}	evil\tsc{-dd}	wind\tsc{=add}	argument	\tsc{npl-}go\tsc{.pfv-pret}\\
	\glt	\sqt{Once upon a time the sun and the evil wind argued.}
\end{exe}

Negation is expressed through the prefix \tit{a-}:
%
\begin{exe}
	\ex	\label{ex:We did not gather plants in Sanzhi}
	\gll	q'ar	Sːanži-d	a-d-ertː-ib=da\\
		plant	Sanzhi\tsc{-npl}	\tsc{neg-npl-}take\tsc{.pfv-pret=1}\\
	\glt	\sqt{We did not gather plants in Sanzhi.}

	\ex	\label{ex:Nobody found out what colour this is}
	\gll	ce	kraska=de-l=ra	a-b-aχ-ur\\
		what	color\tsc{=pst=indq=add}	\tsc{neg-n-}know\tsc{.pfv-pret}\\
	\glt	\sqt{Nobody found out what color this is.}
\end{exe}


% --------------------------------------------------------------------------------------------------------------------------------------------------------------------------------------------------------------------- %

\subsection{The (perfective) resultative}
\label{ssec:The (perfective) resultative}

The perfective \isi{resultative} consists of the preterite and the \isi{copula}. This verb form cannot be used with the first or second \isi{person agreement} controllers. The presence of the \isi{copula} conveys perfectivity/resultativity, i.e. the focus is on the result of a situation \xxref{ex:(The color) has remained inside}{ex:‎‎‎It was at night, there was no fire visible, nobody is there, then he turned and came back}. This form is usually not used in personal narratives, but it is very frequent in other texts such as traditional narratives and other third-person perspective narrations. Negation is expressed through the prefix \tit{a-} \refex{ex:‎‎‎It was at night, there was no fire visible, nobody is there, then he turned and came back}.
%
\begin{exe}
	\ex	\label{ex:(The color) has remained inside}
	\gll	b-ark-le	b-i-b	kelg-un	ca-b\\
		\tsc{n-}inside\tsc{-loc}	\tsc{n-}in\tsc{-n}	remain\tsc{.pfv-pret}	\tsc{cop-n}\\
	\glt	\sqt{(The color) has remained inside.}

	\ex	\label{ex:He has also taken up his hands}
	\gll	nuˁq-be	aq	d-arq'-ib	ca-d	ik'-i-l=ra\\
		arm\tsc{-pl}	high	\tsc{npl-}do\tsc{.pfv-pret}	\tsc{cop-npl}	\tsc{dem.up}\tsc{-obl-erg=add}\\
	\glt	\sqt{He has also raised his hands.}
	
	\ex	\label{ex:‎‎‎It was at night, there was no fire visible, nobody is there, then he turned and came back}
	\gll	dučːilla=q'al	il	ja	c'a	či-b-ig-an	b-a-b-už-ib	ca-b,		ja	insan	w-akːu,		c'il	čar	Ø-iχ-ub-le	ag-ur	ca-w	il\\
		at.night\tsc{=mod}	that	or	fire	\tsc{spr-n-}see\tsc{.ipfv-ptcp}	\tsc{n-neg-n-}be\tsc{-pret}	\tsc{cop-n}		or	person	\tsc{m-}\tsc{cop.neg}	then	back	\tsc{m-}be\tsc{.pfv-pret-cvb}	go\tsc{.pfv-pret}	\tsc{cop-m}	that\\
	\glt	\sqt{‎‎‎It was at night, there was no fire visible, nobody is there, then he turned and came back.}
\end{exe}

Sentence \refex{ex:Now I have forgotten those places analytic} is a corpus example with a first person pronoun in the \isi{dative}. The predicate in this example is an \isi{affective verb} which requires a \isi{dative} \isi{experiencer} and an \isi{absolutive} \isi{stimulus}. In contrast to almost all other bivalent \is{affective verb}affective verbs the \isi{experiencer} does not obligatorily control \isi{person agreement} on the predicate, but the predicate can be used with the \isi{copula} (i.e. third person). See \refsec{sec:Bivalent affective verbs} for more information

\begin{exe}
	\ex	\label{ex:Now I have forgotten those places analytic}
	\gll	qum.ert-ur ca-d	na	hetːi	mus-ne	dam\\
		forget\tsc{.pfv-pret} \tsc{cop-npl}	now	those	place\tsc{-pl}	\tsc{1sg.dat}\\
	\glt	\sqt{Now I have forgotten those places.}
\end{exe}


The focus\is{focus} that the perfective \isi{resultative} puts on the resulting state can lead to an inferential interpretation that becomes particularly obvious to speakers when they are asked to compare the preterite to the perfective \isi{resultative}. For example, the following sentence could be uttered in a situation in which Sanzhiat must wash the dishes, she goes to the kitchen and sees that somebody has already washed the dishes \refex{ex:The dishes have already been washed}. This means that she concludes from the result that someone must have washed them.
%
\begin{exe}
	\ex	\label{ex:The dishes have already been washed}
	\gll	uže	t'alaˁħ-ne	d-irc-ib	ca-d\\
		already	dishes\tsc{-pl}	\tsc{npl-}wash\tsc{.pfv-pret}	\tsc{cop-npl}\\
	\glt	\sqt{The dishes have already been washed.} (E)
\end{exe}

If she then asks \tit{hil dircibe?} \sqt{Who washed (them)?}, an appropriate answer of somebody who attended the event could be \refex{ex:Sanzhiat washed (them)}, that is, now the \isi{agent} is at stake, not the result of the action.
%
\begin{exe}
	\ex	\label{ex:Sanzhiat washed (them)}
	\gll	Sanžijat-li	d-irc-ib\\
		Sanzhiat\tsc{-erg}	\tsc{npl-}wash\tsc{.pfv-pret}\\
	\glt	\sqt{Sanzhiat washed (them).} (E)
\end{exe}

Similarly, when looking out of the window the speaker sees a wet road and concludes from this \refex{ex:It has rained analytic}.
%
\begin{exe}
	\ex	\label{ex:It has rained analytic}
	\gll	marka-l	b-us-ib	ca-b\\
		rain\tsc{-erg}	\tsc{n-}rain\tsc{-pret}	\tsc{cop-n}\\
	\glt	\sqt{It has rained.} (E)
\end{exe}

However, the inferential interpretation can be canceled by a following utterance without leading to a special interpretation \refex{ex:Sanijat has washed the dishes. I saw it myself}.
%
\begin{exe}
	\ex	\label{ex:Sanijat has washed the dishes. I saw it myself}
	\gll	Sanijat-li 	t'alaˁħ-ne	d-irc-ib ca-d.	dam=q'ar	il	či-b-až-ib=da\\
		Sanijat\tsc{-erg}	dishes\tsc{-pl}	\tsc{npl-}wash\tsc{.pfv-pret} \tsc{cop-npl}	\tsc{1sg.dat=prt}	that	\tsc{spr-n-}see\tsc{.pfv-pret=1}\\
	\glt	\sqt{Sanijat has washed the dishes. I saw it myself.} (E)
\end{exe}




% --------------------------------------------------------------------------------------------------------------------------------------------------------------------------------------------------------------------- %

\subsection{The perfect}
\label{ssec:The perfect}

The perfect is formed by adding the \isi{perfective converb} suffix to the preterite, followed by the person enclitics for first and second person and the \isi{copula} \textit{ca-b} for third person. When the \isi{perfective converb} is suffixed to the preterite regular \isi{assimilation} processes take place after the suffixes that end in a sonorant, such that the following allopmorphs result: \tit{-ib-le, -ub-le, -ur-le\slash -ur-re, -un-ne}. Though it can be elicited with imperfective stems, there are no such instances in the corpus and I will therefore restrict myself to the discussion of perfect forms built with perfective stems.

The perfect is not particularly frequent in narratives, but there are enough examples to describe its meaning. Its semantic range primarily covers resulting states; it mostly occurs with verbs such as \sqt{sit}, \sqt{lay down}, \sqt{die}, \sqt{get/become hungry}, etc. that denote a change of state and the perfect expresses the resulting state:
%
\begin{exe}
	\ex	\label{ex:I have been lying (in the hospital) for three days analytic}
	\gll	ka-r-isː-un-ne=da	na	ʡaˁbal	bari\\
		\tsc{down-f-}lay\tsc{.pfv-pret-cvb=1}	now	three	day\\
	\glt	\sqt{I have been lying (in the hospital) for three days.}

	\ex	\label{ex:There is nobody at home, my mother has died analytic}
	\gll	qili-w	ča-k'al	w-aːkːu,	aba	r-ebč'-ib-le	ca-r\\
		home\tsc{-m}	who\tsc{-indef}	\tsc{m-}\tsc{cop.neg}	mother	\tsc{f-}die\tsc{.pfv-pret-cvb}	\tsc{cop-f}\\
	\glt	\sqt{There is nobody at home, my mother has died.}

	\ex	\label{ex:I was born at the time of the prophet Noah}
	\gll	Naħ	idbag-la	zamana	hak'-ub-le=da	du\\
		Noah	prophet\tsc{-gen}	time	appear\tsc{.pfv-pret-cvb=1}	\tsc{1sg}\\
	\glt	\sqt{I was born at the time of the prophet Noah.}
\end{exe}

This includes \is{transitive verb}transitive verbs of which the \isi{agent} is then often omitted because the focus\is{focus} is on the resulting state \refex{ex:Her head has been wounded}.
%
\begin{exe}
	\ex	\label{ex:Her head has been wounded}
	\gll	ik'-i-la	bek'	b-aˁq-ib-le	ca-b	hek'\\
		\tsc{dem.up}\tsc{-obl-gen}	head	\tsc{n-}wound\tsc{.pfv-pret-cvb}	\tsc{cop-n}	\tsc{dem.up}\\
	\glt	\sqt{Her head has been wounded.}
\end{exe}

The following example illustrates one of the traditional greetings for women, used by men and women when the female addressee is seated, for example in front of the house, and the speaker is passing by \refex{ex:Are you sitting (seated) analytic}. Example \refex{ex:I am sitting sat down analytic} shows a minimal pair illustrating the difference between the preterite and the perfect that formally differ only in the absence vs. presence of the \isi{perfective converb}. The preterite conveys past time reference with verbs that express changes of state whereas the perfect refers to the state that obtains at the present moment. \refex{ex:I am sitting analytic} is the standard answer to \refex{ex:Are you sitting (seated) analytic}.
%
\begin{exe}
	\ex	\label{ex:Are you sitting (seated) analytic}
	\gll	ka-r-iž-ib-le=de=w?\\
		\tsc{down-f-}be\tsc{.pfv-pret-cvb=2sg=q}\\
	\glt	\sqt{Are you sitting (seated)?}

	\ex	\label{ex:I am sitting sat down analytic}
	\begin{xlist}
		\ex	\label{ex:I sat down analytic}
		\gll	ka-r-iž-ib=da\\
			\tsc{down-f-}be\tsc{.pfv-pret=1}\\
		\glt	\sqt{I sat down.} (E)

		\ex	\label{ex:I am sitting analytic}
		\gll	ka-r-iž-ib-le=da\\
			\tsc{down-f-}be\tsc{.pfv-pret-cvb=1}\\
		\glt	\sqt{I am sitting.}
	\end{xlist}
\end{exe}

As can be seen in \refex{ex:Well, approximately we already said it analytic}, the \isi{agent} can be overtly expressed and the \isi{ergative} construction is allowed when the perfect is used in Sanzhi, in contrast to the closely related Icari Dargwa variety, which prohibits the perfect with overtly expressed agents inflected for the \isi{ergative} case.
%
\begin{exe}
	\ex	\label{ex:Well, approximately we already said it analytic}
	\gll	nu	hel	priblizitelno	nušːa-l	b-urs-ib-le=da=q'al\\
		well	that	approximately	\tsc{1pl-erg}	\tsc{n-}tell\tsc{-pret-cvb=1=mod}\\
	\glt	\sqt{Well, approximately we already said it.}
	\end{exe}
	
And in contrast to other Dargwa varieties such as Shiri \citep{BelyaevInPreparation}, the Sanzhi perfect can also be used with verbs that do not imply a change of state in the \isi{agent} \refex{ex:‎‎‎(Apparently,) they have gone to the cave of the bear} even though normally the preterite is preferred in such contexts. 

\begin{exe}
	\ex	\label{ex:‎‎‎(Apparently,) they have gone to the cave of the bear}
	\gll	ag-ur-re	ca-b	sːika-la	mergʷ-li-šːu\\
		go\tsc{.pfv-pret-cvb}	\tsc{cop-hpl}	bear\tsc{-gen}	lair\tsc{-obl-ad}\\
	\glt	\sqt{‎‎‎They have gone to the cave of the bear.}
\end{exe}

In the right context, the perfect can imply inferentiality\slash indirect evidentiality similar to the perfective \isi{resultative}. Example \refex{ex:‎(Apparently,) they have not allowed the village to grow} and originates from a narrative about the history of the village of Sanzhi, and the speaker draws a conclusion about the present situation of the village based on past events that he did not witness himself. Similarly, \refex{ex:‎(Apparently,) when he came to Gudermets he did not know that they had been thrown out (of the village)} and \refex{ex:‎‎(Apparently,) he also has not touched the food, he has not eaten} are inferences about past events that the speakers draw from observed results.


%
\begin{exe}
	\ex	\label{ex:‎(Apparently,) they have not allowed the village to grow}
	\gll	heχ	šːi	imc'a	b-iχʷ-ij	b-at-ur-re=kːu	hel-tː-a-li\\
		\tsc{dem.down}	village	additional	\tsc{n-}be\tsc{.pfv-inf}	\tsc{n-}let\tsc{.pfv-pret-cvb=neg}	that\tsc{-pl-obl-erg}\\
	\glt	\sqt{(They) have not allowed the village to grow.}

	\ex	\label{ex:‎(Apparently,) when he came to Gudermets he did not know that they had been thrown out (of the village)}
	\gll	sa-jʁ-ib-le	Gudermec-le,		ix-tːi	tːura	aʁ-ib-il	b-aχ-ur-re=kːu\\
		\tsc{hither}-come\tsc{.pfv.m-pret-cvb}	Gudermets\tsc{-loc}	\tsc{dem.up-pl}	outside	do\tsc{.pfv-pret-ref}	\tsc{n-}know\tsc{.pfv-pret-cvb=neg}\\
	\glt	\sqt{When he came to Gudermets he did not know that they had been thrown out (of the village).}

	\ex	\label{ex:‎‎(Apparently,) he also has not touched the food, he has not eaten}
	\gll	berkʷijce-li-j=ra	qːuc	Ø-ič-ib-le=kːu		b-erk-un-ne=kːu\\
		food\tsc{-obl-dat=add}	touch	\tsc{m-}occur\tsc{.pfv-pret-cvb=neg}	\tsc{n-}eat\tsc{.pfv-pret-cvb=neg}\\
	\glt	\sqt{‎‎(He also has not touched the food, he has not eaten.}
\end{exe}

Negation of verb forms with first and second \isi{person agreement} controllers is expressed by means of the prefix \tit{a-}, but there are no corpus examples. For the third person the negative \isi{copula} \tit{akːu} occurred in its shortened form as an \isi{enclitic} to the verb \xxref{ex:‎(Apparently,) they have not allowed the village to grow}{ex:‎‎(Apparently,) he also has not touched the food, he has not eaten}. 


% --------------------------------------------------------------------------------------------------------------------------------------------------------------------------------------------------------------------- %

\subsection{The past perfect (pluperfect)}
\label{ssec:The past perfect (pluperfect)}

The past perfect is formed by attaching the past \isi{enclitic} \tit{=de} to the \isi{perfective converb}. In elicitation, the past perfect is available for perfective and imperfective stems, but there are no corpus examples of the latter. In addition, there is a variant of the past perfect that makes use of the locative copulas to which \tit{=de} is encliticized (see \refsec{sec:Verb forms with locational copulae}). 

The past perfect has the typical \isi{pluperfect} meaning and also past \isi{resultative} meaning. It refers to an event (or the \isi{resultative} state of an event) that occurred before a definite point in past time. In \refex{ex:‎‎The father came down and stood next to the ladder analytic} the preceding event is mentioned in the first clause of the utterance. In the other examples \refex{ex:Like this (he) had caught (me) analytic} and \refex{ex:‎‎‎We had already given twenty analytic} the reference point in the past was mentioned in the preceding context.
%
\begin{exe}
	\ex	\label{ex:‎‎The father came down and stood next to the ladder analytic}
	\gll	atːa	či-r-ka-w-q-un-ne	na	kːancːupːa-la	šːule	ka-jcː-ur-re=de\\
		father	\tsc{spr-abl-down}\tsc{-m-}go\tsc{.pfv-pret-cvb}	already	ladder\tsc{-gen}	at.side	\tsc{down}-get.up\tsc{.pfv-pret-cvb=pst}\\
	\glt	\sqt{After the father came down he stood next to the ladder.}

	\ex	\label{ex:Like this (he) had caught (me) analytic}
	\gll	heχ-itːe	Ø-uc-ib-le=de\\
		\tsc{dem.down}\tsc{-advz}	\tsc{m-}catch\tsc{.pfv-pret-cvb=pst}\\
	\glt	\sqt{Like this (he) had caught (me).}

	\ex	\label{ex:‎‎‎We had already given twenty analytic}
	\gll	ʁajal	b-ičː-ib-le=de	nušːa-l	uže\\
		twenty	\tsc{n-}give\tsc{.pfv-pret-cvb=pst}	\tsc{1pl-erg}	already\\
	\glt	\sqt{‎‎‎We had already given twenty.}
\end{exe}

Negation is expressed by means of the negative prefix \tit{a-} \refex{ex:‎One boy had turned three, the other was not even one year old analytic} or the negative \isi{past tense} \isi{copula} \tit{akːʷi} \refex{ex:‎ ‎‎We had not given twenty analytic}.

\begin{exe}
	\ex	\label{ex:‎One boy had turned three, the other was not even one year old analytic}
	\gll	ca	ʡaˁbal	dus	w-iχ-ub-le=de	ca	dus	taman	a-jχ-ub-le=de	durħuˁ\\
		one	three	year	\tsc{m-}become\tsc{.pfv-pret-cvb=pst}		one	year	end	\tsc{neg-}become\tsc{.m.pfv-pret-cvb=pst}	boy\\
	\glt	\sqt{‎One boy had turned three, the other was not even one year old.}

	\ex	\label{ex:‎ ‎‎We had not given twenty analytic}
	\gll	ʁajal	b-ičː-ib-le=de=kːʷ-adi	nušːa-l\\
		twenty	\tsc{n-}give\tsc{.pfv-pret-cvb=pst}=\tsc{cop.neg-hab.pst1}	\tsc{1pl-erg}\\
	\glt	\sqt{‎‎‎We had not given twenty.} (E)
\end{exe}

The past perfect also expresses inferentiality. This means that the speaker concludes from an observed result that an event has taken place. Thus, \refex{ex:(He) had (apparently) written Tawlu Zhandaruvich} was uttered in a situation when the speaker found out only afterwards when reading the article that the journalist to whom he had talked had written a wrong name. Example \refex{ex:‎They went to Shara and had grabbed their houses} is from a narrative about past events that were not witnessed by the speaker himself (namely the grabbing that happened at night). But he inferred from the result and from his knowledge of the general circumstances that the people he is talking about in \refex{ex:‎They went to Shara and had grabbed their houses} were the robbers.
%
\begin{exe}
	\ex	\label{ex:(He) had (apparently) written Tawlu Zhandaruvich}
	\gll	Tawlu	žaˁndaruwič	b-elk'-un-ne=de\\
		Tawlu	Zhandaruvich	\tsc{n-}write\tsc{.pfv-pret-cvb=pst}\\
	\glt	\sqt{(He) had (apparently) written Tawlu Zhandaruvich.}

	\ex	\label{ex:‎They went to Shara and had grabbed their houses}
	\gll	šara	ag-ur	itːa-la	qul-be	qaˁm	d-arq'-ib-le=de\\
		S.	go\tsc{.pfv-pret}	those.\tsc{obl-gen}	house\tsc{-pl}	grab	\tsc{npl-}do\tsc{.pfv-pret-cvb=pst}\\
	\glt	\sqt{‎They went to Shara and had grabbed their houses.}
\end{exe}

The past \isi{resultative} meaning can co-occur with the inferential meaning. For instance, in \refex{ex:Before/until I came (home) my mother had already died} the speaker refers to a state (= the death of his mother) that was obtained before another moment in the past (= his return to the village). At the same time the speaker was not present at the relevant event (= the dying of his mother) such that there is an inferential component.
%
\begin{exe}
	\ex	\label{ex:Before/until I came (home) my mother had already died}
	\gll	du	sa-jʁ-ij=satːina 	r-ebč'-ib-le=de	aba\\
		\tsc{1sg}	\tsc{hither}-come\tsc{.pfv-inf=}until	\tsc{f-}die\tsc{.pfv-pret-cvb=pst}	mother\\
	\glt	\sqt{Before I came (home) my mother had already died.}
\end{exe}

When speakers are presented with past perfect sentences out of context that contain predicates that do not denote a change of state the inferential meaning is salient and therefore there is a first-person effect with core arguments that denote first persons. This means that \refex{ex:I saw Arsen. (as it seems, and e.g. I did not recognize him)} can only be uttered if the referent of the first person pronoun did not consciously participate in the situation and therefore did really see Arsen because he did not recognize him.
%
\begin{exe}
	\ex	\label{ex:I saw Arsen. (as it seems, and e.g. I did not recognize him)}
	\gll	dam	Arsen	či-w-až-ib-le=de\\
		\tsc{1sg.dat}	Arsen	\tsc{spr-m-}see\tsc{.pfv-pret-cvb=pst}\\
	\glt	\sqt{I (apparently) saw Arsen.} (as it seems, e.g. I did not recognize him) (E)
\end{exe}


% --------------------------------------------------------------------------------------------------------------------------------------------------------------------------------------------------------------------- %

\subsection{Experiential I and experiential II}
\label{ssec:Experiential I and experiential II}

There are two variants of the \isi{experiential}. They both involve the preterite to which the \is{cross-categorical suffix}cross-categorical suffixes are added (-\textit{ce} and -\textit{il}). The \is{cross-categorical suffix}cross-categorical suffixes are generally used to form referential attributes from various parts of speech, including verbs (\refsec{ssec:The -ce / -te attributive} and \refsec{ssec:The -il attributive}). The resulting word forms largely have the syntactic properties of nominals (e.g. they can be inflected for case, they can take over argument positions, etc.), and this leads to very particular syntactic properties of all experiental and experiental past forms that are discussed below.

The first variant, the \isi{experiential} I, is obtained by suffixing \tit{-ce} (plural \tit{-te}) to the \isi{preterite participle}, followed by the person enclitics or the \isi{copula} \tit{ca-b}. The second variant, the \isi{experiential} II, is formed by adding the suffix \tit{-il} to the preterite, again followed by the person enclitics or the \isi{copula}. The use of the two different suffixes -\textit{ce} and -\textit{il} does not lead to any semantic differences with respect to the \isi{experiential} verb forms. Their distribution rather depends on \isi{number} (this was already explained for the obligative verb forms in \refsec{ssec:Obligative}). For argument controllers in the singular -\textit{il} is almost exclusively used, although -\textit{ce} is also grammatical. For argument controllers in the plural only -\textit{te} is allowed. The \isi{experiential} can also be formed with the locative copulas (see \refsec{sec:Verb forms with locational copulae} for an example).

The \isi{experiential} I and II have perfect-like semantics, but are predominantly used when speakers talk about their own experiences and about situations they were personally involved in, so most of the examples contain first person core arguments:
%
\begin{exe}
	\ex	\label{ex:‎When all Chakhri people moved to the lowlands, we (also) moved}
	\gll	hetːi	li<b>il=ra	čːuˁħrug	ka-b-eʁ-ib=qːel,	ka-d-eʁ-ib-te=da	\\
		those	all\tsc{<hpl>=add}	Chakhri.people	\tsc{down-hpl-}go\tsc{.pfv-pret=}when	\tsc{down-npl-}go\tsc{.pfv-pret-dd.pl=1}\\
	\glt	\sqt{‎When all Chakhri people moved to the lowlands, we (also) moved.}

	\ex	\label{ex:Then Abdulkhalik says, have you come here for making condolence or for singing songs}
	\gll	``jaʁari'',	Ø-ik'ʷ-ar	``ušːa	ʡaˁlħaˁm-le	ha-d-ač'-ib-te=da=w'',	Ø-ik'ʷ-ar	``heštːu,		dalaj	d-ik'ʷ-ij	ha-d-ač'-ib-te=da=w?''\\
		\tsc{prt}	\tsc{m-}say\tsc{.ipfv-prs}	\tsc{2pl}	condolence\tsc{-loc}	\tsc{up-npl-}come\tsc{.pfv-pret-dd.pl=1=q}	\tsc{m-}say\tsc{.ipfv-prs}	here		song	\tsc{npl-}say\tsc{.ipfv-inf}	\tsc{up-npl-}come\tsc{.pfv-pret-dd.pl=1=q}\\
	\glt	\sqt{Then (Abdulkhalik) says, ``Have you come here for condolences or for singing songs?''}

	\ex	\label{ex:‎‎‎I was born in 1935 analytic}
	\gll	w-arq'-ib-il=da	du	azir-lim	urč'em	darš-lim	ʡaˁb-c'anu	xu-ra-ibil\\
		\tsc{m-}do\tsc{.pfv-pret-ref=1}	\tsc{1sg}	thousand\tsc{-num}	nine	hundred\tsc{-num}	three-\tsc{ten}	five\tsc{-num-ord}\\
	\glt	\sqt{‎‎‎I (masc.) was born in 1935.}
\end{exe}

Somewhat more rarely one finds third person examples that, however, usually relate to the personal sphere of the speaker or, more generally, to the sphere of the Sanzhi people \refex{ex:And like this with the hands (they) also made them analytic}, \refex{ex:‎Semalla Xaxa (place name). Our father also fell down there}. For instance, \refex{ex:And like this with the hands (they) also made them analytic} is from a procedural text in which the speaker explained how Sanzhi women used to make carpets. There are only few examples that are not immediately related to personal experience, mostly occurring in texts from the \textit{Family Problems Picture Task} \citep{SanRoqueEtAl2012} \refex{ex:‎He came back from prison analytic}.
%
\begin{exe}
	\ex	\label{ex:And like this with the hands (they) also made them analytic}
	\gll	a	tak	nuˁq-b-a-cːella	hel-tːi=ra	d-arq'-ib-te	ca-d\\
		and	so	hand\tsc{-pl-obl-comit}	that\tsc{-pl=add}	\tsc{npl-}do\tsc{.pfv-pret-dd.pl} 	\tsc{cop-npl}\\
	\glt	\sqt{And like this with the hands (they) also made them.}

	\ex	\label{ex:‎Semalla Xaxa (place name). Our father also fell down there}
	\gll	sːema-la	χaˁχaˁ,	nišːa-la	atːa=ra	k-ag-ur-il ca-w	heχtːu-w\\
		pebble.stone\tsc{-gen}	Xaxa	\tsc{1pl-gen}	father\tsc{=add}	\tsc{down}-go\tsc{.pfv-pret-ref}	\tsc{cop-m}	there.\tsc{down-m}\\
	\glt	\sqt{‎Semalla Xaxa (place name). Our father also fell down there.}

	\ex	\label{ex:‎He came back from prison analytic}
	\gll	it	tusnaq-le-r	sa-jʁ-ib-il	ca-w\\
		that	prison\tsc{-loc-abl}	\tsc{hither}-come\tsc{.pfv-pret-ref} \tsc{cop-m}\\
	\glt	\sqt{‎He came back from prison.}
\end{exe}


From a morphosyntactic point of view, the \isi{experiential} and the \isi{experiential past} are somewhere between a monoclausal and a biclausal structure, which is due to the impact of the \is{cross-categorical suffix}cross-categorical suffixes, because the suffixes form words with largely nominal morphosyntactic features. This means that clauses with \isi{experiential} verb forms resemble clefts with a main \isi{copula} clause that contains only the person enclitics or the \isi{copula} and a subordinate \isi{relative clause}. Thus, instead of \isi{person agreement} enclitics one finds the \isi{copula} despite a first or second person \isi{agent}. For example, the first person \isi{agent} in \refex{ex:‎(I) gave birth to (my children) under a blanket analytic} is not expressed, but clear from the context of the autobiographical narrative. In the elicited example \refex{ex:I cut that reed analytic}, the use of a person marker instead of the \isi{copula} is impossible \refex{Intended meaningI cut that reed.}.
%
\begin{exe}
	\ex	\label{ex:‎(I) gave birth to (my children) under a blanket analytic}
	\gll	julʁan-ni-gu-b	b-arq'-ib-te	ca-b\\
		blanket\tsc{-obl-sub-hpl}	\tsc{hpl-}do\tsc{.pfv-pret-dd.pl} 	\tsc{cop-hpl}\\
	\glt	\sqt{‎(I) gave birth to (my children) under a blanket.}
\end{exe}

\begin{exe}
\ex
\begin{xlist}
	\ex	\label{ex:I cut that reed analytic}
	\gll	itːi	qːamuš	dul	ka-d-ičː-ib-te	ca-d\\
		those	reed	\tsc{1sg.erg}	\tsc{down-npl-}cut.up\tsc{.pfv-pret-dd.pl}	\tsc{cop-npl}\\
	\glt	\sqt{I cut that reed.} (E)
	
	\ex[*]{	\label{Intended meaningI cut that reed.}
	\gll  dul itːi	qːamuš	ka-d-ičː-ib-te=da\\
	 \tsc{1sg.erg} those	reed \tsc{down-npl-}cut.up\tsc{.pfv-pret-dd.pl=1} \\
	\glt	{} (Intended meaning: \sqt{I cut that reed.}) (E)}
\end{xlist}
\end{exe}

This suggests that the structure of \refex{ex:I cut that reed analytic} is as displayed in \refex{ex:I cut that reed Russian relative clause analytic}. In fact, when translating \isi{experiential} clauses speakers sometimes produce \is{relative clause}relative clauses in the Russian translation. Thus, a more literary translation that is closer to the structure of \refex{ex:I cut that reed analytic} would be `It is such that the reed was cut by me.' 
%
\begin{exe}
	\ex	\label{ex:I cut that reed Russian relative clause analytic}
	\gll	[itːi	qːamuš	dul	ka-d-ičː-ib-te]	ca-d\\
		those	reed	\tsc{1sg.erg}	\tsc{down-npl-}cut.up\tsc{.pfv-pret-dd.pl}	\tsc{cop-npl}\\
	\glt	\sqt{I cut that reed.} (E)
	

\end{exe}

The almost biclausal structure becomes especially salient in term focus\is{focus} constructions when the person \isi{enclitic} or the \isi{copula} is not following the verbal complex but an argument or adjunct that is focused \refex{ex:It is the reed that I cut analytic}. In this context, the use of the person marker is allowed, but optional \refex{ex:It is me who cut the reed analytic}. Thus, in the last example we can either employ the person \isi{enclitic} after the pronoun or the \isi{copula}, but not both.
%
\begin{exe}
	\ex	\label{ex:It is the reed that I cut analytic}
	\gll	itːi	qːamuš	ca-d	[dul		ka-d-ičː-ib-te]	\\
		those	reed \tsc{cop-npl}	\tsc{1sg.erg}		\tsc{down-npl-}cut.up\tsc{.pfv-pret-dd.pl}\\
	\glt	\sqt{It is the reed that I cut.} (E) 

	\ex	\label{ex:It is me who cut the reed analytic}
	\gll	itːi	qːamuš		du-l=da	/ du-l ca-d ka-d-ičː-ib-te	\\
		those	reed	\tsc{1sg-erg=1}	/ \tsc{1sg.erg} \tsc{cop-npl} \tsc{down-npl-}cut.up\tsc{.pfv-pret-dd.pl}\\
	\glt	\sqt{It is me who cut the reed.} (E)
\end{exe}

The biclausal-like structure is also apparent in \isi{negation} because here always the negative \isi{copula} \tit{akːu} is used and \isi{person agreement} is suppressed \xxref{ex:I did not see them, you also (did not see them)}{ex:‎A cleanness like theirs I have seen nowhere analytic}. A detailed account of the syntactic structure (i.e. whether it is monoclausal or biclausal or should be analyzed as something else) must be left to future research.
%
\begin{exe}
	\ex	\label{ex:I did not see them, you also (did not see them)}
	\gll	dam	či-b-až-ib-te=kːu,	at	akːu	itːi\\
		\tsc{1sg.dat}	\tsc{spr-hpl-}see\tsc{.pfv-pret-dd.pl=neg}		\tsc{2sg.dat} \tsc{cop.neg}	those\\
	\glt	\sqt{I did not see them, you also (did not see them).}

	\ex	\label{ex:‎The bear was not quiet before they did give it its cub back}
	\gll	parʁat	b-arq'-ib-te=kːu	hel	sːika-l		durħuˁ	čar	b-arq'-ij=sat\\
		quiet	\tsc{n-}do\tsc{.pfv-pret-dd.pl=neg}	that	bear\tsc{-erg}	boy	back	\tsc{n-}do\tsc{.pfv-inf}=until\\
	\glt	\sqt{‎The bear was not quiet before they gave it its cub back.}

	\ex	\label{ex:‎A cleanness like theirs I have seen nowhere analytic}
	\gll	itːa-la=ʁuna	amzu-dex	du-l	nalla	či-b-až-ib-il	akːu\\
		those.\tsc{obl-gen=eq}	clean\tsc{-nmlz}	\tsc{1sg-erg}	until.then	\tsc{spr-n-}see\tsc{.pfv-pret-ref}	\tsc{cop.neg}\\
	\glt	\sqt{‎A cleanliness like theirs I have seen nowhere.}
\end{exe}


% --------------------------------------------------------------------------------------------------------------------------------------------------------------------------------------------------------------------- %

\subsection{Experiential past I and experiential past II}
\label{ssec:Experiential past I and experiential past II}

Corresponding to the \isi{experiential} I and II, there are also two variants of the \isi{experiential past} in which the past \isi{enclitic} =\textit{de} is used instead of the person enclitics\slash \isi{copula}. The lexical verbs appear in the same forms as in the \isi{experiential} I and II. The \isi{experiential past} forms are normally used for the narration of personal experiences or of situations that lie within the personal knowledge sphere of the speaker even if s/he did not personally attend it:
%
\begin{exe}
	\ex	\label{ex:As the very last we moved away from the village analytic}
	\gll	bah	hila-r šːi-l-cːe-r	nušːa	gu-r-ag-ur-te=de\\
		most	last\tsc{-abl}	village\tsc{-obl-in-abl}	\tsc{1pl}	\tsc{sub-abl-}go\tsc{.pfv-pret-dd.pl=pst}\\
	\glt	\sqt{As the very last we moved away from the village.}

	\ex	\label{ex:‎Up there we found a stone analytic}
	\gll	hek'tːu-b	b-arčː-ib-il=de	ca	qːarqːa\\
		there.\tsc{up-n}	\tsc{n-}find\tsc{.pfv-pret-ref=pst}	one	stone\\
	\glt	\sqt{‎Up there (we) found a stone.}
\end{exe}

These tense forms are often employed in summary-like utterances that do not move forward the main storyline \refex{ex:Things like this, how often did we do them, how often} or when providing for background information \refex{ex:All children, ten children (he) rose without his wife then}, \refex{ex:‎Her name was Ajshat, she was sent there as the (leader) of the women}.
%
\begin{exe}
	\ex	\label{ex:Things like this, how often did we do them, how often}
	\gll	hel=ʁuna	cik'al	čujna=ra	d-arq'-ib-te=de	nušːa-l,	čujna=ra\\
		this\tsc{=eq}	something	how.often\tsc{=add}	\tsc{1/2.pl-}do\tsc{.pfv-pret-dd.pl=pst}	\tsc{1pl-erg}	how.often\tsc{=add}\\
	\glt	\sqt{Things like this, how often did we do them, how often.}
	
	\ex	\label{ex:All children, ten children (he) rose without his wife then}
	\gll	libil	durħ-ne	wec'al	durħuˁ	xːunul	akːʷ-ar		ha-b-iq'-un-te=de	c'il\\
		all\tsc{<hpl>}	boy\tsc{-pl}	ten	boy	woman 	\tsc{cop.neg-prs}	\tsc{up-hpl-}bring.up\tsc{-pret-dd.pl=pst}	then\\
	\glt	\sqt{All children, ten children (he) rose without his wife then.}

	\ex	\label{ex:‎Her name was Ajshat, she was sent there as the (leader) of the women}
	\gll	cin-na	zu	ʡaˁjšat=de=q'al,	xːun-r-a-la	či-ka-r-at-ur-il=de\\
		\tsc{refl.sg-gen}	name	Ajshat\tsc{=pst=prt}	woman\tsc{-pl-obl-gen}	\tsc{spr-down-f-}let\tsc{.pfv-pret-ref=pst}\\
	\glt	\sqt{‎Her name was Ajshat, she was sent there as the (leader) of the women.}
\end{exe}

In negated clauses the negative past \isi{copula} \tit{akːʷi} is used \refex{ex:Mahammadhazhi was not working as a teacher or what there (in Sanzhi) analytic}. It can be shortened to the \isi{enclitic} \tit{=kːʷi} \refex{ex:It was not like this analytic} or inflected for person \refex{ex:‎‎(The wolf) said, I did not go to school analytic}. The latter is insofar remarkable as the negative predicate in this case expresses more verbal categories than the affirmative, since person cannot be marked on the predicate in the affirmative because the past \isi{enclitic} does not encode person. For example, in \refex{ex:‎‎(The wolf) said, I did not go to school analytic} the person suffix on the \isi{copula} expresses the first person. By contrast, in affirmative clauses with the same verb form person cannot be expressed \refex{ex:As the very last we moved away from the village analytic}, \refex{ex:Things like this, how often did we do them, how often}.
%
\begin{exe}
	\ex	\label{ex:Mahammadhazhi was not working as a teacher or what there (in Sanzhi) analytic}
	\gll	Maħaˁmmadħaˁži	acːi	učitil-li	kelg-un-il=akːʷ-i=w	ce=ja	ixtːu?\\
		Mahammadhazhi	uncle	teacher\tsc{-erg}	remain\tsc{.pfv-pret-ref=cop.neg-hab.pst=q}	what\tsc{=q}	there.\tsc{up}\\
	\glt	\sqt{Mahammadhazhi was not perhaps working as a teacher there (in Sanzhi)?} 
	
	\ex	\label{ex:It was not like this analytic}
	\gll	itwaj	kelg-un-te=kːʷi\\
		like.this	remain\tsc{.pfv-pret-dd.pl=neg.pst}\\
	\glt	\sqt{It was not like this.}

	\ex	\label{ex:‎‎(The wolf) said, I did not go to school analytic}
	\gll	b-ik'-ul	ca-b	``uškul-le	w-aš-ib-il		akːʷ-adi	du''\\
		\tsc{n-}say\tsc{.ipfv-icvb}	\tsc{cop-n}	school\tsc{-loc}	\tsc{m-}go\tsc{-pret-ref}	\tsc{cop.neg-hab.pst.1}	\tsc{1sg}\\
	\glt	\sqt{‎‎(The wolf) said, ``I did not go to school.''}

\end{exe}
