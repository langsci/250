\chapter{The copula and other auxiliaries}\label{cpt:copulaotherauxiliaries}
\largerpage[2]

The \isi{copula} function in \isi{copula} clauses as well as the formation of \is{periphrastic verb form}periphrastic verb forms is fulfilled by predicative \is{particle}particles (enclitics), a \isi{copula} verb (\refsec{sec:The copula}) and other auxiliaries (\refsec{sec:Other verbs used in copula-functions and as auxiliaries}). In addition, Sanzhi has a \isi{number} of specialized copulas for locational and existential clauses (\refsec{sec:Locational copulae}). The syntactic properties of \isi{copula} clauses with examples of predicative \is{particle}particles and verbs in the \isi{copula} function are treated in \refsec{sec:copulaclauses}.

Predicative enclitics are \tit{=da} (first person singular and plural, second person plural), \tit{=de} (second person singular), \tit{=de} (past time reference), \tit{=q'al} (modal \isi{particle}), \tit{=e}\slash\tit{=ja} (marker for content \isi{questions}), \tit{=w}\slash\tit{=uw}\slash\tit{=ew} (marker for polar \isi{questions}) and \tit{=l}\slash\tit{=jal}\slash\tit{=el} (marker for embedded \isi{questions}). They are not verbs and are therefore treated separately in \refsec{sec:Predicative particles}.


%%%%%%%%%%%%%%%%%%%%%%%%%%%%%%%%%%%%%%%%%%%%%%%%%%%%%%%%%%%%%%%%%%%%%%%%%%%%%%%%

\section{The copula}
\label{sec:The copula}

The affirmative \isi{copula} is \tit{ca-b} or \tit{ca<b>i} with a \isi{gender}/\isi{number} agreement affix (the longer variant is much less used than the shorter one). The variant \tit{ca-b} is homophonous with the singular \isi{reflexive pronoun} in the \isi{absolutive} case, and they seem to be cognates. The \isi{copula} is morphologically defective, as it cannot be inflected like other verbs. The only verbal category it expresses on its own is \isi{gender}/\isi{number} agreement; and it can be inflected for the \isi{masdar} (\tit{ca<b>ni}). If no further predicative \is{particle}particles are encliticized it conveys \isi{present tense} reference with third person arguments and affirmative polarity. It has the same functions as the predicative \is{particle}particles, i.e. it heads \isi{copula} clauses \refex{ex:He is a carpenter} and it is used in analytic tenses of main clauses \refex{ex:He is sitting}, but only when the person \isi{agreement controller} is third person. In \isi{copula} clauses the agreement is always controlled by the subject (see example \refex{ex:‎‎‎That boy is a monster} in \refsec{sec:copulaclauses}).

\begin{exe}
	\ex	\label{ex:He is a carpenter}
	\gll	hej	urcul-la	ustːa	ca-w\\
		this	wood-\tsc{gen}	master	\tsc{cop-m}\\
	\glt	\sqt{He is a carpenter.} (E)

	\ex	\label{ex:He is sitting}
	\gll	hež	ka-jž-ib	ca-w\\
		this	\tsc{down}-remain\tsc{.m.pfv-pret}	\tsc{cop-m}\\
	\glt	\sqt{He is sitting.}
\end{exe}

Predicative \is{particle}particles such as the person markers, the past marker, the modal \isi{particle} or the interrogative markers can be added to the \isi{copula} \refex{ex:‎‎I am your 100th friend}. When person markers are used together with the \isi{copula} they are obligatorily attached to it as in \refex{ex:‎‎I am your 100th friend} and can never be encliticized to another constituent \refex{ex:ungrammaticalIamyoursisterA}. Furthermore, first and second person subjects require the use of the person marker \refex{ex:ungrammaticalIamyoursisterA_1}. The use of the \isi{copula} as in \refex{ex:‎‎I am your 100th friend} is optional and cannot replace the person marker \refex{ex:ungrammaticalIamyoursisterB}.

\begin{exe}
	\ex[]{	\label{ex:‎‎I am your 100th friend}
	\gll	at	du=ra	daršːal-ibil	juldaš	ca-w=da\\
		\tsc{2sg.dat}	\tsc{1sg=add}	hundred\tsc{-ord}	friend	\tsc{cop-m=1}\\
	\glt	\sqt{‎‎I am your 100th friend.}}

	\ex[*]{	\label{ex:ungrammaticalIamyoursisterA}
	\gll	du ala rucːi=da ca-r\\
		\tsc{1sg}	\tsc{2sg.gen}	sister\tsc{=1}	\tsc{cop-f}\\
	\glt	(Intended meaning: \sqt{I am your sister.}) (E)}


	\ex[]{\label{ex:ungrammaticalIamyoursisterA_1}
	\gll	du ala rucːi=da \\
		\tsc{1sg}	\tsc{2sg.gen}	sister\tsc{=1}	\\
	\glt	\sqt{I am your sister.} (E)}	
	
	\ex[*]{	\label{ex:ungrammaticalIamyoursisterB}
	\gll	du ala rucːi ca-r\\
		\tsc{1sg}	\tsc{2sg.gen}	sister	\tsc{cop-f}\\
	\glt	(Intended meaning: \sqt{I am your sister.}) (E)}
\end{exe}

The \isi{past tense} \isi{enclitic} can also be used with \refex{ex:‎When the father came back from work, the boy was happy} or without the \isi{copula} \refex{ex:The boy was happy.COP} without any difference in the semantics. The use of the \isi{copula} alone conveys present time \refex{ex:He is a carpenter}, so it is the past \isi{enclitic} that expresses the past time reference. 

\begin{exe}
	\ex	\label{ex:‎When the father came back from work, the boy was happy}
	\gll	cet'-le	atːa	ʡaˁč-le-r	s-ax-an=qːel	durħuˁ	razi-l	ca-w=de\\
		how\tsc{-advz}	father	work\tsc{-loc-abl}	\tsc{hither}-go\tsc{-ptcp=}when	boy	happy\tsc{-advz}	\tsc{cop-m=pst}\\
	\glt	\sqt{‎When the father came back from work, the boy was happy.}

	\ex	\label{ex:The boy was happy.COP}
	\gll	durħuˁ	razi-l=de\\
		boy	happy\tsc{-advz=pst}\\
	\glt	\sqt{The boy was happy.} (E)
\end{exe}

The \isi{copula} has a \isi{masdar} form built with the normal \isi{masdar} suffix -\textit{ni}. As with other verbs, the \isi{masdar} occurs in complement clauses:

\begin{exe}
	\ex	\label{ex:My friends turned out to be open-hearted.}
	\gll	di-la		juldašːe		[urk'i	ač-te		ca-b-ni]			gu-r-b-uq-un\\
		\tsc{1sg-gen}	friend.\tsc{pl}	heart	open-\tsc{dd.pl} 	\tsc{cop-hpl-msd}	\tsc{down-abl-hpl}-go.\tsc{pfv-pret}\\
	\glt	\sqt{My friends turned out to be open-hearted.}
\end{exe}

For all other functions that verbs fulfill when heading independent or dependent clauses, e.g. the use as \is{participle}participles in \is{relative clause}relative clauses or as \isi{infinitive} in complement clauses, the auxiliaries described in \refsec{sec:Other verbs used in copula-functions and as auxiliaries} below are used.\largerpage

The \isi{copula} is also used as an auxiliary for a \isi{number} of \is{analytic verb form}analytic verb forms with third person arguments that control the agreement (compound present, obligative present, \isi{resultative}, perfect, \isi{experiential} I \& II) \refex{ex:She is laughing}. It can never be used in such verb forms with first or second \isit{person agreement} controllers, not even when person markers are encliticized \refex{ex:ungrammaticalIamlaughing}; in such clauses the person markers on their own must be used \refex{ex:I am laughing}. However, it is possible to encliticize the past marker \tit{=de} to the standard \isit{copula} and then use it in clauses with subjects of all persons, although this has only been attested in elicitation \refex{ex:I was / She was laughing}. In general, the past \isit{enclitic} =\textit{de} is incompatible with the person enclitics, but not with the \isit{copula} (\refsec{sec:Predicative particles}).

\begin{exe}
	\ex[*]{	\label{ex:ungrammaticalIamlaughing}
	\gll	du	ħaˁħaˁ	r-ik'-ul	ca-r=da\\
		\tsc{1sg}	laughter \tsc{f-}say\tsc{.ipfv-icvb}	\tsc{cop-f=1}\\
	\glt	(Intended meaning: \sqt{I am laughing.}) (E)}

	\ex[]{	\label{ex:I am laughing}
	\gll	du	ħaˁħaˁ	r-ik'-ul=da\\
		\tsc{1sg}	laughter \tsc{f-}say\tsc{.ipfv-icvb=1}\\
	\glt	\sqt{I am laughing.} (E)}

	\ex[]{\label{ex:She is laughing}
	\gll	it	ħaˁħaˁ	r-ik'-ul	ca-r\\
		that	laughter \tsc{f-}say\tsc{.ipfv-icvb}	\tsc{cop-f}\\
	\glt	\sqt{She is laughing.} (E)}

	\ex[]{\label{ex:I was / She was laughing}
	\gll	du	/	it	ħaˁħaˁ	r-ik'-ul	ca-r=de\\
		\tsc{1sg}	/	that	laughter \tsc{f-}say\tsc{.ipfv-icvb}	\tsc{cop-f=pst}\\
	\glt	\sqt{I was\slash She was laughing.} (E)}
\end{exe}

The stem of the negative \isi{copula} is \tit{(b-)akːʷ-}. It occurs in the following forms:

\begin{itemize}
	\item	simple present: \tit{(b-)akːʷ-a-} + person suffix\slash\tit{(b-)akː-u} (short form \tit{=kːu})
	\item	simple past: \tit{(b-)akːʷ-i-} + person marker
	\item	\isi{participle}: \tit{(b-)akːʷ-ar}
	\item	\isi{masdar}: \tit{(b-)akʷ-ri}\slash\tit{akʷ-ni}
\end{itemize}

The full paradigms of the negative \isi{copula} in the present and the \isi{past tense} are given in \reftab{tab:thenegativecopulapresent} and \reftab{tab:thenegativecopulapast}. The \isi{present tense} has a short variant that appears as an \isi{enclitic} \tit{=kːu} and the \isi{past tense} has the \isi{enclitic} \tit{=kːʷi}. The \isi{enclitic} variants are only used for third person. Examples are found in \refcpt{cpt:Analytic verb forms}.\largerpage

\begin{table}
	\caption{The negative copula in the present tense}
	\label{tab:thenegativecopulapresent}
	\small
	\begin{tabularx}{0.50\textwidth}[]{%
		>{\arraybackslash}p{10pt}QQ}		
		\lsptoprule
			{}	&	singular			&	plural\\
		\midrule
			1	&	\tit{(b-)akːʷa-di}		&	\tit{(b-)akːʷa-di}\\
			2	&	\tit{(b-)akːʷa-tːe}		&	\tit{(b-)akːʷa-tːa}\\
			3	&	\tit{(b-)akːu}			&	\tit{(b-)akːu}\\
		\lspbottomrule\\
	\end{tabularx}
	\end{table}\vskip-\baselineskip
	
	\begin{table}[H]
	\caption{The negative copula in the past tense}
	\label{tab:thenegativecopulapast}
	\small
	\begin{tabularx}{0.70\textwidth}[]{%
		>{\arraybackslash}p{10pt}
		>{\arraybackslash}X
		>{\arraybackslash}X}
		
		\lsptoprule
			{}	&	singular			&	plural\\
		\midrule
			1	&	\tit{(b-)akːʷa-di\slash  (b-)akːʷi}	&	\tit{(b-)akːʷa-di\slash (b-)akːʷi}\\
			2	&	\tit{(b-)akːʷa-tːe\slash (b-)akːʷi}	&	\tit{(b-)akːʷa-tːa\slash (b-)akːʷi}\\
			3	&	\tit{(b-)akːʷi}			&	\tit{(b-)akːʷi}\\
		\lspbottomrule
	\end{tabularx}
\end{table}

As can be seen from \reftab{tab:thenegativecopulapresent} and \reftab{tab:thenegativecopulapast}, as well as from the examples, there are two complications. The first is the syncretism of the present and the \isi{past tense} in the first and second person forms, which is due to the general syncretism of the simple present and past. The third person form of the simple past can, however, also be used for the first and second person, so that in this tense person marking can be avoided and confusion with the simple present circumvented (\reftab{tab:thenegativecopulapast}). The second complication concerns the \isi{gender} prefix. In principle, the verb can agree, but an agreeing negative \isi{copula} can only have an existential or locational interpretation; it never has the normal \isi{copula} meaning. Thus, in \xxref{ex:I am not deaf}{ex:He was not one of us, he was Icari, the head (of the kolkhoz)} \isi{gender} agreement is prohibited because the clauses have identificational semantics, close to the equals sign (\tit{=}). For instance, in \refex{ex:I am not deaf} the unexpressed \isi{copula} subject is female, but the \isi{copula} does not exhibit feminine agreement. Similarly, in \refex{ex:He was not one of us, he was Icari, the head (of the kolkhoz)} the \isi{copula} subject is male, but the \isi{copula} does not show agreement (and \isi{copula} predicates never control agreement).

\begin{exe}
	\ex	\label{ex:I am not deaf}
	\gll	ʡuˁnc-le	akːʷa-di\\
		deaf\tsc{-adv}	\tsc{cop.neg-1}\\
	\glt	\sqt{I am not deaf.} (said by a woman)

	\ex	\label{ex:He is not the person who beats without anything}
	\gll	zad	akːʷ-ar=q'ar	paˁq	∅-ik'ʷ-an	admi	akːu\\
		nothing	\tsc{cop.neg-prs=mod}	strike	\tsc{m-}say\tsc{.ipfv-ptcp}	person	\tsc{cop.neg}\\
	\glt	\sqt{He is not the person who beats without anything (i.e. without a reason).}

	\ex	\label{ex:He was not one of us, he was Icari, the head (of the kolkhoz)}
	\gll	il	nišːa-la	akːʷ-i,		uc'ran=de,	pirsidatil\\
		that	\tsc{1pl-gen}	\tsc{cop.neg-hab.pst}		Icari\tsc{=pst}	head\\
	\glt	\sqt{He was not one of us, he was Icari, the head (of the kolkhoz).}
\end{exe}

As mentioned above, when the negative \isi{copula} is used with a \isi{gender} \isi{agreement prefix} the meaning is existence or location. For this type of meaning the use of the prefix is obligatory. For instance, in \refex{ex:Nobody is there} the subject is female and the \isi{gender} prefix is the one for the feminine \isi{gender}; in \refex{ex:In the village there was no man}, by contrast, the \isi{agreement controller} is male. In principle, the negative \isi{copula} with \isi{gender} prefixes can be treated as a separate word that is functionally analogous to the negated forms of the locational copulas described in \refsec{sec:Locational copulae}, which consist of the negative \isi{copula} with the \isi{gender} prefix and the roots of the locational copulas \refex{ex:There were no boards there}.

\begin{exe}
	\ex	\label{ex:Nobody is there}
	\gll	insan	w-akːu\\
		person	\tsc{m-}\tsc{cop.neg}\\
	\glt	\sqt{Nobody is there.}

	\ex	\label{ex:At one time I was already not there anymore}
	\gll	ca	zamana	uže	heštːu-r	du	r-akːʷa-di\\
		one	time	already	here\tsc{-f}	\tsc{1sg}	\tsc{f-}\tsc{cop.neg-1}\\
	\glt	\sqt{At one time I (fem.) was already not here anymore.}

	\ex	\label{ex:In the village there was no man}
	\gll	šːi-l-cːe-w	murgul	admi	w-akːʷ-i\\
		village\tsc{-obl-in}\tsc{-m}	man	person	\tsc{m-}\tsc{cop.neg-hab.pst}\\
	\glt	\sqt{In the village there was no man.}
\end{exe}

When the \isi{copula} functions as auxiliary \isi{gender} agreement is prohibited \refex{ex:I do not see well}, \refex{ex:He is not going (with his friends)}.\footnote{The only exception are the occasional use of affirmative locational copulas, which have \isi{gender} prefixes, as auxiliaries in \is{periphrastic verb form}periphrastic verb forms (\refsec{sec:Verb forms with locational copulae}).} In \is{tag question}tag questions, the negative \isi{copula} is always used without the \isi{gender} \isi{agreement prefix} \refex{ex:They are inside the house, aren't they}. This is what one would expect, since in affirmative \is{tag question}tag questions also only the standard \isi{copula} and not a location \isi{copula} is used (see \refsec{sec:Tag questions} on \is{tag question}tag questions).

\begin{exe}
	\ex	\label{ex:I do not see well}
	\gll	ʡaˁħ-le	či-d-ig-ul	akːʷa-di	dam\\
		good\tsc{-adv}	\tsc{spr-npl-}see\tsc{.ipfv-icvb}	\tsc{cop.neg-1}	\tsc{1sg.dat}\\
	\glt	\sqt{I do not see well.} (said by a woman)

	\ex	\label{ex:He is not going (with his friends)}
	\gll	arg-ul	akːu\\
		go\tsc{.ipfv-icvb}	\tsc{cop.neg}\\
	\glt	\sqt{He is not going (with his friends).}

	\ex	\label{ex:They are inside the house, aren't they}
	\gll	qili-b	b-i-b	ca-b,	akːu=w,	iš-tːi?\\
		home	\tsc{hpl-in-hpl}	\tsc{cop-hpl}	\tsc{cop.neg=q}	this\tsc{-pl}\\
	\glt	\sqt{They are inside the house, aren't they?}
\end{exe}

The \isi{participle} of the negative \isi{copula} is \tit{(b-)akːʷ-ar(re)}. It translates as \sqt{not having, without} \refex{ex:There is no place without bottles}, \refex{ex:as if his arrival was unexpected (lit. without news), yes unexpected} and fulfills the function of a postposition (\refsec{ssec:postposition akwar}). It can take further suffixes such as the \is{cross-categorical suffix}cross-categorical suffixes \tit{-te/-ce} and \tit{-il}, the \isi{concessive} marker \tit{=xːar} \refex{ex:Well, you like this, even me not being there, should be able to do the 40 days, probably}, the suffix \tit{-dex} that derives abstracts \isi{nouns}, and others. Again \isi{gender} agreement is, in principle, possible, but very rare in texts \refex{ex:Well, you like this, even me not being there, should be able to do the 40 days, probably} and the semantic differences between the omission of \is{agreement prefix}agreement prefixes and their occurrence are identical to what was said before: no \isi{gender} \isi{agreement prefix} means \isi{copula} function; \isi{gender} \isi{agreement prefix} means locational and/or existential function \refex{ex:Well, you like this, even me not being there, should be able to do the 40 days, probably}.

\begin{exe}
	\ex	\label{ex:There is no place without bottles}
	\gll	šuša	akːʷ-ar	musːa	χe-b-akːu	čina-b-k'al\\
		bottle	\tsc{cop.neg-ptcp}	place	exist.\tsc{down-n}-\tsc{cop.neg}	where\tsc{-n-indef}\\
	\glt	\sqt{There is no place without bottles.}

	\ex	\label{ex:as if his arrival was unexpected (lit. without news), yes unexpected}
	\gll	χabar	akːʷ-arre	sa-jʁ-ib-il	daˁʡle,		e	χabar	akːʷ-ar ...\\
		story	\tsc{cop.neg-ptcp}	\tsc{hither}-come\tsc{.m.pfv-pret-ref}	as	yes	story	\tsc{cop.neg-ptcp}\\
	\glt	\sqt{as if his arrival (was) unexpected (lit. without news), yes unexpected ...}

	\ex	\label{ex:Well, you like this, even me not being there, should be able to do the 40 days, probably}
	\gll	hel-itːe,	du	w-akːʷ-ar=xːar	aʁʷc'alla	d-arq'-ij	d-irχ-an=da	ušːa-l		nawerna\\
		that\tsc{-advz}	\tsc{1sg}	\tsc{m-}\tsc{cop.neg-ptcp=conc}	40.days	\tsc{1/2pl-}do\tsc{.pfv-inf}	\tsc{1/2pl-}be.able\tsc{.ipfv-ptcp=2pl}	\tsc{2pl-erg}		probably\\
	\glt	\sqt{Like that, even me not being there, you should be able to do the 40 days, probably.} (i.e. the religious ceremony held 40 days after the death of a person)
\end{exe}

The \isi{masdar} of the negative \isi{copula} is \textit{{(b-)akʷ-ri\slash akʷ-ni}}. The latter form does not have an \isi{agreement prefix} (not even when it encodes existential or locational meaning as in \refex{ex:Now I know, says Ali, that there is no happy man than the one who has a true friend}). It mainly occurs in complement clauses:

\begin{exe}
	\ex	\label{ex:Now I know, says Ali, that there is no happy man than the one who has a true friend}
	\gll	``hana	b-aχ-ur=da,''	w-ik'ʷ-ar	ʡaˁli,	[mar-ce	juldaš	le-w-il-le-r	taliħ-či-w-il	admi	akʷ-ni]''\\
		now	\tsc{n-}know\tsc{.pfv-pret=1}	\tsc{m-}say\tsc{.ipfv-prs}	Ali	truth\tsc{-dd.sg}	friend	exist\tsc{-m-ref-loc-abl}	happiness\tsc{-adjvz-m-ref}	person	\tsc{cop.neg-msd}\\
	\glt	\sqt{``Now I know,'' says Ali, ``that there is no happier man than the one who has a true friend.''}

	\ex	\label{ex:I guessed late that my sheep were not there}
	\gll	[di-la	macːa	b-akʷ-ri]	q'an-ne	šak	∅-ič-ib=da\\
		\tsc{1sg-gen}	sheep	\tsc{n-}\tsc{cop.neg-msd}	late\tsc{-adv}	feel	\tsc{m-}occur\tsc{.pfv-pret=1}\\
	\glt	\sqt{I guessed late that my sheep were not there.} (E)
\end{exe}


%%%%%%%%%%%%%%%%%%%%%%%%%%%%%%%%%%%%%%%%%%%%%%%%%%%%%%%%%%%%%%%%%%%%%%%%%%%%%%%%

\section{Locational copulas}
\label{sec:Locational copulae}

There are four locational copulas that share a consonant bearing a deictic meaning with the \is{demonstrative pronoun}demonstrative pronouns (\refsec{sec:Demonstrative pronouns}). However, in the case of the copulas this is the initial consonant, whereas with the demonstratives it is the stem-final consonant. Furthermore, the copulas agree in \isi{gender}/\isi{number}, whereas the demonstratives lack agreement. Except for the first \isi{copula} (\tit{le-b})\slash pronoun (\tit{hel}) the semantics of the verbs perfectly match the semantics of the pronouns (\reftab{tab:locationalcopulae}).

\begin{table}
	\caption{Locational copulas and demonstrative pronouns}
	\label{tab:locationalcopulae}
	\small
	\begin{tabularx}{\textwidth}{%
		l
		>{\hangindent=0.5em}Q
		l
		>{\hangindent=0.5em}Q}
		\lsptoprule
			loc. \isi{copula}	&	meaning							&	dem. pro.	&	meaning\\
		\midrule
			\tit{le-b}	&	\sqt{close to the speaker (deictic center)} 				&	\tit{hel}		&	\sqt{that\slash those; away from speaker, can be close to the hearer}\\
			\tit{te-b}	&	\sqt{away from the speaker (deictic center) or undifferentiated} 	&	\tit{het}		&	\sqt{that\slash those; not close to speaker or hearer, undifferentiated}\\
			\tit{k'e-b}	&	\sqt{above the deictic center}				&	\tit{hek'}		&	\sqt{above the deictic center}\\
			\tit{χe-b}	&	\sqt{below the deictic center}					&	\tit{heχ}		&	\sqt{below the deictic center}\\
		\lspbottomrule
	\end{tabularx}
\end{table}

The copulas can attach further suffixes (e.g. \is{participle}participles, temporal markers such as \tit{=qːella} or \tit{=er}, \is{cross-categorical suffix}cross-categorical suffixes, the \isi{masdar} \tit{-ni}) \refex{ex:This means that (people) go where the money is} and predicative enclitics (past marker, person marker) \xxref{ex:There was a spring up there}{ex:‎Khadizhat, where are you? I am in Sanzhi_1}, just like the \isi{copula}. But \tnd\ like the \isi{copula} \tnd\ they are defective in comparison to standard lexical verbs because most of the verbal suffixes cannot be added (e.g. suffixes for the \isi{habitual present} and \isi{habitual past}, \isi{conditional} suffixes, the \isi{infinitive}, etc.). The most frequent \isi{copula} is \tit{le-b}, which fulfills a kind of default function.

\begin{exe}
	\ex	\label{ex:This means that (people) go where the money is}
	\gll	il-tːu	arc	le-b-te 	b-ax-u	značit\\
		that-\tsc{loc}	money	exist\tsc{-n-dd.pl} 	\tsc{hpl-}go\tsc{.ipfv-prs}	thus\\
	\glt	\sqt{This means that (people) go where the money is.}\footnote{Regarding \tit{le-b-te}: the agreement on the \isi{locational copula} should better be \tit{le-d-te} because \tit{arc} \sqt{money} normally controls neuter plural agreement, but neuter singular is also possible.}

	\ex	\label{ex:There was a spring up there}
	\gll	hextːu-b	hin-na	k'arant'	k'e-b=de\\
		there.\tsc{up-n}	water\tsc{-gen}	spring	exist.\tsc{up-n=pst}\\
	\glt	\sqt{There was a spring up there.}

	\ex	\label{ex:That is the woman, the one who was keeping the boy in here hands}
	\gll	xːunul	ca-r	heχ	durħuˁ	kʷi	∅-uc-ib-le	χe-r=de=q'al\\
		woman	\tsc{cop-f}	\tsc{dem.down}	boy	in.the.hands	\tsc{m-}catch\tsc{.pfv-pret-cvb}	exist.\tsc{down-f=pst=mod}\\
	\glt	\sqt{That is the woman, the one who was keeping the boy in her hands.}

	\ex	\label{ex:‎Khadizhat, where are you? I am in Sanzhi_1}
	\gll	χadižat,	čina-r=de	u?	du	Sanži-r=da	/	Sanži-r	le-r=da\\
		Khadizhat	where\tsc{-f=2sg}	\tsc{2sg}	\tsc{1sg}	Sanzhi\tsc{.loc-f=1}	/ Sanzhi\tsc{.loc-f} exist\tsc{-f=1}\\
	\glt	\sqt{‎Khadizhat, where are you? I am in Sanzhi.} (E)

\end{exe}

All locational copulas except \tit{le-b} can be negated by suffixing the negative \isi{copula}, and the \isi{gender} agreement follows the standard rules (which means that it is controlled by the subject), i.e. \tit{te-b-akːu, k'e-b-akːu,} and \tit{χe-bakːu} \refex{ex:There is no place without bottles} in the \isi{present tense} and \tit{te-b-akːʷi, k'e-b-akːʷi,} and \tit{χe-b-akːʷi} in the \isi{past tense} \refex{ex:There were no boards there}.

\begin{exe}
	\ex	\label{ex:There were no boards there}
	\gll	urq'l-e	te-d-akːʷ-i	hitːu-d\\
		board\tsc{-pl}	exist\tsc{.away-npl-}\tsc{cop.neg-hab.pst}	there\tsc{-npl}\\
	\glt	\sqt{There were no boards there.}
\end{exe}

The existential copulas, in particular \tit{le-b} due to its less specific meaning, are occasionally used as auxiliaries in \is{periphrastic verb form}periphrastic verb forms together with lexical verbs that bear the perfective or the \isi{imperfective converb} suffixes \refex{ex:[Talking about how a particular plant grows.] ‎‎[It can become large], at my Asjiat's place it covered one wall of the house}, \refex{ex:[at the time when the fox was born], I went run around (said the wolf)} (\refsec{sec:Verb forms with locational copulae}).

\begin{exe}
	\ex	\label{ex:[Talking about how a particular plant grows.] ‎‎[It can become large], at my Asjiat's place it covered one wall of the house}
	[Talking about how a particular plant grows.]\\
	\gll	di-la	Asijat-la	ca	qal-la	baˁʡ	ka-b-uc-ib-le	k'e-b\\
		\tsc{1sg-gen}	Asiyat\tsc{-gen}	one	house\tsc{-gen}	fassade	\tsc{down-n-}catch\tsc{.pfv-pret-cvb}	exist.\tsc{up-n}\\
	\glt	\sqt{‎‎[It can become large], at my Asiyat's place it covered one wall of the house.}

	\ex	\label{ex:[at the time when the fox was born], I went run around (said the wolf)}
	\gll	du	b-ax-ul	le-b=de\\
		\tsc{1sg}	\tsc{n-}go\tsc{.ipfv-icvb}	exist\tsc{-n=pst}\\
	\glt	\sqt{(At the time when the fox was born), I was walking around (said the wolf).}
\end{exe}


%%%%%%%%%%%%%%%%%%%%%%%%%%%%%%%%%%%%%%%%%%%%%%%%%%%%%%%%%%%%%%%%%%%%%%%%%%%%%%%%

\section{Other verbs used in copula-functions and as auxiliaries}
\label{sec:Other verbs used in copula-functions and as auxiliaries}

There is another copula-like verb \tit{b-el} with the meaning \sqt{stay, remain}. It conveys past time reference, although it does not carry any overt marking \refex{ex:Even nowadays that mill remained (i.e. is still there)}, \refex{ex:I remember this (i.e. it remained in my memory)}. The verb is defective and has a reduced inflectional paradigm. In main clauses usually only the bare stem is used. It is the most frequently used base verb in the compound \sqt{remember} \refex{ex:I remember this (i.e. it remained in my memory)}.

\begin{exe}
	\ex	\label{ex:Even nowadays that mill remained (i.e. is still there)}
	\gll	hana	busːaˁʡaˁt=ra	b-el	hel	urχːab\\
		now	this.time\tsc{=add}	\tsc{n-}remain\tsc{.fv}	that	mill\\
	\glt	\sqt{Even nowadays that mill remained (i.e. is still there).}

	\ex	\label{ex:I remember this (i.e. it remained in my memory)}
	\gll	hel	han	b-el	dam\\
		that	remember	\tsc{n-}remain\tsc{.pfv}	\tsc{1sg.dat}\\
	\glt	\sqt{I remember this (i.e. it remained in my memory).}
\end{exe}

It is possible to add person markers \refex{ex:I stayed together with my wife even until today; a good (strong) woman} or the past \isi{enclitic} \refex{ex:Zhabrail remained}, but it cannot be combined with the \isi{copula} (e.g. \tit{*b-el ca-b}). The use of person markers with first and second agreement controlling arguments is obligatory, i.e. in \refex{ex:I stayed together with my wife even until today; a good (strong) woman} the person \isi{enclitic} \tit{=da} cannot be omitted.

\begin{exe}
	\ex	\label{ex:I stayed together with my wife even until today; a good (strong) woman}
	\gll	xːunul-li-cːella hana	ižal	busːaˁʡaˁt=ra	canille	w-el=da,	c'aq'	xːunul\\
		woman\tsc{-obl-comit}	now	today	this.time\tsc{=add}	together	\tsc{m-}remain\tsc{.pfv=1}		strong	woman\\
	\glt	\sqt{I stayed together with my wife even until today; a good (strong) woman.}

	\ex	\label{ex:Zhabrail remained}
	\gll	žaˁbraˁʔil=ra	w-el=de\\
		Zhabrail\tsc{=add}	\tsc{m-}remain\tsc{.pfv=pst}\\
	\glt	\sqt{Zhabrail remained (i.e. stayed alive).}
\end{exe}

It can be negated in a way that strongly resembles the locative copulas (namely by employing the negative \isi{copula}), but with a small change in the stem vowel, i.e. \tit{b-il akːu} and \tit{b-il-akːʷi} \refex{ex:The river carried them away, roads had not remained, and big ditches were there}, (\tit{*b-el akːu}). When the verb \tit{b-el} bears the \isi{perfective converb} suffix \tit{-le} the stem vowel remains unchanged under \isi{negation} \refex{ex:Now I don't remember how many (rubles) it was. (modified example)}.

\begin{exe}
	\ex	\label{ex:The river carried them away, roads had not remained, and big ditches were there}
	\gll	erk'ʷ-li	gu-r-b-erqː-ib-le,	xːun-be=ra	d-il-akːʷ-i,	qːanaw-te	le-d=de,		χːula	qːanaw-te\\
		river\tsc{-erg}	\tsc{sub-abl-n-}carry\tsc{.pfv-pret-cvb}	road\tsc{-pl=add}	\tsc{npl-}remain-\tsc{cop.neg-hab.pst}		ditch\tsc{-pl}	exist\tsc{-npl=pst}		big	ditch\tsc{-pl}\\
	\glt	\sqt{The river carried them away, roads had not remained, and big ditches were there.}

	\ex	\label{ex:Now I don't remember how many (rubles) it was. (modified example)}
	\gll	nu	hana	han	d-el-le=kːu	čum=de=l\\
		well	now	remember	\tsc{npl-}remain\tsc{.pfv-cvb=}\tsc{cop.neg}	how.many\tsc{=pst=indq}\\
	\glt	\sqt{Now I don't remember how many (rubles) it was.} (modified corpus example)
\end{exe}

It can be inflected for some verb forms that occur in subordinate clauses, namely for the \isi{perfective converb} \refex{ex:Uncle Shamkhal died when he was young}, the referential attributive form with \tit{-ce}\slash\tit{-te} (e.g. \tit{d-el-te}), the referential attributive form with \tit{-il} \refex{ex:No place is left where I (masc.) did not go}, the \isi{temporal enclitic} \tit{=qːella} (\tit{b-el=qːella}) and the \isi{masdar} (\tit{b-el-ni}).

\begin{exe}
	\ex	\label{ex:Uncle Shamkhal died when he was young}
	\gll	šamχal	acːi	žahil-le	w-el-le,	w-ebč'-ib=q'al\\
		Shamkhal	uncle	young\tsc{-advz}	\tsc{m-}remain\tsc{.pfv-cvb}	\tsc{m-}die\tsc{.pfv-pret=mod}\\
	\glt	\sqt{Uncle Shamkhal died when he was young.}

	\ex	\label{ex:No place is left where I (masc.) did not go}
	\gll	a-jteʁ-ib	musːa	b-el-il	akːu\\
		\tsc{neg-}reach\tsc{.m.pfv-pret}	place	\tsc{n-}remain\tsc{.pfv-ref}	\tsc{cop.neg}\\
	\glt	\sqt{No place is left where I (masc.) did not go.}
\end{exe}

In \is{conditional clause}conditional clauses it must occur in a \isi{periphrastic verb form} together with \tit{b-iχʷ-} since it cannot itself be inflected for any \isi{conditional} form \refex{ex:If the concrete blocks remained (are left over)}.

\begin{exe}
	\ex	\label{ex:If the concrete blocks remained (are left over)}
	\gll	hu		šalakbluk-me=ra	d-el	d-iχʷ-ar,	\ldots\\
		well		concrete.block\tsc{-pl=add}	\tsc{npl}-remain\tsc{.pfv}	\tsc{npl}-be\tsc{.pfv}-\tsc{cond.3}\\
	\glt	\sqt{If the concrete blocks remained (are left over), \ldots}
\end{exe}

It can serve as an auxiliary in \is{periphrastic verb form}periphrastic verb forms that head main clauses (\refsec{sec:Verb forms with b-el remain, stay}) or, more commonly, temporal \is{adverbial clause}adverbial clauses (\refsec{sec:periphrastic adverbial construction belle}). The latter function is, alongside with the use in main clauses as illustrated in the examples above, the most frequently attested use of this verb.

For all tenses or subordinate clause types, in which the predicative enclitics\slash negative \isi{copula} cannot be used, the verb \tit{b-irχʷ-\slash b-iχʷ-} \sqt{be, become, occur, can} is employed. This verb has the full inflectional paradigm including \isi{conditional} forms \refex{ex:If it would be like this, it would be interesting} and future forms \refex{ex:It will be a strong medicine} and is negated like any other lexical verb. In addition to its use as a \isi{copula}, as a normal lexical verb and in \is{compound verb}compound verbs (\refsec{sec:Compound verbs}), it also occurs as an auxiliary in epistemic modal constructions \refex{ex:She must have been quarreling} (\refsec{ssec:Epistemic modal constructions}) and in realis \isi{conditional} and irrealis \is{conditional clause}conditional clauses (\refsec{ssec:Periphrastic conditionals}).\largerpage

\begin{exe}
	\ex	\label{ex:If it would be like this, it would be interesting}
	\gll	b-iχʷ-ardel,	intersna	b-irχʷ-an=de\\
		\tsc{n-}be\tsc{.pfv-cond.pst}	interesting	\tsc{n-}be\tsc{.ipfv-ptcp=pst}\\
	\glt	\sqt{If it would be like this, it would be interesting.}

	\ex	\label{ex:It will be a strong medicine}
	\gll	c'aq'	darman-na	b-irχʷ-an	ca-b\\
		strong	medicine\tsc{-gen} \tsc{n-}be\tsc{.ipfv-ptcp}	\tsc{cop-n}\\
	\glt	\sqt{It will be a strong medicine.}

	\ex	\label{ex:She must have been quarreling}
	\gll	ʁaj	r-ik'-ul	r-irχʷ-an=de 	heχ\\
		word	\tsc{f-}say\tsc{.ipfv-icvb} 	\tsc{f-}be\tsc{.ipfv-ptcp=pst} 	\tsc{dem.down}\\
	\glt	\sqt{She must have been quarreling.}
\end{exe}

There are four more verbs that are also used in \isi{copula} function and as auxiliaries. The verb \tit{b-irk-\slash b-ik-} \sqt{be, occur, become, get, receive} is used in \isi{copula} constructions \refex{ex:He became a drinker}, especially with predicates that are marked with the \tsc{in}-essive, and in many \is{compound verb}compound verbs \refex{ex:Be silent}.

\begin{exe}
	\ex	\label{ex:He became a drinker}
	\gll	deč-li-cːe	w-ič-ib	ca-w	iž\\
		drinking\tsc{-obl-in}	\tsc{m-}occur\tsc{.pfv-pret}	\tsc{cop-m}	this\\
	\glt	\sqt{He became a drinker.}

	\ex	\label{ex:Be silent}
	\gll	k'ʷah	r-ič-e!\\
		silent	\tsc{f-}occur\tsc{.pfv-imp}\\
	\glt	\sqt{Be silent!} (said to a woman)
\end{exe}

The imperfective verb \tit{b-urkː-} \sqt{find} is regularly used in epistemic modal constructions similar to those formed with the verb \tit{b-irχʷ-\slash b-iχʷ-} \sqt{be, become, occur, can} just mentioned (\refsec{sec:Epistemic modality with the auxiliary b-urk find}). In this function it can be used together with a lexical verb or as the only verb in a \isi{copula} clause: 

\begin{exe}
	\ex	\label{They are probably a family.COP}
	\gll	kulpat	b-urkː-ar			heχ-tːi\\
		family	\tsc{hpl}-find.\tsc{ipfv-prs.3}	\tsc{dem.down-pl}\\
	\glt	\sqt{They are probably a family.}
\end{exe}

The verb \tit{b-už-} (\tsc{pfv}) \sqt{be, stay, remain} is used in \isi{copula} clauses with evidential semantics \refex{ex:Ah, it turned out to be a chicken} and, more generally, as an auxiliary in evidential constructions (\refsec{sec:Indirect evidentiality with the auxiliary b-uz be, be at, stay, remain}). It is not used in \isi{compounding} and not for \is{analytic verb form}analytic verb forms.

\begin{exe}
	\ex	\label{ex:Ah, it turned out to be a chicken}
	\gll	ha	ʡaˁrʡaˁ	b-už-ib	ca-b\\
		uh	chicken	\tsc{n-}stay\tsc{-pret}	\tsc{cop-n}\\
	\glt	\sqt{Ah, it turned out to be a chicken.}
\end{exe}

The verb \tit{k.elgʷ-} (\tsc{pfv}) \sqt{remain, stay, be} is used in \isi{copula} clauses and as an auxiliary conveying habitual or continuative\slash progressive meaning \refex{ex:‎‎They were arguing for a long time}. It is also not used in \is{compound verb}compound verbs (see \refsec{ssec:compoundswithboundroots} for more examples).

\begin{exe}
	\ex	\label{ex:‎‎They were arguing for a long time}
	\gll	d-aqe	čːal	d-ik'-ul	kelg-un\\
		\tsc{npl-}long	argument	\tsc{npl-}say\tsc{.ipfv-icvb}	remain\tsc{.pfv-pret}\\
	\glt	\sqt{‎‎They were arguing for a long time.}
\end{exe}
