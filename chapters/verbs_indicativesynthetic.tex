\chapter{Indicative synthetic verb forms}
\label{cpt:verbs-indicativesynthetic}

Sanzhi Dargwa has only two indicative synthetic verb forms that head independent clauses, the habitual present (\refsec{sec:vis-habitualpresent}) and the habitual past (\refsec{sec:vis-habitualpast}). They are formed by adding stem augmentation vowels and person agreement markers to verbal stems that have imperfective aspect. The stem augmentation vowels occur only with first and second person forms and are also used in conditional clauses with synthetic verb forms (\refcpt{cpt:conditionalconcessiveclauses}). They are \tit{u} for intransitive verbs and \tit{i} for transitive verbs in the habitual present, and \tit{a} for all verbs in the habitual past (with the exception of the verb \tit{b-aχ-} (\tsc{pfv})\slash\tit{b-alχ-} (\tsc{ipfv}) \sqt{know}, which also has \tit{a} as the stem augmentation in the habitual present). The stem augmentation vowels are not separately glossed in the examples, but given together with the person/tense suffixes.


%%%%%%%%%%%%%%%%%%%%%%%%%%%%%%%%%%%%%%%%%%%%%%%%%%%%%%%%%%%%%%%%%%%%%%%%%%%%%%%%
%%%%%%%%%%%%%%%%%%%%%%%%%%%%%%%%%%%%%%%%%%%%%%%%%%%%%%%%%%%%%%%%%%%%%%%%%%%%%%%%

\section{Habitual present}
\label{sec:vis-habitualpresent}

The habitual present is formed by adding person suffixes to the augmented stem of imperfective verbs (\reftab{tab:habitualpresent}). The third person has always the suffix \tit{-u}, which can thus be interpreted as a person marker, although it most probably originates from the stem augmentation vowel for intransitive verbs. Alternatively there is the suffix \tit{-ar} for the third person (see below for a discussion). \reftab{tab:habitualpresent-examples} shows paradigms of three verbs. The intransitive verb `say' is given in the female form for singular persons.

\begin{table}
	\caption{Person suffixes for the habitual present (without stem augmentation vowels)}
	\label{tab:habitualpresent}
	\small
	\begin{tabularx}{0.40\textwidth}[]{%
		>{\centering\arraybackslash}p{10pt}
		>{\centering\arraybackslash}X
		>{\centering\arraybackslash}X}
		
		\lsptoprule
			{}	&	singular	&	plural\\
		\midrule
			1	&	\multicolumn{2}{c}{\tit{-d}}\\
			2	&	\tit{-tːe}	&	\tit{-tːa}\\
			3	&	\multicolumn{2}{c}{\tit{-u\slash -ar}}\\
		\lspbottomrule
	\end{tabularx}
\end{table}

\begin{table}
	\caption{Some illustrative paradigms of the habitual present}
	\label{tab:habitualpresent-examples}
	\small
	\begin{tabularx}{1\textwidth}[]{%
		>{\centering\arraybackslash\small}p{10pt}
		>{\raggedright\arraybackslash}X
		>{\raggedright\arraybackslash}X
		>{\raggedright\arraybackslash}X
		>{\raggedright\arraybackslash}X
		>{\raggedright\arraybackslash}X
		>{\raggedright\arraybackslash}X}
		
		\lsptoprule
			{}	&	\multicolumn{2}{c}{\sqt{say}}
				&	\multicolumn{2}{c}{\sqt{do}}
				&	\multicolumn{2}{c}{\sqt{know}}\\

			{}	&	\multicolumn{1}{c}{singular} &	\multicolumn{1}{c}{plural}
				&	\multicolumn{1}{c}{singular} &	\multicolumn{1}{c}{plural}
				&	\multicolumn{1}{c}{singular} &	\multicolumn{1}{c}{plural}\\

		\midrule

			1	&	\tit{r-ik'-u-d}		&	\tit{d-ik'-u-d}
				&	\tit{b-irq'-i-d}	&	\tit{b-irq'-i-d}
				&	\tit{b-alχ-a-d}	&	\tit{b-alχ-a-d}\\

			2	&	\tit{r-ik'-u-tːe}	&	\tit{d-ik'-u-tːa}
				&	\tit{b-irq'-i-tːe}	&	\tit{b-irq'-i-tːa}
				&	\tit{b-alχ-a-tːe}	&	\tit{b-alχ-a-tːa}\\

			3	&	\tit{r-ik'-u}		&	\tit{b-ik'-u}
				&	\tit{b-irq'-u}		&	\tit{b-irq'-u}
				&	\tit{b-alχ-u}		&	\tit{b-alχ-u}\\
		\lspbottomrule
	\end{tabularx}
\end{table}


% - - - - - - - - - - - - - - - - - - - - - - - - - - - - - - - - - - - - - - - - - - - - - - - - - - - - - - - - - - - - - - - - - - - - - - - - - - - - - - - - - - - - - - - - - - - - - - - - - - - - - - - - - - - - - - - - - - - - - - - - - - %

\subsubsection*{Semantic domains}
\label{sssec:vis-semanticdomains}

\begin{enumerate}
	\item	habitual: used in time-less utterances that state general characteristics (of people, situations, etc.), in procedural texts, and for the description of (traditional) habits:
	\begin{exe}
		\ex	\label{ex:We make egg khinkal}
		\gll	duq-n-a-lla	χːink'-e	d-irq'-id,	wec'al	duqu	k-ert'-id,	c'il	nejg	k-ert'-id\\
			egg-\tsc{pl}-\tsc{obl}-\tsc{gen}	khinkal-\tsc{pl}	\tsc{npl}-do\tsc{.ipfv-1.prs}	ten	egg	\tsc{down}-pour\tsc{.ipfv-1.prs}	then	milk	\tsc{down}-pour\tsc{.ipfv-1.prs}\\
		\glt	\sqt{We make egg khinkal. We pour ten eggs; then we pour milk.}

		\ex	\label{ex:I don’t forget anything}
		\gll	dam	qum.a.art-id	cik'al\\
			\tsc{1sg.dat}	forget\tsc{.ipfv.neg-1.prs}	anything\\
		\glt	\sqt{I don't forget anything.}

		\ex	\label{ex:Can you do this}
		\gll	u-l	b-arq'-ij	w-irχ-utːe=w?\\
			\tsc{2sg-erg}	\tsc{n-}do\tsc{.pfv-inf}	\tsc{m-}be.able\tsc{.ipfv-2sg.prs=q}\\
		\glt	\sqt{Can you do this?}
	\end{exe}

	\item	future and potential future: \sqt{will}/\sqt{could}/\sqt{should}, including the apodosis of conditionals:
	\begin{exe}
		\ex	\tnm{[Tell your head of administration to wait one more day.]}	\label{ex:Then I willcould be there in his office}\\
		\gll	c'il	du	hextːu-w	w-irχʷ-ud	hek'-i-la	kabinet-le-w\\
			then	\tsc{1sg}	there\tsc{.up-m}	\tsc{m-}be.\tsc{ipfv-1}	\tsc{dem.up-obl-gen}	office\tsc{-loc-m}\\
		\glt	\sqt{Then I will/could be there in his office.}

		\ex	\label{ex:Shouldwill I sing my sad song or not}
		\gll	di-la	šišːim-la	dalaj	b-elč'-id=aw	a-b-elč'-id=aw?\\
			\tsc{1sg-gen}	suffering-\tsc{gen}	song	\tsc{n-}read\tsc{.ipfv-1.prs=q}	\tsc{neg-n-}read\tsc{.ipfv-1.prs=q}\\
		\glt	\sqt{Should\slash will I sing my sad song or not?} (a more literal translation is: \sqt{To sing or not to sing the song about my sufferings?})
	\end{exe}
\end{enumerate}

The habitual/future polysemy is common for Dagestanian languages \citep{Tatevosov2005} and also cross-linguistically well-attested \citep{Haspelmath1998}. The future reading has developed from the habitual reading, but it is only available for predicates that express transitory and accidental properties \refex{ex:Then I willcould be there in his office}, \refex{ex:I do not want to sit with you}. Predicates that denote temporally stable and essential properties that characterize their referents only express the habitual meaning \refex{ex:I don't know anything}, \refex{ex:Madina loves me}.
%
\begin{exe}
	\ex	\label{ex:I do not want to sit with you}
	\gll	hel	prosto,	``dam	a-b-ikː-ar'', Ø-ik'-ul ca-w,	``ašːi-cːella ka-jž-ij w-elqː-un-ne=da''	Ø-ik'-ul	ca-w\\
		that	simply	\tsc{1sg.dat}	\tsc{neg-n-}want\tsc{.ipfv-prs.3}	\tsc{m-}say\tsc{.ipfv-icvb}	be\tsc{-m}	\tsc{2pl-comit}	\tsc{down}-remain\tsc{.pfv-inf	} \tsc{m-}sate\tsc{.pfv-pret-cvb=1} \tsc{m-}say\tsc{.ipfv-icvb}	be\tsc{-m}\\
	\glt	\sqt{He simply says, \dqt{I do not want to sit with you, I had enough of you,} he says.}

	\ex	\label{ex:I don't know anything}
	\gll	ca	cik'al	a-b-alχ-ad\\
		one	something	\tsc{neg-n-}know\tsc{.ipfv-1.prs}\\
	\glt	\sqt{I don't know anything.} (NOT: \sqt{I will\slash should not know anything.})

	\ex	\label{ex:Madina loves me}
	\gll	Madina-j	du	w-ičː-aq-id\\
		Madina\tsc{-dat}	\tsc{1sg}	\tsc{m-}want\tsc{.ipfv-caus-1.prs}\\
	\glt	\sqt{Madina loves me (masc.).} (NOT: \sqt{Madina will\slash should love me.}) (E)
\end{exe}

As can be seen in \reftab{tab:habitualpresent}, the third person has two suffixes, \tit{-u} and \tit{-ar}. The latter suffix is less frequently attested in the corpus. It is homophonous with the third person realis conditional suffix -\textit{ar} (\refsec{sec:realisconditional}), and therefore not always easy to identify in texts. It seems that there is a slight semantic difference such that \tit{-u} can refer to single events whereas \tit{-ar} refers to habitually occurring events, but this difference is hard to detect and not always clear. For instance, \refex{ex:Don't you remember her, Bahamma} means \sqt{remember from time to time, think of}, whereas \tit{han b-irk-u} would just mean \sqt{remember (once)}. Similarly, \tit{bek' icː-u} means \sqt{the head aches (now)}, whereas \tit{bek' icː-ar} means that the head aches again and again, like when people have migraine.
By contrast, example \refex{ex:Can Hamid really throw clay into his eyes} shows an utterance, in which \tit{w-irχʷ-ar} could be replaced by \tit{w-irχ-u} without any change in meaning.  
\begin{exe}
	\ex	\label{ex:Don't you remember her, Bahamma}
	\gll	at	han	a-r-irk-ar=uw,	Baˁħaˁmma?	\\
		\tsc{2sg.dat}	remember	\tsc{neg-f-}occur\tsc{.ipfv-3.prs=q}		Bahamma\\
	\glt	\sqt{Don't you remember her, Bahamma?}
	
	\ex	\label{ex:Can Hamid really throw clay into his eyes}
	\gll	c'il	w-irχʷ-ar=uw	hati	ħaˁmid-li	cin-na	ul-b-a-cːe	lak'	d-arq'-ij	il-tːi	ʡaˁnčːi?\\
		then	\tsc{m-}be.able\tsc{.ipfv-3.prs=q}	really	Hamid\tsc{-erg}	\tsc{refl.sg-gen}	eye\tsc{-pl-obl-in}	throw	\tsc{npl}-do.\tsc{pfv-inf}	that-\tsc{pl}	clay(\tsc{npl})\\
	\glt	\sqt{Can Hamid really throw clay into his eyes?}
\end{exe}

The verb \tit{b-ik'ʷ-ij} \sqt{say} is the most frequently used verb with \tit{-ar}, and it is frequently but not always translated as past tense (\tie\ \sqt{said}) without any habitual seamntics  when it bears this suffix. I do not have an explanation for why the verb form (\textit{r/b-/d-})-\textit{ik'ʷ-ar} conveys non-habitual past time semantics. Example \refex{ex:Zhabrail and his family do not invite you} illustrates the use of both suffixes -\textit{u} and -\textit{ar} with this verb in one sentence. 

\begin{exe}
	\ex	\label{ex:Zhabrail and his family do not invite you}
	\gll	``žaˁbraˁʔil-qal	r-aš	a-b-ik'-u=w?''	haʔ-ib=da.	``a-b-ik'-u'',	r-ik'ʷ-ar\\
		Zhabrail\tsc{-assoc}	\tsc{f-}go	\tsc{neg-n-}say\tsc{.ipfv-3.prs=q}	say\tsc{.pfv-pret=1}	\tsc{neg-n-}say\tsc{.ipfv-3.prs}	\tsc{f-}say\tsc{.ipfv-3.prs}\\
	\glt	\sqt{I said, ``Zhabrail and his family do not invite you?'' She said, ``They don't invite me.''} (lit. \sqt{Don't they say \dqt{Come!}})
\end{exe}


Negation is expressed through the prefix \tit{a-} \refex{ex:Don't you remember her, Bahamma}. Some affective verbs allow for the ergative construction (in addition to the dative construction) with the habitual present (this phenomenon requires future research, but see \refsec{sec:Bivalent affective verbs} for some more examples).


% --------------------------------------------------------------------------------------------------------------------------------------------------------------------------------------------------------------------- %

\section{Habitual past}
\label{sec:vis-habitualpast}

The habitual past is the past-tense counterpart of the habitual present. It is only formed from the imperfective stem by means of a suffix \tit{-a} that is followed by person markers (first and second person). The person markers for second person are identical to the person markers used for the habitual present such that \tit{b-aχ-} (\tsc{pfv})\slash\tit{b-alχ-} (\tsc{ipfv}) \sqt{know} has identical forms for the habitual present and past in the second person (compare \reftab{tab:habitualpresent-examples} and \reftab{tab:habitualpast-examples}). In the third person, \tit{-a} is absent. Instead, the suffix \tit{-i} or alternatively the longer variants \tit{-iri} or, rarely, \tit{-ini} are used (\reftab{tab:habitualpast}). As an alternative to \tit{-a} plus person suffix, \tit{-i(ri)} can also be used with first and second person without any difference in meaning.

\begin{table}
	\caption{Person suffixes for the habitual past}
	\label{tab:habitualpast}
	\small
	\begin{tabularx}{0.40\textwidth}[]{%
		>{\centering\arraybackslash}p{10pt}
		>{\centering\arraybackslash}X
		>{\centering\arraybackslash}X}
		
		\lsptoprule
			{}	&	singular		&	plural\\
		\midrule
			1	&	\multicolumn{2}{c}{\tit{-di\slash -i(ri)}}\\
			2	&	\tit{-tːe\slash -i(ri)}	&	\tit{-tːa\slash -i(ri)}\\
			3	&	\multicolumn{2}{c}{\tit{-i(ri)\slash -ini}}\\
		\lspbottomrule
	\end{tabularx}
\end{table}

\begin{table}
	\caption{Some illustrative paradigms of the habitual past}
	\label{tab:habitualpast-examples}
	\small
	\begin{tabularx}{1\textwidth}[]{%
		>{\centering\arraybackslash\small}p{10pt}
		>{\raggedright\arraybackslash}X
		>{\raggedright\arraybackslash}X
		>{\raggedright\arraybackslash}X
		>{\raggedright\arraybackslash}X
		>{\raggedright\arraybackslash}X
		>{\raggedright\arraybackslash}X}
		
		\lsptoprule
			{}	&	\multicolumn{2}{c}{\sqt{say}}
				&	\multicolumn{2}{c}{\sqt{do}}
				&	\multicolumn{2}{c}{\sqt{know}}\\

			{}	&	\multicolumn{1}{c}{singular} &	\multicolumn{1}{c}{plural}
				&	\multicolumn{1}{c}{singular} &	\multicolumn{1}{c}{plural}
				&	\multicolumn{1}{c}{singular} &	\multicolumn{1}{c}{plural}\\

		\midrule

			1	&	\tit{r-ik'ʷ-a-di}	&	\tit{d-ik'ʷ-a-di}
				&	\tit{b-irq'-a-di}	&	\tit{b-irq'-a-di}
				&	\tit{b-alχ-a-di}	&	\tit{b-alχ-a-di}\\

			2	&	\tit{r-ik'ʷ-a-tːe}	&	\mbox{\tit{d-ik'ʷ-a-tːa}}
				&	\tit{b-irq'-a-tːe}	&	\tit{b-irq'-a-tːa}
				&	\tit{b-alχ-a-tːe}	&	\tit{b-alχ-a-tːa}\\

			3	&	\tit{r-ik'ʷ-i(ri)}	&	\tit{b-ik'ʷ-i(ri)}
				&	\tit{b-irq'-i(ri)}	&	\tit{b-irq'-i(ri)}
				&	\tit{b-alχ-i(ri)}	&	\tit{b-alχ-i(ri)}\\
		\lspbottomrule
	\end{tabularx}
\end{table}

The semantic domain is habitual situations with past time reference. The verb form is used to express habitually occurring actions in the past \refex{ex:He always told us}, employed in characterizing persons \refex{ex:Icari people ate it}, when referring to occupations, and so on. The functional range of the habitual past also includes the expression of future-in-the-past in the protasis of past conditionals and irrealis conditionals \refex{ex:If my mother's brother would not have brought me to Chechnya}. As with the habitual present, negation is expressed through the prefix \tit{a-} \refex{ex:If my mother's brother would not have brought me to Chechnya} and some affective verbs additionally allow for the ergative construction with the habitual past (for more information see \refsec{sec:Bivalent affective verbs}).
%
\begin{exe}
	\ex	\label{ex:He always told us}
	\gll	har	zamana	herʔ-i	nišːi-cːe,	Sanži-le	w-ax-an=da\\
		every	time	say.\tsc{ipfv-hab.pst.3}	\tsc{1pl-in}	Sanzhi\tsc{-loc}	\tsc{m-}go\tsc{.ipfv-ptcp=1}\\
	\glt	\sqt{He always told us, ``I will go to Sanzhi.'' }

	\ex	\label{ex:Icari people ate it}
	\gll	uc'ran-t-a-l	b-uk-i,	nušːa-l	kːač	a-b-irq'-a-di\\
		Icari\tsc{-pl-obl-erg}	\tsc{n-}eat\tsc{.ipfv-hab.pst.3}	\tsc{1pl-erg}	touch	\tsc{neg-n-}do.\tsc{ipfv-hab.pst-1}\\
	\glt	\sqt{Icari people ate it (the meat of boars), we did not touch it.}

	\ex	\label{ex:If my mother's brother would not have brought me to Chechnya}
	\gll	di-la	aba-la	ucːi-l	du	čaˁčaˁn-t-a-cːe	a-r-uk-utːel		r-ubk'-a-di\\
		\tsc{1sg-gen}	mother\tsc{-gen}	brother\tsc{-erg}	\tsc{1sg}	Chechen\tsc{-pl-obl-in}	\tsc{neg-f-}lead\tsc{.ipfv-cond.pst}	\tsc{f-}die\tsc{.ipfv-hab.pst-1}\\
	\glt	\sqt{If my mother's brother would not have brought me to Chechnya, I would have died.}

\end{exe}


The verb \tit{b-ik'ʷ-ij} \sqt{say}, which was mentioned in the previous section as expressing past time reference by means of the third person habitual present suffix for reasons that still await clarification is regularly inflected for the habitual past. However, the meaning is not always clearly habitual but seems also to be just a perfective past \refex{ex:Who did you say was the head}.

 
\begin{exe}
	
	\ex	\label{ex:Who did you say was the head}
	\gll	a	presedatel	ča 	ca-w=de	Ø-ik'ʷ-a-tːe?\\
		but	head	who	be\tsc{-m=pst}	\tsc{m-}say\tsc{.ipfv-hab.pst-2sg}\\
	\glt	\sqt{Who (masc.) did you (masc.) say was the head (of the kolkhoz)?}
\end{exe}

The verb \tit{b-ikː-} (\tsc{ipfv}) \sqt{want, like, love}, which lacks a perfective stem, shows exceptional behavior with the habitual forms. The only available forms of the habitual present are \tit{dam b-ikː-i} \sqt{I want} and \tit{nišːij b-ikː-i} \sqt{we want} and for questions \tit{at b-ikː-i=w?} \sqt{Do you (\tsc{sg}) want?} and \tit{ašːij b-ikː-i=w?} \sqt{Do you (\tsc{pl}) want?}. There are no forms for third person and the second person forms cannot be used in assertions. Furthermore, the habitual past expresses irealis modality with the first person, that is, \tit{dam}\slash\tit{nišːij b-ikː-a-di} translates as \sqt{I\slash we would like, I\slash we would want}. It is not used with other persons apart from the first person.


