\chapter{General remarks on verbal morphology}
\label{cpt:verbs}

The morphosyntactic categories of verbs in Sanzhi are person, \isi{gender}, \isi{number}, polarity, tense, mood, aspect, evidentiality, and voice. This chapter provides an overview of the formal make-up of simple verb stems (\refsec{sec:The structure underived verbal stems}) and the general morpheme template of verbs in Sanzhi (\refsec{sec:The morpheme template of Sanzhi verbs and the structure of morphologically complex verb forms}), the formal means of expressing \isi{gender}/\isi{number} and \isi{person agreement} (\refsec{sec:Gender agreementVerb}, \refsec{sec:Stem augment vowels and person agreement}), spatial \isi{preverbs} and their meanings (\refsec{sec:Preverbs}), and polarity (\refsec{sec:Negation}) since these categories are largely independent of the TAME forms and voice. It concludes with an overview of the morphophonological processes that affect the formation and inflection of verbs (\refsec{sec:Morphophonological processes affecting the formation and inflection of verbs}).


%%%%%%%%%%%%%%%%%%%%%%%%%%%%%%%%%%%%%%%%%%%%%%%%%%%%%%%%%%%%%%%%%%%%%%%%%%%%%%%%

\section{Overview of the general morphological structure of verbs}
\label{sec:Overview about the general morphological structure of verbs}

Based on their morphological make-up, verbs can be divided into the following morphological classes:
%
\begin{itemize}
	\item	underived stems
	\item	derived verbs (using spatial \isi{preverbs}, causativization)
	\item	\isi{compound verbs}
\end{itemize}

There are comparably few simple verbal stems that can be used and are actually used without having undergone additional derivational or compositional operations. Most of the verbs are morphologically complex, either making use of one or more derivational affixes, and/or being compounds.

Examples of simple underived stems (including \isi{gender} prefixes) are:
%
\begin{exe}
	\ex	\label{ex:underived verbs verbs}
	\begin{xlist}
		\ex	 \tit{b-isː-} (\tsc{ipfv})\slash\tit{b-asː-} (\tsc{pfv}) \sqt{take, buy}
		\ex	\tit{b-uq'-} (\tsc{ipfv})\slash\tit{b-elq'-} (\tsc{pfv}) \sqt{grind, mill}
		\ex	\tit{b-isː-} \sqt{cry}
		\ex	\tit{b-ilʡ-} (\tsc{ipfv})\slash\tit{b-iʡ-} (\tsc{pfv}) \sqt{steal}
		\ex	\tit{b-alχ-} (\tsc{ipfv})\slash\tit{b-aχ-} \sqt{know}
	\end{xlist}
\end{exe}

The derived verbs contain spatial \isi{preverbs} (\refsec{sec:Preverbs}) and\slash or the \isi{causative} suffix (\refsec{sec:Formation of causative verbs}). The \isi{compound verbs} are of various types:
%
\begin{itemize}
	\item	\isi{light verb} compounds with intransitive auxiliaries such as, e.g. \tit{b-ik'ʷ-} \sqt{say, move} and \tit{b-iχʷ-}\slash\tit{b-irχʷ-} \sqt{be, become, can}, and transitive auxiliaries such as \tit{b-irq'-}\slash\tit{b-arq'-} \sqt{do, make} and \tit{aʁ-} \sqt{do} (\refsec{sec:Compound verbs})
	\item	\isi{compound verbs} containing an invariant bound morpheme from a closed class (i.e. non-spatial \isi{preverbs}) (\refsec{ssec:compoundswithboundroots})
	\item	\isi{compound verbs} that have the morphosyntactic behavior of phrases (\refsec{ssec:compoundswithnouns})
\end{itemize}

This chapter includes only information on spatial \isi{preverbs} \refsec{sec:Preverbs}, because they form a closed class and mostly are in a particularly tight connection with the verbal root, which clearly differentiates them from non-spatial \isi{preverbs} and other items used in verbal \isi{compounding}. Causativization and \isi{compounding} are treated in a separate chapter on verb formation (\refcpt{cpt:verbformation}).

%%%%%%%%%%%%%%%%%%%%%%%%%%%%%%%%%%%%%%%%%%%%%%%%%%%%%%%%%%%%%%%%%%%%%%%%%%%%%%%%

\section{The structure of underived verbal stems}
\label{sec:The structure underived verbal stems}

Simple underived verbs have the structure \tit{(C$_{1}$)V(C$_{2}$)C$_{3}$(ː)}. The only \isi{consonants} and semi-vowels that can occur in the \textit{C$_{1}$} slot are the two resonants \textit{r}, \textit{l}, the glottal stop, which obligatorily occurs before vowel-initial roots and which is not indicated in the spelling, and the pharyngeal stop, which usually occurs before pharyngealized vowels and is indicated in the spelling because it cannot be predicated. If we include also the verbs with \isi{gender} agreement slots before the root and spatial \isi{preverbs}, which are obligatorily used with some verbal roots, we have to add the exponents of \isi{gender} agreement (\textit{b}-, \textit{r}-, \textit{d}-, \textit{w}-) and the \isi{consonants} of deixis/elevation \isi{preverbs} (\textit{h}, \textit{k}, \textit{s}) as possible in the position of \textit{C$_{1}$}. No other \isi{consonants} are allowed. This clearly differentiates verbs from \isi{nouns} or \isi{adjectives}, which do not have similar restrictions (see \refsec{sec:Syllable and word structure} for the general syllable and word structure of Sanzhi). In the position of \textit{C$_{2}$} only sonorants (\textit{r, l, m}) and \textit{b} are permitted, which conforms to a general requirement of the Sanzhi \isi{syllable structure} (in \isi{nouns} also \textit{n} and \textit{j} are allowed). If we include complex stems with deixis/elevation \isi{preverbs} and with the stem vowel \textit{i}, then we have to add \textit{j} to the list. The slot \textit{C$_{1}$} allows for a far greater variety of \isi{consonants} than the other two consonantal slots because only a few \isi{consonants} are excluded (\textit{p, p', b, l, m, n, r}, and the semivowels \textit{w} and \textit{j}). All vowels that Sanzhi has can occur as root vowel of verbs (including all pharyngealized vowels).

In Sanzhi, just like in all other Dargwa varieties, simple underived stems come in pairs that express the aspectual opposition between perfective and imperfective. This opposition is found in most TAM forms and is also preserved in non-finite verb forms such as \isi{participles} and converbs. A very small \isi{number} of finite and non-finite verb forms are available for perfective as well as for imperfective verb stems; most TAM forms can be built either only from imperfective or only from perfective stems. Here I will only describe the formal side of the aspectual opposition. Its meaning is treated in the sections on the respective inflectional verb forms.

The formal expression of the aspectual pairs is largely lexicalized and cannot be predicted. However, verbs can be divided into groups that follow the same patterns. The two different aspectual stems are cognates that seem to be derived one from the other, but there is no unique direction of \isi{derivation}. They can be distinguished on the basis of stem vowels, infix-like segments from a closed class of phonemes (only \textit{r} and \textit{l}), and the presence or absence of a \isi{gender} \isi{agreement prefix}.

In the following, I will briefly describe all patterns that can be identified and provide examples for them. Since there are verbs that are only used together with \isi{preverbs} or other bound morphemes, the verbs given as examples will be morphologically simple and complex. The structure of complex verbs is indicated by dots and - for morpheme boundaries, and the verbs are given with the \isi{gender} \isi{agreement prefix} \tit{b-} (except for the verbs that have a fixed \isi{agreement prefix}).


% --------------------------------------------------------------------------------------------------------------------------------------------------------------------------------------------------------------------- %

\subsection{Differences in gender agreement}
\label{ssec:Differences in the gender agreement}

The structure of the verbs in \reftab{tab:Differences in the gender agreement} is completely identical. The only difference is the potential for \isi{gender} agreement in their prefixal form only present in perfective stems.
%
\begin{table}
	\caption{Differences in the gender agreement}
	\label{tab:Differences in the gender agreement}
	\small
	\begin{tabularx}{0.56\textwidth}[]{%
		>{\raggedright\arraybackslash\itshape}X
		>{\raggedright\arraybackslash\itshape}X
		>{\raggedright\arraybackslash\itshape}p{36pt}
		>{\raggedright\arraybackslash}p{50pt}}
		
		\lsptoprule
			\centering\upshape\tsc{ipfv}
		&	\centering\upshape\tsc{pfv} 
		&	\centering\upshape preterite
		&	translation\\
		\midrule
			\multicolumn{4}{l}{{\tit{iC} vs. \tit{b-iC}}}\\
		\midrule
			~it-		&	b-it-			&	-ib			&	`beat up'\\
			~iršː-		&	b-iršː-			&	-ib			&	`mow'\\
			~ikː-		&	b-ikːʷ-			&	-ub			&	`burn'\\
		\lspbottomrule
	\end{tabularx}
\end{table}


% --------------------------------------------------------------------------------------------------------------------------------------------------------------------------------------------------------------------- %

\subsection{Differences in the stem vowel}
\label{ssec:Differences in the stem vowel}

The structure of the verbs in \reftab{tab:Differences in the stem vowel} is \tit{V(C$_{1}$)C$_{2}$(ː)} with \tit{C$_{1}$} being \tit{r,} \tit{b,} or \tit{m}. The vowel distinctions attested are \tit{i} vs. \tit{a,} \tit{u} vs. \tit{a,} and \tit{u} vs. \tit{e}.
%
\begin{table}
	\caption{Differences in the stem vowel}
	\label{tab:Differences in the stem vowel}
	\small
	\begin{tabularx}{0.88\textwidth}[]{%
		>{\raggedright\arraybackslash\itshape}X
		>{\raggedright\arraybackslash\itshape}X
		>{\raggedright\arraybackslash\itshape}p{36pt}
		>{\raggedright\arraybackslash\hangindent=0.5em}p{75pt}}
		
		\lsptoprule
		\centering\upshape\tsc{ipfv}
		&	\centering\upshape\tsc{pfv} 
		&	\centering\upshape preterite
		&	translation\\
		\midrule
			\multicolumn{4}{l}{{\tit{i} vs. \tit{a}\slash\tit{aˁ}  (with or without \isi{gender} \isi{agreement prefix})}}\\
			\midrule
			~isː-			&	asː-			&	-ib 		&	`take, buy'\\
			~irʁ-			&	arʁ-			&	-ib		&	`understand'\\
			~ibχ-			&	abχ-			&	-ib		&	`comb'\\
			~b-ig- (g > ž)	&	b-ag- (g > ž)		&	-ib		&	`see'\\
			~b-ic'-			&	b-ac'-			&	-ib		&	`thaw'\\
			~b-it.iq-		&	b-it.aq-		&	-ib		&	`disappear'\\
			~b-irq'-		&	b-arq'-		&	-ib		&	`do, make'\\
			~b-irʡ-		&	b-aˁrʡ-		&	-ib		&	`freeze, get\slash become cold'\\
			~b-ik- (k > č)	&	b-ak- (k > č)		&	-ib		&	`smear, spread'\\
			~b-it.ik'- (k' > č')	&	b-it.ak'- (k' > č')	&	-ib		&	`shove in'\\
			~b-it'.ik'- (k' > č')	&	b-it'.ak'- (k' > č')	&	-ib		&	`pull'\\
			~gu-b-ibkː- (kː > čː)	&	gu-b-abkː- (kː > čː)	&	-ib		&	`yoke'\\
			~ka-b-ibχː-		&	ka-b-aˁbχː-		&	-ib		&	`thresh'\\
	\midrule
			\multicolumn{4}{l}{{\tit{u} vs. \tit{a}\slash\tit{aˁ} (with or without \isi{gender} \isi{agreement prefix})}}\\
			\midrule
			~b-urχ-		&	b-arχ-			&	-ur		&	`sew'\\
			~b-urcː-		&	b-arcː-			&	-ur 		&	`get tired'\\
			~b-urkː- (kː > čː)	&	b-arkː- (kː > čː)	&	-ib		&	`find'\\
			~ha-b-urk'- (k' > č')	&	ha-b-ark'- (k' > č')	&	-ib		&	`throw~upwards'\\
			~ʡuˁmč'-		&	ʡaˁmč'-		&	-un		&	`become bald, lose hair, scrape clean'\\
	\midrule
			\multicolumn{4}{l}{{\tit{u} vs. \tit{e} (with or without \isi{gender} \isi{agreement prefix})}}\\
			\midrule
			~qum.urt- 		& 	qum.ert- 		& 	-ur 		&	`forget'\\
			\multicolumn{4}{l}{~~\tit{(qum.a.art-} when negated)}\\
			~urg-			&	ergʷ-			&	-ur		&	`sieve'\\
			~urč-			&	erč-			&	-ur		&	`saw'\\
			~b-umtː-		&	b-emtː-		&	-un		&	`swell'\\
			~b-urh-		&	b-erh-			&	-ib		&	`knock,~strike,~bang'\\
			~b-uč'-		&	b-elč'-			&	-un		&	`read, learn'\\
		\lspbottomrule
	\end{tabularx}
\end{table}


% --------------------------------------------------------------------------------------------------------------------------------------------------------------------------------------------------------------------- %

\subsection{Insertion of \tit{r} in the imperfective stem}
\label{ssec:Insertion of r in the imperfective stem}

The pattern in \reftab{tab:Insertion of r in the imperfective stem} occurs without and in combination with ablaut of the stem vowel. The structure of the verbal root is always \tit{VrC(ː)} for the imperfective and \tit{VC(ː)} for the \isi{perfective aspect}. Many of these verbs have the same root vowel in the imperfective as well as the perfective stem, with the majority of verbs having \tit{i}. Then there are a \isi{number} of verbs that have diverging root vowels. Among them there are a few that occur only with a spatial \isi{preverb} (\tit{ka-} or \tit{ha-}). Since there are regular morphophonological process of \textit{a} + \tit{i} > \tit{e} and \tit{a} + \tit{a} > \tit{a(ː)} we can assume that the root vowel of these verbs is \textit{i} for the imperfective stem and \textit{a} for the perfective stem.
%
\begin{table}
	\caption{Insertion of \tit{r} in the imperfective stem}
	\label{tab:Insertion of r in the imperfective stem}
	\small
	\begin{tabularx}{0.88\textwidth}[]{%
		>{\raggedright\arraybackslash\itshape}X
		>{\raggedright\arraybackslash\itshape}X
		>{\raggedright\arraybackslash\itshape}p{36pt}
		>{\raggedright\arraybackslash\hangindent=0.5em}p{75pt}}
		
		\lsptoprule
		\centering\upshape\tsc{ipfv}
		&	\centering\upshape\tsc{pfv} 
		&	\centering\upshape preterite
		&	translation\\
		
		\midrule
			\multicolumn{4}{l}{{\tit{VrC} vs. \tit{VC} (with or without \isi{gender} \isi{agreement prefix})}}\\
				\midrule
			~tːura irt'-		&	tːura it'-		&	-ib		&	`spill out'\\
			~irxʷ-			&	ixʷ-			&	-ub		&	`throw, shoot'\\
			~arg(ʷ)- (g > ž)	&	ag- (g > ž)		&	-ur		&	`go'\\
			~b-irk- (k > č)	&	b-ik- (k > č)		&	-ib		&	`occur, happen' (auxiliary)\\
			~b-irc'-		&	b-ic'-			&	-ib		&	`fill'\\
			~b-irc-		&	b-ic-			&	-ib		&	`sell'\\
			~(ka-)b-irg- (g > ž)	&	(ka-)b-ig- (g > ž)	&	-ib		&	`sit, be' (auxiliary)\\
			~gu.r.b-irxː- (xː > šː)	&	gu.r.b-ixː- (xː > šː)	&	-ib		&	`hide' (intr., tr.) \\
			~ka-b-irxː- (xː > šː)	&	ka-b-ixː- (xː > šː)	&	-ib		&	`put'\\
			~b-urc-		&	b-uc-			&	-ib		&	`catch, keep'\\
			~b-urt'-		&	b-ut'-			&	-ib		&	`distribute'\\
			~sa-b-irʁ- 		&	sa-b-eʁ- (sa-jʁ-)	&	-ib		&	`come'\\
			~ka-b-irčː-		&	ka-b-ičː-		&	-ib		&	`cut up'\\
			~či-r-b-irxʷ-		&	či-r-b-ixʷ-		&	-ub		&	`take off, take away' (e.g. clothes)\\
			~b-irχʷ- 		&	b-iχʷ- 			&	-ub		&	`be, become, be able' (auxiliary)\\
			~b-ircː-		&	b-icː-			&	-ur		&	`stand'\\
			~b-erg- (g > ž)	&	b-eg- (g > ž)		&	-ur		&	`enter'\\

	\midrule
			\multicolumn{4}{l}{{\tsc{prv}-\tit{irC} vs. \tsc{prv}-\tit{aC} (\isi{preverb}, no \isi{gender} \isi{agreement prefix})}}\\
				\midrule
				
			~k.ert'-		&	ka.t'-			&	-ib		&	`pour'\\
			~h.erʔ-		&	ha.ʔ-			&	-ib		&	`say'\\
			~k.erxʷ-		&	ka.xʷ-			&	-ub		&	`kill'\\
			~k.erxʷ-\slash h.erxʷ-	&	ka.xʷ-\slash ha.xʷ-		&	-ub		&	`pour, add'\\
	\midrule
			\multicolumn{4}{l}{{\tit{urC} vs. \tit{aC} (with \isi{gender} \isi{agreement prefix})}}\\
				\midrule
			~b-uˁrq-		&	b-aˁq-			&	-ib		&	`strike, hit, wound'\\
			~tu d-uˁrq-		&	tu d-aˁq-		&	-ib		&	`spit'\\
		\lspbottomrule
	\end{tabularx}
\end{table}


% --------------------------------------------------------------------------------------------------------------------------------------------------------------------------------------------------------------------- %

\subsection{Insertion of \tit{l} in the imperfective stem}
\label{ssec:Insertion of l in the imperfective stem}

Apart from two exceptions, the verbs in \reftab{tab:Insertion of l in the imperfective stem} are all of the structure \tit{VlC(ː)} for the imperfective and \tit{VC(ː)} for the \isi{perfective aspect}, with identical vowels for both verbs.
%
\begin{table}
	\caption{Insertion of \tit{l} in the imperfective stem}
	\label{tab:Insertion of l in the imperfective stem}
	\small
	\begin{tabularx}{0.78\textwidth}[]{%
		>{\raggedright\arraybackslash\itshape}X
		>{\raggedright\arraybackslash\itshape}X
		>{\raggedright\arraybackslash\itshape}p{36pt}
		>{\raggedright\arraybackslash\hangindent=0.5em}p{75pt}}
		
		\lsptoprule
		\centering\upshape\tsc{ipfv}
		&	\centering\upshape\tsc{pfv} 
		&	\centering\upshape preterite
		&	translation\\
		
		\midrule
			\multicolumn{4}{l}{{\tit{VlC(ː)} vs. \tit{VC(ː)} (with or without \isi{gender} \isi{agreement prefix})}}\\
			\midrule
			~k.alt'-	&	k.at'-		&	-un		&	`stick, pin, attach'\\
			~ʡaˁlħ-	&	ʡaˁħ-		&	-un		&	`fly'\\
			~icːalχː-	&	icːaχː-		&	-un		&	`start aching,\newline start hurting'\\
			~b-alχː-	&	b-aχː-		&	-un		&	`feet'\\
			~b-alxʷ-	&	b-axʷ-		&	-un		&	`sow'\\
			~b-alsː-	&	b-asː-		&	-un		&	`glue'\\
			~b-alš-	&	b-aš-		&	-un		&	`knead'\\
			~b-alc- 	&	b-ac-		&	-un		&	`plough'\\
			~ha-b-ilq'-	&	ha-b-iq'-	&	-un		&	`bring up'\\
			~ka-b-ilsː-	&	ka-b-isː- 	&	-un		&	`lay down, lie'\\
			~b-ilʡ-		&	b-iʡ-		&	-uˁn		&	`steal'\\
			~b-ilχ-	&	b-iχ-		&	-un		&	`tie, fasten'\\
			~či-b-b-ilš-	&	či-b-b-iš-	&	-un		&	`go out, die out'\\
			~(ha)-b-ulq- 	&	b-uq-		&	-un		&	`go, run (away),\\
			~~(ha-lqʷ-)	&	~(ha-w-q-)	&	{}		&	move, direct'\\
			~či-r-b-ulg	&	či-r-b-ug-	&	-un		&	`cancel, delete'\\
			~b-alχ-	&	b-aχ-		&	-ur		&	`know'\\
			~b-alt-	&	b-at-		&	-ur		&	`let, leave'\\
		\lspbottomrule
	\end{tabularx}
\end{table}


% --------------------------------------------------------------------------------------------------------------------------------------------------------------------------------------------------------------------- %

\subsection{Insertion of \tit{r} in the perfective stem}
\label{ssec:Insertion of r in the perfective stem}

The verbs in \reftab{tab:Insertion of r in the perfective stem} have the root structure \tit{VC(ː)} for the imperfective and \tit{VrC(ː)} for the \isi{perfective aspect}. Vowels can either be identical or diverge. There are a \isi{number} of verbs in this group that lack \isi{gender} \isi{agreement prefixes} for imperfective stems.
%
\begin{table}
	\caption{Insertion of \tit{r} in the perfective stem}
	\label{tab:Insertion of r in the perfective stem}
	\small
	\begin{tabularx}{0.78\textwidth}[]{%
		>{\raggedright\arraybackslash\itshape}X
		>{\raggedright\arraybackslash\itshape}X
		>{\raggedright\arraybackslash\itshape}p{36pt}
		>{\raggedright\arraybackslash\hangindent=0.5em}p{75pt}}
		
		\lsptoprule
		\centering\upshape\tsc{ipfv}
		&	\centering\upshape\tsc{pfv} 
		&	\centering\upshape preterite
		&	translation\\
		
		\midrule
			\multicolumn{4}{l}{{\tit{VC(ː)} vs. \tit{VrC(ː)} (\isi{gender} \isi{agreement prefixes} only with}}\\
			\multicolumn{4}{l}{{\hspace*{1em}perfective stems)}}\\
			
			\midrule
			~icː-		&	b-ircː-		&	-ib		&	`milk'\\
			~isː-		&	b-irsː-		&	-ib		&	`shave'\\
			~ic-		&	b-irc-		&	-ib		&	`wash'\\
			~iq-		&	b-irq-		&	-ib		&	`chop'\\
			~uqː-		&	b-urqː-		&	-ib		&	`dig'\\
\midrule
			\multicolumn{4}{l}{{\tit{uC(ː)} vs. \tit{erC(ː)} (with or without \isi{gender} agreement prefix}}\\
			\multicolumn{4}{l}{{\hspace*{1em}with imperfective stems)}}\\
			\midrule
			
			~ruxː-		&	b-erxː-		&	-ur		&	`color, paint'\\
			~ruˁqː-	&	b-aˁrqː-	&	-ib		&	`educate'\\
			~utː-		&	b-ertː-		&	-ib		&	`tear, burst, cut off, mow'\\
			~d-uz-	&	b-erz-		&	-ib		&	`spin'\\
			~b-uc'- 	&	b-erc'-		&	-ib		&	`bake, fry, roast'\\
			~b-uqː-	&	b-erqː-		&	-ib		&	`carry, take'\\
			~b-učː-	&	b-erčː-		&	-ib		&	`drink, consume, smoke'\\
			~b-uʔ-	&	b-erʔ-		&	-ib		&	`rot'\\
			~b-uq-	&	b-erq-		&	-ib		&	`suck, feed'\\
			~er-b-urk'-	&	er-b-erk'-	&	-ib		&	`look at'\\
			~~(k' > č')	&	~(k' > č')\\
			~b-uk-	&	b-erkʷ-	&	-un		&	`eat'\\
		\lspbottomrule
	\end{tabularx}
\end{table}


% --------------------------------------------------------------------------------------------------------------------------------------------------------------------------------------------------------------------- %

\subsection{Insertion of \tit{l} in the perfective stem (and usually \textit{l}-initial imperfective stem)}
\label{ssec:Insertion of l in the perfective stem}

The last group of verbs has \tit{l} in the perfective stem, see \reftab{tab:Insertion of l in the perfective stem}. Most of these verbs have divergent stem vowels. The morphological make-up of the perfective verbs belonging to this group is always \tit{VlC(ː)}. The structure of the imperfective verbs is either \tit{VC(ː)} (only with very few verbs) or \tit{lVC(ː)} (majority of verbs). In the latter case the verbs do not have a slot for a \isi{gender} agreement within the root.
%
\begin{table}
	\caption{Insertion of l in the perfective stem}
	\label{tab:Insertion of l in the perfective stem}
	\small
	\begin{tabularx}{0.78\textwidth}[]{%
		>{\raggedright\arraybackslash\itshape}X
		>{\raggedright\arraybackslash\itshape}X
		>{\raggedright\arraybackslash\itshape}p{36pt}
		>{\raggedright\arraybackslash\hangindent=0.5em}p{75pt}}
		
		\lsptoprule
		\centering\upshape\tsc{ipfv}
		&	\centering\upshape\tsc{pfv} 
		&	\centering\upshape preterite
		&	translation\\
		
		\midrule
			\multicolumn{4}{l}{{\tit{uC/iC} vs. \tit{elC/ulC} (with \isi{gender} \isi{agreement prefix})}}\\
			\midrule
			~b-uq'-	&	b-elq'-		&	-un		&	`turn, whirl, grind, mill'\\
\midrule

			\multicolumn{4}{l}{{\tit{luC} vs. \tit{alC/elC/iC} (\isi{gender} \isi{agreement prefixes} only with}}\\
			\multicolumn{4}{l}{{\hspace*{1em}perfective stems)}}\\	
			\midrule
	
			~luχ-		&	b-elχʷ-	&	-un		&	`slaughter, cut'\\
			~luk'- 		&	b-elk'ʷ- 	&	-un		&	`write'\\
			~lug-		&	d-elgʷ-	&	-un		&	`count'\\
			~luq'-		&	b-elq'ʷ-	&	-un		&	`break into pieces, wreck, destroy'\\
			~lux-		&	b-elxʷ-	&	-un		&	`cook, boil' (intr.)\\
			~luʁ-		&	b-alʁ-		&	-un		&	`shear, cut (hair)'\\
			~luc'-		&	b-alc'-		&	-un		&	`gather, collect'\\
			~lug-		&	b-alg-		&	-un		&	`furnish, equip' (e.g. the house for newlyweds)\\
			~luqː-		&	b-elqː-		&	-un		&	`eat one's fill'\\
			~lusː-		&	b-elsː-		&	-un		&	`snarl, braid, get tangled up'\\
			~luk-		&	b-elk-		&	-un		&	`rub away, wear off'\\
			~luk'-		&	b-alk'-		&	-un		&	`bend'\\
			~lukː-		&	b-ikː- (kː > čː)	&	-ib		&	`give'\\
		\lspbottomrule
	\end{tabularx}
\end{table}

\subsection{Verbs with only one aspectual stem and other morphologically exceptional verbs}
\label{ssec:Verbs with only one aspectual stem} 
There are a \isi{number} of defective verbs that lack the second member of the aspectual pair and only have one stem (\reftab{tab:Stems inflecting for all TAM forms (imperfective and perfective)}). This single stem inflects for the verb forms that are normally only or at least predominantly formed from the imperfective stem (e.g. \isi{imperfective converb}, \isi{modal participle}, \isi{prohibitive}) as well as for verb forms that are normally only or at least predominantly formed from the perfective stem (e.g. preterite, \isi{perfective converb}, \isi{imperative}). In the following, I will simply call these verb forms perfective and imperfective TAM forms.
%
\begin{table}
	\caption{Stems inflecting for all TAM forms (imperfective and perfective)}
	\label{tab:Stems inflecting for all TAM forms (imperfective and perfective)}
	\small
	\begin{tabularx}{0.98\textwidth}[]{%
		>{\raggedright\arraybackslash\itshape}p{70pt}
		>{\raggedright\arraybackslash\hangindent=0.5em}X
		>{\raggedright\arraybackslash\itshape}X}
		
		\lsptoprule
			{}
		&	{}
		&	\upshape preterite,\\
			\upshape verb
		&	\upshape translation
		&	\upshape \isi{imperfective converb}\\
		\midrule
			b-ax- (x > š) 	&	`go'				&	b-aš-ib, b-ax-ul\\
			b-ibxː- (xː > šː)	&	`escape'			&	b-ibšː-ib, b-ibxː-ul\\
			b-irʡ-			&	`betray'			&	b-irʡ-ib, b-irʡ-uˁl\\
			b-isː-			&	`cry' 				&	b-isː-ib, b-isː-ul\\
			b-iχː-			&	`guard, beware, care for'	&	b-iχː-ib, b-iχː-uˁl\\
			b-iχː-			&	`believe'			&	b-iχː-ib, d-iχː-uˁl\\
			b-ucː-			&	`work'				&	b-uc-ib, b-uc-ul\\
			b-ug- (g > ž)		&	`remain, stay, be'		&	b-už-ib, b-ug-ul\\
			b-uk- (k > č)		&	`lead, gather' (people or animals, not objects)	&	b-uč-ib, b-uk-ul\\
			b-uk'-			&	`leak, flow out'		&	b-uk'-un, b-uk'-unne\\
			b-ukː-			&	`itch'				&	b-ukː-un, b-ukː-unne\\
			b-ulkː-			&	`beg, plead'			&	b-ulkː-un, b-ulkː-unne\\
			b-umʡ-		&	`romp around, frolic, have fun, play around'	&	b-umʡ-uˁn, b-umʡ-uˁnne\\
			b-urʁ-			&	`throw oneself, rush, attack' &	b-urʁ-ib, b-urʁ-ul\\
			b-urs- 		&	`tell'				&	b-urs-ib, b-urs-ul\\
			b-urž-			&	`strain oneself'		&	b-urž-ib, b-urž-ul\\
			b.us-			&	`rain, snow'			&	b-us-ib, b-us-ul\\
			či-karχʷ-		&	`(en)wrap, cover, coat'	&	či-karχ-ur, či-karχ-ul\\
			(gu-)lik'-		&	`listen'				&	(gu-)lik'-un, (gu-)lik'-unne\\
			halkʷ-			&	`light up, catch fire'		&	halk-un, halk-unne\\
			ibkː- (kː > čː)		&	`steal, snaffle'			&	ibčː-ib, ibkː-ul\\
			icː-			&	`ache, hurt'			&	icː-ib, icː-ul\\
			ka-b-urχː-		&	`beg'				&	ka-b-urχː-ib, ka-b-urχː-ul\\
			kemq-			&	`hang' (intr.)			&	kemqun, kemq-unne\\
			lik'-			&	`worry, suffer, endure'	&	lik'-un, lik'-unne\\
			rurc'-			&	`itch, burn, twitch'		&	rurc'-ib, rurc'-ul\\
			ruˁrčː-			&	`tremble, shake, boil'		&	ruˁrčː-ib, ruˁrčː-ul\\
			rurg-			&	`burn the skin'			&	rurg-ib, rurg-ul\\
			ruˁrq-			&	`boil'				&	ruˁrq-ib, ruˁrq-uˁl\\
			umc'-			&	`search'			&	umc'-un, umc'-ul\\
			umc-			&	`measure'			&	umc-un, umc-unne\\
			uχː-			&	`shine, sparkle, glitter'	&	uχː-ib, uχː-ul\\
		\lspbottomrule
	\end{tabularx}
\end{table}

There are a few exceptional verbs that have restricted possibilities for inflection. These verbs are:

\begin{description}
\item[\tit{b-ikː-} \sqt{want, love, like}]
		Inflectional forms available are the \isi{imperfective converb} (\tit{b-ikː-ul}), the \isi{modal participle} (\tit{b-ikː-an}), the \isi{habitual present} in the third person used with third-person experiencers (\tit{b-ikː-u}), another \isi{habitual present} form that formally corresponds to a third person but can only be used with first person experiencers in assertions and second person experiencers in \isi{questions} (\textit{b-ikː-ar}), another word form that contains the suffix of the \isi{habitual past} (\tit{b-ikː-i}) but has the same meaning and distribution as the form just described, and one word form that formally corresponds to the \isi{habitual past}, but expresses irrealis modality and is only used with first person experiencers (\tit{b-ikː-adi}) (\refsec{sec:vis-habitualpast}). Only the derived \isi{causative} of this verb (\tit{b-ičː-aq-}) can regularly be inflected for TAM forms such as the preterite (\isi{perfective converb}), the \isi{imperative}, and the \isi{infinitive} that otherwise predominantly occur with perfective verb forms.

\item[\tit{určː-} \sqt{fit, suit}]
		Inflectional forms available are the \isi{perfective converb} (\tit{určː-ib}), the \isi{imperative} (\tit{určː-e!}), the \isi{prohibitive} (\tit{ma-určː-ut!}), but no other verb forms, e.g., no \isi{infinitive}, no \isi{imperfective converb}, no \isi{modal participle}.
		
\item[\tit{b-uˁq'-} \sqt{go}]
		Inflectional forms available are the \isi{infinitive} (\tit{b-uˁq'-ij}), the \isi{imperative} (e.g. the form used for feminine singular addressees \tit{r-uˁq'-en} or \tit{r-uˁq'-aˁn!} with no difference in meaning; masculine singular \tit{uˁq'-en}\slash\tit{uˁq'-aˁn!}, etc.), the \isi{prohibitive}, \tit{(maˁ-q'-aˁt} (\tsc{sg}), \tit{maˁ-q'aˁtːaja!} (\tsc{pl})), the \isi{masdar} (\tit{b-uˁq'-ni}), and the \isi{modal interrogative} (\tit{r-uˁq'-ide(l))}, whereby the \isi{prohibitive} form omits the \isi{gender} prefix. The verb can take the three deictic \isi{preverbs} \tit{sa-}, \tit{ha}, and \tit{ka-}, in which case the \isi{gender} \isi{agreement prefix} is left out. The resulting verb forms \tit{saˁq'-}, \tit{haˁq'-}, and \tit{kaˁq'-} only inflect for the \isi{prohibitive} (e.g. \tit{sa-maˁ-q'-aˁt} in the singular, \tit{sa-maˁ-q'-aˁtːaja} in the plural), the \isi{imperfective converb} (e.g. \tit{saˁq'-uˁnne)} and the \isi{modal participle} (\tit{saˁq'-aˁn}).
\end{description}

%%%%%%%%%%%%%%%%%%%%%%%%%%%%%%%%%%%%%%%%%%%%%%%%%%%%%%%%%%%%%%%%%%%%%%%%%%%%%%%%

\section{Gender agreement in verb stems}
\label{sec:Gender agreementVerb}

Gender agreement is an important grammatical category of East Caucasian languages, and also present in Sanzhi. Most of the vowel-initial verbal stems and the two \isi{preverbs} \tit{b-i-} \sqt{in(side)} and \tit{b-it-} \sqt{thither} have \isi{gender} \isi{agreement prefixes}. Furthermore, the locational/existential copulas (\refsec{sec:Locational copulae}) and the copula-auxiliary \tit{ca-b} have a slot for \isi{gender} agreement suffixes (or infix in the variant \tit{ca<b>i,} see \refsec{sec:The copula}). The agreement affixes are displayed in \reftab{tab:Gender agreement affixes in Sanzhi}.
%
\begin{table}
	\caption{Gender agreement affixes in Sanzhi}
	\label{tab:Gender agreement affixes in Sanzhi}
	\small
	\begin{tabularx}{0.46\textwidth}[]{%
		>{\raggedright\arraybackslash}X
		>{\centering\arraybackslash}p{24pt}
		>{\centering\arraybackslash}p{24pt}
		>{\centering\arraybackslash}p{24pt}}
		
		\lsptoprule
		{}			&	\tsc{sg}	 	&	\tsc{1/2pl}		&	\tsc{3pl}\\
		\midrule 
		masculine		&	\tit{w}\slash\O		&	\tit{d}			&	\tit{b}\\
		feminine		&	\tit{r}			&	\tit{d}			&	\tit{b}\\
		neuter		&	\tit{b}			&	\multicolumn{2}{c}{\tit{d}}\\
		\lspbottomrule
	\end{tabularx}
\end{table}
%

The agreement affix for masculine singular is always used when it occurs as a suffix. It is regularly omitted when it occurs as a prefix to a verbal root beginning with \tit{u}, for example \tit{ukː-unne=da} (masc.) vs. \tit{r-ukː-unne=da} (fem.) (eat\tsc{.ipfv-icvb=1sg}) \sqt{I will eat}. It is optionally omitted when the root starts with \tit{i}, for example \tit{(w-)ik'-ul} (masc.) vs. \tit{r-ik'-ul} (\tsc{f}-say\tsc{.ipfv-icvb}) \sqt{saying}.

The \isi{agreement prefixes} disappear when the \isi{preverb} \tit{b-it-} is attached, which contains its own \isi{gender} prefix (see \refsec{ssec:Deixis/gravitation preverbs} for examples). However, if the \isi{preverb} is preceded by a \isi{negation} prefix, then the \isi{gender} agreement can be completely omitted, but such an omission is optional. Thus, the verb \tit{b-it-eʁ-ij} (\tsc{n-thither}-go\tsc{.pfv-inf}) \sqt{go there} has the neutral negative form \tit{a-jt-eʁ-}, which is not specified for \isi{gender}, alongside with the forms preserving the \isi{gender} prefixes \tit{a-b-it-eʁ-,} \tit{a-w-it-eʁ-,} \tit{a-r-it-eʁ-,} and \tit{a-d-it-eʁ-}.

Verbal \isi{gender} agreement has the clause as its domain, and in the majority of cases it is controlled by the \isi{absolutive} argument of the agreeing verb. The syntax of \isi{gender} agreement is treated in detail in \refsec{sec:Gender/number agreement}.


%%%%%%%%%%%%%%%%%%%%%%%%%%%%%%%%%%%%%%%%%%%%%%%%%%%%%%%%%%%%%%%%%%%%%%%%%%%%%%%%

\section{Person agreement and stem augment vowels}
\label{sec:Stem augment vowels and person agreement}

Person agreement is rather reduced, with a clear opposition of speech act participants (first and second person) vs. third person. Formally it shows up as suffixes and as enclitics. The form of the agreement exponent varies depending on the TAM form, and not all TAM forms have \isi{person agreement} markers. The following verb forms distinguish \isi{person agreement}:
%
\begin{description}
	\item[suffixal \isi{person agreement}:] \isi{habitual present} and \isi{habitual past}; \isi{conditional} forms; \isi{optative}, \isi{imperative} and \isi{prohibitive}
	\item[enclitical \isi{person agreement}:] compound present and compound past, perfect, preterite, future, etc.
\end{description}
%
In the \isi{habitual present}, the realis \isi{conditional} and the past \isi{conditional}; the person suffix for first and second persons is preceded by a stem augment vowel (\tit{i, u,} or occasionally \tit{a}) that indicates the valency of the verb (monovalent vs. bivalent or trivalent). Throughout this grammar, the stem augment vowel is not glossed separately, but together with the following TAM suffix. For full lists of the agreement exponents and the distribution of stem augment vowels see \refsec{sec:Person agreement}.

Person agreement has the clause as its domain, and the rules are rather complex and subject to variation. With monovalent predicates, it is the single argument that functions as \isi{agreement controller}. With predicates that require more than one argument, only subject-like arguments (agents or experiencers) or object-like arguments (patients or stimuli) control \isi{person agreement}. Person agreement follows the person hierarchy 1, 2 > 3. In the case of two speech act participants, it is often the second person that triggers the agreement, but first person subject-like arguments are also able to control agreement. The syntax of \isi{gender} agreement is treated in detail in \refsec{sec:Person agreement}.


%%%%%%%%%%%%%%%%%%%%%%%%%%%%%%%%%%%%%%%%%%%%%%%%%%%%%%%%%%%%%%%%%%%%%%%%%%%%%%%%

\section[The morpheme template of Sanzhi verbs]{The morpheme template of Sanzhi verbs and the structure of morphologically complex verb forms}
\label{sec:The morpheme template of Sanzhi verbs and the structure of morphologically complex verb forms}

The morphological structure of verbal predicates in Sanzhi is fairly complex. There are up to five morphemes that can precede the root and up to five that can follow it. These morphemes can be prefixes, and suffixes, but also enclitics and lexical stems functioning as first parts of \isi{compound verbs}. Before the root, there are only prefixes in the form of spatial \isi{preverbs}, \isi{gender}/\isi{number} prefixes and \isi{negation} prefixes and lexical stems used in compounds. After the root, suffixes and enclitics follow. There are restrictions on the combinability of markers in the various slots, for instance TAM forms requiring person suffixes exclude the use of \isi{enclitic} person or tense markers.

\reftab{tab:Verb affixation order template} provides a template for Sanzhi verbs. The slots are, from left to right:
%
\begin{description}[leftmargin=*]
	\item[5-]	first part of a \isi{compound verb} (there are a few \isi{preverbs} and stems used in \isi{compounding} that have \isi{gender} prefixes as one example in the table shows) (see \refsec{sec:Compound verbs} on \isi{compounding}); 
	\item[4-]	location \isi{preverb}, optionally followed by a direction suffix that can only occur together with a \isi{preverb}; the \isi{preverb} b-i- `in, inside' has an additional \isi{gender} marker (\refsec{ssec:Location preverbs and spatial cases expressing direction})
	\item[3-]	\isi{negation} (\refsec{sec:Negation})
	\item[2-]	deixis/elevation \isi{preverb} (\refsec{ssec:Deixis/gravitation preverbs})
	\item[1-]	\isi{gender} \isi{agreement prefix} (\refsec{sec:Gender agreementVerb})
	\item[0]	root
	\item[-1]	\isi{causative} suffix (\refsec{sec:Formation of causative verbs})
	\item[-2]	first TAM slot 
	\item[-3]	second TAM slot 
	\item[-4]	person and tense enclitics (\refsec{sec:Stem augment vowels and person agreement})
\end{description}
%
There are two slots ([5-] and [4-] that contain items that can have \isi{gender} prefixes such that the structure can even be a bit more complicated. Since only very few items in both slots are marked for \isi{gender}, I did not add two additional slots for \isi{gender} to the template. The slots and respective morphemes are treated in various sections of this grammar.
%

\begin{table}
	\caption{Verb affixation order template}
	\label{tab:Verb affixation order template}
	\small
	\begin{tabularx}{1.00\textwidth}[]{%
		>{\raggedright\arraybackslash}X@{\hskip 0em} 	% 5-
		>{\raggedright\arraybackslash}X@{\hskip 0em} 	% 4-
		>{\raggedright\arraybackslash}X@{\hskip 0em} 	% 3-
		>{\raggedright\arraybackslash}X@{\hskip 0em} 	% 2-
		>{\raggedright\arraybackslash}X@{\hskip 0em} 	% 1-
		>{\raggedright\arraybackslash}p{45pt}@{\hskip 0em} 	% 0
		>{\raggedright\arraybackslash}X@{\hskip 0em} 	% -1
		>{\raggedright\arraybackslash}X@{\hskip 0em} 	% -2
		>{\raggedright\arraybackslash}X@{\hskip 0em} 	% -3
		>{\raggedright\arraybackslash}X@{\hskip 0em}} 	% -4
		
		\lsptoprule
			5-	&	4-	&	3-	&	2-	&	1-	&	0		&	-1	&	-2	&	-3	&	-4\\
		\midrule 
			{}	&	\tit{či-} &	\tit{a-} &	{}	&	\tit{d-} &	\tit{ig}	&	{}	&	\tit{-ul} &	{}	&	\tit{=de}\\
			{}	&	\tsc{spr-} &	\tsc{neg-} &	{}	&	\tsc{nhpl-} &	see\tsc{.ipfv}  &	{}	&	\tsc{-icvb} &	{}	&	\tsc{=pst}\\[1mm]
			\multicolumn{10}{l}{\sqt{they were not seen}}\\[1mm]
		\midrule\\[-3mm]
			{}	&	{}	&	\tit{a-} &	\tit{ka-} &	\tit{d-} &	\tit{irxː}	&	{}	&	\tit{-an} &	{}	&	\tit{=da}\\
			{}	&	{}	&	\tsc{neg-} &	\tsc{down}-	&	\tsc{nhpl-} &	put\tsc{.ipfv} &	{}	&	\tsc{-ptcp} &	{}	&	\tsc{=1}\\[1mm]
			\multicolumn{10}{l}{\sqt{I will not put them down}}\\[1mm]
		\midrule\\[-3mm]
			{}	&	\tit{či-r-} &	{}	&	\tit{sa-} &	\tit{b-} &	\tit{ertː}	&	{}	&	\tit{-ij}	&	{}	&	{}\\	
			{}	&	\mbox{\tsc{spr-abl-}} & {}	&	\tsc{hither-} & \tsc{n-} &	take\tsc{.pfv} &	{}	&	\tsc{-inf}	&	{}	&	{}\\[1mm]
			\multicolumn{10}{l}{\sqt{to take it off}}\\[1mm]
		\midrule\\[-3mm]
			\tit{d-al-} &	\tit{hitːi-} &	{}	&	{}	&	\tit{d-} &	\tit{irč}	&	\tit{-aq} &	\tit{-ad} &	{}	&	{}\\	
			\tsc{1/2pl-} & behind- & {}	&	{}	&	\tsc{hpl-}	&	occur\tsc{.ipfv} &	\tsc{-caus}	&	\tsc{-fut.1}	&	{}	&	{}\\
			\multicolumn{2}{l}{~together}\\[1mm]
			\multicolumn{10}{l}{\sqt{we will support}}\\[1mm]
		\midrule\\[-3mm]
			\tit{debga}	&	{}	&	{}	&	{}	&	\tit{b-}	&	\tit{arq'}		&	{}	&	\tit{-ib}	&	\tit{-le}	&	{}\\
			hidden &	{}	&	{}	&	{}	&	\tsc{n-}	&	do\tsc{.pfv}	&	{}	&	\tsc{-pret}	&	\tsc{-cvb}	&	{}\\[1mm]
			\multicolumn{10}{l}{\sqt{hid it}}\\[1mm]
		\midrule\\[-3mm]
			\tit{kːač}	&	{}	&	\tit{ma-}	&	{}	&	\tit{b-}	&	\tit{irq'}	&	{}	&	\tit{-itːa}	&	\tit{-ja}	&	{}\\
			touch	&	{}	&	\tsc{neg-}	&	{}	&	\tsc{n-}	&	do\tsc{.ipfv}	&	{}	&	\tsc{-proh}	&	\tsc{-pl}	&	{}\\[1mm]
			\multicolumn{10}{l}{\sqt{Do not touch it!}}\\[1mm]
		\midrule\\[-3mm]
			{}	&	{}	&	{}	&	{}	&	{}	&	\tit{umc'}		&	{}	&	\tit{-e}	&	{}	&	{}\\
			{}	&	{}	&	{}	&	{}	&	{}	&	search\tsc{.ipfv}	&	{}	&	\tsc{-imp}	&	{}	&	{}\\[1mm]
			\multicolumn{10}{l}{\sqt{Search!}}\\
		\lspbottomrule
	\end{tabularx}
\end{table}

In principle, only the verbal root is obligatory because there is a variant of the \isi{optative} that does not make use of any suffixes. There are a \isi{number} of verbal roots that are bound and can only be used in combination with spatial \isi{preverbs}, for example \tit{kerxʷ-} (\tsc{ipfv})\slash\tit{kaxʷ-} (\tsc{pfv}) \sqt{kill}, and \tit{kert'-} (\tsc{ipfv})\slash\tit{kat'-} (\tsc{pfv}) \sqt{pour}.


%%%%%%%%%%%%%%%%%%%%%%%%%%%%%%%%%%%%%%%%%%%%%%%%%%%%%%%%%%%%%%%%%%%%%%%%%%%%%%%%

\section{Spatial preverbs}
\label{sec:Preverbs}

Sanzhi Dargwa has the typical Dargwa system of \isi{preverbs}, which in their original spatial meaning express location, direction, and deixis/elevation (see \citealp{vandenBerg2003} for a useful overview of \isi{preverbs} in Akusha Dargwa). Preverbs are generally optional, because verbs can occur without \isi{preverbs}, but there are bound verbal roots for which the prefixed \isi{preverbs} are obligatory. 

There is a tight connection between spatial \isi{preverbs} and the verbal stem, and normally they form one phonological word. The order of the \isi{preverbs} is given in \refex{ex:preverb affixation order}. Between the complex location/direction and the deictic \isi{preverbs}, only \isi{negation} prefixes and some enclitics (e.g. the \isi{additive} \tit{=ra} and \tit{arrah} \sqt{at least}) can intervene.
%
\begin{exe}
	\ex	\mbox{[(location)-(direction)]-(deixis/elevation)-root}	\label{ex:preverb affixation order}
\end{exe}

Preverbs do not express aspectual differences, but occur with imperfective and perfective stems. The Sanzhi Dargwa system of \isi{preverbs} can be characterized as being somewhere in-between regular, productive, and semantically transparent systems, like the ones found in Agul, Tabasaran, and Rutul, and non-regular systems as, for instance in Budukh, Kryz, Tsakhur, and Lezgian \citep{Tatevosov2000, Nichols2003, Ganenkov2007}. It is at least partially formally compositional, in the sense that all theoretically possible combinations of location/direction and deictic \isi{preverbs} are attested (\refsec{ssec:Combinations of preverbs}). However, not every verbal stem takes all available \isi{preverbs} or logically possible combination of \isi{preverbs}. With verbs of movement and posture, the semantic contribution of the \isi{preverbs} is relatively straightforward and compositional (see \reftab{tab:Location preverbs with berij (pfv) go, come} for an example), but with most other verbs there is no real semantic transparency and the spatial meaning of the \isi{preverb} is lost.

Preverbs have probably developed from \isi{spatial postpositions}/adverbs, but not all \isi{spatial postpositions}/adverbs are used as \isi{preverbs}. For instance, \tit{ilda} \sqt{on the side, sideways} and \isi{spatial adverbs} derived from \isi{demonstrative pronouns} do not occur as \isi{preverbs}. The directional markers are identical to the directional markers used for the formation of \isi{spatial cases}.


% --------------------------------------------------------------------------------------------------------------------------------------------------------------------------------------------------------------------- %

\subsection{Location preverbs and spatial cases expressing direction}
\label{ssec:Location preverbs and spatial cases expressing direction}

Location \isi{preverbs}, just like the \isi{spatial cases}, express location and direction. All \isi{preverbs} in \reftab{tab:Location preverbs and directional cases} except for the last one are identical to \isi{spatial postpositions} (\refsec{sec:Spatialpostpositions}), though there are more \isi{spatial postpositions} that are not used as \isi{preverbs}. They express the location of an item with respect to a reference point. The directional affixes can be added to the \isi{preverbs} (\tit{-r} for the \isi{ablative}, the \isi{gender} marker for the essive, no affix for the lative), which are the same used with nominals or \isi{spatial adverbs} (\refsec{sec:nouncase}). The directional affixes are suffixes to the location \isi{preverbs}, not prefixes to the verbal stem, because they cannot occur without location \isi{preverbs} and semantically modify the meaning of the location \isi{preverbs}. Thus, location \isi{preverbs} and directional suffixes form a tight union.
%
\begin{table}
	\caption{Location preverbs and directional cases}
	\label{tab:Location preverbs and directional cases}
	\small
	\begin{tabularx}{0.98\textwidth}[]{%
		>{\raggedright\arraybackslash}p{64pt}
		>{\raggedright\arraybackslash}p{38pt}
		>{\raggedright\arraybackslash}p{38pt}
		>{\raggedright\arraybackslash}p{38pt}
		>{\raggedright\arraybackslash}X}
		
		\lsptoprule
			meaning			&	lative			&	\isi{ablative}		&	essive				&	origin\\
		\midrule 
			\sqt{on}			&	\tit{či-}		&	\tit{či-r-}		&	\tit{či-}\tsc{gm-}		&	postposition\slash adverb\\
			\sqt{under, down}		&	\tit{gu-}		&	\tit{gu-r-}		&	\tit{gu-}\tsc{gm-}		&	\isi{spatial case} and\\
			{}				&	{}			&	{}			&	{}				&	~postposition\slash adverb\\
			\sqt{in front of}		&	\tit{sa-}		&	\tit{sa-r-}		&	\tit{sa-}\tsc{gm-}		&	\isi{spatial case} and\\
			{}				&	{}			&	{}			&	{}				&	~postposition\slash adverb\\
			\sqt{in, inside}		&	\tsc{gm-}\tit{i-}	&	\tsc{gm-}\tit{i-r-}	&	\mbox{\tsc{gm-}\tit{i-}\tsc{gm-}} &	postposition\slash adverb\\
			\sqt{behind, after}		&	\tit{hitːi-}		&	\tit{hitːi-r-}		&	\tit{hitːi-}\tsc{gm-}		&	postposition\slash adverb\\
			\sqt{out, outside}		&	\tit{tːura-}		&	\tit{tːura-r-}		&	\mbox{\tit{tːura-}\tsc{gm-}} &	postposition\slash adverb\\
			`in(to)\slash to, 			&	\tit{kʷi-} 		&	\tit{kʷi-r-}		&	\tit{kʷi-}\tsc{gm-}		&	adverb\\
			\multicolumn{2}{l}{~~in(to) the hands'\footnote{This \isi{preverb} can also be used with respect to locations that do not have hands (e.g. animals, etc.). Thus, the meaning is not literally \sqt{into the hands} anymore, and speakers do not translate it with \sqt{into the hands}.}}\\
		\lspbottomrule
	\end{tabularx}
\end{table}

Examples of the \isi{preverbs} with and without markers for directed motion are provided in \refexrange{ex:Even though (the hare) run after (the turtle), it did not reach it verbs}{ex:Otherwise not much (hay) fits inside verbs}.
%
\begin{exe}
	\ex	\label{ex:Even though (the hare) run after (the turtle), it did not reach it verbs}
	\gll	hitːi-b-uq-un=xːar,	hitːi-a-jt-eʁ-ib\\
		\tsc{behind-n-}go\tsc{.pfv-pret=conc}	\tsc{behind}\tsc{-neg-thither}-go\tsc{.pfv-pret}\\
	\glt	\sqt{Even though (the hare) run after (the turtle), it did not reach it.}

	\ex	\label{ex:They went out for a walk verbs}
	\gll	šːatːir	tːura-b-uq-un	ca-b	hex-tːi\\
		walk	\tsc{out}\tsc{-hpl-}go\tsc{.pfv-pret}	\tsc{cop-hpl}	\tsc{dem.up-pl}\\
	\glt	\sqt{They went out for a walk.}

	\ex	\label{ex:Now give another (picture) verbs}
	\gll	na	cara	kʷi-b-ikː-a!\\
		now	other	\tsc{in.the.hands}\tsc{-n-}give\tsc{.pfv-imp}\\
	\glt	\sqt{Now give another (picture)!}

	\ex	\label{ex:Take it away (from in front) verbs}
	\gll	sa-r-b-uqː-a	il!\\
		\tsc{ante-abl-n-}carry\tsc{.pfv-imp}	that\\
	\glt	\sqt{Take it away! (from in front)}

	\ex	\label{ex:Otherwise not much (hay) fits inside verbs}
	\gll	itwaj	d-aqil	d-i-d-ax-ul	akːu=q'al	hex-tːi\\
		like.this	\tsc{npl-}much	\tsc{npl-in}\tsc{-npl-}go\tsc{.ipfv-icvb}	\tsc{neg=mod}	\tsc{dem-pl}\\
	\glt	\sqt{Otherwise not much (hay) fits inside.}
\end{exe}

For a \isi{number} of verbs, the spatial semantics has been lost and has developed into a more metaphorical meaning \refex{ex:(It is enough what) I experienced verbs@2}. Furthermore, with verbs that do not denote the position or the movement of an item, the semantic contribution of the \isi{preverbs} is synchronically opaque \xxref{ex:He is listening carefully to him verbs@3}{ex:I did not see it. [There is no lock}.
%
\begin{exe}
	\ex	\label{ex:(It is enough what) I experienced verbs@2}
	\gll	dam	či-d-d-ač'-ib-te\\
		\tsc{1sg.dat}	\tsc{spr-npl-npl-}come\tsc{.pfv-pret-dd.pl}\\
	\glt	\sqt{(It is enough what) I experienced.}

	\ex	\label{ex:He is listening carefully to him verbs@3}
	\gll	iχ	gu-lik'-un	ca-w	ħaˁq'-le	qːuʁa-l\\
		\tsc{dem.down}	\tsc{sub}-listen\tsc{-pret}	\tsc{cop-m}	very\tsc{-advz}	beautiful\tsc{-advz}\\
	\glt	\sqt{He is listening carefully to him.}

	\ex	\label{ex:I did not see it. [There is no lock}
	\gll	dam	il	či-a-b-až-ib=da\\
		\tsc{1sg.dat}	that	\tsc{spr-neg-n-}see\tsc{.pfv-pret=1}\\
	\glt	\sqt{I did not see it.} (E)
\end{exe}

Because \isi{preverbs} are identical to postpositions and adverbials, it is not always possible to determine whether a specific item functions as the one or the other. For instance, \tit{či-r} in the following example \refex{ex:He fell from the horse verbs} is interpreted as \isi{preverb} by my main language assistant Gadzhimurad Gadzhimuradov, although the combination \tit{urči-le-r=či-r} also exists as a postpositional phrase \sqt{from on the horse}. In the example \refex{ex:He fell from the horse verbs} the \isi{constituent order} can be changed to \textit{či-r-ka-jč-ib urči-le-r}, which excludes an interpretation of \tit{či-r} as postposition and supports the \isi{preverb} analysis. It is also possible to use both the postposition/adverbial and the \isi{preverb} \refex{ex:He fell from the horse verbs_1}. See also \refsec{ssec:postposition ci} for some more examples in which \tit{či-r} rather functions as postposition/adverbial and not as \isi{preverb}.
%
\begin{exe}
	\ex	\label{ex:He fell from the horse verbs}
	\gll	urči-le-r	či-r-ka-jč-ib\\
		horse\tsc{-loc-abl}	\tsc{spr-abl-down}-occur\tsc{.pfv.m-pret}\\
	\glt	\sqt{He fell from the horse.} (E)
	
		\ex	\label{ex:He fell from the horse verbs_1}
	\gll	urči-le-r či-r	či-r-ka-jč-ib\\
		horse\tsc{-loc-abl} on-\tsc{abl} \tsc{spr-abl-down}-occur\tsc{.pfv.m-pret}\\
	\glt	\sqt{He fell from the horse.} (E)
	
\end{exe}

For the most part \isi{preverbs} can not be separated from the verb or follow it \refex{ex:The man fell down ungrammatical verbs}, but in certain contexts (that still await clarification) a separation is possible, just like it has been observed for Tanti Dargwa \citep[107]{Sumbatova.Lander2014} \refex{ex:Take off the hat verbs}.
%
\begin{exe}
	\ex	\label{ex:The man fell down ungrammatical verbs}
	\gll	{*}	admi	ka-jč-ib či-r\\
		{}	person	\tsc{down}-occur\tsc{.pfv.m-pret}	\tsc{on-abl}\\
	\glt	(Intended meaning: \sqt{The man fell down.}) (E)

	\ex	\label{ex:Take off the hat verbs}
	\begin{xlist}
		\ex	\label{ex:Take off the hat verbs@A}
		\gll	či-r-ixʷ-a	qːatːi!\\
			\tsc{spr-abl-}remove\tsc{.pfv-imp}	hat\\
		\glt	\sqt{Take off the hat!} (E)

		\ex	\tit{ixʷ-a	či-r	qːatːi!}	\label{ex:Take off the hat verbs@B}
		\glt	\sqt{Take off the hat!} (E)

		\ex	\tit{ixʷ-a	qːatːi 	či-r}	\label{ex:Take off the hat verbs@C}
		\glt	\sqt{Take off the hat!} (E)
	\end{xlist}
\end{exe}


% --------------------------------------------------------------------------------------------------------------------------------------------------------------------------------------------------------------------- %

\subsection{Deixis and elevation preverbs}
\label{ssec:Deixis/gravitation preverbs}

The participant-oriented and elevation \isi{preverbs}, which are all deictic, are:
%
\begin{itemize}
	\item	\tit{ha-} \sqt{up, upwards}
	\item	\tit{ka-} \sqt{down, downwards}
	\item	\tit{sa-} \sqt{to the speaker, hither}
	\item	\tsc{gm-}\tit{it-} \sqt{away from the speaker, thither}
\end{itemize}

These \isi{preverbs} immediately precede the verbal root, and the only items that can intervene are \isi{gender} \isi{agreement prefixes}. However, if the \isi{preverb} \tsc{gm-}\tit{it-} is added to verbal stems possessing a \isi{gender} prefix, this prefix is omitted, e.g. \tit{či-b-uq-ij} \sqt{attack, hit on, fall upon} vs. \tit{či-b-it-uq-ij} \sqt{go on (something)}, and \tit{gu-b-aˁq-ij} \sqt{beat from down} vs. \tit{gu-b-it-aˁq-ij} \sqt{beat from down}. The deictic/elevation \isi{preverbs} cannot take the directional suffixes since they already convey motion.

The \isi{preverbs} express upwards or downwards motion (elevation) with respect to a deictic center and motion to the speaker and away from the deictic center, which is usually the speaker (participant-oriented deixis). Relevant examples are \refexrange{ex:Get up (said to a man) verbs}{ex:Down through that water I drove the car verbs}.
%
\begin{exe}
	\ex	\label{ex:Get up (said to a man) verbs}
	\gll	ha-jcː-e!\\
		\tsc{up}-get.up\tsc{.pfv.m-imp}\\
	\glt	\sqt{Get up! }(said to a man)

	\ex	\label{ex:Sit down (said to a man) verbs}
	\gll	ka-jž-e!\\
		\tsc{down}-remain\tsc{.m-imp}\\
	\glt	\sqt{Sit down!} (said to a man)

	\ex	\label{ex:He came back verbs}
	\gll	heχ	sa-jʁ-ib\\
		\tsc{dem.down}	\tsc{hither}-come\tsc{.pfv.m-pret}\\
	\glt	\sqt{He came back.}

	\ex	\label{ex:Down through that water I drove the car verbs}
	\gll	heχ	hin-ni-cːe-r	itːu-b-a	b-it-erč'-ib=da	mašin\\
		\tsc{dem.down}	water\tsc{-obl-in-abl}	there\tsc{-n-dir}	\tsc{n-thither}-drive\tsc{.pfv-pret=1}	car\\
	\glt	\sqt{I drove the car down through that water.}
\end{exe}

To younger speakers of Sanzhi, the specific meanings of the \isi{preverbs} \tit{ha-} and \tit{ka-} are not fully clear anymore, and they usually employ only \tit{sa-} as the default form. Older speakers differentiate between:
%
\begin{itemize}
	\item	\tit{ha-b-eʁ-ij} \sqt{go, come upwards}, e.g. from Druzhba to Sanzhi, from the sea to Druzhba;
	\item	\tit{ka-b-eʁ-ij} \sqt{go, come downwards}, e.g. from Sanzhi or Bashlikent to Druzhba, from Druzhba to the sea
	\item	\tit{sa-b-eʁ-ij} \sqt{go, come to the speaker}, e.g. from Moscow, Germany, America to Druzhba
\end{itemize}
%
In contrast, the younger speakers use \tit{sa-b-eʁ-ij} for all contexts.

Among the participant-oriented deixis and elevation \isi{preverbs}, the first three \isi{preverbs} \tit{ha-, ka-,} and \tit{sa-} are far more commonly used than the fourth \isi{preverb} \tsc{gm-}\tit{it-}. For instance, the verb \sqt{carry} takes the first three \isi{preverbs}, but not the fourth, that is \tit{haqː-, kaqː-, saqː-, *b-it-aqː-}. It still needs to be clarified whether this is due to formal reasons (presence vs. absence of \isi{gender} prefixes or morphophonological restrictions) or can be explained functionally. 

\citet{Tatevosov2000} observes that the meanings of the deictic and elevation \isi{preverbs} do not exclude each other, that is a movement can be upwards away from the speaker, but only one of these meanings can be realized through the use of the relevant \isi{preverb}. Which \isi{preverb} is actually employed seems to be an idiosyncratic lexical property of the verbal predicate.

The origin of the participant-oriented deixis and elevation \isi{preverbs} is often opaque. The \isi{preverb} \tit{ka-} is possibly related to the second part of the morphologically complex \isi{ablative} suffix \tit{-r-ka,} but the synchronic semantic contribution of \tit{-ka} in the \isi{ablative} suffix is hard to determine. The deictic \isi{preverb} \tit{sa-} \sqt{to the speaker, hither} is formally identical to the locational \isi{preverb} \tit{sa-} \sqt{in front}, the \isi{spatial case} \tit{-sa} \sqt{in front}, and the postpositions/adverbs \tit{sala} \sqt{in front of} and \tit{sa-b} \sqt{in front, ago}. If this is due to chance or due to cognacy requires further investigation.

A detailed account of the elevation \isi{preverbs} in Sanzhi within a general discussion about the semantic category of `elevation' in Sanzhi can be found in \citet{ForkerLTSanzhi}.

% --------------------------------------------------------------------------------------------------------------------------------------------------------------------------------------------------------------------- %

\subsection{Combinations of preverbs}
\label{ssec:Combinations of preverbs}

The two groups of \isi{preverbs} are independent of each other, in the sense that location can be expressed without participant-oriented deixis/elevation and vice versa, but they can also be combined. The order is fixed, with participant-oriented deixis/elevation \isi{preverbs} occurring closer to the stem \refex{ex:Put (it) down verbs}, \refex{ex:I pour this in as well verbs}. Reverse combinations are ungrammatical.
%
\begin{exe}
	\ex	\label{ex:Put (it) down verbs}
	\gll	či-ka-b-ixː-a!\\
		\tsc{spr-down}\tsc{-n-}put\tsc{.pfv-imp}\\
	\glt	\sqt{Put (it) down!}

	\ex	\label{ex:I pour this in as well verbs}
	\gll	hel=ra	b-i-k-ert'-id\\
		that\tsc{=add}	\tsc{n-in-down}-pour\tsc{.ipfv-1}\\
	\glt	\sqt{I pour that in as well.}
\end{exe}

The location \isi{preverbs} have optional allomorphic variants when followed by the participant-oriented deixis/elevation \isi{preverb} \tit{ha-} because the initial fricative of the second \isi{preverb} is omitted. The \isi{preverbs} ending in \tit{-i} change that vowel to \tit{-e} under omission of \tit{h}, for example \tit{či-} + \tit{ha-} > \tit{če-, kʷi-} + \tit{ha} > \tit{kʷe-}, \tit{hitːi-} + \tit{ha} > \tit{hitːe-}. The stop in the \isi{preverb} \tit{gu-} becomes labialized, that is \tit{gu-} + \tit{ha-} > \tit{gʷa-} \refex{ex:(Somebody) set up verbs}.
%
\begin{exe}
	\ex	\label{ex:(Somebody) set up verbs}
	\gll	c'a	gʷa-b-iq'-un	ca-b\\
		fire	\tsc{from.down.up}\tsc{-n-}set.fire\tsc{.pfv-pret}	\tsc{cop-n}\\
	\glt	\sqt{(Somebody) set up a fire.}
\end{exe}

All location \isi{preverbs} can be combined with all participant-oriented deixis/elevation \isi{preverbs}, and almost all of the logically possible combinations are attested in natural texts. However, many if not almost all verbal roots allow only for certain \isi{preverbs} and combinations of \isi{preverbs} to be attached. Verbs of motion and position have, of course, the greatest freedom, since the \isi{preverbs} have spatial and directional meanings. Combinations with the location \isi{preverbs} \tit{či-, hitːi-}, and \textit{tːura-} and the participant-oriented deictic \isi{preverbs} \tit{ka-} and \tit{ha-} are significantly more frequent than combinations with the other \isi{preverbs}. Combinations with \tsc{gm-}\tit{it-} are generally very rare in natural texts. \reftab{tab:Location preverbs with berij (pfv) go, come} illustrates all combinations with the verb \tit{b-eʁ-ij} (\tsc{pfv}) \sqt{go, come}.
%
\begin{table}
	\caption{Location preverbs with \tit{b-eʁ-ij} (\tsc{pfv}) \sqt{go, come}}
	\label{tab:Location preverbs with berij (pfv) go, come}
	\small
	\begin{tabularx}{0.98\textwidth}[]{%
		>{\raggedright\arraybackslash\itshape}p{70pt}
		>{\raggedright\arraybackslash\hangindent=0.5em}X}
		
		\lsptoprule
			\upshape verb form	&	possible contexts of use\\
		\midrule 
			\multicolumn{2}{l}{{\isi{preverb} \tit{ha-}: the movement is always upwards}}\\\midrule
			~či-ha-b-eʁ-ij	&	from the sea to Sanzhi, from down onto something\\
			~gu-ha-b-eʁ-ij	&	under something, e.g. a mouse goes along under a cupboard\\
			~sa-ha-b-eʁ-ij	&	towards from below to Sanzhi\\
			~b-i-ha-b-eʁ-ij	&	inside, e.g. enter a house\\
			~hitːi-ha-b-eʁ-ij	&	after something or someone, e.g. go or come after the daughter\\
			~kʷi-ha-b-eʁ-ij	&	into the hands, e.g. a fish jumping into the hands\\
			~tːura-ha-b-eʁ-ij	&	out of, e.g. a mouse exiting a hole, come out of the house\\\midrule
			\multicolumn{2}{l}{{\isi{preverb} \tit{ka-}: the movement is always downwards}}\\\midrule
			~či-ka-b-eʁ-ij	&	from Sanzhi to Druzhba, from Druzhba to the sea\\
			~gu-ka-b-eʁ-ij	&	under something downwards\\
			~sa-ka-b-eʁ-ij	&	towards, e.g. encounter somebody who comes from Sanzhi to Druzhba\\
			~b-i-ka-b-eʁ-ij	&	into, e.g. down into the cellar of the house (from above)\\
			~hitːi-ka-b-eʁ-ij	&	reach, go after somebody downwards\\
			~kʷi-ka-b-eʁ-ij	&	into the hands, e.g. a child falling from a tree into the hands of the father\\
			~tːura-ka-b-eʁ-ij	&	out of, e.g. a bird exiting a nest, going out of the house\\\midrule
			\multicolumn{2}{l}{{\isi{preverb} \tit{sa-}: the movement is always to the speaker}}\\\midrule
			~či-sa-b-eʁ-ij	&	reach something close to the speaker\\
			~gu-sa-b-eʁ-ij	&	go under something, e.g. the cat climbed under my fur\\
			~sa-sa-b-eʁ-ij	&	towards the speaker, e.g. my friend met me, came towards me\\
			~b-i-sa-b-eʁ-ij	&	enter, e.g. the guests entered our house\\
			~hitːi-sa-b-eʁ-ij	&	go after somebody, to get somebody or something\\
			~kʷi-sa-b-eʁ-ij	&	into the hands, e.g. the cat jumped into my hands\\
			~tːura-sa-b-eʁ-ij	&	exit, leave, go out, e.g. the bear came out of its den\\\midrule
			\multicolumn{2}{l}{{\isi{preverb} \tit{b-it-}: the movement is always away from the speaker}}\\\midrule
			~či-b-it-eʁ-ij		&	reach something away from the speaker, e.g. reach Moscow\\
			~gu-b-it-eʁ-ij	&	go under, e.g. the rope reached under the cupboard away from the speaker\\
			~sa-b-it-eʁ-ij	&	go towards somebody away from the speaker\\
			~b-i-b-it-eʁ-ij	&	into something, enter something further away\\
			~hitːi-b-it-eʁ-ij	&	go after somebody further away, e.g. the dog chased the cat\\
			~kʷi-b-it-eʁ-ij	&	into the hands, e.g. the cat came into the hands of somebody away from the speaker\\
			~tːura-b-it-eʁ-ij	&	out of, e.g. the horse came out of the river on the other side of the river bank\\
		\lspbottomrule
	\end{tabularx}
\end{table}

The \isi{negation} prefixes \tit{a-} and \tit{ma-} follow the location \isi{preverbs} and precede the participant-oriented deixis/elevation \isi{preverbs} if there are any \refex{ex:Do not go out (of the car) (said to a man) verbs}, \refex{ex:He did not even look at his savior verbs}. It is also possible to insert three \isi{particles} that normally occur as enclitics into the same slot following the location \isi{preverbs}. These clitics are the \isi{additive} \tit{=ra} \refex{ex:He did not even look at his savior verbs}, the emphatic \isi{particle} \tit{=q'ar} \refex{ex:As for listening, s/he is listening verbs}, and \tit{=arrah} (see \refsec{ssec:Further enclitics that manipulate the information structure} for an example). If the enclitics occur in that position, they take scope over the verb.
%
\begin{exe}
	\ex	\label{ex:Do not go out (of the car) (said to a man) verbs}
	\gll	tːura-ma-ka-lq-ut!\\
		\tsc{out-proh-}down-direct\tsc{.m-ipfv-proh.sg}\\
	\glt	\sqt{Do not go out (of the car)! (said to a man)}

	\ex	\label{ex:He did not even look at his savior verbs}
	\gll	ca-w	w-erc-aq-ur-il-li-j	er či=ra-a-w-erč'-ib\\
		\tsc{refl-m}	\tsc{m-}save\tsc{.pfv-caus-pret-ref-obl-dat}	look \tsc{spr=add-neg-m-}look\tsc{.pfv-pret}\\
	\glt	\sqt{He did not even look at his savior.}

	\ex	\label{ex:As for listening, s/he is listening verbs}
	\gll	gu=q'ar-lik'-unne	ca-b\\
		\tsc{sub=mod-}listen\tsc{-icvb}	\tsc{cop-n}\\
	\glt	\sqt{As for listening, s/he is listening.}
\end{exe}


%%%%%%%%%%%%%%%%%%%%%%%%%%%%%%%%%%%%%%%%%%%%%%%%%%%%%%%%%%%%%%%%%%%%%%%%%%%%%%%%

\section{Negation}
\label{sec:Negation}

Negation can be expressed through prefixes or through the negative \isi{copula}, depending on the inflected verb forms. In contrast to some other South Dargwa varieties (e.g. Icari, Shiri), Sanzhi Dargwa does not express \isi{negation} through \isi{reduplication} of the verbal stem. There are two negative prefixes \tit{a-} and \tit{ma-} that occur right before participant-oriented deixis/elevation \isi{preverbs} and root-initial \isi{gender} markers if there are any. The prefix \tit{a-} is occasionally preceded by an additional \isi{gender} \isi{agreement prefix}.

The functional distribution of the \isi{negation} prefixes is as follows: the prefix \tit{a-} is used in the imperfect/preterite, \isi{resultative}, \isi{pluperfect}, \isi{experiential past}, and sometimes also with the perfect and with non-finite verb forms. The prefix \tit{ma-} is only used in the \isi{prohibitive} and the negative \isi{optative}. For all other verb forms the negative \isi{copula} is employed. The negative \isi{copula} has the root \tit{akːʷ-}\footnote{The negative copla has prefixal \isi{gender} agreement when it is used with locational or existential meaning, but this is impossible when it is used for the formation of \isi{analytic verb forms}. See \refsec{sec:The copula} for more information.} (allomorphs \tit{akʷ-, akː-)}, of which the initial vowel is dropped when it is encliticized to a preceding predicate, so that we get \tit{=kːu} and \tit{=kːʷi}. The negative \isi{copula} occurs in four forms: present, past, participial, and \isi{masdar}. See the sections on the TAM forms and \refsec{sec:The copula} on the \isi{copula} for examples of negated predicates.


%%%%%%%%%%%%%%%%%%%%%%%%%%%%%%%%%%%%%%%%%%%%%%%%%%%%%%%%%%%%%%%%%%%%%%%%%%%%%%%%

\section[Morphophonological processes affecting verb formation and inflection]{Morphophonological processes affecting the formation and inflection of verbs}
\label{sec:Morphophonological processes affecting the formation and inflection of verbs}

There are a \isi{number} of regular morphophonological processes that occur when verbs are inflected and that lead to the formation of stem allomorphs. These processes are in part optional, but occur frequently. See \refsec{sec:Phonological and morphophonological alternations} for more information about the processes, their application and alternative variants.

\begin{description}
\item[1. Delabialization of consonant:] Labialization as a consonantal feature disappears when the labial vowel \tit{u} follows, e.g. \tit{b-elk'ʷ-ij} (\tsc{n-}write\tsc{.pfv-inf}) vs. \tit{b-elk-un} (\tsc{n-}write\tsc{.pfv-cvb}). Occasionally, this affects the preceding vowel, in which case both forms are given, e.g. \tit{w-i-h.alqʷ-an=da} (\tsc{m-in-}go\tsc{.ipfv.m-ptcp=1}) `I will\slash should go inside' vs. \tit{b-i-ha-b-ulq-an ca-b} (\tsc{n-in-n-}go\tsc{.ipfv-ptcp} \tsc{cop-n}) \sqt{it will/should go inside}.

\item[2. Omission of root vowel:] Disappearance of the labial root vowel when the verb is inflected for masculine singular by means of an overt prefix \tit{w-}, e.g. \tit{sa-w-q-un} (\tsc{hither-m-}go\tsc{.pfv-pret}) \sqt{he came} vs. \tit{sa-b-uq-un} (\tsc{hither-hpl-}go\tsc{.pfv-pret}) \sqt{they went away}.

\item[3. Omission of glottal fricative between vowels:] The glottal fricative disappears when the deictic \isi{preverb} \tit{ha-} is preceded by location \isi{preverbs}. This process, in turn, affects the quality of the adjacent vowels, e.g. \tit{b-i-} + \tit{ha-} > \tit{be-} (see \refsec{ssec:Combinations of preverbs} above for more examples). This process is optional, i.e., the pronunciation \textit{bi-ha-} is also possible and attested, in particular in slow speech.

\item[4. Vowel lowering:] Lowering of the root vowel \tit{i} when a spatial \isi{preverb} \tit{(ka-, ha-, sa-)} or the \isi{negation} prefixes \tit{(a-, ma-)} are added: \tit{a} + \tit{i} > \tit{e}. This occurs when verbs show agreement for masculine singular and the overt \isi{agreement prefix} is omitted or with verbs that lack an \isi{agreement prefix}, e.g. \tit{ka-r-ircː-u} (\tsc{down-f-}stand\tsc{.ipfv-prs}) \sqt{she stands} vs. \tit{k-ercː-u} (\tsc{down}-stand\tsc{.ipfv.m-prs}) \sqt{he stands}.

\item[5. Diphthongization:] The root vowel \tit{i} changes into a diphthong when a spatial \isi{preverb} \tit{(ka-, ha-, sa-)} or a \isi{negation} prefix \tit{(a-, ma-)} is added before the verbal root: \tit{a} + \tit{i} > \tit{aj}. This occurs when verbs show agreement for masculine singular and the overt \isi{agreement prefix} is omitted or with verbs that lack an \isi{agreement prefix}, e.g. \tit{ma-jk'-utːa} (\tsc{neg-}say\tsc{.ipfv.m-proh.sg}) \sqt{Do not talk!} vs. \tit{ma-r-ik'-utːa} (\tsc{neg-f-}say\tsc{.ipfv-proh.sg}) \sqt{Do not talk!}

\item[6. Palatalization of velar \isi{consonants}:] When the front vowel \tit{i,} the \isi{causative} suffix \tit{-aq,} or occasionally when the \isi{masdar} suffix \tit{-ni} follows velar \isi{consonants} undergo \isi{palatalization}, i.e. \tit{x} > \tit{š},\tit{ xː } > \tit{šː}, \tit{g} > \tit{ž}, \tit{k} > \tit{č}, \tit{kː } > \tit{č}ː, \tit{k'} > \tit{č'}. For instance, \tit{či-ka-b-ixː-a} (\tsc{spr-down}\tsc{-n-}put\tsc{.pfv-imp.sg}) \sqt{Put it on!} vs. \tit{či-ka-b-išː-ij} (\tsc{spr-down}\tsc{-n-}put\tsc{.pfv-inf}) \sqt{to put it on}; \textit{b-ikː-ar} (\tsc{n}-want.\tsc{ipfv-hab.prs}) vs. \textit{b-ičː-aq-ar} (\tsc{n}-want.\tsc{ipfv-caus-hab.prs}); \tit{b-ebč'-ni} (\tsc{n-}die\tsc{.pfv-msd}) < \tit{b-ebk'-} (\tsc{n-}die\tsc{.pfv}), \tit{b-arč-ni} (\tsc{n-}find\tsc{.pfv-msd}) < \tit{b-arkː-} (\tsc{n-}find\tsc{.pfv}-). There is also \isi{degemination} in the last example. When the \isi{masdar} suffix is added, \isi{palatalization} is optional, at least with some verbs, and downright ungrammatical with others, e.g. \tit{ubč'-ni}\slash\tit{ubk'-ni} (die\tsc{.m.ipfv-msd}) < \tit{b-ubk'-} (\tsc{n-}die\tsc{.ipfv-}); \tit{er-b-ik'ʷ-ni} (look\tsc{-n-}say\tsc{.ipfv-msd}).
\item[7. Gemination and devoicing of voiced stops:] The \isi{gender} affixes \tit{b-} and \tit{d-} become devoiced geminates when preceded or followed by another stop \tit{b} or \tit{d/t} respectively, e.g. \tit{letːe} (< \tit{le-d=de}, exist-\tsc{npl=pst}) vs. \tit{le-r=de} (exist\tsc{-f=pst}).

\end{description}
