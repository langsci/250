\chapter{Non-finite verb forms}
\label{cpt:nonfiniteverbforms}

There are three types of verb forms that function as heads of subordinate clauses:
%
\begin{itemize}
	\item	plain non-finite verb forms (\refsec{sec:Plain non-finite verb forms})
	\item	verb forms functioning as specialized converbs (\refsec{cpt:specializedconverbssubordinatingenclitics})
		\item	conditional and concessive verb forms (\refsec{cpt:conditionalconcessiveclauses})
\end{itemize}

Plain non-finite verb forms are simple converbs, participles, the infinitive, the subjunctive, and the masdar. The specialized converbs convey more specific temporal and causal relationships. Plain non-finite verb forms (except for the subjunctive) and constructions with specialized converbs lack person agreement. Only the plain non-finite verb forms are part of the inflectional paradigm of the verb and thus formed by suffixes. Some of them are also used for the formation of analytic TAM forms. By contrast, verb forms functioning as specialized converbs mostly employ enclitics, which can also be added to other parts of speech than verbs. Conditional and concessive clauses have person agreement expressed by suffixes that strongly resemble the suffixes used in synthetic verb forms of main clauses. They are treated here as non-finite verb forms because their basic use is restricted to dependent clauses that cannot syntactically function as  main clauses. All non-finite verb forms are normally negated by adding the negative prefix \tit{a-}.

Conditional and concessive forms as well as specialized converbs and some of the plain non-finite verb forms occur in adverbial clauses (\refcpt{cpt:Syntactic properties of adverbial and conditional clauses}). Complement clauses (\refcpt{cpt:Complementation}) are mostly headed by plain non-finite forms such as the infinitive and the masdar, and relative clauses are formed with participles (\refcpt{cpt:Relative clauses}). 


%%%%%%%%%%%%%%%%%%%%%%%%%%%%%%%%%%%%%%%%%%%%%%%%%%%%%%%%%%%%%%%%%%%%%%%%%%%%%%%%

\section{Plain non-finite verb forms}
\label{sec:Plain non-finite verb forms}

The following verb forms are considered to be plain non-finite verb forms:
%
\begin{itemize}
	\item	simple converbs (imperfective and perfective) (\refsec{sssec:The imperfective converb} and \refsec{sssec:The perfective converb})
	\item	participles (preterite, modal and functionally related forms with \tit{-il} and \tit{-ce/-te}, and the locative participle) (\refsec{ssec:Participles and functionally related verb forms})
	\item	infinitive (\refsec{ssec:The infinitive})
	\item	subjunctive, i.e., agreeing infinitive (\refsec{ssec:The subjunctive (agreeing infinitive)})
	\item	masdar (\refsec{ssec:The masdar})
\end{itemize}

%


 

\subsection{Simple converbs}
\label{ssec:Simple converbs}
Like all Dargwa languages, Sanzhi has two simple converbs, an imperfective converb (\refsec{sssec:The imperfective converb}) and a perfective converb (\refsec{sssec:The perfective converb}). For the syntax of adverbial clauses in which these converbs occur see \refcpt{cpt:Syntactic properties of adverbial and conditional clauses}.


\subsubsection{The imperfective converb}
\label{sssec:The imperfective converb}

The imperfective converb has the suffix \tit{-ul} (occasionally also \tit{-ule}) or with a few verbs, which have -\textit{un} as the suffix for the perfective converb, \tit{-un(ne)}. For other Dargwa varieties \tit{-ul} has been analyzed as diachronically composed of \tit{-u} (the stem augment that is used for the habitual present) and a converb suffix \tit{-l} \citep{Sumbatova.Mutalov2003, BelyaevInPreparation}, and it is probable that this analysis can be applied to Sanzhi Dargwa as well. The suffix \tit{-unne} is also diachronically complex, consisting of \tit{-un-ne}. The second part -\textit{ne} is an allomorph of the converb suffix -\textit{le} that is also used for the formation of the perfective converb (\refsec{sssec:The perfective converb}) and, more generally, as an adverbializer (\refsec{ssec:The adverbializer -le}). However, synchronically the imperfective converb is not complex anymore and I will therefore gloss it as one single morpheme.

The imperfective converb can only be formed from imperfective stems and from stems of which the aspect is not specified.
%
The functional range of the imperfective converb is as follows:
%
\begin{enumerate}
	\item	formation of the compound present by adding the person enclitics \refex{I am going to sell sunflower seeds.} or the copula (\refsec{ssec:Compound present}) and formation of the compound past by adding the past enclitic (\refsec{ssec:Compound past}). In interrogative clauses the enclitics and copulas can under certain circumstances be omitted and only the interrogative enclitics occur \refex{ex:What is your wife saying, is she not coming} (\refsec{sec:Predicative particles}). There are also a number of other auxiliaries that co-occur with imperfective converbs in periphrastic verb forms (\refcpt{cpt:Periphrastic verb forms}).
	%
	\begin{exe}
	\ex	\label{I am going to sell sunflower seeds.}
	\gll žimiška	d-ic-ij			arg-ul=da\\
	sunflower.seeds \tsc{npl}-sell.\tsc{pfv-inf}	go.\tsc{ipfv-icvb}=1 \\
	\glt	\sqt{I am going to sell sunflower seeds.}

		\ex	\label{ex:What is your wife saying, is she not coming}
		\gll	ala	xːunul ce	r-ik'-ul=e?	saˁ-q'-unne=kːu=w?\\
			\tsc{2sg.gen}	woman what	\tsc{f-}say\tsc{.ipfv-icvb=q}	\tsc{hither}-go-\tsc{icvb=neg=q}\\
		\glt	\sqt{What is your wife saying? Is she not coming?}
	\end{exe}

	\item	formation of adverbial clauses expressing temporal simultaneity or precedence of the event in the adverbial clause with the event expressed in the main clause \refex{ex:We were going there to the goat canyon, but did not arrive there}, \refex{ex:I did not do a difficult work, driving, bringing the workers (to the wine factory) and bringing them back}. This use is very frequent in texts and includes the use in manner clauses that denote the way in which an event is occurring or the manner in which an action is carried out \refex{ex:Then Patima appeared outside running}.
	%
	\begin{exe}
		\ex	\label{ex:We were going there to the goat canyon, but did not arrive there}
		\gll	[q'aca-la	neqːe	hextːu	arg-ul],	či-a-ha-d-eʁ-ib=da\\
			he.goat\tsc{-gen}	canyon	there.\tsc{up}	go\tsc{.ipfv-icvb}	\tsc{spr-neg-up}\tsc{-1/2pl-}go\tsc{.pfv-pret=1}\\
		\glt	\sqt{We were going there to the goat canyon, but did not arrive there.}
	
		\ex	\label{ex:I did not do a difficult work, driving, bringing the workers (to the wine factory) and bringing them back}
		\gll	guž	ʡaˁči	a-b-arq'-ib=da,	[w-ax-ul],	[raboči	b-ik-ul]	[sa-b-ik-ul]\\
			difficult	work	\tsc{neg-n-}do\tsc{.pfv-pret=1}	\tsc{m-}go\tsc{-icvb} workers	\tsc{hpl-}lead\tsc{.ipfv-icvb}	\tsc{hither}\tsc{-hpl-}lead\tsc{.ipfv-icvb}\\
		\glt	\sqt{I did not do difficult work, driving, bringing the workers (to the wine factory) and bringing them back.}
	
		\ex	\label{ex:Then Patima appeared outside running}
		\gll	c'il	hel-ka	[duc'	r-ik'-ul]	tːura-r-ič-ib	ca-r	Pat'ima\\
			then	that\tsc{-abl}	run	\tsc{f-}move\tsc{.ipfv-icvb}	\tsc{out}\tsc{-f-}occur\tsc{.pfv-pret}	be\tsc{-f}	Patima\\
		\glt	\sqt{Then Patima appeared outside running. (i.e. ran outside)}
	\end{exe}
	
There are plenty of examples in which it is not easy or even impossible to unambiguously identify the main clause to which the converbal clause belongs such that it may seem that the converb can head independent main clauses. In fact, such usages have been reported for Mehweb Dargwa \citep{Kustova2015} (see also \citet{Mithun2008} for a more general account of how and why non-finite verb forms develop into finite verb forms and the use of dependent clauses as independent sentences). However, in elicitation converb clauses are always judged as dependent clauses that need to be related to a superordinate clause in order to form a grammatical sentence. Thus, converb clauses that seem to occur on their own in independent utterances can probably be treated as a feature of colloquial language. In \refex{ex:‎There was bread one for everyone, whatever} only the first clause is morphosyntactically unambiguously a an independent main clause followed by two expressions that indicate the lack of knowledge of the speaker (\textit{ce ca-d=de=l, aχːu}) and two clauses with imperfective converbs without accompanying main clauses.

For instance, the utterance in \refex{ex:‎‎hitting with the crook, disturbing, beating up} is part of a characterization of a person, but there is no preceding or following main clause that could serve as a syntactic anchor for the adverbial clause.
	%
	\begin{exe}
		\ex	\label{ex:‎There was bread one for everyone, whatever}
		\gll	ca	ca	t'ult'=de=w?	ce	ca-d=de=l,	aχːu,	[ču-la	le-b-il	sa-b-iqː-ul]	[lukː-unne]\\
			one	one	bread\tsc{=pst=q}	what	be\tsc{-npl=pst=prt}	not.know	\tsc{refl.pl-gen}	exist\tsc{-n-ref}	\tsc{hither}\tsc{-n-}carry\tsc{.ipfv-icvb}	give\tsc{.ipfv-icvb}\\
		\glt	\sqt{‎There was one bread for everyone? Whatever, I don't know, what they had they were bringing (to the soldiers) and giving it to them.}
	
		\ex	\label{ex:‎‎hitting with the crook, disturbing, beating up}
		\gll	[paˁq	Ø-ik'-ul	q'isːa-l-cːella]	[ʁina.ʁina	b-irq'-ul],	[it-ul], \ldots\\
			strike	\tsc{m-}say\tsc{.ipfv-icvb}	crook\tsc{-obl-comit}	spoil	\tsc{hpl-}do\tsc{.ipfv-icvb}		beat.up\tsc{-icvb}\\
		\glt	\sqt{‎‎hitting with the crook, disturbing, beating up, \ldots}
	\end{exe}

	\item	formation of certain complement clauses, for instance with the verb \sqt{begin} (\refsec{ssec:The imperfective converb}), i.e. the verbal head in the complement clause bears the imperfective converb suffix.

	\item	The imperfective converb of the verb \tit{b-ik'ʷ-} \sqt{say} is used as a quotation marker and, more generally, as a marker of certain complement clauses (\refsec{ssec:The quotative particles}).

\end{enumerate}

% - - - - - - - - - - - - - - - - - - - - - - - - - - - - - - - - - - - - - - - - - - - - - - - - - - - - - - - - - - - - - - - - - - - - - - - - - - - - - - - - - - - - - - - - - - - - - - - - - - - - - - - - - - - - - - - - - - - - - - - - - - %

\subsubsection{The perfective converb}
\label{sssec:The perfective converb}

The perfective converb is formed by adding the suffix \tit{-le} to the preterite participle. The resulting complex suffixes are \tit{-ib-le, -ub-le, -un-ne,} and \tit{-ur-re} (or \tit{-ur-le}), and their distribution is lexicalized (see the tables in \refsec{sec:The structure underived verbal stems} for many example verbs). The suffix -\textit{le} is a cross-categorical suffix that forms adverbials from various parts of speech (\refsec{ssec:The adverbializer -le}). The perfective converb is mainly built from perfective verb stems. Thus, what is treated in this grammar under the label `perfective converb' is a conflation of several components that come with their own properties. It is therefore not ideal to gloss only the suffix -\textit{le} as perfective converb (\tsc{cvb}) because this suffix only contributes to the morphosyntax, but not to the semantics. The meaning of the perfective converb originates from the combination of a (usually) perfective verb stem with the preterite suffix.

The functions of the perfective converb are:
%
\begin{enumerate}
	\item	Formation of analytic verb forms: resultative (\refsec{ssec:The (perfective) resultative}), perfect (\refsec{ssec:The perfect}) and past perfect (\refsec{ssec:The past perfect (pluperfect)}). Furthermore, there are periphrastic verb forms with other auxiliaries that make use of the perfective converb (\refcpt{cpt:Periphrastic verb forms}).

	\item	Formation of temporal adverbial clauses: The adverbial clauses refer to situations that take place before the situation expressed in the main clause or simultaneously with it \refex{ex:‎He got happy, caught his frog, and wants to go home}, \refex{ex:‎He went and complained}. Occasionally, the latter type of converb clauses, which express simultaneously occurring events, are semantically manner clauses \refex{ex:‎He himself run away}. It is common to have sequences of adverbial clauses containing perfective converbs that denote a sequence of events \refex{ex:‎He got happy, caught his frog, and wants to go home}. As with the imperfective converb, it is not always easy to find an adjacent main clause that is the syntactic anchor for perfective converb clauses. For instance, in \refex{ex:we plaster the house, hang the wallpaper, put the windows; the boards were already prepared} the copula clause at the end refers to the same stretch of events that the preceding converbal clauses refer to, namely the building of a house. However, the converb clauses and the copula clauses do not share any arguments. The sharing of arguments is not a syntactic requirement for the use of perfective converbs, but as the preceding examples \refex{ex:‎He got happy, caught his frog, and wants to go home}-\refex{ex:‎He himself run away} show it is very common (see also \refsec{Co-reference and expression of shared arguments} for more information about the syntactic properties of converbal clauses with respect to argument sharing).
	%
	\begin{exe}
		\ex	\label{ex:‎He got happy, caught his frog, and wants to go home}
		\gll	[razi	Ø-iχ-ub-le],	[b-uc-ib-le	ʡaˁt'a=ra	ca-w=ra],	Ø-uq'-ij	b-ikː-ul	ca-b	qili\\
			happy	\tsc{m-}be\tsc{.pfv-pret-cvb}	\tsc{n-}catch\tsc{.pfv-pret-cvb}	frog\tsc{=add}	\tsc{refl-m=add}	\tsc{m-}go\tsc{-inf}	\tsc{n-}want\tsc{.ipfv-icvb}	be\tsc{-n}	home\\
		\glt	\sqt{‎He got happy, caught his frog, and wants to go home.}

		\ex	\label{ex:‎He went and complained}
		\gll	[iž	ag-ur-re]	ʡaˁrz	w-arq'-ib	ca-w\\
			this	go\tsc{.pfv-pret-cvb}	complain	\tsc{m-}do\tsc{.pfv-pret}	be\tsc{-m}\\
		\glt	\sqt{‎He went away and complained.}

		\ex	\label{ex:‎He himself run away}
		\gll	ca-w	duc'	Ø-uq-un-ne	ag-ur	hel\\
			\tsc{refl-m}	run	\tsc{m-}go\tsc{.pfv-pret-cvb}	go\tsc{.pfv-pret}	that\\
		\glt	\sqt{‎He himself run away.}

		\ex	\label{ex:we plaster the house, hang the wallpaper, put the windows; the boards were already prepared}
		\gll	[či-d-aˁq-aˁq-ib-le	qul-be]	[abujta-la	d-asː-aq-un-ne]	[ʡaˁm-re	sa-ka-d-icː-ur-re] iltːi	urq'l-e	ħaˁdur-re	le-d=de\\
			\tsc{spr-npl-}hit\tsc{.pfv-caus-pret-cvb}	house\tsc{-pl}	wall.paper\tsc{-gen}	\tsc{npl-}glue\tsc{.pfv-caus-pret-cvb}	window\tsc{-pl}	\tsc{ante-down}\tsc{-npl-}stand\tsc{.pfv-pret-cvb}	\tsc{3pl}	board\tsc{-pl}	ready\tsc{-advz}	exist\tsc{-npl=pst}\\
		\glt	\sqt{We plastered the house, hung the wallpaper, put the windows; the boards were already prepared.}
	\end{exe}

	\item	Formation of complement clauses: the perfective converb occurs in a range of complement clauses of the fact and of the activity type with cognition, evaluation and emotion predicates and with the verb \sqt{finish} in the matrix clause \refex{ex:‎I remember that he came riding on a horsePFVCVB} (\refsec{ssec:The preterite converb}).
	
\begin{exe}
	\ex	\label{ex:‎I remember that he came riding on a horsePFVCVB}
	\gll	[ca-w	urči-j	murtːa-l	ha-jʁ-ib-le]	han	le-w\\
		\tsc{refl-m}	horse\tsc{-dat}	rider\tsc{-advz}	\tsc{up}-come\tsc{.m.pfv-pret-cvb}	remember	exist\tsc{-m}\\
	\glt	\sqt{(‎I) remember that he came riding on a horse.}	
		\end{exe}
	
\end{enumerate}


% --------------------------------------------------------------------------------------------------------------------------------------------------------------------------------------------------------------------- %

\subsection{Participles}
\label{ssec:Participles and functionally related verb forms}

Participles and functionally related forms occur in relative clauses and partially also in other constructions. Sanzhi has three participle: (i) the preterite participle (\refsec{sssec:The preterite participle}), (ii) the modal participle (\refsec{sssec:The modal participle -an}), and (iii) the locative participle (\refsec{sssec:The locative participle}). To the first two participles the cross-cateogorical suffixes -\textit{ce} and -\textit{il} can be added (\refsec{sssec:The attributive markers -il and -ce / -te in combination with the participles}). For the syntactic properties of relative clauses see \refcpt{cpt:Relative clauses}.


% - - - - - - - - - - - - - - - - - - - - - - - - - - - - - - - - - - - - - - - - - - - - - - - - - - - - - - - - - - - - - - - - - - - - - - - - - - - - - - - - - - - - - - - - - - - - - - - - - - - - - - - - - - - - - - - - - - - - - - - - - - %

\subsubsection{The preterite participle}
\label{sssec:The preterite participle}

The preterite is used for a range of verb forms with past time reference that are used in main clauses. This includes the preterite itself (\refsec{ssec:The preterite}) but also many more analytic (\refsec{sec:Forms based on the preterite}) and periphrastic verb forms based on it (\refcpt{cpt:Periphrastic verb forms}). The preterite is also employed in relative clauses (\refcpt{cpt:Relative clauses}). Formally we deal with one and the same suffix, and I will therefore use only one single gloss for it (\tsc{pret}), although functionally and with respect to morphosyntactic properties the finite verb form `preterite' differs from the participle. The finite verb form is used together with person enclitics, the past enclitic or the copula, which, by contrast, is impossible for the participle in a relative clause. The participle, in turn, attaches further nominalizing suffixes \refex{ex:Give me the one that is hanging} and can then be case marked (\refsec{sec:Headless relative clauses}). Relative clauses that are formed with the preterite participle obligatorily have a nominal head \refex{ex:These are clothes that have been hung up there}, \refex{ex:‎He is remembering his gone by, past life}. 

%
\begin{exe}
	\ex	\label{ex:These are clothes that have been hung up there}
	\gll	hex-tːi	[sa-r-ha-aq-ib]	paltar	ca-d\\
		\tsc{dem.up}\tsc{-pl}	\tsc{ante-abl-up}-hang\tsc{.pfv-pret}	clothes	be\tsc{-npl}\\
	\glt	\sqt{These are clothes that have been hung up there.}

	\ex	\label{ex:‎He is remembering his gone by, past life}
	\gll	han	d-irč-aq-ul	ca-d	cin-na	[b-it-ag-ur],	[ag-ur]	ʡuˁnru\\
		remember	\tsc{npl-}occur\tsc{.ipfv-caus-icvb}	be\tsc{-npl}	\tsc{refl.sg-gen}	\tsc{n-thither}-go\tsc{.pfv-pret}	go\tsc{.pfv-pret}	life\\
	\glt	\sqt{‎He is remembering his gone by, past life.}
\end{exe}

The preterite participle cannot directly take case suffixes or similar grammatical markers used with nominals \refex{ex:ungrammaticalGive me the one that is hanging}. In order to nominalize the preterite participle, one of the cross-categorical participles \tit{-il} \refex{ex:Give me the one that is hanging} or \tit{-ce} needs to be added (see \refsec{sssec:The attributive markers -il and -ce / -te in combination with the participles} below for more details).
%
\begin{exe}
	\ex	\label{ex:ungrammaticalGive me the one that is hanging}
	\gll	*dam	b-iqː-a	sa-r-ha-aq-ib!\\
		\hphantom{*}\tsc{1sg.dat}		\tsc{n-}take\tsc{.ipfv-imp}	\tsc{ante-abl-up}-hang\tsc{.pfv-pret}\\
	\glt	(Intended meaning: \sqt{Give me the one that is hanging!})

	\ex	\label{ex:Give me the one that is hanging}
	\gll	dam	b-iqː-a	sa-r-ha-aq-ib-il!\\
		\tsc{1sg.dat}		\tsc{n-}take\tsc{.ipfv-imp}	\tsc{ante-abl-up}-hang\tsc{.pfv-pret-ref}\\
	\glt	\sqt{Give me the one that is hanging!} (E)
\end{exe}

The preterite participle also attaches a number of temporal enclitics, suffixes and other subordinating enclitics such as \tit{-er} \sqt{when},  \tit{-la} \sqt{since, after} \refex{ex:After that had happened their friendship did not finish.}, \tit{=qːel(la)} \sqt{when, because} and \tit{=xːar} \sqt{although} and is then used in adverbial clauses with various specialized converbs (\refsec{cpt:specializedconverbssubordinatingenclitics}).


\begin{exe}
	\ex	\label{ex:After that had happened their friendship did not finish.}
	\gll	il	ag-ur-ra	hitːi=ra	ʡaˁχuˁl-dex	taman	a-b-iχ-ub\\
		that	go.\tsc{pfv-pret-post}	after=\tsc{add} guest-\tsc{nmlz}	end	\tsc{neg-b}-happen.\tsc{pfv-pret}\\
	\glt	\sqt{After that had happened their friendship did not finish.}
\end{exe}


% - - - - - - - - - - - - - - - - - - - - - - - - - - - - - - - - - - - - - - - - - - - - - - - - - - - - - - - - - - - - - - - - - - - - - - - - - - - - - - - - - - - - - - - - - - - - - - - - - - - - - - - - - - - - - - - - - - - - - - - - - - %

\subsubsection{The modal participle \tit{-an}}
\label{sssec:The modal participle -an}

The modal participle \tit{-an} is only added to imperfective stems. Its semantics covers modality (obligation, deontic necessity) and future time reference. However, in relative clauses the modal meaning is often absent. The modal participle is used for the formation of a range of finite analytic verb forms, namely future (\refsec{ssec:Future analytic}), future in the past (\refsec{ssec:Future in the past}), obligative (\refsec{ssec:Obligative}), obligative present (\refsec{ssec:Obligative present}), and obligative past (\refsec{ssec:Obligative past}). The second functional domain of the modal participle is the formation of relative clauses. They mostly have habitual semantics and refer to stable properties of the referent of the head noun \refex{ex:We are people who were working}-\refex{ex:‎‎‎For the one who wants to quarrel there is always an option}. Thus, there are some participles that have been lexicalized into adjectives expressing characteristic properties, e.g. \tit{b-uz b-ik'ʷ-an barcːik'ʷ} (\tsc{n-}tear \tsc{n-aux.ipfv-ptcp} chudu) \sqt{chudu filled with cheese that can be expanded and stretched when it is melted}, \textit{dircan} `trader, seller' \refex{ex:The one who is seated is probably selling (stuff)}.
	%
	\begin{exe}
		\ex	\label{ex:We are people who were working}
		\gll	nušːa	χalq',	[ʡaˁči-l	d-irq'-an]	χalq'=de=q'al\\
			\tsc{1pl}	people	work\tsc{-erg}	\tsc{1/2pl-}do\tsc{.ipfv-ptcp}	people\tsc{=pst=mod}\\
		\glt	\sqt{We are people, people who work.} (i.e. we are worth to be respected)

		\ex	\label{ex:the boy called Mahammadhazhi}
		\gll	hel	[Maħaˁmmadħaˁži	b-ik'ʷ-an]	durħuˁ\\
			that	Mahammadhazhi	\tsc{hpl-}say\tsc{.ipfv-ptcp}	boy	\\
		\glt	\sqt{the boy called Mahammadhazhi}

		\ex	\label{ex:on the market seated is a trader (a person who sells)}
		\gll	bazar-re	hej	ka-jž-ib-le	[w-irc-an]	admi	ca-w\\
			market\tsc{-loc}	this	\tsc{down}-be\tsc{.m.pfv-pret-cvb}	\tsc{m-}sell\tsc{.ipfv-ptcp}	person	be\tsc{-m}\\
		\glt	\sqt{On the market a trader is sitting.} (lit. a person who sells)
	\end{exe}

The relative clauses with the participle -\textit{an} can be headless \refex{ex:The one who is seated is probably selling (stuff)}-\refex{ex:‎‎‎For the one who wants to quarrel there is always an option}.Tthe participle can take further case suffixes \refex{ex:‎‎‎For the one who wants to quarrel there is always an option}.
	%
	\begin{exe}
		\ex	\label{ex:The one who is seated is probably selling (stuff)}
		\gll	c'il	hež	ka-jž-ib-il	d-irc-an	Ø-iχʷ-ij\\
			then	this	\tsc{down}-remain\tsc{.m.pfv-pret-ref}	\tsc{npl-}sell\tsc{.ipfv-ptcp}	\tsc{m-}be\tsc{.pfv-inf}\\
		\glt	\sqt{The one who is seated is probably selling (stuff).}

		\ex	\label{ex:‎‎It is like this, my beloved one}
		\gll	hel-tːi	ʁunab-te	ca-d,	di-la	w-ikː-an\\
			that\tsc{-pl}	\tsc{eq-dd.pl} 	be\tsc{-npl}	\tsc{1sg-gen}	\tsc{m-}want\tsc{.ipfv-ptcp}\\
		\glt	\sqt{‎‎It is like this, my beloved one.}

		\ex	\label{ex:‎‎‎For the one who wants to quarrel there is always an option}
		\gll	il	ʁaj	Ø-ik'ʷ-ij	b-ikː-an-ni-j=ra	har	zamana	hana=ra	dune	le-b\\
			that	word	\tsc{m-}say\tsc{.ipfv-inf}	\tsc{n-}want\tsc{.ipfv-ptcp-obl-dat=add}	every	time	now\tsc{=add}	world	exist\tsc{-n}\\
		\glt	\sqt{‎‎‎For the one who wants to quarrel there is always an option.} (lit. `a/the world')
	\end{exe}

When the modal participle is followed by the cross-categorical suffix \tit{-ce} it also occurs in complement clauses \refex{ex:‎‎I know that he will come.PTCPAn} (\refsec{ssec:The attributive marker -ce (-te)}), in addition to the possible use in headless relative clauses (\refsec{sec:Headless relative clauses}). The second cross-categorical suffix \tit{-il} also often co-occurs with the modal participle in various relative clauses. 


\begin{exe}
	\ex	\label{ex:‎‎I know that he will come.PTCPAn}
	\gll	dam	b-alχ-a-d				[it	s-erʁ-an-ce]\\
		\tsc{1sg.dat}	\tsc{n}-know.\tsc{ipfv-hab.pst-1}	that	\tsc{hither}-come.\tsc{ipfv-ptcp-dd.sg}\\
	\glt	\sqt{‎‎I know that he will come.} (E)
\end{exe}


Finally, a number of temporal enclitics and other subordinating enclitics such as \tit{-er} \sqt{when} \refex{ex:Then my husband and I were sitting and when we were eating we met four men.}, \tit{=qːel(la)} \sqt{when, because} and \tit{=xːar} \sqt{although} attach to the modal participle yielding adverbial clauses (\refsec{cpt:specializedconverbssubordinatingenclitics}).


\begin{exe}
	\ex	\label{ex:Then my husband and I were sitting and when we were eating we met four men.}
	\gll	c'il	di-la		sub=ra		du=ra		ag-ur-re			ka-d-iž-ib-le,		d-uk-an-er,	suk	b-ič-ib			aʁʷal	admi\\
		then	\tsc{1sg-gen}	husband=\tsc{add}	\tsc{1sg=add}	go.\tsc{pfv-pret-cvb}	\tsc{down-1/2pl}-be.\tsc{pfv-pret-cvb}	\tsc{1/2pl}-eat.\tsc{ipfv-ptcp}-when		meet	\tsc{hpl}-occur.\tsc{pfv-pret}	four	person\\
	\glt	\sqt{Then my husband and I were sitting and when we were eating we met four men.}
\end{exe}



% - - - - - - - - - - - - - - - - - - - - - - - - - - - - - - - - - - - - - - - - - - - - - - - - - - - - - - - - - - - - - - - - - - - - - - - - - - - - - - - - - - - - - - - - - - - - - - - - - - - - - - - - - - - - - - - - - - - - - - - - - - %

\subsubsection{The cross-categorical suffixes \tit{-il} and \tit{-ce}\slash\tit{-te} in combination with the participles}
\label{sssec:The attributive markers -il and -ce / -te in combination with the participles}

The preterite and the modal participle can combine with both types of cross-categorical suffixes, \tit{-il} and \tit{-ce} (\tit{-te} in the plural). The general function of these suffixes can be described as the formation of referential attributes or definite descriptions that have the morphosyntactic properties of nominals (\refsec{ssec:The -ce / -te attributive} and \refsec{ssec:The -il attributive}). When the suffixes are added to the participles we can form relative clause with heads and headless relative clauses. Two participles and two types of cross-categorical suffixes yield four possible combinations that are not all equally common. There seem to be no semantic differences between the two cross-categorical suffixes when occurring in headless relative clauses. But there is a morphosyntactic difference: the suffix \tit{-il} is only used with referents that are not morphologically overtly marked for plural, i.e., the relative clause needs to refer to a singular object or a mass noun such as \tit{χalq'} \sqt{people} or \tit{sungul} \sqt{the community of the Sanzhi people} (even though both nouns control human plural agreement) or something similar as in \refex{ex:the half (of the people) who were going with me}. The referent can be overtly expressed (relative clause with a head) or not (headless relative clause). For overtly marked plural referents or for headless relative clauses denoting a plurality of referents only \tit{-te} can be used \refex{ex:the Sanzhi people who are sitting and drinking}-\refex{ex:the half (of the people) who were going with me}.
%
\begin{exe}
	\ex	\label{ex:the Sanzhi people who are sitting and drinking}
	\gll	b-učː-ul	*ka-b-iž-ib-il	/	ka-b-iž-ib-te	sunglan-te\\
		\tsc{hpl-}drink\tsc{.ipfv-icvb}	\tsc{down-hpl-}be\tsc{.pfv-pret-ref} 	/	\tsc{down-hpl-}be\tsc{.pfv-pret-dd.pl}	Sanzhi.person\tsc{-pl}\\
	\glt	\sqt{the Sanzhi people who are sitting and drinking} (E)

	\ex	\label{ex:the ones who are working PTCP}
	\gll	ʡaˁči-l	b-irq'-an-te	/	*b-irq'-an-il\\
		work\tsc{-erg}	\tsc{hpl-}do\tsc{.ipfv-ptcp-dd.pl}	/	\tsc{hpl-}do\tsc{.ipfv-ptcp-ref}\\
	\glt	\sqt{the ones who are working} (E)
	
		\ex	\label{ex:the half (of the people) who were going with me}
	\gll	il-tːi	[di-cːella	b-alli	b-ax-an-te]	b-abq'i\\
		that-\tsc{pl}	\tsc{1sg-comit}	\tsc{hpl-}together	\tsc{hpl-}go\tsc{-ptcp-dd.pl} 	\tsc{hpl-}half\\
	\glt	\sqt{the half (of the people) who were going with me}
\end{exe}


By far most common in the Sanzhi corpus are headless relative clauses in which the verb bears the preterite participle suffix plus the suffix \tit{-il} \refex{ex:‎the one where they are sitting and drinking, I put this (picture) somewhere}. When the dative case is used, the meaning of the nominalized relative clause can be causal (due to the semantics of the dative case) such that these clauses rather function as adverbial clauses expressing cause or reason \refex{ex:He became sad, in my opinion, he got very sad, because of what he did minor}.
%
\begin{exe}
	\ex	\label{ex:‎the one where they are sitting and drinking, I put this (picture) somewhere}
	\gll	[b-učː-ul	ka-b-iž-ib-il]	ka-b-išː-ib=da	heltːu	čina-del\\
		\tsc{hpl-}drink\tsc{.ipfv-icvb}	\tsc{down-hpl-}be\tsc{.pfv-pret-ref}	\tsc{down-n-}put\tsc{.pfv-pret=1}	there	where\tsc{-indef}\\
	\glt	\sqt{The one where they are sitting and drinking, I put this (picture) somewhere.}
\end{exe}
%
\begin{exe}	
	\ex	\label{ex:He became sad, in my opinion, he got very sad, because of what he did minor}
	\gll	hej	pašman	Ø-iχ-ub	ca-w	ʡaˁħ-le	hel	b-arq'-ib-il-li-j	\\
		this	sad	\tsc{m-}be\tsc{.pfv-pret}	be\tsc{-m}	good\tsc{-advz}	that	\tsc{n-}do\tsc{.pfv-pret-ref-obl-dat}\\
	\glt	\sqt{‎He got very (lit. well) sad, because of what he had done.}

\end{exe}

The preterite participle of the verb \sqt{say} to which \tit{-il} is added and which is used without the spatial preverb \textit{ha}- (i.e. \tit{ʔ-ib-il}, also written as \tit{ibil}\footnote{The usual participial form of this verb is \tit{haʔ-ib-il} with the spatial preverb.}) functions as marker for ordinal numerals \refex{ex:in the year (19)59} (\refsec{sec:ordinalnumerals}).
%
\begin{exe}
	\ex	\label{ex:in the year (19)59}
	\gll	xu-c'anu	urč'em-ra	ibil	dusːi-cːe-w\\
		five-\tsc{ten}	nine\tsc{-num}	\tsc{ord}	year\tsc{.obl-in-m}\\
	\glt	\sqt{in the year (19)59}
\end{exe}

The co-occurrence of both participles (preterite and modal) with \tit{-ce}\slash\tit{-te} is also common and it is easy to find examples with \refex{ex:‎‎‎The ones that gathered (the food) probably divided it among themselves} and without case suffixes \refex{ex:‎‎Foolish Sanzhi people, crazy Sаnzhi people, at night they eat three times, and during day they stay hungry they said}. There is a clear difference in meaning that is due to the participles and the aspectual properties of the verb stems. In \refex{} the preterite participle is used to refer to people who accomplished an action in the past (i.e. they gathered). The modal participle in \refex{} expresses refers to the Sanzhi people by means of a stative characterization as the ones who eat three times at night and stay hungry during the day, i.e., fulfilling the duties of Muslims during the month of Ramadan.
%
\begin{exe}
	\ex	\label{ex:‎‎‎The ones that gathered (the food) probably divided it among themselves}
	\gll	d-erč-ib-t-a-l	ču-j	d-ut'-ib	d-urkː-ar\\
		\tsc{npl-}collect\tsc{.pfv-pret-pl-obl-erg}	\tsc{refl.pl-dat}	\tsc{npl-}divide\tsc{-pret}	\tsc{npl-}find\tsc{.ipfv-prs}\\
	\glt	\sqt{‎‎‎The ones that gathered (the food) probably divided it among themselves.}

	\ex	\label{ex:‎‎Foolish Sanzhi people, crazy Sаnzhi people, at night they eat three times, and during day they stay hungry they said}
	\gll	``abdal	sungul,		baˁħ	sungul,		dučːe		ʡaˁj-na	b-uk-an-te, 					ari 		kːuš-le		b-ug-an-te'' 							b-ik'-ul\\
		fool	Sanzhi.people	crazy	Sanzhi.people	at.night	three-\tsc{time}	\tsc{hpl-}eat\tsc{.ipfv-ptcp-dd.pl} 	during.day	hungry\tsc{-adv}	\tsc{hpl-}stay\tsc{.ipfv-ptcp-dd.pl} 	\tsc{hpl-}say\tsc{.ipfv}\tsc{-icvb}\\
	\glt	\sqt{``‎‎Foolish Sanzhi people, crazy Sаnzhi people, at night they eat three times, and during day they stay hungry,'' they said.}
\end{exe}


The combination of the modal participle with \tit{-il} is not particularly frequent; it occurs mostly together with case suffixes as in \refex{ex:‎now for the one who operates her 5000, 50000 rubles}. 

\begin{exe}
	\ex	\label{ex:‎now for the one who operates her 5000, 50000 rubles}
	\gll	hana	cek'u	r-irq'-an-il-li-j	xujal	azir	xu-c'al	azir\\
		now	whatchamacallit	\tsc{f-}do\tsc{.ipfv-ptcp-ref-obl-dat}	five	thousand	five-\tsc{ten}	thousand\\
	\glt	\sqt{‎now for the one who operates her 5000, 50000 rubles (need to be given)}
\end{exe}

Finally, the suffix \tit{-il} can also be added to the existential copulas, which do not inflect for any of the participles, in order to form headed and headless relative clauses \refex{ex:‎There was bread one for everyone, whatever}. In \refex{ex:‎My dear, may on the place where you (fem.) are rain the rain of happiness} the existential copula with its suffix -\textit{il} is inflected for a spatial case.
%
\begin{exe}
	\ex	\label{ex:‎My dear, may on the place where you (fem.) are rain the rain of happiness}
	\gll	durqa-ce,	u	le-r-il-le	taliħ-la	marka	b-arq'-ab\\
		dear\tsc{-dd.sg}	\tsc{2sg}	exist\tsc{-f-ref-loc}	happiness\tsc{-gen}	rain	\tsc{n-}do\tsc{.pfv-opt.3}\\
	\glt	\sqt{(‎My) dear, may onto the place where you (fem.) are rain the rain of happiness.}
\end{exe}

% - - - - - - - - - - - - - - - - - - - - - - - - - - - - - - - - - - - - - - - - - - - - - - - - - - - - - - - - - - - - - - - - - - - - - - - - - - - - - - - - - - - - - - - - - - - - - - - - - - - - - - - - - - - - - - - - - - - - - - - - - - %

\subsubsection{The locative participle \textit{-na}}
\label{sssec:The locative participle}

The locative participle has the suffix \tit{-an}. It is only available for imperfective verb stems. It has a spatial meaning that corresponds to the semantics of the lative case (\refsec{ssec:Functions of semantic cases}). Similar to spatial adverbs it can take further spatial case suffixes, i.e. the essive and the ablative. It most commonly functions as the head of spatial relative clauses \refex{ex:‎He wanted to go his way (where he is going)}, \refex{ex:‎‎I did not go through where the people (normally) go, but through where the hill is}, but it is also possible to add a head noun \refex{ex:‎‎‎The boys went to the place where cows are slaughtered}.
%
\begin{exe}
	\ex	\label{ex:‎He wanted to go his way (where he is going)}
	\gll	Ø-uq'-ij	b-ikː-ul=de	il-i-j	[cin-na	w-ax-na]\\
		\tsc{m-}go\tsc{-inf}	\tsc{n-}want\tsc{.ipfv-icvb=pst}	that\tsc{-obl-dat}	\tsc{refl.sg-gen}	\tsc{m-}go\tsc{-ptcp.loc}\\
	\glt	\sqt{‎He wanted to go his way.} (lit. where he goes)

	\ex	\label{ex:‎‎I did not go through where the people (normally) go, but through where the hill is}
	\gll	het	[χalq'	b-ax-na-r]	a-ag-ur-re,	c'il	tum-la	hetːu-r	ag-ur=da\\
		that	people	\tsc{hpl-}go\tsc{-ptcp.loc-abl}	\tsc{neg-}go\tsc{.pfv-pret-cvb}	then	hill-\tsc{gen}	there\tsc{-abl}	go\tsc{.pfv-pret=1}\\
	\glt	\sqt{‎‎I did not go through where the people (normally) go, but through where the hill is.}

	\ex	\label{ex:‎‎‎The boys went to the place where cows are slaughtered}
	\gll	durħ-ne	ag-ur	[q'ʷal	luχ-na]	musːa\\
		boy\tsc{-pl}	go\tsc{.pfv-pret}	cow	cut\tsc{.ipfv-ptcp.loc}	place.\tsc{loc}\\
	\glt	\sqt{‎‎‎The boys went to the place where cows are slaughtered.} (this refers to a specific place in Sanzhi)
\end{exe}

In addition, the locative participle can be fully case marked. In order to add case suffixes (other than the suffixes for the essive and the ablative) the participle appears in its oblique forms, just like any other nominal. After suffixing the oblique marker \tit{-l} (which is identical to the ergative), case suffixes follow \refex{ex:‎‎‎Where Napisat was sleeping there was a snake}. But as \refex{ex:‎‎I did not go through where the people (normally) go, but through where the hill is} and the second variant in \refex{ex:‎‎‎Where Napisat was sleeping there was a snak} show, is is also allowed to directly suffix markers that express the spatial cases essive, lative and ablative, because the locative participle has inherent spatial meaning. In elicitation, the suffixation of other than spatial cases leads to a broader variety of relative clauses \refex{ex:‎ ‎‎There was the gossip that Maja was lying}. The case-marked participle also occurs in adverbial clauses with causative semantics \refex{ex:ecause Maja was sleeping I (masc.) could not come.}.
%
\begin{exe}
	\ex	\label{ex:‎‎‎Where Napisat was sleeping there was a snake}
	\gll	[Napisat	ka-r-ils-na-l-le-b	/	ka-r-ils-na-b]	te-b=de	maˁlʡuˁn\\
		Napisat	\tsc{down-f-}sleep\tsc{.ipfv-ptcp.loc-obl-loc-n}	/ \tsc{down-f-}sleep\tsc{.ipfv-ptcp.loc-n}	exist\tsc{.away}-\tsc{n=pst}	snake\\
	\glt	\sqt{‎‎‎Where Napisat was sleeping there was a snake.} (E)

	\ex	\label{ex:‎ ‎‎There was the gossip that Maja was lying}
	\gll	[Maˁʡaˁ	ka-r-ils-na-l-la]	χabar	le-b=de\\
		Maja	\tsc{down-f-}sleep\tsc{.ipfv-ptcp.loc-obl-gen}	story	exist\tsc{-n=pst}\\
	\glt	\sqt{‎‎‎There was the gossip that Maja was lying (in the hospital).} (E)
\end{exe}


% --------------------------------------------------------------------------------------------------------------------------------------------------------------------------------------------------------------------- %

\subsection{The infinitive}
\label{ssec:The infinitive}

The suffix for the infinitive is \tit{-ij}. It is very likely that the final \tit{j} diachronically goes back to the dative case, and the formal identity of infinite and dative case is also attested in other Dargwa languages (e.g. Tanti, Icari) and other East Caucasian languages such as Hinuq. In principle, the infinitive can be formed from imperfective and perfective stems \refex{ex:infinitiveexamples} but in natural texts it is almost exclusively used with perfective stems. There are a number of imperfective verbs for which the infinitive is at least very marginal if not ungrammatical (see the last two examples in \refex{ex:infinitiveexamples}).
%
\begin{exe}
	\ex	\label{ex:infinitiveexamples}
	\begin{xlist}
		\TabPositions{8em,9em,17em}
		\ex	\tit{ha-b-ilq'-ij} (\tsc{ipfv}) 	\tab	/	\tab 	\tit{habiq'ij} (\tsc{pfv}) 		\tab	\sqt{raise, keep up}
		\ex	\tit{lukː-ij} (\tsc{ipfv}) 		\tab	/	\tab  	\tit{b-ičː-ij} (\tsc{pfv}) 		\tab	\sqt{give}
		\ex	\tit{či-b-iž-ij} (\tsc{ipfv}) 		\tab	/	\tab  	\tit{či-b-až-ij} (\tsc{pfv}) 		\tab	\sqt{see}
		\ex	\tit{b-alχ-ij} (\tsc{ipfv}) 		\tab	/	\tab  	\tit{b-aχ-ij} (\tsc{pfv}) 		\tab	\sqt{know}
		\ex	\tit{?b-irc-ij} (\tsc{ipfv}) 		\tab	/	\tab 	\tit{b-ic-ij} (\tsc{pfv}) 		\tab	\sqt{sell}
		\ex	\tit{\#b-ubč'-ij} (\tsc{ipfv}) 	\tab	/	\tab 	\tit{b-ebč'-ij} (\tsc{pfv}) 		\tab	\sqt{die}
	\end{xlist}
\end{exe}

The functions of the bare infinitive are:
%
\begin{enumerate}
	\item	formation of purpose clauses 
	%
	\begin{exe}
		\ex	\label{ex:‎We put it (on the table) to eat}
		\gll	hel	sa-ka-b-iršː-id	b-erkʷ-ij\\
			that	\tsc{hither-down-n-}put\tsc{.ipfv-prs.1}	\tsc{n-}eat\tsc{.pfv-inf}\\
		\glt	\sqt{‎We put it (on the table) to eat.}
	\end{exe}

	\item	formation of complement clauses with different types of complement-taking predicates (volitional, modal, phasal, manipulative, etc.). The complement clauses have potential or activity meaning (\refsec{ssec:Infinitive and subjunctive}). In \refex{ex:There is what I want to say} the infinitival complement clause is embedded into a headless relative clause formed with the modal participle -\textit{an}.
	%
	\begin{exe}
	
				\ex	\label{ex:‎‎I was not able to move (lit. shake) (my legs)}
	\gll	[hak'	d-arq'-ij]	a-r-iχ-ub=da\\
		shake	\tsc{npl-}do\tsc{.pfv-inf}	\tsc{neg-f-}be.able\tsc{.pfv-pret=1}\\
	\glt	\sqt{‎‎I (fem.) was not able to move (lit. shake) (my legs).}
	
	
		\ex	\label{ex:There is what I want to say}
		\gll	le-b=q'al	[dam	[b-urs-ij]	b-ikː-an]\\
			exist\tsc{-n=prt}	\tsc{1sg.dat}	\tsc{n-}say\tsc{.pfv-inf}	\tsc{n-}want\tsc{.ipfv-ptcp}\\
		\glt	\sqt{There exists what I want to say.}
			
	\end{exe}
\end{enumerate}

In addition, the infinitive can take a number of suffixes and enclitics:
%
\begin{enumerate}
	\item[(i)]	the cross-categorical suffix \tit{-ce}\slash\tit{-te} for the formation of complement clauses with potential meaning and purpose clauses \refex{ex:‎There is something to eat and to drink}. Note that in this function the suffix can also be omitted without any change in meaning (i.e. compare with \refex{ex:‎We put it (on the table) to eat}).
	%
	\begin{exe}
		\ex	\label{ex:‎There is something to eat and to drink}
		\gll	na	d-erčː-ij-te	d-erkʷ-ij-te	li<d>il	cik'al	le-d\\
			now	\tsc{npl-}drink\tsc{.pfv-inf-dd.pl} 	\tsc{npl-}eat\tsc{.pfv-inf-dd.pl} 	all\tsc{<pl>}	something	exist\tsc{-npl}\\
		\glt	\sqt{‎There is something to eat and to drink.}
	\end{exe}

	\item[(ii)]	the complementizer\slash embedded question marker \tit{=al} for the formation of embedded polar and content questions \refex{ex:I do not remember how it is called} and very occasionally for rhetoric questions for which the speaker does not expect an answer \refex{ex:‎How is this possible, they four and we also in one car}. The latter use is due to the ongoing grammaticalization of the embedded question enclitic as a marker of epistemic modality (\refsec{sec:Subordinate questions}).
	%
	\begin{exe}
		\ex	\label{ex:I do not remember how it is called}
		\gll	it-i-j	aχːu	han	b-el	akːu	[ce	b-ik'ʷ-ij=al]\\
			that\tsc{-obl-dat}	not.know	remember	\tsc{n-}remain\tsc{.pfv}	be\tsc{.neg}	what	\tsc{n-}say\tsc{.ipfv-inf=indq}\\
		\glt	\sqt{I do not remember how to call it.}

		\ex	\label{ex:‎How is this possible, they four and we also in one car}
		\gll	cet'-le	b-iχʷ-ij=al,	aʁʷal	itːi=ra	nušːa=ra	mašin-ni-cːella\\
			how\tsc{-advz}	\tsc{hpl-}be\tsc{.pfv-inf=indq}	four	\tsc{3pl=add}	\tsc{1pl=add}	car\tsc{-obl-comit	}\\
		\glt	\sqt{‎How is this possible, they four and we also in one car.}
	\end{exe}
	
	\item[(iii)]	subordinating enclitics for the formation of adverbial clauses, e.g. \tit{=sat}\slash\tit{=satːin}\slash\tit{= satːinna} \sqt{until}, \tit{=sar} \sqt{before, until} and \tit{bahandan} \sqt{because of} (\refsec{cpt:specializedconverbssubordinatingenclitics})

	\item[(iv)]	the suffix \tit{-li-j} (\tsc{-obl-dat}): the dative is generally used to express causes (\refsec{sssec:Dative}), and it is the only case that the infinitive can be inflected for \refex{ex:‎‎You tell him, I said silently like this, such that Kurban cannot hear it}. In the causative function the dative is also suffixed to other deverbal nominals such as abstracts nouns with the suffix \tit{-dex} and the masdar (\refsec{ssec:The masdar}).
	%
	\begin{exe}
		\ex	\label{ex:‎‎You tell him, I said silently like this, such that Kurban cannot hear it}
		\gll	``u-l	b-urs-a!''		haʔ-ib=da	qːant'-le	caj-na	cara	daˁʡaˁn-ne		[Q'urban-ni	t'am	a-d-aq'-ij-li-j		heχ-itːe]\\
			\tsc{2sg-erg}	\tsc{n-}tell\tsc{-imp}	say\tsc{.pfv-pret=1}		short\tsc{-advz}	one-\tsc{time}	other	secret\tsc{-advz}		Kurban\tsc{-erg}	sound	\tsc{neg-npl-}hear\tsc{.pfv-inf-obl-dat}		\tsc{dem.down}\tsc{-advz}\\
		\glt	\sqt{``‎‎You tell him!'' I said briefly once more such that Kurban could not hear it.}
	\end{exe}


\end{enumerate}


% --------------------------------------------------------------------------------------------------------------------------------------------------------------------------------------------------------------------- %

\subsection{The subjunctive (i.e. agreeing infinitive)}
\label{ssec:The subjunctive (agreeing infinitive)}

Sanzhi Dargwa has another verb form that is functionally equivalent to the infinitive, but shows person agreement, and will be called `subjunctive' in this grammar. Person agreement of the subjunctive is reduced in comparison to other verb forms such as the habitual present or the habitual past. There is no suffix for the first person and instead the normal infinitive is used. The suffixes for the second and the third person, which do not distinguish number, are displayed in \reftab{tab:subjunctive}. The second person makes use of the same stem augment vowels (\textit{i}, \textit{u}) that occur with other verb forms such as the habitual present or the prohibitive.\footnote{The second person subjunctive suffixes are identical in form with variants of the plural prohibitive.} Diachronically, the suffixes \tit{-itːaj}\slash\tit{-utːaj} consist of the stem augment, followed by a second person suffix -\textit{tː}, and the last part -\textit{aj}, which represents the actual subjunctive marker. This becomes clear when we compare the subjunctive to other verb forms (\refsec{sec:Person agreement}). However, for the sake of understanding I treat the suffixes as units and use a single gloss.

%
\begin{table}
	\caption{The subjunctive}
	\label{tab:subjunctive}
	\small
	\begin{tabularx}{0.40\textwidth}[]{%
		>{\centering\arraybackslash}p{10pt}
		>{\centering\arraybackslash}X
		>{\centering\arraybackslash}X}
		
		\lsptoprule
			{}	&	\tsc{sg}	&	\tsc{pl}\\
		\midrule
			1	&	\tmd		&	\tmd\\
			2	&	\multicolumn{2}{c}{\tit{-itːaj\slash -utːaj}}\\
			3	&	\multicolumn{2}{c}{\tit{-araj\slash -anaj}}\\
		\lspbottomrule
	\end{tabularx}
\end{table}

There is a strong correlation between the stem augment vowel and transitivity, i.e. intransitive verbs mostly take \tit{-u} \refex{ex:subjunctivecryeatgosleep} and transitive and affective verbs usually take \tit{-i} \refex{ex:subjunctivetellseebringlead}. Thus, the subjunctive behaves as other verb forms that have person agreement suffixes (\refsec{sec:Person agreement}). 


%
\begin{exe}
	\ex	\label{ex:subjunctivecryeatgosleep}
	\begin{xlist}
		\ex	\tit{r-isː-utːaj} \sqt{cry} 
		\ex	\tit{r-uk-utːaj} (\tsc{ipfv}) \sqt{eat}
		\ex	\tit{r-uq'-uˁtːaj} \sqt{go}
		\ex	\tit{ka-r-isː-utːaj} (\tsc{pfv}) \sqt{lie, sleep}
	\end{xlist}

	\ex	\label{ex:subjunctivetellseebringlead}
	\begin{xlist}
		\ex	\tit{b-urs-itːaj} \sqt{tell}
		\ex	\tit{či-b-až-itːaj} (\tsc{pfv}) \sqt{see}
		\ex	\tit{r-ič-itːaj} (\tsc{pfv}) \sqt{bring, lead}
	\end{xlist}
\end{exe}

In the third person, the suffixes \tit{-ar-aj} and \tit{-an-aj} are used, which diachronically consist of \tit{-ar} (homophone to one allomorph of the the habitual present and the realis conditional), or \tit{-an} (homophone to the modal participle) and \tit{-aj}. The choice between \tit{-araj} and \tit{-anaj} is mostly lexicalized \refex{ex:subjunctivecrydobekill}, \refex{ex:subjunctiveliewriteeatsupport} but there are a few verbs to which in elicitation both suffixes can be attached, e.g.  \tit{či-ha-b-uq-anaj}\slash\tit{či-ha-b-uq-araj} (\tsc{spr-up}\tsc{-n-}go\tsc{.pfv-subj.3}) \sqt{climb}. In general, \tit{-anaj} is more common both in terms of types (i.e. verb stems to which the suffix is added) as well as in terms of token frequency in my corpus.
%
\begin{exe}
	\ex	\label{ex:subjunctivecrydobekill}
	\begin{xlist}
		\ex	\tit{r-isː-araj} \sqt{cry}
		\ex	\tit{b-arq'-araj} (\tsc{pfv}) \sqt{do}
		\ex	\tit{b-iχʷ-araj} (\tsc{pfv}) \sqt{be, become, can}
		\ex	\tit{kaxʷ-araj} (\tsc{pfv}) \sqt{kill}
	\end{xlist}

	\ex	\label{ex:subjunctiveliewriteeatsupport}
	\begin{xlist}
		\ex	\tit{ka-b-isː-anaj} (\tsc{pfv}) \sqt{lie, sleep}
		\ex	\tit{b-elk'-anaj} (\tsc{pfv}) \sqt{write}
		\ex	\tit{b-uk-anaj} (\tsc{ipfv}) \sqt{eat}
		\ex	\tit{ha-b-irq'-anaj} (\tsc{ipfv}) \sqt{support, bring up, make high}
	\end{xlist}
\end{exe}

The subjunctive, just like the infinitive, is mainly obtained from perfective stems, though a number of imperfective stems can also be inflected for it \refex{ex:subjunctiveexamples}.
%
\begin{exe}
	\ex	\label{ex:subjunctiveexamples}
	\begin{xlist}
		\TabPositions{9em,10em,19em}
		\ex	\tit{ha-b-ilq'-araj} (\tsc{ipfv}) 	\tab	/	\tab 	\tit{habiq'-araj} (\tsc{pfv}) 	\tab	\sqt{raise, keep up}
		\ex	\tit{lukː-araj} (\tsc{ipfv}) 		\tab	/	\tab 	\tit{b-ikː-araj} (\tsc{pfv}) 		\tab	\sqt{give}
		\ex	\tit{či-b-ig-araj} (\tsc{ipfv}) 	\tab	/	\tab 	\tit{či-b-ag-araj} (\tsc{pfv}) 	\tab	\sqt{see}
		\ex	\tit{b-alχ-araj} (\tsc{ipfv}) 	\tab	/	\tab 	\tit{b-aχ-arajj} (\tsc{pfv}) 		\tab	\sqt{know}
		\ex	\tit{b-irc-araj} (\tsc{ipfv}) 	\tab	/	\tab 	\tit{b-ic-araj} (\tsc{pfv}) 		\tab	\sqt{sell}
		\ex	\tit{b-ubk'-arai} (\tsc{ipfv}) 	\tab	/	\tab 	\tit{b-ebk'-araj} (\tsc{pfv}) 	\tab	\sqt{die}
	\end{xlist}
\end{exe}


The functions of the subjunctive are identical to the functions of the normal infinitive and it is always possible to replace the subjunctive with the infinitive. Thus, the subjunctive heads purpose and complement clauses:
%
\begin{exe}
	\ex	\label{ex:‎Then there at the school, there is a place, if it still exists, for you to sleep}
	\gll	c'il	uškul-la		hextːu-b,		musːa=ra		k'e-b b-iχʷ-ar				ka-d-isː-utːaj	\\
		then	school\tsc{-gen}	there\tsc{.up-n}	place\tsc{=add}	exist\tsc{.up-n}		\tsc{n-}be\tsc{.pfv-cond.3}	\tsc{down-1/2.pl-}sleep\tsc{.pfv-subj.2}\\
	\glt	\sqt{‎Then there at the school, there is a place, if it still exists, for you to sleep.}

	\ex	\label{ex:‎The roads to go there were probably bad, for you to go}
	\gll	d-uq'-ij	xːun-be	wahi-l	d-určː-i=q'al	ixtːu	d-uq'-aˁtːaj\\
		\tsc{1/2.pl-}go\tsc{.pfv-inf}	way\tsc{-pl}	bad\tsc{-advz}	\tsc{npl-}be\tsc{.ipfv-hab.pst=mod}	there.\tsc{up}	\tsc{1/2.pl-}go\tsc{.pfv-subj.2}\\
	\glt	\sqt{‎The roads to go there were probably bad, for you to go.}

	\ex	\label{ex:‎The sister started to eat up the tree so it would come down}
	\gll	hel	rucːi-l	r-aʔ	r-išː-ib	ca-r	b-ukː-un-ne	kːalkːi	ka-b-ik-araj\\
		that	sister\tsc{-erg}	\tsc{f-}begin	\tsc{f-}become\tsc{.pfv-pret}	be\tsc{-f}	\tsc{n-}eat\tsc{.ipfv-icvb}	tree	\tsc{down-n-}occur\tsc{.pfv-subj.3}\\
	\glt	\sqt{‎The sister started to eat up the tree so it would come down.}
\end{exe}

As with the infinitive it is possible to suffix the cross-categorical suffix \tit{-ce} to the subjunctive \refex{ex:‎It would be better to build a hospital instead of building a school}. In this example, \tit{b-arq'-araj-ce} could be replaced by \tit{b-arq'-ij-ce} and the meaning would not change.
%
\begin{exe}
	\ex	\label{ex:‎It would be better to build a hospital instead of building a school}
	\gll	uškul	b-arq'-araj-ce	balnicːa	b-arq'-ib-le	b-iχʷ-ardel\\
		school	\tsc{n-}do\tsc{.pfv-subj.3-dd.sg}	hospital	\tsc{n-}do\tsc{.pfv-pret-cvb}	\tsc{n-}be\tsc{.pfv-cond.pst}\\
	\glt	\sqt{‎It would be better to build a hospital instead of building a school.} (E)
\end{exe}

Furthermore, subordinating enclitics for the formation of adverbial clauses can be attached, in particular \tit{=sat}\slash\tit{=satːin}\slash\tit{=satːinna} \sqt{until} \refex{ex:‎‎‎He remained silent, until it became dark he did not tell}, \tit{=sar} \sqt{before, until} and \tit{bahandan} \sqt{because of} (\refsec{cpt:specializedconverbssubordinatingenclitics}):
%
\begin{exe}
	\ex	\label{ex:‎‎‎He remained silent, until it became dark he did not tell}
	\gll	k'ʷah	Ø-ič-ib-le,	ʁera	ag-araj=sat	a-b-urs-ib\\
		silent	\tsc{m-}occur\tsc{.pfv-pret-cvb}	dusk	go\tsc{.pfv-subj.3=}until	\tsc{neg-n-}tell\tsc{.pfv-pret}\\
	\glt	\sqt{‎‎‎He remained silent, until it became dark he did not tell.}
\end{exe}

It seems that the subjunctive, which is absent from the more innovative north Dargwa varieties (e.g. from Akusha\slash Standard Dargwa), is gradually disappearing from south Dargwa varieties. In Icari, it lacks a cell in the transitive paradigm that is replaced with \tit{-ij} (which is not the Icari infinitive, but another suffix.). In Sanzhi, it entirely lacks first person forms. Verb forms similar to the Sanzhi subjunctive are found in other south Dargwa varieties such as Qunqi, and Xuduc, but, e.g., not in Tanti \citep[136]{Sumbatova.Lander2014}. \citet[107]{Sumbatova.Mutalov2003} write that the Icari subjunctive (-\textit{aj}\slash -\textit{j}), which is clearly cognate with the Sanzhi subjunctive, is historically and structurally related to the Standard Dargwa infinitive -\textit{es}. 




% --------------------------------------------------------------------------------------------------------------------------------------------------------------------------------------------------------------------- %

\subsection{The masdar}
\label{ssec:The masdar}

The suffix for the masdar is \tit{-ni}. It has an allomorph -\textit{ri}, which is only used with a handful of verbs such as the negative form of the copula (\tit{(b-)akʷ-ni} and \tit{akːʷ-ri)}). The masdar is available for imperfective and perfective stems as well as for the copula (\tit{ca-b-ni}), and the locative copulas, e.g. \tit{le-b-ni}. The functions of the masdar are:
%
\begin{enumerate}
	\item	formation of complement clauses, e.g. with matrix predicates such as \sqt{know} and \sqt{understand} \refex{ex:Only for you I sing this song; not knowing whether you got tired}
	%
	\begin{exe}
		\ex	\label{ex:Only for you I sing this song; not knowing whether you got tired}
		\gll	il	at=cun	dalaj	w-ik'-ud,	[w-arc-ni=ra]	a-b-alχ-ul\\
			this	\tsc{2sg.dat=}only	song	\tsc{m-}say\tsc{.ipfv-prs.1}	\tsc{m-}get.tired\tsc{.pfv-msd=add}	\tsc{neg-n-}know\tsc{.ipfv-icvb}\\
		\glt	\sqt{Only for you I sing this song, not knowing whether you got tired.}
	\end{exe}
	
	If the masdar is formed from a stem with imperfective aspect, the temporal reference of the complement clause is non-past \refex{ex:I did not know that he dies}, and if it is formed from a stem with perfective aspect the temporal reference is past time \refex{ex:I did not know that he died}.
	%
	\begin{exe}
		\ex	\label{ex:I did not know that he dies}
		\gll	ubk'-ni dam a-b-alχ-ul=de\\
			die\tsc{.m.ipfv-msd}	\tsc{1sg.dat}		\tsc{neg-n-}know\tsc{.ipfv-icvb=pst}\\
		\glt	\sqt{I did not know that he dies.} (E)

		\ex	\label{ex:I did not know that he died}
		\gll	w-ebk'-ni dam a-b-alχ-ul=de\\
			\tsc{m-}die\tsc{.pfv-msd}	\tsc{1sg.dat}		\tsc{neg-n-}know\tsc{.ipfv-icvb=pst}\\
		\glt	\sqt{I did not know that he died.} (E)
	\end{exe}

	\item	formation of deverbal nouns that can be used like other nominals \refex{ex:‎‎‎There was skating on the ice. (referring to a woman skating)}, i.e. in the position of arguments or adjuncts. The masdar can be inflected, e.g. for the dative, which yields the expected causative reading \refex{ex:‎‎‎These (gems) are for you, my friend, because you saved me, said the bear}, and for the genitive when it expresses the topic of a speech act \refex{ex:‎‎‎I don't know if I should tell you the story about how we went somewhere} or other relations \refex{ex:What to do, this pasturing (lit. the work of going after the animals) is work}.
	%
	\begin{exe}
		\ex	\label{ex:‎‎‎There was skating on the ice. (referring to a woman skating)}
		\gll	mig-le-r	qus	r-ik'ʷ-ni	b-irχʷ-i\\
			ice\tsc{-loc-f}	slip	\tsc{f-}say\tsc{.ipfv-msd}	\tsc{n-}be\tsc{.ipfv-hab.pst}\\
		\glt	\sqt{‎‎‎There was skating on the ice.} (referring to a woman skating)

		\ex	\label{ex:‎‎‎These (gems) are for you, my friend, because you saved me, said the bear}
		\gll	``iš-tːi	at	ca-d,	di-la	juldaš,		du	b-erc-aq-ni-li-j''	b-ik'-ul	ca-b\\
			this\tsc{-pl}	\tsc{2sg.dat}	be\tsc{-npl}	\tsc{1sg-gen}	friend		\tsc{1sg}	\tsc{n-}save\tsc{.pfv-caus-msd-obl-dat}		\tsc{n-}say\tsc{.ipfv-icvb}	be\tsc{-n}\\
		\glt	\sqt{‎‎‎These (gems) are for you, my friend, because you saved me, said (the bear).}

		\ex	\label{ex:‎‎‎I don't know if I should tell you the story about how we went somewhere}
		\gll	čina=k'u	ʡaˁlħaˁm-le	d-uˁq'-ni-la		χabar	b-urs-idel	aχːu\\
			where=\tsc{emph}	condolence-\tsc{loc}	\tsc{1/2pl-}go\tsc{-msd-gen}	story	\tsc{n-}tell\tsc{-modq}	not.know\\
		\glt	\sqt{‎‎‎I don't know if I should tell you the story about how we went somewhere, to the condolences.}

		\ex	\label{ex:What to do, this pasturing (lit. the work of going after the animals) is work}
		\gll	značit,	ce	d-arq'-ij	uˁq'-ni-lla	ʡaˁči=q'al	it\\
			thus	what	\tsc{npl-}do\tsc{.pfv-inf}	go\tsc{.m-msd-gen}	work\tsc{=mod}	that\\
		\glt	\sqt{What to do, this pasturing (lit. the work of going after the animals) is work.}
	\end{exe}	
\end{enumerate}


%%%%%%%%%%%%%%%%%%%%%%%%%%%%%%%%%%
\section{Specialized converbs}
\label{cpt:specializedconverbssubordinatingenclitics}

Constructions with special converbs occur in adverbial clauses that express temporal or causal relationships. Formally, they mostly consist of an enclitic or suffix bearing the specific temporal/causal meaning or of a postposition\slash adverb used to express temporal relationships. The converbial marker is attached either to the preterite participle and the modal participle or to the subjunctive and the infinitive (or follows the respective verb forms). Other ways of obtaining specialized adverbial clauses involve the locative participle and the noun \tit{zamana} \sqt{time}. Sanzhi possesses the specialized converbs that are given in \refex{ex:Specialized converbs and subordinating enclitics}. There are two enclitics among them (=\textit{qːel(la)} and =\textit{sat}\slash =\textit{satːin}\slash =\textit{satːinna}), which are not only used with verbs in order to form adverbial clauses, but also with nominals. Their use with nominals is described in \refsec{sec:Temporal enclitics}. The syntax of adverbial clauses is analyzed in \refcpt{cpt:Syntactic properties of adverbial and conditional clauses}.

\begin{exe}
	\ex	\label{ex:Specialized converbs and subordinating enclitics}
	\begin{xlist}
		\TabPositions{14em}
		\ex	\tit{=qːel(la)} \sqt{when, while, because}	\tab	(simultaneity, anteriority, causality)	   
		\ex	\tit{-er} \sqt{when, as}				\tab	(simultaneity)
		\ex	\tit{=sat\slash =satːin\slash =satːinna} \sqt{until, before, as much as, as long as}
		\sn	~\hspace*{1em}					\tab	(posteriority, manner)	   
		\ex	\tit{sar(ka)} \sqt{until, before}			\tab	(posteriority)
		\ex	\tit{(h)itːi} \sqt{after, because}			\tab	(anteriority, causality)
		\ex	\tit{-la} \sqt{since, after}				\tab	(anteriority, causality)	   
		\ex	\tit{b-el-le} \sqt{while, as long as, as soon as, until, when}
		\sn	~\hspace*{1em}					\tab	(simultaneity, immediate anteriority)	   
		\ex	\tit{zamana} \sqt{time}				\tab	(simultaneity)	   
		\ex	\tit{=xːar} \sqt{although, even if}		\tab	(concession)	   
		\ex	the locative participle \tit{-na}			\tab	(causality)	   
		\ex	\tit{bahanne\slash bahandan} \sqt{because of}	\tab	(causality)	
	\end{xlist}
\end{exe}


%%%%%%%%%%%%%%%%%%%%%%%%%%%%%%%%%%%%%%%%%%%%%%%%%%%%%%%%%%%%%%%%%%%%%%%%%%%%%%%%

\subsection{The temporal/causal enclitic \tit{=qːel(la)} \sqt{when, while, because}}
\label{sec:enclitic =qella}

The temporal \tit{=qːel(la)}, which exists in a more commonly used short form and in a less frequently occurring long form, translates as \sqt{when, while, because}. As the other two temporal enclitics it can be hosted by verbs and other parts of speech (\refsec{sec:Temporal enclitics}). When used to form adverbial clauses it is added to the preterite participle \refex{sssec:The preterite participle} or to the modal participle \refex{sssec:The modal participle -an}, and also to forms of the negative copula (usually the participle, but also other forms) \refex{ex:Well because (they) were not Muslims}. It expresses the temporal simultaneity \refex{ex:‎When they went up there they were praying}, and rarely the temporal anteriority \refex{ex:‎When / after / because at that night} of the situation referred to in the adverbial clause with respect to the situation described in the main clause. In many cases, this also implies a causal link between the two events \refex{ex:Well because (they) were not Muslims}.
%

\begin{exe}
	\ex	\label{ex:‎When they went up there they were praying}
	\gll	či-b-a	arg-an=qːella		debʁul-m-a-l	b-irq'-ul	b-už-ib-le=de\\
		up\tsc{-hpl-dir}	go\tsc{.ipfv-ptcp=}when	prayer\tsc{-pl-obl-erg}	\tsc{hpl-}do\tsc{.ipfv-icvb}	\tsc{hpl-}be\tsc{-pret-cvb=pst}\\
	\glt	\sqt{‎When they went up there they were praying.}

	\ex	\label{ex:‎When / after / because at that night}
	\gll	hel	dučːi	a-b-ič-ib=qːella	ikarus	abt'abuz	hila	bar	ag-ur-re	Rast'aw=uw	čina=jal	hetːu	ag-ur	ca-w	hetːu\\
		that	night	\tsc{neg-n-}occur\tsc{.pfv-pret=}when	Icarus	coach	last	day	go\tsc{.pfv-pret-cvb}	Rostov\tsc{=q}	where\tsc{=indq}	there	go\tsc{.pfv-pret}	be\tsc{-m}	there\\
	\glt	\sqt{‎When\slash after\slash because in that night there was no Icarus coach, he went to Rostov or somewhere else the next day, he went there.}

	\ex	\label{ex:Well because (they) were not Muslims}
	\gll	nu	busurman-te	akːu=qːella,	\ldots\\
		well	Muslim\tsc{-pl} 	be\tsc{.neg=}when\\
	\glt	\sqt{Well because (they) were not Muslims, \ldots}
\end{exe}

When the enclitic is hosted by nominals (nouns, demonstrative pronouns, adjectives) it also means `when'. Examples are provided in \refsec{sec:Temporal enclitics}.




%%%%%%%%%%%%%%%%%%%%%%%%%%%%%%%%%%%%%%%%%%%%%%%%%%%%%%%%%%%%%%%%%%%%%%%%%%%%%%%%

\subsection{The temporal marker \tit{-er} \sqt{when, as}}
\label{sec:enclitic =er}

The temporal meaning of simultaneity is also expressed by \tit{-er}, which is, just like \tit{=qːel(la)}, added to the preterite participle or to the modal participle \refex{ex:When I moved to Druzhba, I worked in the sovkhoz in Druzhba}-\refex{ex:‎As you (masc.) say, they are walking around}. This suffix does not imply any causal relationships. In fact, in some examples there is no relationship whatsoever between the situations expressed in the two clauses \refex{ex:‎As you (masc.) say, they are walking around}. The suffix is only added to verbs, never to nominals or other parts of speech.
%
\begin{exe}
	\ex	\label{ex:When I moved to Druzhba, I worked in the sovkhoz in Druzhba}
	\gll	Družba-le	ka-jʁ-ib-er,	Družba-le	sowχoz-la	ʡaˁči	b-irq'-ul=de\\
		Druzhba\tsc{-loc}	\tsc{down}-come\tsc{.m.pfv-pret-}when	Druzhba\tsc{-loc}	sovkhoz\tsc{-gen}	work	\tsc{n-}do\tsc{.ipfv-icvb=pst}\\
	\glt	\sqt{When I moved to Druzhba, I worked in the sovkhoz in Druzhba.}

	\ex	\label{ex:‎When he was drunk and went there, he staggered}
	\gll	heχ	b-erčː-ib-le	haˁ-q'-aˁn-er	cinna	tːartːar	uq-un-ne	heχ\\
		\tsc{dem.down}	\tsc{n-}drink\tsc{.pfv-pret-cvb}	\tsc{up}-go\tsc{-ptcp-}when	pause.filler	shake	go\tsc{.m.pfv-pret-cvb}	\tsc{dem.down}\\
	\glt	\sqt{‎When he was drunk and went there, he staggered.}

	\ex	\label{ex:‎As you (masc.) say, they are walking around}
	\gll	u	Ø-ik'ʷ-an-er,	šːatːir	arg-ul	ca-b\\
		\tsc{2sg}	\tsc{m-}say\tsc{.ipfv-ptcp-}when	walk	go\tsc{.ipfv-icvb}	be\tsc{-hpl}\\
	\glt	\sqt{‎As you (masc.) say, they are walking around.}
\end{exe}


%%%%%%%%%%%%%%%%%%%%%%%%%%%%%%%%%%%%%%%%%%%%%%%%%%%%%%%%%%%%%%%%%%%%%%%%%%%%%%%%

\subsection{The temporal enclitic \tit{=sat}\slash\tit{=satːin}\slash\tit{=satːinna} \sqt{until, before as much/long as}}
\label{sec:enclitic =satin}

This enclitic, which also belongs to the category of enclitics that can be used with nominals (\refsec{sec:Temporal enclitics}), occurs in three variants of different lengths that can be ordered according to their increasing frequency as  \tit{=satːin} < \tit{=satːinna} < \tit{=sat}. It is a cognate of the spatial/temporal adverb \textit{satːi} \sqt{at\slash along the front, as soon as}, which, however, occurs before verbs rather than following them. The adverb can be further decomposed into the postposition \tit{sa} \sqt{in front, ago} and the adverbializer \tit{-tːi}, which is part of a few spatial adverbials (\refsec{ssec:SpatialAdverbsDerivedFromPostpositions}).

The enclitic follows the infinitive and the subjunctive and expresses the meaning of temporal posteriority of the situation denoted by the adverbial clause with respect to the situation referred to in the main clause, i.e. \sqt{before, until}:
%
\begin{exe}
	\ex	\label{ex:Until you take your stick out, you have to stand there}
	\gll	u-l	dirxːa	gu-r-b-uqː-ij=sat	k-ercː-an=de	heštːu\\
		\tsc{2sg-erg}	stick	\tsc{sub-abl-n-}take.out\tsc{.ipfv-inf=}as.much \tsc{down}-stand\tsc{.ipfv-ptcp=2sg}	here\\
	\glt	\sqt{Until you take your stick out, you have to stand there.}

	\ex	\label{ex:Before (the turtle) came back from that place}
	\gll	mus-a-rka	abratna	čar	sa-b-iχʷ-araj=satːin	\ldots\\
		place\tsc{.obl-loc-abl}		back	back	\tsc{hither}\tsc{-n-}be\tsc{.pfv-subj.3=}until\\
	\glt	\sqt{Before (the turtle) came back from that place, ...}
\end{exe}

When the enclitic is attached to the modal participle the meaning is \sqt{as much as, as long as} \refex{ex:‎‎‎The woman was patient as long as she could}. Two more examples are given in \refsec{sec:Temporal enclitics}.
%
\begin{exe}
	\ex	\label{ex:‎‎‎The woman was patient as long as she could}
	\gll	xːunul-li-j	r-irχʷ-an=satːin	jaˁħ	b-irq'-ul=de\\
		woman\tsc{-obl-dat}	\tsc{f-}be.able\tsc{.ipfv-ptcp=}as.much	patience	\tsc{n-}do\tsc{.ipfv-icvb=pst}\\
	\glt	\sqt{‎‎‎The woman was patient as long as she could.}
\end{exe}

The same meaning is also attested when the enclitic follows nominals \refex{ex:‎‎‎Around as much as 700 pictures I made when we went (there) now}, \refex{ex:‎‎‎The ones that I left in passion (i.e. that fell in love with me)}. Finally, the enclitic can be attached to demonstrative pronouns forming manner demonstrative pronouns that are used in comparison \sqt{like this, like that, such} \refex{ex:‎[‎‎From her small finger he pulled out his parents], so big was his sister}.
%



%%%%%%%%%%%%%%%%%%%%%%%%%%%%%%%%%%%%%%%%%%%%%%%%%%%%%%%%%%%%%%%%%%%%%%%%%%%%%%%%

\subsection{The temporal adverb/postposition \tit{sar(ka)} \sqt{until, before}}
\label{sec:enclitic =sarka}

The temporal adverb/postposition \tit{sar(ka)} \sqt{before, until}, (which also has the spatial meaning \sqt{in front}, \refsec{ssec:postposition sa} and \refsec{ssec:SpatialAdverbsDerivedFromPostpositions}) follows the infinitive or the subjunctive. The resulting meaning is purely one of temporal posteriority, i.e. \sqt{until, before} \refex{ex:‎I was a milkmaid until / before I got married} and occasionally more like an apprehensive \refex{ex:‎‎‎Stand there before / lest / otherwise I hit you with the crook on the head}.
%
\begin{exe}
	\ex	\label{ex:‎I was a milkmaid until / before I got married}
	\gll	q'uˁl-a-la	dajark'a	ag-ur=da	bahsar	hel	xadi	r-uˁq'-ij	sarka\\
		cow\tsc{-obl-gen}	milkmaid	go\tsc{.pfv-pret=1}	first	that	married \tsc{f-}go\tsc{.pfv-inf}		before\\
	\glt	\sqt{‎I was a milkmaid until\slash before I got married.}

	\ex	\label{ex:‎‎‎Stand there before / lest / otherwise I hit you with the crook on the head}
	\gll	``q'isːa	bek'-le	ka-b-aˁq-ij	sar	hetːu	ka-jcː-e!''	haʔ-ib=da\\
		crook	head\tsc{-loc}	\tsc{down}\tsc{-n-}hit\tsc{.pfv-inf}	before	there	\tsc{down}-stand\tsc{.m.pfv-imp}	say\tsc{.pfv-pret=1}\\
	\glt	\sqt{``‎‎‎Stand there before\slash lest\slash otherwise I hit you with the crook on the head!'' I said.}
\end{exe}


%%%%%%%%%%%%%%%%%%%%%%%%%%%%%%%%%%%%%%%%%%%%%%%%%%%%%%%%%%%%%%%%%%%%%%%%%%%%%%%%

\subsection{The temporal/causal postposition \tit{(h)itːi} \sqt{after, because}}
\label{sec:temporalcausal postposition hiti}

The postposition/adverb \tit{hitːi} \sqt{after, behind} (see \refsec{ssec:postposition hiti} for the use in postpositional phrases) can occur in temporal adverbial clauses with the meaning \sqt{when, after}. In the majority of occurrences the postposition is shortened to its enclitic variant \tit{=itːi} and directly attached to the preterite participle \refex{ex:‎‎‎After we left from there, I did not return.}. However, it is always possible to replace the enclitic with the full form \tit{hitːi}.
%
\begin{exe}
	\ex	\label{ex:‎‎‎After we left from there, I did not return.}
	\gll	c'il	helka	ag-ur=itːi	c'il	a-ka-r-ač'-ib=da\\
		then	from.there	go\tsc{.pfv-pret=}after	then	\tsc{neg-down}\tsc{-f-}come\tsc{.pfv-pret=1}\\
	\glt	\sqt{‎‎‎After we left from there, I did not return.}
\end{exe}

Occasionally, the temporal relationship also implies a causal relation between the situation referred to in the adverbial clause and the situation expressed in the main clause:
%
\begin{exe}
	\ex	\label{ex:After drinking the heart opened}
	\gll	b-erčː-ib=itːi	urk'i	b-at	b-uq-un-ne,	\ldots\\
		\tsc{n-}drink\tsc{.pfv-pret=}after	heart	\tsc{n-}free	\tsc{n-}go\tsc{.pfv-pret-cvb}\\
	\glt	\sqt{After drinking the heart opened, (and he molested his wife).}
\end{exe}

Instead of using the bare preterite participle it is also possible to suffix another marker \tit{-la} to the participle, which is, in turn, followed by the postposition \refex{ex:After the times of the war finished he married her}, \refex{ex:after he started drinking}. The suffix -\textit{la} goes most probably back to the genitive case suffix since \tit{hitːi} governs the genitive. The suffix -\textit{la} undergoes assimilation after the sonorants /n/ and /r/ (> -\textit{na}, -\textit{ra}). It can also be employed on its own without the following postposition \textit{hitːi} (\refsec{sec:temporal marker -la}).
%
\begin{exe}
	\ex	\label{ex:After the times of the war finished he married her}
	\gll	il-tːi	daˁwi	taman	d-iχ-ub-la	hitːi	xadi	ka-r-iž-ib	hel\\
		that-\tsc{pl}	war	end	\tsc{npl-}be\tsc{.pfv-pret-post}	after	married	down\tsc{-f-}sit\tsc{.pfv-pret}	that\\
	\glt	\sqt{After the times of the war finished he married her.}

	\ex	\label{ex:after he started drinking}
	\gll	deč-la	učː-ul	w-aʔ-ač'-ib-la	hitːi, \ldots\\
		drinking\tsc{-gen}	drink\tsc{.m.ipfv-icvb}	\tsc{m-}begin-come\tsc{.pfv-pret-post}	after\\
	\glt	\sqt{after he started drinking, \ldots}
\end{exe}

In elicitation it is possible to place the postposition \textit{hitːi} after the modal participle -\textit{an} to which again \tit{-la} (in its assimilated allomorphic form \tit{-na}) can be suffixed \refex{ex:‎‎‎If / because she comes, I will not go there.}. In this construction, however, the meaning diverges and instead of the sequential meaning we get a causal\slash conditional meaning. Now the clause containing the postposition expresses a condition or cause for the situation that the main clause denotes.
%
\begin{exe}
	\ex	\label{ex:‎‎‎If / because she comes, I will not go there.}
	\gll	it	sa-r-irʁ-an-na	hitːi	du	itːu	a-ax-an=da\\
		that	\tsc{hither-f-}come\tsc{.ipfv-ptcp-post}	after	\tsc{1sg}	there	\tsc{neg-}go\tsc{.m-ptcp=1}\\
	\glt	\sqt{‎‎‎If\slash because she comes, I will not go there.} (E)

	\ex	\label{ex:Why should I go to Makhachkala, if / since I can will see the concert in Druzhba?}
	\gll	cellij	du	Maˁħaˁčqːala-le	w-ax-an=da=ja,	dam		Družba-le-b		kancert	či-b-ig-an=itːi?\\
		why	\tsc{1sg}	Makhachkala\tsc{-loc}	\tsc{m-}go\tsc{-ptcp=1=q}	\tsc{1sg.dat}	Druzhba\tsc{-loc-n}	concert	\tsc{spr-n-}see\tsc{.ipfv-ptcp=}after\\
	\glt	\sqt{Why should I go to Makhachkala, if\slash since I can will see the concert in Druzhba?} (E)
\end{exe}


%%%%%%%%%%%%%%%%%%%%%%%%%%%%%%%%%%%%%%%%%%%%%%%%%%%%%%%%%%%%%%%%%%%%%%%%%%%%%%%%

\subsection{The temporal marker \tit{-la} \sqt{since, after}}
\label{sec:temporal marker -la}

This marker is suffixed to the preterite participle. As mentioned in \refsec{sec:temporalcausal postposition hiti}, it goes back to the genitive and the construction most probably arose as a simplified variant of the use of the same marker followed by the postposition \tit{hitːi}. The meaning of both constructions is very similar expressing temporal posteriortity of the situation that the adverbial clause refers to with respect to a second situation that is normally expressed by the main clause. However, the adverbial clauses that contain only \tit{-la} without a following postposition mean \sqt{since then, ever since, from then on, after that}.
%
\begin{exe}
	\ex	\label{ex:‎In Sanzhi there was no (such plant), now there is, since we moved here}
	\gll	Sanži-b	b-akːʷ-i,	hana	ca-b	il	nušːa	ka-d-eʁ-ib-la\\
		Sanzhi\tsc{-n}	\tsc{n-neg-hab.pst}	now	be\tsc{-n}	that	\tsc{1pl}	\tsc{down-1/2pl-}go\tsc{.pfv-pret-post}\\
	\glt	\sqt{‎In Sanzhi there was no (such plant), now there is, since we moved here.}
\end{exe}

The postposition can always be added and this results in a slight change of the meaning. In \refex{ex:Since the trial, after the trial was made, he is already sitting in prison} the speaker uses first the construction with \tit{-la} and then the construction containing the encliticized postposition, but lacking \tit{-la}. Both clauses have similar, but not completely identical, semantics. 
%
\begin{exe}
	\ex	\label{ex:Since the trial, after the trial was made, he is already sitting in prison}
	\gll	hej	sud	b-arq'-ib-la,	sud	b-arq'-ib=itːi,	hej	ka-jž-ib	ca-w	uže	tusnaq-le	hež\\
		this	trial	\tsc{n-}do\tsc{.pfv-pret-post}	trial	\tsc{n-}do\tsc{.pfv-pret=}after	this	\tsc{down}-remain\tsc{.m.pfv-pret}	be\tsc{-m}	already	prison\tsc{-loc}	this\\
	\glt	\sqt{Since the trial, after the trial was made, he is already sitting in prison.}
\end{exe}

To sum up, in certain contexts the three options (only \tit{-la}, only \tit{=itːi}, or \tit{-la + hitːi}) have very similar or even identical meaning (\sqt{after, when}). In other contexts when \tit{-la} is used alone it means rather \sqt{since}.


%%%%%%%%%%%%%%%%%%%%%%%%%%%%%%%%%%%%%%%%%%%%%%%%%%%%%%%%%%%%%%%%%%%%%%%%%%%%%%%%

\subsection{The periphrastic adverbial construction with \tit{b-el-le} \sqt{while, as long as, as soon as, until, when}}
\label{sec:periphrastic adverbial construction belle}

The defective verb \tit{b-el} \sqt{remain, stay} when inflected as perfective converb heads periphrastic adverbial clauses. The same type of periphrastic verb form is attested in independent clauses (\refsec{sec:Verb forms with b-el remain, stay}), but the use in dependent clauses is far more common. The verb can occur together with a lexical verb that bears either the imperfective or the perfective converb suffix. It does not assign case to any arguments and therefore shows the same gender/number agreement as the lexical verbs \refex{ex:‎While until / as long as he is not drinking, everything is good for them}. However, it can also be used in the invariant form \tit{b-el-le} \refex{ex:When / as soon as he went out from there, he mounted the horse} in which gender agreement is lost in favor of the petrified prefix \tit{n-} (neuter singular).

When \tit{b-el} co-occurs with a lexical verb inflected for the imperfective converb the meaning of the adverbial clause is \sqt{while, until, as long as} \refex{ex:‎While until / as long as he is not drinking, everything is good for them}. In combination with the past perfect in the main clause the adverbial clause refers to a situation that obtained at a reference point in the past (= the situation expressed in the main clause) and continued to a later point in time at which the other situation had finished \refex{ex:‎‎‎The poor man took the stone}.
%
\begin{exe}
	\ex	\label{ex:‎While until / as long as he is not drinking, everything is good for them}
	\gll	a-w-čː-ul	w-el-le	[\ldots]	li<d>il	iš-tː-a-la	ʡaˁħ-le	ca-d\\
		\tsc{neg-m-}drink\tsc{.ipfv-icvb}	\tsc{m-}remain\tsc{-cvb}	{}	all\tsc{<npl>}	this\tsc{-pl-obl-gen}	good\tsc{-advz}	be\tsc{-npl}\\
	\glt	\sqt{‎While until\slash as long as he is not drinking, everything is good for them.}

	\ex	\label{ex:‎‎‎The poor man took the stone}
	\gll	er=či	Ø-ik'-ul	w-el-le	il-i-la	qːuʁa-dex-li-j,	iltːi	žaniwar-te	li<d>il	ag-ur-re=de\\
		look=on	\tsc{m-}look.at\tsc{.ipfv-icvb}	\tsc{m-}remain\tsc{-cvb}	that\tsc{-obl-gen}	beautiful\tsc{-nmlz-obl-dat}	\tsc{3pl}	animal\tsc{-pl}	all\tsc{<npl>}	go\tsc{.pfv-pret-cvb=pst}\\
	\glt	\sqt{While\slash as long as he was looking at its beauty, all animals had already left.}
\end{exe}

If the lexical verb appears in the form of the perfective converb, the adverbial clause expresses immediate anteriority that can be translated with \sqt{as soon as, immediately when}. It refers to the point in time when an event is completed or was completed or to the moment when a state obtains or obtained rather than to an enduring situation. The relevant state or event immediately precedes the situation denoted by the main clause.
%
\begin{exe}
	\ex	\label{ex:When / as soon as the master saw the stone, he trembled (started to tremble)}
	\gll	il	usta-j	qːarqːa	či-b-až-ib	b-el-le,	gargar	Ø-ik'ʷ-ij	ha-jž-ib	ca-w\\
		that	master\tsc{-dat}	stone	\tsc{spr-n-}see\tsc{.pfv-pret}	\tsc{n-}remain\tsc{-cvb}	trembling	\tsc{m-}move\tsc{.ipfv-inf}	\tsc{up-}remain\tsc{.m.pfv-pret} be\tsc{-m}\\
	\glt	\sqt{When\slash as soon as the master saw the stone, he trembled (started to tremble).}

	\ex	\label{ex:When / as soon as he went out from there, he mounted the horse}
	\gll	helka	tːura	ka-w-q-un-ne	b-el-le,	murtːa	Ø-iž-ib-le	urči-li-j [\ldots]	gu-r-ag-ur-il	ca-w\\
		from.there	outside	\tsc{down-m-}go\tsc{.pfv-pret-cvb}	\tsc{n-}remain\tsc{-cvb}	rider	\tsc{m-}be\tsc{.pfv-pret-cvb}	horse\tsc{-obl-dat}	{}	\tsc{sub-abl}-go.\tsc{pfv-pret-ref}	be\tsc{-m}\\
	\glt	\sqt{When\slash as soon as he went out from there, he mounted the horse, [singing a song he pretended to be drunk] and left.}

	\ex	\label{ex:‎As soon as Aminat makes bread / finishes making bread}
	\gll	Aminat-li	t'ult'-e	d-arq'-ib-le	b-el-le,	nišːi-j	k'ʷel	s-aqː-a!\\
		Aminat\tsc{-erg}	bread\tsc{-pl}	\tsc{npl-}do\tsc{.pfv-pret-cvb}	\tsc{n-}remain\tsc{-cvb}	\tsc{1pl-dat}	two	\tsc{hither}-carry\tsc{.pfv-imp}\\
	\glt	\sqt{‎As soon as Aminat makes bread\slash finishes making bread, bring us two (loaves of bread)!} (E)
\end{exe}


%%%%%%%%%%%%%%%%%%%%%%%%%%%%%%%%%%%%%%%%%%%%%%%%%%%%%%%%%%%%%%%%%%%%%%%%%%%%%%%%

\subsection{The concessive enclitic \tit{=xːar(e)} \sqt{although, even if}}
\label{sec:concessive enclitic =xar}

The concessive enclitic \tit{=xːar(e)} is attached to the preterite participle \refex{ex:Knowledgeable, he was educated even if he did not study}, \refex{ex:Although she eats, she does not get fat}, to the modal participle \refex{ex:‎Although I want, I cannot}, \refex{ex:‎T‎hough they give here (stuff) for debts, there they do not give him (food) for debts}, or to the participial form of the negated copula, and expresses concession \sqt{even if, although, though, even though}. In the main clause following a concessive clause optionally the particle \textit{ja} can occur \refex{ex:Although she eats, she does not get fat}, which is probably a cognate of the disjunctive particle \textit{ja}.
%
\begin{exe}
	\ex	\label{ex:Knowledgeable, he was educated even if he did not study}
	\gll	q'ʷila	b-alχ-an,	itwaj	gramatni	Ø-irχʷ-iri,	a-b-elč'-un=xːar\\
		a.little	\tsc{n-}know\tsc{.ipfv-ptcp}	like.this	educated	\tsc{m-}be\tsc{.ipfv-hab.pst}	\tsc{neg-n-}learn\tsc{.pfv-pret=conc}\\
	\glt	\sqt{Knowledgeable, he was educated even if he did not study.}

	\ex	\label{ex:Although she eats, she does not get fat}
	\gll	r-uk-un=xːar,	ja	c'erx	r-irχ-ul	akːu\\
		\tsc{f-}eat\tsc{.ipfv-icvb=conc}	or	fat	\tsc{f-}become\tsc{.ipfv-icvb}	be\tsc{.neg}\\
	\glt	\sqt{Although she eats, she does not get fat.} (E)

	\ex	\label{ex:‎Although I want, I cannot}
	\gll	b-ikː-an=xːar	r-irχʷ-ul	akːʷa-di\\
		\tsc{n-}want\tsc{.ipfv-ptcp=conc}	\tsc{f-}be.able\tsc{.ipfv-icvb}	be\tsc{.neg-1}\\
	\glt	\sqt{‎Although I want, I (fem.) cannot.}

	\ex	\label{ex:‎T‎hough they give here (stuff) for debts, there they do not give him (food) for debts}
	\gll	ištːu-d	čibla-li-j	lukː-an=xːar=q'al,	itːu-d	il-i-j	čibla-li-j	a-lukː-an\\
		here\tsc{-npl}	debt\tsc{-obl-dat}	give\tsc{.ipfv-ptcp=conc=mod}	there\tsc{-npl}	that\tsc{-obl-dat}	debt\tsc{-obl-dat}	\tsc{neg-}give\tsc{.ipfv-ptcp}\\
	\glt	\sqt{‎‎Though here they give (food) for debts, there they do not give him (food) for debts.}
\end{exe}

Note that the last example in \refex{ex:‎T‎hough they give here (stuff) for debts, there they do not give him (food) for debts} is an independent clause with the modal participle -\textit{an} suffixed to the verbal head of the main clause. Because of the use of the modal participle the sentence has a habitual meaning without any specific temporal reference. If the suffix \tit{-ne} were added, we would obtain future tense (\refsec{ssec:Future analytic}) with a modal meaning and temporal reference to future events.

Frequently concessive clauses are copula constructions without a copula item (`although X is Y'), in which case the temporal reference of the concessive clause depends on the main clause. For instance, in \refex{ex:‎‎I argued, I quarreled with them, though I was guilty (myself).} the main clause refers to the past and therefore the concessive clause also refers to a past event even though it does not contain any morpheme expressing temporal reference (i.e. no preterite or modal participle). The host of the enclitic in such concessive phrases is the copula predicate, which can for instance be an adjective \refex{ex:‎‎I argued, I quarreled with them, though I was guilty (myself).}, and adverbial, \refex{ex:‎‎Even though the heart is sorrowful (lit. `narrowly'), I sing my song loudly.}, or a noun \refex{ex:‎‎‎Every fourth or fifth night there was a circle (of people) in my mother's house though it was a small house}. 

\begin{exe}
		\ex	\label{ex:‎‎I argued, I quarreled with them, though I was guilty (myself).}
	\gll	ʁaj=či=ra	uq-un=da, majmaj=či=ra	uq-un=da	hel-tː-a-j,	du	winawat=xːar \\
word=on=\tsc{add}	go.\tsc{pfv.m-pret=1}		condemnation=on\tsc{=add}	go.\tsc{pfv.m-pret=1}	that-\tsc{pl-obl-dat}	\tsc{1sg}	guilty=\tsc{conc} \\
	\glt	\sqt{‎‎I argued, I quarreled with them, though I was guilty (myself).}
	
	\ex	\label{ex:‎‎Even though the heart is sorrowful (lit. `narrowly'), I sing my song loudly.}
	\gll	urk'i	q'aq'a-le=xːare,		aq-le	dalaj	w-ik'-ul=da \\
heart	narrow-\tsc{advz=conc}		high-\tsc{advz}	song	\tsc{m}-say.\tsc{ipfv-icvb=1}\\
	\glt	\sqt{‎‎Even though the heart is sorrowful (lit. `narrowly'), I sing my song loudly.}
	
			\ex	\label{ex:‎‎‎Every fourth or fifth night there was a circle (of people) in my mother's house though it was a small house}
	\gll	har	aʁʷal	xujal	dučːi	nik'a	qal=xːar	kružok	b-irχʷ-i	di-la	aba-la	qili-b\\
		every	four	five	night	small	house\tsc{=conc}	circle	\tsc{n-}be\tsc{.ipfv-hab.pst}	\tsc{1sg-gen}	mother\tsc{-gen}	home\tsc{-hpl}\\
	\glt	\sqt{‎‎‎Every fourth or fifth night there was a circle (of people) in my mother's house though it was a small house.}
\end{exe}

There is another way of formulating concessive clauses in Sanzhi, namely the use of conditional forms to which the additive is encliticized (\refsec{sec:concessiveconditionals}).


%%%%%%%%%%%%%%%%%%%%%%%%%%%%%%%%%%%%%%%%%%%%%%%%%%%%%%%%%%%%%%%%%%%%%%%%%%%%%%%%

\subsection{Constructions with \tit{zamana} \sqt{time}}
\label{sec:constructions with zamana}

The noun \tit{zamana} \sqt{time}, ultimately an Arabic loan word, is used in temporal adverbial clauses that are relative clauses from a syntactic point of view (\refcpt{cpt:Relative clauses}). The noun \tit{zamana} is the head, and the relative clause contains a verb in the form of the modal or the preterite participle. Clauses with the modal participle refer to events that were ongoing during a reference point in time or a reference period, which is expressed in the main clause (\sqt{while, when}) \refex{ex:‎While he came from Icari, he went with a torch at night home from work}, \refex{ex:‎‎‎When he was attentively listening he heard the sound of frogs}.
%
\begin{exe}
	\ex	\label{ex:‎While he came from Icari, he went with a torch at night home from work}
	\gll	Uc'ari-r	haˁ-q'-aˁn	zamana,	lampučka	ca-w=ra	dučːi	ha-aš-i=q'al	qili	ʡaˁči-le-r\\
		Icari\tsc{-abl}	\tsc{up}-go\tsc{-ptcp}	time	torch	\tsc{refl-m=add}	night	\tsc{up}-go\tsc{-hab.pst=mod}	home	work\tsc{-loc-abl}\\
	\glt	\sqt{‎While he came from Icari, he went with a torch at night home from work.}

	\ex	\label{ex:‎‎‎When he was attentively listening he heard the sound of frogs}
	\gll	ʡaˁħ-ʡaˁħ-le	gulik'-an	zamana,	t'am	b-aq'-ib-le	ca-b	ʡaˁt'-n-a-lla	t'ama\\
		good-good\tsc{-advz}	listen\tsc{.ipfv-ptcp}	time	sound	\tsc{n-}hear\tsc{.pfv-pret-cvb}	be\tsc{-n}	frog\tsc{-pl-obl-gen}	sound\\
	\glt	\sqt{‎‎‎When he was attentively listening he heard the sound of frogs.}
\end{exe}

Clauses with the preterite participle have a comparable meaning of simultaneity:
%
\begin{exe}
	\ex	\label{ex:‎While he presented the condolences (lit. kept his hands in front), Zhapar's wife entered}
	\gll	nuˁq-be	sa-ka-d-uc-ib		zamana,	žaˁpar-ra	xːunul	qili-rka	tːura	sa-r-uq-un	ca-r\\
		arm\tsc{-pl}	\tsc{ante-down}\tsc{-npl-}keep\tsc{.pfv-pret}	time	Zhapar\tsc{-gen}	woman	home\tsc{-abl}	outside 	\tsc{hither-f-}go\tsc{.pfv-pret}	be\tsc{-f}\\
	\glt	\sqt{‎While he presented the condolences (lit. kept his hands in front), Zhapar's wife entered.}

	\ex	\label{ex:‎While I got up, my eyes remained on the porridge}
	\gll	hel	gu-r-ha-jcː-ur	zamana,	di-la	ul-be	het	kaš-le-d	kelg-un\\
		that	\tsc{down}\tsc{-abl-up}-stand\tsc{.m.pfv-pret}	time	\tsc{1sg-gen}	eye\tsc{-pl}	that	porridge\tsc{-loc-npl}	remain\tsc{.pfv-pret}\\
	\glt	\sqt{‎While I got up, my eyes remained on the porridge.}
\end{exe}

The \tit{zamana}-construction can be combined with the particle \tit{bah} \sqt{immediately when} that occurs in the initial position of the relative clause \refex{ex:‎When this tree already felt down, the dogs were coming}. The precise origin of \tit{bah} needs further investigation, but we might suggest that it is related to the superlative particle \tit{bah} \sqt{most} and to the adverbs \tit{bahsala, bahsar} \sqt{first}, which can be decomposed into \tit{bah-} and a following postposition. The particle can also co-occur with the enclitic \tit{=qːel(la)}.
%
\begin{exe}
	\ex	\label{ex:‎When this tree already felt down, the dogs were coming}
	\gll	bah	hel	kːalkːi	ka-b-irk-an	zamana,	či-sa-d-eʁ-ib	ca-d	χu-de\\
		immediately.when	that	tree	\tsc{down-n-}occur\tsc{.ipfv-ptcp}	time	\tsc{spr-hither}\tsc{-npl-}go\tsc{.pfv-pret}	be\tsc{-npl}	dog\tsc{-pl}\\
	\glt	\sqt{‎When this tree already fell down, the dogs were coming.}
\end{exe}


%%%%%%%%%%%%%%%%%%%%%%%%%%%%%%%%%%%%%%%%%%%%%%%%%%%%%%%%%%%%%%%%%%%%%%%%%%%%%%%%

\subsection{Minor ways of forming adverbial clauses}
\label{sec:minor ways of forming adverbial clauses}

The locative participle and the masdar can occur in adverbial clauses expressing causes when they take the dative suffix \refex{ex:ecause Maja was sleeping I (masc.) could not come.}, \refex{ex:This is for you, because you helped me out of the pit}, because the expression of causes is one of the functions of the dative (\refsec{sssec:Dative}):
%
\begin{exe}
	\ex	\label{ex:ecause Maja was sleeping I (masc.) could not come.}
	\gll	[Maˁʡaˁ	ka-r-ils-na-lli-j]	du	uq'-ij	a-jχ-ub=da\\
		Maja	\tsc{down-f-}sleep\tsc{.ipfv-ptcp.loc-obl-dat}	\tsc{1sg}	go\tsc{.m-inf}	\tsc{neg-}be.able\tsc{.m.pfv-pret=1}\\
	\glt	\sqt{‎‎‎Because Maja was sleeping I (masc.) could not come.} (E)

	\ex	\label{ex:This is for you, because you helped me out of the pit}
	\gll	hež	at	ca-b,	du	kur-ri-cːe-r	tːura	ha-qː-ni-li-j\\
		this	\tsc{2sg.dat}	be\tsc{-n}	\tsc{1sg}	pit\tsc{-obl-in-abl}	outside	\tsc{up}-carry\tsc{.pfv-msd-obl-dat}\\
	\glt	\sqt{This is for you, because you helped me out of the pit.}
\end{exe}

When the postposition \tit{bahanne\slash bahandan} \sqt{because of} (\refsec{ssec:postposition bahanne}) follows the masdar, the resulting clause also expresses causation \refex{ex:‎Because they cut it, they died}. By contrast, when it follows the infinitive or the subjunctive we get purpose clauses \refex{ex:‎in order to shake those, in order to shake the threshing boards}:
%
\begin{exe}
	\ex	\label{ex:‎Because they cut it, they died}
	\gll	ka-b-ičː-ni	bahanne	b-ebč'-ib\\
		\tsc{down-n-}cut.up\tsc{.pfv-msd}	because.of	\tsc{hpl-}die\tsc{.pfv-pret}\\
	\glt	\sqt{‎Because they cut it, they died.}


	\ex	\label{ex:‎in order to shake those, in order to shake the threshing boards}
	\gll	hel-tːi	ce	hak'	ka-d-arq'-ar-aj	bahanne	irk-me	hak'	ka-d-arq'-ij	bahanne\\
		that\tsc{-pl}	what	shake	\tsc{down-npl-}do\tsc{.pfv-prs-subj.3}	in.order.to	threshing.board\tsc{-pl}	shake	\tsc{down-npl}-do.\tsc{pfv-inf}	in.order.to\\
	\glt	\sqt{‎in order to shake those, in order to shake the threshing boards}
\end{exe}


%\begin{exe}
%	\ex	\label{ex:}
%	\gll	\\
%		\\
%	\glt	\sqt{}
%\end{exe}

\section{Conditional and concessive verb forms and clauses}
\label{cpt:conditionalconcessiveclauses}

Conditional and concessive clauses are adverbial clauses that contain specialized verb forms expressing realis and irrealis conditional and concessive meaning. All verb forms here are obtained by means of suffixes that are added to a stem augmentation vowel. The vowel is the same that is used for synthetic verb forms (\refcpt{cpt:verbs-indicativesynthetic}) and some non-declarative verb forms (\refcpt{cpt:verbs-nondeclarative}) and not separately glossed in the examples. The suffixes express conditional meaning and person agreement and bear resemblance to the suffixes of the synthetic tenses. The following forms are treated in this section:
%
\begin{itemize}
	\item	realis conditional (\refsec{sec:realisconditional}) 
	\item	past conditional (\refsec{sec:pastconditional})
	\item	imperfective realis conditional (\refsec{sec:imperfectiverealisconditional})
	\item	imperfective past conditional (\refsec{sec:imperfectivepastconditional})
	\item	periphrastic conditional clauses (\refsec{sec:periphrasticconditionalclauses})
	\item	concessive conditionals (\refsec{sec:concessiveconditionals})
\end{itemize}

All conditional forms head dependent clauses, thus they are normally followed by a main clause. The conditional suffixes alone suffice to convey conditional meaning, but optionally the conjunction \textit{raχle} `if' can co-occur in conditional clauses \refex{ex:If (he) caught him, he must stand there}. However, the use of the subordinating conjunction is rare.

\begin{exe}
	\ex	\label{ex:If (he) caught him, he must stand there}
	\gll	raχle	uc-arre	het	k-ercː-an	ca-w	heštːu\\
		if	catch\tsc{.m.pfv-cond.3}	that	\tsc{down}-stand\tsc{.ipfv-ptcp}	be\tsc{-m}	here\\
	\glt	\sqt{If (he) caught him, he must stand there.}
\end{exe}

For more information on the general syntactic properties of adverbial clauses see \refcpt{cpt:Syntactic properties of adverbial and conditional clauses}.


%%%%%%%%%%%%%%%%%%%%%%%%%%%%%%%%%%%%%%%%%%%%%%%%%%%%%%%%%%%%%%%%%%%%%%%%%%%%%%%%

\subsection{Realis conditional}
\label{sec:realisconditional}

The realis conditional is formed from perfective verb stems (for those verbs that occur in pairs of imperfective and perfective stems). To the verbal stem the stem augments (vowels \tit{-u} or \tit{-i}) are added, followed by the conditional suffixes (\reftab{tab:realisconditional}). In the second person singular, there are two variants possible, \tit{-tːe} and \tit{-tːel}, but the first is clearly preferred. In the third person, there is again largely lexically determined allomorphy between the suffixes \tit{-an} and \tit{-ar}. The latter suffix has a longer variant \tit{-arre}, but the shorter variant is more common. In negative realis conditional clauses the verb bears the negative prefix \tit{a-} \refex{ex:‎‎‎If you (= masc.) do not calm down, I make you calm, he says}.
%
\begin{table}
	\caption{The realis conditional}
	\label{tab:realisconditional}
	\small
	\begin{tabularx}{0.40\textwidth}[]{%
		>{\centering\arraybackslash}p{10pt}
		>{\centering\arraybackslash}X
		>{\centering\arraybackslash}X}
		
		\lsptoprule
			{}	&	\tsc{sg}	&	\tsc{pl}\\
		\midrule
			1	&	\multicolumn{2}{c}{\tit{-lle}}\\
			2	&	\tit{-tːe(l)}	&	\tit{-tːal}\\
			3	&	\multicolumn{2}{c}{\tit{-ar(re)\slash -an}}\\
		\lspbottomrule
	\end{tabularx}
\end{table}
%
\begin{table}
	\renewcommand{\tit}[1]{\mbox{\textit{#1}}}
	\caption{Some illustrative paradigms of the realis conditional}
	\label{tab:realisconditional-examples}
	\small
	\begin{tabularx}{1\textwidth}[]{%
		>{\centering\arraybackslash\small}p{10pt}
		>{\raggedright\arraybackslash}X
		>{\raggedright\arraybackslash}X
		>{\raggedright\arraybackslash}X
		>{\raggedright\arraybackslash}X
		>{\raggedright\arraybackslash}X
		>{\raggedright\arraybackslash}X}
		
		\lsptoprule
			{}	&	\multicolumn{2}{c}{\sqt{say}}
				&	\multicolumn{2}{c}{\sqt{do}}
				&	\multicolumn{2}{c}{\sqt{know}}\\

			{}	&	\multicolumn{1}{c}{\tsc{sg}} &	\multicolumn{1}{c}{\tsc{pl}}
				&	\multicolumn{1}{c}{\tsc{sg}} &	\multicolumn{1}{c}{\tsc{pl}}
				&	\multicolumn{1}{c}{\tsc{sg}} &	\multicolumn{1}{c}{\tsc{pl}}\\

		\midrule

			1	&	\tit{r-ik'-u-lle}	&	\tit{d-ik'-u-lle}
				&	\tit{b-arq'-i-lle}	&	\tit{b-arq'-i-lle}
				&	\tit{b-aχ-i-lle}	&	\tit{b-aχ-i-lle}\\

			2	&	\tit{r-ik'-u-tːe(l)}	&	\tit{d-ik'-u-tːal}
				&	\tit{b-arq'-i-tːe(l)}	&	\tit{b-arq'-i-tːal}
				&	\tit{b-aχ-i-tːe(l)}	&	\tit{b-aχ-i-tːal}\\

			3	&	\tit{r-ik'ʷ-arre}	&	\tit{b-ik'ʷ-arre}
				&	\tit{b-arq'-ar(re)}	&	\tit{b-arq'-ar(re)}
				&	\tit{b-aχ-ar(re)}	&	\tit{b-aχ-ar(re)}\\
		\lspbottomrule
	\end{tabularx}
\end{table}

The function of the conditional is the expression of real non-past conditions: 
%
\begin{exe}
	\ex	\label{ex:‎The bride who just married}
	\gll	``hana	ka-r-iž-ib	c'ikuri-li,''	Ø-ik'-ul	ca-w,	``misa či-b-ikː-arre	urχːab	kʷir	ka-b-irg-an-ne''\\
		now	\tsc{down-f}-sit.\tsc{pfv}-\tsc{pret}	bride\tsc{-erg}	\tsc{m-}say\tsc{.ipfv-icvb}	be\tsc{-m}	mouth	\tsc{spr-n-}give\tsc{.pfv-}\tsc{cond.3}	mill	stop	\tsc{down-n}-be.\tsc{ipfv-ptcp-fut.3}\\
	\glt	\sqt{``‎The bride who just married,'' says one, ``if she kisses the mill it will stop.''}

	\ex	\label{ex:If I do not read (my song), I burst inside.}
	\gll	a-b-elč'-ille,	w-ark	Ø-utː-ud\\
		\tsc{neg-n-}read\tsc{.pfv-cond.1}	\tsc{m-}inside	\tsc{m-}burst\tsc{.ipfv-prs.1}\\
	\glt	\sqt{If I do not read (my song), I (masc.) burst inside.}

	\ex	\label{ex:‎‎‎If you (= masc.) do not calm down, I make you calm, he says}
	\gll	``a-w-g-utːe,	u	parʁat	Ø-irq'-an=de,''	Ø-ik'-ul	ca-w	ik'\\
		\tsc{neg-m-}stay\tsc{-cond.2sg}	\tsc{2sg}	quiet	\tsc{m-}do\tsc{.ipfv-ptcp=2sg}	\tsc{m-}say\tsc{.ipfv-icvb}	be\tsc{-m}	\tsc{dem.up}\\
	\glt	\sqt{``‎‎‎If you (= masc.) do not calm down, I make you calm,'' he says.}
\end{exe}

It also occurs utterances in which the conditional is not a condition for the apodosis because there is no conditional connection between the two clauses. This includes the common idiomatic expression \sqt{to be honest} (lit. \sqt{if I tell correctly}) \refex{ex:I, to be honest, remained for three years.}.
%
\begin{exe}
	\ex	\label{ex:‎‎There is much there (i.e. the graveyard is large), if you go there}
	\gll	celi d-aqil	k'e-d,	či-d-uˁq'-uˁtːal\\
		whole	\tsc{npl-}much	exist\tsc{.up-npl}		\tsc{spr-1/2pl-}go\tsc{.pfv-cond.2pl}\\
	\glt	\sqt{‎‎There is much there (i.e. the graveyard is large), if you go there.}

	\ex	\label{ex:I, to be honest, remained for three years.}
	\gll	du,	b-arx-le	b-urs-ille	ʡaˁbal	dus	kelg-un=da\\
		\tsc{1sg}	\tsc{n-}direct\tsc{-advz}	\tsc{n-}say\tsc{-cond.1}		three	year	remain\tsc{.pfv-pret=1}\\
	\glt	\sqt{I, to be honest, remained for three years.}
\end{exe}

As with the two indicative analytic verb forms, the habitual present and the habitual past, in conditional clauses ergative alignment is, in addition to the dative construction, possible with some affective verbs. 


%%%%%%%%%%%%%%%%%%%%%%%%%%%%%%%%%%%%%%%%%%%%%%%%%%%%%%%%%%%%%%%%%%%%%%%%%%%%%%%%

\subsection{Past conditional}
\label{sec:pastconditional}

The past conditional bears strong formal resemblances to the realis conditional and the second person is almost identical for both conditional forms (\reftab{tab:pastconditional}). Before the conditional past suffixes the stem augment vowels occur that are the same as for the realis conditional and for a number of other verb forms such as the subjunctive. In the third person, the first part of the two allomorphic suffixes \tit{-ar-del} and \tit{-an-del} is identical with the suffixes used in the realis conditional (\reftab{tab:realisconditional}). The second part probably originates from the past enclitic \tit{=de}. Negation is marked with the prefix \tit{a-}. Only perfective verb stems can function as the basis for the past conditional.
%
\begin{table}
	\caption{The past conditional}
	\label{tab:pastconditional}
	\small
	\begin{tabularx}{0.40\textwidth}[]{%
		>{\centering\arraybackslash}p{10pt}
		>{\centering\arraybackslash}X
		>{\centering\arraybackslash}X}
		
		\lsptoprule
			{}	&	\tsc{sg}	&	\tsc{pl}l\\
		\midrule
			1	&	\multicolumn{2}{c}{\tit{-tːel}}\\
			2	&	\tit{-tːel}	&	\tit{-tːal}\\
			3	&	\multicolumn{2}{c}{\tit{-ar-del\slash -an-del}}\\
		\lspbottomrule
	\end{tabularx}
\end{table}

The semantic range of the past conditional comprises the expression of realis conditions that obtained in the past.
%
\begin{exe}
	\ex	\label{ex:If we were too late}
	\gll	raχle	q'an	d-iχ-utːel, \ldots\\
		if	late	\tsc{1/2pl-}be\tsc{.pfv-cond.pst.1}\\
	\glt	\sqt{If we were too late, \ldots}

	\ex	\label{ex:If you hold the tail (of the fish) and lift it up and shake it}
	\gll	č'imi	b-uc-ib-le,	aq	b-arq'-ib-le,	hak'	ka-b-arq'-itːel,	dig-be	k-arž-i	skelet	kalž-i\\
		tail	\tsc{n-}catch\tsc{.pfv-pret-cvb}	tall	\tsc{n-}do\tsc{.pfv-pret-cvb} shake	\tsc{down-n-}do\tsc{.pfv-cond.pst.2}	meat\tsc{-pl}	\tsc{down}-go.\tsc{ipfv-hab.pst}	skeleton	remain\tsc{-hab.pst}\\
	\glt	\sqt{If you held the tail (of the fish) and lifted it up and shook it, the meat fell down and the skeleton remained.}

	\ex	\label{ex:When/if graves fell down, (grandfather) put them up again}
	\gll	χːuˁrbe	ka-d-ik-ardel, 	χːuˁrbe	ʡaˁħ	d-irq'-ul=de\\
		graves	\tsc{down-npl-}occur\tsc{.pfv-cond.pst} graves	good	\tsc{1/2.pl-}do\tsc{.ipfv-icvb=pst}\\
	\glt	\sqt{When/if graves fell down, (grandfather) put them up again.} (lit. made them good)
\end{exe}

Furthermore, it conveys irrealis conditional meanings, i.e. conditions with low probability and counterfactual conditions \refex{ex:‎if he would have said it with his own mouth}, and those sentences can lack the apodosis \refex{ex:‎‎‎If they would go! They bore (me)}, \refex{ex:if (you) would eat bread made of rye, it is the best thing for you}. The apodosis of past conditional clauses often contains a verb marked for future in the past (\refsec{ssec:Future in the past}) or habitual past (\refsec{sec:vis-habitualpast}) \refex{ex:If you hold the tail (of the fish) and lift it up and shake it}.
%
\begin{exe}
	\ex	\label{ex:‎if he would have said it with his own mouth}
	\gll	cin-ni	b-urs-ardel=ra	cin-na	tːutːu-l	\ldots\\
		\tsc{refl.sg-erg}	\tsc{n-}tell\tsc{-cond.pst=add}	\tsc{refl.sg-gen}	beak\tsc{-erg}\\
	\glt	\sqt{‎if he would have said it with his own mouth, \ldots}

	\ex	\label{ex:‎‎‎If they would go! They bore (me)}
	\gll	b-uˁq'-aˁndel!	či-b-b-et'-ib	ca-b\\
		\tsc{hpl-}go\tsc{-cond.pst}	\tsc{spr-hpl-hpl-}bore\tsc{.pfv-pret}	be\tsc{-hpl}\\
	\glt	\sqt{‎‎‎If they would go! They bore (me).} (E)

	\ex	\label{ex:if (you) would eat bread made of rye, it is the best thing for you}
	\gll	sːusːul-la	t'ult'	b-erkʷ-itːel=ra,	at	bahlalla	ʡaˁħ-ce	ca-b	žan-ni-j,	q'arq'ala-li-j\\
		rye\tsc{-gen}	bread	\tsc{n-}eat\tsc{.pfv-cond.pst.2=add}	\tsc{2sg.dat}	most\tsc{.emph}	good\tsc{-dd.sg}	be\tsc{-n}	body\tsc{-obl-dat}	body\tsc{-obl-dat}\\
	\glt	\sqt{If (you) would eat bread made of rye, it is the best thing for you, for the body, the organism.}
\end{exe} 


%%%%%%%%%%%%%%%%%%%%%%%%%%%%%%%%%%%%%%%%%%%%%%%%%%%%%%%%%%%%%%%%%%%%%%%%%%%%%%%%

\subsection{Imperfective realis conditional}
\label{sec:imperfectiverealisconditional}

The imperfective realis conditional is formed from imperfective verb stems (for those verbs that occur in pairs of imperfective and perfective stems) by means of the suffix \tit{-aχː}, followed by the vowel \tit{-a} that functions as a stem augment without expressing transitivity, and finally by (almost) the same person suffixes that are used for the realis conditional (\refsec{sec:realisconditional}).  As with all conditional forms treated in this Section, negation is marked by means of \tit{a-} \refex{ex:If I cannot educate (up bring) her (myself), she said, then the life is of no need for me.}. Some affective verbs can occur in the dative experiencer construction and in the ergative construction when inflected for the imperfective realis conditional.
%
\begin{table}
	\caption{The imperfective realis conditional}
	\label{tab:imperfectiverealisconditional}
	\small
	\begin{tabularx}{0.40\textwidth}[]{%
		>{\centering\arraybackslash}p{10pt}
		>{\centering\arraybackslash}X
		>{\centering\arraybackslash}X}
		
		\lsptoprule
			{}	&	\tsc{sg}	&	\tsc{pl}\\
		\midrule
			1	&	\multicolumn{2}{c}{\tit{-aχː-a-lle}}\\
			2	&	\tit{-aχː-a-t(te)}	&	\tit{-aχː-a-t(tal) }\\
			3	&	\multicolumn{2}{c}{\tit{-aχː-a-n(ne)\slash -aχː-a-r(re)}}\\
		\lspbottomrule
	\end{tabularx}
\end{table}
%
\begin{table}
	\renewcommand{\tit}[1]{\mbox{\textit{#1}}}
	\caption{Some illustrative paradigms of the imperfective realis conditional}
	\label{tab:realisconditional-examples}
	\small
	\begin{tabularx}{0.9\textwidth}[]{%
		>{\centering\arraybackslash\small}p{10pt}
		>{\raggedright\arraybackslash}X
		>{\raggedright\arraybackslash}X
		>{\raggedright\arraybackslash}X
		>{\raggedright\arraybackslash}X}
		
		\lsptoprule
			{}	&	\multicolumn{2}{c}{\sqt{say}}
				&	\multicolumn{2}{c}{\sqt{do}}\\

			{}	&	\multicolumn{1}{c}{\tsc{sg}} &	\multicolumn{1}{c}{\tsc{pl}}
				&	\multicolumn{1}{c}{\tsc{sg}} &	\multicolumn{1}{c}{\tsc{pl}}\\

		\midrule

			1	&	\tit{r-ik'ʷ-aχː-alle}	&	\tit{d-ik'ʷ-aχː-alle }
				&	\multicolumn{2}{c}{\tit{b-irq'-aχː-alle}}\\

			2	&	\tit{r-ik'ʷ-aχː-at}	&	\tit{d-ik'ʷ-aχː-t(tal)}
				&	\tit{b-irq'-aχː-at(te)}	&	\tit{b-irq'-aχː-atːal}\\

			3	&	\tit{r-ik'ʷ-an(ne)}	&	\tit{b-ik'ʷ-an(ne)}
				&	\multicolumn{2}{c}{\tit{b-irq'-aχː-an(ne)}}\\
		\lspbottomrule
	\end{tabularx}
\end{table}

The imperfective realis conditional is basically the imperfective counterpart of the realis conditional. According to Sanzhi speakers, it covers the same meanings, with the only difference being the aspectual value that the stem carries. Thus, we have realis conditional semantics with present and future time reference \refex{ex:If I cannot educate (up bring) her (myself), she said, then the life is of no need for me.}, \refex{ex:‎‎‎If she is not able to carry those (sacks), of what use is she for me?} and occassionally in utterance in which no genuine conditional semantics is expressed \refex{ex:After having pulled (him) out, if they look, there is not head}. 
%
\begin{exe}
	\ex	\label{ex:If I cannot educate (up bring) her (myself), she said, then the life is of no need for me.}
	\gll	``hej	ha-r-iq'-ij	a-r-irχʷ-aχː-alle,''	r-ik'ʷ-ar,	``dam	ʡuˁmru	ħaˁžat-le=kːu''\\
		this	\tsc{up-f-}bring.up\tsc{-inf}	\tsc{neg-f-}be.able\tsc{.ipfv-cond-cond.1}	\tsc{f-}say\tsc{.ipfv-prs}		\tsc{1sg.dat}	life	need\tsc{-advz=neg}\\
	\glt	\sqt{``If I cannot educate (i.e. bring up) her (myself),'' she said, ``then life is of no need for me.''}

	\ex	\label{ex:‎‎‎If she is not able to carry those (sacks), of what use is she for me?}
	\gll	hel-tːi	ha-qː-ij	a-r-irχʷ-aχː-an		il	ce	r-irq'-an=e	dam?\\
		that\tsc{-pl}	\tsc{up}-carry\tsc{.pfv-inf}	\tsc{neg-f-}be.able\tsc{.ipfv-cond-cond.prs.3}	that	what	\tsc{f-}do\tsc{.ipfv-ptcp=q}	\tsc{1sg.dat}\\
	\glt	\sqt{‎‎‎If she is not able to carry those (sacks), of what use is she for me?} (i.e. a wife that is unable to carry the sacks of flour is useless)

	\ex	\label{ex:After having pulled (him) out, if they look, there is not head}
	\gll	qus	tːura-k-aˁq-ib-le,	er	Ø-ik'ʷ-aχː-an,	il	bek'	b-akːu	\\
		slip	\tsc{out-down}-drag\tsc{.pfv-pret-cvb}	look	\tsc{m-}look.at\tsc{.ipfv-cond-cond.prs.3}	that		head	\tsc{n-}be\tsc{.neg}\\
	\glt	\sqt{After having pulled (him) out, if they look, there is no head.}
\end{exe}

The verbs that do not have an aspectual distinction can form the realis conditional as well as the imperfective realis conditional without any noticeable semantic difference between the two forms \refex{ex:‎‎‎if we go (E)}. For verbs with two aspectual stems the semantic difference is restricted to the aspectual difference between imperfective and perfective aspect; the conditional meaning is identical for both forms \refex{ex:‎ if I see the horse (regularly / once) (E)}.
%
\begin{exe}
	\ex	\label{ex:‎‎‎if we go (E)}
	\gll	nušːa	d-ax-ulle /		d-ax-aχː-alle\\
		\tsc{1pl}	\tsc{npl-}go\tsc{-cond.1}	/ \tsc{npl-}go\tsc{-cond-cond.1}\\
	\glt	\sqt{‎‎‎if we go} (E)

	\ex	\label{ex:‎ if I see the horse (regularly / once) (E)}
	\gll	dam	urči	či-b-ig-aχː-alle	\quad/	či-b-až-ille\\
		\tsc{1sg.dat}	horse	\tsc{spr-n-}see\tsc{.ipfv-cond-cond.1}	\quad/ \tsc{spr-n-}see\tsc{.pfv-cond.1}\\
	\glt	\sqt{‎if I see the horse (regularly\slash once)} (E)
\end{exe}


%%%%%%%%%%%%%%%%%%%%%%%%%%%%%%%%%%%%%%%%%%%%%%%%%%%%%%%%%%%%%%%%%%%%%%%%%%%%%%%%

\subsection{Imperfective past conditional}
\label{sec:imperfectivepastconditional}

There is also a past version of the imperfective conditional formed only from imperfective verbs. The precise formal make-up is still to be clarified since the form is only very rarely used. There are no corpus examples and elicitation is hard due to the insecurity of the speakers. It seems that the suffix \tit{-aχː-an-del} can be used with all persons. It expresses irrealis conditional \refex{ex:‎‎‎If a car would go to the sea, we would go.}, \refex{ex:If I would see a nice car, I would want it.} and past conditional meaning \refex{ex:At that time when I saw a nice car in the city, I wanted it.}, depending on the sentence and the further context.
%
\begin{exe}
	\ex	\label{ex:‎‎‎If a car would go to the sea, we would go.}
	\gll	mašin	b-ax-aχː-andel	urx-n-a-cːe	nušːa	d-ax-adi\\
		car	\tsc{n-}go\tsc{-cond-cond.pst}	sea\tsc{-pl-obl-in}	\tsc{1pl}	\tsc{1/2pl-}go\tsc{-hab.pst.1}\\
	\glt	\sqt{‎‎‎If a car would go to the sea, we would go.} (i.e. if somebody would go to the sea by car, we would go with him.) (E) 

	\ex	\label{ex:If I would see a nice car, I would want it.}
	\gll	dam	qːuʁa	mašin	či-b-ig-aχː-andel,	b-ikː-ul	hajq-i\\
		\tsc{1sg.dat}	beautiful	car	\tsc{spr-n-}see\tsc{.ipfv-cond-cond.pst}	\tsc{n-}want\tsc{.ipfv-icvb}	be.enough\tsc{.ipfv-hab.pst.3}\\
	\glt	\sqt{If I would see a nice car, I would want it.}  (E)

	\ex	\label{ex:At that time when I saw a nice car in the city, I wanted it.}
	\gll	dam	šahar-ri-cːe-b	het=qːel qːuʁa	mašin	či-b-ig-aχː-andel,	b-ikː-ul	hajq-i\\
		\tsc{1sg.dat}	town\tsc{-obl-in-n} \tsc{dem=}when beautiful	car	\tsc{spr-n-}see\tsc{.ipfv-cond-cond.pst}	\tsc{n-}want\tsc{.ipfv-icvb}	be.enough\tsc{.ipfv-hab.pst.3}\\
	\glt	\sqt{At that time when I saw a nice car in the city, I wanted it.} (But now I do not care about cars anymore) (E)
\end{exe}


%%%%%%%%%%%%%%%%%%%%%%%%%%%%%%%%%%%%%%%%%%%%%%%%%%%%%%%%%%%%%%%%%%%%%%%%%%%%%%%%

\subsection{Periphrastic conditional clauses}
\label{sec:periphrasticconditionalclauses}

As shown in \refex{ex:‎If I do not tell it correctly, correct me!}-\refex{ex:If the woman had stolen, they would/should have imprisoned her}, conditional clauses can be periphrastic, i.e., make use of the additional auxiliary \tit{b-iχʷ-} \tsc{(pfv)} \sqt{be, become, can}. In such clauses, the lexical verb bears a converb or occasionally a participial suffix and the auxiliary \tit{b-iχʷ-} takes one of the conditional forms, e.g. realis conditional \refex{ex:‎If I do not tell it correctly, correct me!} or past conditional\slash irrealis conditional \refex{ex:There must have been people there}-\refex{ex:If the woman had stolen, they would/should have imprisoned her}. More examples can be found in \refsec{ssec:Periphrastic conditionals}, which describes all uses of \tit{b-iχʷ-} as auxiliary.
%
\begin{exe}
	\ex	\label{ex:‎If I do not tell it correctly, correct me!}
	\gll	du	b-arx-le	Ø-ik'-ul	a-jχ-ulle	raχle	u-l	ʡaˁħ	či-b-arq'-a!	\\
		\tsc{1sg}	\tsc{n-}direct\tsc{-advz}	\tsc{m-}say\tsc{.ipfv-icvb}	\tsc{neg-}be\tsc{.pfv-cond.1}	if	\tsc{2sg-erg}	good	\tsc{spr-n-}do\tsc{.pfv-imp}	\\
	\glt	\sqt{‎If (masc.) I do not tell it correctly, correct me!}

	\ex	\label{ex:Let's go if you did not go there!}
	\gll	w-aš-e	a-ag-ur-il	Ø-iχ-utːe!\\
		\tsc{m-}go\tsc{.ipfv-imp}	\tsc{neg-}go\tsc{.pfv-pret-ref}	\tsc{m-}be\tsc{.pfv-cond.2sg}\\
	\glt	\sqt{Let's go if you (masc.) did not go there!}

	\ex	\label{ex:There must have been people there}
	\gll	itːu-b	adim-te	te-b	b-irχʷ-an=de.	a-b-iχʷ-ardel,	itːi	ʁaj	a-d-ik'ʷ-an=de\\
		there\tsc{-hpl}	person\tsc{-pl}	exist\tsc{-hpl}	\tsc{hpl-}be\tsc{.ipfv-ptcp=pst}		\tsc{neg-hpl-}be\tsc{.pfv-cond.pst}	\tsc{3pl}	word	\tsc{neg-1/2pl-}say\tsc{.ipfv-ptcp=pst}\\
	\glt	\sqt{There must have been people there. If there were (no people there), you would not have said so.}

	\ex	\label{ex:‎If I (masc.) had known that I will look at them, I had brought my glasses}
	\gll	iš-tː-a-j	er	Ø-ik'ʷ-ni	b-alχ-ul	Ø-iχ-utːel,	ulbasne d-alli	ha-d-iqː-adi=q'al\\
		this\tsc{-pl-obl-dat}	look	\tsc{m-}look.at\tsc{.ipfv-msd}	\tsc{n-}know\tsc{.ipfv-icvb}	\tsc{m-}be\tsc{.pfv-cond.pst.1}	glasses	\tsc{npl-}together	\tsc{up-npl-}carry\tsc{.ipfv-hab.pst.1=mod}\\
	\glt	\sqt{‎If I (masc.) had known that I will look at them, I would have brought my glasses.}

	\ex	\label{ex:If the woman had stolen, they would/should have imprisoned her}
	\gll	r-ilʡ-uˁn-ne	r-iχʷ-ardel,	xːunul	r-i-ka-jʁ-an=de=q'al\\
		\tsc{f-}steal\tsc{.ipfv-icvb-cvb}	\tsc{f-}be\tsc{.pfv-cond.pst}	woman	 \tsc{f-in-down}-drive\tsc{.pfv-ptcp=pst=mod}\\
	\glt	\sqt{If the woman had stolen, they would/should have imprisoned her.}
\end{exe}


%%%%%%%%%%%%%%%%%%%%%%%%%%%%%%%%%%%%%%%%%%%%%%%%%%%%%%%%%%%%%%%%%%%%%%%%%%%%%%%%

\subsection{Concessive conditionals}
\label{sec:concessiveconditionals}

The conditional forms presented in the preceding sections can acquire a concessive conditional meaning (\sqt{even if}) when the additive is encliticized to the conditional suffixes. For instance, the realis conditional \refex{ex:‎‎‎Even if I die today, I am not worried, she said} or the past conditional \refex{ex:‎‎Even if I had money, I would not buy a car} can serve as the base for concessives.
%
\begin{exe}
	\ex	\label{ex:‎‎‎Even if I die today, I am not worried, she said}
	\gll	``ižal	r-ebk'-ulle=ra	awara	b-akːu,''	r-ik'ʷ-ar\\
		today	\tsc{f-}die\tsc{.pfv-cond.1=add}	worries	\tsc{n-}be\tsc{.neg}	\tsc{f-}say\tsc{.ipfv-prs.3}\\
	\glt	\sqt{``‎‎‎Even if I die today, I am not worried,'' she said.}

	\ex	\label{ex:‎‎Even if I had money, I would not buy a car}
	\gll	di-la	arc	d-iχʷ-ardel=ra		du-l	mašin	a-jsː-adi\\
		\tsc{1sg-gen}	money	\tsc{npl-}be\tsc{.pfv-cond.pst=add}	\tsc{1sg-erg}	car	\tsc{neg-}buy\tsc{.pfv-hab.pst.1}\\
	\glt	\sqt{‎‎Even if I had money, I would not buy a car.} (E)
\end{exe}

The third person concessive conditional form of the auxiliary \tit{b-iχʷ-} (\tsc{n-}be\tsc{.pfv}) used in combination with interrogative pronouns lexicalized into a universal indefinite free choice pronoun similar to the English \tit{-ever} series \refex{ex:Are you able to do whatever? he says} (\refsec{ssec:Free-choice indefinite pronouns}). Similarly, the verb \tit{b-ikː-} \sqt{like, want, love} can function as universal indefinite free choice when it takes a concessive conditional form and co-occurs with an interrogative pronoun \refex{ex:How often whoever you love, I am your compass}.
%
\begin{exe}
	\ex	\label{ex:Are you able to do whatever? he says}
	\gll	``u-l	b-arq'-ij	w-irχ-utːe=w''	w-ik'-ul	ca-w	``ce-k'a	b-iχʷ-ar=ra?''\\
		\tsc{2sg-erg}	\tsc{n-}do\tsc{.pfv-inf}	\tsc{m-}be.able\tsc{.ipfv-prs.2sg=q}	\tsc{m-}say\tsc{.ipfv-icvb}	be\tsc{-m}	what\tsc{-indef}		\tsc{n-}be\tsc{.pfv-cond.3=add}\\
	\glt	\sqt{``Are you able to do whatever?'' he says.}

	\ex	\label{ex:How often whoever you love, I am your compass}
	\gll	čujna	ča	w-ikː-aχː-at=ra		du=da	ala	q'iblama\\
		how.often	who	\tsc{m-}want\tsc{.ipfv-cond-cond.2=add}	\tsc{1sg=1}	\tsc{2sg.gen}	compass\\
	\glt	\sqt{How often whoever you love, I am your compass.}
\end{exe}

However, conditional forms with an additional additive enclitic do not always express conditional concessive meaning. For present conditional forms the concessive semantics can be very weak \refex{ex:If you let the ashes of your cigarette fall down} or even absent, in which case only the conditional meaning is conveyed. For past conditionals the meaning is irrealis conditional instead of concessive \refex{ex:if (you) would eat bread made of rye, it is the best thing for you}.
%
\begin{exe}
	\ex	\label{ex:If you let the ashes of your cigarette fall down}
	\gll	hel=de		hel	pepel	p'aq'	ka-b-arq'-itːe=ra,	``uberi!''	b-ik'-ul=de	``hetːi''\\
		that\tsc{=pst}	that	ashes	shake.off	\tsc{down-n}-do.\tsc{pfv-cond.2sg=add}	take.away	\tsc{hpl-}say\tsc{.ipfv-icvb=pst}	those\\
	\glt	\sqt{(Even) if you let the ashes of your cigarette fall down, they said, ``Put it away!''} (i.e. make it clean)
\end{exe}

There is another way of forming concessive clauses by means of the enclitic \tit{=xːar} (\refsec{sec:concessive enclitic =xar}).




%\begin{exe}
%	\ex	\label{ex:}
%	\gll	\\
%		\\
%	\glt	\sqt{}
%\end{exe}
