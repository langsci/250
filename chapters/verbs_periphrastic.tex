\chapter{Periphrastic verb forms}
\label{cpt:Periphrastic verb forms}

Periphrastic verb forms are morphologically complex in the same manner as \is{analytic verb form}analytic verb forms (\refcpt{cpt:Analytic verb forms}) and make use of the same range of non-finite inflectional forms (perfective and \isi{imperfective converb}, and occasionally \is{participle}participles), but employ different auxiliaries that have, by themselves, particular semantic values. Therefore, the resulting verb forms differ in their meaning from the \is{analytic verb form}analytic verb forms. The auxiliaries employed are:

\begin{itemize}
	\item	locational copulas (\refsec{sec:Verb forms with locational copulae})
	\item	\tit{kelg-} \sqt{remain} (\refsec{sec:Verb forms with kelgw- remain})
	\item	\tit{b-el} `remain, stay' (\refsec{sec:Verb forms with b-el remain, stay})
	\item	\tit{b-iχʷ-} \sqt{be, become, be able} (\refsec{sec:Verb forms with the auxiliary b-irxw- (ipfv) / b-ixw- (pfv) be, become, be able})
	\item	\tit{b-urkː-} \sqt{find} (\refsec{sec:Epistemic modality with the auxiliary b-urk find})
	\item	\tit{b-už-} \sqt{be, be at, stay, remain} (\refsec{sec:Indirect evidentiality with the auxiliary b-uz be, be at, stay, remain})
\end{itemize}

Some of the resulting verb forms have similar meanings although the auxiliaries differ. However, I will take a form-to-function approach and treat all formally distinct combinations of lexical verbs and auxiliaries separately.

There is a very large \isi{number} of morphologically complex verb forms that can, in theory, be produced and can thus be obtained in elicitation, because various auxiliaries can be employed and partially combined. But since it is impossible to gain an understanding of verb forms if one has only one or two elicited examples, I restrict myself to the examination of commonly attested periphrastic forms and describe the meaning of these forms based on their occurrences in natural texts.

The auxiliaries are inflected according to their morphological possibilities (i.e. the existential copulas have reduced paradigms, see \refsec{sec:Verb forms with locational copulae}). Consequently, the auxiliaries can themselves be inflected for verb forms heading subordinate clauses, that is, there are also \is{periphrastic verb form}periphrastic verb forms that occur in subordinate clauses.


%%%%%%%%%%%%%%%%%%%%%%%%%%%%%%%%%%%%%%%%%%%%%%%%%%%%%%%%%%%%%%%%%%%%%%%%%%%%%%%%

\section{Verb forms with locational copulas}
\label{sec:Verb forms with locational copulae}

Sanzhi has four locational copulas that are morphologically defective in a way similar to the standard \isi{copula}. They have locational and existential meaning that includes elevation (\refsec{sec:Locational copulae}). The use of locational copulas (instead of the standard \isi{copula}) for the formation of \is{periphrastic verb form}periphrastic verb forms is not extremely frequent, but it is repeatedly attested. The most widely used \isi{locational copula} verb is \tit{le-b} \sqt{be located\slash exist close to the speaker and the hearer}, because its semantics is somewhat less specific in comparison to the other three \isi{locational copula} verbs, and because of its meaning of proximity. The other three locational copulas are \tit{te-b} \sqt{be located\slash exist away from the speaker}, \tit{k'e-b} \sqt{be located\slash exist above the deictic center}, and \tit{χe-b} \sqt{be located\slash exist below the deictic center}. The semantics of the locational copulas partially determine the meaning of the \is{periphrastic verb form}periphrastic verb forms. For example, the use of \tit{le-w} in \refex{ex:Ilja is going downwards periphrastic} implies that the situation took place close to the speaker, and that the speaker consequently saw the event with her own eyes. If \tit{ca-w} had been used instead, then there would be no such implication. Similarly, if \tit{te-b} is used, the situation takes place or took place (far) away from the speaker who did not participate and did not witness the event himself/herself. For instance, the utterance in \refex{ex:Then, badly like this, she is not able to move} comes from a report about a woman who was in a hospital in Makhachkala and whom the speaker did not visit there.

\begin{exe}
	\ex	\label{ex:Ilja is going downwards periphrastic}
	\gll	Iljas	arg-ul	le-w	[\ldots]	kat'\\
		Ilyas	go\tsc{.ipfv-icvb}	exist\tsc{-m}	{}	down\\
	\glt	\sqt{Ilyas is going downwards.}
	
	\ex	\label{ex:Then, badly like this, she is not able to move}
	\gll	c'il	hel-itːe	wahi-l	šiq'	r-uq-an-aj	a-r-irχ-ul	te-r\\
		then	that\tsc{-advz}	bad\tsc{-advz}	stir	\tsc{f-}go\tsc{.pfv-ptcp-subj.3} \tsc{neg-f-}be.able\tsc{.ipfv-icvb} exist\tsc{.away-f}\\
	\glt	\sqt{Then, badly like this, she is not able to move.}

\end{exe}

The \isi{copula} \tit{χe-b} refers to events occurring in an area lower than the deictic center, which is often the speaker or a default reference point \refex{ex:There bottles have fallen down periphrastic}. This example and also \refex{ex:Then there one the picture it shows there are his T-shirt, jacket, and shoes.} were produced during the \textit{Family Problems Picture Task} \citep{SanRoqueEtAl2012} and the deictic center for \refex{ex:There bottles have fallen down periphrastic} is not the speaker who uttered this sentence (his location is irrelevant) but the people on the picture. 

\begin{exe}
	\ex	\label{ex:There bottles have fallen down periphrastic}
	\gll	ka-d-ič-ib-le=q'ar	χe-d	heχtːu-d	šuš-ne\\
		\tsc{down-npl-}occur\tsc{.pf=pret-cvb=mod}	exist.\tsc{down-npl}	there.\tsc{down-pl}	bottle\tsc{-pl}\\
	\glt	\sqt{There bottles have fallen down.}
\end{exe}

The \isi{copula} \tit{k'e-b} refers to events occurring in an area higher than the deictic center \refex{ex:Then there one the picture it shows there are his T-shirt, jacket, and shoes.}. Example \refex{ex:Then there one the picture it shows there are his T-shirt, jacket, and shoes.} is a description of pictures arranged on a table in front of the speaker that were put higher than some other pictures on the same table.  

\begin{exe}
	\ex	\label{ex:Then there one the picture it shows there are his T-shirt, jacket, and shoes.}
	\gll	c'il	hek'-tːi,	heχtːu-d	sːurrat-le-d	či-d-iž-aq-ul	k'e-d	maˁjk'a=ra	koftːa=ra		tːapri=ra	cin-na\\
		then	\tsc{dem.up}\tsc{-pl}	there.\tsc{down-npl}	picture\tsc{-loc-npl}	\tsc{spr-npl-}see\tsc{.ipfv-caus-icvb}	exist.\tsc{up-npl}	T-shirt\tsc{=add}	jacket\tsc{=add}		shoe\tsc{=add}	\tsc{refl.sg-gen}\\
	\glt	\sqt{Then those, there on the picture, it shows that there are his T-shirt, jacket, and shoes.}
\end{exe}

If the lexical verb takes the \isi{imperfective converb} suffix, the resulting verb form corresponds to the compound present (\refsec{ssec:Compound present}) and has a comparable semantic range covering progressive\slash continuative \refex{ex:One time he comes home drunk} and habitual \refex{ex:Then, badly like this, she is not able to move}.

\begin{exe}
	\ex	\label{ex:One time he comes home drunk}
	\gll	ca	zamana	b-erčː-ib-le	saˁ-q'-uˁnne	le-w	hel\\
		one	time	\tsc{n-}drink\tsc{.pfv-pret-cvb}	\tsc{hither}-go\tsc{-icvb}	exist\tsc{-m}	that\\
	\glt	\sqt{One time he is coming home drunk.}
\end{exe}



When the \isi{perfective converb} (i.e. preterite plus suffix -\textit{le}) is employed, the perfect or other forms are obtained. As the normal perfect (\refsec{ssec:The perfect}), the perfect with locational copulas mostly expresses states that obtain after a preceding event \refex{ex:There bottles have fallen down periphrastic}, \xxref{ex:‎(Then I relaxed) and we lived normally}{ex:‎One Kubachi person, one Agul person are buried (in Sanzhi)}.

\begin{exe}
	\ex	\label{ex:‎(Then I relaxed) and we lived normally}
	\gll	ka-d-iž-ib-le	le-d=da	hana\\
		\tsc{down-1/2pl-}be\tsc{.pfv-pret-cvb}	exist\tsc{-1/2pl=1}	now\\
	\glt	\sqt{‎(Then I relaxed) and we lived normally.} (lit. \sqt{We have sat down.})

	\ex	\label{ex:(The plant) has covered one wall of the house (by growing upwards).}
	\gll	ca	qal-la	baˁʔ	ka-b-uc-ib-le	k'e-b\\
		one	house\tsc{-gen}	façade	down\tsc{-n-}catch\tsc{.pfv-pret-cvb}	exist.\tsc{up-n}\\
	\glt	\sqt{(The plant) has covered one wall of the house (by growing upwards).}
	
		\ex	\label{ex:‎One Kubachi person, one Agul person are buried (in Sanzhi)}
	\gll	ca	Kubači-lan,	ca	aʁul-an	gu-b-aˁʁ-ib-le	k'e-b\\
		one	Kubachi\tsc{-nmlz}	one	Agul\tsc{-nmlz}	\tsc{sub-hpl-}release\tsc{.pfv-pret-cvb}	exist.\tsc{up-hpl}\\
	\glt	\sqt{‎One Kubachi person, one Agul person are buried (in Sanzhi).}
\end{exe}


The locational copulas can be followed by the past \isi{enclitic} =\textit{de}, so that we get a variant of the past perfect or \isi{pluperfect} (\refsec{ssec:The past perfect (pluperfect)}), usually referring to states that obtained in the past as the result of preceding situations \refex{ex:The sack was full filled}, but also occasionally in reference to actions and events that happened before a reference point in the past. Thus, example \refex{ex:The people started well} was uttered when the speaker compared the life of the family before and after an important event that served as a temporal anchoring point in the past.

\begin{exe}
	\ex	\label{ex:The sack was full filled}
	\gll	qːap	b-ic'-ib-le	χe-b=de\\
		sack	\tsc{n-}fill\tsc{.pfv-pret-cvb}	exist.\tsc{down-n=pst}\\
	\glt	\sqt{The sack was full (filled).}

	\ex	\label{ex:The people started well}
	\gll	ca	ʡaˁħ-le	b-aʔ	b-išː-ib-le	le-b=de	iš-tːi	χalq'\\
		one	good\tsc{-advz}	\tsc{n-}begin	\tsc{n-}put\tsc{.pfv-pret-cvb}	exist\tsc{-n=pst}	this\tsc{-pl}	people\\
	\glt	\sqt{The people started off well.}
\end{exe}

Another possible \isi{periphrastic verb form} corresponds to the \isi{experiential} II (\refsec{ssec:Experiential I and experiential II}) for which the lexical verb takes the suffixes of the preterite plus the \isi{cross-categorical suffix} \tit{-il}:

\begin{exe}
	\ex	\label{ex:‎Well, he is like sitting, probably, at the side (of the road)}
	\gll	nu	hej=ʁuna	ka-jž-ib-il	te-w	∅-urkː-ar	het	šːal-le-w=ra\\
		well	this\tsc{=eq}	\tsc{down}-remain\tsc{.m.pfv-pret-ref}	exist\tsc{.away-m}	\tsc{m-}find\tsc{.ipfv-cond.3}	that	side\tsc{-loc-m=add}\\
	\glt	\sqt{‎Well, he is like sitting, probably, at the side (of the road).}
\end{exe}

As can be seen from the examples in this section, many of the lexical verbs that are used in \is{periphrastic verb form}periphrastic verb forms with locational copulas are position verbs or verbs of movement \refex{ex:There bottles have fallen down periphrastic}, \refex{ex:One time he comes home drunk}, \refex{ex:‎(Then I relaxed) and we lived normally}, but other verbs are also allowed \refex{ex:Then there one the picture it shows there are his T-shirt, jacket, and shoes.}, \refex{ex:The sack was full filled}, \refex{ex:The people started well}.

Finally, the use of \tit{le-b} and other locational copulas in periphrasis is more common in other Dargwa varieties such as Mehweb \citep{Daniel2015}, Ashti \citep{Belyaev2012} and Shiri \citep{BelyaevInPreparation}.


%%%%%%%%%%%%%%%%%%%%%%%%%%%%%%%%%%%%%%%%%%%%%%%%%%%%%%%%%%%%%%%%%%%%%%%%%%%%%%%%

\section{Verb forms with \protect\tit{kelgʷ-} \protect\sqt{remain}}
\label{sec:Verb forms with kelgw- remain}

The verb \tit{kelgʷ-} (\tsc{pfv}) \sqt{remain, stay, be} is used as an auxiliary in constructions conveying continuous, enduring and sometimes habitually occurring situations and actions in the past. The auxiliary verb is either inflected for the preterite \tit{(kelg-un)} or some other verb form derived from the preterite. The imperfective stem \tit{kalg-} is not used in the auxiliary function. This means that the resulting clauses always have  past time reference. The \isi{periphrastic verb form} can be used with verbs of various \is{valency class}valency classes, e.g. \is{intransitive verb}intransitive verbs \refex{ex:‎‎‎Mine (i.e. my stick) remained upright standing}, \is{transitive verb}transitive verbs \refex{ex:‎He wrote the letter in one hour}, or \is{affective verb}affective verbs \refex{‎‎He was seeing the girls.}. 

When the lexical verb bears the \isi{imperfective converb} suffix, the resulting verb forms have habitual or continuative\slash progressive semantics \xxref{ex:‎He wrote the letter in one hour}{‎‎He was seeing the girls.}. Note that the verb `see' in \refex{‎‎He was seeing the girls.} has the literary meaning and thus the sentence refers to a continuous situation of seeing.

\begin{exe}
	\ex	\label{ex:‎He wrote the letter in one hour}
	\gll	it-i-l	ca	sːaˁʡaˁt	kaʁar	luk'-unne	kelg-un\\
		that\tsc{-obl-erg}	one	hour	letter	write\tsc{.ipfv-icvb}	remain\tsc{.pfv-pret}\\
	\glt	\sqt{‎He wrote the letter in one hour.} (E)

	\ex	\label{ex:[When the man was in prison he remembered a lot], how he constantly beat up his wife}
	\gll	cet'-le	cin-ni	xːunul	it-ul	kelg-un-ce=de=l\\
		how\tsc{-advz}	\tsc{refl.sg-erg}	woman	beat.up\tsc{-icvb}	remain\tsc{.pfv-pret-dd.sg=pst=indq}\\
	\glt	\sqt{[When the man was in prison he remembered a lot], how he constantly beat up his wife.}

	\ex	\label{‎‎He was seeing the girls.}
	\gll it-i-j	rurs-be	či-b-ig-ul	kelg-un\\
	that-\tsc{obl-dat}	girl-\tsc{pl}	\tsc{spr-hpl}-see.\tsc{ipfv-icvb}	remain.\tsc{pfv-pret}\\
	\glt \sqt{‎‎He was watching at (lit. seeing) the girls.} (E)
\end{exe}
	
Periphrasis with \tit{kelgʷ-} is frequently used when talking about, for example, professions and more generally about the kind of work someone is/was doing \refex{ex:‎Which work was uncle Mahammadhazhi doing}.	
	
\begin{exe}
	\ex	\label{ex:‎Which work was uncle Mahammadhazhi doing}
	\gll	Maħaˁmmadħaˁži	acːi-l	ce	ʡaˁči	b-irq'-ul	kelg-un-il=de?	\\
		Mahammadhazhi	uncle\tsc{-erg}	what	work	\tsc{n-}do\tsc{.ipfv-icvb}	remain\tsc{.pfv-pret-ref=pst}\\
	\glt	\sqt{‎Which work was uncle Mahammadhazhi doing?}
\end{exe}
%
With perfective converbs the construction is used for the expression of enduring states that obtain during a longer stretch of time \xxref{ex:The calf stayed alive periphrastic}{ex:In the morning, I slept until eleven}. The verbs in \refex{ex:‎‎‎Mine (i.e. my stick) remained upright standing} and \refex{ex:In the morning, I slept until eleven} refer to the actions of getting up and lying down, but when they are used with the perfective converbs they denote the states that obtain after having carried out the respective actions. 

\begin{exe}
	\ex	\label{ex:The calf stayed alive periphrastic}
	\gll	a-b-ebč'-ib-le	kelg-un	hel	qːačːa\\
		\tsc{neg-n-}die\tsc{.pfv-pret-cvb}	remain\tsc{.pfv-pret}	that	calf\\
	\glt	\sqt{The calf stayed alive.} (lit. \sqt{not died})

	\ex	\label{ex:‎‎‎Mine (i.e. my stick) remained upright standing}
	\gll	di-la	ka-b-icː-ur-re	kelg-un-ne	\ldots\\
		\tsc{1sg-gen}	\tsc{down-n-}stand\tsc{.pfv-pret-cvb}	remain\tsc{.pfv-pret-cvb}\\
	\glt	\sqt{‎‎‎Mine (i.e. my stick) remained upright standing, \ldots}

	\ex	\label{ex:In the morning, I slept until eleven}
	\gll	čːaˁʡaˁl-la	sːaˁʡaˁt	wec'-nu	ca-ra	d-ik-ar-aj	∅-usː-un-ne	kelg-un=da	du\\
		morning\tsc{-gen}	hour	ten-\tsc{ten}	one\tsc{-num}	\tsc{npl-}occur\tsc{.pfv-prs-subj.3}	\tsc{m-}sleep\tsc{.pfv-pret-cvb}		remain\tsc{.pfv-pret=1}	\tsc{1sg}\\
	\glt	\sqt{In the morning, I (masc.) slept until eleven.} (lit. \sqt{I remained lying})
\end{exe}

However, the use with perfective verb stems is restricted and it is possible that \refex{ex:The calf stayed alive periphrastic} is rather a biclausal sentence that consists of an \isi{adverbial clause} with the \isi{perfective converb} (`not having died') followed by a main clause (`the calf remained'). Thus, in \refex{‎‎He was seeing the girls.} the use of the perfective verb stem would lead to ungrammaticality \refex{‎‎He was seeing the girls.UNGRAM}. The precise conditions for this \isi{periphrastic verb form} when it is used with verbs of different aktionsart classes  remains open to future research.

\begin{exe}
	\ex[*]{	\label{‎‎He was seeing the girls.UNGRAM}
	\gll  it-i-j	rurs-be	či-b-až-ib-le	kelg-un\\
	 that-\tsc{obl-dat}	girl-\tsc{pl}	\tsc{spr-hpl}-see.\tsc{pfv-pret-cvb}	remain.\tsc{pfv-pret}\\
	\glt {} (Intended meaning: \sqt{‎‎He was watching at the girls.}) (E)}
\end{exe}




%%%%%%%%%%%%%%%%%%%%%%%%%%%%%%%%%%%%%%%%%%%%%%%%%%%%%%%%%%%%%%%%%%%%%%%%%%%%%%%%

\section{Verb forms with \protect\tit{b-el} \protect\sqt{remain, stay}}
\label{sec:Verb forms with b-el remain, stay}

The defective verb \tit{b-el} \sqt{remain, stay}, which refers to enduring states that obtained in the past and still obtain at the moment of speech, is occasionally used in periphrastic constructions that express the continuation of a state similar to periphrasis with \tit{kelgʷ-} described above in \refsec{sec:Verb forms with kelgw- remain}. The verb \tit{b-el} is almost exclusively used with lexical verbs inflected for the \isi{imperfective converb}. When the bare stem \tit{b-el} is used the construction conveys present time reference, referring to an ongoing event \refex{ex:Here he is crying (i.e. he continues to cry).} or existing state. When the past \isi{enclitic} is added to the verb, the \isi{periphrastic verb form} denotes a past state or an ongoing situation in the past \refex{ex:‎‎‎He was taken away by the water}, \refex{ex:‎‎Even though now (my hand) does not work, (at that time) it worked}.

\begin{exe}
	\ex	\label{ex:Here he is crying (i.e. he continues to cry).}
	\gll	ištːu-w	w-isː-ul	w-el	iž\\
		here\tsc{-m}	\tsc{m-}cry\tsc{-icvb}	\tsc{m-}remain	this\\
	\glt	\sqt{Here he is crying.} (i.e. he continues to cry).

	\ex	\label{ex:‎‎‎He was taken away by the water}
	\gll	hin-ni	∅-uqː-ul	w-el=de\\
		water\tsc{-erg}	\tsc{m-}carry\tsc{-icvb}	\tsc{m-}remain\tsc{=pst}\\
	\glt	\sqt{‎‎‎He was taken away by the water.} (lit. \sqt{He remained being carried away by the water.})

	\ex	\label{ex:‎‎Even though now (my hand) does not work, (at that time) it worked}
	\gll	hana	a-b-ucː-an=xːar,	ij	b-ucː-ul	b-el=de\\
		now	\tsc{neg-n-}work\tsc{-ptcp=conc}	this	\tsc{n-}work\tsc{-icvb}	\tsc{n-}remain\tsc{=pst}\\
	\glt	\sqt{‎‎Even though now (my hand) does not work, (at that time) it worked.}
	
\end{exe}

As mentioned above, the forms with \tit{b-el} and \tit{kelgʷ-} show similarities in their semantics, especially when they bear the past \isi{enclitic} =\textit{de}. For instance, we can replace \tit{kelg-un} in \refex{‎‎He was seeing the girls.} with \tit{b-el=de} and the meaning does not noticeably change \refex{‎‎He was seeing the girls.BEL}. As with the \isi{periphrastic verb form} with \tit{kelgʷ-}, the use of the perfective verb stem is not allowed in this sentence \refex{‎‎He was seeing the girls.UNGRAM2}.

\begin{exe}
	\ex[]{\label{‎‎He was seeing the girls.BEL}
	\gll it-i-j	rurs-be	či-b-ig-ul	b-el=de\\
	that-\tsc{obl-dat}	girl-\tsc{pl}	\tsc{spr-hpl}-see.\tsc{ipfv-icvb}	\tsc{hpl-}remain\tsc{=pst}\\
	\glt \sqt{‎‎He was watching at (lit. seeing) the girls.} (E)}
	
	\ex[*]{	\label{‎‎He was seeing the girls.UNGRAM2}
	\gll  it-i-j	rurs-be	či-b-až-ib-le	b-el\\
	 that-\tsc{obl-dat}	girl-\tsc{pl}	\tsc{spr-hpl}-see.\tsc{pfv-pret-cvb}	\tsc{hpl-}remain\\
	\glt {} (Intended meaning: \sqt{‎‎He is watching at the girls.}) (E)}

\end{exe}

Semantic differences are only perceptible when \tit{b-el} without the past enclitc is compared to \tit{kelg-un}, because the former has present time reference whereas the latter has past time reference. Thus, if we use \tit{kelg-un} instead of \tit{w-el} in \refex{ex:Here he is crying (i.e. he continues to cry).}, the resulting sentence conveys past time reference \refex{ex:Here he remained crying kelg-un}.

\begin{exe}
	\ex	\label{ex:Here he remained crying kelg-un}
	\gll	ištːu-w	w-isː-ul	kelg-un	iž\\
		here\tsc{-m}	\tsc{m-}cry\tsc{-icvb}	remain\tsc{.pfv-pret}	this\\
	\glt	\sqt{Here he remained crying.} (E)

\end{exe}


The example in \refex{ex:The calf stayed alive periphrastic} with \tit{kelg-un} can be used in a context in which the calf was about to die, but stayed alive (in fact, it fell down a slope but survived). The sentence in \refex{ex:The calf, not having died, is alive b-el}, which is a modified version of \refex{ex:The calf stayed alive periphrastic}, simply means that the calf is alive and has not died. The semantics indicates that the sentence is actually biclausal and the verb `die' and \textit{b-el} do not form a verbal complex. Periphrastic constructions with \textit{b-el} are only marginally acceptable if the lexical verb has \isi{perfective aspect} and carries the \isi{perfective converb} suffix. Combinations of a \isi{perfective converb} followed by \textit{b-el} are interpreted as separate clauses. Similarly, \refex{ex:The rye was still not taken to Icari (it was left untaken) b-el} is a complex clause expressing two situations: the situation that the rye had not been taken away to the neighboring village of Icari and the situation that the rye had remained in the village of Sanzhi. 

\begin{exe}
	\ex	\label{ex:The calf, not having died, is alive b-el}
	\gll	a-b-ebč'-ib-le,	b-el	hel	qːačːa\\
		\tsc{neg-n-}die\tsc{.pfv-pret-cvb}	\tsc{n-}remain	that	calf\\
	\glt	\sqt{The calf, not having died, is alive.} (E)
	
	\ex	\label{ex:The rye was still not taken to Icari (it was left untaken) b-el}
	\gll	sːusːul	gu-r-a-d-erqː-ib-le,	d-el=de	\\
		rye	\tsc{sub-abl-neg-npl-}take\tsc{.pfv-pret-cvb}	\tsc{npl-}remain\tsc{=pst}	\\
	\glt	\sqt{The rye was still not taken (to Icari) and had remained (in Sanzhi).}
\end{exe}



More frequent than the use of \tit{b-el} in finite periphrastic constructions as discussed in this section is the use in periphrastic constructions that function as heads of temporal \is{adverbial clause}adverbial clauses (\refsec{sec:periphrastic adverbial construction belle}).


%%%%%%%%%%%%%%%%%%%%%%%%%%%%%%%%%%%%%%%%%%%%%%%%%%%%%%%%%%%%%%%%%%%%%%%%%%%%%%%%

\section[Verb forms with the auxiliary \protect\tit{b-irχʷ-} (\protect\tsc{ipfv})/\protect\tit{b-iχʷ-} (\protect\tsc{pfv})]{Verb forms with the auxiliary \protect\tit{b-irχʷ-} (\protect\tsc{ipfv})/\protect\tit{b-iχʷ-} (\protect\tsc{pfv}) \protect\sqt{be, become, be able}}
\label{sec:Verb forms with the auxiliary b-irxw- (ipfv) / b-ixw- (pfv) be, become, be able}


% --------------------------------------------------------------------------------------------------------------------------------------------------------------------------------------------------------------------- %

\subsection{Periphrastic conditionals}
\label{ssec:Periphrastic conditionals}

The auxiliary \tit{b-irχʷ-} (\tsc{ipfv})\slash\tit{b-iχʷ-} (\tsc{pfv}) \sqt{be, become, be able} is used in periphrastic \isi{conditional} constructions, where it is inflected for various \isi{conditional} forms such as the realis \isi{conditional} \refex{ex:‎If you want stories, (here is one), she had such a father (and these were the stories about him)} or the past \isi{conditional} \refex{ex:‎If I would have known what you want, by God, I would also have said something} (see \refsec{sec:realisconditional} and \refsec{sec:pastconditional} for more examples and \refsec{sec:The syntax of conditional clauses} for the syntax of \is{conditional clause}conditional clauses). The periphrastic conditionals are functional equivalents of the simple conditionals and according to Sanzhi speakers there is no semantic difference between them. Thus, in \refex{ex:‎If you want stories, (here is one), she had such a father (and these were the stories about him)} \tit{b-ikː-ul ∅-iχ-utːe} could be replaced by \tit{b-ikː-aχː-at(te)}, and in \refex{ex:‎If I would have known what you want, by God, I would also have said something} \tit{b-alχ-ul r-iχ-utːel} could be substituted by \tit{b-alχ-aχː-at(te)} without any change in the meaning of the sentences.

\begin{exe}
	\ex	\label{ex:‎If you want stories, (here is one), she had such a father (and these were the stories about him)}
	\gll	at	χabar	b-ikː-ul	∅-iχ-utːe,	hel=ʁuna	atːa	∅-irχʷ-i	hel-i-la\\
		\tsc{2sg.dat}	story	\tsc{n-}want\tsc{.ipfv-icvb}	\tsc{m-}be\tsc{.pfv-cond.2sg}	that\tsc{=eq}	father	\tsc{m-}be\tsc{.ipfv-hab.pst}	that\tsc{-obl-gen}\\
	\glt	\sqt{‎If you want stories, (here is one), she had such a father (and these were the stories about him).}

	\ex	\label{ex:‎If I would have known what you want, by God, I would also have said something}
	\gll	d-ikː-an-ce	b-alχ-ul	r-iχ-utːel,	wallah,	ce-k'a	b-iχʷ-ar=ra	herʔ-adi		du=ra	cek'u\\
		\tsc{npl-}want\tsc{.ipfv-ptcp-dd.sg}	\tsc{n-}know\tsc{.ipfv-icvb}	\tsc{f-}be\tsc{.pfv-cond.pst}	by.God	what\tsc{-indef}	\tsc{n-}be\tsc{.pfv-cond.3=add}	say\tsc{.ipfv-hab.pst.1}	\tsc{1sg=add}	whatchamacallit\\
	\glt	\sqt{‎If I had known what you want, by God, I would also have said something.}
\end{exe}


% --------------------------------------------------------------------------------------------------------------------------------------------------------------------------------------------------------------------- %

\subsection{Epistemic modal constructions}\label{ssec:Epistemic modal constructions}
\largerpage

The same auxiliary is widely used in epistemic modal clauses to convey the meaning \sqt{probably, possibly, presumably}. In such constructions it is mostly inflected for the \isi{infinitive} (suffix -\textit{ij}) \refex{ex:These two are probably drinking} or the \isi{modal interrogative} \tit{(-ide)} \refex{ex:‎Probably the men had already told it}, and very rarely also for the realis \isi{conditional} \tit{(-ar)} \refex{ex:‎‎This is probably mowed (grass).}. Since these are suffixes that are predominantly or exclusively used with perfective stems, it is mostly the perfective stem \tit{b-iχʷ-} that occurs in the epistemic modal constructions. However, it is also possible to use the future in the past, in which case the imperfective stem of the auxiliary must be employed \refex{ex:‎She must have been scolding / probably she was scolding}. The auxiliary agrees in \isi{gender} and \isi{number} with the \isi{absolutive} argument, just like the lexical verb. It does not assign case to the arguments; case assignment is determined by the lexical verb. Thus, it behaves just like any other auxiliary. There is, however, one important difference. In normal \is{analytic verb form}analytic verb forms and other periphrastic constructions, the auxiliary cannot occur in the form of an \isi{infinitive} or \isi{conditional}, since these suffixes are only used in subordinate clauses such as complement or \is{conditional clause}conditional clauses. By contrast, in epistemic modal constructions, such a use is possible. For instance, example \refex{ex:These two are probably drinking} contains a lexical verb bearing the \isi{imperfective converb} suffix and the auxiliary \tit{b-iχʷ-}, to which the \isi{infinitive} is suffixed. The resulting clause is nevertheless a grammatical independent main clause. The lexical verb appears in a finite or non-finite verb form. A similar construction is attested in Icari Dargwa \citep[110]{Sumbatova.Mutalov2003}.

\begin{exe}
	\ex	\label{ex:These two are probably drinking}
	\gll	heštːi	k'ʷel=ra	b-učː-ul	b-iχʷ-ij\\
		these	two\tsc{=add}	\tsc{hpl-}drink\tsc{.ipfv-icvb}	\tsc{hpl-}be\tsc{.pfv-inf}\\
	\glt	\sqt{These two are probably drinking.}

	\ex	\label{ex:(He) probably has big sorrows}
	\gll	χːula	dard	χe-b	b-iχʷ-ij\\
		big	sorrow	exist.\tsc{down-n}		\tsc{n-}be\tsc{.pfv-inf}\\
	\glt	\sqt{(He) probably has big sorrows.}
\end{exe}

The auxiliary can be used as the only verb of the clause. It occurs in the form of the \isi{infinitive} but nevertheless functions as the head of an independent clause \refex{ex:This is probably a man.}.

\begin{exe}
	\ex	\label{ex:This is probably a man.}
	\gll	ik'	admi	∅-iχʷ-ij\\
		this.\tsc{up}	person	\tsc{m-}be\tsc{.pvf-inf}\\
	\glt	\sqt{This is probably a man.}
\end{exe}

As mentioned above, the auxiliary can also be inflected for future in the past, which itself already has epistemic modal semantics (\refsec{ssec:Future in the past}) \refex{ex:‎She must have been scolding / probably she was scolding}. Alternatively, the \isi{modal interrogative} form is attested \refex{ex:‎Probably the men had already told it}; this form is otherwise only used in \isi{questions} with first person subjects and epistemic and deontic modality (\refsec{sec:modalinterrogative}). Very occasionally the auxiliary appears in the form of the realis \isi{conditional} \refex{ex:‎‎This is probably mowed (grass).}.\largerpage

\begin{exe}
	\ex	\label{ex:‎She must have been scolding / probably she was scolding}
	\gll	ʁaj	r-ik'-ul	r-irχʷ-an=de	heχ\\
		word	\tsc{f-}say\tsc{.ipfv-icvb}	\tsc{f-}become\tsc{.ipfv-ptcp=pst}	\tsc{dem.down}\\
	\glt	\sqt{‎She must have been scolding\slash she was probably scolding.}

	\ex	\label{ex:‎Probably the men had already told it}
	\gll	b-urs-ib-le	b-iχʷ-ide	murgl-a-l\\
		\tsc{n-}tell\tsc{-pret-cvb}	\tsc{n-}be\tsc{.pfv-modq}	man\tsc{-obl-erg}	\\
	\glt	\sqt{‎Probably the men had already told it.}

	\ex	\label{ex:‎‎This is probably mowed (grass).}
	\gll	d-ertː-ib-te	a-d-iχʷ-ar\\
		\tsc{npl-}mow\tsc{.pfv-pret-dd.pl} 	\tsc{neg-npl-}be\tsc{.pfv-cond.3}\\
	\glt	\sqt{‎‎This is probably mowed (grass).} OR \sqt{If this is not mowed grass.}
\end{exe}

Together with the \isi{infinitive}, only third person controllers of \isi{person agreement} (which are nevertheless suppressed, since the auxiliary is in the \isi{infinitive}) are allowed. The use of the \isi{modal interrogative} also permits first and second person subject-like arguments:

\begin{exe}
	\ex	\label{ex:She / I was probably right}
	\gll	it	/	du	r-arx-le	r-iχʷ-ide\\
		\tsc{dem}	/	\tsc{1sg}	\tsc{f-}right\tsc{-advz}	\tsc{f-}be\tsc{.pfv-modq}\\
	\glt	\sqt{She\slash I was probably right.} (E)
\end{exe}

Negation can be expressed on the auxiliary \refex{ex:He also raised his arms; he probably does not want to (be taken away)} or on the lexical verb \refex{ex:‎Probably she was not scolding reprise}. In each case it has scope over the entire clause.

\begin{exe}
	\ex	\label{ex:He also raised his arms; he probably does not want to (be taken away)}
	\gll	nuˁq-be	aq d-arq'-ib ca-d	ik'-i-l=ra.	b-ikː-ul		a-b-iχʷ-ij\\
		arm\tsc{-pl}	high \tsc{npl-}do\tsc{.pfv-pret} \tsc{cop-npl}	\tsc{dem.up-obl-erg=add}	\tsc{n-}want\tsc{.ipfv-icvb}	\tsc{neg-n-}be\tsc{.pfv-inf}\\
	\glt	\sqt{He also raised his arms. He probably does not want to (be taken away).}

	\ex	\label{ex:‎Probably she was not scolding reprise}
	\begin{xlist}
		\ex	\label{ex:‎Probably she was not scolding@A}
		\gll	ʁaj	a-r-ik'-ul	r-irχʷ-an=de	heχ\\
			word	\tsc{neg-f-}say\tsc{.ipfv-icvb}	\tsc{f-}become\tsc{.ipfv-ptcp=pst}	\tsc{dem.down}\\
		\glt	\sqt{‎Probably she was not scolding.} (E)
	
		\ex	\label{ex:‎Probably she was not scolding@B}
		\gll	ʁaj	r-ik'-ul	a-r-irχʷ-an=de	heχ\\
			word	\tsc{f-}say\tsc{.ipfv-icvb}	\tsc{neg-f-}become\tsc{.ipfv-ptcp=pst}	\tsc{dem.down}\\
		\glt	\sqt{Probably she was not scolding.} (E)
	\end{xlist}
\end{exe}

The use of the future in the past and the \isi{modal interrogative} in a construction expressing epistemic modality is not particularly surprising, since (i) these forms have meanings that are similar to epistemic modality, and (ii) they are finite, that is, they can function as heads of main clauses. The use of the realis \isi{conditional} and the \isi{infinitive}, however, deserves further explanation. A plausible path of development is conventionalized ellipses of the main clause similar to examples of \isi{insubordination} that have been investigated by \citet{Evans2007} and \citet{EvansWatanabe2016}. Full \isi{conditional} constructions consist of an apodosis with the \isi{conditional} form and a protasis, in which the verb can choose from a rich array of possible morphosyntactic forms (\refcpt{cpt:conditionalconcessiveclauses}). In periphrastic conditionals, Sanzhi makes use of \tit{b-iχʷ}, as was shown above in \refsec{ssec:Periphrastic conditionals}. If in a periphrastic \isi{conditional} such as \refex{ex:Ashura mows the lawn} the protasis is omitted, we are left with a clause expressing a likely condition for an unspecified situation (\sqt{if X obtains}). The \isi{conditional} force has been lost and instead the proposition is judged as probable or possible, i.e. \sqt{if X obtains} > \sqt{X probably obtains} \refex{ex:‎‎This is probably mowed (grass).}. In fact, even if there is a protasis, it is nevertheless possible to have two readings for some apodosis clauses, namely a \isi{conditional} reading and an epistemic modal reading.

\begin{exe}
	\ex	\label{ex:Ashura mows the lawn}
	\begin{xlist}
		\ex	\label{ex:‎‎If Ashura has mowed this grass@A}
		\gll	heštːi	Ašura-l d-ertː-ib-te	q'ar	d-iχʷ-ar \ldots\\
			these	Ashura\tsc{-erg}	\tsc{npl-}mow\tsc{-pret-dd.pl} 	grass \tsc{npl-}be\tsc{.pfv-cond.3}\\
		\glt	\sqt{‎‎If Ashura has mowed this grass, \ldots}

		\ex	\label{ex:‎‎If Ashura has not mowed this grass@B}
		\gll	heštːi	Ašura-l d-ertː-ib-te	q'ar	a-d-iχʷ-ar \\
			these	Ashura\tsc{-erg}	\tsc{npl-}mow\tsc{.pfv-pret-dd.pl} 	grass \tsc{neg-npl-}be\tsc{.pfv-cond.3}\\
		\glt	\sqt{‎‎If Ashura has not mowed this grass, \ldots} > \sqt{Ashura has probably mowed this grass.}
	\end{xlist}
\end{exe}

A similar development might also be posited for the epistemic modals that are formed with the \isi{infinitive} of \tit{b-iχʷ-}. They possibly go back to epistemic and perhaps also deontic modal constructions with main predicates such as \tit{belki} \sqt{be possible} or \tit{ʡaˁʁunil} \sqt{necessary, needed, must, should} that take infinitival complements \refex{ex:Maybe they (= my thoughts) are not right}. If the main clause is omitted, only the clause with the \isi{infinitive} remains, which in examples such as \refex{ex:They (= my thoughts) are probably not right} has undergone a re-interpretation from deontic to epistemic modality: \sqt{X should obtain} \refex{ex:He must be a good man / He should be a good man} > \sqt{X probably obtains} \refex{ex:He is probably a good man}.

\begin{exe}
	\ex	\label{ex:Maybe they (= my thoughts) are not right}
	\gll	iχ-tːi	d-arx-le	a-d-iχʷ-ij	belki\\
		\tsc{dem.down}\tsc{-pl}	\tsc{1/2pl-}direct\tsc{-advz}	\tsc{neg-npl-}be\tsc{.pfv-inf}	it.is.possible\\
	\glt	\sqt{Maybe they (= my thoughts) are not right.}

	\ex	\label{ex:He must be a good man / He should be a good man}
	\gll	ʡaˁħ-ce	admi	∅-iχʷ-ij	ʡaˁʁuni-l	ca-w\\
		good\tsc{-dd.sg}	person	\tsc{m-}be\tsc{.pfv-inf}	needed\tsc{-advz}	\tsc{cop-m}\\
	\glt	\sqt{He must be a good man.\slash He should be a good man.}

	\ex	\label{ex:They (= my thoughts) are probably not right}
	\gll	iχ-tːi	d-arx-le	a-d-iχʷ-ij\\
		\tsc{dem.down}\tsc{-pl}	\tsc{1/2pl-}direct\tsc{-advz}	\tsc{neg-npl-}be\tsc{.pfv-inf}\\
	\glt	\sqt{They (= my thoughts) are probably not right.} (E)

	\ex	\label{ex:He is probably a good man}
	\gll	ʡaˁħ-ce	admi	∅-iχʷ-ij\\
		good\tsc{-dd.sg}	person	\tsc{m-}be\tsc{.pfv-inf}\\
	\glt	\sqt{He is probably a good man.}
\end{exe}


%%%%%%%%%%%%%%%%%%%%%%%%%%%%%%%%%%%%%%%%%%%%%%%%%%%%%%%%%%%%%%%%%%%%%%%%%%%%%%%%

\section{Epistemic modality with the auxiliary \protect\tit{b-urkː-} \protect\sqt{find}}
\label{sec:Epistemic modality with the auxiliary b-urk find}

In addition to the epistemic modal construction described in \refsec{ssec:Epistemic modal constructions}, there is another construction, which makes use of the verb \tit{b-urkː-} \sqt{find}. The perfective stem of this verb (\tit{b-arkː-}) is only used as an \isi{affective verb} with the meaning \sqt{find}, which means that it requires an \isi{experiencer} in the \isi{dative} and a \isi{stimulus} in the \isi{absolutive} case. The imperfective stem \tit{b-urkː-} is used both with the meaning \sqt{find} and as an auxiliary with the epistemic meaning \sqt{probably, be possible}.

The \tit{b-urkː-} constructions bears a strong similarity to the other epistemic modal construction because the lexical verb can be finite or non-finite. The default position of \tit{b-urkː-} is the default position for auxiliaries, namely following the lexical verb. It can be the only verb in the clause and still have the epistemic meaning. The auxiliary can be negated, then behaving like any other auxiliary: it expresses the \isi{negation} of the predication, that is, it has scope over the lexical verb \refex{ex:(He) probably does not want to get involved}. But \isi{negation} can also be expressed on the lexical verb with the same semantic effect \refex{ex:‎Then he will probably not do (this again)}.

\begin{exe}
	\ex	\label{ex:(He) probably does not want to get involved}
	\gll	vmešiwatsa	iχʷ-ij	b-ikː-ul	a-b-urkː-ar\\
		mingle	be\tsc{.pfv-inf}	\tsc{n-}want\tsc{.ipfv-icvb}	\tsc{neg-n-}find\tsc{.ipfv-prs}\\
	\glt	\sqt{(He) probably does not want to get involved.}

	\ex	\label{ex:‎Then he will probably not do (this again)}
	\gll	c'il	a-b-irq'-an-ne	∅-urkː-ar\\
		then	\tsc{neg-n-}do\tsc{.ipfv-ptcp-fut.3}	\tsc{m-}find\tsc{.ipfv-prs}\\
	\glt	\sqt{‎Then he will probably not do (this again).}
\end{exe}

The \isi{gender}/\isi{number} agreement can follow the \isi{ergative} pattern and thus be with the \isi{absolutive} argument \refex{ex:He probably knows their scandal@19}, or it can follow the accusative pattern. In the latter case it is controlled by the subject-like argument, which can be in the \isi{ergative} or \isi{dative} case (`deviant \isi{gender} agreement') \refex{ex:He also drank probably agreement with ergative@18}. Such behavior is not attested for all auxiliaries, but the \isi{copula} allows for it (\refsec{ssec:Gender agreement with arguments in other than the absolutive case}).

\begin{exe}
	\ex	\label{ex:He probably knows their scandal@19}
	\gll	iχ-i-j	b-alχ-ul	b-urkː-ar	ču-la	t'ama.hama\\
		\tsc{dem.down-obl-dat}	\tsc{n-}know\tsc{.ipfv-icvb}	\tsc{n-}find\tsc{.ipfv-prs}	\tsc{refl.pl-gen}	scandal\\
	\glt	\sqt{He probably knows their scandal.}

	\ex	\label{ex:He also drank probably agreement with ergative@18}
	\gll	b-erčː-ib-le	∅-urkː-ar	hel-i-l=ra\\
		\tsc{n-}drink\tsc{.pfv-pret-cvb}	\tsc{m-}find\tsc{.ipfv-prs}		that\tsc{-obl-erg=add}\\
	\glt	\sqt{He also drank, probably.} [\isi{gender} agreement with the \isi{ergative}]
\end{exe}

In all examples discussed so far, the auxiliary is inflected with the suffix \tit{-ar} \xxref{ex:(He) probably does not want to get involved}{ex:He also drank probably agreement with ergative@18}. The lexical verb (if there is any) is responsible for the temporal reference. The suffix \tit{-ar} is also used in the epistemic modal construction with \tit{b-iχʷ-} (\refsec{ssec:Epistemic modal constructions}), and it looks like the realis \isi{conditional} suffix for the third person. However, the realis \isi{conditional} is normally only formed from perfective stems (\refsec{sec:realisconditional}) and the form (\tit{b})-\textit{urkː-ar} never expresses \isi{conditional} semantics. Therefore, although we can suppose that there is a diachronic relationship with the \isi{conditional}, synchronically the form cannot be analyzed as \isi{conditional}, but is glossed with \tsc{prs}. Instead of the suffix \tit{-ar}, it is also possible to inflect the auxiliary regularly for the \isi{habitual present} \refex{ex:‎(It is made from) mint; I probably also told it (= how to make it) the last time} or the \isi{habitual past}, resulting in regular \isi{person agreement}  \refex{ex:‎‎Probably they did not always go (to drink milk)}, \refex{ex:You should know / you probably know, where (in which place) they (the berries) were}.\largerpage[-2]

\begin{exe}
	\ex	\label{ex:‎(It is made from) mint; I probably also told it (= how to make it) the last time}
	\gll	šːalme,	sala-r=ra	d-urs-ib	d-urkː-ud\\
		mint	front\tsc{-abl=add}	\tsc{npl-}tell\tsc{-pret}	\tsc{npl-}find\tsc{.ipfv-1.prs}\\
	\glt	\sqt{‎(It is made from) mint; I probably also told it (= how to make it) the last time.}
	
	\ex	\label{ex:‎‎Probably they did not always go (to drink milk)}
	\gll	har	zamana	b-ax-ul	a-b-určː-i=q'al\\
		every	time	\tsc{hpl-}go\tsc{-icvb}	\tsc{neg-hpl-}find\tsc{.ipfv-hab.pst=prt}\\
	\glt	\sqt{‎‎Probably they did not always go (to drink milk).}

	\ex	\label{ex:You should know / you probably know, where (in which place) they (the berries) were}
	\gll	ašːi-j	b-alχ-ul	d-urkː-a-tːa	čina	musːa-d=de=l\\
		\tsc{2pl-dat}	\tsc{n-}know\tsc{.ipfv-icvb}	\tsc{npl-}find\tsc{.ipfv-hab.pst-2pl}	where	place\tsc{.loc-npl=pst=indq}\\
	\glt	\sqt{You should know\slash you probably know, where (in which place) they (the berries) were.}
\end{exe}

It is possible that the construction goes back to a complement construction with \tit{b-urkː-} as matrix predicate; it is currently grammaticalizing and therefore one finds variation between those sub-constructions that show \isi{person agreement} and those that do not, and between the locus of \isi{negation} and the expression of temporal reference (i.e. whether the auxiliary or the main verb conveys the temporal reference). Similar epistemic modal constructions involving a verb \sqt{find} are attested in many other Dagestanian languages including other Dargwa varieties, Hinuq and other Tsezic languages, Avar, and Archi \citep{Forker2018a, Forker2018b}.

In summary, the epistemic modal constructions with \tit{b-iχʷ-} \sqt{be, become, be able} and \tit{b-urkː-} \sqt{find} have approximately the same range of meanings, and further research is needed to clarify if it is possible to establish semantic differences between them. The only difference observed so far pertains to morphosyntax. The verb \tit{b-iχʷ-} is most commonly used in the form of the \isi{infinitive}, which only allows for third person subject-like arguments, as the corpus examples in \refsec{ssec:Epistemic modal constructions} illustrate. By contrast, \tit{b-urkː-} is also attested in inflected forms that have first or second \isi{person agreement} controllers \refex{ex:‎(It is made from) mint; I probably also told it (= how to make it) the last time}, \refex{ex:You should know / you probably know, where (in which place) they (the berries) were}.


%%%%%%%%%%%%%%%%%%%%%%%%%%%%%%%%%%%%%%%%%%%%%%%%%%%%%%%%%%%%%%%%%%%%%%%%%%%%%%%%

\section{Indirect evidentiality with the auxiliary \protect\tit{b-ug-} \protect\sqt{be, be at, stay, remain}}
\label{sec:Indirect evidentiality with the auxiliary b-uz be, be at, stay, remain}

The verb \tit{b-ug-} \sqt{be, be at, stay, remain}, in addition to its use as the only predicate of a main clause, occurs as an auxiliary with evidential\slash inferential meaning. The auxiliary predominantly has the form of the \isi{resultative} (preterite + \isi{copula}), but the perfect or \isi{pluperfect} are also attested. In other words, it is inflected for verb forms that by themselves express resultativity. The lexical verb appears in the form of the imperfective or \isi{perfective converb}. The alignment of the auxiliary is identical to that of the lexical verb, that is, it agrees in \isi{gender} with the \isi{absolutive} argument.

The use of this auxiliary for conveying indirect evidentiality is a common strategy in many (if not all) Dargwa varieties, especially in traditional stories. For instance, the introductory formula for tales in Sanzhi is \tit{b-už-ib ca-b b-už-ib-le=kːu} (\tsc{n-}be\tsc{-pret} \tsc{cop-n} \tsc{n-}be\tsc{-pret-cvb=}\tsc{cop.neg}) \sqt{once upon a time}, with the second occurrence of \tit{b-už-} being optional.

The construction expresses non-firsthand evidentiality, in particular propositions\linebreak based on inferences from traces or results \refex{ex:(It turned out that) she did not chew the people, but swallowed them@13}, \refex{ex:Then (apparently) his hat remained there@14} or reasoning \refex{ex:They came from Shari (to us)@16}. For instance, example \refex{ex:(It turned out that) she did not chew the people, but swallowed them@13} occured in a fairy tale in which the villain ate all the people of a village. After she was killed and her belly was opened, the people could be rescued because they were still alive.

\begin{exe}
	\ex	\label{ex:(It turned out that) she did not chew the people, but swallowed them@13}
	\gll	il-i-l	č'aˁm	b-irq'-ul	b-už-ib-le=kːu,	qurt' 	iʁ-ul	b-už-ib ca-b	χalq'\\
		that\tsc{-obl-erg}	chew	\tsc{hpl-}do\tsc{.ipfv-icvb}	\tsc{hpl-}be\tsc{-pret-cvb=neg}	swallow	do\tsc{.pfv-icvb}	\tsc{hpl-}stay\tsc{-pret} \tsc{cop-hpl} people\\
	\glt	\sqt{(It turned out that) she did not chew the people, but swallowed them.}

	\ex	\label{ex:Then (apparently) his hat remained there@14}
	\gll	c'il	il-i-la	šljaˁp'a	kelg-un-ne	b-už-ib-le=de\\
		then	that\tsc{-obl-gen}	hat	remain\tsc{.pfv-pret-cvb}	\tsc{n-}stay\tsc{-pret-cvb=pst}\\
	\glt	\sqt{Then (apparently) his hat remained there.}
\end{exe}

Negation is expressed on the auxiliary, but negated lexical verb forms that have scope over the auxiliary are also possible. Thus, in elicitation, both ways of negating can be obtained, although with a slight semantic difference that becomes apparent if both verbs are negated \refex{ex:Then it did not turn out that his hat did not remain there, but (by contrast) we found it (there)@14c}. In the latter case we can see that the scope of the \isi{negation} prefix is the evidential auxiliary together with the whole clause if the auxiliary bears the prefix \refex{ex:Then (apparently) his hat did not remain there@14a}, \refex{ex:Then it did not turn out that his hat did not remain there, but (by contrast) we found it (there)@14c}. By contrast, the scope is the lexical verb together with its arguments, but excluding the evidential auxiliary, if the prefix appears on the lexical verb \refex{ex:Then (apparently) his hat did not remain there@14b}.

\begin{exe}
		\ex	\label{ex:Then (apparently) his hat did not remain there@14a}
		\gll	c'il	il-i-la	šljaˁp'a	kelg-un-ne	a-b-už-ib-le=de\\
			then	that\tsc{-obl-gen}	hat	remain\tsc{.pfv-pret-cvb}	\tsc{neg-n-}stay\tsc{-pret-cvb=pst}\\
		\glt	\sqt{Then (apparently) his hat did not remain there. } (= it did not turn out that his hat remained there) (E)

		\ex	\label{ex:Then (apparently) his hat did not remain there@14b}
		\gll	c'il	il-i-la	šljaˁp'a	a-kelg-un-ne	b-už-ib-le=de\\
			then	that\tsc{-obl-gen}	hat	\tsc{neg-}remain\tsc{.pfv-pret-cvb}	\tsc{n-}stay\tsc{-pret-cvb=pst}\\
		\glt	\sqt{Then (apparently) his hat did not remain there.} (= it turned out that his hat did not remain there) (E)

		\ex	{[context: we were betrayed by some other people who wanted to make us believe that his hat was not there anymore, but we found out the truth]}\label{ex:Then it did not turn out that his hat did not remain there, but (by contrast) we found it (there)@14c}\\%
		\gll	c'il	il-i-la	šljaˁp'a	a-kelg-un-ne	a-b-už-ib-le=de; nušːa-l	b-arčː-ib=da\\
			then	that\tsc{-obl-gen}	hat	\tsc{neg-}remain\tsc{.pfv-pret-cvb}	\tsc{neg-n-}stay\tsc{-pret-cvb=pst}	\tsc{1pl-erg}	\tsc{n-}find\tsc{.pfv-pret=1}\\
		\glt	\sqt{Then it did not turn out that his hat did not remain there, (but by contrast) we found it (there).}
\end{exe}

Indirect evidentiality can also include surprise about the inference if it contradicts the expectations of the speaker \refex{ex:(It turned out) they lived well first, look@15}.

\begin{exe}
	\ex	\label{ex:(It turned out) they lived well first, look@15}
	\gll	ix-tːi	bahsar,	ix-tːi	qːuʁa-l	er b-irχ-ul	b-už-ib ca-b	hex-tːi,	er	r-erč'-e!\\
		\tsc{dem.up-pl}	first	\tsc{dem.up-pl}	beautiful\tsc{-advz}	life \tsc{hpl-}be\tsc{.ipfv-icvb}	\tsc{hpl-}stay\tsc{-pret} \tsc{cop-hpl}	\tsc{dem.up-pl}	look	\tsc{f-}look\tsc{.pfv-imp}\\
	\glt	\sqt{(It turned out) they lived well first, look!} (said to a woman)
\end{exe}

Sometimes only evidential meaning is expressed, for example in narrations about past events of which no traces remained. In other cases, the speakers acquired their knowledge from the narrations of other people including their ancestors, such that the auxiliary expresses \isi{hearsay evidentiality}. For example, \refex{ex:pagans invaded from Shari} is part of a longer account about the history of the Sanzhi people, and the speaker speculates about other people who are said to have lived close to Sanzhi, and others who are said to have come to Sanzhi and destroyed the village. There are no visible results of these events. Instead, the speaker, based on his knowledge of the topography of Sanzhi and of stories about assaults on the village, hypothesizes from where enemies could have reached Sanzhi.

\begin{exe}
	\ex	\label{ex:pagans invaded from Shari}
	\begin{xlist}
		\ex	\label{ex:They were (apparently) called pagans}
		\gll	hel-tː-a-j	kapur-te	b-ik'-ul	b-už-ib	ca-b\\
			that\tsc{-pl-obl-dat}	pagan\tsc{-pl}	\tsc{hpl-}say\tsc{.ipfv-icvb}	\tsc{n-}stay\tsc{-pret}	\tsc{cop-n}\\
		\glt	\sqt{They were (apparently) called pagans.}
	
		\ex	\label{ex:They apparently killed our (people)}
		\gll	kerx-ul	b-už-ib	ca-b		nišːa-lla\\
			kill\tsc{.ipfv-icvb}	\tsc{hpl-}stay\tsc{-pret}	\tsc{cop-hpl}		\tsc{1pl-gen}\\
		\glt	\sqt{They apparently killed our (people).}
	
		\ex	\label{ex:They came from Shari (to us)@16}
		\gll	het	šaˁrʡaˁ-rka	sa-b-ax-ul	b-už-ib ca<b>i\\
			that	Shari\tsc{-abl}	\tsc{hither-hpl-}go\tsc{.ipfv-icvb}	\tsc{hpl-}stay\tsc{-pret} \tsc{cop<hpl>}\\
		\glt	\sqt{They came from Shari (to us).}
	\end{xlist}
\end{exe}

In general, the use of \tit{b-už-} can be considered to represent a stylistic device for traditional narratives and other traditional stories about the past, including funny and fictional anecdotes that Sanzhi people recite about their ancestors \refex{ex:they had no machete nor ax, they did not know (these tools)}.

\begin{exe}
	\ex	\label{ex:they had no machete nor ax, they did not know (these tools)}
	\gll	il-tːa-lla	kʷiriž,	il-tːa-lla	beretːa	cik'al	b-a-b-už-ib;	a-b-alχ-ul	b-už-ib	ca-b\\
		that-\tsc{obl.pl-gen}	machete	that-\tsc{obl.pl-gen}	ax	something	\tsc{n-neg}\tsc{-n-}be\tsc{-pret}	\tsc{neg-n-}know\tsc{.ipfv-icvb}	\tsc{n-}stay\tsc{-pret}	\tsc{cop-n}\\
	\glt	\sqt{They had no machete nor ax; they did not know (these tools).}
\end{exe}

When the auxiliary is used with the first person we get the reading that the speaker does not consider himself as an active, conscious participant in the event, and was rather informed about its true properties and implications afterwards, in other words, we obtain the first-person effect \refex{ex:(It turned out, that) I had seen Stalin@17}.

\begin{exe}
	\ex	{[When I was a small child my father took me to Moscow to a meeting of the Party.]}\label{ex:(It turned out, that) I had seen Stalin@17}\\%
	\gll	dam	Stalin	či-w-až-ib-le	už-ib-le=de\\
		\tsc{1sg.dat}	Stalin	\tsc{spr-m-}see\tsc{.pfv-pret-cvb}	stay\tsc{.m-pret-cvb=pst}\\
	\glt	\sqt{(It turned out, that) I (masc.) had seen Stalin.} (E)
\end{exe}
