\chapter{Verb formation}
\label{cpt:verbformation}

There are three types of operations that allow for the formation of complex verbal lexemes from base verbs:
%
\begin{itemize}
	\item	spatial preverbs (treated in \refsec{sec:Preverbs})
	\item	valency-changing derivation, i.e., causativization (\refsec{sec:Formation of causative verbs})
	\item	compounding (\refsec{sec:Compound verbs})
\end{itemize}

In this section, causativization and compounding are discussed. 

%%%%%%%%%%%%%%%%%%%%%%%%%%%%%%%%%%%%%%%%%%%%%%%%%%%%%%%%%%%%%%%%%%%%%%%%%%%%%%%%


\section{Formation of causative verbs}
\label{sec:Formation of causative verbs}
Causativization is a productive means of deriving causative verbs from base verbs. It can be applied to most if not all verbs, including intransitive, transitive and affective verbs of imperfective and perfective aspect. The causative suffix -\textit{aq} is added directly to the stem prior to TAM suffixes and it does not have any impact on the aspectual value of the verb or on the choice of certain inflectional suffixes (e.g. which suffix is used for the preterite). It has a pharyngealized allomorph \textit{-aˁq}. Furthermore, suffixation of the causative marker triggers palatalization of velar consonants in the verbal root. Examples of causativized verbs and their meanings are given in \refex{ex:causativizedVerbForms}.

\begin{exe}
	\ex	\label{ex:causativizedVerbForms}
	\begin{xlist}
	\ex \textit{b-ič-ib} ‘occurred, happend' (intr.) (\tsc{n}-occur.\tsc{pfv-pret}) \\
	> \textit{b-ič-aq-ib} ‘make occur, hit, strike' (tr.)
	\ex \textit{b-ikː-ul} ‘wanting, liking, loving' (aff.) (\tsc{n}-want.\tsc{ipfv-icvb}) \\
	> \textit{b-ičː-aq-ul} ‘make wanting, liking, loving' (tr.)
		\end{xlist}
\end{exe}

In the majority of cases, causativization adds one argument to the valency frame of the base verb, i.e. intransitive verbs become transitive and transitive verbs become ditransitive.
Causativization normally applies only once to the verbal stem, but in elicitation the causative suffix can also be added twice to a small number of verbs. However, due to the scarcity of examples the syntax and semantic properties of verbs that underwent double causativization could not be clarified. With the verb exemplified in \refex{ex:causativizedVerbFight} the meaning seems to be more emphatic, and the valency frame is transitive (as after single causativization). 

\begin{exe} 
	\ex	\label{ex:causativizedVerbFight} 
	\textit{b-iħ-ib} `(they) fought' (\tsc{hpl}-fight.\tsc{pfv-pret}) \\
	> \textit{b-iħ-aˁq-ib} `made fight' (tr.) \\
	> \textit{b-iħ-aˁq-aˁq-ib} `made fight' (tr.)
\end{exe}


In addition to morphological causativization, there are other formal means for making causative constructions such as light verb change and suppletion. This operation is applied to compound verbs. Intransitive compound verbs make use of the light verbs \tit{b-irχʷ-} (\tsc{ipfv})\slash\tit{b-iχʷ-} (\tsc{pfv}) \sqt{be, become, can} \refex{ex:‎At that time they (the trousers) opened even more}. For causativization these light verbs are replaced by \tit{b-irq'-} (\tsc{ipfv})\slash\tit{b-arq'-} (\tsc{pfv}) \sqt{do, make} \refex{ex:‎He opened the room together with them}. A full list of available light verbs is given in \refsec{sec:Light verbs used in compounding and general remarks on compounds}.  

%
\begin{exe}
	\ex	\label{ex:‎At that time they (the trousers) opened even more}
	\gll	hel	zamana	hati=ra	ač	d-iχ-ub	heχ-tːi	d-el-te=ra	uže\\
		that	time	more\tsc{=add}	open	\tsc{npl-}be\tsc{.pfv-pret}	\tsc{dem.down}\tsc{-pl}	\tsc{npl-}remain\tsc{.pfv-dd.pl=add}	already\\
	\glt	\sqt{‎At that time they (the trousers) opened even more including the remaining parts (that had been closed up to now).}

	\ex	\label{ex:‎He opened the room together with them}
	\gll	hel	hel-tː-a-cːella	canille	taχna	ač	b-arq'-ib\\
		that	that\tsc{-pl-obl-comit}	together	room	open	\tsc{n-}do\tsc{.pfv-pret}\\
	\glt	\sqt{‎He opened the room together with them.}
\end{exe}

See Section \refsec{sec:Causativization} for more information on the syntactic properties of causativization and more examples of causativized verbs.

%%%%%%%%%%%%%%%%%%%%%%%%%%%%%%%%%%%%%%%%%%%%%%%%%%%%%%%%%%%%%%%%%%%%%%%%%%%%%%%%
\section{Compound verbs}
\label{sec:Compound verbs}
Verbal compounds consist of two parts, the first of which can be a noun, short adjective, ideophone, bound lexical stem, or, very rarely, another verbal stem. It can be a native lexical item or a loan word. Thus, compounding is a convenient way of extending the verbal lexicon. The second part is a light verb from a closed class of verbs. It is only the light verb that is inflected and that determines all morphosyntactic properties of the compound. All light verbs used in compounding are given at the beginning of this Section in \refsec{sec:Light verbs used in compounding and general remarks on compounds} with both the imperfective and the perfective stems.

%%%%%%%%%%%%%%%%%%%%%%%%%%%%%%%%%%%%%%%%%%%%%%%%%%%%%%%%%%%%%%%%%%%%%%%%%%%%%%%%

\subsection{Light verbs used in compounding and general remarks on compounds}
\label{sec:Light verbs used in compounding and general remarks on compounds}

There are a fair number of light verbs occurring in compounding. The most frequent ones are \tit{b-iχʷ-ij} \sqt{ be, become, can}, \tit{b-ik'ʷ-ij} \sqt{say} and \tit{b-arq'-ij} \sqt{do}. The verb \tit{b-ik'ʷ-ij} is widely used in compounds that denote verbs of speech and the production of other sounds, but also in many verbs of movement. Since it means \sqt{say} when used on its own, I stick to this as an overall gloss. The two tables below display intransitive (\reftab{tab:Intransitive light verbs}) and transitive (\reftab{tab:Transitive light verbs}) light verbs.
%
\begin{table}
	\caption{Intransitive light verbs}
	\label{tab:Intransitive light verbs}
	\small
	\begin{tabularx}{1.0\textwidth}[]{%
		>{\raggedright\arraybackslash}p{63pt}
		>{\raggedright\arraybackslash}p{83pt}
		>{\raggedright\arraybackslash}X}
		
		\lsptoprule
			\tsc{ipfv}\slash\tsc{pfv} 	&	translation			&	example verbs\\
		\midrule
			\tit{b-irχʷ-}\slash\tit{b-iχʷ-} 	&	\sqt{be, become, can}	&	\tit{razi b-iχʷ-} \sqt{be happy, agree},\newline\tit{halak b-iχʷ-} \sqt{hurry, be fast},\newline\tit{uruχ b-iχʷ-} \sqt{get afraid},\newline\tit{uruc b-iχʷ-} \sqt{get embarrassed, ashamed}\\
			\tit{b-irk-}\slash\tit{b-ik-} 	&	\sqt{occur, get, receive}	&	\tit{ʡaˁʁni b-ik-} \sqt{need},\newline\tit{šak b-ik-} \sqt{feel, suppose},\newline\tit{suk b-ik-} \sqt{meet},\newline\tit{han b-ik-} \sqt{remember}\\
			\tit{b-ik'ʷ-} (\tsc{ipfv})	&	\sqt{say}			&	\tit{uf b-ik'ʷ-} \sqt{blow},\newline\tit{ʁumku b-ik'ʷ-} \sqt{swear},\newline\tit{zuruq sa-b-ik'ʷ-} \sqt{wriggle},\newline\tit{iχtilat b-ik'ʷ-} \sqt{chat},\newline\tit{qus b-ik'ʷ-} \sqt{slide},\newline\tit{zuq'-sa-b-ik'ʷ-} \sqt{swinging back and forth}\\
			\tit{b-ulq-}\slash\tit{b-uq-}	&	\sqt{go}			&	\tit{čːal b-uq-} \sqt{argue},\newline\tit{duc' b-uq-} \sqt{run},\newline\tit{ʡuˁt' b-uq-} \sqt{fall into pieces}\\
			\tit{b-irg-}\slash\tit{b-ig-}  	&	\sqt{be}			&	\tit{ʡuˁt' ka-b-ig-} \sqt{fall apart, be destroyed},\newline\tit{qus ka-b-ig-} \sqt{slip (off), slide down}\\
			\tit{b-ircː-}\slash\tit{b-icː-}	&	\sqt{stand, get up}		&	\tit{t'aš b-icː-} \sqt{stop} (intr.),\newline\tit{ʡaˁħ ka-b-icː-} \sqt{like, be pleased by},\newline\tit{hitːi ka-b-icː-} `back, stand behind,\\
			{}				&	{}				&	~~be committed to, support'\\
			\tit{argʷ-}\slash\tit{ag-}	&	\sqt{go}			&	\tit{b-iχči ag-} \sqt{believe},\newline\tit{xadi ag-} \sqt{marry} (woman marries a man)\\
		\lspbottomrule
	\end{tabularx}
\end{table}
%
\begin{table}
	\caption{Transitive light verbs}
	\label{tab:Transitive light verbs}
	\small
	\begin{tabularx}{0.92\textwidth}[]{%
		>{\raggedright\arraybackslash}p{63pt}
		>{\raggedright\arraybackslash}p{83pt}
		>{\raggedright\arraybackslash}X}
		
		\lsptoprule
			\tsc{ipfv}\slash\tsc{pfv} 	&	translation			&	example verbs\\
		\midrule
			\tit{b-irq'-}\slash\tit{b-arq'-}	&	\sqt{do}			&	\tit{ħuˁrmat b-arq'-} \sqt{respect},\newline\tit{jangi či-b-arq'-} \sqt{renovate, renew},\newline\tit{k'ap (ka-)b-arq'-} \sqt{wrap},\newline\tit{bursːi b-arq'-} \sqt{teach},\newline\tit{t'int' b-arq'-} \sqt{spread},\newline\tit{q'aˁq' b-arq'-} \sqt{squint},\newline\tit{qaˁm b-arq'-} \sqt{grab},\newline\tit{ʁina ʁina b-arq'-} \sqt{spoil}\\
			\tit{iʁ-}\slash\tit{aʁ-}		&	\sqt{do}			&	\tit{qaˁš kaʁ-} \sqt{cut into pieces},\newline\tit{t'aš aʁ-} \sqt{stop},\newline\tit{xurt' aʁ-} \sqt{swallow},\newline\tit{taˁħ aʁ-} \sqt{cut, chop},\newline\tit{b-aˁʡči aʁ-} \sqt{direct},\newline\tit{b-at-čir aʁ-} \sqt{release, set free},\newline\tit{b-at aʁ-} \sqt{send},\newline\tit{ʡuˁt' aʁ-} \sqt{destroy}\\
			\tit{b-uˁrq-}\slash\tit{b-aˁq-}	&	\sqt{hit, strike, wound}	&	\tit{guči b-aˁq-} \sqt{gather, collect},\newline\tit{ink b-aˁq-} \sqt{assemble, gather},\newline\tit{xʷit' d-aˁq-} \sqt{whistle},\newline\tit{tilipun d-aˁq-} \sqt{call on the phone}\\
			\tit{b-irxː-}\slash\tit{b-ixː}	&	\sqt{put}			&	\tit{daˁʡaˁna b-ixː-} \sqt{hide} (intr.),\newline\tit{can ka-b-ixː-} \sqt{put together}\\
		\lspbottomrule
	\end{tabularx}
\end{table}

The verb-forming processes listed in the following sections of this Chapter are relatively freely combinable. If a specific combination is available depends largely on the semantics of the resulting verb. Thus, compound verbs can contain spatial preverbs \refex{ex:laughing about (me)}- \refex{ex:Do not look at them (the treesCOMPOUNDING} and causativized verb stems.
%
\begin{exe}
	\ex	\label{ex:laughing about (me)}
	\gll	ħaˁħaˁ=tːi	ka-jk'-ul	\ldots\\
		laughter=after	\tsc{down}-say\tsc{.ipfv.m-icvb}\\
	\glt	\sqt{laughing about (me) \ldots}

	\ex	\label{ex:all villagers were mobilized}
	\gll	li<b>il	šːan-te	aq	či-ha-b-arq'-ib-le	\ldots\\
		all<\tsc{hpl}>	villager\tsc{-pl}	high	\tsc{spr-up}\tsc{-hpl-}do\tsc{.pfv-pret-cvb}\\
	\glt	\sqt{all villagers were mobilized, \ldots}

	\ex	\label{ex:Do not look at them (the treesCOMPOUNDING}
	\gll	ix-tː-a-j	er	či-ma-ha-rk'-utːa!\\
		\tsc{dem.up}\tsc{-pl-obl-dat}	look	\tsc{spr-proh-up}-look\tsc{.ipfv.m-proh.sg}\\
	\glt	\sqt{Do not look at them (the trees)!} (said to a man)
\end{exe}

Compounding is possible with loans. Some of the nouns listed in \refsec{ssec:compoundswithnouns} and adjectives listed in \refsec{sec:compoundswithshortadjectives} have been borrowed from other languages such as Arabic, Persian or Turkic. In the last 50 years mainly Russian borrowings entered the language. Two examples with Russian loans are given in \refex{ex:russianloancompounds}. The first compound verb contains the infinitive of a Russian verb and the second an adverb.

%
\begin{exe}
	\ex	\label{ex:russianloancompounds}
	\begin{xlist}
		\ex	\tit{kupatsa b-ik'ʷ-} \sqt{bathe, take a bath}
		\ex	\tit{žalka ag-} \sqt{feel sorry for}
	\end{xlist}
\end{exe}

%%%%%%%%%%%%%%%%%%%%%%%%%%%%%%%%%%%%%%%%%%%%%%%%%%%%%%%%%
\subsection{Compounds with nouns}
\label{ssec:compoundswithnouns}

Many compound verbs contain a noun. These nouns are often loan words. The nouns in the compounds are non-specific indefinite and can normally not be modified or referred to by anaphoric pronouns. They occur in the absolutive case, or occasionally in the genitive or are marked by spatial postpositions.

The compound verbs can be intransitive or transitive. For intransitive verbs, the noun that is part of the compound verb cannot control the agreement (\reftab{tab:Examples of intransitive compound verbs}) \refex{ex:Our (people) were fasting like nowadays}. By contrast, the noun that serves as the subject-like argument and occurs in the absolutive controls the gender agreement on the verb. Some of the compound verbs can take clausal complements \refex{ex:Once the sun and the evil wind argued about who is stronger}.
%
\begin{table}
	\caption{Examples of intransitive compound verbs}
	\label{tab:Examples of intransitive compound verbs}
	\small
	\begin{tabularx}{0.9\textwidth}[]{%
		>{\raggedright\arraybackslash}p{95pt}
		>{\raggedright\arraybackslash}X
		>{\raggedright\arraybackslash}X}
		
		\lsptoprule
			noun						&	light verb						&	translation of compound verb\\
		\midrule
			\tit{abdal} \sqt{fool}			&	\tit{b-iχʷ-}~~(\tsc{hpl-}be\tsc{.pfv-})		&	\sqt{be a fool}\\   
			\tit{er} \sqt{life}				&	\tit{b-iχʷ-}~~(\tsc{hpl-}be\tsc{.pfv-}) 		&	\sqt{live}\\
			\tit{taman} \sqt{end}			&	\tit{b-iχʷ-}~~(\tsc{n-}be\tsc{.pfv-}) 		&	\sqt{end, finish}\\
			\tit{tiladi} \sqt{request}			&	\tit{b-ik'ʷ-}~~(\tsc{hpl-}say\tsc{.ipfv-}) 	&	\sqt{request, ask, beg}\\
			\tit{tilipun} \sqt{telephone}		&	\tit{b-ik'ʷ-}~~(\tsc{hpl-}say\tsc{.ipfv-}) 	&	\sqt{talk on the phone}\\
			\tit{pikri} \sqt{thought}			&	\tit{b-uq-}~~(\tsc{hpl-}go\tsc{.pfv-})\slash 	&	\sqt{think}\\
			{}						&	\tit{b-ik'ʷ-}~~(\tsc{hpl-}say\tsc{.ipfv-}) 	&	{}\\
			\tit{čːal} \sqt{argument, quarrel} 	&	\tit{b-uq-}~~(\tsc{hpl-}go\tsc{.pfv-})\slash 	&	\sqt{argue, quarrel}\\
			{}						&	\tit{b-ik'ʷ-}~~(\tsc{hpl-}say\tsc{.ipfv-}) 	&	{}\\
			\tit{dum} \sqt{edge} 			&	\tit{b-uc-}~~(\tsc{hpl-}keep\tsc{.pfv-}) 	&	\sqt{fast}\\
			\tit{b-aʔ} \sqt{edge, begin}		&	\tit{b-ač'-}~~(\tsc{hpl-}come\tsc{.pfv-}) 	&	\sqt{begin, start}\\
			\tit{b-aʔ} \sqt{edge, begin} 		&	\tit{b-axː-}~~(\tsc{hpl-}put\tsc{.pfv-}) 	&	\sqt{begin, start}\\
			\tit{waˁw} 	\sqt{cry, call}		&	\tit{b-ik'ʷ-}~~(\tsc{hpl-}say\tsc{.ipfv-})	&	\sqt{cry, shout, call}\\
			\tit{gap} \sqt{praise}			&	\tit{b-ik'ʷ-}~~(\tsc{n-}say\tsc{.ipfv-})		&	\sqt{praise}\\
		\lspbottomrule
	\end{tabularx}
\end{table}
%
\begin{exe}
	\ex	\label{ex:Our (people) were fasting like nowadays}
	\gll	nišːa-la	dum	b-urc-ul=q'al	hana	daˁʡle\\
		\tsc{1pl-gen}	edge	\tsc{hpl-}keep\tsc{.ipfv-icvb=mod}	now	as\\
	\glt	\sqt{Our (people) were fasting like nowadays.}

	\ex	\label{ex:Once the sun and the evil wind argued about who is stronger}
	\gll	ca	zamana	bari=ra	wahi-ce	č'an=ra	čːal	d-uq-un	[kutːi	ču-cːe-rka	c'aq'-ce=de=l]\\
		one	time	sun\tsc{=add}	evil\tsc{-dd.sg}	wind\tsc{=add}	argument	\tsc{npl-}go\tsc{.pfv-pret}	which	\tsc{refl.pl-in-abl}	mighty\tsc{-dd.sg}=\tsc{pst=indq}\\
	\glt	\sqt{Once the sun and the evil wind argued about who is stronger.}
\end{exe}

For transitive verbs the subject-like argument is in the ergative and the gender agreement is almost always controlled by the noun that is part of the compound. This means that the gender agreement is fixed, mostly for neuter singular. Additional arguments fulfill the semantic functions of addressees, recipients or beneficiaries and occur in the cases that are used to express these semantic roles, e.g. dative or \tsc{in}-lative. Examples are provided in \reftab{tab:Examples of transitive compound verbs} and \refex{ex:when we asked other people}.
%
\begin{table}
	\caption{Examples of transitive compound verbs}
	\label{tab:Examples of transitive compound verbs}
	\small
	\begin{tabularx}{0.9\textwidth}[]{%
		>{\raggedright\arraybackslash}p{95pt}
		>{\raggedright\arraybackslash}X
		>{\raggedright\arraybackslash}X}
		
		\lsptoprule
			noun						&	light verb						&	translation of compound verb\\
		\midrule
			\tit{er} \sqt{look}				&	\tit{b-arq'-}~~(\tsc{n-}do\tsc{.pfv-})		&	\sqt{take a look}\\
			\tit{gap} \sqt{praise}			&	\tit{b-arq'-}~~(\tsc{n-}do\tsc{.pfv-})		&	\sqt{praise}\\
			\tit{jašaw}~\sqt{being,~existence}	&	\tit{b-arq'-}~~(\tsc{n-}do\tsc{.pfv-})		&	\sqt{make a living}\\	
			\tit{jašaw}~\sqt{being,~existence}	&	\tit{b-ucː-aq-}~~ 			&	\sqt{make a living}\\	
			{}						&	~~~(\tsc{n-}work\tsc{-caus-})					&	{}\\
			\tit{kumek} \sqt{help} 			&	\tit{b-arq'-}~~(\tsc{n-}do\tsc{.pfv-})		&	\sqt{help}\\
			\tit{mar} \sqt{truth} 			&	\tit{ka-b-icː-aq-}~~(\tsc{down}\tsc{-n-}		&	\sqt{prove}\\
			{}						&	~~~stand\tsc{.pfv-caus-})			&	{}\\
			\tit{pikri} \sqt{thought} 			&	\tit{b-arq'-}~~(\tsc{n-}do\tsc{.pfv-}) 		&	\sqt{think, give thought to}\\
			\tit{sːalam} \sqt{greeting} 		&	\tit{b-ikː-}~~(\tsc{n-}give\tsc{.pfv-})		&	\sqt{greet}\\
			\tit{taman} \sqt{end} 			&	\tit{b-arq'-}~~(\tsc{n-}do\tsc{.pfv-}) /		&	\sqt{finish}\\
			{}						&	\tit{aʁ-}~~(do\tsc{.pfv-})			&	{}\\
			\tit{tiladi} \sqt{request}			&	\tit{b-arq'-}~~(\tsc{n-}do\tsc{.pfv-})		&	\sqt{request}\\
			\tit{tilipun} \sqt{telephone}		&	\tit{d-arq'-}~~(\tsc{npl-}do\tsc{.pfv-}) /	&	\sqt{call on the phone}\\
			{}						&	\tit{d-aˁq-}~~(\tsc{npl-}hit\tsc{.pfv-})		&	{}\\
			\tit{ul} \sqt{eye}				&	\tit{b-ixː-}~~(\tsc{n-}put\tsc{.pfv-})		&	\sqt{blink}\\
			\tit{ʡaˁjib} \sqt{blame} 			&	\tit{b-arq'-}~~(\tsc{n-}do\tsc{.pfv-})		&	\sqt{take offence, feel hurt}\\
		\lspbottomrule
	\end{tabularx}
\end{table}
%
\begin{exe}
	\ex	\label{ex:when we asked other people}
	\gll	cara	adim-t-a-cːe	[\ldots]	heχ	tiladi	b-arq'-ib-le	\ldots\\
		other	person\tsc{-pl-obl-in}	{} \tsc{dem.down}		request	\tsc{n-}do\tsc{.pfv-pret-cvb}\\
	\glt	\sqt{when (we) asked other people, \ldots}
\end{exe}

However, there is at least one transitive compound verb containing a noun in the absolutive case for which not the subject-like argument, but the noun that serves as the direct object triggers the gender agreement, namely \tit{taman} \sqt{end} + \tit{b-arq'-} (\tsc{hpl}-do\tsc{.pfv-}) \sqt{finish (off), terminate} \refex{ex:(They) shot him into the forehead}. And in the example in \refex{ex:when (they) called the brother on the phone} the agreement prefix on the light verb does not agree with any overt noun. The first part of the compound, the noun \textit{tilipun} `telephone' belongs to the neuter gender. If it functioned as the object of the light verb it would trigger the prefix \textit{b}-.
%
\begin{exe}
	\ex	\label{ex:(They) shot him into the forehead}
	\gll	antːa-le	ix-ub-le,	taman	w-arq'-ib	le-w	musːa-w\\
		forehead\tsc{-loc}	throw\tsc{.pfv-pret-cvb}	end	\tsc{m-}do\tsc{.pfv-pret}	exist\tsc{-m}	place\tsc{.loc-m}\\
	\glt	\sqt{(They) shot him in the forehead and finished (i.e. killed) him on the spot where he was.}

	\ex	\label{ex:when (they) called the brother on the phone}
	\gll	cin-na	ucːi-li-j	tilipun	d-arq'-ib-le	\ldots\\
		\tsc{refl.sg-gen}	brother\tsc{-obl-dat}	telephone	\tsc{npl-}do\tsc{.pfv-pret-cvb}\\
	\glt	\sqt{when (they) called the brother on the phone, \ldots}
\end{exe}

There are some nouns that are particularly productive for the formation of compounds verbs and can combine with a variety of light verbs. One is the noun \tit{ʁaj} \sqt{word, talk, language}, that occurs in the following compounds:
%
\begin{exe}
	\ex	intransitive compound verbs (with subject in the absolutive)	\label{ex:wordcompoundsintransitive}
	\begin{xlist}
		\ex	\tit{ʁaj} \tit{(ka-)b-ik'ʷ-ij} (\tsc{down-hpl-}say\tsc{.ipfv-inf}) \sqt{say, tell} 
		\ex	\tit{ʁaj} \tit{(ka-)b-uq-ij} (\tsc{down-hpl-}go\tsc{.pfv-inf}) \sqt{chat, talk, communicate, converse} 
		\ex	\tit{ʁaj} \tit{ha-b-iž-ij} (\tsc{up-hpl-}be\tsc{.pfv-inf}) \sqt{chat} 
	\end{xlist}

	\ex	transitive (with subject in the ergative)	\label{ex:wordcompoundstransitive}
	\begin{xlist}
		\ex	\tit{ʁaj} \tit{d-arq'-ij} (\tsc{npl}-do\tsc{.pfv-inf}) \sqt{say, tell} 
		\ex	\tit{ʁaj} \tit{d-urs-ij} (\tsc{npl}-tell\tsc{.pfv-inf}) \sqt{say, tell} 
		\ex	\tit{ʁaj} \tit{b-ičː-ij} (\tsc{n-}give\tsc{.pfv-inf}) \sqt{promise}
	\end{xlist}
\end{exe}

In addition to the noun+verb compounds there are constructions that resemble those compounds but contain nouns in the genitive. The verbs used are \tit{b-arq'-} \sqt{do, make} and \tit{b-iχʷ-} \sqt{be, become} and a few other intransitive and transitive verbs \refex{ex:markednouncpgenitive},  \refex{ex:They took him for a foolCOMPOUND}. The nouns in the genitive case do not serve any argument functions in the clause, but form compounds together with the verb and thus contribute to the semantics of the predicate. In the predicates in \refex{ex:IRON} and \refex{ex:WEED} the genitive-marked nouns resemble instruments, but this cannot be said about the other predicates. Note that the last two examples in \refex{ex:markednouncpgenitive} differ from the others because in both cases the genitive can be explained by the morphosyntactic properties of the construction. The postposition \textit{hitːi} in \refex{exGossip} generally requires the genitive case and thus \tit{ʁaj-la hitːi d-urs-ij} consists of a postpositional phrase followed by a verb. In \refex{ex:considerguilty} the genitive functions as the modifier of the following noun such that we have a genitive phrase together with a verb. However, semantically both constructions function as compound predicates analogously to the other constructions with genitive-marked nouns. All compound predicates derive their transitivity from the transitivity of the base verb. If the base verb is intransitive the compound verb is also intransitive  \refex{ex:markednouncpgenitiveget married}; if the base verb is transitive, then the compound is also transitive \refex{ex:They took him for a foolCOMPOUND}. Some more examples sentences can be found in \refsec{sssec:Genitive} and in \refsec{sec:Intransitive verbs}. 

%
\begin{exe}
	\ex	genitive case	\label{ex:markednouncpgenitive}
	\begin{xlist}
		\ex \label{ex:markednouncpgenitiveget married} \textit{qal-la b-iχʷ-ij\slash qal-la ka-b-iž-ij} (house-\tsc{gen} \tsc{hpl}-be.\tsc{pfv-inf}\slash house-\tsc{gen} \tsc{down-hpl}-be.\tsc{pfv-inf}) \sqt{get married} 
		\ex \textit{waˁʡda-la b-iχʷ-ij} (contract-\tsc{gen} \tsc{hpl}-be.\tsc{pfv-inf}) \sqt{negotiate, conspire}
	\ex	\tit{abdal-la b-arq'-ij} (fool-\tsc{gen} \tsc{hpl-}do\tsc{.pfv-inf}) \sqt{take for a fool} 
	\ex	\tit{qaˁb-la b-arq'-ij\slash qaˁb-la + b-aˁq-ij} (neck-\tsc{gen} \tsc{hpl-}do\tsc{.pfv-inf}\slash neck-\tsc{gen} \tsc{hpl-}strike\tsc{.pfv-inf}) \sqt{behead}
	\ex	\tit{qal-la r-arq'-ij\slash qal-la ka-r-at-ij} (house-\tsc{gen} \tsc{f-}do\tsc{.pfv-inf}\slash house-\tsc{gen} \tsc{down-f-}let\tsc{.pfv-inf}) \sqt{marry off}
	\ex	\tit{χːaˁb-la b-arq'-ij} (grave-\tsc{gen} \tsc{hpl-}do\tsc{.pfv-inf}) \sqt{bury}
	\ex \textit{dawla-lla b-arq'-ij} (wealth-\tsc{gen} \tsc{hpl-}do\tsc{.pfv-inf}) \sqt{congratulate, bless}
	\ex \textit{itul-la b-arq'-ij} (iron-\tsc{gen} \tsc{n-}do\tsc{.pfv-inf}) \sqt{iron with an iron} \label{ex:IRON}
	\ex \textit{qːupi-lla b-arq'-ij} (hoe-\tsc{gen} \tsc{n-}do\tsc{.pfv-inf}) \sqt{weed}	\label{ex:WEED}
	\ex	\tit{ʁaj-la hitːi d-urs-ij} \sqt{gossip} (word\tsc{-gen} after \tsc{nhpl-}tell\tsc{.pfv-inf}) \label{exGossip}
	\ex	\tit{ʡaˁjib-la (w-ah) w-arq'-ij} (blame\tsc{-gen} \tsc{hpl-}owner \tsc{hpl-}do\tsc{.pfv-inf})  \sqt{consider to be guilty}\footnote{The noun \textit{w-ah} `owner' can be omitted in this construction.} \label{ex:considerguilty} 
	\end{xlist}
\end{exe}

\begin{exe}
		\ex	\tit{abdalla + b-arq'-ij} \sqt{take for a fool} \label{ex:They took him for a foolCOMPOUND}\\
		\gll	abdal-la	w-arq'-ib=q'al	itːa-l	it\\
			fool-\textsc{gen}	\textsc{m}-do.\textsc{pfv}-\textsc{pret}=\textsc{mod}	3\textsc{pl}.\textsc{obl}-\textsc{erg}	3\textsc{sg}\\
		\glt	\sqt{They took him for a fool.} (E)
\end{exe}

There is also one compound verb, which contains a noun marked with a spatial case \refex{ex:markednouncplocative}.

\begin{exe}
			\ex	\tsc{loc}-lative case	\label{ex:markednouncplocative} \\ \tit{(cin-na) ʁaj-le či-ka-b-icː-ij} \sqt{to be true to one's word} \\
			(\tsc{refl.sg-gen} word\tsc{-loc} \tsc{spr-down}\tsc{-hpl}-stand\tsc{.pfv-inf}) 
\end{exe}


Finally, there are compound verbs that contain nouns with the encliticized postpositions \tit{=či} `on' and \tit{=(i)tːi} `after'.  These postpositions govern the genitive or spatial cases (\refsec{ssec:postposition hiti}, \refsec{ssec:postposition ci}), but when they are used in verbal compounding, they are directly added to the nouns without case marking:

\begin{exe}
	\ex	spatial postposition/adverb \tit{=či} \label{ex:markednouncpon}
	\begin{xlist}
		\ex	\tit{ʁaj=či b-uq-ij} (word=on \tsc{hpl-}go\tsc{.pfv-inf}) \sqt{instruct, advice, blame}
		\ex	\tit{majmaj=či b-uq-ij} (quarrel=on \tsc{hpl-}go\tsc{.pfv-inf}) \sqt{educate, swear at, abuse, condemn}
		\ex	\tit{b-aˁʡ=ci aʁ-ij} (\tsc{n-}side=on do\tsc{.pfv-inf}) \sqt{direct}
	\end{xlist}

	\ex	spatial postposition/adverb \tit{=(i)tːi} (< \tit{hitːi})	\label{ex:markednouncpafter}
	\begin{xlist}
		\ex	\tit{er=itːi sa-b-erč'-ij} (look=after \tsc{ante-hpl-}look\tsc{.pfv-inf}) \sqt{look around, check, inspect}	
		\ex	\tit{er=itːi b-ik'ʷ-ij} (look=after \tsc{hpl-}say\tsc{.ipfv-inf}) \sqt{look (at)}
		\ex	\tit{qus=itːi b-aˁq-ij} (slip=after \tsc{n-}drag\tsc{.pfv-inf}) \sqt{pull, drag along, after oneself}	
		\ex	\tit{dukal=tːi ka-b-iħ-ij} (smile=after \tsc{down-hpl-}\tsc{aux.pfv-inf}) \sqt{smile about somebody} 
		\ex	\tit{ħaˁħaˁ=tːi b-ik'ʷ-ij} (laughter=after \tsc{hpl-}say\tsc{.ipfv-inf}) \sqt{laugh at/about someone}
	\end{xlist}
\end{exe}






%%%%%%%%%%%%%%%%%%%%%%%%%%%%%%%%%%%%%%%%%%%%%%%%%%%%%%%%%%%%%%%%%%%%%%%%%%%%%%%%

\subsection{Compounds with short adjectives}
\label{sec:compoundswithshortadjectives}

The short adjectival stems (\refsec{sec:adjmorphclasses}) can easily occur in compound verbs together with the light verbs \tit{b-iχʷ-} (\tsc{pfv}) \sqt{be, become, can} \refex{ex:compound verbs with short adjectives intransitive become}, \tit{b-ik-} (\tsc{pfv}) \sqt{occur}  \refex{ex:compound verbs with short adjectives intransitive occur}, and \tit{b-arq'-} (\tsc{pfv}) \sqt{do, make} \refex{ex:compound verbs with short adjectives transitive}. These verbs occur in pairs of intransitive verbs that normally have an inchoative meaning and transitive verbs \refexrange{ex:when Nursijat gets better (healthy)}{ex:I will repair them}. 
%
\begin{exe}
	\TabPositions{12em,14em}
	\ex	intransitive compounds with the light verb \tit{b-iχʷ-} (\tsc{n-}become\tsc{.pfv-})	\label{ex:compound verbs with short adjectives intransitive become}
	\begin{xlist}
		\ex	\tit{ʡaˁħ} \sqt{good}	\tab	>	\tab	\sqt{be, become good, get healthy}
		\ex	\tit{durha} \sqt{cheap}	\tab	>	\tab	\sqt{become, get cheap}
		\ex	\tit{ħaˁdur} \sqt{ready}	\tab	>	\tab	\sqt{prepare oneself}
		\ex	\tit{ač} \sqt{open}		\tab	>	\tab	\sqt{to open}
	\end{xlist}

	\ex	intransitive compounds with the light verb \tit{b-ik-} (\tsc{n-}occur\tsc{.pfv-})	\label{ex:compound verbs with short adjectives intransitive occur}
	\begin{xlist}
		\ex	\tit{dik'ar} \sqt{separate, different}	\tab	>	\tab	\sqt{separate, divorce, disjoin}
		\ex	\tit{tašmiš} \sqt{sad}	\tab	>	\tab	\sqt{get sad}
	\end{xlist}

	\ex	transitive compounds with the light verb \tit{b-arq'-} (\tsc{n-}do\tsc{.pfv-})	\label{ex:compound verbs with short adjectives transitive}
	\begin{xlist}
		\ex	\tit{ʡaˁħ} \sqt{good}	\tab	>	\tab	\sqt{improve, correct}
		\ex	\tit{durha} \sqt{cheap}	\tab	>	\tab	\sqt{make cheap}
		\ex	\tit{ħaˁdur} \sqt{ready}	\tab	>	\tab	\sqt{prepare}
		\ex	\tit{ač} \sqt{open}		\tab	>	\tab	\sqt{to open}
		\ex	\tit{dik'ar} \sqt{separate, different}	\tab	>	\tab	\sqt{separate, choose}
		\ex	\tit{tašmiš} \sqt{sad}	\tab	>	\tab	\sqt{make miserable, sadden}
	\end{xlist}
\end{exe}
%
\begin{exe}
	\ex	\label{ex:when Nursijat gets better (healthy)}
	\gll	Nursijat	ʡaˁħ	r-iχ-ub-le,	\ldots\\
		Nursijat	good	\tsc{f-}be\tsc{.pfv-pret-cvb}\\
	\glt	\sqt{when Nursijat gets better (healthy), \ldots}

	\ex	\label{ex:I will repair them}
	\gll	hel-tːi	du-l	ʡaˁħ	d-irq'-id\\
		that\tsc{-pl}	\tsc{1sg-erg}	good	\tsc{npl-}do\tsc{.ipfv-1}\\
	\glt	\sqt{I will repair them.}
\end{exe}

Occasionally, other light verbs are used, which leads to more idiosyncratic meanings \refex{ex:This is a beautiful, pleasant place}:
%
\begin{exe}
	\ex	\label{ex:This is a beautiful, pleasant place}
	\gll	qːuʁa-ce,	ʡaˁħ	ka-b-icː-ur	musːa	het	ca-b\\
		beautiful\tsc{-dd.sg}	good	\tsc{down-n-}stand\tsc{.pfv-pret}	place	that	be\tsc{-n}\\
	\glt	\sqt{This is a beautiful, pleasant place.}
	

\end{exe}


%%%%%%%%%%%%%%%%%%%%%%%%%%%%%%%%%%%%%%%%%%%%%%%%%%%%%%%%%%%%%%%%%%%%%%%%%%%%%%%%

\subsection{Compounds with ideophones}
\label{ssec:compoundswithideophones}

Sanzhi has a fair number of ideophones that combine not only with verbs of speech, but also with other light verbs and auxiliaries (\reftab{tab:Examples of compound verbs with ideophones}). The resulting compound verbs denote the production of various sounds as well as verbs of movement and other activities that are accompanied by typical sounds \xxref{ex:But she is screaming}{ex:(The wolf) swallowed all her sisters}.
%
\begin{table}
	\caption{Examples of compound verbs with ideophones}
	\label{tab:Examples of compound verbs with ideophones}
	\small
	\begin{tabularx}{1.0\textwidth}[]{%
		>{\raggedright\arraybackslash}p{68pt}
		>{\raggedright\arraybackslash}p{115pt}
		>{\raggedright\arraybackslash}X}
		
		\lsptoprule
			ideophone					&	light verb						&	translation of compound verb\\
		\midrule
			\tit{č'aˁm}					&	\tit{b-arq'-}~~(\tsc{n-}do\tsc{.pfv-})		&	\sqt{chew}\\
			\tit{čaˁχ}					&	\tit{b-ik'ʷ-}~~(\tsc{n-}say\tsc{.ipfv-})		&	\sqt{pour}\\
			\tit{c'ip}					&	\tit{či-r-aʁ-}~~(\tsc{spr-abl-}do\tsc{.pfv-})	&	\sqt{chop off, cut off}\\
			\tit{č'uˁp} 					&	\tit{b-ik'ʷ-}~~(\tsc{n-}say\tsc{.ipfv-}) /	&	\sqt{suck} (intr.)\slash (tr.)\\
			{}						&	\tit{b-arq'-}~~(\tsc{n-}do\tsc{.pfv-})		&	{}\\
			\tit{laˁħ, lap'}				&	\tit{(ha-)b-arq'-}~~(\tsc{up-n-}do\tsc{.pfv-})	&	\sqt{flap, wave}\\
			\tit{paˁq}					&	\tit{(či-ka-)b-ik'ʷ-}					&	\sqt{strike, hit on, beat}\\
			{}						&	~~(\tsc{spr-down}\tsc{-n-}say\tsc{.ipfv-})	&	{}\\
			\tit{paˁqaˁr, p'aq'}				&	\tit{b-uq-}~~(\tsc{hpl-}go\tsc{.pfv-})		&	\sqt{shake off}\\
			\tit{pas}					&	\tit{b-ik'ʷ-}~~(\tsc{n-}say\tsc{.ipfv-}) /	&	\sqt{scatter}\\
			{}						&	\tit{b-arq'-}~~(\tsc{n-}do\tsc{.pfv-})		&	{}\\
			\tit{pirχ}					&	\tit{b-arq'-}~~(\tsc{n-}do\tsc{.pfv-})		&	\sqt{light up}\\
			\tit{qːeh}					&	\tit{b-ik'ʷ-}~~(\tsc{n-}say\tsc{.ipfv-})		&	\sqt{cough}\\
			\tit{q'ac'}					&	\tit{b-ikː-}~~(\tsc{n-}bite\tsc{.pfv-}) /		&	\sqt{gnaw, bit}\\
			{}						&	\tit{b-ik'ʷ-}~~(\tsc{n-}say\tsc{.ipfv-}) /	&	{}\\
			{}						&	\tit{b-ax-}~~(\tsc{n-}go\tsc{-}) 		&	{}\\
			\tit{qaˁč'}					&	\tit{b-arq'-}~~(\tsc{n-}do\tsc{.pfv-})		&	\sqt{push, shove}\\
			\tit{qaˁš}					&	\tit{k-aʁ-}~~(\tsc{down}-do\tsc{.pfv-})		&	\sqt{cut off, cut into pieces}\\
			\tit{qit}					&	\tit{b-ik'ʷ-}~~(\tsc{n-}say\tsc{.ipfv-})		&	\sqt{whisper}\\
			\tit{ʁaˁʁ}					&	\tit{b-ik'ʷ-}~~(\tsc{n-}say\tsc{.ipfv-})		&	\sqt{scream}\\
			\tit{ʁuˁč'} 					&	\tit{b-arq'-}~~(\tsc{n-}do\tsc{.pfv-})		&	\sqt{squeeze, press down, compress}\\
			\tit{ʁʷaˁr, qamš}				&	\tit{b-arq'-}~~(\tsc{n-}do\tsc{.pfv-})		&	\sqt{scratch}\\
			\tit{sːul}					&	\tit{d-aˁq-}~~(\tsc{npl-}hit\tsc{.pfv-})		&	\sqt{die out, be extinguished, fade}\\	% \sqt{die out, be extinguished; fade}
			\tit{sːurk'} 					&	\tit{b-arq'-}~~(\tsc{n-}do\tsc{.pfv-})		&	\sqt{press}\\
			\tit{sːurk'}					&	\tit{b-ik'ʷ-}~~(\tsc{n-}say\tsc{.ipfv-}) /	&	\sqt{rub, polish}\\
			{}						&	\tit{b-arq'-}~~(\tsc{n-}do\tsc{.pfv-})		&	{}\\	   
			\tit{tːarʁar} 					&	\tit{b-ik'ʷ-}~~(\tsc{n-}say\tsc{.ipfv-}) /	&	\sqt{shake}\\
			{}						&	\tit{b-arq'-}~~(\tsc{n-}do\tsc{.pfv-})		&	{}\\	   
			\tit{tːartːar, tːamqːar} 			&	\tit{b-uq-}~~(\tsc{hpl-}go\tsc{.pfv-}) /		&	\sqt{stagger}\\
			{}						&	\tit{b-ik'ʷ-}~~(\tsc{n-}say\tsc{.ipfv-})		&	{}\\
			\tit{t'aˁq'}					&	\tit{b-ertː-}~~(\tsc{n-}burst\tsc{.pfv-})	&	\sqt{crack, split}\\
			\tit{tu} 					&	\tit{b-arq'-}~~(\tsc{n-}do\tsc{.pfv-})		&	\sqt{spit}\\
			\tit{xurt'}					&	\tit{aʁ-}~~(do\tsc{.pfv-})			&	\sqt{swallow}\\
			\tit{zuz} 					&	\tit{b-ik'ʷ-}~~(\tsc{n-}say\tsc{.ipfv-})		&	\sqt{stretch}\\
			\tit{χuˁrχ}					&	\tit{b-ik'ʷ-}~~(\tsc{n-}say\tsc{.ipfv-})		&	\sqt{snore}\\
			\tit{χʷaˁrt}					&	\tit{b-uq-}~~(\tsc{hpl-}go\tsc{.pfv-})		&	\sqt{flinch, cringe, wince}\\
			\tit{xʷit'}					&	\tit{d-ik'ʷ-}~~(\tsc{npl-}say\tsc{.ipfv-}) /	&	\sqt{whistle}\\
			{}						&	\tit{d-aˁq-}~~(\tsc{npl-}hit\tsc{.pfv-})		&	{}\\
		\lspbottomrule
	\end{tabularx}
\end{table}
%
\begin{exe}
	\ex	\label{ex:But she is screaming}
	\gll	amma	ʁaˁʁ	r-ik'-ul	ca-r	ik'\\
		but	scream	\tsc{f-}say\tsc{.ipfv-icvb}	be\tsc{-f}	\tsc{dem.up}\\
	\glt	\sqt{But she is screaming.}

	\ex	\label{ex:(He) is beating}
	\gll	paˁq	Ø-ik'-ul	ca-w\\
		strike	\tsc{m-}say\tsc{.ipfv-icvb}	be\tsc{-m}\\
	\glt	\sqt{(He) is beating.}

	\ex	\label{ex:(The wolf) swallowed all her sisters}
	\gll	li<b>il	xurt'	aʁ-ib	ca-b	hel-i-la	ruc-be\\
		all<\tsc{hpl}>	swallow	do\tsc{.pfv}-\tsc{pret} be\tsc{-hpl}	that\tsc{-obl-gen}	sister\tsc{-pl}\\
	\glt	\sqt{(The wolf) swallowed all her sisters.}
\end{exe}


%%%%%%%%%%%%%%%%%%%%%%%%%%%%%%%%%%%%%%%%%%%%%%%%%%%%%%%%%%%%%%%%%%%%%%%%%%%%%%%%

\subsection{Compounds with bound lexical stems}
\label{ssec:compoundswithboundroots}

There is a closed class of bound lexemes that occur only in compound verbs and of which thus the meaning out of the context of a compound verb is impossible to determine. These items do not belong to any of the lexical categories that Sanzhi has. Some of the bound stems are flexible with respect to the light verbs with which they combine leading to a variety of different compound verbs containing the same bound stem (\reftab{tab:Compound verbs with bound lexical stems (Part 1)}).
%
\begin{table}
	\caption{Compound verbs with bound lexical stems (Part 1)}
	\label{tab:Compound verbs with bound lexical stems (Part 1)}
	\small
	\begin{tabularx}{1.0\textwidth}[]{%
		>{\raggedright\arraybackslash}p{63pt}
		>{\raggedright\arraybackslash}X
		>{\raggedright\arraybackslash}X}
		
		\lsptoprule
			bound stem			&	light verb							&	translation of compound verb\\
		\midrule
			\tit{kːač}			&	\tit{b-ik'ʷ-}~~(\tsc{n-}say\tsc{.ipfv-}) /		&	\sqt{touch}\\
			{}				&	\tit{b-arq'-}~~(\tsc{n-}do\tsc{.pfv-}) /			&	{}\\
			{}				&	\tit{b-ik-}~~(\tsc{n-}occur\tsc{.pfv-})			&	{}\\
			\tit{can}			&	\tit{ka-b-ixː-}~~(\tsc{down-n-}put\tsc{.pfv-}) /	&	\sqt{mix, unite, meet}\\
			{}				&	\tit{ka-b-ig-}~~(\tsc{down-n-}be\tsc{.pfv-}) /	&	{}\\
			{}				&	\tit{b-ik-}~~(\tsc{n-}occur\tsc{.pfv-}) /		&	{}\\
			{}				&	\tit{b-ič-aq-}~~(\tsc{n-}occur\tsc{.pfv-caus}) /	&	{}\\
			{}				&	\tit{b-arq'-}~~(\tsc{n-}do\tsc{.pfv-})			&	{}\\
			\tit{taˁħ}			&	\tit{b-ik'ʷ-}~~(\tsc{n-}say\tsc{.ipfv-})\slash 		&	\sqt{jump}\\
			{}				&	\tit{b-uq-}~~(\tsc{n-}go\tsc{.pfv-}) /			&	{}\\
			{}				&	\tit{b-ax-}~~(\tsc{hpl-}go\tsc{.ipfv}) /			&	{}\\
			{}				&	\tit{(či-r)-b-ig-}~~(\tsc{(spr-abl)-n-}be\tsc{.pfv-})	&	{}\\
			\tit{taˁħ}			&	\tit{aʁ-}~~(do\tsc{.pfv-})				&	\sqt{cut off; make jump}\\
			\tit{b-at, b-atčir}		&	\tit{(k-)aʁ-}~~((\tsc{down}-)do\tsc{.pfv-}) /				&	\sqt{send, free, set out for}\\
			{}				&	\tit{b-uq-}~~(\tsc{hpl-}go\tsc{.pfv-}) /			&	{}\\
			{}				&	\tit{ka-b-ixː-}~~(\tsc{down-n-}put\tsc{.pfv-}) 		&	{}\\
			\tit{t'ut'u;}			&	\tit{b-ik'ʷ-}~~(\tsc{n-}say\tsc{.ipfv-}) /		&	`drive out, throw out,\\
			~~\tit{t'ut'u-q'aˁt'}		&	\tit{b-arq'-}~~(\tsc{n-}do\tsc{.pfv-}) /			&	~~leave, separate, distribute'\\
			{}				&	\tit{b-ig-}~~(\tsc{n-}be\tsc{.pfv-}) /			&	{}\\
			{}				&	\tit{b-iχʷ-}~~(\tsc{n-}become\tsc{.pfv-})		&	{}\\
			\tit{lus} \sqt{around}	&	\tit{b-uq-}~~(\tsc{hpl-}go\tsc{.pfv-}) /			&	\sqt{turn around}\\
			{}				&	\tit{b-ig-}~~(\tsc{n-}be\tsc{.pfv-}) /			&	{}\\
			{}				&	\tit{b-ik'ʷ-}~~(\tsc{n-}say\tsc{.ipfv-}) /		&	{}\\
			{}				&	\tit{b-ik'-aq-}~~(\tsc{n-}say\tsc{.ipfv-caus-}) /	&	{}\\
			{}				&	\tit{b-arq'-}~~(\tsc{n-}do\tsc{.pfv-})			&	{}\\
			\tit{han} 			&	\tit{b-ik-}~~(\tsc{n-}occur\tsc{.pfv-}) /		&	\sqt{seem, remember}\\
			{}				&	\tit{b-ič-aq}~~(\tsc{n-}occur\tsc{.pfv-}) /		&	{}\\
			{}				&	\tit{k.elg-}~~(\tsc{down}.remain\tsc{.pfv-}) /		&	{}\\
			{}				&	\tit{le-b}~~(exist\tsc{-n})					&	{}\\   
			\tit{čar} 			&	\tit{b-uq-}~~(\tsc{hpl-}go\tsc{.pfv-}) /			&	\sqt{return}\\
			{}				&	\tit{b-iχʷ-}~~(\tsc{n-}become\tsc{.pfv-}) /		&	{}\\
			{}				&	\tit{b-arq'-}~~(\tsc{n-}do\tsc{.pfv-})			&	{}\\
			\tit{t'a, t'aš}			&	\tit{b-icː-}~~(\tsc{hpl-}stand\tsc{.pfv-}) /		&	\sqt{stop}\\
			{}				&	\tit{aʁ-}~~(do\tsc{.pfv-}) /				&	{}\\
			{}				&	\tit{b-icː-aq-}~~(\tsc{hpl-}stand\tsc{.pfv-caus-}) 	&	{}\\
			\tit{qus} \sqt{slip}		&	\tit{b-ik'ʷ-}~~(\tsc{n-}say\tsc{.ipfv-}) /		&	\sqt{slip, slide, drag}\\
			{}				&	\tit{b-ig-}~~(\tsc{n-}be\tsc{.pfv-}) /			&	{}\\
			{}				&	\tit{b-aˁq-}~~(\tsc{n-}hit\tsc{.pfv-})			&	{}\\
			\tit{er} \sqt{look}		&	\tit{(či-)b-ik'ʷ-}~~(\tsc{(spr-)n-}say\tsc{.ipfv-}), 	&	\sqt{look}\\
			{}				&	\tit{(či-)b-erk'-}	&	{}\\
			{}				&  ~~(\tsc{(spr)-hpl-}look\tsc{.pfv-})\\
		\lspbottomrule
	\end{tabularx}
\end{table}

For instance, the bound stem \tit{taˁħ} occurs together with verbs of movement or posture to yield the meaning \sqt{jump}, but it also combines with other verbs. The resulting compounds always denote movement away from a source \xxref{ex:One (boar) jumped (down)}{ex:when (they) cut off the wood}.
%
\begin{exe}
	\ex	\label{ex:One (boar) jumped (down)}
	\gll	ca	taˁħ	b-uq-un	ca-b\\
		one	jump	\tsc{n-}go\tsc{.pfv-pret}	be\tsc{-n}\\
	\glt	\sqt{One (boar) jumped (down).}

	\ex	\label{ex:I would distract from the sorrows, right}
	\gll	šišːim-te	taˁħ	či-r-d-irg-an=de,	b-arx=ew?\\
		suffering\tsc{-pl}	jump	\tsc{spr-abl-npl-}be\tsc{.ipfv-ptcp=pst}	\tsc{n-}right\tsc{=q}\\
	\glt	\sqt{I would distract from the sorrows, right?}

	\ex	\label{ex:when (they) cut off the wood}
	\gll	taˁħ	aʁ-ib-le	hel-tːi	urcul,	\ldots\\
		cut	do\tsc{.pfv-pret-cvb}	that\tsc{-pl}	firewood\\
	\glt	\sqt{when (they) cut off the wood, \ldots}
\end{exe}

Other bound stems combine only with one or two light verbs (\reftab{tab:Compound verbs with bound lexical stems (Part 2a)}, \reftab{tab:Compound verbs with bound lexical stems (Part 2b)}). Among them the verbs \tit{b-ik'ʷ-} (\tsc{n-}say\tsc{.ipfv-}), \tit{b-arq'-} (\tsc{n-}do\tsc{.pfv-}) and \tit{b-uq-} (\tsc{hpl-}go\tsc{.pfv-}) are particularly frequent.
%
\begin{table}
	\caption{Compound verbs with bound lexical stems (Part 2)}
	\label{tab:Compound verbs with bound lexical stems (Part 2a)}
	\small
	\begin{tabularx}{0.95\textwidth}[]{%
		>{\raggedright\arraybackslash}p{63pt}
		>{\raggedright\arraybackslash}X
		>{\raggedright\arraybackslash}X}
		
		\lsptoprule
			bound stem		&	light verb							&	translation of compound verb\\
		\midrule
			\tit{akːa}		&	\tit{b-aˁq-}~~(\tsc{n-}hit\tsc{.pfv-}),			&	`leave in a huff,\\
			{}			&	\tit{aq-}~~(go.through\tsc{.pfv-})			&	~~withdraw offended'\\
			\tit{b-al} 		&	{}								&	\sqt{in order, fit, matching}\\
			\tit{b-iχči(t)}\slash 	&	\tit{ag-}~~(go\tsc{.pfv-})				&	\sqt{believe}	\\
			~~\tit{b-iχ-b-it-} 	&	{}								&	{}\\
			\tit{bursːi}		&	\tit{b-ik-}~~(\tsc{n-}occur\tsc{.pfv-}) /		&	\sqt{teach}\\
			{}			&	\tit{b-arq'-}~~(\tsc{n}-do.pfv-)					&	{}\\
			\tit{b-uz} 		&	\tit{b-it'-}~~(\tsc{n-}tear\tsc{.pfv-}) /			&	\sqt{stretch, lengthen}\\
			{}			&	\tit{b-ik'ʷ-}~~(\tsc{n-}say\tsc{.ipfv-})			&	{}\\
			\tit{duc'} 		&	\tit{b-ik'ʷ-}~~(\tsc{n-}say\tsc{.ipfv-}) /		&	\sqt{run}\\
			{}			&	\tit{b-uq-}~~(\tsc{hpl}-go\tsc{.pfv-})				&	{}\\
			\tit{guči} 		&	\tit{b-ik-}~~(\tsc{n-}occur\tsc{.pfv-}) /		&	\sqt{gather, collect, unite}\\
			{}			&	\tit{b-aˁq-}~~(\tsc{n-}hit\tsc{.pfv-})			&	{}\\
			\tit{hak'ar} 		&	\tit{b-ik'ʷ-}~~(\tsc{n-}say\tsc{.ipfv-}) /		&	\sqt{swing, shake}\\
			{}			&	\tit{b-arq'-}~~(\tsc{n-}do\tsc{.pfv-})			&	{}\\
			\tit{ħaˁsib} 		&	\tit{b-arq'-}~~(\tsc{n-}do\tsc{.pfv-})			&	\sqt{test, check, pay attention}\\
			\tit{ink}		&	\tit{b-aˁq-}~~(\tsc{n-}hit\tsc{.pfv-})			&	\sqt{meet, gather}\\
			\tit{k'ap}		&	\tit{b-ik-}~~(\tsc{n-}occur\tsc{.pfv-}) /		&	\sqt{wrap}\\
			{}			&	\tit{b-arq'-}~~(\tsc{n-}do\tsc{.pfv-})			&	{}\\
			\tit{k'ʷah} 		&	\tit{b-ik-}~~(\tsc{n-}occur\tsc{.pfv-})			&	\sqt{silent}\\
			\tit{kʷir} 		&	\tit{ka-b-ig-}~~(\tsc{down}\tsc{-n-}be\tsc{.pfv-}) /	&	\sqt{stop, lie down, sleep}\\
			{}			&	\tit{ka-b-isː-}~~(\tsc{down}\tsc{-n-}lie\tsc{.pfv-})		&	{}\\
			\tit{lak'} 		&	\tit{b-ik'ʷ-}~~(\tsc{n-}say\tsc{.ipfv-}) /		&	\sqt{throw, fling oneself}\\
			{}			&	\tit{b-arq'-}~~(\tsc{n-}do\tsc{.pfv-})			&	{}\\
			\tit{laˁk'} 		&	\tit{b-ig-}~~(\tsc{n-}be\tsc{.pfv-}) /			&	\sqt{leave, drive away}\\
			{}			&	\tit{b-arq'-}~~(\tsc{n-}do\tsc{.pfv-})			&	{}\\
			\tit{lakːa} 		&	\tit{b-arq'-}~~(\tsc{n-}do\tsc{.pfv-})			&	\sqt{throw hurl, fling}\\
			\tit{mucːa} 		&	\tit{b-uq-}~~(\tsc{hpl}-go\tsc{.pfv-})				&	\sqt{search}\\
			\tit{qːaˁp} 		&	\tit{b-ik'ʷ-}~~(\tsc{n-}say\tsc{.ipfv-}) /		&	\sqt{pull}\\
			{}			&	\tit{b-arq'-}~~(\tsc{n-}do\tsc{.pfv-})			&	{}\\
			\tit{qːuc}		&	\tit{b-ik-}~~(\tsc{n-}occur\tsc{.pfv-}) /		&	`touch, dip into, prick,\\
			{}			&	\tit{b-ik'ʷ-}~~(\tsc{n-}say\tsc{.ipfv-}) /		&	~~stick into'\\
			{}			&	\tit{b-arq'-}~~(\tsc{n-}do\tsc{.pfv-})			&	{}\\
			\tit{qaˁm} 		&	\tit{b-ik'ʷ-}~~(\tsc{n-}say\tsc{.ipfv-}) /		&	\sqt{grab}\\
			{}			&	\tit{b-arq'-}~~(\tsc{n-}do\tsc{.pfv-})			&	{}\\
			\tit{q'aˁq'} 		&	\tit{či-b-ig-}~~(\tsc{spr-n-}see\tsc{.pfv-}) /		&	\sqt{stare, peer at}\\
			{}			&	\tit{(či-)aʁ-}~~(\tsc{(spr)}-do\tsc{.pfv-})			&	{}\\
			\tit{qum} 		&	\tit{(k)ert-}~~(\tsc{(down)}.forget\tsc{.pfv-})		&	\sqt{forget}\\
		\lspbottomrule
	\end{tabularx}
\end{table}
%
\begin{table}
	\caption{Compound verbs with bound lexical stems (Part 3)}
	\label{tab:Compound verbs with bound lexical stems (Part 2b)}
	\small
	\begin{tabularx}{0.95\textwidth}[]{%
		>{\raggedright\arraybackslash}p{63pt}
		>{\raggedright\arraybackslash}X
		>{\raggedright\arraybackslash}X}
		
		\lsptoprule
			bound stem		&	light verb						&	translation of compound verb\\
		\midrule
			\tit{ʁina ʁina} 	&	\tit{b-iχʷ-}~~(\tsc{n-}become\tsc{.pfv-}) /	&	\sqt{spoil}\\
			{}			&	\tit{b-arq'-}~~(\tsc{n-}do\tsc{.pfv-})		&	{}\\
			\tit{šak}  		&	\tit{b-ik-}~~(\tsc{n-}occur\tsc{.pfv-})		&	\sqt{guess, suspect, feel}\\
			\tit{šiq'} 		&	\tit{b-ig-}~~(\tsc{n-}be\tsc{.pfv-}) /		&	\sqt{sway, rock, shake}\\
			{}			&	\tit{b-uq-}~~(\tsc{hpl-}go\tsc{.pfv-})		&	{}\\
			\tit{suk} 		&	\tit{b-ik-}~~(\tsc{n-}occur\tsc{.pfv-})		&	\sqt{meet, gather}\\
			\tit{t'int'} 		&	\tit{b-ik-}~~(\tsc{n-}occur\tsc{.pfv-}) /	&	\sqt{spread out}\\
			{}			&	\tit{b-arq'-}~~(\tsc{n-}do\tsc{.pfv-})		&	{}\\
			\tit{urk'} 		&	\tit{b-uq-}~~(\tsc{hpl-}go\tsc{.pfv-})		&	\sqt{wonder, fright}\\
			\tit{xar} 		&	\tit{b-eʁ-}~~(\tsc{n-}go\tsc{.pfv-})		&	\sqt{ask}\\
			\tit{ʡuˁt'}		&	\tit{aʁ-}~~(do\tsc{.pfv-})			&	\sqt{destroy}\\
		\lspbottomrule
	\end{tabularx}
\end{table}

As with the compound verbs containing short adjectives (\refsec{sec:compoundswithshortadjectives}), there are often pairs of intransitive and transitive verbs. They can be divided into groups depending on the intransitive verbs that they make use of.
%
Firstly, there are bound stems that are combined with \tit{b-iχʷ-} (\tsc{pfv}) \sqt{be, become, can} and \tit{b-arq'-} (\tsc{pfv}) \sqt{do, make} to form intransitive and transitive verbs, see, for instance, the examples in \refex{ex:hajbars}. Other light verbs cannot be used together with these stems.
%
\begin{exe}
	\ex	\label{ex:hajbars}
	\begin{xlist}
		\ex	\tit{haj}  \tit{b-iχʷ-}\slash\tit{haj} \tit{b-arq'-}  \sqt{move, drive}
		\ex	\tit{b-ars} \tit{b-iχʷ-}\slash\tit{b-ars} \tit{b-arq'-} \sqt{change}
	\end{xlist}
\end{exe}

And secondly, there are bound stems that are combined with \tit{b-ik-} (\tsc{pfv}) \sqt{occur} and \tit{b-arq'-} (\tsc{pfv}) \sqt{do, make} to form intransitive and transitive verbs:
%
\begin{exe}
	\ex	\label{ex:cansuksak}
	\begin{xlist}
		\ex	\tit{can} \tit{b-ik-}\slash\tit{can} \tit{b-arq'-} \sqt{mix, unite, meet}
		\ex	\tit{suk} \tit{b-ik-}\slash\tit{suk} \tit{b-arq'-} \sqt{meet, gather}
		\ex	\tit{šak} \tit{b-ik-}\slash\tit{šak} \tit{b-arq'-} \sqt{guess, suspect, feel}
	\end{xlist}
\end{exe}

Occasionally, stems can be combined with more than one intransitive auxiliary, e.g. \refex{ex:rudurqus}. 
%
\begin{exe}
	\ex	\label{ex:rudurqus}
	\begin{xlist}
		\TabPositions{6em}
		\ex	\tit{ʁudur} \sqt{mix} +	\tab	\tit{b-iχʷ-} (\tsc{pfv}) \sqt{be, become, can} and \tit{b-ik-} (\tsc{pfv}) \sqt{occur}
		\ex	\tit{qus} \sqt{slip} +		\tab	\tit{b-ik'ʷ-} (\tsc{n-}say\tsc{.ipfv-}) and \tit{b-ig-} (\tsc{n-}be\tsc{.pfv-})
	\end{xlist}
\end{exe}

There are a couple of compound verbs in which the first part synchronically seems to be a verb or diachronically to originate from a verb \refex{ex:cansuksak}. However, the compounds express verbal aspect only via the stem alternation of the second verb; the first part is invariable and not inflected except for the gender/number prefixes, which agree in exactly the same way as the prefixes, which belong to the inflecting verb \xxref{ex:He left her (at home)}{ex:He will / should go to sleep}.
%
\begin{exe}
	\ex	\label{ex:cansuksak}
	\begin{xlist}
		\ex	\tit{b-ax-b-at-} (\tsc{pfv})\slash\tit{b-ax-b-alt-} (\tsc{ipfv}) \sqt{leave, let} \\ 
		< \tit{b-ax-} (\tsc{hpl-}go\tsc{.ipfv}?) + \tit{b-at-} (\tsc{pfv}) `\tsc{hpl-}let'
		\ex	\tit{icːaχː-} (\tsc{pfv})\slash\tit{icːalχː} (\tsc{ipfv}) \sqt{start to hurt} \\
		< \tit{icː-} (\tsc{ipfv}) \sqt{hurt, ache} + ?
		\ex	\tit{b-it'-b-ak'-} (\tsc{pfv})\slash\tit{b-it'-b-ik'-} \sqt{pull, draw, move} \\
		< \tit{b-it'-} (\tsc{pfv}) \sqt{lure out of, from} + \tit{b-ak'-} \sqt{grow}?
		\ex	\tit{us.kelg-} (\tsc{pfv})\slash\tit{us.kalg-} (\tsc{ipfv}) \sqt{go to sleep, fall asleep}\\
		 < \tit{usː-} \sqt{lie} (\tsc{pfv}) + \tit{kelg-} (\tsc{pfv}) \sqt{remain, stay}
		\ex	\tit{b-iχ-(b)-it-ag-} (\tsc{pfv})\slash\tit{b-iχ-(b)-it-arg-} (\tsc{ipfv}) \sqt{believe} \\
		<  \tit{b-iχː-} \sqt{believe} + preverb \tit{b-it-} \sqt{thither} + \tit{ag-} (\tsc{pfv}) \sqt{go}
		\ex	\tit{b-iχ-čeg-} (\tsc{pfv})\slash\tit{b-iχ-čerg-} (\tsc{ipfv}) \sqt{believe} \\
		<  \tit{b-iχː-} \sqt{believe} + \tit{či-ag-} (\tsc{pfv}) \sqt{\tsc{spr}-go}
	\end{xlist}
\end{exe}

\begin{exe}
	\ex	\label{ex:He left her (at home)}
	\gll	it	r-ax-r-at-ur\\
		that	\tsc{f-}go\tsc{-f-}let\tsc{.pfv-pret}\\
	\glt	\sqt{(They/She/He) left her (at home).}

	\ex	\label{ex:Move here}
	\gll	w-it'-k-ač'-e	heštːu!\\
		\tsc{m-}pull-\tsc{down}-grow?\tsc{.pfv-imp}	here\\
	\glt	\sqt{Move here!} (E)

	\ex	\label{ex:(What) if the car does not move}
	\gll	mašin	b-it'-a-jk'-aχː-an	raχle?\\
		car	\tsc{n-}pull\tsc{-neg-}grow(?)\tsc{.ipfv-cond-prs.3}	if\\
	\glt	\sqt{(What) if the car does not move?}

	\ex	\label{ex:He will / should go to sleep}
	\gll	it	us-kalg-an	ca-w\\
		that	lie-remain\tsc{.ipfv-ptcp}	be\tsc{-m}\\
	\glt	\sqt{He will\slash should go to sleep.}
\end{exe}



%\begin{exe}
%	\ex	\label{ex:}
%	\gll	\\
%		\\
%	\glt	\sqt{}
%\end{exe}
