%add all your local new commands to this file

\newcommand{\smiley}{:)}

\renewbibmacro*{index:name}[5]{%
  \usebibmacro{index:entry}{#1}
    {\iffieldundef{usera}{}{\thefield{usera}\actualoperator}\mkbibindexname{#2}{#3}{#4}{#5}}}

% \newcommand{\noop}[1]{}

\makeatletter
\def\blx@maxline{77}
\makeatother

\newcommand{\appref}[1]{Appendix \ref{#1}}
\newcommand{\fnref}[1]{Appendix \ref{#1}}
\newcommand{\regel}[1]{#1}
\newcommand{\vernacular}[1]{\emph{#1}}
\newcommand{\gloss}[1]{#1}




% editing
	\definecolor{cornellred}{RGB}{179,28,28}						%	cornell red
	\definecolor{cerulean}{RGB}{0,122,166}						% 	cerulean
	\definecolor{britishracinggreen}{RGB}{0,66,38}					% 	british racing green
	
	\newcommand{\none}{\textcolor{cornellred}{\ding{56}}\thinspace} 		% 	red 'none' mark
	\newcommand{\hone}{\textcolor{cerulean}{\ding{50}}\thinspace} 		% 	blue 'hone' mark
	\newcommand{\done}{\textcolor{britishracinggreen}{\ding{51}}\thinspace} 	% 	green 'done' mark
	
	\newcommand{\parn}{\textcolor{cornellred}} 						% 	red colouration
	\newcommand{\parh}{\textcolor{cerulean}} 						% 	blue colouration
	\newcommand{\pard}{\textcolor{britishracinggreen}} 				% 	green colouration

% typesetting
	\newcommand{\tit}[1]{\textit{#1}}							%	shorthand for italics
	\newcommand{\tbf}[1]{\textbf{#1}}							%	shorthand for bold
	\newcommand{\tsc}[1]{{\rmfamily\textsc{#1}}}					%	shorthand for small caps
	\newcommand{\tup}[1]{{\upshape#1}}						%	shorthand for upright
	\newcommand{\tnm}[1]{\textnormal{#1}}						%	shorthand for clearing font styles

	\newcommand{\tang}[1]{\textless#1\textgreater}					%	enclose in angular brackets
	\newcommand{\tchv}[1]{$\langle$#1$\rangle$}					%	enclose in chevrons
	\newcommand{\tcbr}[1]{\{#1\}}								%	enclose in curly brackets
	\newcommand{\sqt}[1]{{}`{}#1{}'{}}							%	enclose in single quotation marks
	\newcommand{\dqt}[1]{{}``{}#1{}''{}}						%	enclose in double quotation marks
	\newcommand{\sqtx}[1]{{\upshape`#1'}}						%	enclose in single quotation marks, not italic

	\newcommand{\tsup}{\textsuperscript}							%	shorthand for superscript
	\newcommand{\tsub}{\textsubscript}							%	shorthand for subscript
	\newcommand{\tcol}{\textcolor}								%	shorthand for colour

	\newcommand{\hph}{\hphantom}							%	shorthand for horizontal phantom space
	\newcommand{\vph}{\vphantom}							%	shorthand for vertical phantom space

% characters
	\newcommand{\tnd}{\textendash}							%	shorthand for en-dash
	\newcommand{\tmd}{\textemdash}							%	shorthand for em-dash
	\newcommand{\ths}{\thinspace}								%	shorthand for thin space
	\newcommand{\tls}{\textless}								%	shorthand for less-than sign
	\newcommand{\tgr}{\textgreater}							%	shorthand for greater-than sign

% abbreviations 											%	(need \protect to be usable in float captions!)
	\makeatletter
		\newcommand{\tcf}{cf\protect\@ifnextchar{.}{}{.\protect\@}}		%	shorthand for properly spaced cf.
		\newcommand{\tch}{ch\protect\@ifnextchar{.}{}{.\protect\@}}		%	shorthand for properly spaced ch.
		\newcommand{\teg}{e.g\protect\@ifnextchar{.}{}{.\protect\@}}		%	shorthand for properly spaced e.g.
		\newcommand{\tie}{i.e\protect\@ifnextchar{.}{}{.\protect\@}}		%	shorthand for properly spaced i.e.
		\newcommand{\tviz}{viz\protect\@ifnextchar{.}{}{.\protect\@}}		%	shorthand for properly spaced viz.
		\newcommand{\tetc}{etc\protect\@ifnextchar{.}{}{.\protect\@}}	%	shorthand for properly spaced etc.
		\newcommand{\tincl}{incl\protect\@ifnextchar{.}{}{.\protect\@}}	%	shorthand for properly spaced incl.
		\newcommand{\tcc}{c\protect\@ifnextchar{.}{}{.\protect\@}}		%	shorthand for properly spaced c. (circa)
		\newcommand{\txm}{ex\protect\@ifnextchar{.}{}{.\protect\@}}		%	shorthand for properly spaced ex. (example, in citations)
		\newcommand{\ttab}{tab\protect\@ifnextchar{.}{}{.\protect\@}}	%	shorthand for properly spaced tab. (table, in citations)
		\newcommand{\tfig}{fig\protect\@ifnextchar{.}{}{.\protect\@}}		%	shorthand for properly spaced fig. (figure, in citations)
		\newcommand{\tfn}{fn\protect\@ifnextchar{.}{}{.\protect\@}}		%	shorthand for properly spaced fn. (footnote, in citations)
		\newcommand{\ten}{en\protect\@ifnextchar{.}{}{.\protect\@}}		%	shorthand for properly spaced en. (endnote, in citations)
	\makeatother

% references
%	\newcommand{\refx}{\cref}								%	alias for references using cleveref
%
%	\crefname{xnumi}{}{}									%	reference name for linguistic examples, level 1
%	\creflabelformat{xnumi}{#2(#1)#3}							%	label format for linguistic examples, level 1
%	\crefrangelabelformat{xnumi}{#3(#1#4\textendash#5#2)#6}			%	label format for ranges of linguistic examples, level 1
%
%	\crefname{xnumii}{}{}									%	reference name for linguistic examples, level 1
%	\creflabelformat{xnumii}{#2(#1)#3}							%	label format for linguistic examples, level 1
%	\crefrangelabelformat{xnumii}{#3(#1#4\textendash#5#2)#6}			%	label format for ranges of linguistic examples, level 1
%
%	\crefname{enumi}{}{}									%	reference name for enumerated items, level 1
%	\creflabelformat{enumi}{#2#1#3.}							%	label format for enumerated items, level 1
%	\crefrangelabelformat{enumi}{#3#1#4.\textendash#5#2#6.}			%	label format for ranges of enumerated items, level 1
%
%	\crefname{alphabetizei}{}{}								%	reference name for alphabetical enumerated items, level 1
%
%	\newcommand{\crefrangeconjunction}{\textendash}					%	conjunction for ranges
%	\newcommand{\creflastconjunction}{, and~}						%	conjunction before final item in lists with more than two elements
%	\newcommand{\creflastgroupconjunction}{, and~}					%	conjunction before final item in mixed lists with more than two types


	\newcommand{\exsource}[1]{~\\[-0.7cm]{\hspace*{\fill}\footnotesize{\textit{#1}}}} 	% for indicating example sources


	\newcommand{\refex}[1]{(\ref{#1})} 							%	for example references
	\newcommand{\refexrange}[2]{(\ref{#1}\textendash\ref{#2})} 			%	for ranges of example references, e.g. (1-2)

	\newcommand{\refcpt}[1]{\chapref{#1}}						%	alias for referencing chapters
	\newcommand{\refsec}[1]{\sectref{#1}}						%	alias for referencing sections
	\newcommand{\reftab}[1]{\tabref{#1}}						%	alias for referencing tables
	\newcommand{\reffig}[1]{\figref{#1}}							%	alias for referencing figures
	
	\newcommand{\citea}[1]{\citeauthor{#1}} 						%	shorthand for author-only citations
	\newcommand{\citey}[1]{\citeyear{#1}} 						%	shorthand for year-only citations
	\newcommand{\citeb}[1]{\citealp{#1}} 						%	shorthand for bare citations

% IPA shorthands (legacy)
	\newcommand{\pafr}{š}	%{\v{s}}							%	voiceless post-alveolar fricative
	\newcommand{\paaf}{č}	%{\v{c}}							% 	voiceless post-alveolar affricate
	\newcommand{\vuvfr}{ʁ}	%{\textinvscr}						% 	voiced uvular fricative
	\newcommand{\uvfr}{χ}	%{\textchi}							% 	voiceless uvular fricative
	\newcommand{\phfr}{ħ}	%{\textcrh}							% 	voiceless pharyngeal fricative
	\newcommand{\eppl}{ʡ}	%{\textbarglotstop}					% 	epiglottal plosive
	\newcommand{\glpl}{ʔ}	%{\textglotstop}						% 	glottal stop
	\newcommand{\glst}{ʔ}	%{\textglotstop}						% 	alias for glottal stop
	
	\newcommand{\ej}{’}									% 	ejective
	\newcommand{\lab}{ʷ}	%{\textsuperscript{w}}					% 	labialized
	\newcommand{\pha}{ˤ}	%{\textsuperscript{\textrevglotstop}}			% 	pharyngealized
	\newcommand{\lmk}{ː}	%{\textlengthmark}					% 	long / geminated

