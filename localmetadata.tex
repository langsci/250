\title{A Grammar of Sanzhi Dargwa}
% \subtitle{Change subtitle}
%\BackTitle{Change your backtitle in localmetadata.tex} % Change if BackTitle != Title
\BackBody{Sanzhi Dargwa belongs to the Dargwa (Dargi) languages (ISO dar; Glottocode darg1242) which form a subgroup of the Nakh-Daghestanian language family. Sanzhi Dargwa is spoken by approximately 250 speakers and is severely endangered. This book is the first comprehensive descriptive grammar of Sanzhi, written from a typological perspective. It treats all major levels of grammar (phonology, morphology, syntax) and also information structure. Sanzhi Dargwa is structurally similar to other East Caucasian languages, in particular Dargwa languages. It has a relatively large consonant inventory including pharyngeal and ejective consonants. Sanzhi morphology is concatenative and mainly suffixing. The language exhibits a mixture of dependent-marking in the form of a rich case inventory and head-marking in the form of verbal agreement. Nouns are divided into three genders. Verbal inflection conflates tense\slash aspect\slash mood\slash evidentiality in a rich array of synthetic and analytic verb forms as well as participles, converbs, a masdar, and infinitive and some other forms used in analytic tenses and subordinate clauses. Salient traits of the grammar are two independently operating agreement systems: gender\slash number agreement and person agreement. Within the nominal domain, modifiers agree with the head nominal in gender\slash number. Agreement within the clausal domain is mainly controlled by the argument in the absolutive case. Person agreement operates only at the clausal level and according to the person hierarchy 1, 2 > 3. Sanzhi has ergative alignment in the form of gender\slash number agreement and ergative case marking. The most frequent word order at the clause level is APV, though all other logically possible word orders are also attested. In subordinate clauses, word order is almost exclusively head-final.}
%\dedication{Change dedication in localmetadata.tex}
\typesetter{Diana Forker}
%\proofreader{Change proofreaders in localmetadata.tex}
\author{Diana Forker}
\BookDOI{10.5281/zenodo.3339225}%ask coordinator for DOI
\renewcommand{\lsISBNdigital}{978-3-96110-196-2}
\renewcommand{\lsISBNhardcover}{978-3-96110-197-9}
\renewcommand{\lsSeries}{loc} % use lowercase acronym, e.g. sidl, eotms, tgdi
\renewcommand{\lsSeriesNumber}{2} %will be assigned when the book enters the proofreading stage
\renewcommand{\lsID}{250} % contact the coordinator for the right number
